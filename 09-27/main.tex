
\documentclass[a4paper,11pt]{article}
\usepackage[a4paper, margin=8em]{geometry}

% usa i pacchetti per la scrittura in italiano
\usepackage[french,italian]{babel}
\usepackage[T1]{fontenc}
\usepackage[utf8]{inputenc}
\frenchspacing 

% usa i pacchetti per la formattazione matematica
\usepackage{amsmath, amssymb, amsthm, amsfonts}

% usa altri pacchetti
\usepackage{gensymb}
\usepackage{hyperref}
\usepackage{standalone}

% imposta il titolo
\title{Appunti Reti Logiche}
\author{Luca Seggiani}
\date{2024}

% imposta lo stile
% usa helvetica
\usepackage[scaled]{helvet}
% usa palatino
\usepackage{palatino}
% usa un font monospazio guardabile
\usepackage{lmodern}

\renewcommand{\rmdefault}{ppl}
\renewcommand{\sfdefault}{phv}
\renewcommand{\ttdefault}{lmtt}

% disponi il titolo
\makeatletter
\renewcommand{\maketitle} {
	\begin{center} 
		\begin{minipage}[t]{.8\textwidth}
			\textsf{\huge\bfseries \@title} 
		\end{minipage}%
		\begin{minipage}[t]{.2\textwidth}
			\raggedleft \vspace{-1.65em}
			\textsf{\small \@author} \vfill
			\textsf{\small \@date}
		\end{minipage}
		\par
	\end{center}

	\thispagestyle{empty}
	\pagestyle{fancy}
}
\makeatother

% disponi teoremi
\usepackage{tcolorbox}
\newtcolorbox[auto counter, number within=section]{theorem}[2][]{%
	colback=blue!10, 
	colframe=blue!40!black, 
	sharp corners=northwest,
	fonttitle=\sffamily\bfseries, 
	title=Teorema~\thetcbcounter: #2, 
	#1
}

% disponi definizioni
\newtcolorbox[auto counter, number within=section]{definition}[2][]{%
	colback=red!10,
	colframe=red!40!black,
	sharp corners=northwest,
	fonttitle=\sffamily\bfseries,
	title=Definizione~\thetcbcounter: #2,
	#1
}

% disponi codice
\usepackage{listings}
\usepackage[table]{xcolor}

\definecolor{codegreen}{rgb}{0,0.6,0}
\definecolor{codegray}{rgb}{0.5,0.5,0.5}
\definecolor{codepurple}{rgb}{0.58,0,0.82}
\definecolor{backcolour}{rgb}{0.95,0.95,0.92}

\lstdefinestyle{codestyle}{
		backgroundcolor=\color{black!5}, 
		commentstyle=\color{codegreen},
		keywordstyle=\bfseries\color{magenta},
		numberstyle=\sffamily\tiny\color{black!60},
		stringstyle=\color{green!50!black},
		basicstyle=\ttfamily\footnotesize,
		breakatwhitespace=false,         
		breaklines=true,                 
		captionpos=b,                    
		keepspaces=true,                 
		numbers=left,                    
		numbersep=5pt,                  
		showspaces=false,                
		showstringspaces=false,
		showtabs=false,                  
		tabsize=2
}

\lstdefinestyle{shellstyle}{
		backgroundcolor=\color{black!5}, 
		basicstyle=\ttfamily\footnotesize\color{black}, 
		commentstyle=\color{black}, 
		keywordstyle=\color{black},
		numberstyle=\color{black!5},
		stringstyle=\color{black}, 
		showspaces=false,
		showstringspaces=false, 
		showtabs=false, 
		tabsize=2, 
		numbers=none, 
		breaklines=true
}


\lstdefinelanguage{assembler}{
  keywords={AAA, AAD, AAM, AAS, ADC, ADCB, ADCW, ADCL, ADD, ADDB, ADDW, ADDL, AND, ANDB, ANDW, ANDL,
        ARPL, BOUND, BSF, BSFL, BSFW, BSR, BSRL, BSRW, BSWAP, BT, BTC, BTCB, BTCW, BTCL, BTR, 
        BTRB, BTRW, BTRL, BTS, BTSB, BTSW, BTSL, CALL, CBW, CDQ, CLC, CLD, CLI, CLTS, CMC, CMP,
        CMPB, CMPW, CMPL, CMPS, CMPSB, CMPSD, CMPSW, CMPXCHG, CMPXCHGB, CMPXCHGW, CMPXCHGL,
        CMPXCHG8B, CPUID, CWDE, DAA, DAS, DEC, DECB, DECW, DECL, DIV, DIVB, DIVW, DIVL, ENTER,
        HLT, IDIV, IDIVB, IDIVW, IDIVL, IMUL, IMULB, IMULW, IMULL, IN, INB, INW, INL, INC, INCB,
        INCW, INCL, INS, INSB, INSD, INSW, INT, INT3, INTO, INVD, INVLPG, IRET, IRETD, JA, JAE,
        JB, JBE, JC, JCXZ, JE, JECXZ, JG, JGE, JL, JLE, JMP, JNA, JNAE, JNB, JNBE, JNC, JNE, JNG,
        JNGE, JNL, JNLE, JNO, JNP, JNS, JNZ, JO, JP, JPE, JPO, JS, JZ, LAHF, LAR, LCALL, LDS,
        LEA, LEAVE, LES, LFS, LGDT, LGS, LIDT, LMSW, LOCK, LODSB, LODSD, LODSW, LOOP, LOOPE,
        LOOPNE, LSL, LSS, LTR, MOV, MOVB, MOVW, MOVL, MOVSB, MOVSD, MOVSW, MOVSX, MOVSXB,
        MOVSXW, MOVSXL, MOVZX, MOVZXB, MOVZXW, MOVZXL, MUL, MULB, MULW, MULL, NEG, NEGB, NEGW,
        NEGL, NOP, NOT, NOTB, NOTW, NOTL, OR, ORB, ORW, ORL, OUT, OUTB, OUTW, OUTL, OUTSB, OUTSD,
        OUTSW, POP, POPA, POPAD, POPF, POPFD, PUSH, PUSHA, PUSHAD, PUSHF, PUSHFD, RCL, RCLB, RCLW,
        RCLL, RCR, RCRB, RCRW, RCRL, RDMSR, RDPMC, RDTSC, REP, REPE, REPNE, RET, ROL, ROLB, ROLW,
        ROLL, ROR, RORB, RORW, RORL, SAHF, SAL, SALB, SALW, SALL, SAR, SARB, SARW, SARL, SBB,
        SBBB, SBBW, SBBL, SCASB, SCASD, SCASW, SETA, SETAE, SETB, SETBE, SETC, SETE, SETG, SETGE,
        SETL, SETLE, SETNA, SETNAE, SETNB, SETNBE, SETNC, SETNE, SETNG, SETNGE, SETNL, SETNLE,
        SETNO, SETNP, SETNS, SETNZ, SETO, SETP, SETPE, SETPO, SETS, SETZ, SGDT, SHL, SHLB, SHLW,
        SHLL, SHLD, SHR, SHRB, SHRW, SHRL, SHRD, SIDT, SLDT, SMSW, STC, STD, STI, STOSB, STOSD,
        STOSW, STR, SUB, SUBB, SUBW, SUBL, TEST, TESTB, TESTW, TESTL, VERR, VERW, WAIT, WBINVD,
        XADD, XADDB, XADDW, XADDL, XCHG, XCHGB, XCHGW, XCHGL, XLAT, XLATB, XOR, XORB, XORW, XORL},
  keywordstyle=\color{blue}\bfseries,
  ndkeywordstyle=\color{darkgray}\bfseries,
  identifierstyle=\color{black},
  sensitive=false,
  comment=[l]{\#},
  morecomment=[s]{/*}{*/},
  commentstyle=\color{purple}\ttfamily,
  stringstyle=\color{red}\ttfamily,
  morestring=[b]',
  morestring=[b]"
}

\lstset{language=assembler, style=codestyle}

% disponi sezioni
\usepackage{titlesec}

\titleformat{\section}
	{\sffamily\Large\bfseries} 
	{\thesection}{1em}{} 
\titleformat{\subsection}
	{\sffamily\large\bfseries}   
	{\thesubsection}{1em}{} 
\titleformat{\subsubsection}
	{\sffamily\normalsize\bfseries} 
	{\thesubsubsection}{1em}{}

% tikz
\usepackage{tikz}

% float
\usepackage{float}

% grafici
\usepackage{pgfplots}
\pgfplotsset{width=10cm,compat=1.9}

% disponi alberi
\usepackage{forest}

\forestset{
	rectstyle/.style={
		for tree={rectangle,draw,font=\large\sffamily}
	},
	roundstyle/.style={
		for tree={circle,draw,font=\large}
	}
}

% disponi algoritmi
\usepackage{algorithm}
\usepackage{algorithmic}
\makeatletter
\renewcommand{\ALG@name}{Algoritmo}
\makeatother

% disponi numeri di pagina
\usepackage{fancyhdr}
\fancyhf{} 
\fancyfoot[L]{\sffamily{\thepage}}

\makeatletter
\fancyhead[L]{\raisebox{1ex}[0pt][0pt]{\sffamily{\@title \ \@date}}} 
\fancyhead[R]{\raisebox{1ex}[0pt][0pt]{\sffamily{\@author}}}
\makeatother

\begin{document}
% sezione (data)
\section{Lezione del 27-09-24}

% stili pagina
\thispagestyle{empty}
\pagestyle{fancy}

% testo
\subsection{Divisioni}
La divisone è l'operazione più complessa fra le 4 operazioni aritmetiche fondamentali.
I risultati, di base, sono due: \textbf{quoziente} e \textbf{resto}.
Inoltre, l'operazione non è ben definita quando il divisore vale 0.

Facciamo innanzitutto delle considerazioni di dimensione dei risultati:
$$
X / Y \rightarrow (Q, R), \quad
0 \leq R \leq Y - 1, \quad
0 \leq Q \leq X
$$

In assembler, si assume il quoziente e il resto stiano sulla metà dei bit che rappresentano il dividendo.
Bisogna fare attenzione in quanto questo non è sempre il caso.

\subsubsection{DIVIDE}
\begin{itemize}
	\item \textbf{Formato:} \lstinline|DIV source|
	\item \textbf{Azione:} considera l'operando sorgente come un divisore, l'operando destinatario (implicito) come un dividendo, e effettua la divisione assumendo i numeri naturali. Nello specifico:
	\begin{itemize}
	\item Sorgente a 8 bit, si ha $\text{AL} = \text{AX} \div \text{source}$, e $ \text{AH} = \text{AX} \mod \text{source} $;
	\item Sorgente a 16 bit, si ha $\text{AX} = \text{DX\_AX} \div \text{source}$, e $ \text{DX} = \text{DX\_AX} \mod \text{source} $;
	\item Sorgente a 32 bit, si ha $\text{EAX} = \text{EDX\_EAX} \div \text{source}$, e $ \text{EDX} = \text{EDX\_EAX} \mod \text{source} $;
	\end{itemize}
		Nel caso il quoziente non sia esprimibile su un numero di bit pari a quello del divisore, allora si genera un'eccezione interna, che mette in esecuzione un sottoprogramma.
		Da lì in poi i risultati generati non sono più attendibili
	\item \textbf{Flag:} imposta tutti i bit, ma non è attendibile. 
\end{itemize}

		\begin{table}[H]
		\center \rowcolors{2}{white}{black!10}
			\begin{tabular} { c | p{10cm} }
				\bfseries Operandi & \bfseries Esempi \\
				\hline
				Memoria & \lstinline!DIVB (\%ESI)	\# AX destinazione implicita! \\ 
				Registro Generale & \lstinline|DIV \%ECX	\# EDX\_EAX destinazione implicita|
			\end{tabular}
		\end{table}

Attenzione: la destinazione implicita non è quella che va a contenere il risultato, ma quella che contiene il dividendo.
Negli esempi, le destinazioni quoziente resto sono rispettivamente AL e AH, EAX e EDX.

\subsubsection{INTEGER DIVIDE}
\begin{itemize}
	\item \textbf{Formato:} \lstinline|IMUL source|
	\item \textbf{Azione:} considera l'operando sorgente come un divisore, l'operando destinatario (implicito) come un dividendo, e effettua la divisione assumendo i numeri interi. Nello specifico:
	\begin{itemize}
	\item Sorgente a 8 bit, si ha $\text{AL} = \text{AX} / \text{source}$, e $ \text{AH} = \text{AX} \mod \text{source} $;
	\item Sorgente a 16 bit, si ha $\text{AX} = \text{DX\_AX} / \text{source}$, e $ \text{DX} = \text{DX\_AX} \mod \text{source} $;
	\item Sorgente a 32 bit, si ha $\text{EAX} = \text{EDX\_EAX} / \text{source}$, e $ \text{EDX} = \text{EDX\_EAX} \mod \text{source} $;
	\end{itemize}
	\item \textbf{Flag:} li imposta tutti, ma non è attendibile.
\end{itemize}

		\begin{table}[H]
		\center \rowcolors{2}{white}{black!10}
			\begin{tabular} { c | p{10cm} }
				\bfseries Operandi & \bfseries Esempi \\
				\hline
				Memoria & \lstinline|IDIVB (\%ESI)	\# AX destinazione implicita| \\ 
				Registro Generale & \lstinline|IDIV \%ECX	\# EDX\_EAX destinazione implicita|
			\end{tabular}
		\end{table}

Bisogna stare attenti ai segni della divisione intera.
Nella divisione intera il resto ha sempre il segno del dividendo, ed è minore in modulo del divisore.
Ciò significa che il quoziente si approssima sempre all'intero più vicino allo zero (\textit{per troncamento}).
Ad esempio, $-7 \ \mathrm{idiv} \ 3 = -2, -1$ e $7 \ \mathrm{idiv} \ -3 = -2, +1$.

\par\medskip
\noindent
\textbf{\textsf{Funzionamento delle DIVIDE e INTEGER DIVIDE}} \\
Esistono quindi, come per le moltiplicazioni, tre tipi di divisione, con operando e destinatario impliciti:
\begin{itemize}
	\item Sorgente a 8 bit, si ha $\text{AL} = \text{AX} / \text{source}$, e $ \text{AH} = \text{AX} \mod \text{source} $;
	\item Sorgente a 16 bit, si ha $\text{AX} = \text{DX\_AX} / \text{source}$, e $ \text{DX} = \text{DX\_AX} \mod \text{source} $;
	\item Sorgente a 32 bit, si ha $\text{EAX} = \text{EDX\_EAX} / \text{source}$, e $ \text{EDX} = \text{EDX\_EAX} \mod \text{source} $;
\end{itemize}

In tabella questo significa:

\begin{table}[h!]
	\center \rowcolors{2}{white}{black!10}
	\begin{tabular} { c | c | c | c | c }
		\bfseries Dim. sorgente (divisore) & \bfseries Dim. dividendo & \bfseries Dividendo & \bfseries Quoziente & \bfseries Resto \\ 
		\hline 
		8 bit & 16 bit & AX & AL & AH \\ 
		16 bit & 32 bit & DX\_AX & AX & DX \\ 
		32 bit & 64 bit & EDX\_EAX & EAX & EDX
	\end{tabular}
\end{table}

Se il quoziente non sta nel numero di bit previsto, viene sollevata un'eccezione, e il programma va in HALT.
Bisogna quindi decidere quali versioni usare tenendo conto delle dimensioni dei possibili quoziente.
Questo è importante in quanto non è cosi raro avere divisioni dove il quoziente non sta nella metà dei bit del dividendo, ad esempio:

\begin{lstlisting}[language=assembler,style=codestyle]	
MOV $3, %CL
MOV $15000, %AX
DIV %CL	# come metto 5000 su una locazione da 8 bit?
\end{lstlisting}

per risolvere il problema, dobbiamo costringere il processore ad usare un altro tipo di divisione, quindi:
\begin{lstlisting}[language=assembler,style=codestyle]	
MOV $3, %CX
MOV $15000, %AX
MOV $0, %DX	# devo ripulire DX, verra' usato il dividendo DX_AX 
DIV %CX	# il risultato va in AX, tutto bene 
\end{lstlisting}

\subsection{Note conclusive su moltiplicazioni e divisioni}
Dobbiamo quindi ricordarci, riguardo a moltiplicazioni e divisioni, di:
\begin{itemize}
	\item Scegliere con cura la versione che usiamo (sopratutto nel caso di divisioni dove il quoziente potrebbe non stare nella metà del numero di bit del dividendo);
	\item Azzerare di azzerare i registri DX o EDX prima della divisione, se è a più di 8 bit;
	\item Ricordare che il contenuto di DX o EDX viene modificato per operazioni su più di 8 bit.
\end{itemize}

\subsection{Estensione di campo}
Attraverso l'estensione di campo si rappresenta lo stesso numero su più cifre.
Questo è banale sui naturali (si aggiunge uno zero), ma più complicato per gli interi.
In questo caso si estende con il bit più significativo (quello di segno).

\subsubsection{CONVERT BYTE TO WORD}
\begin{itemize}
	\item \textbf{Formato:} \lstinline|CBX|
	\item \textbf{Azione:} interpreta il contenuto di AL come un numero intero a 8 bit, la rappresenta su 16 bit e quindi lo memorizza in AX.
	\item \textbf{Flag:} nessuno.
\end{itemize}

\subsubsection{CONVERT WORD TO DOUBLEWORD}
\begin{itemize}
	\item \textbf{Formato:} \lstinline|CWDE|
	\item \textbf{Azione:} interpreta il contenuto di AX come un numero intero a 16 bit, la rappresenta su 32 bit e quindi lo memorizza in EAX.
	\item \textbf{Flag:} nessuno.
\end{itemize}

Poniamo ad esempio di voler sommare due interi, uno in AX e l'altro in EBX:
\begin{lstlisting}[language=assembler,style=codestyle]	
MOV $-5, %AX
MOV $100000, $EBX
CWDE
ADD %EAX, %EBX
\end{lstlisting}

\subsection{Istruzioni di traslazione e rotazione}
Queste istruzioni variano l'ordine dei bit in un operando destinatario.
Hanno due formati: \lstinline|OCPODE source, destination| o \lstinline|OPCODE destination|.

Quando si specifica un sorgente, esso rappresenta il numero di iterazioni per cui si ripete l'operazione.
Il sorgente può essere ad indirizzamento immediato o essere il registro CL.
Inoltre, deve essere $\leq 31$ (sarebbe inutile fare $\geq32$ trasformazioni di 32 bit).
Quando è omesso, il sorgente vale di default 1.

\subsubsection{SHIFT LOGICAL LEFT}
\begin{itemize}
	\item \textbf{Formato:} \lstinline|SHL source, destination|
	\item \textbf{Azione:} interpreta l'operando sorgente come un naturale $n$, e per $n$ iterazioni:
		\begin{itemize}
			\item Sostituisce il bit in CF con il MSB;
			\item Sostituisce ogni bit (tranne il LSB) con il bit immediatamente a destra  (il meno significativo);
			\item Sostituisce il LSB con 0.
		\end{itemize}
	\item \textbf{Flag:} nessuno.
\end{itemize}

		\begin{table}[H]
		\center \rowcolors{2}{white}{black!10}
			\begin{tabular} { c | p{5cm} }
				\bfseries Operandi & \bfseries Esempi \\
				\hline
				Immediato, Registro Generale & \lstinline|SHL \$1, \%EAX| \\
				Immediato, Memoria & \lstinline|SHLB \$7, 0x00002000| \\
				Registro CL, Registro Generale & \lstinline|SHL \%CL, \%EAX| \\
				Registro CL, Memoria & \lstinline|SHLL \%CL, (\%EDI)| \\
				Memoria & \lstinline|SHLL (\%EDI)| \\ 
				Registro Generale & \lstinline|SHL \%AX|
			\end{tabular}
		\end{table}

La SHL è utile per effettuare moltiplicazioni per 2 (shift a sinistra in binario significa $\times 2$), tranne nei casi in cui il prodotto non sta sul numero di bit del destinatario.

Per questo si controlla il CF, facendo però attenzione che per $n$ iterazioni (date dal sorgente) vengono effettuati $n$ sovrascrizioni del CF.
Ergo, se la moltiplicazione fallisce, non sappiamo \textit{quando} fallisce.

\subsubsection{SHIFT ARITHMETIC LEFT}
\begin{itemize}
	\item \textbf{Formato:} \lstinline|SAL source, destination|
	\item \textbf{Azione:} è identica alla SHL. 
		Quindi equivale a moltiplicare per $2^\text{source}$.
	\item \textbf{Flag:} nessuno.
\end{itemize}

Esiste come duale della SAR, ma in questo caso non deve fare nulla di diverso dalla SHL.

\subsubsection{SHIFT LOGICAL RIGHT}
\begin{itemize}
	\item \textbf{Formato:} \lstinline|SHR source, destination|
	\item \textbf{Azione:} interpreta l'operando sorgente come un naturale $n$, e per $n$ iterazioni:
		\begin{itemize}
			\item Sostituisce il bit in CF con il LSB;
			\item Sostituisce ogni bit (tranne il MSB) con il bit immediatamente a sinistra (il più significativo);
			\item Sostituisce il MSB con 0.
		\end{itemize}
	\item \textbf{Flag:} nessuno.
\end{itemize}

		\begin{table}[H]
		\center \rowcolors{2}{white}{black!10}
			\begin{tabular} { c | p{5cm} }
				\bfseries Operandi & \bfseries Esempi \\
				\hline
				Immediato, Registro Generale & \lstinline|SHR \$1, \%EAX| \\
				Immediato, Memoria & \lstinline|SHRB \$7, 0x00002000| \\
				Registro CL, Registro Generale & \lstinline|SHR \%CL, \%EAX| \\
				Registro CL, Memoria & \lstinline|SHRL \%CL, (\%EDI)| \\
				Memoria & \lstinline|SHRL (\%EDI)| \\ 
				Registro Generale & \lstinline|SHR \%AX|
			\end{tabular}
		\end{table}

La SHR, come la SHL, è utile per effettuare divisioni per 2 (shift a destra in binario significa $\div 2$), concessa approssimazione del bit perso, tranne nei casi in cui il numero è un intero (lo 0 al MSB corrompe il segno). 
Per questo motivo si definisce la:

\subsubsection{SHIFT ARITHMETIC RIGHT}
\begin{itemize}
	\item \textbf{Formato:} \lstinline|SAR source, destination|
	\item \textbf{Azione:} è identica alla SHR, ma non sostituisce il MSB con 0, lasciandolo tale.
		Questo equivale a dividere per $2^\text{source}$.
	\item \textbf{Flag:} nessuno.
\end{itemize}

La SAR ci permette di dividere velocemente interi per 2, come avremmo fatto sui naturali con la SHR.

\subsubsection{Divisioni intere}
Le IDIV e SAR approssimano diversamente: la IDIV approssima per troncamento, mentre la SAR approssima sempre a sinistra.
Quindi, IDIV e SAR danno lo stesso quoziente solo quando il dividendo è positivo, o il resto nullo.

\subsection{Istruzioni di rotazione}
Le istruzioni di rotazione ruotano i bit, cioè effettuano uno shift con rientro dei bit in uscita dal lato opposto, con la possibilità di includere o meno CF nella rotazione.

\subsubsection{ROTATE LEFT}
\begin{itemize}
	\item \textbf{Formato:} \lstinline|ROL source, destination|
	\item \textbf{Azione:} interpreta l'operando sorgente come un naturale $n$, e per $n$ iterazioni ruota verso sinistra senza usare il carry.
	\item \textbf{Flag:} nessuno.
\end{itemize}

		\begin{table}[H]
		\center \rowcolors{2}{white}{black!10}
			\begin{tabular} { c | p{5cm} }
				\bfseries Operandi & \bfseries Esempi \\
				\hline
				Immediato, Registro Generale & \lstinline|ROL \$1, \%EAX| \\
				Immediato, Memoria & \lstinline|ROLB \$7, 0x00002000| \\
				Registro CL, Registro Generale & \lstinline|ROL \%CL, \%EAX| \\
				Registro CL, Memoria & \lstinline|ROLL \%CL, (\%EDI)| \\
				Memoria & \lstinline|ROLL (\%EDI)| \\ 
				Registro Generale & \lstinline|ROL \%AX|
			\end{tabular}
		\end{table}

\subsubsection{ROTATE RIGHT}
\begin{itemize}
	\item \textbf{Formato:} \lstinline|ROR source, destination|
	\item \textbf{Azione:} interpreta l'operando sorgente come un naturale $n$, e per $n$ iterazioni ruota verso destra senza usare il carry.
	\item \textbf{Flag:} nessuno.
\end{itemize}

		\begin{table}[H]
		\center \rowcolors{2}{white}{black!10}
			\begin{tabular} { c | p{5cm} }
				\bfseries Operandi & \bfseries Esempi \\
				\hline
				Immediato, Registro Generale & \lstinline|ROR \$1, \%EAX| \\
				Immediato, Memoria & \lstinline|RORB \$7, 0x00002000| \\
				Registro CL, Registro Generale & \lstinline|ROR \%CL, \%EAX| \\
				Registro CL, Memoria & \lstinline|RORL \%CL, (\%EDI)| \\
				Memoria & \lstinline|RORL (\%EDI)| \\ 
				Registro Generale & \lstinline|ROR \%AX|
			\end{tabular}
		\end{table}

\subsubsection{ROTATE CARRY LEFT}
\begin{itemize}
	\item \textbf{Formato:} \lstinline|RCL source, destination|
	\item \textbf{Azione:} interpreta l'operando sorgente come un naturale $n$, e per $n$ iterazioni ruota verso sinistra usando il carry.
	\item \textbf{Flag:} imposta il carry assumendolo a sinistra del MSB.
\end{itemize}

		\begin{table}[H]
		\center \rowcolors{2}{white}{black!10}
			\begin{tabular} { c | p{5cm} }
				\bfseries Operandi & \bfseries Esempi \\
				\hline
				Immediato, Registro Generale & \lstinline|RCL \$1, \%EAX| \\
				Immediato, Memoria & \lstinline|RCLB \$7, 0x00002000| \\
				Registro CL, Registro Generale & \lstinline|RCL \%CL, \%EAX| \\
				Registro CL, Memoria & \lstinline|RCLL \%CL, (\%EDI)| \\
				Memoria & \lstinline|RCLL (\%EDI)| \\ 
				Registro Generale & \lstinline|RCL \%AX|
			\end{tabular}
		\end{table}

\subsubsection{ROTATE CARRY RIGHT}
\begin{itemize}
	\item \textbf{Formato:} \lstinline|RCR source, destination|
	\item \textbf{Azione:} interpreta l'operando sorgente come un naturale $n$, e per $n$ iterazioni ruota verso destra usando il carry.
	\item \textbf{Flag:} imposta il carry assumendolo a destra del LSB.
\end{itemize}

		\begin{table}[H]
		\center \rowcolors{2}{white}{black!10}
			\begin{tabular} { c | p{5cm} }
				\bfseries Operandi & \bfseries Esempi \\
				\hline
				Immediato, Registro Generale & \lstinline|RCR \$1, \%EAX| \\
				Immediato, Memoria & \lstinline|RCRB \$7, 0x00002000| \\
				Registro CL, Registro Generale & \lstinline|RCR \%CL, \%EAX| \\
				Registro CL, Memoria & \lstinline|RCRL \%CL, (\%EDI)| \\
				Memoria & \lstinline|RCRL (\%EDI)| \\ 
				Registro Generale & \lstinline|RCR \%AX|
			\end{tabular}
		\end{table}

\subsection{Istruzioni logiche}
Queste istruzioni applicano gli operatori dell'algebra di Boole, e solitamente modificano flag.

\subsubsection{NOT}
\begin{itemize}
	\item \textbf{Formato:} \lstinline|NOT destination|
	\item \textbf{Azione:} modifica il destinatario applicandogli il NOT bit a bit. 
	\item \textbf{Flag:} nessuno. 
\end{itemize}

		\begin{table}[H]
		\center \rowcolors{2}{white}{black!10}
			\begin{tabular} { c | p{5cm} }
				\bfseries Operandi & \bfseries Esempi \\
				\hline
				Memoria & \lstinline|NOTL (\%ESI)| \\ 
				Registro Generale & \lstinline|NOT \%CX| 
			\end{tabular}
		\end{table}

\subsubsection{AND}
\begin{itemize}
	\item \textbf{Formato:} \lstinline|AND source, destination|
	\item \textbf{Azione:} modifica il destinatario applicando l'AND bit a bit degli operandi. 
	\item \textbf{Flag:} modifica tutti i flag (annulla CF e OF).
\end{itemize}

		\begin{table}[H]
		\center \rowcolors{2}{white}{black!10}
			\begin{tabular} { c | p{5cm} }
				\bfseries Operandi & \bfseries Esempi \\
				\hline
				Memoria, Registro Generale & \lstinline|AND 0x00002000, \%EDX| \\ 
				Registro Generale, Memoria & \lstinline|AND \%CL, 0x12AB1024| \\ 
				Registro Generale, Registro Generale & \lstinline|AND \%AX, \%DX| \\ 
				Immediato, Memoria & \lstinline|AND 5x5B, (\%EDI)| \\ 
				Immediato, Registro Generale & \lstinline|AND \$0x45AB54A3, \%EAX|
			\end{tabular}
		\end{table}

\subsubsection{OR}
\begin{itemize}
	\item \textbf{Formato:} \lstinline|OR source, destination|
	\item \textbf{Azione:} modifica il destinatario applicando l'OR bit a bit degli operandi. 
	\item \textbf{Flag:} modifica tutti i flag (annulla CF e OF).
\end{itemize}

		\begin{table}[H]
		\center \rowcolors{2}{white}{black!10}
			\begin{tabular} { c | p{5cm} }
				\bfseries Operandi & \bfseries Esempi \\
				\hline
				Memoria, Registro Generale & \lstinline|OR 0x00002000, \%EDX| \\ 
				Registro Generale, Memoria & \lstinline|OR \%CL, 0x12AB1024| \\ 
				Registro Generale, Registro Generale & \lstinline|OR \%AX, \%DX| \\ 
				Immediato, Memoria & \lstinline|OR 5x5B, (\%EDI)| \\ 
				Immediato, Registro Generale & \lstinline|OR \$0x45AB54A3, \%EAX|
			\end{tabular}
		\end{table}

\subsubsection{XOR}
\begin{itemize}
	\item \textbf{Formato:} \lstinline|XOR source, destination|
	\item \textbf{Azione:} modifica il destinatario applicando l'OR bit a bit degli operandi. 
	\item \textbf{Flag:} modifica tutti i flag (annulla CF e OF).
\end{itemize}

		\begin{table}[H]
		\center \rowcolors{2}{white}{black!10}
			\begin{tabular} { c | p{5cm} }
				\bfseries Operandi & \bfseries Esempi \\
				\hline
				Memoria, Registro Generale & \lstinline|XOR 0x00002000, \%EDX| \\ 
				Registro Generale, Memoria & \lstinline|XOR \%CL, 0x12AB1024| \\ 
				Registro Generale, Registro Generale & \lstinline|XOR \%AX, \%DX| \\ 
				Immediato, Memoria & \lstinline|XOR 5x5B, (\%EDI)| \\ 
				Immediato, Registro Generale & \lstinline|XOR \$0x45AB54A3, \%EAX|
			\end{tabular}
		\end{table}

\subsubsection{Uso delle istruzioni logiche}
Le istruzioni logiche vengono usate per operare su singoli bit degli operandi, usando uno specifico operatore sorgente immediato detto maschera (\textbf{bitmask}).
Nello specifico:
\begin{itemize}
	\item \textbf{AND:} 
		\begin{itemize}
			\item si usa per testare singoli bit di un operando.
			Ad esempio, si può implementare un salto condizionale se il quinto bit di AL vale zero:
			\begin{lstlisting}[language=assembler,style=codestyle]	
AND $0x20, %AL	# 0x20 = 00100000
JZ # vale zero
\end{lstlisting} 
			\item si usa per resettare singoli bit di un operando.
			Ad esempio, si può resettare il sesto bit di BH:
			\begin{lstlisting}[language=assembler,style=codestyle]	
AND $0xBF, $BH	# 0xBF = 10111111
\end{lstlisting}
			\item si usa per l'estensione di operandi \textit{naturali}.
				Ad esempio, si possono sommare due numeri naturali, di cui uno in AL e l'altro in EBX:
				\begin{lstlisting}[language=assembler,style=codestyle]	
MOV $5, $AL
MOV $100000, %EBX
AND $0x000000FF, $EAX
ADD %EAX, %EBX	
\end{lstlisting}
		\end{itemize} 
	\item \textbf{OR:} si usa per settare singoli bit di un operando.
		Ad esempio, si può settare il quarto bit di CL:
		\begin{lstlisting}[language=assembler,style=codestyle]	
OR $0x10, %CL	# =x10 = 00010000
\end{lstlisting}
	\item \textbf{XOR:}
		\begin{itemize}
			\item si usa per invertire singoli bit.
		Ad esempio, si può invertire il quinto bit del registro AH:
		\begin{lstlisting}[language=assembler,style=codestyle]	
XOR $0x20, %AH	# 0x20 = 00100000
\end{lstlisting}
	\item si usa per resettare registri.
		Ad esempio, si può resettare EAX come:
		\begin{lstlisting}[language=assembler,style=codestyle]	
XOR %EAX, %EAX	# equivale a dire MOV $0, %EAX, ma occupa 
								# 1 byte invece di 5
\end{lstlisting}
		\end{itemize}
\end{itemize}

\subsection{Istruzioni di controllo}
Le istruzioni di controllo permettono di alterare il flusso del programma, che altrimenti scorrerebbe normalmente in sequenza (le istruzioni vengono eseguite come vengono lette in memoria).

Conosciamo il ciclo fetch-execute: il processore carica un'istruzione, incrementa EIP, e la esegue.
Alcune istruzioni alterano il valore di EIP, implementando quindi alterazioni del flusso di esecuzione:
\begin{itemize}
	\item \textbf{Istruzioni di salto:} JMP, Jcon;
	\item \textbf{Istruzioni di gestione sottoprgrammi}: CALL, RET.
\end{itemize}

\subsubsection{JUMP}
\begin{itemize}
	\item \textbf{Formato:} \lstinline|JMP  \%EIP +/- displacement|, \lstinline|JMP *extended\_register|, \lstinline|JMP *memory|
	\item \textbf{Azione:} calcola un'indrizzo di salto e lo immette nel registro EIP. 
	\item \textbf{Flag:} nessuno. 
\end{itemize}

Solitamente le istruzioni di salto si riferiscono ad un nome simbolico, ed è quindi compito dell'assemblatore ricondurre la sintassi ad una delle forme sopra riportate.

\subsubsection{JUMP if CONDITION MET}
\begin{itemize}
	\item \textbf{Formato:} \lstinline|Jcon \%EIP +/- displacement|
	\item \textbf{Azione:} esamina il contenuto dei flag.
		Se da questo esame risulta che la condizione \textit{con} è soddisfatta, si comporta come \lstinline|JMP \%EIP +/- displacement|, altrimenti non fa nulla.
	\item \textbf{Flag:} nessuno. 
\end{itemize}

I prossimi paragrafi riguardano tutti i di condizione supportati.

\subsubsection{Condizioni sui flag}
Esistono le seguenti condizioni sui singoli flag:

\begin{table}[h!]
	\center \rowcolors{1}{white}{black!5}
	\begin{tabular} { c  p{10cm} }
		\bfseries Condizione & \bfseries Funzionamento \\
		\hline 
		JZ & Jump If Zero, la condizione è soddisfatta se ZF è impostato, ergo se il risultato dell'istruzione precedente è stato 0. \\ 
		JNZ & Jump If Not Zero, la condizione è soddisfatta se ZF non è impostato, ergo se il risultato dell'istruzione precedente non è stato 0. \\ 
		JC & Jump if Carry, la condizione è soddisfatta se CF è impostato. \\
		JNC & Jump if No Carry, la condizione è soddisfatta se CF non è impostato. \\ 
		JO & Jump if Overflow, la condizione è soddisfatta se OF è impostato. \\
		JNO & Jump if No Overflow, la condizione è soddisfatta se OF non è impostato. \\ 
		JS & Jump if Sign, la condizione è soddisfatta se SF è impostato. \\
		JNS & Jump if No Sign, la condizione è soddisfatta se SF non è impostato. \\ 
	\end{tabular}
\end{table}

\par\medskip
\noindent
\textbf{\textsf{Esempi}} \\
\begin{itemize}
	\item 
\begin{lstlisting}[language=assembler,style=codestyle]	
ADD %AX, %BX
JC ...
# continua
\end{lstlisting}
Se la somma dei contenuti di AX e BX presi come naturali non è rappresentabile su 16 bit, salta.

	\item 
\begin{lstlisting}[language=assembler,style=codestyle]	
ADD %AX, %BX
JO ...
# continua
\end{lstlisting}
Se la somma dei contenuti di AX e BX presi come interi non è rappresentabile su 16 bit, salta.

	\item 
\begin{lstlisting}[language=assembler,style=codestyle]	
SUB %AL, %BL
JS ...
# continua
\end{lstlisting}
Se la somma differenza dei contenuti di BL ed AL (in quest'ordine) presi come interi è negativa, salta.
\end{itemize}

\subsubsection{Condizioni sui naturali}
Esistono le seguenti condizioni sui confronti fra naturali:

\begin{table}[h!]
	\center \rowcolors{2}{white}{black!5}
	\begin{tabular} { c  p{10cm} }
		\bfseries Condizione & \bfseries Funzionamento \\
		\hline 
		JE & Jump if Equal, la condizione è soddisfatta se ZF contiene 1, cioè dopo CMP su due numeri uguali. \\
		JNE & Jump if Not Equal, la condizione è soddisfatta se ZF contiene 0, cioè dopo CMP su due numeri non uguali. \\ 
		JA & Jump if Above, la condizione è soddisfatta se CF contiene 0 e ZF contiene 1, cioè dopo CMP su un destinatario maggiore del sorgente. \\
		JAE & Jump if Above or Equal, la condizione è soddisfatta se CF contiene 0, cioè dopo CMP su un destinatario maggiore o uguale del sorgente. \\ 
		JB & Jump if Below, la condizione è soddisfatta se CF contiene 1, cioè dopo CMP su un destinatario minore del sorgente. \\
		JBE & Jump if Below or Equal, la condizione è soddisfatta se CF contiene 1 o ZF contiene 1, cioè dopo CMP su un destinatario minore o uguale del sorgente. \\ 
	\end{tabular}
\end{table}

Tutte queste condizioni seguono sempre una CMP, che aggiorna i flag in modo da permettere il confronto.
I risultati dei confronti possono sempre evincersi dai flag.

\par\medskip
\noindent
\textbf{\textsf{Esempi}} \\
\begin{itemize}
	\item 
\begin{lstlisting}[language=assembler,style=codestyle]	
CMP %AX, %BX
JAE ...
# continua
\end{lstlisting}
Se BX è maggiore o uguale di AX, presi come naturali, salta.

	\item 
\begin{lstlisting}[language=assembler,style=codestyle]	
CMP %EDX, %ECX
JB ...
# continua
\end{lstlisting}
Se ECX è minore stretto di EDX, presi come naturali, salta.
\end{itemize}

\subsubsection{Condizioni sugli interi}
Esistono le seguenti condizioni sui confronti fra interi:

\begin{table}[h!]
	\center \rowcolors{2}{white}{black!5}
	\begin{tabular} { c  p{10cm} }
		\bfseries Condizione & \bfseries Funzionamento \\
		\hline 
		JE & Jump if Equal, la condizione è soddisfatta se ZF contiene 1, cioè dopo CMP su due numeri uguali. \\
		JNE & Jump if Not Equal, la condizione è soddisfatta se ZF contiene 0, cioè dopo CMP su due numeri non uguali. \\ 
		JG & Jump if Greater, la condizione è soddisfatta se ZF contiene 0 o se SF è uguale a OF, cioè dopo CMP su un destinatario maggiore del sorgente. \\
		JGE & Jump if Greater or Equal, la condizione è soddisfatta se SF è uguale a OF, cioè dopo CMP su un destinatario maggiore o uguale del sorgente. \\ 
		JB & Jump if Less, la condizione è soddisfatta se SF è diverso da OF, cioè dopo CMP su un destinatario minore del sorgente. \\
		JBE & Jump if Less or Equal, la condizione è soddisfatta se ZF contiene 1 o se Sf è diverso da OF, cioè dopo CMP su un destinatario minore o uguale del sorgente. \\ 
	\end{tabular}
\end{table}

Come prima, queste operazioni seguono sempre una CMP ed evincono il risultato del confronto dai flag.

\par\medskip
\noindent
\textbf{\textsf{Esempi}} \\
\begin{itemize}
	\item 
\begin{lstlisting}[language=assembler,style=codestyle]	
CMP %AX, %BX
JGE ...
# continua
\end{lstlisting}
Se BX è maggiore o uguale di AX, presi come interi, salta.

	\item 
\begin{lstlisting}[language=assembler,style=codestyle]	
CMP %EDX, %ECX
JL ...
# continua
\end{lstlisting}
Se ECX è minore stretto di EDX, presi come interi, salta.
\end{itemize}
\end{document}
