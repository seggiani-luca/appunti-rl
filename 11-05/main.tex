\documentclass[a4paper,11pt]{article}
\usepackage[a4paper, margin=8em]{geometry}

% usa i pacchetti per la scrittura in italiano
\usepackage[french,italian]{babel}
\usepackage[T1]{fontenc}
\usepackage[utf8]{inputenc}
\frenchspacing 

% usa i pacchetti per la formattazione matematica
\usepackage{amsmath, amssymb, amsthm, amsfonts}

% usa altri pacchetti
\usepackage{gensymb}
\usepackage{hyperref}
\usepackage{standalone}

\usepackage{colortbl}

\usepackage{xstring}
\usepackage{karnaugh-map}

% imposta il titolo
\title{Appunti Reti Logiche}
\author{Luca Seggiani}
\date{2024}

% imposta lo stile
% usa helvetica
\usepackage[scaled]{helvet}
% usa palatino
\usepackage{palatino}
% usa un font monospazio guardabile
\usepackage{lmodern}

\renewcommand{\rmdefault}{ppl}
\renewcommand{\sfdefault}{phv}
\renewcommand{\ttdefault}{lmtt}

% circuiti
\usepackage{circuitikz}
\usetikzlibrary{babel}

% disponi il titolo
\makeatletter
\renewcommand{\maketitle} {
	\begin{center} 
		\begin{minipage}[t]{.8\textwidth}
			\textsf{\huge\bfseries \@title} 
		\end{minipage}%
		\begin{minipage}[t]{.2\textwidth}
			\raggedleft \vspace{-1.65em}
			\textsf{\small \@author} \vfill
			\textsf{\small \@date}
		\end{minipage}
		\par
	\end{center}

	\thispagestyle{empty}
	\pagestyle{fancy}
}
\makeatother

% disponi teoremi
\usepackage{tcolorbox}
\newtcolorbox[auto counter, number within=section]{theorem}[2][]{%
	colback=blue!10, 
	colframe=blue!40!black, 
	sharp corners=northwest,
	fonttitle=\sffamily\bfseries, 
	title=Teorema~\thetcbcounter: #2, 
	#1
}

% disponi definizioni
\newtcolorbox[auto counter, number within=section]{definition}[2][]{%
	colback=red!10,
	colframe=red!40!black,
	sharp corners=northwest,
	fonttitle=\sffamily\bfseries,
	title=Definizione~\thetcbcounter: #2,
	#1
}

% disponi codice
\usepackage{listings}
\usepackage[table]{xcolor}

\definecolor{codegreen}{rgb}{0,0.6,0}
\definecolor{codegray}{rgb}{0.5,0.5,0.5}
\definecolor{codepurple}{rgb}{0.58,0,0.82}
\definecolor{backcolour}{rgb}{0.95,0.95,0.92}

\lstdefinestyle{codestyle}{
		backgroundcolor=\color{black!5}, 
		commentstyle=\color{codegreen},
		keywordstyle=\bfseries\color{magenta},
		numberstyle=\sffamily\tiny\color{black!60},
		stringstyle=\color{green!50!black},
		basicstyle=\ttfamily\footnotesize,
		breakatwhitespace=false,         
		breaklines=true,                 
		captionpos=b,                    
		keepspaces=true,                 
		numbers=left,                    
		numbersep=5pt,                  
		showspaces=false,                
		showstringspaces=false,
		showtabs=false,                  
		tabsize=2
}

\lstdefinestyle{shellstyle}{
		backgroundcolor=\color{black!5}, 
		basicstyle=\ttfamily\footnotesize\color{black}, 
		commentstyle=\color{black}, 
		keywordstyle=\color{black},
		numberstyle=\color{black!5},
		stringstyle=\color{black}, 
		showspaces=false,
		showstringspaces=false, 
		showtabs=false, 
		tabsize=2, 
		numbers=none, 
		breaklines=true
}


\lstdefinelanguage{assembler}{ 
  keywords={AAA, AAD, AAM, AAS, ADC, ADCB, ADCW, ADCL, ADD, ADDB, ADDW, ADDL, AND, ANDB, ANDW, ANDL,
        ARPL, BOUND, BSF, BSFL, BSFW, BSR, BSRL, BSRW, BSWAP, BT, BTC, BTCB, BTCW, BTCL, BTR, 
        BTRB, BTRW, BTRL, BTS, BTSB, BTSW, BTSL, CALL, CBW, CDQ, CLC, CLD, CLI, CLTS, CMC, CMP,
        CMPB, CMPW, CMPL, CMPS, CMPSB, CMPSD, CMPSW, CMPXCHG, CMPXCHGB, CMPXCHGW, CMPXCHGL,
        CMPXCHG8B, CPUID, CWDE, DAA, DAS, DEC, DECB, DECW, DECL, DIV, DIVB, DIVW, DIVL, ENTER,
        HLT, IDIV, IDIVB, IDIVW, IDIVL, IMUL, IMULB, IMULW, IMULL, IN, INB, INW, INL, INC, INCB,
        INCW, INCL, INS, INSB, INSD, INSW, INT, INT3, INTO, INVD, INVLPG, IRET, IRETD, JA, JAE,
        JB, JBE, JC, JCXZ, JE, JECXZ, JG, JGE, JL, JLE, JMP, JNA, JNAE, JNB, JNBE, JNC, JNE, JNG,
        JNGE, JNL, JNLE, JNO, JNP, JNS, JNZ, JO, JP, JPE, JPO, JS, JZ, LAHF, LAR, LCALL, LDS,
        LEA, LEAVE, LES, LFS, LGDT, LGS, LIDT, LMSW, LOCK, LODSB, LODSD, LODSW, LOOP, LOOPE,
        LOOPNE, LSL, LSS, LTR, MOV, MOVB, MOVW, MOVL, MOVSB, MOVSD, MOVSW, MOVSX, MOVSXB,
        MOVSXW, MOVSXL, MOVZX, MOVZXB, MOVZXW, MOVZXL, MUL, MULB, MULW, MULL, NEG, NEGB, NEGW,
        NEGL, NOP, NOT, NOTB, NOTW, NOTL, OR, ORB, ORW, ORL, OUT, OUTB, OUTW, OUTL, OUTSB, OUTSD,
        OUTSW, POP, POPL, POPW, POPB, POPA, POPAD, POPF, POPFD, PUSH, PUSHL, PUSHW, PUSHB, PUSHA, 
				PUSHAD, PUSHF, PUSHFD, RCL, RCLB, RCLW, MOVSL, MOVSB, MOVSW, STOSL, STOSB, STOSW, LODSB, LODSW,
				LODSL, INSB, INSW, INSL, OUTSB, OUTSL, OUTSW
        RCLL, RCR, RCRB, RCRW, RCRL, RDMSR, RDPMC, RDTSC, REP, REPE, REPNE, RET, ROL, ROLB, ROLW,
        ROLL, ROR, RORB, RORW, RORL, SAHF, SAL, SALB, SALW, SALL, SAR, SARB, SARW, SARL, SBB,
        SBBB, SBBW, SBBL, SCASB, SCASD, SCASW, SETA, SETAE, SETB, SETBE, SETC, SETE, SETG, SETGE,
        SETL, SETLE, SETNA, SETNAE, SETNB, SETNBE, SETNC, SETNE, SETNG, SETNGE, SETNL, SETNLE,
        SETNO, SETNP, SETNS, SETNZ, SETO, SETP, SETPE, SETPO, SETS, SETZ, SGDT, SHL, SHLB, SHLW,
        SHLL, SHLD, SHR, SHRB, SHRW, SHRL, SHRD, SIDT, SLDT, SMSW, STC, STD, STI, STOSB, STOSD,
        STOSW, STR, SUB, SUBB, SUBW, SUBL, TEST, TESTB, TESTW, TESTL, VERR, VERW, WAIT, WBINVD,
        XADD, XADDB, XADDW, XADDL, XCHG, XCHGB, XCHGW, XCHGL, XLAT, XLATB, XOR, XORB, XORW, XORL},
  keywordstyle=\color{blue}\bfseries,
  ndkeywordstyle=\color{darkgray}\bfseries,
  identifierstyle=\color{black},
  sensitive=false,
  comment=[l]{\#},
  morecomment=[s]{/*}{*/},
  commentstyle=\color{purple}\ttfamily,
  stringstyle=\color{red}\ttfamily,
  morestring=[b]',
  morestring=[b]"
}

\lstset{language=assembler, style=codestyle}

% disponi sezioni
\usepackage{titlesec}

\titleformat{\section}
	{\sffamily\Large\bfseries} 
	{\thesection}{1em}{} 
\titleformat{\subsection}
	{\sffamily\large\bfseries}   
	{\thesubsection}{1em}{} 
\titleformat{\subsubsection}
	{\sffamily\normalsize\bfseries} 
	{\thesubsubsection}{1em}{}

% tikz
\usepackage{tikz}

% float
\usepackage{float}

% grafici
\usepackage{pgfplots}
\pgfplotsset{width=10cm,compat=1.9}

% disponi alberi
\usepackage{forest}

\forestset{
	rectstyle/.style={
		for tree={rectangle,draw,font=\large\sffamily}
	},
	roundstyle/.style={
		for tree={circle,draw,font=\large}
	}
}

% disponi algoritmi
\usepackage{algorithm}
\usepackage{algorithmic}
\makeatletter
\renewcommand{\ALG@name}{Algoritmo}
\makeatother

% disponi numeri di pagina
\usepackage{fancyhdr}
\fancyhf{} 
\fancyfoot[L]{\sffamily{\thepage}}

\makeatletter
\fancyhead[L]{\raisebox{1ex}[0pt][0pt]{\sffamily{\@title \ \@date}}} 
\fancyhead[R]{\raisebox{1ex}[0pt][0pt]{\sffamily{\@author}}}
\makeatother

\begin{document}
% sezione (data)
\section{Lezione del 05-11-24}

% stili pagina
\thispagestyle{empty}
\pagestyle{fancy}

% testo
\subsection{D-Latch trasparente}
Introduciamo una nuova rete sequenziale dotata di due ingressi, $d$ (data) e $c$ (control), e un'uscita $q$.
Il D-latch memorizza il bit in $d$ quando $c$ (\textbf{trasparenza}) vale 1.
Quando $c$ vale 0, invece, si dice che è in \textbf{conservazione}, ergo memorizza l'ultimo valore che $d$ ha assunto quando $c$ valeva 1.

La tabella di flusso di questa rete è la seguente, assunti in quest'ordine $c$ e $d$:
\begin{table}[h!]
	\center 
	\begin{tabular} { c | c  c  c  c | c }
		& 00 & 01 & 10 & 11 & q \\ 
		\hline 
		$S_0$ & \circled{$S_0$} & \circled{$S_0$} & \circled{$S_0$} & $S_1$ & 0 \\ 
		$S_1$ & \circled{$S_1$} & \circled{$S_1$} & $S_0$ & \circled{$S_1$} & 1\\
	\end{tabular}
\end{table}

cioè quando si è in conservazione, qualsiasi valore di $d$ viene ignorato e si memorizza il valore passato.
Quando si è in trasparenza, invece, $q$ si asegua a $d$.

Si può realizzare un D-latch attraverso un latch SR, con in ingresso una certa rete combinatoria.
Quello che vogliamo fare è portare $d$ e $c$ in $s$ e $r$, attraverso la tabella di verità:
\begin{table}[H]
	\center 
	\begin{tabular} { c  c | c  c }
		$c$ & $d$ & $s$ & $r$ \\
		\hline
		0 & 0 & 0 & 0 \\ 
		0 & 1 & 0 & 0 \\ 
		1 & 0 & 0 & 1 \\ 
		1 & 1 & 1 & 0
	\end{tabular}
\end{table}

Questo si sintetizza in $s = c \cdot d$ e $r = c \cdot \overline{d}$.
Si ha che le porte AND che rappresentano le congiunzioni in questa rete combinatoria possono collassare con le porte AND che formavano la rete combinatoria del latch SR che permetteva preset e preclear.

\subsubsection{Pilotaggio del D-Latch}
Nel pilotaggio del D-latch, dobbiamo assicurarci che $d$ sia costante a cavallo della transizione di $c$ da 1 a 0, in quanto potremmo finire per memorizzare dati ignoti (l'ultima cosa che il D-latch ha "visto" prima del reset di $c$).
I tempi per cui $d$ deve essere costante, rispettivamente \textbf{prima} e \textbf{dopo} della transizione di $c$, si dicono $T_{setup}$ e $T_{hold}$, e sono dati di progetto.

\subsubsection{Trasparenza}
Quando il D-latch è in \textbf{trasparenza}. il suo ingresso è direttamente connesso, in \textbf{senso logico} (ci sono comunque ritardi nella logica delle reti), all'uscita.
Per questo motivo, se $q$ e $d$ sono collegati in \textbf{retroazione negativa} (un feedback loop negato), si ha che con $c=1$ abbiamo oscillazioni incontrollate, e che con $c=0$ in $q$ (cioè lo stato interno) resta un valore casuale (l'ultimo rilevato durante le oscillazioni casuali prima che $c$ sia transito a 0).

Questo significa che il D-latch è una rete \textbf{trasparente}, cioè \textit{la sua uscita cambia mentre la rete è sensibile alle variazioni di ingresso}.
Questo significa che non possiamo memorizzare niente che sia funzione dell'uscita (saremmo nel caso della retroazine negativa di prima).

Poniamo di voler eseguire un'istruzione semplice come \lstinline|INC %AX|. 
A livello hardware, questo significà connettere un registro (quindi una serie di D-latch) ad una rete combinatoria per l'incremento (probabilmente un half adder), e quindi l'uscita di questa rete di nuovo al D-latch.
Quello che abbiamo essenzialmente creato è un ciclo di retrazione: il sistema devolverà velocemente in uno stato di oscillazione incontrollata.

\subsection{D flip-flop}
Il \textbf{D flip-flop} è una rete sequenziale \textbf{non trasparente} che si pone di risolvere i problemi di trasparenza del D-latch.
Quello che vedremo nel dettaglio è il \textbf{positive edge-triggered D flip-flop}, che è una rete che si comporta come segue, sulla base degli ingressi $d$ (data) e $p$, e l'uscita $q$: quando $p$ ha un fronte di salita, memorizza $d$, \textit{attendi} un determinato istante temporale e adegua l'uscita.

Possiamo concettualizzare il D flip-flop come composto, alla base, da un D-latch.
Mettiamo a $c$, invece dell'ingresso $p$, il \textbf{generatore di impulso} $P^+$ sul fronte di salita di $p$.
In uscita a $q$, poi, abbiamo un buffer $\Delta$, che introduce ritardo.
La proprietà fondamentale che desideriamo è:
$$
\Delta > P^+ 
$$
Questo significa che $q$ si adegua al valore campionato di $d$ soltanto \textit{dopo} che la rete ha smesso di essere sensibile a $d$.
È questa proprietà a rendere il D flip-flop una rete non trasparente.

\subsubsection{Pilotaggio del D flip-flop}
Innanzitutto, a cavallo del fronte di salita di $p$ l'ingresso $d$ deve rimanere costante, ergo si hanno gli stessi $_{setup}$ e $T_{hold}$ del D-latch.
Inoltre, si ha il ritardo di adeguamento dell'uscita, che denominiamo $T_{prop}$ (dall'inglese \textit{propagation}).
Qui la diseguaglianza d prima si traduce come:
$$
T_{prop} > T_{hold}
$$

\par\medskip 

Si che l'uscita di un D-FF non oscilla mai, a differenza di quella del D-Latch: l'adeguamento avviene in modo "secco", sul fronte di salita, e di lì in poi fino a reset e successivo set di $p$, l'uscita $q$ è in conservazione e ignora il comportamento di $d$.

\subsubsection{Sintesi Master-Slave di un D flip-flop}
Storicamente, un D flip-flop è stato realizzato attraverso un montaggio master/slave, attraverso due D-latch in cascata (di cui uno master, e l'altro chiaramente slave).
Si invia quindi l'ingresso $p$ allo slave, e il suo negato al master, e si fa passare la linea $d$ prima dal master, poi dalla sua uscita all'ingresso del slave, e poi al $q$ del D flip-flop.
Si ha che negli stati:
\begin{itemize}
	\item $p=0$: \textbf{master} e in \textit{trasparenza}, \textbf{slave} in \textit{conservazione};
	\item $p=1$: \textbf{master} in \textit{conservazione}, \textbf{slave} in \textit{trasparenza}.
\end{itemize}

Quando $p$ è a 0, lo slave è in conservazione, quindi la rete memorizza.
Nel frattempo il master è in trasparenza, quindi reagisce al valore in entrata di $d$.
Quando $p$ transisce a $1$, lo slavean automa in automaton theory  passa in trasparenza, e quindi risponde a quello che esce dal master, che invece si trova in conservazione del valore che aveva un'attimo prima della transizione.
Il risultato è un comportamento effettivamente analogo a quello della struttura a generatore di impulso e buffer vista prima.

Si possono avere problemi nel funzionamento transitorio dei due D-latch: per questo si agisce elettronicamente, sviluppando questi per commutare $c$ a valori di tensione diversi.
In particolare, vogliamo che in transizione di $p$ da 1 a 0 lo slave conservi il valore prima che il master passi a trasparenza, quindi che $c$ dello slave commuti prima di $c$ del master.

Nella pratica, infine, si ha che la sintesi reale di un D flip-flop è fatta a partire da un latch SR, prima del quale si dispone una rete sequenziale asincrona la cui sintesi è fuori dagli scopi del corso.

\subsection{Memorie RAM statiche}
Esistono due tipi principali di memoria:
\begin{itemize}
	\item \textbf{S-RAM}, costituite da matrici di D latch;
	\item \textbf{D-RAM}, realizzate in modo diverso, che per adesso ingnoreremo. 
\end{itemize}

Una riga di D latch rappresenta quindi una \textbf{locazione di memoria}, che può essere \textbf{letta} o \textbf{scritta} con apposite operazioni, strettamente \textbf{non simultanee}.

Una SRAM è presenta gli ingressi e le uscite:
\begin{itemize}
	\item \textbf{Ingressi di indirizzo:} in numero sufficiente per indirizzare tutte le celle di memoria. Ad esempio, con $2^{23}$ celle di 4 bit, 23 ingressi;
	\item \textbf{Ingressi/uscite di dati:} che andranno forchettati con porte \textbf{tri-state};
	\item \textbf{Memory read} e \textbf{memory write}, segnali attivi bassi;
	\item \textbf{Select}, segnale attivo basso che fa da \textbf{enabler}, in modo simile a quanto avevamo visto nei decoder.
\end{itemize}

Il comportamento che vogliamo dalla SRAM è il seguente:
\begin{table}[h!]
	\center \rowcolors{2}{white}{black!10}
	\begin{tabular} { c | c | c || c }
		\bfseries \lstinline|/s| & \bfseries \lstinline|/mr| & \bfseries \lstinline|/mw| & \bfseries Azione \\
		\hline 
		1 & - & - & Nulla \\ 
		0 & 1 & 1 & Nulla \\ 
		0 & 0 & 1 & Lettura in corso \\ 
		1 & 1 & 0 & Scrittura in corso \\ 
	\end{tabular}
\end{table}

\subsubsection{Temporizzazione delle RAM statiche}
Facciamo innanzitutto la divisione lettura/scrittura:
\begin{itemize}
	\item \textbf{Lettura:} per fare una lettura bisogna dare il comando (attivo basso) di memory read (\lstinline|/mr|), e impostare l'indirizzo di lettura.
Il comando di select (\lstinline|/s|) arriva in ritardo, e a quel punto, quando sia \lstinline|/s| che \lstinline|/mr| sono in conduzione, i multiplexer vanno a regime e si può fare una lettura sull'uscita dei dati.
Infine, quando \lstinline|/mr| torna a 1, i dati tornano ad alta impedenza, e l'indirizzo di lettura e la select possono assumere valori arbitrari.
	\item \textbf{Scrittura:} si ha che la scrittura è \textbf{distruttiva} (manda i D-latch in trasparenza). Bisogna quindi attendere che il select \lstinline|/s| e gli indirizzi siano stabili prima di abbassare \lstinline|mw| per dare il comando di scrittura (l'opposto di quanto avevamo fatto in lettura, qui vogliamo scrivere solo quando siamo sicuri di poterlo fare, ergo i multiplexer sono a regime).
		A questo punto, abbiamo che quando \lstinline|mw| torna alto dobbiamo assicurarci che i dati in scrittura siano fermi, in quanto i multiplexer riportano gli ingressi di controllo dei D-latch a 0 e l'indirizzo di lettura e la select possono, nuovamente, assumere valori arbitrari.
\end{itemize}

\subsubsection{Montaggio di banchi di memoria}
Vediamo come combinare più banchi di memoria per aumentare lo spazio di memoria indirizzabile.
\begin{itemize}
	\item \textbf{Montaggio in parallelo:} 
prendiamo in considerazione due banchi di memoria da $8 \text{M} \times 4$ bit, e vediamo come collegarli per formare un singolo banco di memoria da $8 \text{M} \times 8$ bit, quindi raddoppiando la dimensione delle locazioni.
\begin{itemize}
	\item Per quanto riguarda gli \textbf{indirizzi di lettura}, basta inviare l'indirizzo ad entrambi i banchi, da cui preleveremo la parte \textit{alta} e \textit{bassa} della locazione;
	\item Per quanto riguarda gli \textbf{ingressi/uscite di dati}, avremo che la combinazione delle linee sui due banchi, da 4 bit ciascuna, formano un singolo byte da 8 bit, ergo la locazione di memoria completa.
\end{itemize}
	\item \textbf{Montaggio in serie:}  
prendiamo in considerazione due banchi di memoria da $8 \text{M} \times 8$ bit, e vediamo come collegarli per formare un singolo banco di memoria da $16 \text{M} \times 8$ bit, quindi raddoppiamdo il numero di locazioni.
\begin{itemize}
	\item Per quanto riguarda gli \textbf{indirizzi di lettura}, discrimina dal MSB dell'indirizzo se selezionare dal primo o dal secondo banco, che faranno quindi da parte \textit{alta} e \textit{bassa} dello spazio di memoria indirizzabile. Facciamo questo attraverso l'ingresso di select \lstinline|/s|, che useremo per determinare altri due segnali di select \lstinline|/sl| e \lstinline|sh| (\textit{select low} e \textit{select high}), che a loro volta ci permettono di discriminare sulla base del MSD quale banco andiamo a selezionare (effettivamente rendere attivo); 
	\item Per quanto riguarda gli \textbf{ingressi/uscite di dati}, avremo che il banco attivo in un dato momento determina completamente le uscite. Potremmo pensare di dover inserire porte tri-state in uscita ai singoli banchi di memoria sulla linea di ingresso/uscita, ma questo non è necessario: le \lstinline|sl| e \lstinline|sh| sono mutualmente esclusive, e quindi non si verificherà mai il caso in cui le linee di uscita di entrambi i banchi sono in conduzione contemporaneamente.
\end{itemize}
\end{itemize}
\end{document}
