
\documentclass[a4paper,11pt]{article}
\usepackage[a4paper, margin=8em]{geometry}

% usa i pacchetti per la scrittura in italiano
\usepackage[french,italian]{babel}
\usepackage[T1]{fontenc}
\usepackage[utf8]{inputenc}
\frenchspacing 

% usa i pacchetti per la formattazione matematica
\usepackage{amsmath, amssymb, amsthm, amsfonts}

% usa altri pacchetti
\usepackage{gensymb}
\usepackage{hyperref}
\usepackage{standalone}

\usepackage{colortbl}

\usepackage{xstring}
\usepackage{karnaugh-map}

% imposta il titolo
\title{Appunti Reti Logiche}
\author{Luca Seggiani}
\date{2024}

% imposta lo stile
% usa helvetica
\usepackage[scaled]{helvet}
% usa palatino
\usepackage{palatino}
% usa un font monospazio guardabile
\usepackage{lmodern}

\renewcommand{\rmdefault}{ppl}
\renewcommand{\sfdefault}{phv}
\renewcommand{\ttdefault}{lmtt}

% circuiti
\usepackage{circuitikz}
\usetikzlibrary{babel}

% testo cerchiato
\newcommand*\circled[1]{\tikz[baseline=(char.base)]{
            \node[shape=circle,draw,inner sep=2pt] (char) {#1};}}

% disponi il titolo
\makeatletter
\renewcommand{\maketitle} {
	\begin{center} 
		\begin{minipage}[t]{.8\textwidth}
			\textsf{\huge\bfseries \@title} 
		\end{minipage}%
		\begin{minipage}[t]{.2\textwidth}
			\raggedleft \vspace{-1.65em}
			\textsf{\small \@author} \vfill
			\textsf{\small \@date}
		\end{minipage}
		\par
	\end{center}

	\thispagestyle{empty}
	\pagestyle{fancy}
}
\makeatother

% disponi teoremi
\usepackage{tcolorbox}
\newtcolorbox[auto counter, number within=section]{theorem}[2][]{%
	colback=blue!10, 
	colframe=blue!40!black, 
	sharp corners=northwest,
	fonttitle=\sffamily\bfseries, 
	title=Teorema~\thetcbcounter: #2, 
	#1
}

% disponi definizioni
\newtcolorbox[auto counter, number within=section]{definition}[2][]{%
	colback=red!10,
	colframe=red!40!black,
	sharp corners=northwest,
	fonttitle=\sffamily\bfseries,
	title=Definizione~\thetcbcounter: #2,
	#1
}

% disponi codice
\usepackage{listings}
\usepackage[table]{xcolor}

\definecolor{codegreen}{rgb}{0,0.6,0}
\definecolor{codegray}{rgb}{0.5,0.5,0.5}
\definecolor{codepurple}{rgb}{0.58,0,0.82}
\definecolor{backcolour}{rgb}{0.95,0.95,0.92}

\lstdefinestyle{codestyle}{
		backgroundcolor=\color{black!5}, 
		commentstyle=\color{codegreen},
		keywordstyle=\bfseries\color{magenta},
		numberstyle=\sffamily\tiny\color{black!60},
		stringstyle=\color{green!50!black},
		basicstyle=\ttfamily\footnotesize,
		breakatwhitespace=false,         
		breaklines=true,                 
		captionpos=b,                    
		keepspaces=true,                 
		numbers=left,                    
		numbersep=5pt,                  
		showspaces=false,                
		showstringspaces=false,
		showtabs=false,                  
		tabsize=2
}

\lstdefinestyle{shellstyle}{
		backgroundcolor=\color{black!5}, 
		basicstyle=\ttfamily\footnotesize\color{black}, 
		commentstyle=\color{black}, 
		keywordstyle=\color{black},
		numberstyle=\color{black!5},
		stringstyle=\color{black}, 
		showspaces=false,
		showstringspaces=false, 
		showtabs=false, 
		tabsize=2, 
		numbers=none, 
		breaklines=true
}


\lstdefinelanguage{assembler}{ 
  keywords={AAA, AAD, AAM, AAS, ADC, ADCB, ADCW, ADCL, ADD, ADDB, ADDW, ADDL, AND, ANDB, ANDW, ANDL,
        ARPL, BOUND, BSF, BSFL, BSFW, BSR, BSRL, BSRW, BSWAP, BT, BTC, BTCB, BTCW, BTCL, BTR, 
        BTRB, BTRW, BTRL, BTS, BTSB, BTSW, BTSL, CALL, CBW, CDQ, CLC, CLD, CLI, CLTS, CMC, CMP,
        CMPB, CMPW, CMPL, CMPS, CMPSB, CMPSD, CMPSW, CMPXCHG, CMPXCHGB, CMPXCHGW, CMPXCHGL,
        CMPXCHG8B, CPUID, CWDE, DAA, DAS, DEC, DECB, DECW, DECL, DIV, DIVB, DIVW, DIVL, ENTER,
        HLT, IDIV, IDIVB, IDIVW, IDIVL, IMUL, IMULB, IMULW, IMULL, IN, INB, INW, INL, INC, INCB,
        INCW, INCL, INS, INSB, INSD, INSW, INT, INT3, INTO, INVD, INVLPG, IRET, IRETD, JA, JAE,
        JB, JBE, JC, JCXZ, JE, JECXZ, JG, JGE, JL, JLE, JMP, JNA, JNAE, JNB, JNBE, JNC, JNE, JNG,
        JNGE, JNL, JNLE, JNO, JNP, JNS, JNZ, JO, JP, JPE, JPO, JS, JZ, LAHF, LAR, LCALL, LDS,
        LEA, LEAVE, LES, LFS, LGDT, LGS, LIDT, LMSW, LOCK, LODSB, LODSD, LODSW, LOOP, LOOPE,
        LOOPNE, LSL, LSS, LTR, MOV, MOVB, MOVW, MOVL, MOVSB, MOVSD, MOVSW, MOVSX, MOVSXB,
        MOVSXW, MOVSXL, MOVZX, MOVZXB, MOVZXW, MOVZXL, MUL, MULB, MULW, MULL, NEG, NEGB, NEGW,
        NEGL, NOP, NOT, NOTB, NOTW, NOTL, OR, ORB, ORW, ORL, OUT, OUTB, OUTW, OUTL, OUTSB, OUTSD,
        OUTSW, POP, POPL, POPW, POPB, POPA, POPAD, POPF, POPFD, PUSH, PUSHL, PUSHW, PUSHB, PUSHA, 
				PUSHAD, PUSHF, PUSHFD, RCL, RCLB, RCLW, MOVSL, MOVSB, MOVSW, STOSL, STOSB, STOSW, LODSB, LODSW,
				LODSL, INSB, INSW, INSL, OUTSB, OUTSL, OUTSW
        RCLL, RCR, RCRB, RCRW, RCRL, RDMSR, RDPMC, RDTSC, REP, REPE, REPNE, RET, ROL, ROLB, ROLW,
        ROLL, ROR, RORB, RORW, RORL, SAHF, SAL, SALB, SALW, SALL, SAR, SARB, SARW, SARL, SBB,
        SBBB, SBBW, SBBL, SCASB, SCASD, SCASW, SETA, SETAE, SETB, SETBE, SETC, SETE, SETG, SETGE,
        SETL, SETLE, SETNA, SETNAE, SETNB, SETNBE, SETNC, SETNE, SETNG, SETNGE, SETNL, SETNLE,
        SETNO, SETNP, SETNS, SETNZ, SETO, SETP, SETPE, SETPO, SETS, SETZ, SGDT, SHL, SHLB, SHLW,
        SHLL, SHLD, SHR, SHRB, SHRW, SHRL, SHRD, SIDT, SLDT, SMSW, STC, STD, STI, STOSB, STOSD,
        STOSW, STR, SUB, SUBB, SUBW, SUBL, TEST, TESTB, TESTW, TESTL, VERR, VERW, WAIT, WBINVD,
        XADD, XADDB, XADDW, XADDL, XCHG, XCHGB, XCHGW, XCHGL, XLAT, XLATB, XOR, XORB, XORW, XORL},
  keywordstyle=\color{blue}\bfseries,
  ndkeywordstyle=\color{darkgray}\bfseries,
  identifierstyle=\color{black},
  sensitive=false,
  comment=[l]{\#},
  morecomment=[s]{/*}{*/},
  commentstyle=\color{purple}\ttfamily,
  stringstyle=\color{red}\ttfamily,
  morestring=[b]',
  morestring=[b]"
}

\lstset{language=assembler, style=codestyle}

% disponi sezioni
\usepackage{titlesec}

\titleformat{\section}
	{\sffamily\Large\bfseries} 
	{\thesection}{1em}{} 
\titleformat{\subsection}
	{\sffamily\large\bfseries}   
	{\thesubsection}{1em}{} 
\titleformat{\subsubsection}
	{\sffamily\normalsize\bfseries} 
	{\thesubsubsection}{1em}{}

% tikz
\usepackage{tikz}

% float
\usepackage{float}

% grafici
\usepackage{pgfplots}
\pgfplotsset{width=10cm,compat=1.9}

% disponi alberi
\usepackage{forest}

\forestset{
	rectstyle/.style={
		for tree={rectangle,draw,font=\large\sffamily}
	},
	roundstyle/.style={
		for tree={circle,draw,font=\large}
	}
}

% disponi algoritmi
\usepackage{algorithm}
\usepackage{algorithmic}
\makeatletter
\renewcommand{\ALG@name}{Algoritmo}
\makeatother

% disponi numeri di pagina
\usepackage{fancyhdr}
\fancyhf{} 
\fancyfoot[L]{\sffamily{\thepage}}

\makeatletter
\fancyhead[L]{\raisebox{1ex}[0pt][0pt]{\sffamily{\@title \ \@date}}} 
\fancyhead[R]{\raisebox{1ex}[0pt][0pt]{\sffamily{\@author}}}
\makeatother

\begin{document}
% sezione (data)
\section{Lezione del 30-10-24}

% stili pagina
\thispagestyle{empty}
\pagestyle{fancy}

% testo
\subsection{La funzione di memoria}
Finora abbiamo visto reti combinatorie, cioè \textbf{reti prive di memoria}, dove lo stato di uscita ad un istante dipende solo dallo stato di ingresso corrente.
Nelle \textbf{reti sequenziali}, invece, l'uscita dipende dalla sequenza degli stati di ingresso visti dalla rete fino a quel momento. 
Questa memoria si implementa attraverso \textbf{anelli di retroazione}.

Prendiamo ad esempio un buffer con un anello di retrazione, cioè una linea che porta la sua uscita al suo ingresso, e che estrae in uscita $q$.

Questo potrà quindi esistere in due situazioni di stabilità:
\begin{itemize}
	\item L'uscita vale 0, e va in ingresso al buffer, dove si \textbf{rigenera} (o si \textit{autosostiene});
	\item L'uscita vale 1, e va in ingresso al buffer, dove ancora una volta si rigenera e mantiene il suo valore.
\end{itemize}

La presenza del buffer è fondamentale: mantiene l'uscita $q$ a 0 e 1, e sopratutto ci assicura di poter associare a quel punto della rete uno stato logico.

Il problema di una rete di questo genere è che è fondamentalmente inutile: non si può controllare lo stato di stabilità del buffer, a quindi non si possono immagazzinare bit diversi a tempi diversi.

\subsubsection{Uscita negata}
Realizziamo allora il nostro buffer, sostituendolo cone due porte NOR disposte come invertitori (quindi un doppio invertitore, che equivale al buffer).
Si ha che fra le due porte NOR abbiamo il valore complementato del buffer, cioè 1 a 0 e 0 a 1.
Possiamo quindi dotare la rete di un'ulteriore uscita $q_N$, che equivale appunto alla negazione di $q$.
Per questo motivo, avevamo detto, nella valutazione dei livelli di logica si ignorano le porte NOT: solitamente abbiamo già un valore negato a disposizione dai registri.

\subsubsection{Stato all'accensione}
Ora, se all'accensione $q$ e $q_N$ sono discordi, la rete si troverà già in uno degli stati stabili, e lì restera.
In caso contrario, se sono concordi, teoricamente ciascuna delle due uscite dovrebbe oscillare all'infinito, con un periodo pari al doppio del tempo di risposta delle porte (in due passaggi si completa un ciclo, cioè l'ingresso della prima porta torna al neutro).
Nella pratica, la rete si stabilizza, in quanto il tempo delle porte sarà necessariamente diverso e quindi si creerà prima o poi una condizione analoga a prima, dove le uscite sono discordi e la rete resta stabile. 

\subsubsection{Latch SR}
Vediamo quini come rendere pilotabile lo stato del circuito.
Introduciamo due ingressi negli input (finora duplicati) delle porte NOR: in entrata alla prima porta avremo il comando S, per SET, e in entranta alla seconda porta avremo il comando R, per RESET.
Questi ingressi sono \textit{attivi alti}, cioè i comandi S e R vengono dati quando le rispettive entrate sono in tensione.
Chiamiamo questa rete \textbf{latch SR}, a volte impropriamente detta \textit{flip-flop SR}.

Vediamo il funzionamento della rete nei diversi casi di attivazione degli ingressi:
\begin{itemize}
	\item $S=1$, $R=0$: si ha che la prima NOR ha un ingresso 1, ergo mette l'uscita a 0. Quindi, la seconda NOR ha un ingresso 0, ergo mette l'uscita a 1. Ci troviamo nella configurazione stabile $q=1$, $q_N = 0$, cioè abbiamo memorizzato un bit. 
	\item $S=0$, $R=1$: si ha che la seconda NOR ha un ingresso 1, ergo mette l'uscita a 0. Quindi, la prima NOR ha un ingresso 0, quindi mette l'uscita a 1. Ci troviamo nella configurazione stabile $q=0$, $q_N = 1$, cioè abbiamo resettato un bit.
	\item $S=0$, $R=0$: l'uscita della prima NOR vale 0 se $q=1$, e 1 se $q=0$, quindi $q_N$ dà semplicemente $\bar{q}$ e viceversa, e la rete conserva il valore che aveva precedentemente. Questo comportamento rende la rete \textbf{sequenziale}: nello stato di \textbf{conservazione}, cioè quello a ingressi disattivati, si ha che la rete rimane nello stato stabile $S_0$ o $S_1$ nel quale si era portata in un momento precedente nella sequenza di stati.
		Si può anche dire che la rete \textbf{ricorda} l'ultimo SET o RESET ricevuto.
		Comunque, è una rete \textbf{asincrona}, in quanto l'uscita si aggiorna subito rispetto agli ingressi (e non in sincronia ad un clock).
	\item $S=1$, $R=1$: semanticamente, questa istruzione non ha molto significato. In uno stato di pilotaggio corretto, diciamo che questo stato \textbf{non è permesso}.
		Se si venisse a verificare, avremmo che alla prima porta un entrata è 1, e quindi l'uscita è 0. Alla seconda porta, quindi, un'uscita sarà 1, e avremo di nuovo uscita 0.
		Forzeremmo quindi la rete in uno stato $q=0$, $q_N=0$, che non significa nulla dal punto di vista della rappresentazione in bit della memoria.
\end{itemize}

In Verilog, possiamo descrivere il latch SR come segue:

\lstinputlisting[language=verilog, style=codestyle]{../verilog/10-30/sr_latches/sr_latch.v}

\subsubsection{Tabella di applicazione}
Per descrivere il comportamento delle reti con memoria usiamo le \textbf{tabelle di applicazione}.
Queste rappresentano, a sinistra, il valore attuale della variabile e il valore successivo che si vuole questa assuma, e a destra il comando necessario perchè l'uscita passi dal valore attuale a quello successivo.
Nel caso del latch SR, si ha che questa è:
\begin{table}[h!]
	\center 
	\begin{tabular} { c c | c c }
		\bfseries $q$ & \bfseries $q'$ & \bfseries $s$ & \bfseries $r$ \\
		\hline 
		0 & 0 & 0 & - \\ 
		0 & 1 & 1 & 0 \\ 
		1 & 0 & 0 & 1 \\ 
		1 & 1 & - & 0 
	\end{tabular}
\end{table}

\subsubsection{Regole di pilotaggio}
Avevamo visto le regole per le reti combinatorie:
\begin{itemize}
	\item Siamo in \textbf{pilotaggio in modo fondamentale}: si cambiano gli ingressi solo quando la rete è a regime;
	\item Gli stati di ingresso consecutivi devono essere adiacenti (per evitare \textit{race condition}).
\end{itemize}

Vogliamo definire una serie di regole simili per le reti sequenziali.
Abbiamo che la regola di \textbf{pilotaggio in modo fondamentale} va rispettata comunque: la rete avrà un certo \textbf{tempo di attraversamento} di cui tenere conto.
Anche la seconda regola, degli \textbf{stati di ingressi consecutivi adiacenti}, è fondamentale: se non viene rispettata, si possono presentare in ingresso stati transitori spuri, e l'evoluzione delle uscite diventa imprevedebile.

Nel latch SR, però, vale che questa legge può essere violata: cioè il latch SR è robusto nei confronti di pilotaggi scorretti.
Questo è il punto di forza che lo rende la rete alla base dei registri e di tutti gli elementi di memoria.

\subsubsection{Lo stato iniziale}
Abbiamo detto che l'SR è l'elemento alla base dei circuiti di memoria.
Un SR può contenere informazioni, che corrispondono allo \textit{stato} $S_0$ o $S_1$ in cui si trova.
Si ha, però, che all'accensione il bit contenuto nell'SR è \textbf{casuale} (da quello che avevamo visto dalle modalità di pilotaggio).
All'accensione di un calcolatore, si ha che alcuni elementi possono avere un valore casuale (ad esempio la RAM), altri no (ad esmpio l'instruction pointer).
Si definisce quindi una \textbf{fase di reset}, distinta dalla \textbf{fase operativa}, cioè quella di operatività standard.
Nella fase di reset si inizializzano gli elementi di memoria: notiamo che questo reset non corrisponde al comando R, di RESET, che diamo ai latch.
In generale, quindi, non è vero che gli elementi di memoria contengono tutti zero all'accensione del calcolatore.

Vediamo quindi il circuito:
\begin{center}
	\begin{circuitikz}
		\draw (0, 0) node[spdt, xscale=-1] (on) {};	
		\draw (0, -2) node[spdt, xscale=-1] (reset) {};
		\draw (on.out 2) node[ground] {};
		\draw (reset.out 2) node[ground] {};
		\node[anchor=east] at (on.out 1) {Vcc};

		\draw (on.in) -- (2, 0)
			to [ resistor, l = $R$ ] (2, -2)
			to [ capacitor, l = $C$] (2, -3);
		\draw (2, -3) node[ground] {};
		\draw (4, -2) node [ schmitt ] (schm) {};
		\draw (reset.in) -- (schm.in);
		\node[anchor=west] at (4.75, -2) {/reset};

		\node at (0, 0.75) {ON/OFF};
		\node at (0, -1.25) {reset};
	\end{circuitikz}
\end{center}
Abbiamo che la circuiteria di trigger è realizzata attraverso un circuito RC, fra l'interruttore ON/OFF e il pulsante di reset, con $\tau = R \cdot C \approx 0 \, \mathrm{\mu s}$, dove si collega il nodo fra R e C ad un trigger di Schmitt.
Il trigger di Schmitt effettivamente "quantizza" la tensione, cioè scatta ad un valore 1 di tensione solamente quando la tensione in entranta è maggiore di una certa soglia.
Abbiamo quindi che, spostando l'interruttore nella posizione ON, il circuito raggiunge il regime in un tempo $\approx \tau$, e quindi il trigger va a 1 in un tempo $\approx \tau$. Lo stesso quando si preme il pulsante di reset, il capacitore C si scarica e dobbiamo riportare nuovamente il circuito a regime, per cui abbiamo un istante $\approx \tau$ dove il trigger è a 0. 

Abbiamo che l'uscita di questa rete va ad un ingresso detto /reset (che ricordiamo è distinto dai singoli reset dei latch SR), che è \textit{attivo basso}: cioè nella fase iniziale dell'accensione, e ad ogni pressione successiva del pulsante reset, si ha che dal trigger esce per un tempo $\approx \tau$ il comando di /reset.

Per implementare effettivamente il meccanismo di reset si dota il latch SR di due ingressi aggiuntivi: /preset e /preclear, entrambi attivi bassi.
Si distinguono quindi i seguenti casi:
\begin{itemize}
	\item $\mathrm{/preset} = \mathrm{/preclear} = 1$: la rete si comporta come un latch SR normale;
	\item $\mathrm{/preset} = 0$: la rete si trova nello stato $S_1$ (indipendentemente dallo stato di $s$ e $r$);
	\item $\mathrm{/preclear} = 0$: la rete si trova nello stato $S_0$ (indipendentemente dallo stato di $s$ e $r$);
	\item $\mathrm{/preset} = \mathrm{/preclear} = 0$: abbiamo, come nel caso già visto dei semplici ingressi Set e Reset, che questo stato non è permesso, e quindi non è interessante conoscere il funzionamento della rete in tale stato.
\end{itemize}

Abbiamo quindi che per inizializzare un latch SR a 1 si porta /preset a /reset, e /preclear al Vcc.
Viceversa, per inizializzare il latch a 0 si porta /preset al Vcc e /preclear a /reset.

Vogliamo quindi modificare la sintesi del latch SR: conviene unirlo ad una rete combinatoria, che ha per ingresso S, R, /preset e /preclear, e in uscita $z_s$ e $z_r$ (che andranno in ingresso al latch vero e proprio).
L'obiettivo di questa rete è di impostare i corrispondenti comandi di SET e RESET se uno fra /preset e /preclear è attivo basso, o di restituire S e R così come sono in caso entrambi siano alti. 

Abbiamo, dalla sintesi con le mappe di Karnaugh, riportando i valori in coppie $(z_s, z_r)$:

\begin{center}
	\begin{karnaugh-map}[4][4][1][/preclear][/preset][R][S]
		\manualterms{--, 10, 01, 00, --, 10, 01, 01, --, 10, 01, 10, --, 10, 01, 10}
	\end{karnaugh-map}
\end{center}

Si visualizzano i sottocubi nelle mappe presi separatamente $z_s$ e $z_r$:
\begin{itemize}
	\item $z_s$:
\begin{center}
\noindent
\begin{minipage}{0.3\textwidth}
	\begin{karnaugh-map}[4][4][1][/preclear][/preset][R][S]
		\manualterms{-, 1, 0, 0, -, 1, 0, 0, -, 1, 0, 1, -, 1, 0, 1}
		\implicant{0}{9}
		\implicant{13}{11}
	\end{karnaugh-map}
\end{minipage}%
\hspace{3cm}
\begin{minipage}{0.3\textwidth}
	\begin{table}[H]
		\center \rowcolors{2}{white}{black!10}
		\begin{tabular} { c || c | c | c | c}
			& S & R & /preset & /preclear \\ 
			\hline 
			\rowcolor{red!20!white} A & - & - & 0 & - \\
			\rowcolor{green!20!white} B & 1 & - & - & 1 \\
		\end{tabular}
	\end{table}
\end{minipage}
\end{center}
\item $s_r$:
\begin{center}
\noindent
\begin{minipage}{0.3\textwidth}
	\begin{karnaugh-map}[4][4][1][/preclear][/preset][R][S]
		\manualterms{-, 0, 1, 0, -, 0, 1, 1, -, 0, 1, 0, -, 0, 1, 1}
		\implicantedge{0}{8}{2}{10}
		\implicant{7}{14}
	\end{karnaugh-map}
\end{minipage}%
\hspace{3cm}
\begin{minipage}{0.3\textwidth}
	\begin{table}[H]
		\center \rowcolors{2}{white}{black!10}
		\begin{tabular} { c || c | c | c | c}
			& S & R & /preset & /preclear \\ 
			\hline 
			\rowcolor{red!20!white} A & - & - & - & 0 \\
			\rowcolor{green!20!white} B & - & 1 & 1 & - \\
		\end{tabular}
	\end{table}
\end{minipage}
\end{center}
\end{itemize}


Da cui si ricavano le due sintesi di $z_s$ e $z_r$:
\[
	\begin{cases}
		z_s = \overline{\mathrm{/preset}} + (\mathrm{/preclear} \cdot s) \\ 
		z_r = \overline{\mathrm/preclear} + (\mathrm{/preset} \cdot r)
	\end{cases}
\]

A questo punto, visto che il latch SR è realizzato a porte NOR, possiamo semplificare gli OR e i NOR in cascata: se assumiamo una NOR come una OR in serie ad una NOT, si ha che due OR equivalgono a una singola OR, ergo si possono mandare le uscite delle reti combinatorie appena sintetizzate direttamente ai NOR del latch SR, rimuovendo le OR che avremmo normalmento introdotto in una sintesi SP. 
Questo processo viene a volte detto \textit{compenetrazione}.

In Verilog, possiamo quindi descrivere il latch SR aggiornato con le entrate di $\mathrm{/preset}$ e $\mathrm{/preclear come segue:

\lstinputlisting[language=verilog, style=codestyle]{../verilog/10-30/sr_latches/sr_reset_latch.v}

\subsection{Tabelle e grafi di flusso}
Le reti sequenziali, pi+ spesso che con la tabella di applicazione, si descrivono usando \textbf{tabelle di flusso} e \textbf{grafi di flusso}.

\subsubsection{Tabelle di flusso}
Una tabella di flusso è una tabella che descrive come si evolvono lo stato interno e l'uscita al variare degli stati di ingresso.
Ad esempio, per un latch SR, ignorando /preclear e /preset:

\begin{table}[h!]
	\center 
	\begin{tabular} { c | c  c  c  c | c }
		& 00 & 01 & 11 & 10 & q \\ 
		\hline 
		$S_0$ & \circled{$S_0$} & \circled{$S_0$} & - & $s_1$ & 0 \\ 
		$S_1$ & \circled{$S_1$} & $s_0$ & - & \circled{$S_1$} & 1\\
	\end{tabular}
\end{table}

Si ha che nella tabella, le righe rappresentano gli \textbf{stati interni presenti} (SIP) e le colonne i possibili ingressi in entrata: all'intersezione fra uno stato e un ingresso si ha lo \textbf{stato interno successivo} (SIS).
Si indicano con la barra ($-$) gli stati non definiti (in questo caso \textit{non permessi}).
Inoltre, l'ultima colonna indica il valore effettivo delle uscite in ogni stato (qui si è riportato solo $q$, e non $q_N$).
Gli stati interni successivi cerchiati sono quelli che restano invariati dagli stati interni presenti precedenti: cioè, le coppie di stati interni presenti e ingressi che individuano uno stato interno successivo cerchiato sono coppie \textbf{stabili}.

\subsubsection{Grafi di flusso}
Un formalismo del tutto identico è quello del grafo di flusso: si prendono gli stati come nodi, e si disegnano archi (orientati) etichettati con gli stati di ingresso.
Gli archi uscenti da un nodo simboleggiano quindi i possibili ingressi di quello stato, e entrano nei nodi che rappresentano gli stati interni successivi.
Ad esempio, il grafo corrispondente alla tabella di flusso precedente è:
\begin{center}
	\begin{circuitikz}
		\node[draw, circle] (1) at (0, 0) {$S_0 / 0$};
		\node[draw, circle] (2) at (4, 0) {$S_1 / 1$};
		
		\draw[->] (1) to [out=45, in=135, looseness=1] (2);
		\node at (2,1.5) {sr = 10};

		\draw[->] (2) to [out=225, in=-45, looseness=1] (1);
		\node at (2,-1.5) {sr = 01};
		
		\draw[->] (1) to [out=70, in=110, looseness=5] (1);
		\node at (0, 1.5) {sr = 00};
		\draw[->] (2) to [out=70, in=110, looseness=5] (2);
		\node at (4, 1.5) {sr = 00};
	
		\draw[->] (1) -- (0, -1.2); 
		\node at (0, -1.5) {sr = 11};
		\draw[->] (2) -- (4, -1.2); 
		\node at (4, -1.5) {sr = 11};
	\end{circuitikz}
\end{center}

Notiamo come ad ogni nodo si può associare, separato dalla barra ($/$), l'uscita corrispondente a un dato stato interno presente.
Inoltre, gli stati non definiti vengono indicati con frecce non dirette verso alcun nodo.

\par\smallskip 

Questi strumenti sono utili per la descrizione e la verifica (in questo caso, sotto \textbf{ispezione}, \textbf{statica}) delle reti logiche.
Nel caso di reti sequenziali, poi, la verifica \textbf{dinamica} si fa attraverso un \textbf{diagramma di temporizzazione}, cioè un grafico temporale del valore logico di ogni variabile di interesse, creato seguendo i passaggi:
\begin{enumerate}
	\item Si decide uno stato iniziale;
	\item Si attribuiscono valori agli ingressi nel tempo;
	\item Si osserva l'evoluzione temporale della rete.
\end{enumerate}

\end{document}
