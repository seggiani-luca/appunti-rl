
\documentclass[a4paper,11pt]{article}
\usepackage[a4paper, margin=8em]{geometry}

% usa i pacchetti per la scrittura in italiano
\usepackage[french,italian]{babel}
\usepackage[T1]{fontenc}
\usepackage[utf8]{inputenc}
\frenchspacing 

% usa i pacchetti per la formattazione matematica
\usepackage{amsmath, amssymb, amsthm, amsfonts}

% usa altri pacchetti
\usepackage{gensymb}
\usepackage{hyperref}
\usepackage{standalone}

\usepackage{colortbl}

\usepackage{xstring}
\usepackage{karnaugh-map}

% imposta il titolo
\title{Appunti Reti Logiche}
\author{Luca Seggiani}
\date{2024}

% imposta lo stile
% usa helvetica
\usepackage[scaled]{helvet}
% usa palatino
\usepackage{palatino}
% usa un font monospazio guardabile
\usepackage{lmodern}

\renewcommand{\rmdefault}{ppl}
\renewcommand{\sfdefault}{phv}
\renewcommand{\ttdefault}{lmtt}

% circuiti
\usepackage{circuitikz}
\usetikzlibrary{babel}

% testo cerchiato
\newcommand*\circled[1]{\tikz[baseline=(char.base)]{
            \node[shape=circle,draw,inner sep=2pt] (char) {#1};}}

% disponi il titolo
\makeatletter
\renewcommand{\maketitle} {
	\begin{center} 
		\begin{minipage}[t]{.8\textwidth}
			\textsf{\huge\bfseries \@title} 
		\end{minipage}%
		\begin{minipage}[t]{.2\textwidth}
			\raggedleft \vspace{-1.65em}
			\textsf{\small \@author} \vfill
			\textsf{\small \@date}
		\end{minipage}
		\par
	\end{center}

	\thispagestyle{empty}
	\pagestyle{fancy}
}
\makeatother

% disponi teoremi
\usepackage{tcolorbox}
\newtcolorbox[auto counter, number within=section]{theorem}[2][]{%
	colback=blue!10, 
	colframe=blue!40!black, 
	sharp corners=northwest,
	fonttitle=\sffamily\bfseries, 
	title=Teorema~\thetcbcounter: #2, 
	#1
}

% disponi definizioni
\newtcolorbox[auto counter, number within=section]{definition}[2][]{%
	colback=red!10,
	colframe=red!40!black,
	sharp corners=northwest,
	fonttitle=\sffamily\bfseries,
	title=Definizione~\thetcbcounter: #2,
	#1
}

% disponi codice
\usepackage{listings}
\usepackage[table]{xcolor}

\definecolor{codegreen}{rgb}{0,0.6,0}
\definecolor{codegray}{rgb}{0.5,0.5,0.5}
\definecolor{codepurple}{rgb}{0.58,0,0.82}
\definecolor{backcolour}{rgb}{0.95,0.95,0.92}

\lstdefinestyle{codestyle}{
		backgroundcolor=\color{black!5}, 
		commentstyle=\color{codegreen},
		keywordstyle=\bfseries\color{magenta},
		numberstyle=\sffamily\tiny\color{black!60},
		stringstyle=\color{green!50!black},
		basicstyle=\ttfamily\footnotesize,
		breakatwhitespace=false,         
		breaklines=true,                 
		captionpos=b,                    
		keepspaces=true,                 
		numbers=left,                    
		numbersep=5pt,                  
		showspaces=false,                
		showstringspaces=false,
		showtabs=false,                  
		tabsize=2
}

\lstdefinestyle{shellstyle}{
		backgroundcolor=\color{black!5}, 
		basicstyle=\ttfamily\footnotesize\color{black}, 
		commentstyle=\color{black}, 
		keywordstyle=\color{black},
		numberstyle=\color{black!5},
		stringstyle=\color{black}, 
		showspaces=false,
		showstringspaces=false, 
		showtabs=false, 
		tabsize=2, 
		numbers=none, 
		breaklines=true
}


\lstdefinelanguage{assembler}{ 
  keywords={AAA, AAD, AAM, AAS, ADC, ADCB, ADCW, ADCL, ADD, ADDB, ADDW, ADDL, AND, ANDB, ANDW, ANDL,
        ARPL, BOUND, BSF, BSFL, BSFW, BSR, BSRL, BSRW, BSWAP, BT, BTC, BTCB, BTCW, BTCL, BTR, 
        BTRB, BTRW, BTRL, BTS, BTSB, BTSW, BTSL, CALL, CBW, CDQ, CLC, CLD, CLI, CLTS, CMC, CMP,
        CMPB, CMPW, CMPL, CMPS, CMPSB, CMPSD, CMPSW, CMPXCHG, CMPXCHGB, CMPXCHGW, CMPXCHGL,
        CMPXCHG8B, CPUID, CWDE, DAA, DAS, DEC, DECB, DECW, DECL, DIV, DIVB, DIVW, DIVL, ENTER,
        HLT, IDIV, IDIVB, IDIVW, IDIVL, IMUL, IMULB, IMULW, IMULL, IN, INB, INW, INL, INC, INCB,
        INCW, INCL, INS, INSB, INSD, INSW, INT, INT3, INTO, INVD, INVLPG, IRET, IRETD, JA, JAE,
        JB, JBE, JC, JCXZ, JE, JECXZ, JG, JGE, JL, JLE, JMP, JNA, JNAE, JNB, JNBE, JNC, JNE, JNG,
        JNGE, JNL, JNLE, JNO, JNP, JNS, JNZ, JO, JP, JPE, JPO, JS, JZ, LAHF, LAR, LCALL, LDS,
        LEA, LEAVE, LES, LFS, LGDT, LGS, LIDT, LMSW, LOCK, LODSB, LODSD, LODSW, LOOP, LOOPE,
        LOOPNE, LSL, LSS, LTR, MOV, MOVB, MOVW, MOVL, MOVSB, MOVSD, MOVSW, MOVSX, MOVSXB,
        MOVSXW, MOVSXL, MOVZX, MOVZXB, MOVZXW, MOVZXL, MUL, MULB, MULW, MULL, NEG, NEGB, NEGW,
        NEGL, NOP, NOT, NOTB, NOTW, NOTL, OR, ORB, ORW, ORL, OUT, OUTB, OUTW, OUTL, OUTSB, OUTSD,
        OUTSW, POP, POPL, POPW, POPB, POPA, POPAD, POPF, POPFD, PUSH, PUSHL, PUSHW, PUSHB, PUSHA, 
				PUSHAD, PUSHF, PUSHFD, RCL, RCLB, RCLW, MOVSL, MOVSB, MOVSW, STOSL, STOSB, STOSW, LODSB, LODSW,
				LODSL, INSB, INSW, INSL, OUTSB, OUTSL, OUTSW
        RCLL, RCR, RCRB, RCRW, RCRL, RDMSR, RDPMC, RDTSC, REP, REPE, REPNE, RET, ROL, ROLB, ROLW,
        ROLL, ROR, RORB, RORW, RORL, SAHF, SAL, SALB, SALW, SALL, SAR, SARB, SARW, SARL, SBB,
        SBBB, SBBW, SBBL, SCASB, SCASD, SCASW, SETA, SETAE, SETB, SETBE, SETC, SETE, SETG, SETGE,
        SETL, SETLE, SETNA, SETNAE, SETNB, SETNBE, SETNC, SETNE, SETNG, SETNGE, SETNL, SETNLE,
        SETNO, SETNP, SETNS, SETNZ, SETO, SETP, SETPE, SETPO, SETS, SETZ, SGDT, SHL, SHLB, SHLW,
        SHLL, SHLD, SHR, SHRB, SHRW, SHRL, SHRD, SIDT, SLDT, SMSW, STC, STD, STI, STOSB, STOSD,
        STOSW, STR, SUB, SUBB, SUBW, SUBL, TEST, TESTB, TESTW, TESTL, VERR, VERW, WAIT, WBINVD,
        XADD, XADDB, XADDW, XADDL, XCHG, XCHGB, XCHGW, XCHGL, XLAT, XLATB, XOR, XORB, XORW, XORL},
  keywordstyle=\color{blue}\bfseries,
  ndkeywordstyle=\color{darkgray}\bfseries,
  identifierstyle=\color{black},
  sensitive=false,
  comment=[l]{\#},
  morecomment=[s]{/*}{*/},
  commentstyle=\color{purple}\ttfamily,
  stringstyle=\color{red}\ttfamily,
  morestring=[b]',
  morestring=[b]"
}

\lstset{language=assembler, style=codestyle}

% disponi sezioni
\usepackage{titlesec}

\titleformat{\section}
	{\sffamily\Large\bfseries} 
	{\thesection}{1em}{} 
\titleformat{\subsection}
	{\sffamily\large\bfseries}   
	{\thesubsection}{1em}{} 
\titleformat{\subsubsection}
	{\sffamily\normalsize\bfseries} 
	{\thesubsubsection}{1em}{}

% tikz
\usepackage{tikz}

% float
\usepackage{float}

% grafici
\usepackage{pgfplots}
\pgfplotsset{width=10cm,compat=1.9}

% disponi alberi
\usepackage{forest}

\forestset{
	rectstyle/.style={
		for tree={rectangle,draw,font=\large\sffamily}
	},
	roundstyle/.style={
		for tree={circle,draw,font=\large}
	}
}

% disponi algoritmi
\usepackage{algorithm}
\usepackage{algorithmic}
\makeatletter
\renewcommand{\ALG@name}{Algoritmo}
\makeatother

% disponi numeri di pagina
\usepackage{fancyhdr}
\fancyhf{} 
\fancyfoot[L]{\sffamily{\thepage}}

\makeatletter
\fancyhead[L]{\raisebox{1ex}[0pt][0pt]{\sffamily{\@title \ \@date}}} 
\fancyhead[R]{\raisebox{1ex}[0pt][0pt]{\sffamily{\@author}}}
\makeatother

\begin{document}
% sezione (data)
\section{Lezione del 10-12-24}

% stili pagina
\thispagestyle{empty}
\pagestyle{fancy}

% testo
\subsection{Conversione digitale/analogico e analogico/digitale}
Finora abbiamo assunto che le interfacce lavorino solo e soltanto con segnali digitali.
In verità nel mondo esterno al computer le grandezze variano su un una scala continua.
Occoronno appositi convertitori, detti convertitori digitale/analogico e analogico/digitale.


La grandezza analogica che consideriamo è un voltaggi appartenente alla scala FSR (Full Scale Range) $[5, 30] \mathrm{V}$.
Questa verrà convertita in un intero $x$ rappresentato su $N$ bit con $N \in \{8, 16\}$.
A seconda dell'interpolazione scelta, potremo distinguere fra:
\begin{itemize}
	\item \textbf{Conversione ubipolare:} $v \in [0, FSR]$, $x \in [0, 2^N -1]$;
	\item \textbf{Conversione bipolare:} $v \in \left[ -\frac{FSR}{2}, \frac{FSR}{2} \right]$, $x \in \left[ -2^{N-1}, +2^{N-1} -1 \right]$
\end{itemize}

\subsubsection{Errori di conversione}
Dato $K = \frac{FSR}{2^N}$, nel caso ideale vorremmo $v = Kx$.
In realtà, avremo che $|v - Kx| \leq \varepsilon$, con un $\varepsilon$ dovuto a errori di conversione:
\begin{itemize}
	\item \textbf{Errore di non linearità};
	\item \textbf{Errore di quantizzazione}.
\end{itemize}

\subsubsection{Tempi di risposta}
I convertitori D/A sono praticamente "combinatori", e quindi estrememente veloci (pochi nanosecondi).
I convertitori A/D, di contro, hanno tempi di risposta variabili in base alle architetture.
Noi vedremo i convertitori ad \textbf{approssimazioni successive} (\textbf{SAR}), che hanno tempi di risposta di qualche centinaio di nanosecondi.

\subsubsection{Convertitori unipolari/bipolari}
I convertitori D/A che lavorano con i numeri naturali usano la stessa rappresentazione in codice binario a cui siamo abituati.
I convertitori bipolari che lavorano con interi usano invece rappresentazioni in traslazione (detta appunto anche \textit{binaria bipolare}), anziché in complemento a 2.
Il numero intero $x$ viene quindi rappresentato dal naturale $X = x + 2^{N-1}$.
In ogni caso, per riportare in complemento a 2 basterà complementare il MSB, o analogamente sommare all'intero in questione $2^{n-1}$.

\subsection{Convertitore D/A}
Un convertitore D/A può essere schematizzato come segue:

\begin{center}\begin{circuitikz}\draw
  (-1.5,6) to[resistor, l = $R$, i<=$i_N$] ++(3,0)
  to[resistor, l = $R$] ++(3,0)
  to[resistor, l = $2R$, i<=$i_0$] ++(3,0)
  (7.5,6) node[ground]{}
  (4.5,6) to[resistor, l = $2R$] ++(0,-2)
  (1.5,6) to[resistor, l = $2R$] ++(0,-2)
  (-1.5,6) to[resistor, l = $2R$] ++(0,-2)
  (-1.5,6) to[short] ++(-1,0)
	(-2,2.5) to[short, i = $i$] ++(1,0) to[short] ++(7,0)
  (-1,3) to[short] ++(0,-1.5)
  to[short] ++(7,0)
  (8,2) to[short] ++(2,0)
  (5.5,2.5) to[short] ++(0,1)
	to[resistor, l = $R_\text{gain}$] ++(3,0)
  (8.5,2) to[short, i = $i_a$] ++(0,1.5)
  (5.5,3.5) to[short] ++(0,1)
	(5.5,4.5) to[resistor, l = $R_\text{offset}$] ++(3,0)
  (-2.5,6) node[circ]{}
  (8.5,4.5) node[circ]{}
  (-2,2.5) to[short] ++(0,0.5)
  (1,3) to[short] ++(0,-0.5)
  (2,1.5) to[short] ++(0,1.5)
  (4,3) to[short] ++(0,-0.5)
  (5,3) to[short] ++(0,-1.5)
  (5.5,1.5) node[ground]{};

	\node at (-3.175, 6) {$FSR$};
	\node at (9, 4.5) {$V_{pol}$};
	\node at (10.5, 2) {$V_{out}$};

	\draw (-1.5,3.45) node[spdt, rotate=-90, scale=1.59] {};
	\node at (-1, 3.5) {$x_2$};
	
	\draw (1.5,3.45) node[spdt, rotate=-90, scale=1.59] {};
	\node at (2, 3.5) {$x_1$};
	
	\draw (4.5, 3.45) node[spdt, rotate=-90, scale=1.59] {};
	\node at (5, 3.5) {$x_0$};

	\draw (7,2) node[op amp, scale = 1.02] {};

	\draw(5.5,2.5) node[circ] {};
	\node at (5.7,2.8) {A};

	\node at (-2.5, 2.5) {$V^-$};
	\node at (-1.5, 1.5) {$V^+$};

\end{circuitikz}\end{center}

La linea superiore è collegata a tensione $FSR$, cioè il valore massimo (fondo scala) del convertitore.
Si ha che, per resistenze parallele, la resistenza a destra di ogni ramo verticale (con le resistenze da $2R$) vale $2R$, e quindi la resistenza a sinistra del ramo vale $R$.
Assunte le due resistenze da $2R$ a destra collegate a massa (come sarà con gli interruttori commutati a destra, che per adesso assumiamo), si ha che la corrente che passa su di esse sarà $i_0$, quella che passa sulle due resistenze alla loro destra sarà $2 i_0$, e quindi che la corrente che passa su ogni ramo verticale sarà il doppio della corrente che passa sul ramo verticale immediatamente a destra.
Il valore di questa $i_0$ è dato prendendo la corrente che esce al nodo $FSR$, che sarà chiaramente:
$$
i_n = \frac{FSR}{R}
$$
e osservando che la corrente che passa sulla prima resistenza da $2R$ (quella collegata a massa) dovrà essere:
$$
i_0 = \frac{FSR}{R} \cdot \frac{1}{2^n} = \frac{K}{R}
$$

Vediamo velocemente che la cosa si mantiene anche commutando gli interruttori a sinistra. L'amplificatore operazionale mantiene l'uscita $V_{out}$ a tensione:
$$
V_{out} = \alpha \cdot (V^+ - V^-) = \alpha \cdot V^-
$$
visto che che la linea $V^+$ è collegata a massa.
Allo stesso tempo, dal lato sinistro dell'amplificatore, la resistenza $R_\text{gain}$ (che vedremo in seguito ha valenza, assieme alla resistenza $R_{offset}$, per la correzione dell'errore di non linearità del convertitore), ci fornisce:
$$
V_{out} = R_{gain} \cdot i_a + V^-
$$
da cui:
$$
R_{gain} \cdot i_a + V^- = \alpha \cdot V^-, \quad R_{gain} \cdot i_a = V^- (\alpha - 1) \implies V^- = \frac{R_{gain} \cdot i_a}{\alpha - 1}
$$
da cui, con $\alpha >> 1$, $V^- \approx 0$, cioè molto vicina a massa.

Si ha quindi che la corrente che passa sulla linea $V^-$ vale:
$$
i = i_0 \cdot x_0 + 2 i_0 \cdot x_1 + 4 i_0 \cdot x_2
$$
che è esattamente il naturale codificato da $ X = \{x_2 x_1 x_0\}$, cioè su 3 cifre.
Potremo infatti generalizzare a un numero $n$ di cifre arbitrario:
$$
i = i_0 \cdot x_0 + 2 i_0 \cdot x_1 + ... + 2^{n-1} i_0 \cdot x_{n-1} = i_0 \sum_{i=0}^{n-1} 2^i \cdot x_i = i_0 \cdot X
$$

Infine, sostituendo il valore ricavato prima per $i_0$, otteniamo:
$$
i = \frac{K}{R} \sum_{i = 0}^{n - 1} 2^i \cdot x_i = \frac{K}{R} \cdot X
$$

Vediamo allora come i resistori $R_\text{gain}$ e $R_\text{offset}$ influenzano il segnale $V_{out}$ in uscita, e come la tensione $V_{pol}$ trasforma il comportamento del convertitore da unipolare a bipolare.
Sostituendo la formula trovata prima per $V^-$ nell'equazione dell'amplificatore operazionale, si trova immediatamente:
$$
V_{out} = \alpha \cdot \frac{R_{gain} \cdot i_a}{\alpha - 1}
$$
che assumendo come prima $\alpha >> 1$ restituisce:
$$
V_{out} = R_{gain} \cdot i_a
$$
cioè la resistenza $R_\text{gain}$ funge da fattore di scala per il voltaggio $V_{out}$ in uscita.

Impostando poi il bilancio delle correnti al nodo A, si ottiene:
$$
i = \frac{K}{R} \cdot X = \frac{V_{out}}{R_\text{gain}} + \frac{V_{pol}}{R_\text{offset}} \implies V_{out} = R_\text{gain} \left( \frac{K}{R} \cdot X - \frac{V_{pol}}{R_\text{offset}} \right)
$$
da cui notiamo che la resistenza $R_\text{offset}$ regola l'intercetta, e la resistenza $R_\text{gain}$ la pendenza della retta che lega $X$ a $V_{out}$.
Inoltre, impostando $V_{pol} = 0$ si ottiene un \textbf{converitore unipolare}, mentre impostando $V_{pol} = \frac{FSR}{2}$ si ottiene un \textbf{converitore bipolare}.

\par\smallskip

Anche se non si considerano resistori e amplificatori operazionali come componenti combinatori, il circuito è effettivamente "combinatorio" nel senso che ha tempi di risposta estremamente veloci.
Il problema è però quello delle transizioni multiple dello stato di uscita: questo si risolve attraverso un filtro \textit{passa-basso} in uscita.

\lstset{language=verilog}

In Verilog, un implementazione semplificata di un convertitore D/A può essere la seguente, che emula un segnale analogico usando variabili di tipo \lstinline|real|:

\lstinputlisting[language=verilog, style=codestyle]{../verilog/12-10/digital_analog_converters/digital_analog_converter.v}

\subsubsection{Interfaccia per la conversione D/A}
Vediamo un'interfaccia parallela per l'operazione di un convertitore D/A.
Sul lato di uscita non si avranno handshake, in quanto il convertitore è in sé più veloce del clock del processore.

\lstinputlisting[language=verilog, style=codestyle]{../verilog/12-10/digital_analog_converters/digital_analog_interface.v}

\subsection{Convertitore A/D}
Descriviamo un particolare tipo di convertitori A/D detto convertitore ad \textbf{approssimazioni successive} (alternative potrebbero essere i convertitori \textit{paralleli} o i convertitori \textit{a rampa}, anch'essi basati su comparatori).
Il cuore di un convertitore di questo tipo è una rete sequenziale detta \textbf{SAR} (Successive Approximation Register).
L'uscita del SAR viene fatta passare attraverso un convertitore D/A dello stesso tipo dell'A/D, e confrontata attraverso un \textbf{comparatore} con l'ingresso corrente in modo da migliorare la previsione, in quella che è effettivamente una \textbf{ricerca logaritmica} (o \textit{binaria} o \textit{dicotomica}).
In particolare, ad ogni iterazione della ricerca si ricava il valore di un singolo bit, per cui $n$ bit richiedono $n$ iterazioni.
Lato processore, il SAR dovrà implementare inoltre un handshake, che scegliamo \lstinline|soc|/\lstinline|eoc|.

Una descrizione in linguaggio Verilog della SAR potrebbe essere la seguente.
Si noti che si presentano due versioni: il problema della prima è che abbiamo bisogno di un nuovo stato per ogni iterazione di aggiornamento di RBR; si introducono quindi nella seconda versione un registro COUNT e una rete combinatoria per il calcolo dei bit di RBR.

\lstinputlisting[language=verilog, style=codestyle]{../verilog/12-10/analog_digital_converters/analog_digital_converter.v}

\subsubsection{Interfaccia per la conversione A/D}
Vediamo un'interfaccia parallela per l'operazione di un convertitore A/D.
Lato processore si implementerà, come abbiamo visto, un handshake \lstinline|soc|/\lstinline|eoc|.

Un'implementazione in Verilog può essere la seguente:

\lstinputlisting[language=verilog, style=codestyle]{../verilog/12-10/analog_digital_converters/analog_digital_interface.v}

Si nota che l'interfaccia lascia al processore il compito di completare manualmente l'handshake, fornendo i bit soc e oc rispettivamente alle posizioni 1 e 0 della linea dati \lstinline|d7_d0|. 

\end{document}
