
\documentclass[a4paper,11pt]{article}
\usepackage[a4paper, margin=8em]{geometry}

% usa i pacchetti per la scrittura in italiano
\usepackage[french,italian]{babel}
\usepackage[T1]{fontenc}
\usepackage[utf8]{inputenc}
\frenchspacing 

% usa i pacchetti per la formattazione matematica
\usepackage{amsmath, amssymb, amsthm, amsfonts}

% usa altri pacchetti
\usepackage{gensymb}
\usepackage{hyperref}
\usepackage{standalone}

\usepackage{colortbl}

\usepackage{xstring}
\usepackage{karnaugh-map}

% imposta il titolo
\title{Appunti Reti Logiche}
\author{Luca Seggiani}
\date{2024}

% imposta lo stile
% usa helvetica
\usepackage[scaled]{helvet}
% usa palatino
\usepackage{palatino}
% usa un font monospazio guardabile
\usepackage{lmodern}

\renewcommand{\rmdefault}{ppl}
\renewcommand{\sfdefault}{phv}
\renewcommand{\ttdefault}{lmtt}

% circuiti
\usepackage{circuitikz}
\usetikzlibrary{babel}

% disponi il titolo
\makeatletter
\renewcommand{\maketitle} {
	\begin{center} 
		\begin{minipage}[t]{.8\textwidth}
			\textsf{\huge\bfseries \@title} 
		\end{minipage}%
		\begin{minipage}[t]{.2\textwidth}
			\raggedleft \vspace{-1.65em}
			\textsf{\small \@author} \vfill
			\textsf{\small \@date}
		\end{minipage}
		\par
	\end{center}

	\thispagestyle{empty}
	\pagestyle{fancy}
}
\makeatother

% disponi teoremi
\usepackage{tcolorbox}
\newtcolorbox[auto counter, number within=section]{theorem}[2][]{%
	colback=blue!10, 
	colframe=blue!40!black, 
	sharp corners=northwest,
	fonttitle=\sffamily\bfseries, 
	title=Teorema~\thetcbcounter: #2, 
	#1
}

% disponi definizioni
\newtcolorbox[auto counter, number within=section]{definition}[2][]{%
	colback=red!10,
	colframe=red!40!black,
	sharp corners=northwest,
	fonttitle=\sffamily\bfseries,
	title=Definizione~\thetcbcounter: #2,
	#1
}

% disponi codice
\usepackage{listings}
\usepackage[table]{xcolor}

\definecolor{codegreen}{rgb}{0,0.6,0}
\definecolor{codegray}{rgb}{0.5,0.5,0.5}
\definecolor{codepurple}{rgb}{0.58,0,0.82}
\definecolor{backcolour}{rgb}{0.95,0.95,0.92}

\lstdefinestyle{codestyle}{
		backgroundcolor=\color{black!5}, 
		commentstyle=\color{codegreen},
		keywordstyle=\bfseries\color{magenta},
		numberstyle=\sffamily\tiny\color{black!60},
		stringstyle=\color{green!50!black},
		basicstyle=\ttfamily\footnotesize,
		breakatwhitespace=false,         
		breaklines=true,                 
		captionpos=b,                    
		keepspaces=true,                 
		numbers=left,                    
		numbersep=5pt,                  
		showspaces=false,                
		showstringspaces=false,
		showtabs=false,                  
		tabsize=2
}

\lstdefinestyle{shellstyle}{
		backgroundcolor=\color{black!5}, 
		basicstyle=\ttfamily\footnotesize\color{black}, 
		commentstyle=\color{black}, 
		keywordstyle=\color{black},
		numberstyle=\color{black!5},
		stringstyle=\color{black}, 
		showspaces=false,
		showstringspaces=false, 
		showtabs=false, 
		tabsize=2, 
		numbers=none, 
		breaklines=true
}


\lstdefinelanguage{assembler}{ 
  keywords={AAA, AAD, AAM, AAS, ADC, ADCB, ADCW, ADCL, ADD, ADDB, ADDW, ADDL, AND, ANDB, ANDW, ANDL,
        ARPL, BOUND, BSF, BSFL, BSFW, BSR, BSRL, BSRW, BSWAP, BT, BTC, BTCB, BTCW, BTCL, BTR, 
        BTRB, BTRW, BTRL, BTS, BTSB, BTSW, BTSL, CALL, CBW, CDQ, CLC, CLD, CLI, CLTS, CMC, CMP,
        CMPB, CMPW, CMPL, CMPS, CMPSB, CMPSD, CMPSW, CMPXCHG, CMPXCHGB, CMPXCHGW, CMPXCHGL,
        CMPXCHG8B, CPUID, CWDE, DAA, DAS, DEC, DECB, DECW, DECL, DIV, DIVB, DIVW, DIVL, ENTER,
        HLT, IDIV, IDIVB, IDIVW, IDIVL, IMUL, IMULB, IMULW, IMULL, IN, INB, INW, INL, INC, INCB,
        INCW, INCL, INS, INSB, INSD, INSW, INT, INT3, INTO, INVD, INVLPG, IRET, IRETD, JA, JAE,
        JB, JBE, JC, JCXZ, JE, JECXZ, JG, JGE, JL, JLE, JMP, JNA, JNAE, JNB, JNBE, JNC, JNE, JNG,
        JNGE, JNL, JNLE, JNO, JNP, JNS, JNZ, JO, JP, JPE, JPO, JS, JZ, LAHF, LAR, LCALL, LDS,
        LEA, LEAVE, LES, LFS, LGDT, LGS, LIDT, LMSW, LOCK, LODSB, LODSD, LODSW, LOOP, LOOPE,
        LOOPNE, LSL, LSS, LTR, MOV, MOVB, MOVW, MOVL, MOVSB, MOVSD, MOVSW, MOVSX, MOVSXB,
        MOVSXW, MOVSXL, MOVZX, MOVZXB, MOVZXW, MOVZXL, MUL, MULB, MULW, MULL, NEG, NEGB, NEGW,
        NEGL, NOP, NOT, NOTB, NOTW, NOTL, OR, ORB, ORW, ORL, OUT, OUTB, OUTW, OUTL, OUTSB, OUTSD,
        OUTSW, POP, POPL, POPW, POPB, POPA, POPAD, POPF, POPFD, PUSH, PUSHL, PUSHW, PUSHB, PUSHA, 
				PUSHAD, PUSHF, PUSHFD, RCL, RCLB, RCLW, MOVSL, MOVSB, MOVSW, STOSL, STOSB, STOSW, LODSB, LODSW,
				LODSL, INSB, INSW, INSL, OUTSB, OUTSL, OUTSW
        RCLL, RCR, RCRB, RCRW, RCRL, RDMSR, RDPMC, RDTSC, REP, REPE, REPNE, RET, ROL, ROLB, ROLW,
        ROLL, ROR, RORB, RORW, RORL, SAHF, SAL, SALB, SALW, SALL, SAR, SARB, SARW, SARL, SBB,
        SBBB, SBBW, SBBL, SCASB, SCASD, SCASW, SETA, SETAE, SETB, SETBE, SETC, SETE, SETG, SETGE,
        SETL, SETLE, SETNA, SETNAE, SETNB, SETNBE, SETNC, SETNE, SETNG, SETNGE, SETNL, SETNLE,
        SETNO, SETNP, SETNS, SETNZ, SETO, SETP, SETPE, SETPO, SETS, SETZ, SGDT, SHL, SHLB, SHLW,
        SHLL, SHLD, SHR, SHRB, SHRW, SHRL, SHRD, SIDT, SLDT, SMSW, STC, STD, STI, STOSB, STOSD,
        STOSW, STR, SUB, SUBB, SUBW, SUBL, TEST, TESTB, TESTW, TESTL, VERR, VERW, WAIT, WBINVD,
        XADD, XADDB, XADDW, XADDL, XCHG, XCHGB, XCHGW, XCHGL, XLAT, XLATB, XOR, XORB, XORW, XORL},
  keywordstyle=\color{blue}\bfseries,
  ndkeywordstyle=\color{darkgray}\bfseries,
  identifierstyle=\color{black},
  sensitive=false,
  comment=[l]{\#},
  morecomment=[s]{/*}{*/},
  commentstyle=\color{purple}\ttfamily,
  stringstyle=\color{red}\ttfamily,
  morestring=[b]',
  morestring=[b]"
}

\lstset{language=assembler, style=codestyle}

% disponi sezioni
\usepackage{titlesec}

\titleformat{\section}
	{\sffamily\Large\bfseries} 
	{\thesection}{1em}{} 
\titleformat{\subsection}
	{\sffamily\large\bfseries}   
	{\thesubsection}{1em}{} 
\titleformat{\subsubsection}
	{\sffamily\normalsize\bfseries} 
	{\thesubsubsection}{1em}{}

% tikz
\usepackage{tikz}

% float
\usepackage{float}

% grafici
\usepackage{pgfplots}
\pgfplotsset{width=10cm,compat=1.9}

% disponi alberi
\usepackage{forest}

\forestset{
	rectstyle/.style={
		for tree={rectangle,draw,font=\large\sffamily}
	},
	roundstyle/.style={
		for tree={circle,draw,font=\large}
	}
}

% disponi algoritmi
\usepackage{algorithm}
\usepackage{algorithmic}
\makeatletter
\renewcommand{\ALG@name}{Algoritmo}
\makeatother

% disponi numeri di pagina
\usepackage{fancyhdr}
\fancyhf{} 
\fancyfoot[L]{\sffamily{\thepage}}

\makeatletter
\fancyhead[L]{\raisebox{1ex}[0pt][0pt]{\sffamily{\@title \ \@date}}} 
\fancyhead[R]{\raisebox{1ex}[0pt][0pt]{\sffamily{\@author}}}
\makeatother

\begin{document}
% sezione (data)
\section{Lezione del 10-10-24}

% stili pagina
\thispagestyle{empty}
\pagestyle{fancy}

% testo
\subsection{Sintesi di reti in forma SP a costo minimo}
Esistono due criteri di costo per le reti:
\begin{itemize}
	\item \textbf{A porte:} ogni \textbf{porta} conta per un'unità di costo;
	\item \textbf{A diodi:} ogni \textbf{ingresso} conta per un'unità di costo.
\end{itemize}

Presentiamo un metoodo, applicabile a reti con'un uscita, che produce reti in forma SP a 2 livelli di logica in quanto, per una legge combinatoria F, si ha::
$$
\text{Sintesi di F a 2 L.L. in forma SP} \subset \text{Sintesi di F a 2 L.L.} \subset \text{Sintesi di F}
$$

\subsubsection{Espansione di Shannon}
Si può dimostrare il seguente risultato:
\begin{theorem}{Espansione di Shannon}
	Si può sempre scrivere qualunque legge combinatoria $f$ come somma di prodotti degli ingressi (diretti o negati).
\end{theorem}

Questo significa che, se ho una legge combinatoria $z = f(x_{N-1}, ..., x_0)$, posso dire:
\[
	\begin{aligned}
		z = f(0, ..., 0,  0) \cdot \overline{x_{N-1}} \cdot \overline{x_{N-2}} \cdot ... \cdot \overline{x_1} \cdot \overline{x_0}	+ \\
		f(0, ..., 0, 1) \cdot \overline{x_{N-1}} \cdot \overline{x_{N-2}} \cdot ... \cdot \overline{x_1} \cdot x_0	+\\
		... + \\
		f(1, ..., 1, 0) \cdot x_{N-1} \cdot x_{N-2} \cdot ... \cdot x_1 \cdot \overline{x_0}	+\\
		f(1, ..., 1, 1) \cdot x_{N-1} \cdot x_{N-2} \cdot ... \cdot x_1 \cdot x_0	\\
	\end{aligned}
\]
che equivale a quanto avevamo visto con la sintesi di reti combinatorie a $N$ ingressi con multiplexer a $N$ variabili di comando.

A questo punto possiamo ottenere la cosiddetta \textbf{forma canonica SP}, applicando le proprietà:
\[
	\begin{cases}
		1 \cdot \alpha = \alpha \\ 
		0 \cdot \alpha = 0 \\ 
		0 + \beta = \beta
	\end{cases}
\]
all'espansione di Shannon (sostanzialmente rimuoviamo tutti i termini a cui corrispondono uscite negate).
Della forma canonica SP possiamo dire che è:
\begin{itemize}
	\item \textbf{SP:} è fatta da somme e prodotti;
	\item \textbf{Canonica:} ogni prodotto ha come fattori tutti gli ingressi, diretti o negati;
	\item Ciascuno dei termini della somma si chiama \textbf{mintermine};
	\item Ogni mintermine corrisponde ad uno stato riconosciuto dalla rete.
\end{itemize}

L'insieme dei termini (mintermini) sommati fra di loro che otteniamo dall'espansione di Shannon prende il nome di \textbf{lista di mintermini}.

\subsection{Semplificazione della forma canonica SP}
Definiamo quindi un metodo per la semplificazione della lista dei mintermini.
Divideremo quest'operazione in due passaggi principali:

\begin{itemize}
	\item \textbf{Identificazione degli implicanti principali}: si ricava una lista di termini ricavati da quelli di partenza, e di dimensioni più piccole, che rappresentano la stessa legge combinatoria;
	\item \textbf{Eliminazione delle ridondanze}: si rimuovono gli implicanti che non portano informazioni utili alla legge combinatoria.
\end{itemize}

\subsubsection{Metodo di Quine-McCluskey}
Si presenta il metodo di Quine-McCluskey per l'identificazione degli implicanti principali.
Questo metodo prevede di:
\begin{itemize}
	\item \textbf{Fondere i mintermini} applicando \textbf{esaustivamente} la regola:
$$
\alpha x + \alpha \bar{x} = \alpha
$$
che possiamo dimostrare come:
$$
\alpha x + \alpha \bar{x} = \alpha (x + \bar{x}) = \alpha, \quad x + \bar{x} = 1
$$
alla lista dei mintermini.

Ripetiamo questo passaggio $N - 1$ volte per la dimensione $N$ dei termini, riducendo ogni volta la dimensione degli implicanti di 1.
Si ricava una forma SP, detta \textbf{lista di implicanti}.
	\item \textbf{Rimuovere i duplicati} dalla lista dei duplicanti, applicando l'altra regola:
$$
\alpha x + \alpha = \alpha 
$$
sugli implicanti che hanno elementi in comune.
\end{itemize}

Troviamo quindi quella che è detta \textbf{lista degli implicanti principali}.
Questa lista contiene meno elementi della forma canonica SP, non è ancora di costo minimo: potrebbe contenere ridondanze, cioè implicanti non necessari alla corretta modelizzazione della legge combinatoria.

Vediamo un modo per eliminare queste ridondanze.

\subsubsection{Liste di copertura ridondanti}
Una \textbf{lista di copertura} è una lista di implicanti, la cui somma è una forma SP per la funzione $f$.
La \textbf{lista di copertura non ridondante} è la lista che smette di essere una lista di copertura appena si toglie un elemento.

La lista dei mintermini è una lista non ridondante, mentre la lista degli implicanti principali può esserlo.

Si introduce quindi uno strumento per la visualizzazione di ridondanze.

\subsubsection{Mappe di Karnaugh}
Per una rete a $N$ ingressi la mappa di Karnaugh è una matrice di $2^N$ celle, dove le coordinate rappresentano gli ingressi, e gli elementi della matrice le uscite.
Sono diagrammi che tornano utili per rappresentare graficamente gli implicanti, ed eliminarne le ridonanze.
Vediamo, ad esempio, mappe con $N = 2$, $3$ e $4$:

\begin{center}
\begin{karnaugh-map}[2][2][1]
		\minterms{1,2}
		\maxterms{0,3}
\end{karnaugh-map}
\end{center}

\noindent
\begin{minipage}{0.45\textwidth}
	\begin{karnaugh-map}[4][2][1][$X_1X_0$][$X_2$]
			\minterms{3,4}
			\maxterms{0,1,6,7}
			\indeterminants{2,5}
	\end{karnaugh-map}
\end{minipage}%
\hfill
\begin{minipage}{0.45\textwidth}
\begin{karnaugh-map}
		\manualterms{0,0,0,0,0,0,0,0,0,0,0,0,0,0,0,0}
\end{karnaugh-map}
\end{minipage}


In una mappa di Karnaugh, celle \textbf{contigue} hanno coordinate \textbf{adiacenti}, e viceversa.
Oltre le 4 coordinate, per le mappe non possiamo più rappresentare queste mappe senza la terza dimensione.

Definiamo:
\begin{itemize}
	\item \textbf{Sottocubo di ordine 1:} una casella che contiene un 1, corrispondente quindi ad uno stato di ingresso riconosciuto dalla rete, si indica come SO1;
	\item \textbf{Coordinate} di un SO1: stato di ingresso corrispondente al sottocubo;
	\item \textbf{Adiacenza} fra SO1: due SO1 sono adiacenti se differiscono fra loro di una sola coordinata.
\end{itemize}

Vediamo, ad esempio, una mappa di Karnaugh con $N=2$, una serie di sottocubi di ordine 1 con la tabella associata:

\begin{center}
\noindent
\begin{minipage}{0.15\textwidth}
\begin{karnaugh-map}[2][2][1]
		\minterms{1,2}
		\maxterms{0,3}
		\implicant{1}{1}
		\implicant{2}{2}
\end{karnaugh-map}
\end{minipage}%
\hspace{3cm}
\begin{minipage}{0.15\textwidth}
	\begin{table}[H]
		\center \rowcolors{2}{white}{black!10}
		\begin{tabular} { c || c | c }
			& $x_1$ & $x_0$ \\ 
			\hline 
			\rowcolor{red!20!white} A & 0 & 1 \\
			\rowcolor{green!20!white} B & 1 & 0 \\
		\end{tabular}
	\end{table}
\end{minipage}
\end{center}

Notiamo come A corrisponde all'implicante $\overline{x_1}x_0$, e B all'implicante $x_1\overline{x_0}$.

Possiamo continuare:
\begin{itemize}
	\item \textbf{Sottocubo di ordine 2:} costituito da SO1 adiacenti, e si dice che \textbf{copre} i SO1 che lo formano. Si indica come SO2;
	\item \textbf{Sottocubo di ordine 4:} costituito da SO2 adiacenti, e si dice che \textbf{copre} i SO2 che lo formano. Si indica come SO4;
	\item \textbf{Sottocubo di ordine 8:} costituito da SO4 adiacenti, e si dice che \textbf{copre} i SO4 che lo formano. Si indica come SO8;
\end{itemize}

Vediamo un'ultimo esempio, con $N=4$: 

\begin{center}
\noindent
\begin{minipage}{0.3\textwidth}
\begin{karnaugh-map}
		\manualterms{0,0,0,0,0,0,0,0,0,0,0,0,0,0,0,0}
		\implicant{0}{1}
		\implicant{7}{14}
		\implicant{2}{10}
		\implicantcorner
\end{karnaugh-map}
\end{minipage}%
\hspace{3cm}
\begin{minipage}{0.3\textwidth}
	\begin{table}[H]
		\center \rowcolors{2}{white}{black!10}
		\begin{tabular} { c || c | c | c | c}
			& $x_3$ & $x_2$ & $x_1$ & $x_0$ \\ 
			\hline 
			\rowcolor{red!20!white} A & 0 & - & - & 1 \\
			\rowcolor{green!20!white} B & 1 & - & - & 1 \\
			\rowcolor{yellow!20!white} C & 1 & 0 & - & - \\
			\rowcolor{cyan!20!white} D & - & 0 & - & 0 \\
		\end{tabular}
	\end{table}
\end{minipage}
\end{center}

Notiamo dall'esempio che le mappe di Karnaugh rispettano il cosiddetti \textit{effetto pacman}: lo stesso implicante può esistere su lati opposti della mappa.
Il bisogno di rappresentare le adiacenze dà origine a questa particolarità, come determina l'ordine particolare delle attivazioni degli ingressi.
Inoltre, notiamo come i trattini nelle tabelle delle coordinate denotano che la variabile non influenza l'implicante, cioè rappresentano, in inglese, un \textit{don't care}.

\subsubsection{Ricerca delle liste di copertura non ridondanti}

Si dice che un sottocubo è \textbf{principale} quando non esiste nessun sottocubo più grande che lo copre completamente.
Si ha quindi che sottocubi e implicanti sono correlati: un sottocubo principale di ordine $p$ rappresenta un implicante principale di $N - \log_2(p)$ variabili.

Una \textbf{lista di copertura} è l'insieme (qualunque) di sottocubi che coprono tutti i SO1.
Una \textbf{lista di copertura non ridondante} è una lista di copertura che smette di essere tale quando si toglie un sottocubo.

Si presenta finalmente l'algoritmo:
\begin{algorithm}
\caption{per la ricerca dei sottocubi principali}
\begin{algorithmic}
	\STATE \textbf{Input:} i sottocubi di ordine più grande trovati sulla mappa 
	\STATE \textbf{Output:} i sottocubi principali della mappa
	\STATE \textsf{ciclo:}
	\STATE Considera tutti i sottocubi di ordine $p$ non interamente contenuti in sottocubi di ordine più grande, e
	segnali tutti: questi sono sicuramente principali
	\IF{l'insieme trovato finora basta a coprire tutta a mappa}
		\STATE L'algoritmo è terminato 
	\ELSE 
		\STATE Poni $p \leftarrow \frac{p}{2}$ e vai a \textsf{ciclo}
	\ENDIF 
\end{algorithmic}
\end{algorithm}

Ad esempio, nello scorso esempio, l'algoritmo rimuoverebbe l'implicante C.

\par\smallskip

Notiamo che alcuni sottocubi sono gli unici a coprire un dato sottocubo di ordine 1. In questo caso, si chiamano sottocubi \textbf{essenziali}, e costituiscono il \textbf{cuore} (\textit{core}) della mappa.

\end{document}

