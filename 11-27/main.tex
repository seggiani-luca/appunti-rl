
\documentclass[a4paper,11pt]{article}
\usepackage[a4paper, margin=8em]{geometry}

% usa i pacchetti per la scrittura in italiano
\usepackage[french,italian]{babel}
\usepackage[T1]{fontenc}
\usepackage[utf8]{inputenc}
\frenchspacing 

% usa i pacchetti per la formattazione matematica
\usepackage{amsmath, amssymb, amsthm, amsfonts}

% usa altri pacchetti
\usepackage{gensymb}
\usepackage{hyperref}
\usepackage{standalone}

\usepackage{colortbl}

\usepackage{xstring}
\usepackage{karnaugh-map}

% imposta il titolo
\title{Appunti Reti Logiche}
\author{Luca Seggiani}
\date{2024}

% imposta lo stile
% usa helvetica
\usepackage[scaled]{helvet}
% usa palatino
\usepackage{palatino}
% usa un font monospazio guardabile
\usepackage{lmodern}

\renewcommand{\rmdefault}{ppl}
\renewcommand{\sfdefault}{phv}
\renewcommand{\ttdefault}{lmtt}

% circuiti
\usepackage{circuitikz}
\usetikzlibrary{babel}

% testo cerchiato
\newcommand*\circled[1]{\tikz[baseline=(char.base)]{
            \node[shape=circle,draw,inner sep=2pt] (char) {#1};}}

% disponi il titolo
\makeatletter
\renewcommand{\maketitle} {
	\begin{center} 
		\begin{minipage}[t]{.8\textwidth}
			\textsf{\huge\bfseries \@title} 
		\end{minipage}%
		\begin{minipage}[t]{.2\textwidth}
			\raggedleft \vspace{-1.65em}
			\textsf{\small \@author} \vfill
			\textsf{\small \@date}
		\end{minipage}
		\par
	\end{center}

	\thispagestyle{empty}
	\pagestyle{fancy}
}
\makeatother

% disponi teoremi
\usepackage{tcolorbox}
\newtcolorbox[auto counter, number within=section]{theorem}[2][]{%
	colback=blue!10, 
	colframe=blue!40!black, 
	sharp corners=northwest,
	fonttitle=\sffamily\bfseries, 
	title=Teorema~\thetcbcounter: #2, 
	#1
}

% disponi definizioni
\newtcolorbox[auto counter, number within=section]{definition}[2][]{%
	colback=red!10,
	colframe=red!40!black,
	sharp corners=northwest,
	fonttitle=\sffamily\bfseries,
	title=Definizione~\thetcbcounter: #2,
	#1
}

% disponi codice
\usepackage{listings}
\usepackage[table]{xcolor}

\definecolor{codegreen}{rgb}{0,0.6,0}
\definecolor{codegray}{rgb}{0.5,0.5,0.5}
\definecolor{codepurple}{rgb}{0.58,0,0.82}
\definecolor{backcolour}{rgb}{0.95,0.95,0.92}

\lstdefinestyle{codestyle}{
		backgroundcolor=\color{black!5}, 
		commentstyle=\color{codegreen},
		keywordstyle=\bfseries\color{magenta},
		numberstyle=\sffamily\tiny\color{black!60},
		stringstyle=\color{green!50!black},
		basicstyle=\ttfamily\footnotesize,
		breakatwhitespace=false,         
		breaklines=true,                 
		captionpos=b,                    
		keepspaces=true,                 
		numbers=left,                    
		numbersep=5pt,                  
		showspaces=false,                
		showstringspaces=false,
		showtabs=false,                  
		tabsize=2
}

\lstdefinestyle{shellstyle}{
		backgroundcolor=\color{black!5}, 
		basicstyle=\ttfamily\footnotesize\color{black}, 
		commentstyle=\color{black}, 
		keywordstyle=\color{black},
		numberstyle=\color{black!5},
		stringstyle=\color{black}, 
		showspaces=false,
		showstringspaces=false, 
		showtabs=false, 
		tabsize=2, 
		numbers=none, 
		breaklines=true
}


\lstdefinelanguage{assembler}{ 
  keywords={AAA, AAD, AAM, AAS, ADC, ADCB, ADCW, ADCL, ADD, ADDB, ADDW, ADDL, AND, ANDB, ANDW, ANDL,
        ARPL, BOUND, BSF, BSFL, BSFW, BSR, BSRL, BSRW, BSWAP, BT, BTC, BTCB, BTCW, BTCL, BTR, 
        BTRB, BTRW, BTRL, BTS, BTSB, BTSW, BTSL, CALL, CBW, CDQ, CLC, CLD, CLI, CLTS, CMC, CMP,
        CMPB, CMPW, CMPL, CMPS, CMPSB, CMPSD, CMPSW, CMPXCHG, CMPXCHGB, CMPXCHGW, CMPXCHGL,
        CMPXCHG8B, CPUID, CWDE, DAA, DAS, DEC, DECB, DECW, DECL, DIV, DIVB, DIVW, DIVL, ENTER,
        HLT, IDIV, IDIVB, IDIVW, IDIVL, IMUL, IMULB, IMULW, IMULL, IN, INB, INW, INL, INC, INCB,
        INCW, INCL, INS, INSB, INSD, INSW, INT, INT3, INTO, INVD, INVLPG, IRET, IRETD, JA, JAE,
        JB, JBE, JC, JCXZ, JE, JECXZ, JG, JGE, JL, JLE, JMP, JNA, JNAE, JNB, JNBE, JNC, JNE, JNG,
        JNGE, JNL, JNLE, JNO, JNP, JNS, JNZ, JO, JP, JPE, JPO, JS, JZ, LAHF, LAR, LCALL, LDS,
        LEA, LEAVE, LES, LFS, LGDT, LGS, LIDT, LMSW, LOCK, LODSB, LODSD, LODSW, LOOP, LOOPE,
        LOOPNE, LSL, LSS, LTR, MOV, MOVB, MOVW, MOVL, MOVSB, MOVSD, MOVSW, MOVSX, MOVSXB,
        MOVSXW, MOVSXL, MOVZX, MOVZXB, MOVZXW, MOVZXL, MUL, MULB, MULW, MULL, NEG, NEGB, NEGW,
        NEGL, NOP, NOT, NOTB, NOTW, NOTL, OR, ORB, ORW, ORL, OUT, OUTB, OUTW, OUTL, OUTSB, OUTSD,
        OUTSW, POP, POPL, POPW, POPB, POPA, POPAD, POPF, POPFD, PUSH, PUSHL, PUSHW, PUSHB, PUSHA, 
				PUSHAD, PUSHF, PUSHFD, RCL, RCLB, RCLW, MOVSL, MOVSB, MOVSW, STOSL, STOSB, STOSW, LODSB, LODSW,
				LODSL, INSB, INSW, INSL, OUTSB, OUTSL, OUTSW
        RCLL, RCR, RCRB, RCRW, RCRL, RDMSR, RDPMC, RDTSC, REP, REPE, REPNE, RET, ROL, ROLB, ROLW,
        ROLL, ROR, RORB, RORW, RORL, SAHF, SAL, SALB, SALW, SALL, SAR, SARB, SARW, SARL, SBB,
        SBBB, SBBW, SBBL, SCASB, SCASD, SCASW, SETA, SETAE, SETB, SETBE, SETC, SETE, SETG, SETGE,
        SETL, SETLE, SETNA, SETNAE, SETNB, SETNBE, SETNC, SETNE, SETNG, SETNGE, SETNL, SETNLE,
        SETNO, SETNP, SETNS, SETNZ, SETO, SETP, SETPE, SETPO, SETS, SETZ, SGDT, SHL, SHLB, SHLW,
        SHLL, SHLD, SHR, SHRB, SHRW, SHRL, SHRD, SIDT, SLDT, SMSW, STC, STD, STI, STOSB, STOSD,
        STOSW, STR, SUB, SUBB, SUBW, SUBL, TEST, TESTB, TESTW, TESTL, VERR, VERW, WAIT, WBINVD,
        XADD, XADDB, XADDW, XADDL, XCHG, XCHGB, XCHGW, XCHGL, XLAT, XLATB, XOR, XORB, XORW, XORL},
  keywordstyle=\color{blue}\bfseries,
  ndkeywordstyle=\color{darkgray}\bfseries,
  identifierstyle=\color{black},
  sensitive=false,
  comment=[l]{\#},
  morecomment=[s]{/*}{*/},
  commentstyle=\color{purple}\ttfamily,
  stringstyle=\color{red}\ttfamily,
  morestring=[b]',
  morestring=[b]"
}

\lstset{language=assembler, style=codestyle}

% disponi sezioni
\usepackage{titlesec}

\titleformat{\section}
	{\sffamily\Large\bfseries} 
	{\thesection}{1em}{} 
\titleformat{\subsection}
	{\sffamily\large\bfseries}   
	{\thesubsection}{1em}{} 
\titleformat{\subsubsection}
	{\sffamily\normalsize\bfseries} 
	{\thesubsubsection}{1em}{}

% tikz
\usepackage{tikz}

% float
\usepackage{float}

% grafici
\usepackage{pgfplots}
\pgfplotsset{width=10cm,compat=1.9}

% disponi alberi
\usepackage{forest}

\forestset{
	rectstyle/.style={
		for tree={rectangle,draw,font=\large\sffamily}
	},
	roundstyle/.style={
		for tree={circle,draw,font=\large}
	}
}

% disponi algoritmi
\usepackage{algorithm}
\usepackage{algorithmic}
\makeatletter
\renewcommand{\ALG@name}{Algoritmo}
\makeatother

% disponi numeri di pagina
\usepackage{fancyhdr}
\fancyhf{} 
\fancyfoot[L]{\sffamily{\thepage}}

\makeatletter
\fancyhead[L]{\raisebox{1ex}[0pt][0pt]{\sffamily{\@title \ \@date}}} 
\fancyhead[R]{\raisebox{1ex}[0pt][0pt]{\sffamily{\@author}}}
\makeatother

\begin{document}
% sezione (data)
\section{Lezione del 27-11-24}

% stili pagina
\thispagestyle{empty}
\pagestyle{fancy}

% testo
\subsection{Letture e scritture nello spazio di memoria}
Durante la fase di fetch, abbiamo eseguito solo \textbf{letture} in memoria.
Vediamo adesso che nella fase \textbf{execute} abbiamo bisogno di effettuare sia letture che scritture.
Vediamo come si effettuano a livello di $\mu$-istruzioni queste letture e scritture, in maniera compatibile con le temporizzazioni già definite sulle memorie.

\subsubsection{Lettura}
Innanzitutto ripassiamo le temporizzazioni nel caso della lettura:
\begin{itemize}
	\item A indirizzi stabili arriva il comando \lstinline|/mr|;
	\item \lstinline|/s| arriva con ritardo;
	\item A \lstinline|/mr| e \lstinline|/s| bassi si effettua la lettura, cioè le tri-state vanno in conduzione;
	\item I multiplexer alle uscite vanno a regime dopo gli indirizzi, da qui in puoi i dati sono buoni e si prelevano;
	\item Quando \lstinline|/mr| torna a 1 i dati tornano ad alta impedenza, da lì in poi \lstinline|/s| e gli indirizzi possono tornare instabili.
\end{itemize}

Definiamo allora un $\mu$-programma per la lettura in memoria:
\begin{lstlisting}[language=verilog, style=codestyle]	
mem_r0: begin A23_A0 <= address; DIR <= 0; MR <= 0; STAR <= mem_r1; end
mem_r1: begin STAR <= mem_r2; end // stato di wait, da qui in poi omesso 
mem_r2: begin cpu_register <= d7_d0; MR <= 1; ...; end 
\end{lstlisting}
Notiamo che allo stato R2 si possono cambiare tutte le linee (indirizzo, ecc...) tranne che la DIR, in quanto impostando solo a quel punto MR non si è sicuri che la RAM risponda in tempo.

\subsubsection{Scrittura}
Ricordiamo che la scrittura è distruttiva.
Ricordiamo quindi le temporizzazioni:
\begin{itemize}
	\item Si abbassa \lstinline|/s| e ci si assicura che gli indirizzi siano stabili;
	\item Si abbassa \lstinline|/mr|;
	\item I dati dovranno essere corretti fino al fronte di salita di \lstinline|/mr|.
\end{itemize}

Definiamo un'altro $\mu$-programma, stavolta per la scrittura in memoria:
\begin{lstlisting}[language=verilog, style=codestyle]	
mem_w0: begin A23_A0 <= address; D7_D0 <= new_byte; DIR <= 1; STAR <= mem_w1; end
mem_w1: begin MW <= 0; STAR <= mem_w2; end
mem_w2: begin MW <= 1; STAR <= mem_w3; end
mem_w3: begin DIR <= 0; ...; end
\end{lstlisting}
Notiamo di non poter abbassare DIR o gli indirizzi fino allo stato W3, in quanto non si può essere sicuri che a quel punto la RAM abbia finito di scrivere.

\subsection{Letture e scritture nello spazio di I/O}
Le letture e le scritture nello spazio di I/O sono diverse, in quanto qui \textbf{la lettura è distruttiva}.
Inoltre, dobbiamo ricordarci di operare sui registri IOR e IOW anzichè MR e MW.

\subsubsection{Lettura}
Scriviamo quindi un $\mu$-programma per la lettura nello spazio di I/O dove teniamo conto di dover abbassare IOR \textbf{dopo} che gli indirizzi sono stabili, in maniera simile alla lettura:
\begin{lstlisting}[language=verilog, style=codestyle]	
io_r0: begin A23_A0 <= address; DIR <= 0; STAR <= io_r1; end
io_r1: begin IOR <= 0; STAR <= io_r2; end
io_r2: begin STAR <= io_r3; end
io_r3: begin cpu_register <= d7_d0; IOR <= 1; ...; end
\end{lstlisting}

\subsubsection{Scrittura}
Ridefiniamo quindi il $\mu$-programma di scrittura:
\begin{lstlisting}[language=verilog, style=codestyle]	
io_w0: begin A23_A0 <= address; D7_D0 <= new_byte; DIR <= 1; STAR <= io_w1; end
io_w1: begin IOW <= 0; STAR <= io_w2; end
io_w2: begin IOW <= 1; STAR <= io_w3; end
io_w3: begin DIR <= 0; ...; end
\end{lstlisting}

Notiamo che, in questo caso, la scrittura si fa sul fronte di discesa anziché di salita, e quindi l'assegnamento di \lstinline|D7_D0| al nuovo byte va fatto esclusivamente in W0, e non in W1 com'era possibile nella scrittura nello spazio di memoria. 

# metti gli address giusti, qui sarebbe 'H00 + offset

\subsection{Accessi multipli in memoria}
Il processore potrebbe fare accessi non solo ad un byte, ma 2 byte (per operandi su 16 bit) o 3 byte (per indirizzi).
Occasionalmente dovremo leggere anche 4 byte, ma questo non è considerato nel corso.

Per fare letture su locazioni multiple, si usano $\mu$-sottoprogrammi di lettura/scrittura modulari.
Utilizzeremo:
\begin{itemize}
	\item Il registro \textbf{MJR} per contenere il $\mu$-indirizzo di ritorno;
	\item Il registro \textbf{NUMLOC} come contatore del numero di byte da leggere/scrivere;
	\item Il registro A23\_A0 per contenere l'indirizzo del primo byte da leggere/scrivere;
	\item I registri APP0, ..., APP3 per contenere i byte letti/da scrivere.
\end{itemize}

\subsubsection{Lettura}
Vediamo quindi il $\mu$-programma principale di lettura:
\begin{lstlisting}[language=verilog, style=codestyle]	
s_x: begin ... A23_A0 <= address; MJR <= s_x+1; STAR <= subprogram; end
s_x+1: begin ... /* qui si usa APP0 */ end
\end{lstlisting}

Definiamo 4 $\mu$-sottoprogrammi:
\begin{itemize}
	\item readB: legge 1 byte;
	\item readW: legge 2 byte;
	\item readM: legge 3 byte;
	\item readL: legge 4 byte.
\end{itemize}

I parametri di ingresso saranno l'indirizzo in memoria della prima locazione e la DIR impostata a 0, i parametri di uscita i registri APP da 0 a 3 (che conterranno i byte letti).

Vediamo quindi i $\mu$-sottoprogrammi di lettura:
\begin{lstlisting}[language=verilog, style=codestyle]	
readB: begin MR <= 0; NUMLOC <= 1; STAR <= read0; end;
readW: begin MR <= 0; NUMLOC <= 2; STAR <= read0; end;
readM: begin MR <= 0; NUMLOC <= 3; STAR <= read0; end;
readL: begin MR <= 0; NUMLOC <= 4; STAR <= read0; end;

read0: begin APP0 <= d7_d0; A23_A0 <= A23_A0 + 1; NUMLOC <= NUMLOC - 1; STAR <= ( NUMLOC == 1 ) ? read4 : read1; end
read1: begin APP1 <= d7_d0; A23_A0 <= A23_A0 + 1; NUMLOC <= NUMLOC - 1; STAR <= ( NUMLOC == 1 ) ? read4 : read2; end
read2: begin APP2 <= d7_d0; A23_A0 <= A23_A0 + 1; NUMLOC <= NUMLOC - 1; STAR <= ( NUMLOC == 1 ) ? read4 : read3; end
read3: begin APP3 <= d7_d0; A23_A0 <= A23_A0 + 1; STAR <= read4; end
read4: begin MR <= 1; STAR <= MJR; end
\end{lstlisting}

# non sono sicuro, comunque commentali

\subsubsection{Scrittura}
Vediamo il $\mu$-programma principale di scrittura:
\begin{lstlisting}[language=verilog, style=codestyle]	
s: begin ... APP1 <= datum(15:8); APP0 <= datum(7:0); A23_A= <= address; MJR = s_x+1; STAR <= subprogram; end
\end{lstlisting}

E definiamo i 4 $\mu$-sottoprogrammi:
\begin{itemize}
	\item writeB: scrive 1 byte;
	\item writeW: scrive 2 byte;
	\item writeM: scrive 3 byte;
	\item writeL: scrive 4 byte.
\end{itemize}

I parametri di ingresso saranno l'indirizzo in memoria della prima locazione, la DIR impostata a 0, e i registri APP da 0 a 3 (che conterranno i byte da scrivere).

Implementiamo i $\mu$-sottoprogrammi come:
\begin{lstlisting}[language=verilog, style=codestyle]	
writeB: begin D7_D0 <= APP0; DIR <= 1; NUMLOC <= 1; STAR <= write0; end 
writeW: begin D7_D0 <= APP0; DIR <= 2; NUMLOC <= 1; STAR <= write0; end
writeM: begin D7_D0 <= APP0; DIR <= 3; NUMLOC <= 1; STAR <= write0; end
writeL: begin D7_D0 <= APP0; DIR <= 4; NUMLOC <= 1; STAR <= write0; end

write0: begin MW <= 0; STAR <= write1; end;
write1: begin MW <= 1; STAR <= ( NUMLOC == 1) ? write11 : write2; end
\end{lstlisting}
# finisci qua sopra

\subsection{Descrizione in Verilog del processore}
Iniziamo quindi a definire il nostro processore col linguaggio Verilog.
# qui per l'amor di cristo riscrivi per benino e metti nella cartella /verilog che così è drammatico


\end{document}
