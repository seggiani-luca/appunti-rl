
\documentclass[a4paper,11pt]{article}
\usepackage[a4paper, margin=8em]{geometry}

% usa i pacchetti per la scrittura in italiano
\usepackage[french,italian]{babel}
\usepackage[T1]{fontenc}
\usepackage[utf8]{inputenc}
\frenchspacing 

% usa i pacchetti per la formattazione matematica
\usepackage{amsmath, amssymb, amsthm, amsfonts}

% usa altri pacchetti
\usepackage{gensymb}
\usepackage{hyperref}
\usepackage{standalone}

% imposta il titolo
\title{Appunti Reti Logiche}
\author{Luca Seggiani}
\date{2024}

% imposta lo stile
% usa helvetica
\usepackage[scaled]{helvet}
% usa palatino
\usepackage{palatino}
% usa un font monospazio guardabile
\usepackage{lmodern}

\renewcommand{\rmdefault}{ppl}
\renewcommand{\sfdefault}{phv}
\renewcommand{\ttdefault}{lmtt}

% disponi il titolo
\makeatletter
\renewcommand{\maketitle} {
	\begin{center} 
		\begin{minipage}[t]{.8\textwidth}
			\textsf{\huge\bfseries \@title} 
		\end{minipage}%
		\begin{minipage}[t]{.2\textwidth}
			\raggedleft \vspace{-1.65em}
			\textsf{\small \@author} \vfill
			\textsf{\small \@date}
		\end{minipage}
		\par
	\end{center}

	\thispagestyle{empty}
	\pagestyle{fancy}
}
\makeatother

% disponi teoremi
\usepackage{tcolorbox}
\newtcolorbox[auto counter, number within=section]{theorem}[2][]{%
	colback=blue!10, 
	colframe=blue!40!black, 
	sharp corners=northwest,
	fonttitle=\sffamily\bfseries, 
	title=Teorema~\thetcbcounter: #2, 
	#1
}

% disponi definizioni
\newtcolorbox[auto counter, number within=section]{definition}[2][]{%
	colback=red!10,
	colframe=red!40!black,
	sharp corners=northwest,
	fonttitle=\sffamily\bfseries,
	title=Definizione~\thetcbcounter: #2,
	#1
}

% disponi codice
\usepackage{listings}
\usepackage[table]{xcolor}

\lstdefinestyle{codestyle}{
		backgroundcolor=\color{black!5}, 
		commentstyle=\color{codegreen},
		keywordstyle=\bfseries\color{magenta},
		numberstyle=\sffamily\tiny\color{black!60},
		stringstyle=\color{green!50!black},
		basicstyle=\ttfamily\footnotesize,
		breakatwhitespace=false,         
		breaklines=true,                 
		captionpos=b,                    
		keepspaces=true,                 
		numbers=left,                    
		numbersep=5pt,                  
		showspaces=false,                
		showstringspaces=false,
		showtabs=false,                  
		tabsize=2
}

\lstdefinestyle{shellstyle}{
		backgroundcolor=\color{black!5}, 
		basicstyle=\ttfamily\footnotesize\color{black}, 
		commentstyle=\color{black}, 
		keywordstyle=\color{black},
		numberstyle=\color{black!5},
		stringstyle=\color{black}, 
		showspaces=false,
		showstringspaces=false, 
		showtabs=false, 
		tabsize=2, 
		numbers=none, 
		breaklines=true
}

\lstdefinelanguage{javascript}{
	keywords={typeof, new, true, false, catch, function, return, null, catch, switch, var, if, in, while, do, else, case, break},
	keywordstyle=\color{blue}\bfseries,
	ndkeywords={class, export, boolean, throw, implements, import, this},
	ndkeywordstyle=\color{darkgray}\bfseries,
	identifierstyle=\color{black},
	sensitive=false,
	comment=[l]{//},
	morecomment=[s]{/*}{*/},
	commentstyle=\color{purple}\ttfamily,
	stringstyle=\color{red}\ttfamily,
	morestring=[b]',
	morestring=[b]"
}

% disponi sezioni
\usepackage{titlesec}

\titleformat{\section}
	{\sffamily\Large\bfseries} 
	{\thesection}{1em}{} 
\titleformat{\subsection}
	{\sffamily\large\bfseries}   
	{\thesubsection}{1em}{} 
\titleformat{\subsubsection}
	{\sffamily\normalsize\bfseries} 
	{\thesubsubsection}{1em}{}

% tikz
\usepackage{tikz}

% float
\usepackage{float}

% grafici
\usepackage{pgfplots}
\pgfplotsset{width=10cm,compat=1.9}

% disponi alberi
\usepackage{forest}

\forestset{
	rectstyle/.style={
		for tree={rectangle,draw,font=\large\sffamily}
	},
	roundstyle/.style={
		for tree={circle,draw,font=\large}
	}
}

% disponi algoritmi
\usepackage{algorithm}
\usepackage{algorithmic}
\makeatletter
\renewcommand{\ALG@name}{Algoritmo}
\makeatother

% disponi numeri di pagina
\usepackage{fancyhdr}
\fancyhf{} 
\fancyfoot[L]{\sffamily{\thepage}}

\makeatletter
\fancyhead[L]{\raisebox{1ex}[0pt][0pt]{\sffamily{\@title \ \@date}}} 
\fancyhead[R]{\raisebox{1ex}[0pt][0pt]{\sffamily{\@author}}}
\makeatother

\begin{document}
% sezione (data)
\section{Lezione del 26-09-24}

% stili pagina
\thispagestyle{empty}
\pagestyle{fancy}

% testo
\subsection{Istruzioni di trasferimento}
Le istruzioni di trasferimento spostano memoria:
\begin{itemize}
	\item Dalla memoria a un registro;
	\item Da un registro a un registro;
	\item Dallo spazio I/O a un regsitro.
\end{itemize}

Non esistono altre possibilità, ergo non si può (per quanto interessa a noi) spostare da memoria a memoria.
In verità esistono alcune istruzioni nei processori di nuova generazione che ottimizzano operazioni di questo tipo, che verrano viste in seguito.
Sfruttando i registri, il trasferimento da memoria a memoria si fa attraverso un registro, in due istruzioni.

Nessuna istruzione di trasferimento modifica i flag.

\subsubsection{MOVE}
\begin{itemize}
	\item \textbf{Formato:} \texttt{MOV source, destination}
	\item \textbf{Azione:} sostituisce l'operando destinatario con una copia dell'operando sorgente.
	\item \textbf{Flag:} nessuno.

		\begin{table}[h!]
			\center \rowcolors{2}{white}{black!10}
			\begin{tabular} { c | p{5cm} }
				\bfseries Operandi & \bfseries Esempi \\
				\hline 
				Memoria, Registro Generale & \texttt{MOV 0x00002000, \%EDX} \\
				Registro Generale, Memoria & \texttt{MOV \%CL, 0x12AB1024} \\
				Registro Generale, Registro Generale & \texttt{MOV \%AX, \%DX} \\
				Immediato, Memoria & \texttt{MOVB \$0x5B, (\%EDI)} \\ 
				Immediato, Registro generale & \texttt{MOV \$0x54A3, \%AX}
			\end{tabular}
		\end{table}
\end{itemize}

\subsubsection{LOAD EFFECTIVE ADDRESS}
\begin{itemize}
	\item \textbf{Formato:} \texttt{LEA source, destination}
	\item \textbf{Azione:} sostituisce l'operando destinatario con l'espressione indirizzo contenuta nell'operando sorgente.
	\item \textbf{Flag:} nessuno.

		\begin{table}[h!]
			\center \rowcolors{2}{white}{black!10}
			\begin{tabular} { c | p{7cm} }
				\bfseries Operandi & \bfseries Esempi \\
				\hline 
				Memoria, Registro Generale a 32 bit & \texttt{LEA 0x00002000, \%EDX} \\
																						& \texttt{LEA 0x00213AB1 (\%EAX,\%EBX,4), \%ECX}
			\end{tabular}
		\end{table}
\end{itemize}

A differenza di MOV, LEA calcola l'indirizzo della locazione di memoria cercata come $ \text{base} + \text{index} \times \text{scala} \pm \text{displacement} $, e carica quell'indirizzo nella destinazione, non il valore contenuto in esso.
Nel primo esempio, questo equivale alla MOV con indirizzamento immediato.
In altri casi permette di ricavare esplicitamente il valore ottenuto dall'indirizzamento complesso.

\subsubsection{EXCHANGE}
\begin{itemize}
	\item \textbf{Formato:} \texttt{XCHG source, destination}
	\item \textbf{Azione:} sostituisce l'operando destinatario con l'operando sorgente e viceversa. Questa operazione è l'unica che modifica il sorgente.
	\item \textbf{Flag:} nessuno.

		\begin{table}[h!]
			\center \rowcolors{2}{white}{black!10}
			\begin{tabular} { c | p{7cm} }
				\bfseries Operandi & \bfseries Esempi \\
				\hline 
				Memoria, Registro Generale & \texttt{XCHG 0x00002000, \%DX} \\
				Registro Generale, Memoria & \texttt{XCHG \%AL, 0x000A2003} \\
				Registro Generale, Registro Generale & \texttt{XCHG \%EAX, \%EDX}
			\end{tabular}
		\end{table}

		Grazie a quest'istruzione in assembler si possono scambiare due operandi con una sola istruzione (\textbf{non trasparenza} dei registri) \textbf{atomica}.
		Questo è particolarmente utile nel caso di esecuzione concorrente.
\end{itemize}

\subsubsection{INPUT}
\begin{itemize}
	\item \textbf{Formato:}
		\begin{itemize}
			\item \texttt{IN indirizzo, \%AL} (8 bit)
			\item \texttt{IN indirizzo, \%AX} (16 bit)
			\item \texttt{IN (\%DX), \%AX} (8 bit) 
			\item \texttt{IN (\%DX), \%Al} (16 bit)
		\end{itemize}
	\item \textbf{Azione:} sostituisce il contenuto del registro destinatario (AL 8 bit, AX 16 bit) con il contenuto di un adeguato numero di porte consecutive.
		L'indirizzo è specificato direttamente (per porte con indirizzo $<256$), o indirettamente usando il registro DX.
	\item \textbf{Flag:} nessuno.
\end{itemize}

\subsubsection{OUTPUT}
\begin{itemize}
	\item \textbf{Formato:}
		\begin{itemize}
			\item \texttt{OUT \%AL, indirizzo} (8 bit)
			\item \texttt{IN \%AX, indirizzo} (16 bit)
			\item \texttt{IN \%AX}, (\%DX) (8 bit) 
			\item \texttt{IN \%Al, (\%DX)} (16 bit)
		\end{itemize}
	\item \textbf{Azione:} copia il contenuto del registro sorgente (AL 8 bit, AX 16 bit) su un adeguato numero di porte consecutive.
		L'indirizzo è specificato direttamente (per porte con indirizzo $<256$), o indirettamente usando il registro DX.
	\item \textbf{Flag:} nessuno.
\end{itemize}

\subsubsection{Non ortogonalità INPUT/OUTPUT}
Le uniche due operazioni che gestiscono l'input e l'output possono trasferire solo dai o nei registri AL e AX, e indirizzare indirettamente la memoria puntando col registro DX.
Questo rende le operazioni non ortogonali: non si possono usare altri registri, ed eventuali operazioni vanno fatte nel processore,

\subsection{Pila}
La pila, o \textbf{stack}, è una regione di memoria gestita con politica Last In First Out (LIFO), essenziale al funzionamento del calcolatore.
Permette di annidare sottoprogrammi, funzionalità per cui l'assembler è organizzato.

Generalmente, la pila viene usata come segue per eseguire i sottoprogrammi:
\begin{itemize}
	\item Prima di saltare al sottoprogramma, si fa \textbf{PUSH} sulla pila dell'indirizzo di ritorno (e.g. l'indirizzo della prossima istruzione);
	\item Si esegue il sottoprogramma;
	\item Al termine del sottoprogramma, si fa \textbf{POP} dalla pila del prossimo indirizzo.
\end{itemize}

Più sottoprogrammi possono chiamarsi a vicenda (annidarsi), ponendosi su livelli via via superiori della pila.
Al termine della sua esecuzione, ogni sottoprogramma tornerà all'indirizzo di ripresa del sottoprogramma precedente, finché tutti i sottoprogrammi non termineranno l'esecuzione.

Il registro \textbf{ESP} punta al top della pila, ergo non va usato per altri scopi.
Va però inizializzato prima che parta il programma.
Si deve inoltre notare che la pila in assembler si estende \textit{verso il basso}: aggiungere alla pila significa decrementare ESP, e rimuovere dalla pila significa incrementare ESP.
I frame successivi della pila si vanno a disporre via via sotto (o "a sinistra") del frame corrente.

Per lavorare sulla pila si usano le istruzioni:

\subsubsection{PUSH}
\begin{itemize}
	\item \textbf{Formato:} \texttt{PUSH source}
	\item \textbf{Azione:} decrementa ESP e copia il sorgente nell'indirizzo puntato da ESP.
		Il sorgente deve essere a 16 bit o a 32 bit.
		Nello specifico, compie le seguenti azioni:
		\begin{itemize}
			\item Decrementa l'indirizzo contenuto nel registro ESP di 2 o 4;
			\item Memorizza una copia dell'operando sorgente nella word o long il cui indirizzo è contenuto in ESP.
		\end{itemize}
	\item \textbf{Flag:} nessuno.
\end{itemize}

		\begin{table}[h!]
			\center \rowcolors{2}{white}{black!10}
			\begin{tabular} { c | p{5cm} }
				\bfseries Operandi & \bfseries Esempi \\
				\hline 
				Memoria & \texttt{PUSHW 0x3214200A} \\ 
				Immediato & \texttt{PUSHL \$0x4871A000} \\ 
				Registro Generale & \texttt{PUSH \%BX}
			\end{tabular}
		\end{table}

\subsubsection{POP}
\begin{itemize}
	\item \textbf{Formato:} \texttt{POP destination}
	\item \textbf{Azione:} copia una word o un long dall'indirzzo puntato dall'ESP nel destinatario e incrementa ESP.
		Nello specifico compie le seguenti azioni:
		\begin{itemize}
			\item Sostituisce all'operando destinatario una copia del contenuto nella word o long il cui indirizzo è contenuto in ESP;
			\item Incrementa di due o quattro l'indirizzo contenuto in ESP, rimuovendo la word o il long copiato.
		\end{itemize}
	\item \textbf{Flag:} nessuno.
\end{itemize}
	
		\begin{table}[h!]
			\center \rowcolors{2}{white}{black!10}
			\begin{tabular} { c | p{5cm} }
				\bfseries Operandi & \bfseries Esempi \\
				\hline 
				Memoria & \texttt{POPW 0x02AB2000} \\ 
				Registro Generale & \texttt{POP \%BX}
			\end{tabular}
		\end{table}

\par\medskip

\noindent
\textsf{\textbf{Dati temporanei nella pila}} \\
Solitamente la pila viene usata per memorizzare dati temporanei, visto che i registri sono pochi e spesso hanno scopi diversi in momenti diversi. Ad esempio:

\begin{lstlisting}[style=codestyle]	
# sto usando %EAX, mi serve un dato da una porta
PUSH %EAX
IN 0x001A, %AL
...
POP %EAX # ritorno da dove ero
\end{lstlisting}

\subsubsection{PUSHAD}
\begin{itemize}
	\item \textbf{Formato:} \texttt{PUSHAD}
	\item \textbf{Azione:}: salva nella pila corrente una copia degli 8 registri generali a 32 bit, nell'ordine: EAX, ECX, EDX, EBX, ESP, EBP, ESI, EDI.
	\item \textbf{Flag:} nessuno.
\end{itemize}

\subsubsection{POPAD}
\begin{itemize}
	\item \textbf{Formato:} \texttt{POPAD}
	\item \textbf{Azione:}: copia dalla pila corrente gli 8 registri generali a 32 bit, nell'ordine: EAX, ECX, EDX, EBX, ESP, EBP, ESI, EDI. 
	\item \textbf{Flag:} nessuno.
\end{itemize}

\subsection{Istruzioni aritmetiche}
Molte operazioni aritmetiche di base non distinguono numeri naturali e numeri interi, distinzione che viene fatta solo per moltiplicazioni e divisioni.

Le operazioni possono modificare i flag, e in questo caso i flag da controllare dipenderanno dal tipo di numeri su cui si è fatta l'operazione (informazione nota soltanto al programmatore).

Abbiamo quindi che un'operazione aritmetica si svolge seguendo i passi:
\begin{itemize}
	\item Si esegue l'operazione;
	\item Si controllano i flag interessati (OF, SF e ZF sugli interi, CF e ZF sui naturali) per verificarne l'esito.
\end{itemize}

Vediamo quindi le operazioni aritmetiche:

\subsubsection{ADD}
\begin{itemize}
	\item \textbf{Formato:} \texttt{ADD source, destination}
	\item \textbf{Azione:} modifica l'operando destinatario sommandovi l'operando sorgente.
		Il risultato è consistente sia che si interpretino i numeri come naturali, che come interi.
	\item \textbf{Flag:} attiva CF se, interpretando i numeri come naturali, si è verificato un riporto; attiva OF se, interpretando gli operandi come interi, si è verificato un traboccamento.
		Inoltre attiva opportunamente ZF e SF se il numero è rispettivamente zero o negativo (in complemento a 2).
\end{itemize}

		\begin{table}[h!]
			\center \rowcolors{2}{white}{black!10}
			\begin{tabular} { c | p{5cm} }
				\bfseries Operandi & \bfseries Esempi \\
				\hline
				Memoria, Registro Generale & \texttt{ADD 0x00002000, \%EDX} \\ 
				Registro Generale, Memoria & \texttt{ADD \%CL, 0x12AB1024} \\ 
				Registro Generale, Registro Generale & \texttt{ADD \%AX, \%DX} \\ 
				Immediato, Memoria & \texttt{ADDB \$0x5B, (\%EDI)} \\ 
				Immediato, Registro Generale & \texttt{ADD \$0x54A3, \%AX}
			\end{tabular}
		\end{table}

\par\medskip
\noindent
\textbf{\textsf{Funzionamento della ADD}} \\
Il passo elementare di una somma consiste nel sommare due addendi (propriamente due cifre degli addendi) e un riporto entrante per produrre:
	\begin{itemize}
		\item Una cifra;
		\item Un riporto uscente (cioè il riporto entrante per il prossimo passo).
	\end{itemize}
L'ultimo riporto, se non entra in memoria, attiva il carry flag (CF).

L'operazione di somma ha lo stesso effetto sia su naturali che su interi in complemento a 2: la differenza sta nel controllo dell'attivazione dei flag.
Il carry flag non ha infatti alcun significato nella somma fra interi: dobbiamo controllare l'OF.

In generale, si ha overflow (OF) quando il risultato esce dall'intervallo di rappresentabilità.
Si può capire se si è verificato un overflow controllando i segni degli operandi:
\begin{itemize}
	\item \textbf{Segni discordi:} non c'é overflow;
	\item \textbf{Segni concordi:} il risultato è concorde se è concorde con gli operandi.
\end{itemize}
La ADD imposta quindi OF secondo queste regole.
Il ZF viene poi impostato se il risultato è fatto da tutti zeri, e il SF viene impostato se il MSB è uno.

\subsubsection{INCREMENT}
\begin{itemize}
	\item \textbf{Formato:} \texttt{INC destination}
	\item \textbf{Azione:} equivale all'istruzione \texttt{ADD \$1, destination}. 
	\item \textbf{Flag:} modifica tutti i flag di ADD tranne CF (il riporto).
\end{itemize}

		\begin{table}[H]
			\center \rowcolors{2}{white}{black!10}
			\begin{tabular} { c | c }
				\bfseries Operandi & \bfseries Esempi \\
				\hline 
				Memoria & \texttt{INCB (\%ESI)} \\
				Registro Generale & \texttt{INC \%CX}
			\end{tabular}
		\end{table}

Quest'istruzione è più compatta di ADD, e storicamente era anche più veloce.
Questo deriva dal fatto che la circuiteria che implementava l'incremento era più efficiente di quella che implementa le somme.

\subsubsection{SUBTRACT}
\begin{itemize}
	\item \textbf{Formato:} \texttt{SUB source, destination}
	\item \textbf{Azione:} modifica l'operando destinatario sottraendovi l'operando sorgente. 
		Il risultato è consistente sia che si interpretino i numeri come naturali, che come interi.
	\item \textbf{Flag:} attiva CF se, interpretando i numeri come naturali, si è verificato un riporto; attiva OF se, interpretando gli operandi come interi, si è verificato un traboccamento.
\end{itemize}

		\begin{table}[h!]
			\center \rowcolors{2}{white}{black!10}
			\begin{tabular} { c | p{5cm} }
				\bfseries Operandi & \bfseries Esempi \\
				\hline
				Memoria, Registro Generale & \texttt{SUB 0x00002000, \%EDX} \\ 
				Registro Generale, Memoria & \texttt{SUB \%CL, 0x12AB1024} \\ 
				Registro Generale, Registro Generale & \texttt{SUB \%AX, \%DX} \\ 
				Immediato, Memoria & \texttt{SUBB \$0x5B, (\%EDI)} \\ 
				Immediato, Registro Generale & \texttt{SUB \$0x54A3, \%AX}
			\end{tabular}
		\end{table}

\par\medskip
\noindent
\textbf{\textsf{Funzionamento della SUBTRACT}} \\
Il passo elementare della sottrazione è effettivamente il contrario di quello della somma: si sottraggono il sottraendo e un prestito entrante al minuendo, producendo:
\begin{itemize}
	\item Una cifra;
	\item Un prestito uscente.
\end{itemize}

Il carry flag (CF) memorizza il prestito.
Se alla fine dell'operazione il CF è impostato, significa che il risultato è un numero intero.

Questo funziona anche sugli interi: in questo caso, come prima, non si controlla il CF, ma l'OF, che conterrà la seguente informazione:
\begin{itemize}
	\item La differenza di numeri concordi è sempre rappresentabile;
	\item La differenza di numeri discordi è rappresentabile solo se il risultato ha il segno del minuendo.
\end{itemize}

Il ZF e il SF vengono attivati secondo le regole già note.

\subsubsection{DECREMENT}
\begin{itemize}
	\item \textbf{Formato:} \texttt{DEC destination}
	\item \textbf{Azione:} equivale all'istruzione \texttt{SUB \$1, destination}. 
	\item \textbf{Flag:} modifica tutti i flag di SUBTRACT tranne CF (il prestito).
\end{itemize}

		\begin{table}[h!]
			\center \rowcolors{2}{white}{black!10}
			\begin{tabular} { c | c }
				\bfseries Operandi & \bfseries Esempi \\
				\hline 
				Memoria & \texttt{DECB (\%EDI)} \\
				Registro Generale & \texttt{DEC \%CX}
			\end{tabular}
		\end{table}

\subsubsection{ADD WITH CARRY}
\begin{itemize}
	\item \textbf{Formato:} \texttt{ADC source, destination}
	\item \textbf{Azione:} modifica l'operando destinatario sommandovi sia l'operando sorgente sia il contenuto del flag CF.
	\item \textbf{Flag:} modifica tutti i flag come ADD. 
\end{itemize}

\begin{table}[h!]
			\center \rowcolors{2}{white}{black!10}
			\begin{tabular} { c | p{5cm} }
				\bfseries Operandi & \bfseries Esempi \\
				\hline
				Memoria, Registro Generale & \texttt{ADC 0x00002000, \%EDX} \\ 
				Registro Generale, Memoria & \texttt{ADC \%CL, 0x12AB1024} \\ 
				Registro Generale, Registro Generale & \texttt{ADC \%AX, \%DX} \\ 
				Immediato, Memoria & \texttt{ADCB \$0x5B, (\%EDI)} \\ 
				Immediato, Registro Generale & \texttt{ADC \$0x54A3, \%AX}
			\end{tabular}
		\end{table}

Quest'istruzione è utile per effettuare somme di numeri più grandi di 32 bit.
In questo caso si:
\begin{itemize}
	\item Effettua la somma dei 32 bit meno significativi con ADD;
	\item Sommano i successivi 32 bit con ADC portandosi quindi dietro il carry.
\end{itemize}

\subsubsection{SUBTRACT WITH BORROW}
\begin{itemize}
	\item \textbf{Formato:} \texttt{SBB source, destination}
	\item \textbf{Azione:} modifica l'operando destinatario sottraendovi sia l'operando sorgente sia il contenuto del flag CF.
	\item \textbf{Flag:} modifica tutti i flag come SUBTRACT. 
\end{itemize}

		\begin{table}[H]
		\center \rowcolors{2}{white}{black!10}
			\begin{tabular} { c | p{5cm} }
				\bfseries Operandi & \bfseries Esempi \\
				\hline
				Memoria, Registro Generale & \texttt{SBB 0x00002000, \%EDX} \\ 
				Registro Generale, Memoria & \texttt{SBB \%CL, 0x12AB1024} \\ 
				Registro Generale, Registro Generale & \texttt{SBB \%AX, \%DX} \\ 
				Immediato, Memoria & \texttt{SBBB \$0x255B, (\%EDI)} \\ 
				Immediato, Registro Generale & \texttt{SBB \$0x54A3, \%AX}
			\end{tabular}
		\end{table}

Come ormai dovrebbe essere chiaro, è la duale dell'ADC, e si usa per effettuare sottrazioni di numeri più grandi di 32 bit.

\subsubsection{NEGATE}
\begin{itemize}
	\item \textbf{Formato:} \texttt{NEG destination}
	\item \textbf{Azione:} interpreta l'operando destinatario come un numero intero e lo sostituisce con il suo opposto in complemento a 2. 
	\item \textbf{Flag:} quando l'operazione non è possibile (l'intervallo di rappresentabilità degli interi in complemento a 2 non è simmetrico) imposta il flag OF.
		Imposta inoltre il flag CF quando l'operando è diverso da zero, e tutti gli altri flag in base a nullità e segno del risultato. 
\end{itemize}

		\begin{table}[H]
		\center \rowcolors{2}{white}{black!10}
			\begin{tabular} { c | p{5cm} }
				\bfseries Operandi & \bfseries Esempi \\
				\hline
				Memoria & \texttt{NEGB (\%EDI)} \\ 
				Registro Generale & \texttt{NEG \%CX}
			\end{tabular}
		\end{table}


\par\medskip
\noindent
\textbf{\textsf{Funzionamento della NEGATE}} \\
L'opposto di un numero $X$ in complemento a due è:
$$
-X = \bar{X} + 1
$$

Si ricordi che questo ha senso \textit{solamente} se il numero è rappresentato in complemento a due.

\subsubsection{COMPARE}
\begin{itemize}
	\item \textbf{Formato:} \texttt{CMP source, destination}
	\item \textbf{Azione:} verifica se l'operando destinatario è maggiore, uguale o minore dell'operando sorgente, sia interpretando gli operandi come naturali che come interi, e aggiorna i flag di conseguenza.
		Più propriamente, la compare si comporta come la SUB, ma senza sovrascrivere nessuno degli operandi.
	\item \textbf{Flag:} come la SUB. 
\end{itemize}


		\begin{table}[H]
		\center \rowcolors{2}{white}{black!10}
			\begin{tabular} { c | p{5cm} }
				\bfseries Operandi & \bfseries Esempi \\
				\hline
				Memoria, Registro Generale & \texttt{CMP 0x00002000, \%EDX} \\ 
				Registro Generale, Memoria & \texttt{CMP \%CL, 0x12AB1024} \\ 
				Registro Generale, Registro Generale & \texttt{CMP \%AX, \%DX} \\ 
				Immediato, Memoria & \texttt{CMPB \$0x255B, (\%EDI)} \\ 
				Immediato, Registro Generale & \texttt{CMP \$0x54A3, \%AX}
			\end{tabular}
		\end{table}

\subsubsection{Funzionamento della COMPARE}
Solitamente la CMP si usa nei salti condizionati come:
\begin{lstlisting}[style=codestyle]	
CMP %AX, %BX
JCOND # salto condizionato
\end{lstlisting}
\noindent
Ciò che fa la CMP è effettivamente creare un'oggetto temporaneo:
$$
\text{tmp} = \text{dest} - \text{source}
$$
che viene poi rimosso.

I flag restano però aggiornati, e questo valore può essere interpretato correttamente dalla JE per effettuare un salto condizionale.


\subsection{Moltiplicazioni}
Le moltiplicazioni, a differenza delle somme e delle differenze, sono diverse fra naturali ed interi.
Bisogna inoltre notare che le dimensioni il risultato della somma di un numero a $n$ cifre sta su $n$ o $n+1$ cifre, mentre il prodotto di due numeri a $n$ cifre sta su $2n$ cifre.
In altre parole, il numero di bit necessari a memorizzare il risultato non è più confrontabile con quello degli operatori.


\subsubsection{MULTIPLY}
\begin{itemize}
	\item \textbf{Formato:} \texttt{MUL source}
	\item \textbf{Azione:} considera l'operando sorgente come un moltiplicando, l'operando destinatario (implicito) come un moltiplicatore, e effettua la moltiplicazione assumendo i numeri naturali. Nello specifico:
	\begin{itemize}
		\item Sorgente a 8 bit, si ha $\text{AX} = \text{AL} \times \text{source}$;
		\item Sorgente a 16 bit, si ha $\text{DX}\_\text{AX} = \text{AX} \times \text{source}$;
		\item Sorgente a 32 bit, si ha $\text{EDX}\_\text{EAX} = \text{EAX} \times \text{source}$.
	\end{itemize}
	\item \textbf{Flag:} imposta CF e OF se il risultato non sta nel numero di bit di source. SF e ZF sono indefiniti.
\end{itemize}

		\begin{table}[H]
		\center \rowcolors{2}{white}{black!10}
			\begin{tabular} { c | p{5cm} }
				\bfseries Operandi & \bfseries Esempi \\
				\hline
				Memoria & \texttt{MULB (\%ESI)} \\ 
				Registro Generale & \texttt{MUL \%ECX}
			\end{tabular}
		\end{table}

\subsubsection{INTEGER MULTIPLY}
\begin{itemize}
	\item \textbf{Formato:} \texttt{MUL source}
	\item \textbf{Azione:} considera l'operando sorgente come un moltiplicando, l'operando destinatario (implicito) come un moltiplicatore, e effettua la moltiplicazione assumendo i numeri interi. Nello specifico:
	\begin{itemize}
		\item Sorgente a 8 bit, si ha $\text{AX} = \text{AL} \times \text{source}$;
		\item Sorgente a 16 bit, si ha $\text{DX}\_\text{AX} = \text{AX} \times \text{source}$;
		\item Sorgente a 32 bit, si ha $\text{EDX}\_\text{EAX} = \text{EAX} \times \text{source}$.
	\end{itemize}
	\item \textbf{Flag:} li imposta tutti, ma non è attendibile.
\end{itemize}

		\begin{table}[H]
		\center \rowcolors{2}{white}{black!10}
			\begin{tabular} { c | p{5cm} }
				\bfseries Operandi & \bfseries Esempi \\
				\hline
				Memoria & \texttt{IMULB (\%ESI)} \\ 
				Registro Generale & \texttt{IMUL \%ECX}
			\end{tabular}
		\end{table}

\par\medskip
\noindent
\textbf{\textsf{Funzionamento delle MULTIPLY e INTEGER MULTIPLY}} \\
Queste operazioni hanno sia un operando che il destinatario impliciti, in base al tipo dell'operando fornito.
Questo deriva dal fatto che il risultato di una moltiplicazione raramente sta nello stesso numero di bit dei fattori.
Di preciso, abbiamo visto i 3 tipi di moltiplicazione concessi:
\begin{itemize}
	\item Sorgente a 8 bit, si ha $\text{AX} = \text{AL} \times \text{source}$;
	\item Sorgente a 16 bit, si ha $\text{DX}\_\text{AX} = \text{AX} \times \text{source}$;
	\item Sorgente a 32 bit, si ha $\text{EDX}\_\text{EAX} = \text{EAX} \times \text{source}$.
\end{itemize}

La differenza fra le prime due operazioni e l'ultima, in particolare con sorgente a 16 bit, che usa una due registri da 16 bit separati, ha principalmente motivi storici (il registro EAX è stato introdotto dopo).

Si può rimettere il valore dai due registri a 16 bit in un registro a 32 bit attraverso la pila:
\begin{lstlisting}[style=codestyle]	
PUSH \%DX
PUSH \%AX
POP \%EAX
\end{lstlisting}

\end{document}
