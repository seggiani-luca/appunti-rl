
\documentclass[a4paper,11pt]{article}
\usepackage[a4paper, margin=8em]{geometry}

% usa i pacchetti per la scrittura in italiano
\usepackage[french,italian]{babel}
\usepackage[T1]{fontenc}
\usepackage[utf8]{inputenc}
\frenchspacing 

% usa i pacchetti per la formattazione matematica
\usepackage{amsmath, amssymb, amsthm, amsfonts}

% usa altri pacchetti
\usepackage{gensymb}
\usepackage{hyperref}
\usepackage{standalone}

\usepackage{colortbl}

\usepackage{xstring}
\usepackage{karnaugh-map}

% imposta il titolo
\title{Appunti Reti Logiche}
\author{Luca Seggiani}
\date{2024}

% imposta lo stile
% usa helvetica
\usepackage[scaled]{helvet}
% usa palatino
\usepackage{palatino}
% usa un font monospazio guardabile
\usepackage{lmodern}

\renewcommand{\rmdefault}{ppl}
\renewcommand{\sfdefault}{phv}
\renewcommand{\ttdefault}{lmtt}

% circuiti
\usepackage{circuitikz}
\usetikzlibrary{babel}

% disponi il titolo
\makeatletter
\renewcommand{\maketitle} {
	\begin{center} 
		\begin{minipage}[t]{.8\textwidth}
			\textsf{\huge\bfseries \@title} 
		\end{minipage}%
		\begin{minipage}[t]{.2\textwidth}
			\raggedleft \vspace{-1.65em}
			\textsf{\small \@author} \vfill
			\textsf{\small \@date}
		\end{minipage}
		\par
	\end{center}

	\thispagestyle{empty}
	\pagestyle{fancy}
}
\makeatother

% disponi teoremi
\usepackage{tcolorbox}
\newtcolorbox[auto counter, number within=section]{theorem}[2][]{%
	colback=blue!10, 
	colframe=blue!40!black, 
	sharp corners=northwest,
	fonttitle=\sffamily\bfseries, 
	title=Teorema~\thetcbcounter: #2, 
	#1
}

% disponi definizioni
\newtcolorbox[auto counter, number within=section]{definition}[2][]{%
	colback=red!10,
	colframe=red!40!black,
	sharp corners=northwest,
	fonttitle=\sffamily\bfseries,
	title=Definizione~\thetcbcounter: #2,
	#1
}

% disponi codice
\usepackage{listings}
\usepackage[table]{xcolor}

\definecolor{codegreen}{rgb}{0,0.6,0}
\definecolor{codegray}{rgb}{0.5,0.5,0.5}
\definecolor{codepurple}{rgb}{0.58,0,0.82}
\definecolor{backcolour}{rgb}{0.95,0.95,0.92}

\lstdefinestyle{codestyle}{
		backgroundcolor=\color{black!5}, 
		commentstyle=\color{codegreen},
		keywordstyle=\bfseries\color{magenta},
		numberstyle=\sffamily\tiny\color{black!60},
		stringstyle=\color{green!50!black},
		basicstyle=\ttfamily\footnotesize,
		breakatwhitespace=false,         
		breaklines=true,                 
		captionpos=b,                    
		keepspaces=true,                 
		numbers=left,                    
		numbersep=5pt,                  
		showspaces=false,                
		showstringspaces=false,
		showtabs=false,                  
		tabsize=2
}

\lstdefinestyle{shellstyle}{
		backgroundcolor=\color{black!5}, 
		basicstyle=\ttfamily\footnotesize\color{black}, 
		commentstyle=\color{black}, 
		keywordstyle=\color{black},
		numberstyle=\color{black!5},
		stringstyle=\color{black}, 
		showspaces=false,
		showstringspaces=false, 
		showtabs=false, 
		tabsize=2, 
		numbers=none, 
		breaklines=true
}


\lstdefinelanguage{assembler}{ 
  keywords={AAA, AAD, AAM, AAS, ADC, ADCB, ADCW, ADCL, ADD, ADDB, ADDW, ADDL, AND, ANDB, ANDW, ANDL,
        ARPL, BOUND, BSF, BSFL, BSFW, BSR, BSRL, BSRW, BSWAP, BT, BTC, BTCB, BTCW, BTCL, BTR, 
        BTRB, BTRW, BTRL, BTS, BTSB, BTSW, BTSL, CALL, CBW, CDQ, CLC, CLD, CLI, CLTS, CMC, CMP,
        CMPB, CMPW, CMPL, CMPS, CMPSB, CMPSD, CMPSW, CMPXCHG, CMPXCHGB, CMPXCHGW, CMPXCHGL,
        CMPXCHG8B, CPUID, CWDE, DAA, DAS, DEC, DECB, DECW, DECL, DIV, DIVB, DIVW, DIVL, ENTER,
        HLT, IDIV, IDIVB, IDIVW, IDIVL, IMUL, IMULB, IMULW, IMULL, IN, INB, INW, INL, INC, INCB,
        INCW, INCL, INS, INSB, INSD, INSW, INT, INT3, INTO, INVD, INVLPG, IRET, IRETD, JA, JAE,
        JB, JBE, JC, JCXZ, JE, JECXZ, JG, JGE, JL, JLE, JMP, JNA, JNAE, JNB, JNBE, JNC, JNE, JNG,
        JNGE, JNL, JNLE, JNO, JNP, JNS, JNZ, JO, JP, JPE, JPO, JS, JZ, LAHF, LAR, LCALL, LDS,
        LEA, LEAVE, LES, LFS, LGDT, LGS, LIDT, LMSW, LOCK, LODSB, LODSD, LODSW, LOOP, LOOPE,
        LOOPNE, LSL, LSS, LTR, MOV, MOVB, MOVW, MOVL, MOVSB, MOVSD, MOVSW, MOVSX, MOVSXB,
        MOVSXW, MOVSXL, MOVZX, MOVZXB, MOVZXW, MOVZXL, MUL, MULB, MULW, MULL, NEG, NEGB, NEGW,
        NEGL, NOP, NOT, NOTB, NOTW, NOTL, OR, ORB, ORW, ORL, OUT, OUTB, OUTW, OUTL, OUTSB, OUTSD,
        OUTSW, POP, POPL, POPW, POPB, POPA, POPAD, POPF, POPFD, PUSH, PUSHL, PUSHW, PUSHB, PUSHA, 
				PUSHAD, PUSHF, PUSHFD, RCL, RCLB, RCLW, MOVSL, MOVSB, MOVSW, STOSL, STOSB, STOSW, LODSB, LODSW,
				LODSL, INSB, INSW, INSL, OUTSB, OUTSL, OUTSW
        RCLL, RCR, RCRB, RCRW, RCRL, RDMSR, RDPMC, RDTSC, REP, REPE, REPNE, RET, ROL, ROLB, ROLW,
        ROLL, ROR, RORB, RORW, RORL, SAHF, SAL, SALB, SALW, SALL, SAR, SARB, SARW, SARL, SBB,
        SBBB, SBBW, SBBL, SCASB, SCASD, SCASW, SETA, SETAE, SETB, SETBE, SETC, SETE, SETG, SETGE,
        SETL, SETLE, SETNA, SETNAE, SETNB, SETNBE, SETNC, SETNE, SETNG, SETNGE, SETNL, SETNLE,
        SETNO, SETNP, SETNS, SETNZ, SETO, SETP, SETPE, SETPO, SETS, SETZ, SGDT, SHL, SHLB, SHLW,
        SHLL, SHLD, SHR, SHRB, SHRW, SHRL, SHRD, SIDT, SLDT, SMSW, STC, STD, STI, STOSB, STOSD,
        STOSW, STR, SUB, SUBB, SUBW, SUBL, TEST, TESTB, TESTW, TESTL, VERR, VERW, WAIT, WBINVD,
        XADD, XADDB, XADDW, XADDL, XCHG, XCHGB, XCHGW, XCHGL, XLAT, XLATB, XOR, XORB, XORW, XORL},
  keywordstyle=\color{blue}\bfseries,
  ndkeywordstyle=\color{darkgray}\bfseries,
  identifierstyle=\color{black},
  sensitive=false,
  comment=[l]{\#},
  morecomment=[s]{/*}{*/},
  commentstyle=\color{purple}\ttfamily,
  stringstyle=\color{red}\ttfamily,
  morestring=[b]',
  morestring=[b]"
}

\lstset{language=assembler, style=codestyle}

% disponi sezioni
\usepackage{titlesec}

\titleformat{\section}
	{\sffamily\Large\bfseries} 
	{\thesection}{1em}{} 
\titleformat{\subsection}
	{\sffamily\large\bfseries}   
	{\thesubsection}{1em}{} 
\titleformat{\subsubsection}
	{\sffamily\normalsize\bfseries} 
	{\thesubsubsection}{1em}{}

% tikz
\usepackage{tikz}

% float
\usepackage{float}

% grafici
\usepackage{pgfplots}
\pgfplotsset{width=10cm,compat=1.9}

% disponi alberi
\usepackage{forest}

\forestset{
	rectstyle/.style={
		for tree={rectangle,draw,font=\large\sffamily}
	},
	roundstyle/.style={
		for tree={circle,draw,font=\large}
	}
}

% disponi algoritmi
\usepackage{algorithm}
\usepackage{algorithmic}
\makeatletter
\renewcommand{\ALG@name}{Algoritmo}
\makeatother

% disponi numeri di pagina
\usepackage{fancyhdr}
\fancyhf{} 
\fancyfoot[L]{\sffamily{\thepage}}

\makeatletter
\fancyhead[L]{\raisebox{1ex}[0pt][0pt]{\sffamily{\@title \ \@date}}} 
\fancyhead[R]{\raisebox{1ex}[0pt][0pt]{\sffamily{\@author}}}
\makeatother

\begin{document}
% sezione (data)
\section{Lezione del 07-11-24}

% stili pagina
\thispagestyle{empty}
\pagestyle{fancy}

% testo
\subsection{Reti sequenziali sincronizzate}
Le reti sequenziali sincronizzate (RSS), a differenza della asincrone (RSA), non si aggiornano per la sola variazione degli ingressi, ma per l'arrivo di un determinato segnale periodico, che chiamiamo \textbf{clock}.

Il clock è un segnale con forma d'onda periodica, di frequenza $1 \over T$ periodo, e \textit{duty cicle} (ciclo di lavoro) $\tau \over T$ intorno al 50\%.
Solitamente l'evento di sincronia delle reti sequenziali sincronizzate è il \textbf{fronte di salita} del clock.

\subsection{Registri}
Un registro a $W$ bit è una collezione di $W$ D flip-flop positive edge-triggered, che hanno:
\begin{itemize}
	\item $W$ ingressi $d_i$ e $W$ uscite $q_i$ separate (in verità ricordiamo che troviamo sempre $q$ e $\overline{q}$ negata, noi riporteremo solo la prima per semplicità);
	\item Un ingresso $p$ in parallelo a tutti gli ingressi $p_i$ dei singoli D flip-flop.
\end{itemize}

Si ha che $p$ funge da \textbf{segnale di sincronizzazione} (effettivamente il nostro \textit{clock}).
Consideriamo quindi le variabili di entrata e di uscita di un registro come due singole variabili a più bit, $d_{W-1}\_d_0$ e $q_{W-1}\_q_0$.

\subsubsection{Pilotaggio di registri}
Per il corretto pilotaggio di un registro gli ingressi $d_i$ devono essere stabili intorno al fronte di salita del clock, per un tempo $T_{setup}$ prima e $T_{hold}$ dopo il fronte stesso.
L'uscita cambia dopo, come avevamo visto per i D flip-flop, un tempo $T_{prop} > T_{hold}$.

Tutto cio che accade in ingresso fra due istanti di sincronizzazione è irrilevante e non viene memorizzato.

\par\smallskip

Il registro \textit{memorizza} lo stato di ingresso al \textbf{fronte di salita}.
Gli stati di ingresso fra due fronti di salita adiacenti possono essere identici, adiacenti o non adiacenti: è irrilevante in quanto, come abbiamo detto, l'aggiornamento accade soltanto nelle condizioni di stabilità intorno al fronte di salita del clock.

Dopo il fronte di salita, le uscite cambiano il loro valore dopo $T_{prop}$.

\par\medskip

Possiamo quindi aggiornare la nostra definizione di RSS come \textit{collezione di registri e reti combinatorie}, montati arbitrariamente, purchè non ci siano anelli di retroazione di reti combinatorie (costituirebbero reti sequenziali asincrone).
I registri hanno tutti lo stesso clock in comune, e possono formare anelli, in quanto abbiamo visto dal loro pilotaggio, questo non genera problemi.

\subsubsection{Regole di pilotaggio per RSS}
Dato l'$i$-esimo fronte di salita del clock al tempo $t_i$, lo stato di ingresso ai registri dovrà essere stabile, dalle loro regole di pilotaggio, nell'intervallo $[t_i - T_{setup}, t_i + T_{hold}]$.
Non potrò quindi scegliere periodi $T$ del clock piccoli a piacere: dovrò lasciare tempo ai registri di produrre nuovi valori (in tempo $T_{prop}$), e alle reti combinatorie di elaborare tali valori coi loro tempi di ritardo interni,e quindi di propagarsi nuovamente fino ai registri.

Definiamo, nello specifico, i ritardi:
\begin{itemize}
	\item $T_{in\_to\_reg}$: il tempo di attraversamento massimo della catena di sole reti combinatorie che da uno degli ingressi della rete all'ingresso di un registro;
	\item $T_{reg\_to\_reg}$: il tempo di attravarsamento massimo fra l'uscita e l'ingresso di un registro;
	\item $T_{in\_to\_out}$: il tempo di attraversamento massimo fra un ingresso e un uscita dell'intera rete;
	\item $T_{reg\_to\_out}$: il tempo di attraversamento massimo fra l'uscita di un registro e un uscita della rete.
\end{itemize}

Dobbiamo introdurre poi i tempi $T_{a\_monte}$ e $T_{a\_valle}$, cioè i tempi necessari all'utente della rete per, rispettivamente, \textbf{modificare} gli ingressi e \textbf{leggere} le uscite.
Questi formano due ulteriori vincoli di pilotaggio in ingresso e in uscita.

Queste variabili di temporizzazione daranno vita ad un sistema di 4 diseguaglianze. Vediamole nel dettaglio:
\begin{itemize}
	\item $T \geq T_{hold} + T_{a\_monte} + T_{in\_to\_reg} + T_{setup}$ \\ 
		Questa diseguaglianza assicura che un registro abbia tempo $T_{hold}$ di immagazinare il valore dello scorso ciclo, l'utente esterno abbia tempo $T_{a\_monte}$ di modificare l'ingresso della rete, e che questo ingresso abbia tempo di arrivare ai registri $T_{in\_to\_reg}$ prima del tempo di setup $T_{setup}$ degli stessi, che sappiamo essere necessario perchè al clock seguente i registri memorizzino effettivamente il valore (dopo $T_{hold}$, e lo replichino dopo $T_{prop}$);
	\item $T \geq T_{prop} + T_{reg\_to\_reg} + T_{setup}$ \\ 
		Questa diseguaglianza assicura che il valore generato dai registri possa propagarsi dopo $T_{prop}$, arrivare ai registri stessi in tempo $T_{reg\_to\_reg}$ per il $T_{setup}$ necessario perche lo memorizzino.
		In sostanza, è come la precedente ma riferita alle uscite dei registri anziché dell'utente;
	\item $T \geq T_{hold} + T_{a\_monte} + T_{in\_to\_out} + T_{a\_valle}$ \\
		Questa diseguaglianza assicura che la rete abbia tempo di aggiornarsi dopo un ingresso dell'utente ($T_{a\_monte}$), e restituire il risultato per un tempo che basti all'utente per leggere l'uscita ($T_{a\_valle}$).
		Nello specifico, sappiamo che l'utente non proverà a modificare gli ingressi della rete prima del $T_{hold}$ necessario ad aggiornare i registri al positive edge del clock, e quindi impiegherà un tempo $T_{a\_monte}$ per farlo.
		A questo punto, il segnale di uscità dovra viaggiare almeno dall'ingresso all'uscita, quindi si dovrà aspettare un tempo $T_{in\_to\_out}$, e infine resterà il tempo $T_{a\_valle}$ perchè l'utente abbia modo di effettuare la lettura.
		Notiamo che questa legge si rende necessaria in quanto un aggiornamento degli ingressi può comportare un aggiornamento delle uscite \textit{prima} che i registri ne rispondano.
		In altre parole, reti di questo tipo non sono automaticamente \textbf{trasparenti};
	\item $T \geq T_{prop} + T_{reg\_to\_out} + T_{a\_valle}$ \\ 
		Quest'ultima diseguaglianza assicura che la rete abbia tempo di aggiornare le sue uscite, e quindi farle leggere all'utente ($T_{a\_valle}$), a memorizzazione effettuata dei registri.
		Nello specifico, i registri otterrano il valore al ciclo corrente nel tempo compreso fra $T_{setup}$ e $T_{hold}$ centrato sul positive edge dello scorso clock, e quindi si adegueranno dopo un tempo $T_{prop}$ rispetto al positive edge stesso.
		Di qui in poi dovremo aspettare un tempo $T_{reg\_to\_out}$ perchè questo valore attraversi la rete fino alle uscite, e infine il tempo $T_{a\_valle}$ perchè l'utente abbia modo di effettuare la lettura.
		Questa legge si rende necessaria, al contrario della precedente, sia per reti \textit{trasparenti} che per reti \textbf{non trasparenti}, e anzi vedremo che reti non trasparenti saranno proprio i registri a fornire le uscite. 
\end{itemize}

Possiamo quindi porre il sistema completo: 
\[
	\begin{cases}
		T \geq T_{hold} + T_{a\_monte} + T_{in\_to\_reg} + T_{setup} \\ 
		T \geq T_{prop} + T_{reg\_to\_reg} + T_{setup} \\ 
		T \geq T_{hold} + T_{a\_monte} + T_{in\_to\_out} + T_{a\_valle} \\
		T \geq T_{prop} + T_{reg\_to\_out} + T_{a\_valle}
	\end{cases}
\]

Dove, riassumendo, le prime due condizioni garantiscono che lo stato delle variabili di ingresso resti stabile negli intervalli $(-T_{setup}, T_{hold})$ centrati sui positive edge di ogni clock; la prima e la terza tengono conto del mondo esterno \textit{a monte}, quindi in fase di scrittura; la seconda e la quarta tengono conto del mondo esterno \textit{a valle}, quindi in fase di lettura. 

In verità, avremo altri due ritardi di cui tenere conto:
\begin{itemize}
	\item $T_{sfas}$: il \textbf{massimo sfasamento} fra due clock.
		Visto che questo viene portato a elementi diversi, a qualche registro arrivera prima e a qualche registro arrivera dopo;
	\item $T_{reg}$: se un registro è formato da $W$ > 1 bit, questi non cambieranno tutti contemporaneamente: dovremmo aggiungere $T_{prop} + T_{reg} = T'_{prop}$. A questo punto, però, possiamo considerare solo $T_{prop} \leftarrow T'_{prop}$ e ignorare $T_{reg}$.
\end{itemize}

\subsubsection{Anticipazioni sui modelli di Moore e di Mealy ritardato}
Potremmo voler determinare qual'è la più vincolante fra le diseguaglianze riportate prima.
Questa, chiaramente, è quella che copre il percorso più lungo, cioè la terza.
Se decidiamo di vietare il percorso che copre, cioè quello diretto fra ingressi e uscite, otteniamo il cosiddetto \textbf{modello di Moore}: cioè, un modello di RSS dove non si ammettono reti combinatorie che collegano gli ingressi direttamente alle uscite.

Un'altro vincolo che potremmo voler rilassare è il quarto, nel cosiddetto \textbf{modello di Mealy ritardato}.
Questo equivale a prelevare le uscite direttamente dalle uscite dei registri, cioè a eliminare il tempo $T_{reg\_to\_out}$.

\subsection{Contatori}
Un contatore è una RSS il cui stato di uscita può essere visto come un \textbf{numero naturale} ad $n$ cifre in base $\beta$.
Ad ogni clock, il contatore \textbf{incrementa} o \textbf{decrementa}.

Abbiamo che si può realizzare un contatore collegando un modulo sommatore a $n$ cifre a un registro a $n$ cifre.
L'uscita del registro viene collegata in anello di retrazione a uno degli ingressi del sommatore.
Impostando il $C_{in}$ del sommatore a 1, e il suo secondo ingresso ad un'array di $n$ generatori di costante 0, si ha un contatore \textbf{incrementatore}, cioè che incrementa il suo valore ad ogni ciclo di clock.
	L'equivalente \textbf{decrementatore} si può creare usando un sottrattore a $n$ cifre invece di un sommatore.veri

Si può creare un contatore con ingresso di abilitazione (sostanzialmente una \textbf{variabile di controllo}), cioè che incrementa o decrementa solo se è alto un certo bit di controllo, collegando tale bit al carry (o al borrow) del sommatore (sottrattore).

\subsection{Contatore a una cifra in base 2}
Vediamo quindi come realizzare un contatore a una cifra in base $\beta = 2$.
Se l'intenzione è di creare un contatore per $N$ cifre in codifica binaria, questo rappresenterà l'elemento fondamentale (che andremo a combinare nei prossimi paragrafi, attraverso catene di \textbf{ripple carry}).

Avremo bisogno di un input, $e_i$ oltre al clock e al reset, che rappresenterà il riporto entrante dell'incrementatore (che come abbiamo visto può fungere da variabile di controllo, in quanto lasciare $e_i$ a 0 significa sommare 0 a un numero, quindi lasciarlo invariato),
Prenderemo poi due uscite: $q$, cioè l'uscita vera e propria dal registro del contatore (che chiameremo \lstinline|OUTR|), e $e_u$, il riporto uscente.
Nel caso di un contatore a una cifra in base 2, il riporto uscente si riduce al valore dell'AND fra $q$ ed $e_i$, ergo se $q$ è alto e introduciamo un riporto entrante $e_i$, andremo al di fuori della rappresentazione possibile su un bit e dovremo passare il riporto (che possiamo anche qui intendere come segnale di controllo) al prossimo contatore della catena.
L'uscita dell'incrementatore (chiamiamola $a$) andrà messa in \lstinline|OUTR| ad ogni aggiornamento, e verrà aggiornata dai valori di $q$, cioè l'uscita stessa di \lstinline|OUTR| (ciclo di retroazione) e $e_i$, attraverso la logica dell'incrementatore che riportiamo sotto forma di tabella di verità:

\begin{table}[H]
	\center 
	\begin{tabular} { c c | c c }
		$q$ & $e_i$ & $a$ & $e_u$ \\ 
		\hline
		0 & 0 & 0 & 0 \\ 
		0 & 1 & 1 & 0 \\ 
		1 & 0 & 1 & 0 \\ 
		1 & 1 & 0 & 1 \\
	\end{tabular}
\end{table}

Notiamo che non abbiamo mai l'uscita $1, 1$ su $a, e_u$, in quanto uno degli ingressi del sommatore che forma l'incrementatore è "fissato" a 0.
Possiamo quindi dare la sintesi in Verilog:

\lstinputlisting[language=verilog, style=codestyle]{../verilog/11-07/up_counters/b2_up_counter.v}
sfruttando la definizione già data di incrementatore (cioè un \textit{half adder}).

\subsubsection{Contatore a una cifra in base 3}
Con procedimenti simili a quelli dell'esempio precedente si può ricavare un contatore a una cifra in base $\beta = 3$.
In questo caso la tabella di verità dell'incrementatore sarà:
\begin{table}[H]
	\center 
	\begin{tabular} { c c c | c c c }
		$q_1$ & $q_0$ & $e_i$ & $a_1$ & $a_0$ & $e_u$ \\ 
		\hline
		0 & 0 & 0 & 0 & 0 & 0 \\ 
		0 & 0 & 1 & 0 & 1 & 0 \\ 
		0 & 1 & 0 & 0 & 1 & 0 \\ 
		0 & 1 & 1 & 1 & 0 & 0 \\
		1 & 0 & 0 & 1 & 0 & 0 \\ 
		1 & 0 & 1 & 0 & 0 & 1 \\ 
		1 & 1 & 0 & - & - & - \\ 
		1 & 1 & 1 & - & - & -
	\end{tabular}
\end{table}

e la sintesi sarà:
\lstinputlisting[language=verilog, style=codestyle]{../verilog/11-07/up_counters/b3_up_counter.v}
sfruttando la definizione già data di incrementatore in base 3.

\subsubsection{Contatore a una cifra in base 10}
Infine, vediamo come si può ricavare un contatore a una cifra in base $\beta = 10$.
In questo caso la tabella di verità dell'incrementatore sarà:

\begin{table}[H]
	\center 
	\begin{tabular} { c c  c  c  c | c  c  c  c  c }
		$x_3$ & $x_2$ & $x_1$ & $x_0$ & $e_i$ & $z_3$ & $z_2$ & $z_1$ & $z_0$ & $e_u$ \\ 
		\hline
		0 & 0 & 0 & 0 & 0 & 0 & 0 & 0 & 0 & 0 \\
		0 & 0 & 0 & 0 & 1 & 0 & 0 & 0 & 1 & 0 \\
		0 & 0 & 0 & 1 & 0 & 0 & 0 & 0 & 1 & 0 \\
		0 & 0 & 0 & 1 & 1 & 0 & 0 & 1 & 0 & 0 \\
		0 & 0 & 1 & 0 & 0 & 0 & 0 & 1 & 0 & 0 \\
		0 & 0 & 1 & 0 & 1 & 0 & 0 & 1 & 1 & 0 \\
		0 & 0 & 1 & 1 & 0 & 0 & 0 & 1 & 1 & 0 \\
		0 & 0 & 1 & 1 & 1 & 0 & 1 & 0 & 0 & 0 \\
		0 & 1 & 0 & 0 & 0 & 0 & 1 & 0 & 0 & 0 \\
		0 & 1 & 0 & 0 & 1 & 0 & 1 & 0 & 1 & 0 \\
		0 & 1 & 0 & 1 & 0 & 0 & 1 & 0 & 1 & 0 \\
		0 & 1 & 0 & 1 & 1 & 0 & 1 & 1 & 0 & 0 \\
		0 & 1 & 1 & 0 & 0 & 0 & 1 & 1 & 0 & 0 \\
		0 & 1 & 1 & 0 & 1 & 0 & 1 & 1 & 1 & 0 \\
		0 & 1 & 1 & 1 & 0 & 0 & 1 & 1 & 1 & 0 \\
		0 & 1 & 1 & 1 & 1 & 1 & 0 & 0 & 0 & 0 \\
		1 & 0 & 0 & 0 & 0 & 1 & 0 & 0 & 0 & 0 \\
		1 & 0 & 0 & 0 & 1 & 1 & 0 & 0 & 1 & 0 \\
		1 & 0 & 0 & 1 & 0 & 1 & 0 & 0 & 1 & 0 \\
		1 & 0 & 0 & 1 & 1 & 0 & 0 & 0 & 0 & 1 \\
		1 & 0 & 1 & 0 & 0 & - & - & - & - & - \\
		ecc...
	\end{tabular}
\end{table}

e la sintesi sarà:
\lstinputlisting[language=verilog, style=codestyle]{../verilog/11-07/up_counters/b10_up_counter.v}
sfruttando la definizione già data di incrementatore in base 10.

\subsubsection{Scomposizione in moduli di contatori}
Un contatore può essere scomposto, in qualsiasi base, in una serie di contatori ad una cifra collegati a \textbf{catena di riporti} (\textit{ripple carry}).
In questo caso il registro è dato dalla combinazione di $n$ registri, uno per ogni cifra (e quindi per ogni contatore), tutti sincronizzati sullo stesso clock.
All'attivarsi del riporto di uno dei contatori, il successivo nella catena si attiva, e così via in un processo del tutto identico al propagarsi delle somme nei sommatori a più cifre.
Si riportano le descrizioni in Verilog di sommatori a quattro cifre, in base $\beta = 2$:

\lstinputlisting[language=verilog, style=codestyle]{../verilog/11-07/up_counters/n4_b2_up_counter.v}

in base $\beta = 3$:

\lstinputlisting[language=verilog, style=codestyle]{../verilog/11-07/up_counters/n4_b3_up_counter.v}

e infine in base $\beta = 10$:

\lstinputlisting[language=verilog, style=codestyle]{../verilog/11-07/up_counters/n4_b10_up_counter.v}

\end{document}
