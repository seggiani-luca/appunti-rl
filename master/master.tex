
\documentclass[a4paper,11pt]{article}
\usepackage[a4paper, margin=8em]{geometry}

% usa i pacchetti per la scrittura in italiano
\usepackage[french,italian]{babel}
\usepackage[T1]{fontenc}
\usepackage[utf8]{inputenc}
\frenchspacing 

% usa i pacchetti per la formattazione matematica
\usepackage{amsmath, amssymb, amsthm, amsfonts}

% usa altri pacchetti
\usepackage{gensymb}
\usepackage{hyperref}
\usepackage{standalone}

% imposta il titolo
\title{Appunti Reti Logiche}
\author{Luca Seggiani}
\date{2024}

% imposta lo stile
% usa helvetica
\usepackage[scaled]{helvet}
% usa palatino
\usepackage{palatino}
% usa un font monospazio guardabile
\usepackage{lmodern}

\renewcommand{\rmdefault}{ppl}
\renewcommand{\sfdefault}{phv}
\renewcommand{\ttdefault}{lmtt}

% disponi il titolo
\makeatletter
\renewcommand{\maketitle} {
	\begin{center} 
		\begin{minipage}[t]{.8\textwidth}
			\textsf{\huge\bfseries \@title} 
		\end{minipage}%
		\begin{minipage}[t]{.2\textwidth}
			\raggedleft \vspace{-1.65em}
			\textsf{\small \@author} \vfill
			\textsf{\small \@date}
		\end{minipage}
		\par
	\end{center}

	\thispagestyle{empty}
	\pagestyle{fancy}
}
\makeatother

% disponi teoremi
\usepackage{tcolorbox}
\newtcolorbox[auto counter, number within=section]{theorem}[2][]{%
	colback=blue!10, 
	colframe=blue!40!black, 
	sharp corners=northwest,
	fonttitle=\sffamily\bfseries, 
	title=Teorema~\thetcbcounter: #2, 
	#1
}

% disponi definizioni
\newtcolorbox[auto counter, number within=section]{definition}[2][]{%
	colback=red!10,
	colframe=red!40!black,
	sharp corners=northwest,
	fonttitle=\sffamily\bfseries,
	title=Definizione~\thetcbcounter: #2,
	#1
}

% disponi codice
\usepackage{listings}
\usepackage[table]{xcolor}

\definecolor{codegreen}{rgb}{0,0.6,0}
\definecolor{codegray}{rgb}{0.5,0.5,0.5}
\definecolor{codepurple}{rgb}{0.58,0,0.82}
\definecolor{backcolour}{rgb}{0.95,0.95,0.92}

\lstdefinestyle{codestyle}{
		backgroundcolor=\color{black!5}, 
		commentstyle=\color{codegreen},
		keywordstyle=\bfseries\color{magenta},
		numberstyle=\sffamily\tiny\color{black!60},
		stringstyle=\color{green!50!black},
		basicstyle=\ttfamily\footnotesize,
		breakatwhitespace=false,         
		breaklines=true,                 
		captionpos=b,                    
		keepspaces=true,                 
		numbers=left,                    
		numbersep=5pt,                  
		showspaces=false,                
		showstringspaces=false,
		showtabs=false,                  
		tabsize=2
}

\lstdefinestyle{shellstyle}{
		backgroundcolor=\color{black!5}, 
		basicstyle=\ttfamily\footnotesize\color{black}, 
		commentstyle=\color{black}, 
		keywordstyle=\color{black},
		numberstyle=\color{black!5},
		stringstyle=\color{black}, 
		showspaces=false,
		showstringspaces=false, 
		showtabs=false, 
		tabsize=2, 
		numbers=none, 
		breaklines=true
}


\lstdefinelanguage{assembler}{
  keywords={AAA, AAD, AAM, AAS, ADC, ADCB, ADCW, ADCL, ADD, ADDB, ADDW, ADDL, AND, ANDB, ANDW, ANDL,
        ARPL, BOUND, BSF, BSFL, BSFW, BSR, BSRL, BSRW, BSWAP, BT, BTC, BTCB, BTCW, BTCL, BTR, 
        BTRB, BTRW, BTRL, BTS, BTSB, BTSW, BTSL, CALL, CBW, CDQ, CLC, CLD, CLI, CLTS, CMC, CMP,
        CMPB, CMPW, CMPL, CMPS, CMPSB, CMPSD, CMPSW, CMPXCHG, CMPXCHGB, CMPXCHGW, CMPXCHGL,
        CMPXCHG8B, CPUID, CWDE, DAA, DAS, DEC, DECB, DECW, DECL, DIV, DIVB, DIVW, DIVL, ENTER,
        HLT, IDIV, IDIVB, IDIVW, IDIVL, IMUL, IMULB, IMULW, IMULL, IN, INB, INW, INL, INC, INCB,
        INCW, INCL, INS, INSB, INSD, INSW, INT, INT3, INTO, INVD, INVLPG, IRET, IRETD, JA, JAE,
        JB, JBE, JC, JCXZ, JE, JECXZ, JG, JGE, JL, JLE, JMP, JNA, JNAE, JNB, JNBE, JNC, JNE, JNG,
        JNGE, JNL, JNLE, JNO, JNP, JNS, JNZ, JO, JP, JPE, JPO, JS, JZ, LAHF, LAR, LCALL, LDS,
        LEA, LEAVE, LES, LFS, LGDT, LGS, LIDT, LMSW, LOCK, LODSB, LODSD, LODSW, LOOP, LOOPE,
        LOOPNE, LSL, LSS, LTR, MOV, MOVB, MOVW, MOVL, MOVSB, MOVSD, MOVSW, MOVSX, MOVSXB,
        MOVSXW, MOVSXL, MOVZX, MOVZXB, MOVZXW, MOVZXL, MUL, MULB, MULW, MULL, NEG, NEGB, NEGW,
        NEGL, NOP, NOT, NOTB, NOTW, NOTL, OR, ORB, ORW, ORL, OUT, OUTB, OUTW, OUTL, OUTSB, OUTSD,
        OUTSW, POP, POPL, POPW, POPB, POPA, POPAD, POPF, POPFD, PUSH, PUSHL, PUSHW, PUSHB, PUSHA, 
				PUSHAD, PUSHF, PUSHFD, RCL, RCLB, RCLW,
        RCLL, RCR, RCRB, RCRW, RCRL, RDMSR, RDPMC, RDTSC, REP, REPE, REPNE, RET, ROL, ROLB, ROLW,
        ROLL, ROR, RORB, RORW, RORL, SAHF, SAL, SALB, SALW, SALL, SAR, SARB, SARW, SARL, SBB,
        SBBB, SBBW, SBBL, SCASB, SCASD, SCASW, SETA, SETAE, SETB, SETBE, SETC, SETE, SETG, SETGE,
        SETL, SETLE, SETNA, SETNAE, SETNB, SETNBE, SETNC, SETNE, SETNG, SETNGE, SETNL, SETNLE,
        SETNO, SETNP, SETNS, SETNZ, SETO, SETP, SETPE, SETPO, SETS, SETZ, SGDT, SHL, SHLB, SHLW,
        SHLL, SHLD, SHR, SHRB, SHRW, SHRL, SHRD, SIDT, SLDT, SMSW, STC, STD, STI, STOSB, STOSD,
        STOSW, STR, SUB, SUBB, SUBW, SUBL, TEST, TESTB, TESTW, TESTL, VERR, VERW, WAIT, WBINVD,
        XADD, XADDB, XADDW, XADDL, XCHG, XCHGB, XCHGW, XCHGL, XLAT, XLATB, XOR, XORB, XORW, XORL},
  keywordstyle=\color{blue}\bfseries,
  ndkeywordstyle=\color{darkgray}\bfseries,
  identifierstyle=\color{black},
  sensitive=false,
  comment=[l]{\#},
  morecomment=[s]{/*}{*/},
  commentstyle=\color{purple}\ttfamily,
  stringstyle=\color{red}\ttfamily,
  morestring=[b]',
  morestring=[b]"
}

\lstset{language=assembler, style=codestyle}

% disponi sezioni
\usepackage{titlesec}

\titleformat{\section}
	{\sffamily\Large\bfseries} 
	{\thesection}{1em}{} 
\titleformat{\subsection}
	{\sffamily\large\bfseries}   
	{\thesubsection}{1em}{} 
\titleformat{\subsubsection}
	{\sffamily\normalsize\bfseries} 
	{\thesubsubsection}{1em}{}

% tikz
\usepackage{tikz}

% float
\usepackage{float}

% grafici
\usepackage{pgfplots}
\pgfplotsset{width=10cm,compat=1.9}

% disponi alberi
\usepackage{forest}

\forestset{
	rectstyle/.style={
		for tree={rectangle,draw,font=\large\sffamily}
	},
	roundstyle/.style={
		for tree={circle,draw,font=\large}
	}
}

% disponi algoritmi
\usepackage{algorithm}
\usepackage{algorithmic}
\makeatletter
\renewcommand{\ALG@name}{Algoritmo}
\makeatother

% disponi numeri di pagina
\usepackage{fancyhdr}
\fancyhf{} 
\fancyfoot[L]{\sffamily{\thepage}}

\makeatletter
\fancyhead[L]{\raisebox{1ex}[0pt][0pt]{\sffamily{\@title \ \@date}}} 
\fancyhead[R]{\raisebox{1ex}[0pt][0pt]{\sffamily{\@author}}}
\makeatother

\begin{document}
\maketitle
\documentclass[a4paper,11pt]{article}
\usepackage[a4paper, margin=8em]{geometry}

% usa i pacchetti per la scrittura in italiano
\usepackage[french,italian]{babel}
\usepackage[T1]{fontenc}
\usepackage[utf8]{inputenc}
\frenchspacing 

% usa i pacchetti per la formattazione matematica
\usepackage{amsmath, amssymb, amsthm, amsfonts}

% usa altri pacchetti
\usepackage{gensymb}
\usepackage{hyperref}
\usepackage{standalone}

% imposta il titolo
\title{Appunti Reti Logiche}
\author{Luca Seggiani}
\date{24-09-24}

% imposta lo stile
% usa helvetica
\usepackage[scaled]{helvet}
% usa palatino
\usepackage{palatino}
% usa un font monospazio guardabile
\usepackage{lmodern}

\renewcommand{\rmdefault}{ppl}
\renewcommand{\sfdefault}{phv}
\renewcommand{\ttdefault}{lmtt}

% disponi teoremi
\usepackage{tcolorbox}
\newtcolorbox[auto counter, number within=section]{theorem}[2][]{%
	colback=blue!10, 
	colframe=blue!40!black, 
	sharp corners=northwest,
	fonttitle=\sffamily\bfseries, 
	title=Teorema~\thetcbcounter: #2, 
	#1
}

% disponi definizioni
\newtcolorbox[auto counter, number within=section]{definition}[2][]{%
	colback=red!10,
	colframe=red!40!black,
	sharp corners=northwest,
	fonttitle=\sffamily\bfseries,
	title=Definizione~\thetcbcounter: #2,
	#1
}

% disponi codice
\usepackage{listings}
\usepackage[table]{xcolor}

\lstdefinestyle{codestyle}{
		backgroundcolor=\color{black!5}, 
		commentstyle=\color{codegreen},
		keywordstyle=\bfseries\color{magenta},
		numberstyle=\sffamily\tiny\color{black!60},
		stringstyle=\color{green!50!black},
		basicstyle=\ttfamily\footnotesize,
		breakatwhitespace=false,         
		breaklines=true,                 
		captionpos=b,                    
		keepspaces=true,                 
		numbers=left,                    
		numbersep=5pt,                  
		showspaces=false,                
		showstringspaces=false,
		showtabs=false,                  
		tabsize=2
}

\lstdefinestyle{shellstyle}{
		backgroundcolor=\color{black!5}, 
		basicstyle=\ttfamily\footnotesize\color{black}, 
		commentstyle=\color{black}, 
		keywordstyle=\color{black},
		numberstyle=\color{black!5},
		stringstyle=\color{black}, 
		showspaces=false,
		showstringspaces=false, 
		showtabs=false, 
		tabsize=2, 
		numbers=none, 
		breaklines=true
}

\lstdefinelanguage{javascript}{
	keywords={typeof, new, true, false, catch, function, return, null, catch, switch, var, if, in, while, do, else, case, break},
	keywordstyle=\color{blue}\bfseries,
	ndkeywords={class, export, boolean, throw, implements, import, this},
	ndkeywordstyle=\color{darkgray}\bfseries,
	identifierstyle=\color{black},
	sensitive=false,
	comment=[l]{//},
	morecomment=[s]{/*}{*/},
	commentstyle=\color{purple}\ttfamily,
	stringstyle=\color{red}\ttfamily,
	morestring=[b]',
	morestring=[b]"
}

% disponi sezioni
\usepackage{titlesec}

\titleformat{\section}
	{\sffamily\Large\bfseries} 
	{\thesection}{1em}{} 
\titleformat{\subsection}
	{\sffamily\large\bfseries}   
	{\thesubsection}{1em}{} 
\titleformat{\subsubsection}
	{\sffamily\normalsize\bfseries} 
	{\thesubsubsection}{1em}{}

% disponi alberi
\usepackage{forest}

\forestset{
	rectstyle/.style={
		for tree={rectangle,draw,font=\large\sffamily}
	},
	roundstyle/.style={
		for tree={circle,draw,font=\large}
	}
}

% disponi algoritmi
\usepackage{algorithm}
\usepackage{algorithmic}
\makeatletter
\renewcommand{\ALG@name}{Algoritmo}
\makeatother

% disponi numeri di pagina
\usepackage{fancyhdr}
\fancyhf{} 
\fancyfoot[L]{\sffamily{\thepage}}

\makeatletter
\fancyhead[L]{\raisebox{1ex}[0pt][0pt]{\sffamily{\@title \ \@date}}} 
\fancyhead[R]{\raisebox{1ex}[0pt][0pt]{\sffamily{\@author}}}
\makeatother

\begin{document}
% sezione (data)
\section{Lezione del 24-09-24}

% stili pagina
\thispagestyle{empty}
\pagestyle{fancy}

% testo
\subsection{Introduzione}
Il corso di reti logiche tratta di:
\begin{enumerate}
	\item \textbf{Linguaggio assembler:} come scrivere programmi semplici, come avviene la compilazione in linguaggio macchina;
	\item \textbf{Reti logiche:} reti combinatorie, reti combinatorie per l'aritmetica, reti sequenziali asincrone e sincronizzate;
	\item \textbf{Microprogammazione:} reti sequenziali sincronizzate, come realizzare una rete logica da specifiche. 
		"Micro" qui sta per \textit{hardware};
	\item \textbf{Il calcolatore:} processore, interfacce comuni e convertitori.
\end{enumerate}

\subsubsection{Introduzione alle reti logiche}
Si parla di reti \textit{logiche} in quanto si guarda all'hardware da una prospettiva funzionale, indipendente dalla sua tecnologia.
Ad esempio, una porta NOR sarà implementata con determinati circuiti, ma tutto ciò che interessa a questo corso è come si comporta logicamente: $ y = 1 \Leftrightarrow A = B = 0 $.

\subsection{Programmazione assembly}
Il nome corretto del linguaggio sarebbe Assembly, ma noi lo chiameremo Assembler per ragioni storiche.
L'assembler è il linguaggio con cui si scrivono le istruzioni eseguite dal processore.
Il processore implementa effettivamente un ciclo fetch-execute dove preleva la prossima istruzione macchina (in assembler) dalla memoria e la esegue.

\subsubsection{Linguaggio macchina}
Il linguaggio macchina (LM) è dato dal contenuto effettivo della memoria che contiene le istruzioni, ergo una sequenza di zero e uno.
Il linguaggio assembler adotta una sintassi simbolica per il linguaggio macchina: ad esempio, \texttt{MOV \%AX, \%BX}.

Il processo di trasformazione dall'assembler all'LM si chiama \textbf{assemblaggio}, mentre il processo di traasformazione da un linguaggio ad alto livello all'assembler si chiama \textbf{compilazione}.

\subsubsection{Generalità sull'assembler}
Si dice che assembler è un linguaggio a basso livello.
Mancano i costrutti a cui siamo abituati da i linguaggi di alto livello:
\begin{enumerate}
	\item Non esistono costrutti di flow control (for, if-else, ecc...), tutto si fa con istruzioni di salto.
	\item Non esistono tipi variabile: gli operandi sono stringhe di bit che si riferiscono a locazioni in memoria.
\end{enumerate}

Inoltre, l'assembler è strettamente legato all'hardware, ed è specifico per ogni processore.
Noi vedremo l'assembler dei processori della famiglia Intel x86, che non è uguale all'assembler dei processori Arm Cortex, ecc...
Questo rende il codice in assembler mai portatile.
Fatta questa precisazione, possiamo dire che i principi generali restano comunque validi fra famiglie di processori diverse.

Esiste ancora oggi una nicchia di utilizzo del linguaggio assembler: quello dello sviluppo di sistemi embedded.
Inoltre, il linguaggio ha un importante significato didattico e culturale.

\subsection{Schema a blocchi del calcolatore}

# illustrazione modello funzionale

Un calcolatore è formato, in linea generale, da una rete di interconnessione (bus) che collega fra di loro:
\begin{itemize}
	\item Interfacce che comunicano con dispositivi;
	\item La memoria principale che contiene dati e programmi;
	\item Il processore, che esegue il ciclo fetch-execute. Possiamo aggiungere che ogni processore, oggi, contiene almeno due blocchi:
		\begin{itemize}
			\item L'\textbf{ALU}, Arithmetic Logic Unit, che si occupa di calcoli aritmetici su numeri interi (interpretando le stringhe di bit come numeri naturali o interi in complemento a 2) e operazioni logiche;
			\item L'\textbf{FPU}, Floating Point Unit, che si occupa dei numeri a virgola mobile.
		\end{itemize}
\end{itemize}

\subsection{Riassunto di rappresentazione dell'informazione}
\subsubsection{Numeri naturali}
$N$ bit rappresentano $2^N$ naturali sull'intervallo $[0, 2^N - 1]$, ovvero:

$$
b_{N-1}, b_{N-2}, ... , b_1, b_0 \Leftrightarrow X = \sum_{i=0}^{N-1} b_i \cdot 2^i
$$

Il bit più a sinistra è il Most Significant Bit (MSB) (nell'esempio $b_{N-1}$), quello più a destra il Least Significant Bit (LSD) (nell'esempio $b_0$).
Le cifre in base due a partire da un numero in un'altra base si trovano con l'algoritmo div-mod.

\subsubsection{Numeri interi in complemento a due}
$N$ bit rappresentano $2^N$ interi sull'intervallo $ [-2^{N-1}, 2^{N-1} - 1]$, ovvero:

$$
X = 
	\begin{cases}
		x \quad \quad \quad \ \ x \geq 0 \\
		2^N + x \quad x < 0
	\end{cases}
$$

oppure, usando l'operatore modulo:
$$ |x|_2N $$ # poco chiaro, copiaci fondamenti

La legge inversa, che mi permette di trovare l'intero $x$ dalla sua rappresentazione $X$, è:

$$
x =
	\begin{cases}
		X \quad \quad \quad \quad \  X_{N-1} = 0 \\
		-(\bar{X} + 1) \quad X_{N-1} = 1
	\end{cases}
$$

dove la barra rappresenta l'operazione complemento.

\subsubsection{Notazione esadecimale}
Scrivere lunghe stringhe binarie diventa velocemente complicato. 
Per questo si adotta una notazione esadecimale per stringhe di 4 bit ($[0, 15]$):

# riporta tabella stringhe esadecimali

A questo punto, possiamo denotare qualsiasi stringa binaria come una lista di numeri esadecimali prefissi da \texttt{0x} (che serve ad indicare la rappresentazione esadecimale stessa), ad esempio \texttt{0xC1}.

\subsection{Struttura del calcolatore}
\subsubsection{Spazio di memoria}
La memoria del calcolatore, vista dal programmatore assembler, è uno spazio lineare di $2^{32}$ (su calcolatori a 32 bit) locazioni (celle) di memoria, dalla capacità di un btye ciascuna.
Ogni cella è quindi identificata da un numero di 32 bit, detto \textbf{indirizzo}.

\par\smallskip

Lo spazio di memoria è in larga parte implementato attraverso Random Access Memory (RAM), ovvero memoria volatile.
Solo una piccola parte dello spazio è implementata attraverso Read Only Memory (ROM), ovvero memoria permanente, che contiene le istruzioni da eseguire al reset.

\subsubsection{Accesso allo spazio di memoria}
Il processore può accedere (leggere/scrivere) a:
\begin{itemize}
	\item Singole locazioni (byte) da 8 bit;
	\item Doppie locazioni (word) da 16 bit;
	\item Quadruple locazioni (double word) da 32 bit.
\end{itemize}

Per gli accessi 16/32 bit si usa l'indirizzo più piccolo delle 2/4 locazioni.
Si ricorda che l'indirizzo più grande contiene i bit più significativi.

Gli indirizzi di memoria assembler sono solo simbolici, e vengono tradotti dall'assemblatore, e in parte runtime.
Questo significa che non si può accedere a memoria appartenente al sistema operativo, o memoria fuori dai limiti fisici del sistema, ecc...

\subsubsection{Spazio di Input/Output}
Lo spazio di Input/Output è formato da $2^{16}$, ovvero 64k, locazioni o \textbf{porte}.
Ogni porta ha una capacità di un byte ed è indirizzata da un numero di 16 bit.

Il processore accede alle porte attraverso operazioni particolari di lettura o scrittura (in o out).
Spesso le porte sono configurate per un solo tipo di operazione: sola lettura o sola scrittura.

\par\smallskip

Le locazioni di memoria sono solitamente identifiche fra di loro, le porte di I/O no.
Indirizzi diversi significano dispositivi diversi, e si rende quindi necessario conoscere fisicamente gli indirizzi.

\subsubsection{Processore}
Il processore è dotato di una memoria interna formata da locazioni di memoria da 32 bit (\textbf{registri}).
Questi si dividono in registri \textbf{generali}, riservati alle elaborazioni, e \textbf{di stato}, riservati a compiti speciali.

# sii piu chiaro

I 16 bit bassi dei registri sono riferibili autonomamente (retro-compatibili).
DI alcuni registri si possono riferire parti ad 8 bit.

\subsubsection{Registri generali}
Alcuni registri vengono utilizzati per particolari funzioni, per motivi storici.
\begin{itemize}
	\item EAX (AX, AH od AL) è utilizzato da alcune istruzioni aritmetiche per contenere operandi e risultati. Viene detto \textbf{accumulatore}.
	\item ESI, EDI, EBX e EBP sono a volte utilizzati come registri puntatore, base (B) e indice (I).
		\begin{itemize}
			\item ESI
			\item EDI
			\item EBX veniva usato come indirizzo di base per l'accesso in memoria. Viene solitamente detto \textbf{base}.
			\item EBP # guarda slide!
		\end{itemize}
	\item ECX è utilizzato come contatore nei cicli. Vienedetto \textbf{contatore}.
	\item EDX è utilizzato come operando di operazioni aritmetiche. Viene detto \textbf{data}.
	\item ESP è utilizzato per indirizzare la \textbf{pila} o \textbf{stack}, ovvero una parte di memoria con disciplina LIFO che serve a gestire sottoprogrammi.
\end{itemize}

\subsubsection{Registri di stato}
L'EIP viene detto instruction pointer, o \textbf{program counter}.
Viene usato per contenere l'indirizzo della locazione dalla quale sarà prelevata la prossima istruzione da eseguire.
Il contenuto dell'EIP è fissato al reset iniziale, e impostato sulla prima istruzione da eseguire (in memoria ROM).

Possiamo quindi dire che il ciclo fetch-loop si svolge come segue:
\begin{itemize}
	\item Il processore preleva dalla memoria, all'indirizzo EIP, una nuova istruzione;
	\item Incrementa EIP del numero di byte dell'istruzione prelevata;
	\item Esegue l'istruzione e ripete.
\end{itemize}

Da questo si ha che le istruzioni in memoria vengono eseguite sequenzialmente nell'ordine in cui sono incontrate, a meno che non si definiscano salti attraverso altre determinate istruzioni.

\end{document}



\documentclass[a4paper,11pt]{article}
\usepackage[a4paper, margin=8em]{geometry}

% usa i pacchetti per la scrittura in italiano
\usepackage[french,italian]{babel}
\usepackage[T1]{fontenc}
\usepackage[utf8]{inputenc}
\frenchspacing 

% usa i pacchetti per la formattazione matematica
\usepackage{amsmath, amssymb, amsthm, amsfonts}

% usa altri pacchetti
\usepackage{gensymb}
\usepackage{hyperref}
\usepackage{standalone}

% imposta il titolo
\title{Appunti Reti Logiche}
\author{Luca Seggiani}
\date{25-09-24}

% imposta lo stile
% usa helvetica
\usepackage[scaled]{helvet}
% usa palatino
\usepackage{palatino}
% usa un font monospazio guardabile
\usepackage{lmodern}

\renewcommand{\rmdefault}{ppl}
\renewcommand{\sfdefault}{phv}
\renewcommand{\ttdefault}{lmtt}

% disponi teoremi
\usepackage{tcolorbox}
\newtcolorbox[auto counter, number within=section]{theorem}[2][]{%
	colback=blue!10, 
	colframe=blue!40!black, 
	sharp corners=northwest,
	fonttitle=\sffamily\bfseries, 
	title=Teorema~\thetcbcounter: #2, 
	#1
}

% disponi definizioni
\newtcolorbox[auto counter, number within=section]{definition}[2][]{%
	colback=red!10,
	colframe=red!40!black,
	sharp corners=northwest,
	fonttitle=\sffamily\bfseries,
	title=Definizione~\thetcbcounter: #2,
	#1
}

% disponi codice
\usepackage{listings}
\usepackage[table]{xcolor}

\lstdefinestyle{codestyle}{
		backgroundcolor=\color{black!5}, 
		commentstyle=\color{codegreen},
		keywordstyle=\bfseries\color{magenta},
		numberstyle=\sffamily\tiny\color{black!60},
		stringstyle=\color{green!50!black},
		basicstyle=\ttfamily\footnotesize,
		breakatwhitespace=false,         
		breaklines=true,                 
		captionpos=b,                    
		keepspaces=true,                 
		numbers=left,                    
		numbersep=5pt,                  
		showspaces=false,                
		showstringspaces=false,
		showtabs=false,                  
		tabsize=2
}

\lstdefinestyle{shellstyle}{
		backgroundcolor=\color{black!5}, 
		basicstyle=\ttfamily\footnotesize\color{black}, 
		commentstyle=\color{black}, 
		keywordstyle=\color{black},
		numberstyle=\color{black!5},
		stringstyle=\color{black}, 
		showspaces=false,
		showstringspaces=false, 
		showtabs=false, 
		tabsize=2, 
		numbers=none, 
		breaklines=true
}

\lstdefinelanguage{javascript}{
	keywords={typeof, new, true, false, catch, function, return, null, catch, switch, var, if, in, while, do, else, case, break},
	keywordstyle=\color{blue}\bfseries,
	ndkeywords={class, export, boolean, throw, implements, import, this},
	ndkeywordstyle=\color{darkgray}\bfseries,
	identifierstyle=\color{black},
	sensitive=false,
	comment=[l]{//},
	morecomment=[s]{/*}{*/},
	commentstyle=\color{purple}\ttfamily,
	stringstyle=\color{red}\ttfamily,
	morestring=[b]',
	morestring=[b]"
}

% disponi sezioni
\usepackage{titlesec}

\titleformat{\section}
	{\sffamily\Large\bfseries} 
	{\thesection}{1em}{} 
\titleformat{\subsection}
	{\sffamily\large\bfseries}   
	{\thesubsection}{1em}{} 
\titleformat{\subsubsection}
	{\sffamily\normalsize\bfseries} 
	{\thesubsubsection}{1em}{}

% disponi alberi
\usepackage{forest}

\forestset{
	rectstyle/.style={
		for tree={rectangle,draw,font=\large\sffamily}
	},
	roundstyle/.style={
		for tree={circle,draw,font=\large}
	}
}

% disponi algoritmi
\usepackage{algorithm}
\usepackage{algorithmic}
\makeatletter
\renewcommand{\ALG@name}{Algoritmo}
\makeatother

% disponi numeri di pagina
\usepackage{fancyhdr}
\fancyhf{} 
\fancyfoot[L]{\sffamily{\thepage}}

\makeatletter
\fancyhead[L]{\raisebox{1ex}[0pt][0pt]{\sffamily{\@title \ \@date}}} 
\fancyhead[R]{\raisebox{1ex}[0pt][0pt]{\sffamily{\@author}}}
\makeatother

\begin{document}
% sezione (data)
\section{Lezione del 25-09-24}

% stili pagina
\thispagestyle{empty}
\pagestyle{fancy}

% testo
\subsection{Introduzione all'Assembler}

\subsubsection{Codifica macchina e codifica mnemonica}
Possiamo adottare 2 metodi per codificare le istruzioni eseguite dal processore:

\begin{itemize}
	\item \textbf{Codifica macchina:} la serie di zeri e di uni che codificano, nel linguaggio del processore, le operazioni che esegue.
		Il formato macchina è, nell'architettura che ci interessa, il seguente:

		\begin{table}[h!]
			\center \rowcolors{2}{white}{black!10}
			\begin{tabular} { c | c | c }
				\bfseries Segmento & \bfseries Byte & \bfseries Funzione \\
				\hline 
				I Prefix (Instruction Prefix) & 0/1 byte & Usato per modificare l'istruzione \\ 
				O Prefix (Operand-size prefix) & 0/1 byte & Usato per modificare la dimensione degli operandi \\
				Opcode &
			\end{tabular}
		\end{table}

	\item \textbf{Codifica mnemonica:} un modo \textbf{simbolico} per riferirsi alle istruzioni.
		Un'istruzione può quindi essere semplicemente: \texttt{MOV \%EAX, 0x01F4E39}.
\end{itemize}

Il linguaggio assembler usa la codifica mnemonica delle istruzioni, e dispone di sovrastrutture sintattiche che lo rendono più comprensibile agli umani.
Ad esempio, permette l'uso di nomi simbolici per locazioni di memoria: \texttt{MOV \%EAX, pippo}.

\subsubsection{Istruzioni in codifica mnemonica}
Un'istruzione ha 3 campi:
\begin{itemize}
	\item \textbf{Codice operativo:} stabilisce quale operazione eseguire;
	\item \textbf{Suffisso di lunghezza:} stabilisce la lunghezza (che può variare) degli operandi;	
	\item \textbf{Operandi:} gli operandi su cui si applica l'operazione. 
		Possono essere contenuti in registri, in celle di memoria, nelle porte I/O o direttamente nell'istruzione (\textbf{costanti}).
\end{itemize}

Il suffisso di lunghezza può essere omesso quando è chiaro (essenzialmente quando si usa un registro).

Sintatticamente la struttura è \texttt{OPCODEsuffix source, dest}, che diventa qualcosa come \texttt{ADD \%BX, pluto}.
Questa istruzione effettua l'operazione \texttt{ADD} (aggiungi), aggiungendo al registro \texttt{BX} ciò che è contenuto nel simbolo \texttt{pluto}.

\par\medskip
\noindent
\textsf{\textbf{Operandi di istruzioni}} \\
Le istruzioni ammettono 0, 1 o 2 operandi.
Quando sono 2, il primo operando si chiama \textbf{sorgente} e il secondo \textbf{destinatario}, e solitamente hanno la stessa lunghezza.
Quando è 1, l'operando può essere sia sorgente che destinatario a seconda dell'istruzione.

\subsubsection{Primo esempio di programma}

Si presenta un programma per contare il numero di uno trovati dalla locazione \texttt{0x00000100} a \texttt{0x0000010i3}e scriverlo nella locazione \texttt{0x00000104}. 

\begin{lstlisting}[style=codestyle]	
MOVB $0x00, %CL					% sposta $0x00 in %CL
MOVL 0x00000100, %EAX		% sposta 32 bit da 0x00000100 a %EAX
CMPL $0x00000000, %EAX	% confronta 32 bit di 0 con il registro %EAX
JE   %EIP+$0x07					% salta se uguale a %EIP+$0x07, 
												%	ergo 0x0000020C + 0x07 = 0x00000213
SHRL %EAX								% trasla a destra %EAX
ADCB $0x00, %CL					% aggiungi a %CL 0 + carry 
JMP  %EIP-$0x0C					% salta incondizionato a %EIP-$0x0C,
												% ergo 0x00000213 - 0x0C = 0x00000207
MOVB %CL, 0x00000104		% sposta byte da %CL a 0x00000104
\end{lstlisting}

Il programma svolge i seguenti passi:
\begin{algorithm}
\caption{Conta 0}
\begin{algorithmic}
	\STATE Inizializza il registro CL (Counter Low) a 0
	\STATE Sposta i 32 bit da \texttt{0x00000000} a \texttt{0x00000103} in EAX
	\WHILE{true}	
		\IF{EAX è vuoto (tutti zeri)}
			\STATE Salta all'ultima istruzione
		\ENDIF
		\STATE Sposta EAX a destra
		\STATE Aggiungi il flag carry (che prende il valore rimosso da EAX) al registro CL
	\ENDWHILE
	\STATE Sposta il byte in CL nella locazione \texttt{0x00000104}
\end{algorithmic}
\end{algorithm}

\subsubsection{Istruzioni assembler}
Le istruzioni assembler si dividono in:
\begin{itemize}
	\item \textbf{Operative:} ovvero quelle che svolgono qualche operazione (ADD, SHR, MOV, CMP, ....);
	\item \textbf{Di controllo}: cioè che si occupano di altreare il flusso del programma (JMP, JE, ecc...).
\end{itemize}

\par\medskip
\noindent
\textsf{\textbf{Indirizzamento delle istruzioni operative}} \\
Le istruzioni operative si indirizzano attraverso l'\textbf{OPCODE} (codice operazione, ADD, MOV, ecc...), seguito da un suffisso (\textbf{B}, \textit{byte} da 8 bit, \textbf{W}, \textit{word} da 16 bit o \textbf{L}, \textit{long} da 32 bit) che può essere omesso, e gli indirizzi sorgente e destinazione.

\begin{itemize}
	\item 
Si possono \textbf{indirizzare i registri} sia come sorgenti che come destinatari, ovvero gli 8 registri generali da 32 bit, gli 8 registri generali da 16 bit, e gli 8 registri generali da 8 bit (disponibili solo sui registri A, B, C e D).
Bisogna precedere i nomi dei registri con \textbf{\%}.
	\item
Si può avere \textbf{indirizzamento immediato}, ovvero di costanti preceduti da \textbf{\$}, solo sull'operando sogente.
	\item
		Si può \textbf{indirizzare la memoria}, ma solo da sorgente o solo da destinatario, specificando un'indirizzo di memoria da 32 bit.
Ergo non posso scrivere:

\begin{lstlisting}[style=codestyle]	
MOVB pippo, pluto
\end{lstlisting}

ma devo scrivere:

\begin{lstlisting}[style=codestyle]	
MOV pippo, %EAX	% qua il suffisso di lunghezza e' implicito
MOVL %EAX, pluto
\end{lstlisting}

L'indirizzamento della memoria, nel caso più generale, è dato da: 

$$ \text{indirizzo} = \text{base} + \text{indice} \times \text{scala} \pm \text{displacement} $$

dove base e indice sono due registri generali da 32 bit, scala una costante dal valore 1 (default), 2, 4, 8, e displacement una costante intera.

La sintassi è \texttt{OPCODEsfx $\pm$disp(base,indice,scala)}.

Si distingue poi l'indirizzamento di tipo:

\begin{itemize}
	\item 
		\textbf{Diretto}, dove si indica soltanto il displacement, che coincide con l'indirizzo. \texttt{OPCODEW 0x00002001} significa prendi la word a partire da \texttt{0x00002001}.
	\item
		\textbf{Indiretto}, o con registro puntatore, dove si sfrutta un registro: \texttt{OPCODEL (\%EBX)} significa indirizzare il valore indirizzato da EBX. Si può specificare una scala: \texttt{OPCODEL (,\%EBX,4)} significa il valore nel registro EBX moltiplicato per 4.
		Si noti come a essere moltiplicato è l'indice e non la base.
	\item
		\textbf{Displacement e registro di modifica}, ad esempio da \texttt{OPCODEW 0x002A3A2B (\%EDI)} si ottiene l'operando a 16 bit ottenuto sommando al displacement \texttt{0x002A3A2B} il contenuto di EDI, modulo $2^32$.
	\item \textbf{Bimodificato senza displacement}, ad esempio \texttt{OPCODEW (\%EBX, \%EDI)}, che dipende sia da EBX che da EDI. Si può anche includere una scala: \texttt{OPCODEW (\%EBX, \%EDI, 8)}.
\item \textbf{Bimodificato con displacement}, come prima ma con displacement: \texttt{OPCODEB 0x002F9000 (\%EBX, \%EDI)}, ovvero l'indirizzo dato da base in EBX + indice in EDI + l'offset modulo $2^32$. Si può avere anche negativo: \texttt{OPCODEB -0x9000 (\%EBX, \%EDI)}, dove si sottrae l'offset invece di sommarlo.
\end{itemize}

Notare che senza il \$ i numeri in formato esadecimale sono interpretati automaticamente come indirizzi.

\item
Si possono \textbf{indirizzare le porte I/O}, come prima in uno solo dei due operandi. 
Questo si fa con le istruzioni specifiche IN e OUT.
In particolare si ha indirizzamento di tipo:

\begin{itemize}
	\item \textbf{Diretto}, solo per indirizzi $ < 256 $, in quanto nel formato macchina ci sono 8 bit.
		Ad esempio \texttt{IN 0x001A, \%AL} o \texttt{OUT \%AL, 0x003A}.
	\item \textbf{Indiretto con registro puntatore}, usando come registro puntatore soltanto DX.
		Ad esempio \texttt{IN (\%DX), \%AX} o \texttt{OUT \%AL, (\%DX)}.
\end{itemize}

\end{itemize}

\subsection{Panoramica sulle istruzioni}
Abbiamo diviso le istruzioni in \textbf{operative} e \textbf{di controllo}.
Possiamo fare ulteriori suddivisioni:

\begin{itemize}
	\item \textbf{Operative:}
		\begin{itemize}
			\item Di trasferimento;
			\item Aritmetiche;
			\item Di traslazione/rotazione:
			\item Logiche.
		\end{itemize}
	\item \textbf{Di controllo:}
		\begin{itemize}
			\item Di salto;
			\item Di gestione di sottoprogrammi.
		\end{itemize}
\end{itemize}

Conviene definire formato, funzionamento, comportamento sui flag e modalità di indirizzamento ammesse per gli operandi di ogni operazione, in quanto l'assembler non è \textbf{ortogonale}, ergo ci sono particolari restrizioni su \textit{quali} operandi e modalità di indirizzamento possono essere combinate.

\end{document}


\documentclass[a4paper,11pt]{article}
\usepackage[a4paper, margin=8em]{geometry}

% usa i pacchetti per la scrittura in italiano
\usepackage[french,italian]{babel}
\usepackage[T1]{fontenc}
\usepackage[utf8]{inputenc}
\frenchspacing 

% usa i pacchetti per la formattazione matematica
\usepackage{amsmath, amssymb, amsthm, amsfonts}

% usa altri pacchetti
\usepackage{gensymb}
\usepackage{hyperref}
\usepackage{standalone}

% imposta il titolo
\title{Appunti Reti Logiche}
\author{Luca Seggiani}
\date{2024}

% imposta lo stile
% usa helvetica
\usepackage[scaled]{helvet}
% usa palatino
\usepackage{palatino}
% usa un font monospazio guardabile
\usepackage{lmodern}

\renewcommand{\rmdefault}{ppl}
\renewcommand{\sfdefault}{phv}
\renewcommand{\ttdefault}{lmtt}

% disponi il titolo
\makeatletter
\renewcommand{\maketitle} {
	\begin{center} 
		\begin{minipage}[t]{.8\textwidth}
			\textsf{\huge\bfseries \@title} 
		\end{minipage}%
		\begin{minipage}[t]{.2\textwidth}
			\raggedleft \vspace{-1.65em}
			\textsf{\small \@author} \vfill
			\textsf{\small \@date}
		\end{minipage}
		\par
	\end{center}

	\thispagestyle{empty}
	\pagestyle{fancy}
}
\makeatother

% disponi teoremi
\usepackage{tcolorbox}
\newtcolorbox[auto counter, number within=section]{theorem}[2][]{%
	colback=blue!10, 
	colframe=blue!40!black, 
	sharp corners=northwest,
	fonttitle=\sffamily\bfseries, 
	title=Teorema~\thetcbcounter: #2, 
	#1
}

% disponi definizioni
\newtcolorbox[auto counter, number within=section]{definition}[2][]{%
	colback=red!10,
	colframe=red!40!black,
	sharp corners=northwest,
	fonttitle=\sffamily\bfseries,
	title=Definizione~\thetcbcounter: #2,
	#1
}

% disponi codice
\usepackage{listings}
\usepackage[table]{xcolor}

\lstdefinestyle{codestyle}{
		backgroundcolor=\color{black!5}, 
		commentstyle=\color{codegreen},
		keywordstyle=\bfseries\color{magenta},
		numberstyle=\sffamily\tiny\color{black!60},
		stringstyle=\color{green!50!black},
		basicstyle=\ttfamily\footnotesize,
		breakatwhitespace=false,         
		breaklines=true,                 
		captionpos=b,                    
		keepspaces=true,                 
		numbers=left,                    
		numbersep=5pt,                  
		showspaces=false,                
		showstringspaces=false,
		showtabs=false,                  
		tabsize=2
}

\lstdefinestyle{shellstyle}{
		backgroundcolor=\color{black!5}, 
		basicstyle=\ttfamily\footnotesize\color{black}, 
		commentstyle=\color{black}, 
		keywordstyle=\color{black},
		numberstyle=\color{black!5},
		stringstyle=\color{black}, 
		showspaces=false,
		showstringspaces=false, 
		showtabs=false, 
		tabsize=2, 
		numbers=none, 
		breaklines=true
}

\lstdefinelanguage{javascript}{
	keywords={typeof, new, true, false, catch, function, return, null, catch, switch, var, if, in, while, do, else, case, break},
	keywordstyle=\color{blue}\bfseries,
	ndkeywords={class, export, boolean, throw, implements, import, this},
	ndkeywordstyle=\color{darkgray}\bfseries,
	identifierstyle=\color{black},
	sensitive=false,
	comment=[l]{//},
	morecomment=[s]{/*}{*/},
	commentstyle=\color{purple}\ttfamily,
	stringstyle=\color{red}\ttfamily,
	morestring=[b]',
	morestring=[b]"
}

% disponi sezioni
\usepackage{titlesec}

\titleformat{\section}
	{\sffamily\Large\bfseries} 
	{\thesection}{1em}{} 
\titleformat{\subsection}
	{\sffamily\large\bfseries}   
	{\thesubsection}{1em}{} 
\titleformat{\subsubsection}
	{\sffamily\normalsize\bfseries} 
	{\thesubsubsection}{1em}{}

% tikz
\usepackage{tikz}

% float
\usepackage{float}

% grafici
\usepackage{pgfplots}
\pgfplotsset{width=10cm,compat=1.9}

% disponi alberi
\usepackage{forest}

\forestset{
	rectstyle/.style={
		for tree={rectangle,draw,font=\large\sffamily}
	},
	roundstyle/.style={
		for tree={circle,draw,font=\large}
	}
}

% disponi algoritmi
\usepackage{algorithm}
\usepackage{algorithmic}
\makeatletter
\renewcommand{\ALG@name}{Algoritmo}
\makeatother

% disponi numeri di pagina
\usepackage{fancyhdr}
\fancyhf{} 
\fancyfoot[L]{\sffamily{\thepage}}

\makeatletter
\fancyhead[L]{\raisebox{1ex}[0pt][0pt]{\sffamily{\@title \ \@date}}} 
\fancyhead[R]{\raisebox{1ex}[0pt][0pt]{\sffamily{\@author}}}
\makeatother

\begin{document}
% sezione (data)
\section{Lezione del 26-09-24}

% stili pagina
\thispagestyle{empty}
\pagestyle{fancy}

% testo
\subsection{Istruzioni di trasferimento}
Le istruzioni di trasferimento spostano memoria:
\begin{itemize}
	\item Dalla memoria a un registro;
	\item Da un registro a un registro;
	\item Dallo spazio I/O a un regsitro.
\end{itemize}

Non esistono altre possibilità, ergo non si può (per quanto interessa a noi) spostare da memoria a memoria.
In verità esistono alcune istruzioni nei processori di nuova generazione che ottimizzano operazioni di questo tipo, che verrano viste in seguito.
Sfruttando i registri, il trasferimento da memoria a memoria si fa attraverso un registro, in due istruzioni.

Nessuna istruzione di trasferimento modifica i flag.

\subsubsection{MOVE}
\begin{itemize}
	\item \textbf{Formato:} \texttt{MOV source, destination}
	\item \textbf{Azione:} sostituisce l'operando destinatario con una copia dell'operando sorgente.
	\item \textbf{Flag:} nessuno.

		\begin{table}[h!]
			\center \rowcolors{2}{white}{black!10}
			\begin{tabular} { c | p{5cm} }
				\bfseries Operandi & \bfseries Esempi \\
				\hline 
				Memoria, Registro Generale & \texttt{MOV 0x00002000, \%EDX} \\
				Registro Generale, Memoria & \texttt{MOV \%CL, 0x12AB1024} \\
				Registro Generale, Registro Generale & \texttt{MOV \%AX, \%DX} \\
				Immediato, Memoria & \texttt{MOVB \$0x5B, (\%EDI)} \\ 
				Immediato, Registro generale & \texttt{MOV \$0x54A3, \%AX}
			\end{tabular}
		\end{table}
\end{itemize}

\subsubsection{LOAD EFFECTIVE ADDRESS}
\begin{itemize}
	\item \textbf{Formato:} \texttt{LEA source, destination}
	\item \textbf{Azione:} sostituisce l'operando destinatario con l'espressione indirizzo contenuta nell'operando sorgente.
	\item \textbf{Flag:} nessuno.

		\begin{table}[h!]
			\center \rowcolors{2}{white}{black!10}
			\begin{tabular} { c | p{7cm} }
				\bfseries Operandi & \bfseries Esempi \\
				\hline 
				Memoria, Registro Generale a 32 bit & \texttt{LEA 0x00002000, \%EDX} \\
																						& \texttt{LEA 0x00213AB1 (\%EAX,\%EBX,4), \%ECX}
			\end{tabular}
		\end{table}
\end{itemize}

A differenza di MOV, LEA calcola l'indirizzo della locazione di memoria cercata come $ \text{base} + \text{index} \times \text{scala} \pm \text{displacement} $, e carica quell'indirizzo nella destinazione, non il valore contenuto in esso.
Nel primo esempio, questo equivale alla MOV con indirizzamento immediato.
In altri casi permette di ricavare esplicitamente il valore ottenuto dall'indirizzamento complesso.

\subsubsection{EXCHANGE}
\begin{itemize}
	\item \textbf{Formato:} \texttt{XCHG source, destination}
	\item \textbf{Azione:} sostituisce l'operando destinatario con l'operando sorgente e viceversa. Questa operazione è l'unica che modifica il sorgente.
	\item \textbf{Flag:} nessuno.

		\begin{table}[h!]
			\center \rowcolors{2}{white}{black!10}
			\begin{tabular} { c | p{7cm} }
				\bfseries Operandi & \bfseries Esempi \\
				\hline 
				Memoria, Registro Generale & \texttt{XCHG 0x00002000, \%DX} \\
				Registro Generale, Memoria & \texttt{XCHG \%AL, 0x000A2003} \\
				Registro Generale, Registro Generale & \texttt{XCHG \%EAX, \%EDX}
			\end{tabular}
		\end{table}

		Grazie a quest'istruzione in assembler si possono scambiare due operandi con una sola istruzione (\textbf{non trasparenza} dei registri) \textbf{atomica}.
		Questo è particolarmente utile nel caso di esecuzione concorrente.
\end{itemize}

\subsubsection{INPUT}
\begin{itemize}
	\item \textbf{Formato:}
		\begin{itemize}
			\item \texttt{IN indirizzo, \%AL} (8 bit)
			\item \texttt{IN indirizzo, \%AX} (16 bit)
			\item \texttt{IN (\%DX), \%AX} (8 bit) 
			\item \texttt{IN (\%DX), \%Al} (16 bit)
		\end{itemize}
	\item \textbf{Azione:} sostituisce il contenuto del registro destinatario (AL 8 bit, AX 16 bit) con il contenuto di un adeguato numero di porte consecutive.
		L'indirizzo è specificato direttamente (per porte con indirizzo $<256$), o indirettamente usando il registro DX.
	\item \textbf{Flag:} nessuno.
\end{itemize}

\subsubsection{OUTPUT}
\begin{itemize}
	\item \textbf{Formato:}
		\begin{itemize}
			\item \texttt{OUT \%AL, indirizzo} (8 bit)
			\item \texttt{IN \%AX, indirizzo} (16 bit)
			\item \texttt{IN \%AX}, (\%DX) (8 bit) 
			\item \texttt{IN \%Al, (\%DX)} (16 bit)
		\end{itemize}
	\item \textbf{Azione:} copia il contenuto del registro sorgente (AL 8 bit, AX 16 bit) su un adeguato numero di porte consecutive.
		L'indirizzo è specificato direttamente (per porte con indirizzo $<256$), o indirettamente usando il registro DX.
	\item \textbf{Flag:} nessuno.
\end{itemize}

\subsubsection{Non ortogonalità INPUT/OUTPUT}
Le uniche due operazioni che gestiscono l'input e l'output possono trasferire solo dai o nei registri AL e AX, e indirizzare indirettamente la memoria puntando col registro DX.
Questo rende le operazioni non ortogonali: non si possono usare altri registri, ed eventuali operazioni vanno fatte nel processore,

\subsection{Pila}
La pila, o \textbf{stack}, è una regione di memoria gestita con politica Last In First Out (LIFO), essenziale al funzionamento del calcolatore.
Permette di annidare sottoprogrammi, funzionalità per cui l'assembler è organizzato.

Generalmente, la pila viene usata come segue per eseguire i sottoprogrammi:
\begin{itemize}
	\item Prima di saltare al sottoprogramma, si fa \textbf{PUSH} sulla pila dell'indirizzo di ritorno (e.g. l'indirizzo della prossima istruzione);
	\item Si esegue il sottoprogramma;
	\item Al termine del sottoprogramma, si fa \textbf{POP} dalla pila del prossimo indirizzo.
\end{itemize}

Più sottoprogrammi possono chiamarsi a vicenda (annidarsi), ponendosi su livelli via via superiori della pila.
Al termine della sua esecuzione, ogni sottoprogramma tornerà all'indirizzo di ripresa del sottoprogramma precedente, finché tutti i sottoprogrammi non termineranno l'esecuzione.

Il registro \textbf{ESP} punta al top della pila, ergo non va usato per altri scopi.
Va però inizializzato prima che parta il programma.
Si deve inoltre notare che la pila in assembler si estende \textit{verso il basso}: aggiungere alla pila significa decrementare ESP, e rimuovere dalla pila significa incrementare ESP.
I frame successivi della pila si vanno a disporre via via sotto (o "a sinistra") del frame corrente.

Per lavorare sulla pila si usano le istruzioni:

\subsubsection{PUSH}
\begin{itemize}
	\item \textbf{Formato:} \texttt{PUSH source}
	\item \textbf{Azione:} decrementa ESP e copia il sorgente nell'indirizzo puntato da ESP.
		Il sorgente deve essere a 16 bit o a 32 bit.
		Nello specifico, compie le seguenti azioni:
		\begin{itemize}
			\item Decrementa l'indirizzo contenuto nel registro ESP di 2 o 4;
			\item Memorizza una copia dell'operando sorgente nella word o long il cui indirizzo è contenuto in ESP.
		\end{itemize}
	\item \textbf{Flag:} nessuno.
\end{itemize}

		\begin{table}[h!]
			\center \rowcolors{2}{white}{black!10}
			\begin{tabular} { c | p{5cm} }
				\bfseries Operandi & \bfseries Esempi \\
				\hline 
				Memoria & \texttt{PUSHW 0x3214200A} \\ 
				Immediato & \texttt{PUSHL \$0x4871A000} \\ 
				Registro Generale & \texttt{PUSH \%BX}
			\end{tabular}
		\end{table}

\subsubsection{POP}
\begin{itemize}
	\item \textbf{Formato:} \texttt{POP destination}
	\item \textbf{Azione:} copia una word o un long dall'indirzzo puntato dall'ESP nel destinatario e incrementa ESP.
		Nello specifico compie le seguenti azioni:
		\begin{itemize}
			\item Sostituisce all'operando destinatario una copia del contenuto nella word o long il cui indirizzo è contenuto in ESP;
			\item Incrementa di due o quattro l'indirizzo contenuto in ESP, rimuovendo la word o il long copiato.
		\end{itemize}
	\item \textbf{Flag:} nessuno.
\end{itemize}
	
		\begin{table}[h!]
			\center \rowcolors{2}{white}{black!10}
			\begin{tabular} { c | p{5cm} }
				\bfseries Operandi & \bfseries Esempi \\
				\hline 
				Memoria & \texttt{POPW 0x02AB2000} \\ 
				Registro Generale & \texttt{POP \%BX}
			\end{tabular}
		\end{table}

\par\medskip

\noindent
\textsf{\textbf{Dati temporanei nella pila}} \\
Solitamente la pila viene usata per memorizzare dati temporanei, visto che i registri sono pochi e spesso hanno scopi diversi in momenti diversi. Ad esempio:

\begin{lstlisting}[style=codestyle]	
# sto usando %EAX, mi serve un dato da una porta
PUSH %EAX
IN 0x001A, %AL
...
POP %EAX # ritorno da dove ero
\end{lstlisting}

\subsubsection{PUSHAD}
\begin{itemize}
	\item \textbf{Formato:} \texttt{PUSHAD}
	\item \textbf{Azione:}: salva nella pila corrente una copia degli 8 registri generali a 32 bit, nell'ordine: EAX, ECX, EDX, EBX, ESP, EBP, ESI, EDI.
	\item \textbf{Flag:} nessuno.
\end{itemize}

\subsubsection{POPAD}
\begin{itemize}
	\item \textbf{Formato:} \texttt{POPAD}
	\item \textbf{Azione:}: copia dalla pila corrente gli 8 registri generali a 32 bit, nell'ordine: EAX, ECX, EDX, EBX, ESP, EBP, ESI, EDI. 
	\item \textbf{Flag:} nessuno.
\end{itemize}

\subsection{Istruzioni aritmetiche}
Molte operazioni aritmetiche di base non distinguono numeri naturali e numeri interi, distinzione che viene fatta solo per moltiplicazioni e divisioni.

Le operazioni possono modificare i flag, e in questo caso i flag da controllare dipenderanno dal tipo di numeri su cui si è fatta l'operazione (informazione nota soltanto al programmatore).

Abbiamo quindi che un'operazione aritmetica si svolge seguendo i passi:
\begin{itemize}
	\item Si esegue l'operazione;
	\item Si controllano i flag interessati (OF, SF e ZF sugli interi, CF e ZF sui naturali) per verificarne l'esito.
\end{itemize}

Vediamo quindi le operazioni aritmetiche:

\subsubsection{ADD}
\begin{itemize}
	\item \textbf{Formato:} \texttt{ADD source, destination}
	\item \textbf{Azione:} modifica l'operando destinatario sommandovi l'operando sorgente.
		Il risultato è consistente sia che si interpretino i numeri come naturali, che come interi.
	\item \textbf{Flag:} attiva CF se, interpretando i numeri come naturali, si è verificato un riporto; attiva OF se, interpretando gli operandi come interi, si è verificato un traboccamento.
		Inoltre attiva opportunamente ZF e SF se il numero è rispettivamente zero o negativo (in complemento a 2).
\end{itemize}

		\begin{table}[h!]
			\center \rowcolors{2}{white}{black!10}
			\begin{tabular} { c | p{5cm} }
				\bfseries Operandi & \bfseries Esempi \\
				\hline
				Memoria, Registro Generale & \texttt{ADD 0x00002000, \%EDX} \\ 
				Registro Generale, Memoria & \texttt{ADD \%CL, 0x12AB1024} \\ 
				Registro Generale, Registro Generale & \texttt{ADD \%AX, \%DX} \\ 
				Immediato, Memoria & \texttt{ADDB \$0x5B, (\%EDI)} \\ 
				Immediato, Registro Generale & \texttt{ADD \$0x54A3, \%AX}
			\end{tabular}
		\end{table}

\par\medskip
\noindent
\textbf{\textsf{Funzionamento della ADD}} \\
Il passo elementare di una somma consiste nel sommare due addendi (propriamente due cifre degli addendi) e un riporto entrante per produrre:
	\begin{itemize}
		\item Una cifra;
		\item Un riporto uscente (cioè il riporto entrante per il prossimo passo).
	\end{itemize}
L'ultimo riporto, se non entra in memoria, attiva il carry flag (CF).

L'operazione di somma ha lo stesso effetto sia su naturali che su interi in complemento a 2: la differenza sta nel controllo dell'attivazione dei flag.
Il carry flag non ha infatti alcun significato nella somma fra interi: dobbiamo controllare l'OF.

In generale, si ha overflow (OF) quando il risultato esce dall'intervallo di rappresentabilità.
Si può capire se si è verificato un overflow controllando i segni degli operandi:
\begin{itemize}
	\item \textbf{Segni discordi:} non c'é overflow;
	\item \textbf{Segni concordi:} il risultato è concorde se è concorde con gli operandi.
\end{itemize}
La ADD imposta quindi OF secondo queste regole.
Il ZF viene poi impostato se il risultato è fatto da tutti zeri, e il SF viene impostato se il MSB è uno.

\subsubsection{INCREMENT}
\begin{itemize}
	\item \textbf{Formato:} \texttt{INC destination}
	\item \textbf{Azione:} equivale all'istruzione \texttt{ADD \$1, destination}. 
	\item \textbf{Flag:} modifica tutti i flag di ADD tranne CF (il riporto).
\end{itemize}

		\begin{table}[H]
			\center \rowcolors{2}{white}{black!10}
			\begin{tabular} { c | c }
				\bfseries Operandi & \bfseries Esempi \\
				\hline 
				Memoria & \texttt{INCB (\%ESI)} \\
				Registro Generale & \texttt{INC \%CX}
			\end{tabular}
		\end{table}

Quest'istruzione è più compatta di ADD, e storicamente era anche più veloce.
Questo deriva dal fatto che la circuiteria che implementava l'incremento era più efficiente di quella che implementa le somme.

\subsubsection{SUBTRACT}
\begin{itemize}
	\item \textbf{Formato:} \texttt{SUB source, destination}
	\item \textbf{Azione:} modifica l'operando destinatario sottraendovi l'operando sorgente. 
		Il risultato è consistente sia che si interpretino i numeri come naturali, che come interi.
	\item \textbf{Flag:} attiva CF se, interpretando i numeri come naturali, si è verificato un riporto; attiva OF se, interpretando gli operandi come interi, si è verificato un traboccamento.
\end{itemize}

		\begin{table}[h!]
			\center \rowcolors{2}{white}{black!10}
			\begin{tabular} { c | p{5cm} }
				\bfseries Operandi & \bfseries Esempi \\
				\hline
				Memoria, Registro Generale & \texttt{SUB 0x00002000, \%EDX} \\ 
				Registro Generale, Memoria & \texttt{SUB \%CL, 0x12AB1024} \\ 
				Registro Generale, Registro Generale & \texttt{SUB \%AX, \%DX} \\ 
				Immediato, Memoria & \texttt{SUBB \$0x5B, (\%EDI)} \\ 
				Immediato, Registro Generale & \texttt{SUB \$0x54A3, \%AX}
			\end{tabular}
		\end{table}

\par\medskip
\noindent
\textbf{\textsf{Funzionamento della SUBTRACT}} \\
Il passo elementare della sottrazione è effettivamente il contrario di quello della somma: si sottraggono il sottraendo e un prestito entrante al minuendo, producendo:
\begin{itemize}
	\item Una cifra;
	\item Un prestito uscente.
\end{itemize}

Il carry flag (CF) memorizza il prestito.
Se alla fine dell'operazione il CF è impostato, significa che il risultato è un numero intero.

Questo funziona anche sugli interi: in questo caso, come prima, non si controlla il CF, ma l'OF, che conterrà la seguente informazione:
\begin{itemize}
	\item La differenza di numeri concordi è sempre rappresentabile;
	\item La differenza di numeri discordi è rappresentabile solo se il risultato ha il segno del minuendo.
\end{itemize}

Il ZF e il SF vengono attivati secondo le regole già note.

\subsubsection{DECREMENT}
\begin{itemize}
	\item \textbf{Formato:} \texttt{DEC destination}
	\item \textbf{Azione:} equivale all'istruzione \texttt{SUB \$1, destination}. 
	\item \textbf{Flag:} modifica tutti i flag di SUBTRACT tranne CF (il prestito).
\end{itemize}

		\begin{table}[h!]
			\center \rowcolors{2}{white}{black!10}
			\begin{tabular} { c | c }
				\bfseries Operandi & \bfseries Esempi \\
				\hline 
				Memoria & \texttt{DECB (\%EDI)} \\
				Registro Generale & \texttt{DEC \%CX}
			\end{tabular}
		\end{table}

\subsubsection{ADD WITH CARRY}
\begin{itemize}
	\item \textbf{Formato:} \texttt{ADC source, destination}
	\item \textbf{Azione:} modifica l'operando destinatario sommandovi sia l'operando sorgente sia il contenuto del flag CF.
	\item \textbf{Flag:} modifica tutti i flag come ADD. 
\end{itemize}

\begin{table}[h!]
			\center \rowcolors{2}{white}{black!10}
			\begin{tabular} { c | p{5cm} }
				\bfseries Operandi & \bfseries Esempi \\
				\hline
				Memoria, Registro Generale & \texttt{ADC 0x00002000, \%EDX} \\ 
				Registro Generale, Memoria & \texttt{ADC \%CL, 0x12AB1024} \\ 
				Registro Generale, Registro Generale & \texttt{ADC \%AX, \%DX} \\ 
				Immediato, Memoria & \texttt{ADCB \$0x5B, (\%EDI)} \\ 
				Immediato, Registro Generale & \texttt{ADC \$0x54A3, \%AX}
			\end{tabular}
		\end{table}

Quest'istruzione è utile per effettuare somme di numeri più grandi di 32 bit.
In questo caso si:
\begin{itemize}
	\item Effettua la somma dei 32 bit meno significativi con ADD;
	\item Sommano i successivi 32 bit con ADC portandosi quindi dietro il carry.
\end{itemize}

\subsubsection{SUBTRACT WITH BORROW}
\begin{itemize}
	\item \textbf{Formato:} \texttt{SBB source, destination}
	\item \textbf{Azione:} modifica l'operando destinatario sottraendovi sia l'operando sorgente sia il contenuto del flag CF.
	\item \textbf{Flag:} modifica tutti i flag come SUBTRACT. 
\end{itemize}

		\begin{table}[H]
		\center \rowcolors{2}{white}{black!10}
			\begin{tabular} { c | p{5cm} }
				\bfseries Operandi & \bfseries Esempi \\
				\hline
				Memoria, Registro Generale & \texttt{SBB 0x00002000, \%EDX} \\ 
				Registro Generale, Memoria & \texttt{SBB \%CL, 0x12AB1024} \\ 
				Registro Generale, Registro Generale & \texttt{SBB \%AX, \%DX} \\ 
				Immediato, Memoria & \texttt{SBBB \$0x255B, (\%EDI)} \\ 
				Immediato, Registro Generale & \texttt{SBB \$0x54A3, \%AX}
			\end{tabular}
		\end{table}

Come ormai dovrebbe essere chiaro, è la duale dell'ADC, e si usa per effettuare sottrazioni di numeri più grandi di 32 bit.

\subsubsection{NEGATE}
\begin{itemize}
	\item \textbf{Formato:} \texttt{NEG destination}
	\item \textbf{Azione:} interpreta l'operando destinatario come un numero intero e lo sostituisce con il suo opposto in complemento a 2. 
	\item \textbf{Flag:} quando l'operazione non è possibile (l'intervallo di rappresentabilità degli interi in complemento a 2 non è simmetrico) imposta il flag OF.
		Imposta inoltre il flag CF quando l'operando è diverso da zero, e tutti gli altri flag in base a nullità e segno del risultato. 
\end{itemize}

		\begin{table}[H]
		\center \rowcolors{2}{white}{black!10}
			\begin{tabular} { c | p{5cm} }
				\bfseries Operandi & \bfseries Esempi \\
				\hline
				Memoria & \texttt{NEGB (\%EDI)} \\ 
				Registro Generale & \texttt{NEG \%CX}
			\end{tabular}
		\end{table}


\par\medskip
\noindent
\textbf{\textsf{Funzionamento della NEGATE}} \\
L'opposto di un numero $X$ in complemento a due è:
$$
-X = \bar{X} + 1
$$

Si ricordi che questo ha senso \textit{solamente} se il numero è rappresentato in complemento a due.

\subsubsection{COMPARE}
\begin{itemize}
	\item \textbf{Formato:} \texttt{CMP source, destination}
	\item \textbf{Azione:} verifica se l'operando destinatario è maggiore, uguale o minore dell'operando sorgente, sia interpretando gli operandi come naturali che come interi, e aggiorna i flag di conseguenza.
		Più propriamente, la compare si comporta come la SUB, ma senza sovrascrivere nessuno degli operandi.
	\item \textbf{Flag:} come la SUB. 
\end{itemize}


		\begin{table}[H]
		\center \rowcolors{2}{white}{black!10}
			\begin{tabular} { c | p{5cm} }
				\bfseries Operandi & \bfseries Esempi \\
				\hline
				Memoria, Registro Generale & \texttt{CMP 0x00002000, \%EDX} \\ 
				Registro Generale, Memoria & \texttt{CMP \%CL, 0x12AB1024} \\ 
				Registro Generale, Registro Generale & \texttt{CMP \%AX, \%DX} \\ 
				Immediato, Memoria & \texttt{CMPB \$0x255B, (\%EDI)} \\ 
				Immediato, Registro Generale & \texttt{CMP \$0x54A3, \%AX}
			\end{tabular}
		\end{table}

\subsubsection{Funzionamento della COMPARE}
Solitamente la CMP si usa nei salti condizionati come:
\begin{lstlisting}[style=codestyle]	
CMP %AX, %BX
JCOND # salto condizionato
\end{lstlisting}
\noindent
Ciò che fa la CMP è effettivamente creare un'oggetto temporaneo:
$$
\text{tmp} = \text{dest} - \text{source}
$$
che viene poi rimosso.

I flag restano però aggiornati, e questo valore può essere interpretato correttamente dalla JE per effettuare un salto condizionale.


\subsection{Moltiplicazioni}
Le moltiplicazioni, a differenza delle somme e delle differenze, sono diverse fra naturali ed interi.
Bisogna inoltre notare che le dimensioni il risultato della somma di un numero a $n$ cifre sta su $n$ o $n+1$ cifre, mentre il prodotto di due numeri a $n$ cifre sta su $2n$ cifre.
In altre parole, il numero di bit necessari a memorizzare il risultato non è più confrontabile con quello degli operatori.


\subsubsection{MULTIPLY}
\begin{itemize}
	\item \textbf{Formato:} \texttt{MUL source}
	\item \textbf{Azione:} considera l'operando sorgente come un moltiplicando, l'operando destinatario (implicito) come un moltiplicatore, e effettua la moltiplicazione assumendo i numeri naturali. Nello specifico:
	\begin{itemize}
		\item Sorgente a 8 bit, si ha $\text{AX} = \text{AL} \times \text{source}$;
		\item Sorgente a 16 bit, si ha $\text{DX}\_\text{AX} = \text{AX} \times \text{source}$;
		\item Sorgente a 32 bit, si ha $\text{EDX}\_\text{EAX} = \text{EAX} \times \text{source}$.
	\end{itemize}
	\item \textbf{Flag:} imposta CF e OF se il risultato non sta nel numero di bit di source. SF e ZF sono indefiniti.
\end{itemize}

		\begin{table}[H]
		\center \rowcolors{2}{white}{black!10}
			\begin{tabular} { c | p{5cm} }
				\bfseries Operandi & \bfseries Esempi \\
				\hline
				Memoria & \texttt{MULB (\%ESI)} \\ 
				Registro Generale & \texttt{MUL \%ECX}
			\end{tabular}
		\end{table}

\subsubsection{INTEGER MULTIPLY}
\begin{itemize}
	\item \textbf{Formato:} \texttt{MUL source}
	\item \textbf{Azione:} considera l'operando sorgente come un moltiplicando, l'operando destinatario (implicito) come un moltiplicatore, e effettua la moltiplicazione assumendo i numeri interi. Nello specifico:
	\begin{itemize}
		\item Sorgente a 8 bit, si ha $\text{AX} = \text{AL} \times \text{source}$;
		\item Sorgente a 16 bit, si ha $\text{DX}\_\text{AX} = \text{AX} \times \text{source}$;
		\item Sorgente a 32 bit, si ha $\text{EDX}\_\text{EAX} = \text{EAX} \times \text{source}$.
	\end{itemize}
	\item \textbf{Flag:} li imposta tutti, ma non è attendibile.
\end{itemize}

		\begin{table}[H]
		\center \rowcolors{2}{white}{black!10}
			\begin{tabular} { c | p{5cm} }
				\bfseries Operandi & \bfseries Esempi \\
				\hline
				Memoria & \texttt{IMULB (\%ESI)} \\ 
				Registro Generale & \texttt{IMUL \%ECX}
			\end{tabular}
		\end{table}

\par\medskip
\noindent
\textbf{\textsf{Funzionamento delle MULTIPLY e INTEGER MULTIPLY}} \\
Queste operazioni hanno sia un operando che il destinatario impliciti, in base al tipo dell'operando fornito.
Questo deriva dal fatto che il risultato di una moltiplicazione raramente sta nello stesso numero di bit dei fattori.
Di preciso, abbiamo visto i 3 tipi di moltiplicazione concessi:
\begin{itemize}
	\item Sorgente a 8 bit, si ha $\text{AX} = \text{AL} \times \text{source}$;
	\item Sorgente a 16 bit, si ha $\text{DX}\_\text{AX} = \text{AX} \times \text{source}$;
	\item Sorgente a 32 bit, si ha $\text{EDX}\_\text{EAX} = \text{EAX} \times \text{source}$.
\end{itemize}

La differenza fra le prime due operazioni e l'ultima, in particolare con sorgente a 16 bit, che usa una due registri da 16 bit separati, ha principalmente motivi storici (il registro EAX è stato introdotto dopo).

Si può rimettere il valore dai due registri a 16 bit in un registro a 32 bit attraverso la pila:
\begin{lstlisting}[style=codestyle]	
PUSH \%DX
PUSH \%AX
POP \%EAX
\end{lstlisting}

\end{document}


\documentclass[a4paper,11pt]{article}
\usepackage[a4paper, margin=8em]{geometry}

% usa i pacchetti per la scrittura in italiano
\usepackage[french,italian]{babel}
\usepackage[T1]{fontenc}
\usepackage[utf8]{inputenc}
\frenchspacing 

% usa i pacchetti per la formattazione matematica
\usepackage{amsmath, amssymb, amsthm, amsfonts}

% usa altri pacchetti
\usepackage{gensymb}
\usepackage{hyperref}
\usepackage{standalone}

% imposta il titolo
\title{Appunti Reti Logiche}
\author{Luca Seggiani}
\date{2024}

% imposta lo stile
% usa helvetica
\usepackage[scaled]{helvet}
% usa palatino
\usepackage{palatino}
% usa un font monospazio guardabile
\usepackage{lmodern}

\renewcommand{\rmdefault}{ppl}
\renewcommand{\sfdefault}{phv}
\renewcommand{\ttdefault}{lmtt}

% disponi il titolo
\makeatletter
\renewcommand{\maketitle} {
	\begin{center} 
		\begin{minipage}[t]{.8\textwidth}
			\textsf{\huge\bfseries \@title} 
		\end{minipage}%
		\begin{minipage}[t]{.2\textwidth}
			\raggedleft \vspace{-1.65em}
			\textsf{\small \@author} \vfill
			\textsf{\small \@date}
		\end{minipage}
		\par
	\end{center}

	\thispagestyle{empty}
	\pagestyle{fancy}
}
\makeatother

% disponi teoremi
\usepackage{tcolorbox}
\newtcolorbox[auto counter, number within=section]{theorem}[2][]{%
	colback=blue!10, 
	colframe=blue!40!black, 
	sharp corners=northwest,
	fonttitle=\sffamily\bfseries, 
	title=Teorema~\thetcbcounter: #2, 
	#1
}

% disponi definizioni
\newtcolorbox[auto counter, number within=section]{definition}[2][]{%
	colback=red!10,
	colframe=red!40!black,
	sharp corners=northwest,
	fonttitle=\sffamily\bfseries,
	title=Definizione~\thetcbcounter: #2,
	#1
}

% disponi codice
\usepackage{listings}
\usepackage[table]{xcolor}

\definecolor{codegreen}{rgb}{0,0.6,0}
\definecolor{codegray}{rgb}{0.5,0.5,0.5}
\definecolor{codepurple}{rgb}{0.58,0,0.82}
\definecolor{backcolour}{rgb}{0.95,0.95,0.92}

\lstdefinestyle{codestyle}{
		backgroundcolor=\color{black!5}, 
		commentstyle=\color{codegreen},
		keywordstyle=\bfseries\color{magenta},
		numberstyle=\sffamily\tiny\color{black!60},
		stringstyle=\color{green!50!black},
		basicstyle=\ttfamily\footnotesize,
		breakatwhitespace=false,         
		breaklines=true,                 
		captionpos=b,                    
		keepspaces=true,                 
		numbers=left,                    
		numbersep=5pt,                  
		showspaces=false,                
		showstringspaces=false,
		showtabs=false,                  
		tabsize=2
}

\lstdefinestyle{shellstyle}{
		backgroundcolor=\color{black!5}, 
		basicstyle=\ttfamily\footnotesize\color{black}, 
		commentstyle=\color{black}, 
		keywordstyle=\color{black},
		numberstyle=\color{black!5},
		stringstyle=\color{black}, 
		showspaces=false,
		showstringspaces=false, 
		showtabs=false, 
		tabsize=2, 
		numbers=none, 
		breaklines=true
}


\lstdefinelanguage{assembler}{
  keywords={AAA, AAD, AAM, AAS, ADC, ADCB, ADCW, ADCL, ADD, ADDB, ADDW, ADDL, AND, ANDB, ANDW, ANDL,
        ARPL, BOUND, BSF, BSFL, BSFW, BSR, BSRL, BSRW, BSWAP, BT, BTC, BTCB, BTCW, BTCL, BTR, 
        BTRB, BTRW, BTRL, BTS, BTSB, BTSW, BTSL, CALL, CBW, CDQ, CLC, CLD, CLI, CLTS, CMC, CMP,
        CMPB, CMPW, CMPL, CMPS, CMPSB, CMPSD, CMPSW, CMPXCHG, CMPXCHGB, CMPXCHGW, CMPXCHGL,
        CMPXCHG8B, CPUID, CWDE, DAA, DAS, DEC, DECB, DECW, DECL, DIV, DIVB, DIVW, DIVL, ENTER,
        HLT, IDIV, IDIVB, IDIVW, IDIVL, IMUL, IMULB, IMULW, IMULL, IN, INB, INW, INL, INC, INCB,
        INCW, INCL, INS, INSB, INSD, INSW, INT, INT3, INTO, INVD, INVLPG, IRET, IRETD, JA, JAE,
        JB, JBE, JC, JCXZ, JE, JECXZ, JG, JGE, JL, JLE, JMP, JNA, JNAE, JNB, JNBE, JNC, JNE, JNG,
        JNGE, JNL, JNLE, JNO, JNP, JNS, JNZ, JO, JP, JPE, JPO, JS, JZ, LAHF, LAR, LCALL, LDS,
        LEA, LEAVE, LES, LFS, LGDT, LGS, LIDT, LMSW, LOCK, LODSB, LODSD, LODSW, LOOP, LOOPE,
        LOOPNE, LSL, LSS, LTR, MOV, MOVB, MOVW, MOVL, MOVSB, MOVSD, MOVSW, MOVSX, MOVSXB,
        MOVSXW, MOVSXL, MOVZX, MOVZXB, MOVZXW, MOVZXL, MUL, MULB, MULW, MULL, NEG, NEGB, NEGW,
        NEGL, NOP, NOT, NOTB, NOTW, NOTL, OR, ORB, ORW, ORL, OUT, OUTB, OUTW, OUTL, OUTSB, OUTSD,
        OUTSW, POP, POPA, POPAD, POPF, POPFD, PUSH, PUSHA, PUSHAD, PUSHF, PUSHFD, RCL, RCLB, RCLW,
        RCLL, RCR, RCRB, RCRW, RCRL, RDMSR, RDPMC, RDTSC, REP, REPE, REPNE, RET, ROL, ROLB, ROLW,
        ROLL, ROR, RORB, RORW, RORL, SAHF, SAL, SALB, SALW, SALL, SAR, SARB, SARW, SARL, SBB,
        SBBB, SBBW, SBBL, SCASB, SCASD, SCASW, SETA, SETAE, SETB, SETBE, SETC, SETE, SETG, SETGE,
        SETL, SETLE, SETNA, SETNAE, SETNB, SETNBE, SETNC, SETNE, SETNG, SETNGE, SETNL, SETNLE,
        SETNO, SETNP, SETNS, SETNZ, SETO, SETP, SETPE, SETPO, SETS, SETZ, SGDT, SHL, SHLB, SHLW,
        SHLL, SHLD, SHR, SHRB, SHRW, SHRL, SHRD, SIDT, SLDT, SMSW, STC, STD, STI, STOSB, STOSD,
        STOSW, STR, SUB, SUBB, SUBW, SUBL, TEST, TESTB, TESTW, TESTL, VERR, VERW, WAIT, WBINVD,
        XADD, XADDB, XADDW, XADDL, XCHG, XCHGB, XCHGW, XCHGL, XLAT, XLATB, XOR, XORB, XORW, XORL},
  keywordstyle=\color{blue}\bfseries,
  ndkeywordstyle=\color{darkgray}\bfseries,
  identifierstyle=\color{black},
  sensitive=false,
  comment=[l]{\#},
  morecomment=[s]{/*}{*/},
  commentstyle=\color{purple}\ttfamily,
  stringstyle=\color{red}\ttfamily,
  morestring=[b]',
  morestring=[b]"
}

\lstset{language=assembler, style=codestyle}

% disponi sezioni
\usepackage{titlesec}

\titleformat{\section}
	{\sffamily\Large\bfseries} 
	{\thesection}{1em}{} 
\titleformat{\subsection}
	{\sffamily\large\bfseries}   
	{\thesubsection}{1em}{} 
\titleformat{\subsubsection}
	{\sffamily\normalsize\bfseries} 
	{\thesubsubsection}{1em}{}

% tikz
\usepackage{tikz}

% float
\usepackage{float}

% grafici
\usepackage{pgfplots}
\pgfplotsset{width=10cm,compat=1.9}

% disponi alberi
\usepackage{forest}

\forestset{
	rectstyle/.style={
		for tree={rectangle,draw,font=\large\sffamily}
	},
	roundstyle/.style={
		for tree={circle,draw,font=\large}
	}
}

% disponi algoritmi
\usepackage{algorithm}
\usepackage{algorithmic}
\makeatletter
\renewcommand{\ALG@name}{Algoritmo}
\makeatother

% disponi numeri di pagina
\usepackage{fancyhdr}
\fancyhf{} 
\fancyfoot[L]{\sffamily{\thepage}}

\makeatletter
\fancyhead[L]{\raisebox{1ex}[0pt][0pt]{\sffamily{\@title \ \@date}}} 
\fancyhead[R]{\raisebox{1ex}[0pt][0pt]{\sffamily{\@author}}}
\makeatother

\begin{document}
% sezione (data)
\section{Lezione del 27-09-24}

% stili pagina
\thispagestyle{empty}
\pagestyle{fancy}

% testo
\subsection{Divisioni}
La divisone è l'operazione più complessa fra le 4 operazioni aritmetiche fondamentali.
I risultati, di base, sono due: \textbf{quoziente} e \textbf{resto}.
Inoltre, l'operazione non è ben definita quando il divisore vale 0.

Facciamo innanzitutto delle considerazioni di dimensione dei risultati:
$$
X / Y \rightarrow (Q, R), \quad
0 \leq R \leq Y - 1, \quad
0 \leq Q \leq X
$$

In assembler, si assume il quoziente e il resto stiano sulla metà dei bit che rappresentano il dividendo.
Bisogna fare attenzione in quanto questo non è sempre il caso.

\subsubsection{DIVIDE}
\begin{itemize}
	\item \textbf{Formato:} \lstinline|DIV source|
	\item \textbf{Azione:} considera l'operando sorgente come un divisore, l'operando destinatario (implicito) come un dividendo, e effettua la divisione assumendo i numeri naturali. Nello specifico:
	\begin{itemize}
	\item Sorgente a 8 bit, si ha $\text{AL} = \text{AX} \div \text{source}$, e $ \text{AH} = \text{AX} \mod \text{source} $;
	\item Sorgente a 16 bit, si ha $\text{AX} = \text{DX\_AX} \div \text{source}$, e $ \text{DX} = \text{DX\_AX} \mod \text{source} $;
	\item Sorgente a 32 bit, si ha $\text{EAX} = \text{EDX\_EAX} \div \text{source}$, e $ \text{EDX} = \text{EDX\_EAX} \mod \text{source} $;
	\end{itemize}
		Nel caso il quoziente non sia esprimibile su un numero di bit pari a quello del divisore, allora si genera un'eccezione interna, che mette in esecuzione un sottoprogramma.
		Da lì in poi i risultati generati non sono più attendibili
	\item \textbf{Flag:} imposta tutti i bit, ma non è attendibile. 
\end{itemize}

		\begin{table}[H]
		\center \rowcolors{2}{white}{black!10}
			\begin{tabular} { c | p{10cm} }
				\bfseries Operandi & \bfseries Esempi \\
				\hline
				Memoria & \lstinline!DIVB (\%ESI)	\# AX destinazione implicita! \\ 
				Registro Generale & \lstinline|DIV \%ECX	\# EDX\_EAX destinazione implicita|
			\end{tabular}
		\end{table}

Attenzione: la destinazione implicita non è quella che va a contenere il risultato, ma quella che contiene il dividendo.
Negli esempi, le destinazioni quoziente resto sono rispettivamente AL e AH, EAX e EDX.

\subsubsection{INTEGER DIVIDE}
\begin{itemize}
	\item \textbf{Formato:} \lstinline|IMUL source|
	\item \textbf{Azione:} considera l'operando sorgente come un divisore, l'operando destinatario (implicito) come un dividendo, e effettua la divisione assumendo i numeri interi. Nello specifico:
	\begin{itemize}
	\item Sorgente a 8 bit, si ha $\text{AL} = \text{AX} / \text{source}$, e $ \text{AH} = \text{AX} \mod \text{source} $;
	\item Sorgente a 16 bit, si ha $\text{AX} = \text{DX\_AX} / \text{source}$, e $ \text{DX} = \text{DX\_AX} \mod \text{source} $;
	\item Sorgente a 32 bit, si ha $\text{EAX} = \text{EDX\_EAX} / \text{source}$, e $ \text{EDX} = \text{EDX\_EAX} \mod \text{source} $;
	\end{itemize}
	\item \textbf{Flag:} li imposta tutti, ma non è attendibile.
\end{itemize}

		\begin{table}[H]
		\center \rowcolors{2}{white}{black!10}
			\begin{tabular} { c | p{10cm} }
				\bfseries Operandi & \bfseries Esempi \\
				\hline
				Memoria & \lstinline|IDIVB (\%ESI)	\# AX destinazione implicita| \\ 
				Registro Generale & \lstinline|IDIV \%ECX	\# EDX\_EAX destinazione implicita|
			\end{tabular}
		\end{table}

Bisogna stare attenti ai segni della divisione intera.
Nella divisione intera il resto ha sempre il segno del dividendo, ed è minore in modulo del divisore.
Ciò significa che il quoziente si approssima sempre all'intero più vicino allo zero (\textit{per troncamento}).
Ad esempio, $-7 \ \mathrm{idiv} \ 3 = -2, -1$ e $7 \ \mathrm{idiv} \ -3 = -2, +1$.

\par\medskip
\noindent
\textbf{\textsf{Funzionamento delle DIVIDE e INTEGER DIVIDE}} \\
Esistono quindi, come per le moltiplicazioni, tre tipi di divisione, con operando e destinatario impliciti:
\begin{itemize}
	\item Sorgente a 8 bit, si ha $\text{AL} = \text{AX} / \text{source}$, e $ \text{AH} = \text{AX} \mod \text{source} $;
	\item Sorgente a 16 bit, si ha $\text{AX} = \text{DX\_AX} / \text{source}$, e $ \text{DX} = \text{DX\_AX} \mod \text{source} $;
	\item Sorgente a 32 bit, si ha $\text{EAX} = \text{EDX\_EAX} / \text{source}$, e $ \text{EDX} = \text{EDX\_EAX} \mod \text{source} $;
\end{itemize}

In tabella questo significa:

\begin{table}[h!]
	\center \rowcolors{2}{white}{black!10}
	\begin{tabular} { c | c | c | c | c }
		\bfseries Dim. sorgente (divisore) & \bfseries Dim. dividendo & \bfseries Dividendo & \bfseries Quoziente & \bfseries Resto \\ 
		\hline 
		8 bit & 16 bit & AX & AL & AH \\ 
		16 bit & 32 bit & DX\_AX & AX & DX \\ 
		32 bit & 64 bit & EDX\_EAX & EAX & EDX
	\end{tabular}
\end{table}

Se il quoziente non sta nel numero di bit previsto, viene sollevata un'eccezione, e il programma va in HALT.
Bisogna quindi decidere quali versioni usare tenendo conto delle dimensioni dei possibili quoziente.
Questo è importante in quanto non è cosi raro avere divisioni dove il quoziente non sta nella metà dei bit del dividendo, ad esempio:

\begin{lstlisting}[language=assembler,style=codestyle]	
MOV $3, %CL
MOV $15000, %AX
DIV %CL	# come metto 5000 su una locazione da 8 bit?
\end{lstlisting}

per risolvere il problema, dobbiamo costringere il processore ad usare un altro tipo di divisione, quindi:
\begin{lstlisting}[language=assembler,style=codestyle]	
MOV $3, %CX
MOV $15000, %AX
MOV $0, %DX	# devo ripulire DX, verra' usato il dividendo DX_AX 
DIV %CX	# il risultato va in AX, tutto bene 
\end{lstlisting}

\subsection{Note conclusive su moltiplicazioni e divisioni}
Dobbiamo quindi ricordarci, riguardo a moltiplicazioni e divisioni, di:
\begin{itemize}
	\item Scegliere con cura la versione che usiamo (sopratutto nel caso di divisioni dove il quoziente potrebbe non stare nella metà del numero di bit del dividendo);
	\item Azzerare di azzerare i registri DX o EDX prima della divisione, se è a più di 8 bit;
	\item Ricordare che il contenuto di DX o EDX viene modificato per operazioni su più di 8 bit.
\end{itemize}

\subsection{Estensione di campo}
Attraverso l'estensione di campo si rappresenta lo stesso numero su più cifre.
Questo è banale sui naturali (si aggiunge uno zero), ma più complicato per gli interi.
In questo caso si estende con il bit più significativo (quello di segno).

\subsubsection{CONVERT BYTE TO WORD}
\begin{itemize}
	\item \textbf{Formato:} \lstinline|CBX|
	\item \textbf{Azione:} interpreta il contenuto di AL come un numero intero a 8 bit, la rappresenta su 16 bit e quindi lo memorizza in AX.
	\item \textbf{Flag:} nessuno.
\end{itemize}

\subsubsection{CONVERT WORD TO DOUBLEWORD}
\begin{itemize}
	\item \textbf{Formato:} \lstinline|CWDE|
	\item \textbf{Azione:} interpreta il contenuto di AX come un numero intero a 16 bit, la rappresenta su 32 bit e quindi lo memorizza in EAX.
	\item \textbf{Flag:} nessuno.
\end{itemize}

Poniamo ad esempio di voler sommare due interi, uno in AX e l'altro in EBX:
\begin{lstlisting}[language=assembler,style=codestyle]	
MOV $-5, %AX
MOV $100000, $EBX
CWDE
ADD %EAX, %EBX
\end{lstlisting}

\subsection{Istruzioni di traslazione e rotazione}
Queste istruzioni variano l'ordine dei bit in un operando destinatario.
Hanno due formati: \lstinline|OCPODE source, destination| o \lstinline|OPCODE destination|.

Quando si specifica un sorgente, esso rappresenta il numero di iterazioni per cui si ripete l'operazione.
Il sorgente può essere ad indirizzamento immediato o essere il registro CL.
Inoltre, deve essere $\leq 31$ (sarebbe inutile fare $\geq32$ trasformazioni di 32 bit).
Quando è omesso, il sorgente vale di default 1.

\subsubsection{SHIFT LOGICAL LEFT}
\begin{itemize}
	\item \textbf{Formato:} \lstinline|SHL source, destination|
	\item \textbf{Azione:} interpreta l'operando sorgente come un naturale $n$, e per $n$ iterazioni:
		\begin{itemize}
			\item Sostituisce il bit in CF con il MSB;
			\item Sostituisce ogni bit (tranne il LSB) con il bit immediatamente a destra  (il meno significativo);
			\item Sostituisce il LSB con 0.
		\end{itemize}
	\item \textbf{Flag:} nessuno.
\end{itemize}

		\begin{table}[H]
		\center \rowcolors{2}{white}{black!10}
			\begin{tabular} { c | p{5cm} }
				\bfseries Operandi & \bfseries Esempi \\
				\hline
				Immediato, Registro Generale & \lstinline|SHL \$1, \%EAX| \\
				Immediato, Memoria & \lstinline|SHLB \$7, 0x00002000| \\
				Registro CL, Registro Generale & \lstinline|SHL \%CL, \%EAX| \\
				Registro CL, Memoria & \lstinline|SHLL \%CL, (\%EDI)| \\
				Memoria & \lstinline|SHLL (\%EDI)| \\ 
				Registro Generale & \lstinline|SHL \%AX|
			\end{tabular}
		\end{table}

La SHL è utile per effettuare moltiplicazioni per 2 (shift a sinistra in binario significa $\times 2$), tranne nei casi in cui il prodotto non sta sul numero di bit del destinatario.

Per questo si controlla il CF, facendo però attenzione che per $n$ iterazioni (date dal sorgente) vengono effettuati $n$ sovrascrizioni del CF.
Ergo, se la moltiplicazione fallisce, non sappiamo \textit{quando} fallisce.

\subsubsection{SHIFT ARITHMETIC LEFT}
\begin{itemize}
	\item \textbf{Formato:} \lstinline|SAL source, destination|
	\item \textbf{Azione:} è identica alla SHL. 
		Quindi equivale a moltiplicare per $2^\text{source}$.
	\item \textbf{Flag:} nessuno.
\end{itemize}

Esiste come duale della SAR, ma in questo caso non deve fare nulla di diverso dalla SHL.

\subsubsection{SHIFT LOGICAL RIGHT}
\begin{itemize}
	\item \textbf{Formato:} \lstinline|SHR source, destination|
	\item \textbf{Azione:} interpreta l'operando sorgente come un naturale $n$, e per $n$ iterazioni:
		\begin{itemize}
			\item Sostituisce il bit in CF con il LSB;
			\item Sostituisce ogni bit (tranne il MSB) con il bit immediatamente a sinistra (il più significativo);
			\item Sostituisce il MSB con 0.
		\end{itemize}
	\item \textbf{Flag:} nessuno.
\end{itemize}

		\begin{table}[H]
		\center \rowcolors{2}{white}{black!10}
			\begin{tabular} { c | p{5cm} }
				\bfseries Operandi & \bfseries Esempi \\
				\hline
				Immediato, Registro Generale & \lstinline|SHR \$1, \%EAX| \\
				Immediato, Memoria & \lstinline|SHRB \$7, 0x00002000| \\
				Registro CL, Registro Generale & \lstinline|SHR \%CL, \%EAX| \\
				Registro CL, Memoria & \lstinline|SHRL \%CL, (\%EDI)| \\
				Memoria & \lstinline|SHRL (\%EDI)| \\ 
				Registro Generale & \lstinline|SHR \%AX|
			\end{tabular}
		\end{table}

La SHR, come la SHL, è utile per effettuare divisioni per 2 (shift a destra in binario significa $\div 2$), concessa approssimazione del bit perso, tranne nei casi in cui il numero è un intero (lo 0 al MSB corrompe il segno). 
Per questo motivo si definisce la:

\subsubsection{SHIFT ARITHMETIC RIGHT}
\begin{itemize}
	\item \textbf{Formato:} \lstinline|SAR source, destination|
	\item \textbf{Azione:} è identica alla SHR, ma non sostituisce il MSB con 0, lasciandolo tale.
		Questo equivale a dividere per $2^\text{source}$.
	\item \textbf{Flag:} nessuno.
\end{itemize}

La SAR ci permette di dividere velocemente interi per 2, come avremmo fatto sui naturali con la SHR.

\subsubsection{Divisioni intere}
Le IDIV e SAR approssimano diversamente: la IDIV approssima per troncamento, mentre la SAR approssima sempre a sinistra.
Quindi, IDIV e SAR danno lo stesso quoziente solo quando il dividendo è positivo, o il resto nullo.

\subsection{Istruzioni di rotazione}
Le istruzioni di rotazione ruotano i bit, cioè effettuano uno shift con rientro dei bit in uscita dal lato opposto, con la possibilità di includere o meno CF nella rotazione.

\subsubsection{ROTATE LEFT}
\begin{itemize}
	\item \textbf{Formato:} \lstinline|ROL source, destination|
	\item \textbf{Azione:} interpreta l'operando sorgente come un naturale $n$, e per $n$ iterazioni ruota verso sinistra senza usare il carry.
	\item \textbf{Flag:} nessuno.
\end{itemize}

		\begin{table}[H]
		\center \rowcolors{2}{white}{black!10}
			\begin{tabular} { c | p{5cm} }
				\bfseries Operandi & \bfseries Esempi \\
				\hline
				Immediato, Registro Generale & \lstinline|ROL \$1, \%EAX| \\
				Immediato, Memoria & \lstinline|ROLB \$7, 0x00002000| \\
				Registro CL, Registro Generale & \lstinline|ROL \%CL, \%EAX| \\
				Registro CL, Memoria & \lstinline|ROLL \%CL, (\%EDI)| \\
				Memoria & \lstinline|ROLL (\%EDI)| \\ 
				Registro Generale & \lstinline|ROL \%AX|
			\end{tabular}
		\end{table}

\subsubsection{ROTATE RIGHT}
\begin{itemize}
	\item \textbf{Formato:} \lstinline|ROR source, destination|
	\item \textbf{Azione:} interpreta l'operando sorgente come un naturale $n$, e per $n$ iterazioni ruota verso destra senza usare il carry.
	\item \textbf{Flag:} nessuno.
\end{itemize}

		\begin{table}[H]
		\center \rowcolors{2}{white}{black!10}
			\begin{tabular} { c | p{5cm} }
				\bfseries Operandi & \bfseries Esempi \\
				\hline
				Immediato, Registro Generale & \lstinline|ROR \$1, \%EAX| \\
				Immediato, Memoria & \lstinline|RORB \$7, 0x00002000| \\
				Registro CL, Registro Generale & \lstinline|ROR \%CL, \%EAX| \\
				Registro CL, Memoria & \lstinline|RORL \%CL, (\%EDI)| \\
				Memoria & \lstinline|RORL (\%EDI)| \\ 
				Registro Generale & \lstinline|ROR \%AX|
			\end{tabular}
		\end{table}

\subsubsection{ROTATE CARRY LEFT}
\begin{itemize}
	\item \textbf{Formato:} \lstinline|RCL source, destination|
	\item \textbf{Azione:} interpreta l'operando sorgente come un naturale $n$, e per $n$ iterazioni ruota verso sinistra usando il carry.
	\item \textbf{Flag:} imposta il carry assumendolo a sinistra del MSB.
\end{itemize}

		\begin{table}[H]
		\center \rowcolors{2}{white}{black!10}
			\begin{tabular} { c | p{5cm} }
				\bfseries Operandi & \bfseries Esempi \\
				\hline
				Immediato, Registro Generale & \lstinline|RCL \$1, \%EAX| \\
				Immediato, Memoria & \lstinline|RCLB \$7, 0x00002000| \\
				Registro CL, Registro Generale & \lstinline|RCL \%CL, \%EAX| \\
				Registro CL, Memoria & \lstinline|RCLL \%CL, (\%EDI)| \\
				Memoria & \lstinline|RCLL (\%EDI)| \\ 
				Registro Generale & \lstinline|RCL \%AX|
			\end{tabular}
		\end{table}

\subsubsection{ROTATE CARRY RIGHT}
\begin{itemize}
	\item \textbf{Formato:} \lstinline|RCR source, destination|
	\item \textbf{Azione:} interpreta l'operando sorgente come un naturale $n$, e per $n$ iterazioni ruota verso destra usando il carry.
	\item \textbf{Flag:} imposta il carry assumendolo a destra del LSB.
\end{itemize}

		\begin{table}[H]
		\center \rowcolors{2}{white}{black!10}
			\begin{tabular} { c | p{5cm} }
				\bfseries Operandi & \bfseries Esempi \\
				\hline
				Immediato, Registro Generale & \lstinline|RCR \$1, \%EAX| \\
				Immediato, Memoria & \lstinline|RCRB \$7, 0x00002000| \\
				Registro CL, Registro Generale & \lstinline|RCR \%CL, \%EAX| \\
				Registro CL, Memoria & \lstinline|RCRL \%CL, (\%EDI)| \\
				Memoria & \lstinline|RCRL (\%EDI)| \\ 
				Registro Generale & \lstinline|RCR \%AX|
			\end{tabular}
		\end{table}

\subsection{Istruzioni logiche}
Queste istruzioni applicano gli operatori dell'algebra di Boole, e solitamente modificano flag.

\subsubsection{NOT}
\begin{itemize}
	\item \textbf{Formato:} \lstinline|NOT destination|
	\item \textbf{Azione:} modifica il destinatario applicandogli il NOT bit a bit. 
	\item \textbf{Flag:} nessuno. 
\end{itemize}

		\begin{table}[H]
		\center \rowcolors{2}{white}{black!10}
			\begin{tabular} { c | p{5cm} }
				\bfseries Operandi & \bfseries Esempi \\
				\hline
				Memoria & \lstinline|NOTL (\%ESI)| \\ 
				Registro Generale & \lstinline|NOT \%CX| 
			\end{tabular}
		\end{table}

\subsubsection{AND}
\begin{itemize}
	\item \textbf{Formato:} \lstinline|AND source, destination|
	\item \textbf{Azione:} modifica il destinatario applicando l'AND bit a bit degli operandi. 
	\item \textbf{Flag:} modifica tutti i flag (annulla CF e OF).
\end{itemize}

		\begin{table}[H]
		\center \rowcolors{2}{white}{black!10}
			\begin{tabular} { c | p{5cm} }
				\bfseries Operandi & \bfseries Esempi \\
				\hline
				Memoria, Registro Generale & \lstinline|AND 0x00002000, \%EDX| \\ 
				Registro Generale, Memoria & \lstinline|AND \%CL, 0x12AB1024| \\ 
				Registro Generale, Registro Generale & \lstinline|AND \%AX, \%DX| \\ 
				Immediato, Memoria & \lstinline|AND 5x5B, (\%EDI)| \\ 
				Immediato, Registro Generale & \lstinline|AND \$0x45AB54A3, \%EAX|
			\end{tabular}
		\end{table}

\subsubsection{OR}
\begin{itemize}
	\item \textbf{Formato:} \lstinline|OR source, destination|
	\item \textbf{Azione:} modifica il destinatario applicando l'OR bit a bit degli operandi. 
	\item \textbf{Flag:} modifica tutti i flag (annulla CF e OF).
\end{itemize}

		\begin{table}[H]
		\center \rowcolors{2}{white}{black!10}
			\begin{tabular} { c | p{5cm} }
				\bfseries Operandi & \bfseries Esempi \\
				\hline
				Memoria, Registro Generale & \lstinline|OR 0x00002000, \%EDX| \\ 
				Registro Generale, Memoria & \lstinline|OR \%CL, 0x12AB1024| \\ 
				Registro Generale, Registro Generale & \lstinline|OR \%AX, \%DX| \\ 
				Immediato, Memoria & \lstinline|OR 5x5B, (\%EDI)| \\ 
				Immediato, Registro Generale & \lstinline|OR \$0x45AB54A3, \%EAX|
			\end{tabular}
		\end{table}

\subsubsection{XOR}
\begin{itemize}
	\item \textbf{Formato:} \lstinline|XOR source, destination|
	\item \textbf{Azione:} modifica il destinatario applicando l'OR bit a bit degli operandi. 
	\item \textbf{Flag:} modifica tutti i flag (annulla CF e OF).
\end{itemize}

		\begin{table}[H]
		\center \rowcolors{2}{white}{black!10}
			\begin{tabular} { c | p{5cm} }
				\bfseries Operandi & \bfseries Esempi \\
				\hline
				Memoria, Registro Generale & \lstinline|XOR 0x00002000, \%EDX| \\ 
				Registro Generale, Memoria & \lstinline|XOR \%CL, 0x12AB1024| \\ 
				Registro Generale, Registro Generale & \lstinline|XOR \%AX, \%DX| \\ 
				Immediato, Memoria & \lstinline|XOR 5x5B, (\%EDI)| \\ 
				Immediato, Registro Generale & \lstinline|XOR \$0x45AB54A3, \%EAX|
			\end{tabular}
		\end{table}

\subsubsection{Uso delle istruzioni logiche}
Le istruzioni logiche vengono usate per operare su singoli bit degli operandi, usando uno specifico operatore sorgente immediato detto maschera (\textbf{bitmask}).
Nello specifico:
\begin{itemize}
	\item \textbf{AND:} 
		\begin{itemize}
			\item si usa per testare singoli bit di un operando.
			Ad esempio, si può implementare un salto condizionale se il quinto bit di AL vale zero:
			\begin{lstlisting}[language=assembler,style=codestyle]	
AND $0x20, %AL	# 0x20 = 00100000
JZ # vale zero
\end{lstlisting} 
			\item si usa per resettare singoli bit di un operando.
			Ad esempio, si può resettare il sesto bit di BH:
			\begin{lstlisting}[language=assembler,style=codestyle]	
AND $0xBF, $BH	# 0xBF = 10111111
\end{lstlisting}
			\item si usa per l'estensione di operandi \textit{naturali}.
				Ad esempio, si possono sommare due numeri naturali, di cui uno in AL e l'altro in EBX:
				\begin{lstlisting}[language=assembler,style=codestyle]	
MOV $5, $AL
MOV $100000, %EBX
AND $0x000000FF, $EAX
ADD %EAX, %EBX	
\end{lstlisting}
		\end{itemize} 
	\item \textbf{OR:} si usa per settare singoli bit di un operando.
		Ad esempio, si può settare il quarto bit di CL:
		\begin{lstlisting}[language=assembler,style=codestyle]	
OR $0x10, %CL	# =x10 = 00010000
\end{lstlisting}
	\item \textbf{XOR:}
		\begin{itemize}
			\item si usa per invertire singoli bit.
		Ad esempio, si può invertire il quinto bit del registro AH:
		\begin{lstlisting}[language=assembler,style=codestyle]	
XOR $0x20, %AH	# 0x20 = 00100000
\end{lstlisting}
	\item si usa per resettare registri.
		Ad esempio, si può resettare EAX come:
		\begin{lstlisting}[language=assembler,style=codestyle]	
XOR %EAX, %EAX	# equivale a dire MOV $0, %EAX, ma occupa 
								# 1 byte invece di 5
\end{lstlisting}
		\end{itemize}
\end{itemize}

\subsection{Istruzioni di controllo}
Le istruzioni di controllo permettono di alterare il flusso del programma, che altrimenti scorrerebbe normalmente in sequenza (le istruzioni vengono eseguite come vengono lette in memoria).

Conosciamo il ciclo fetch-execute: il processore carica un'istruzione, incrementa EIP, e la esegue.
Alcune istruzioni alterano il valore di EIP, implementando quindi alterazioni del flusso di esecuzione:
\begin{itemize}
	\item \textbf{Istruzioni di salto:} JMP, Jcon;
	\item \textbf{Istruzioni di gestione sottoprgrammi}: CALL, RET.
\end{itemize}

\subsubsection{JUMP}
\begin{itemize}
	\item \textbf{Formato:} \lstinline|JMP  \%EIP +/- displacement|, \lstinline|JMP *extended\_register|, \lstinline|JMP *memory|
	\item \textbf{Azione:} calcola un'indrizzo di salto e lo immette nel registro EIP. 
	\item \textbf{Flag:} nessuno. 
\end{itemize}

Solitamente le istruzioni di salto si riferiscono ad un nome simbolico, ed è quindi compito dell'assemblatore ricondurre la sintassi ad una delle forme sopra riportate.

\subsubsection{JUMP if CONDITION MET}
\begin{itemize}
	\item \textbf{Formato:} \lstinline|Jcon \%EIP +/- displacement|
	\item \textbf{Azione:} esamina il contenuto dei flag.
		Se da questo esame risulta che la condizione \textit{con} è soddisfatta, si comporta come \lstinline|JMP \%EIP +/- displacement|, altrimenti non fa nulla.
	\item \textbf{Flag:} nessuno. 
\end{itemize}

I prossimi paragrafi riguardano tutti i di condizione supportati.

\subsubsection{Condizioni sui flag}
Esistono le seguenti condizioni sui singoli flag:

\begin{table}[h!]
	\center \rowcolors{1}{white}{black!5}
	\begin{tabular} { c  p{10cm} }
		\bfseries Condizione & \bfseries Funzionamento \\
		\hline 
		JZ & Jump If Zero, la condizione è soddisfatta se ZF è impostato, ergo se il risultato dell'istruzione precedente è stato 0. \\ 
		JNZ & Jump If Not Zero, la condizione è soddisfatta se ZF non è impostato, ergo se il risultato dell'istruzione precedente non è stato 0. \\ 
		JC & Jump if Carry, la condizione è soddisfatta se CF è impostato. \\
		JNC & Jump if No Carry, la condizione è soddisfatta se CF non è impostato. \\ 
		JO & Jump if Overflow, la condizione è soddisfatta se OF è impostato. \\
		JNO & Jump if No Overflow, la condizione è soddisfatta se OF non è impostato. \\ 
		JS & Jump if Sign, la condizione è soddisfatta se SF è impostato. \\
		JNS & Jump if No Sign, la condizione è soddisfatta se SF non è impostato. \\ 
	\end{tabular}
\end{table}

\par\medskip
\noindent
\textbf{\textsf{Esempi}} \\
\begin{itemize}
	\item 
\begin{lstlisting}[language=assembler,style=codestyle]	
ADD %AX, %BX
JC ...
# continua
\end{lstlisting}
Se la somma dei contenuti di AX e BX presi come naturali non è rappresentabile su 16 bit, salta.

	\item 
\begin{lstlisting}[language=assembler,style=codestyle]	
ADD %AX, %BX
JO ...
# continua
\end{lstlisting}
Se la somma dei contenuti di AX e BX presi come interi non è rappresentabile su 16 bit, salta.

	\item 
\begin{lstlisting}[language=assembler,style=codestyle]	
SUB %AL, %BL
JS ...
# continua
\end{lstlisting}
Se la somma differenza dei contenuti di BL ed AL (in quest'ordine) presi come interi è negativa, salta.
\end{itemize}

\subsubsection{Condizioni sui naturali}
Esistono le seguenti condizioni sui confronti fra naturali:

\begin{table}[h!]
	\center \rowcolors{2}{white}{black!5}
	\begin{tabular} { c  p{10cm} }
		\bfseries Condizione & \bfseries Funzionamento \\
		\hline 
		JE & Jump if Equal, la condizione è soddisfatta se ZF contiene 1, cioè dopo CMP su due numeri uguali. \\
		JNE & Jump if Not Equal, la condizione è soddisfatta se ZF contiene 0, cioè dopo CMP su due numeri non uguali. \\ 
		JA & Jump if Above, la condizione è soddisfatta se CF contiene 0 e ZF contiene 0, cioè dopo CMP su un destinatario maggiore del sorgente. \\
		JAE & Jump if Above or Equal, la condizione è soddisfatta se CF contiene 0, cioè dopo CMP su un destinatario maggiore o uguale del sorgente. \\ 
		JB & Jump if Below, la condizione è soddisfatta se CF contiene 1, cioè dopo CMP su un destinatario minore del sorgente. \\
		JBE & Jump if Below or Equal, la condizione è soddisfatta se CF contiene 1 o ZF contiene 1, cioè dopo CMP su un destinatario minore o uguale del sorgente. \\ 
	\end{tabular}
\end{table}

Tutte queste condizioni seguono sempre una CMP, che aggiorna i flag in modo da permettere il confronto.
I risultati dei confronti possono sempre evincersi dai flag.

\par\medskip
\noindent
\textbf{\textsf{Esempi}} \\
\begin{itemize}
	\item 
\begin{lstlisting}[language=assembler,style=codestyle]	
CMP %AX, %BX
JAE ...
# continua
\end{lstlisting}
Se BX è maggiore o uguale di AX, presi come naturali, salta.

	\item 
\begin{lstlisting}[language=assembler,style=codestyle]	
CMP %EDX, %ECX
JB ...
# continua
\end{lstlisting}
Se ECX è minore stretto di EDX, presi come naturali, salta.
\end{itemize}

\subsubsection{Condizioni sugli interi}
Esistono le seguenti condizioni sui confronti fra interi:

\begin{table}[h!]
	\center \rowcolors{2}{white}{black!5}
	\begin{tabular} { c  p{10cm} }
		\bfseries Condizione & \bfseries Funzionamento \\
		\hline 
		JE & Jump if Equal, la condizione è soddisfatta se ZF contiene 1, cioè dopo CMP su due numeri uguali. \\
		JNE & Jump if Not Equal, la condizione è soddisfatta se ZF contiene 0, cioè dopo CMP su due numeri non uguali. \\ 
		JG & Jump if Greater, la condizione è soddisfatta se ZF contiene 0 e se SF è uguale a OF, cioè dopo CMP su un destinatario maggiore del sorgente. \\
		JGE & Jump if Greater or Equal, la condizione è soddisfatta se SF è uguale a OF, cioè dopo CMP su un destinatario maggiore o uguale del sorgente. \\ 
		JL & Jump if Less, la condizione è soddisfatta se SF è diverso da OF, cioè dopo CMP su un destinatario minore del sorgente. \\
		JLE & Jump if Less or Equal, la condizione è soddisfatta se ZF contiene 1 o se Sf è diverso da OF, cioè dopo CMP su un destinatario minore o uguale del sorgente. \\ 
	\end{tabular}
\end{table}

Come prima, queste operazioni seguono sempre una CMP ed evincono il risultato del confronto dai flag.

\par\medskip
\noindent
\textbf{\textsf{Esempi}} \\
\begin{itemize}
	\item 
\begin{lstlisting}[language=assembler,style=codestyle]	
CMP %AX, %BX
JGE ...
# continua
\end{lstlisting}
Se BX è maggiore o uguale di AX, presi come interi, salta.

	\item 
\begin{lstlisting}[language=assembler,style=codestyle]	
CMP %EDX, %ECX
JL ...
# continua
\end{lstlisting}
Se ECX è minore stretto di EDX, presi come interi, salta.
\end{itemize}
\end{document}

\end{document}