
\documentclass[a4paper,11pt]{article}
\usepackage[a4paper, margin=8em]{geometry}

% usa i pacchetti per la scrittura in italiano
\usepackage[french,italian]{babel}
\usepackage[T1]{fontenc}
\usepackage[utf8]{inputenc}
\frenchspacing 

% usa i pacchetti per la formattazione matematica
\usepackage{amsmath, amssymb, amsthm, amsfonts}

% usa altri pacchetti
\usepackage{gensymb}
\usepackage{hyperref}
\usepackage{standalone}

\usepackage{colortbl}

\usepackage{xstring}
\usepackage{karnaugh-map}

% imposta il titolo
\title{Appunti Reti Logiche}
\author{Luca Seggiani}
\date{2025}

% imposta lo stile
% usa helvetica
\usepackage[scaled]{helvet}
% usa palatino
\usepackage{palatino}
% usa un font monospazio guardabile
\usepackage{lmodern}

\renewcommand{\rmdefault}{ppl}
\renewcommand{\sfdefault}{phv}
\renewcommand{\ttdefault}{lmtt}

% circuiti
\usepackage{circuitikz}
\usetikzlibrary{babel}

% testo cerchiato
\newcommand*\circled[1]{\tikz[baseline=(char.base)]{
            \node[shape=circle,draw,inner sep=2pt] (char) {#1};}}

% disponi il titolo
\makeatletter
\renewcommand{\maketitle} {
	\begin{center} 
		\begin{minipage}[t]{.8\textwidth}
			\textsf{\huge\bfseries \@title} 
		\end{minipage}%
		\begin{minipage}[t]{.2\textwidth}
			\raggedleft \vspace{-1.65em}
			\textsf{\small \@author} \vfill
			\textsf{\small \@date}
		\end{minipage}
		\par
	\end{center}

	\thispagestyle{empty}
	\pagestyle{fancy}
}
\makeatother

% disponi teoremi
\usepackage{tcolorbox}
\newtcolorbox[auto counter, number within=section]{theorem}[2][]{%
	colback=blue!10, 
	colframe=blue!40!black, 
	sharp corners=northwest,
	fonttitle=\sffamily\bfseries, 
	title=Teorema~\thetcbcounter: #2, 
	#1
}

% disponi definizioni
\newtcolorbox[auto counter, number within=section]{definition}[2][]{%
	colback=red!10,
	colframe=red!40!black,
	sharp corners=northwest,
	fonttitle=\sffamily\bfseries,
	title=Definizione~\thetcbcounter: #2,
	#1
}

% disponi codice
\usepackage{listings}
\usepackage[table]{xcolor}

\definecolor{codegreen}{rgb}{0,0.6,0}
\definecolor{codegray}{rgb}{0.5,0.5,0.5}
\definecolor{codepurple}{rgb}{0.58,0,0.82}
\definecolor{backcolour}{rgb}{0.95,0.95,0.92}

\lstdefinestyle{codestyle}{
		backgroundcolor=\color{black!5}, 
		commentstyle=\color{codegreen},
		keywordstyle=\bfseries\color{magenta},
		numberstyle=\sffamily\tiny\color{black!60},
		stringstyle=\color{green!50!black},
		basicstyle=\ttfamily\footnotesize,
		breakatwhitespace=false,         
		breaklines=true,                 
		captionpos=b,                    
		keepspaces=true,                 
		numbers=left,                    
		numbersep=5pt,                  
		showspaces=false,                
		showstringspaces=false,
		showtabs=false,                  
		tabsize=2
}

\lstdefinestyle{shellstyle}{
		backgroundcolor=\color{black!5}, 
		basicstyle=\ttfamily\footnotesize\color{black}, 
		commentstyle=\color{black}, 
		keywordstyle=\color{black},
		numberstyle=\color{black!5},
		stringstyle=\color{black}, 
		showspaces=false,
		showstringspaces=false, 
		showtabs=false, 
		tabsize=2, 
		numbers=none, 
		breaklines=true
}


\lstdefinelanguage{assembler}{ 
  keywords={AAA, AAD, AAM, AAS, ADC, ADCB, ADCW, ADCL, ADD, ADDB, ADDW, ADDL, AND, ANDB, ANDW, ANDL,
        ARPL, BOUND, BSF, BSFL, BSFW, BSR, BSRL, BSRW, BSWAP, BT, BTC, BTCB, BTCW, BTCL, BTR, 
        BTRB, BTRW, BTRL, BTS, BTSB, BTSW, BTSL, CALL, CBW, CDQ, CLC, CLD, CLI, CLTS, CMC, CMP,
        CMPB, CMPW, CMPL, CMPS, CMPSB, CMPSD, CMPSW, CMPXCHG, CMPXCHGB, CMPXCHGW, CMPXCHGL,
        CMPXCHG8B, CPUID, CWDE, DAA, DAS, DEC, DECB, DECW, DECL, DIV, DIVB, DIVW, DIVL, ENTER,
        HLT, IDIV, IDIVB, IDIVW, IDIVL, IMUL, IMULB, IMULW, IMULL, IN, INB, INW, INL, INC, INCB,
        INCW, INCL, INS, INSB, INSD, INSW, INT, INT3, INTO, INVD, INVLPG, IRET, IRETD, JA, JAE,
        JB, JBE, JC, JCXZ, JE, JECXZ, JG, JGE, JL, JLE, JMP, JNA, JNAE, JNB, JNBE, JNC, JNE, JNG,
        JNGE, JNL, JNLE, JNO, JNP, JNS, JNZ, JO, JP, JPE, JPO, JS, JZ, LAHF, LAR, LCALL, LDS,
        LEA, LEAVE, LES, LFS, LGDT, LGS, LIDT, LMSW, LOCK, LODSB, LODSD, LODSW, LOOP, LOOPE,
        LOOPNE, LSL, LSS, LTR, MOV, MOVB, MOVW, MOVL, MOVSB, MOVSD, MOVSW, MOVSX, MOVSXB,
        MOVSXW, MOVSXL, MOVZX, MOVZXB, MOVZXW, MOVZXL, MUL, MULB, MULW, MULL, NEG, NEGB, NEGW,
        NEGL, NOP, NOT, NOTB, NOTW, NOTL, OR, ORB, ORW, ORL, OUT, OUTB, OUTW, OUTL, OUTSB, OUTSD,
        OUTSW, POP, POPL, POPW, POPB, POPA, POPAD, POPF, POPFD, PUSH, PUSHL, PUSHW, PUSHB, PUSHA, 
				PUSHAD, PUSHF, PUSHFD, RCL, RCLB, RCLW, MOVSL, MOVSB, MOVSW, STOSL, STOSB, STOSW, LODSB, LODSW,
				LODSL, INSB, INSW, INSL, OUTSB, OUTSL, OUTSW
        RCLL, RCR, RCRB, RCRW, RCRL, RDMSR, RDPMC, RDTSC, REP, REPE, REPNE, RET, ROL, ROLB, ROLW,
        ROLL, ROR, RORB, RORW, RORL, SAHF, SAL, SALB, SALW, SALL, SAR, SARB, SARW, SARL, SBB,
        SBBB, SBBW, SBBL, SCASB, SCASD, SCASW, SETA, SETAE, SETB, SETBE, SETC, SETE, SETG, SETGE,
        SETL, SETLE, SETNA, SETNAE, SETNB, SETNBE, SETNC, SETNE, SETNG, SETNGE, SETNL, SETNLE,
        SETNO, SETNP, SETNS, SETNZ, SETO, SETP, SETPE, SETPO, SETS, SETZ, SGDT, SHL, SHLB, SHLW,
        SHLL, SHLD, SHR, SHRB, SHRW, SHRL, SHRD, SIDT, SLDT, SMSW, STC, STD, STI, STOSB, STOSD,
        STOSW, STR, SUB, SUBB, SUBW, SUBL, TEST, TESTB, TESTW, TESTL, VERR, VERW, WAIT, WBINVD,
        XADD, XADDB, XADDW, XADDL, XCHG, XCHGB, XCHGW, XCHGL, XLAT, XLATB, XOR, XORB, XORW, XORL},
  keywordstyle=\color{blue}\bfseries,
  ndkeywordstyle=\color{darkgray}\bfseries,
  identifierstyle=\color{black},
  sensitive=false,
  comment=[l]{\#},
  morecomment=[s]{/*}{*/},
  commentstyle=\color{purple}\ttfamily,
  stringstyle=\color{red}\ttfamily,
  morestring=[b]',
  morestring=[b]"
}

\lstset{language=assembler, style=codestyle}

% disponi sezioni
\usepackage{titlesec}

\titleformat{\section}
	{\sffamily\Large\bfseries} 
	{\thesection}{1em}{} 
\titleformat{\subsection}
	{\sffamily\large\bfseries}   
	{\thesubsection}{1em}{} 
\titleformat{\subsubsection}
	{\sffamily\normalsize\bfseries} 
	{\thesubsubsection}{1em}{}

% tikz
\usepackage{tikz}

% float
\usepackage{float}

% grafici
\usepackage{pgfplots}
\pgfplotsset{width=10cm,compat=1.9}

% disponi alberi
\usepackage{forest}

\forestset{
	rectstyle/.style={
		for tree={rectangle,draw,font=\large\sffamily}
	},
	roundstyle/.style={
		for tree={circle,draw,font=\large}
	}
}

% disponi algoritmi
\usepackage{algorithm}
\usepackage{algorithmic}
\makeatletter
\renewcommand{\ALG@name}{Algoritmo}
\makeatother

% disponi numeri di pagina
\usepackage{fancyhdr}
\fancyhf{} 
\fancyfoot[L]{\sffamily{\thepage}}

\makeatletter
\fancyhead[L]{\raisebox{1ex}[0pt][0pt]{\sffamily{\@title \ \@date}}} 
\fancyhead[R]{\raisebox{1ex}[0pt][0pt]{\sffamily{\@author}}}
\makeatother

\begin{document}

\pagestyle{fancy}
\thispagestyle{empty}
\renewcommand{\thispagestyle}[1]{}

\maketitle
\documentclass[a4paper,11pt]{article}
\usepackage[a4paper, margin=8em]{geometry}

% usa i pacchetti per la scrittura in italiano
\usepackage[french,italian]{babel}
\usepackage[T1]{fontenc}
\usepackage[utf8]{inputenc}
\frenchspacing 

% usa i pacchetti per la formattazione matematica
\usepackage{amsmath, amssymb, amsthm, amsfonts}

% usa altri pacchetti
\usepackage{gensymb}
\usepackage{hyperref}
\usepackage{standalone}

% imposta il titolo
\title{Appunti Reti Logiche}
\author{Luca Seggiani}
\date{24-09-24}

% imposta lo stile
% usa helvetica
\usepackage[scaled]{helvet}
% usa palatino
\usepackage{palatino}
% usa un font monospazio guardabile
\usepackage{lmodern}

\renewcommand{\rmdefault}{ppl}
\renewcommand{\sfdefault}{phv}
\renewcommand{\ttdefault}{lmtt}

% disponi teoremi
\usepackage{tcolorbox}
\newtcolorbox[auto counter, number within=section]{theorem}[2][]{%
	colback=blue!10, 
	colframe=blue!40!black, 
	sharp corners=northwest,
	fonttitle=\sffamily\bfseries, 
	title=Teorema~\thetcbcounter: #2, 
	#1
}

% disponi definizioni
\newtcolorbox[auto counter, number within=section]{definition}[2][]{%
	colback=red!10,
	colframe=red!40!black,
	sharp corners=northwest,
	fonttitle=\sffamily\bfseries,
	title=Definizione~\thetcbcounter: #2,
	#1
}

% disponi codice
\usepackage{listings}
\usepackage[table]{xcolor}

\lstdefinestyle{codestyle}{
		backgroundcolor=\color{black!5}, 
		commentstyle=\color{codegreen},
		keywordstyle=\bfseries\color{magenta},
		numberstyle=\sffamily\tiny\color{black!60},
		stringstyle=\color{green!50!black},
		basicstyle=\ttfamily\footnotesize,
		breakatwhitespace=false,         
		breaklines=true,                 
		captionpos=b,                    
		keepspaces=true,                 
		numbers=left,                    
		numbersep=5pt,                  
		showspaces=false,                
		showstringspaces=false,
		showtabs=false,                  
		tabsize=2
}

\lstdefinestyle{shellstyle}{
		backgroundcolor=\color{black!5}, 
		basicstyle=\ttfamily\footnotesize\color{black}, 
		commentstyle=\color{black}, 
		keywordstyle=\color{black},
		numberstyle=\color{black!5},
		stringstyle=\color{black}, 
		showspaces=false,
		showstringspaces=false, 
		showtabs=false, 
		tabsize=2, 
		numbers=none, 
		breaklines=true
}

\lstdefinelanguage{javascript}{
	keywords={typeof, new, true, false, catch, function, return, null, catch, switch, var, if, in, while, do, else, case, break},
	keywordstyle=\color{blue}\bfseries,
	ndkeywords={class, export, boolean, throw, implements, import, this},
	ndkeywordstyle=\color{darkgray}\bfseries,
	identifierstyle=\color{black},
	sensitive=false,
	comment=[l]{//},
	morecomment=[s]{/*}{*/},
	commentstyle=\color{purple}\ttfamily,
	stringstyle=\color{red}\ttfamily,
	morestring=[b]',
	morestring=[b]"
}

% disponi sezioni
\usepackage{titlesec}

\titleformat{\section}
	{\sffamily\Large\bfseries} 
	{\thesection}{1em}{} 
\titleformat{\subsection}
	{\sffamily\large\bfseries}   
	{\thesubsection}{1em}{} 
\titleformat{\subsubsection}
	{\sffamily\normalsize\bfseries} 
	{\thesubsubsection}{1em}{}

% disponi alberi
\usepackage{forest}

\forestset{
	rectstyle/.style={
		for tree={rectangle,draw,font=\large\sffamily}
	},
	roundstyle/.style={
		for tree={circle,draw,font=\large}
	}
}

% disponi algoritmi
\usepackage{algorithm}
\usepackage{algorithmic}
\makeatletter
\renewcommand{\ALG@name}{Algoritmo}
\makeatother

% disponi numeri di pagina
\usepackage{fancyhdr}
\fancyhf{} 
\fancyfoot[L]{\sffamily{\thepage}}

\makeatletter
\fancyhead[L]{\raisebox{1ex}[0pt][0pt]{\sffamily{\@title \ \@date}}} 
\fancyhead[R]{\raisebox{1ex}[0pt][0pt]{\sffamily{\@author}}}
\makeatother

\begin{document}
% sezione (data)
\section{Lezione del 24-09-24}

% stili pagina
\thispagestyle{empty}
\pagestyle{fancy}

% testo
\subsection{Introduzione}
Il corso di reti logiche tratta di:
\begin{enumerate}
	\item \textbf{Linguaggio assembler:} come scrivere programmi semplici, come avviene la compilazione in linguaggio macchina;
	\item \textbf{Reti logiche:} reti combinatorie, reti combinatorie per l'aritmetica, reti sequenziali asincrone e sincronizzate;
	\item \textbf{Microprogammazione:} reti sequenziali sincronizzate, come realizzare una rete logica da specifiche. 
		"Micro" qui sta per \textit{hardware};
	\item \textbf{Il calcolatore:} processore, interfacce comuni e convertitori.
\end{enumerate}

\subsubsection{Introduzione alle reti logiche}
Si parla di reti \textit{logiche} in quanto si guarda all'hardware da una prospettiva funzionale, indipendente dalla sua tecnologia.
Ad esempio, una porta NOR sarà implementata con determinati circuiti, ma tutto ciò che interessa a questo corso è come si comporta logicamente: $ y = 1 \Leftrightarrow A = B = 0 $.

\subsection{Programmazione assembly}
Il nome corretto del linguaggio sarebbe Assembly, ma noi lo chiameremo Assembler per ragioni storiche.
L'assembler è il linguaggio con cui si scrivono le istruzioni eseguite dal processore.
Il processore implementa effettivamente un ciclo fetch-execute dove preleva la prossima istruzione macchina (in assembler) dalla memoria e la esegue.

\subsubsection{Linguaggio macchina}
Il linguaggio macchina (LM) è dato dal contenuto effettivo della memoria che contiene le istruzioni, ergo una sequenza di zero e uno.
Il linguaggio assembler adotta una sintassi simbolica per il linguaggio macchina: ad esempio, \texttt{MOV \%AX, \%BX}.

Il processo di trasformazione dall'assembler all'LM si chiama \textbf{assemblaggio}, mentre il processo di traasformazione da un linguaggio ad alto livello all'assembler si chiama \textbf{compilazione}.

\subsubsection{Generalità sull'assembler}
Si dice che assembler è un linguaggio a basso livello.
Mancano i costrutti a cui siamo abituati da i linguaggi di alto livello:
\begin{enumerate}
	\item Non esistono costrutti di flow control (for, if-else, ecc...), tutto si fa con istruzioni di salto.
	\item Non esistono tipi variabile: gli operandi sono stringhe di bit che si riferiscono a locazioni in memoria.
\end{enumerate}

Inoltre, l'assembler è strettamente legato all'hardware, ed è specifico per ogni processore.
Noi vedremo l'assembler dei processori della famiglia Intel x86, che non è uguale all'assembler dei processori Arm Cortex, ecc...
Questo rende il codice in assembler mai portatile.
Fatta questa precisazione, possiamo dire che i principi generali restano comunque validi fra famiglie di processori diverse.

Esiste ancora oggi una nicchia di utilizzo del linguaggio assembler: quello dello sviluppo di sistemi embedded.
Inoltre, il linguaggio ha un importante significato didattico e culturale.

\subsection{Schema a blocchi del calcolatore}

# illustrazione modello funzionale

Un calcolatore è formato, in linea generale, da una rete di interconnessione (bus) che collega fra di loro:
\begin{itemize}
	\item Interfacce che comunicano con dispositivi;
	\item La memoria principale che contiene dati e programmi;
	\item Il processore, che esegue il ciclo fetch-execute. Possiamo aggiungere che ogni processore, oggi, contiene almeno due blocchi:
		\begin{itemize}
			\item L'\textbf{ALU}, Arithmetic Logic Unit, che si occupa di calcoli aritmetici su numeri interi (interpretando le stringhe di bit come numeri naturali o interi in complemento a 2) e operazioni logiche;
			\item L'\textbf{FPU}, Floating Point Unit, che si occupa dei numeri a virgola mobile.
		\end{itemize}
\end{itemize}

\subsection{Riassunto di rappresentazione dell'informazione}
\subsubsection{Numeri naturali}
$N$ bit rappresentano $2^N$ naturali sull'intervallo $[0, 2^N - 1]$, ovvero:

$$
b_{N-1}, b_{N-2}, ... , b_1, b_0 \Leftrightarrow X = \sum_{i=0}^{N-1} b_i \cdot 2^i
$$

Il bit più a sinistra è il Most Significant Bit (MSB) (nell'esempio $b_{N-1}$), quello più a destra il Least Significant Bit (LSD) (nell'esempio $b_0$).
Le cifre in base due a partire da un numero in un'altra base si trovano con l'algoritmo div-mod.

\subsubsection{Numeri interi in complemento a due}
$N$ bit rappresentano $2^N$ interi sull'intervallo $ [-2^{N-1}, 2^{N-1} - 1]$, ovvero:

$$
X = 
	\begin{cases}
		x \quad \quad \quad \ \ x \geq 0 \\
		2^N + x \quad x < 0
	\end{cases}
$$

oppure, usando l'operatore modulo:
$$ |x|_2N $$ # poco chiaro, copiaci fondamenti

La legge inversa, che mi permette di trovare l'intero $x$ dalla sua rappresentazione $X$, è:

$$
x =
	\begin{cases}
		X \quad \quad \quad \quad \  X_{N-1} = 0 \\
		-(\bar{X} + 1) \quad X_{N-1} = 1
	\end{cases}
$$

dove la barra rappresenta l'operazione complemento.

\subsubsection{Notazione esadecimale}
Scrivere lunghe stringhe binarie diventa velocemente complicato. 
Per questo si adotta una notazione esadecimale per stringhe di 4 bit ($[0, 15]$):

# riporta tabella stringhe esadecimali

A questo punto, possiamo denotare qualsiasi stringa binaria come una lista di numeri esadecimali prefissi da \texttt{0x} (che serve ad indicare la rappresentazione esadecimale stessa), ad esempio \texttt{0xC1}.

\subsection{Struttura del calcolatore}
\subsubsection{Spazio di memoria}
La memoria del calcolatore, vista dal programmatore assembler, è uno spazio lineare di $2^{32}$ (su calcolatori a 32 bit) locazioni (celle) di memoria, dalla capacità di un btye ciascuna.
Ogni cella è quindi identificata da un numero di 32 bit, detto \textbf{indirizzo}.

\par\smallskip

Lo spazio di memoria è in larga parte implementato attraverso Random Access Memory (RAM), ovvero memoria volatile.
Solo una piccola parte dello spazio è implementata attraverso Read Only Memory (ROM), ovvero memoria permanente, che contiene le istruzioni da eseguire al reset.

\subsubsection{Accesso allo spazio di memoria}
Il processore può accedere (leggere/scrivere) a:
\begin{itemize}
	\item Singole locazioni (byte) da 8 bit;
	\item Doppie locazioni (word) da 16 bit;
	\item Quadruple locazioni (double word) da 32 bit.
\end{itemize}

Per gli accessi 16/32 bit si usa l'indirizzo più piccolo delle 2/4 locazioni.
Si ricorda che l'indirizzo più grande contiene i bit più significativi.

Gli indirizzi di memoria assembler sono solo simbolici, e vengono tradotti dall'assemblatore, e in parte runtime.
Questo significa che non si può accedere a memoria appartenente al sistema operativo, o memoria fuori dai limiti fisici del sistema, ecc...

\subsubsection{Spazio di Input/Output}
Lo spazio di Input/Output è formato da $2^{16}$, ovvero 64k, locazioni o \textbf{porte}.
Ogni porta ha una capacità di un byte ed è indirizzata da un numero di 16 bit.

Il processore accede alle porte attraverso operazioni particolari di lettura o scrittura (in o out).
Spesso le porte sono configurate per un solo tipo di operazione: sola lettura o sola scrittura.

\par\smallskip

Le locazioni di memoria sono solitamente identifiche fra di loro, le porte di I/O no.
Indirizzi diversi significano dispositivi diversi, e si rende quindi necessario conoscere fisicamente gli indirizzi.

\subsubsection{Processore}
Il processore è dotato di una memoria interna formata da locazioni di memoria da 32 bit (\textbf{registri}).
Questi si dividono in registri \textbf{generali}, riservati alle elaborazioni, e \textbf{di stato}, riservati a compiti speciali.

# sii piu chiaro

I 16 bit bassi dei registri sono riferibili autonomamente (retro-compatibili).
DI alcuni registri si possono riferire parti ad 8 bit.

\subsubsection{Registri generali}
Alcuni registri vengono utilizzati per particolari funzioni, per motivi storici.
\begin{itemize}
	\item EAX (AX, AH od AL) è utilizzato da alcune istruzioni aritmetiche per contenere operandi e risultati. Viene detto \textbf{accumulatore}.
	\item ESI, EDI, EBX e EBP sono a volte utilizzati come registri puntatore, base (B) e indice (I).
		\begin{itemize}
			\item ESI
			\item EDI
			\item EBX veniva usato come indirizzo di base per l'accesso in memoria. Viene solitamente detto \textbf{base}.
			\item EBP # guarda slide!
		\end{itemize}
	\item ECX è utilizzato come contatore nei cicli. Vienedetto \textbf{contatore}.
	\item EDX è utilizzato come operando di operazioni aritmetiche. Viene detto \textbf{data}.
	\item ESP è utilizzato per indirizzare la \textbf{pila} o \textbf{stack}, ovvero una parte di memoria con disciplina LIFO che serve a gestire sottoprogrammi.
\end{itemize}

\subsubsection{Registri di stato}
L'EIP viene detto instruction pointer, o \textbf{program counter}.
Viene usato per contenere l'indirizzo della locazione dalla quale sarà prelevata la prossima istruzione da eseguire.
Il contenuto dell'EIP è fissato al reset iniziale, e impostato sulla prima istruzione da eseguire (in memoria ROM).

Possiamo quindi dire che il ciclo fetch-loop si svolge come segue:
\begin{itemize}
	\item Il processore preleva dalla memoria, all'indirizzo EIP, una nuova istruzione;
	\item Incrementa EIP del numero di byte dell'istruzione prelevata;
	\item Esegue l'istruzione e ripete.
\end{itemize}

Da questo si ha che le istruzioni in memoria vengono eseguite sequenzialmente nell'ordine in cui sono incontrate, a meno che non si definiscano salti attraverso altre determinate istruzioni.

\end{document}



\documentclass[a4paper,11pt]{article}
\usepackage[a4paper, margin=8em]{geometry}

% usa i pacchetti per la scrittura in italiano
\usepackage[french,italian]{babel}
\usepackage[T1]{fontenc}
\usepackage[utf8]{inputenc}
\frenchspacing 

% usa i pacchetti per la formattazione matematica
\usepackage{amsmath, amssymb, amsthm, amsfonts}

% usa altri pacchetti
\usepackage{gensymb}
\usepackage{hyperref}
\usepackage{standalone}

% imposta il titolo
\title{Appunti Reti Logiche}
\author{Luca Seggiani}
\date{25-09-24}

% imposta lo stile
% usa helvetica
\usepackage[scaled]{helvet}
% usa palatino
\usepackage{palatino}
% usa un font monospazio guardabile
\usepackage{lmodern}

\renewcommand{\rmdefault}{ppl}
\renewcommand{\sfdefault}{phv}
\renewcommand{\ttdefault}{lmtt}

% disponi teoremi
\usepackage{tcolorbox}
\newtcolorbox[auto counter, number within=section]{theorem}[2][]{%
	colback=blue!10, 
	colframe=blue!40!black, 
	sharp corners=northwest,
	fonttitle=\sffamily\bfseries, 
	title=Teorema~\thetcbcounter: #2, 
	#1
}

% disponi definizioni
\newtcolorbox[auto counter, number within=section]{definition}[2][]{%
	colback=red!10,
	colframe=red!40!black,
	sharp corners=northwest,
	fonttitle=\sffamily\bfseries,
	title=Definizione~\thetcbcounter: #2,
	#1
}

% disponi codice
\usepackage{listings}
\usepackage[table]{xcolor}

\lstdefinestyle{codestyle}{
		backgroundcolor=\color{black!5}, 
		commentstyle=\color{codegreen},
		keywordstyle=\bfseries\color{magenta},
		numberstyle=\sffamily\tiny\color{black!60},
		stringstyle=\color{green!50!black},
		basicstyle=\ttfamily\footnotesize,
		breakatwhitespace=false,         
		breaklines=true,                 
		captionpos=b,                    
		keepspaces=true,                 
		numbers=left,                    
		numbersep=5pt,                  
		showspaces=false,                
		showstringspaces=false,
		showtabs=false,                  
		tabsize=2
}

\lstdefinestyle{shellstyle}{
		backgroundcolor=\color{black!5}, 
		basicstyle=\ttfamily\footnotesize\color{black}, 
		commentstyle=\color{black}, 
		keywordstyle=\color{black},
		numberstyle=\color{black!5},
		stringstyle=\color{black}, 
		showspaces=false,
		showstringspaces=false, 
		showtabs=false, 
		tabsize=2, 
		numbers=none, 
		breaklines=true
}

\lstdefinelanguage{javascript}{
	keywords={typeof, new, true, false, catch, function, return, null, catch, switch, var, if, in, while, do, else, case, break},
	keywordstyle=\color{blue}\bfseries,
	ndkeywords={class, export, boolean, throw, implements, import, this},
	ndkeywordstyle=\color{darkgray}\bfseries,
	identifierstyle=\color{black},
	sensitive=false,
	comment=[l]{//},
	morecomment=[s]{/*}{*/},
	commentstyle=\color{purple}\ttfamily,
	stringstyle=\color{red}\ttfamily,
	morestring=[b]',
	morestring=[b]"
}

% disponi sezioni
\usepackage{titlesec}

\titleformat{\section}
	{\sffamily\Large\bfseries} 
	{\thesection}{1em}{} 
\titleformat{\subsection}
	{\sffamily\large\bfseries}   
	{\thesubsection}{1em}{} 
\titleformat{\subsubsection}
	{\sffamily\normalsize\bfseries} 
	{\thesubsubsection}{1em}{}

% disponi alberi
\usepackage{forest}

\forestset{
	rectstyle/.style={
		for tree={rectangle,draw,font=\large\sffamily}
	},
	roundstyle/.style={
		for tree={circle,draw,font=\large}
	}
}

% disponi algoritmi
\usepackage{algorithm}
\usepackage{algorithmic}
\makeatletter
\renewcommand{\ALG@name}{Algoritmo}
\makeatother

% disponi numeri di pagina
\usepackage{fancyhdr}
\fancyhf{} 
\fancyfoot[L]{\sffamily{\thepage}}

\makeatletter
\fancyhead[L]{\raisebox{1ex}[0pt][0pt]{\sffamily{\@title \ \@date}}} 
\fancyhead[R]{\raisebox{1ex}[0pt][0pt]{\sffamily{\@author}}}
\makeatother

\begin{document}
% sezione (data)
\section{Lezione del 25-09-24}

% stili pagina
\thispagestyle{empty}
\pagestyle{fancy}

% testo
\subsection{Introduzione all'Assembler}

\subsubsection{Codifica macchina e codifica mnemonica}
Possiamo adottare 2 metodi per codificare le istruzioni eseguite dal processore:

\begin{itemize}
	\item \textbf{Codifica macchina:} la serie di zeri e di uni che codificano, nel linguaggio del processore, le operazioni che esegue.
		Il formato macchina è, nell'architettura che ci interessa, il seguente:

		\begin{table}[h!]
			\center \rowcolors{2}{white}{black!10}
			\begin{tabular} { c | c | c }
				\bfseries Segmento & \bfseries Byte & \bfseries Funzione \\
				\hline 
				I Prefix (Instruction Prefix) & 0/1 byte & Usato per modificare l'istruzione \\ 
				O Prefix (Operand-size prefix) & 0/1 byte & Usato per modificare la dimensione degli operandi \\
				Opcode &
			\end{tabular}
		\end{table}

	\item \textbf{Codifica mnemonica:} un modo \textbf{simbolico} per riferirsi alle istruzioni.
		Un'istruzione può quindi essere semplicemente: \texttt{MOV \%EAX, 0x01F4E39}.
\end{itemize}

Il linguaggio assembler usa la codifica mnemonica delle istruzioni, e dispone di sovrastrutture sintattiche che lo rendono più comprensibile agli umani.
Ad esempio, permette l'uso di nomi simbolici per locazioni di memoria: \texttt{MOV \%EAX, pippo}.

\subsubsection{Istruzioni in codifica mnemonica}
Un'istruzione ha 3 campi:
\begin{itemize}
	\item \textbf{Codice operativo:} stabilisce quale operazione eseguire;
	\item \textbf{Suffisso di lunghezza:} stabilisce la lunghezza (che può variare) degli operandi;	
	\item \textbf{Operandi:} gli operandi su cui si applica l'operazione. 
		Possono essere contenuti in registri, in celle di memoria, nelle porte I/O o direttamente nell'istruzione (\textbf{costanti}).
\end{itemize}

Il suffisso di lunghezza può essere omesso quando è chiaro (essenzialmente quando si usa un registro).

Sintatticamente la struttura è \texttt{OPCODEsuffix source, dest}, che diventa qualcosa come \texttt{ADD \%BX, pluto}.
Questa istruzione effettua l'operazione \texttt{ADD} (aggiungi), aggiungendo al registro \texttt{BX} ciò che è contenuto nel simbolo \texttt{pluto}.

\par\medskip
\noindent
\textsf{\textbf{Operandi di istruzioni}} \\
Le istruzioni ammettono 0, 1 o 2 operandi.
Quando sono 2, il primo operando si chiama \textbf{sorgente} e il secondo \textbf{destinatario}, e solitamente hanno la stessa lunghezza.
Quando è 1, l'operando può essere sia sorgente che destinatario a seconda dell'istruzione.

\subsubsection{Primo esempio di programma}

Si presenta un programma per contare il numero di uno trovati dalla locazione \texttt{0x00000100} a \texttt{0x0000010i3}e scriverlo nella locazione \texttt{0x00000104}. 

\begin{lstlisting}[style=codestyle]	
MOVB $0x00, %CL					% sposta $0x00 in %CL
MOVL 0x00000100, %EAX		% sposta 32 bit da 0x00000100 a %EAX
CMPL $0x00000000, %EAX	% confronta 32 bit di 0 con il registro %EAX
JE   %EIP+$0x07					% salta se uguale a %EIP+$0x07, 
												%	ergo 0x0000020C + 0x07 = 0x00000213
SHRL %EAX								% trasla a destra %EAX
ADCB $0x00, %CL					% aggiungi a %CL 0 + carry 
JMP  %EIP-$0x0C					% salta incondizionato a %EIP-$0x0C,
												% ergo 0x00000213 - 0x0C = 0x00000207
MOVB %CL, 0x00000104		% sposta byte da %CL a 0x00000104
\end{lstlisting}

Il programma svolge i seguenti passi:
\begin{algorithm}
\caption{Conta 0}
\begin{algorithmic}
	\STATE Inizializza il registro CL (Counter Low) a 0
	\STATE Sposta i 32 bit da \texttt{0x00000000} a \texttt{0x00000103} in EAX
	\WHILE{true}	
		\IF{EAX è vuoto (tutti zeri)}
			\STATE Salta all'ultima istruzione
		\ENDIF
		\STATE Sposta EAX a destra
		\STATE Aggiungi il flag carry (che prende il valore rimosso da EAX) al registro CL
	\ENDWHILE
	\STATE Sposta il byte in CL nella locazione \texttt{0x00000104}
\end{algorithmic}
\end{algorithm}

\subsubsection{Istruzioni assembler}
Le istruzioni assembler si dividono in:
\begin{itemize}
	\item \textbf{Operative:} ovvero quelle che svolgono qualche operazione (ADD, SHR, MOV, CMP, ....);
	\item \textbf{Di controllo}: cioè che si occupano di altreare il flusso del programma (JMP, JE, ecc...).
\end{itemize}

\par\medskip
\noindent
\textsf{\textbf{Indirizzamento delle istruzioni operative}} \\
Le istruzioni operative si indirizzano attraverso l'\textbf{OPCODE} (codice operazione, ADD, MOV, ecc...), seguito da un suffisso (\textbf{B}, \textit{byte} da 8 bit, \textbf{W}, \textit{word} da 16 bit o \textbf{L}, \textit{long} da 32 bit) che può essere omesso, e gli indirizzi sorgente e destinazione.

\begin{itemize}
	\item 
Si possono \textbf{indirizzare i registri} sia come sorgenti che come destinatari, ovvero gli 8 registri generali da 32 bit, gli 8 registri generali da 16 bit, e gli 8 registri generali da 8 bit (disponibili solo sui registri A, B, C e D).
Bisogna precedere i nomi dei registri con \textbf{\%}.
	\item
Si può avere \textbf{indirizzamento immediato}, ovvero di costanti preceduti da \textbf{\$}, solo sull'operando sogente.
	\item
		Si può \textbf{indirizzare la memoria}, ma solo da sorgente o solo da destinatario, specificando un'indirizzo di memoria da 32 bit.
Ergo non posso scrivere:

\begin{lstlisting}[style=codestyle]	
MOVB pippo, pluto
\end{lstlisting}

ma devo scrivere:

\begin{lstlisting}[style=codestyle]	
MOV pippo, %EAX	% qua il suffisso di lunghezza e' implicito
MOVL %EAX, pluto
\end{lstlisting}

L'indirizzamento della memoria, nel caso più generale, è dato da: 

$$ \text{indirizzo} = \text{base} + \text{indice} \times \text{scala} \pm \text{displacement} $$

dove base e indice sono due registri generali da 32 bit, scala una costante dal valore 1 (default), 2, 4, 8, e displacement una costante intera.

La sintassi è \texttt{OPCODEsfx $\pm$disp(base,indice,scala)}.

Si distingue poi l'indirizzamento di tipo:

\begin{itemize}
	\item 
		\textbf{Diretto}, dove si indica soltanto il displacement, che coincide con l'indirizzo. \texttt{OPCODEW 0x00002001} significa prendi la word a partire da \texttt{0x00002001}.
	\item
		\textbf{Indiretto}, o con registro puntatore, dove si sfrutta un registro: \texttt{OPCODEL (\%EBX)} significa indirizzare il valore indirizzato da EBX. Si può specificare una scala: \texttt{OPCODEL (,\%EBX,4)} significa il valore nel registro EBX moltiplicato per 4.
		Si noti come a essere moltiplicato è l'indice e non la base.
	\item
		\textbf{Displacement e registro di modifica}, ad esempio da \texttt{OPCODEW 0x002A3A2B (\%EDI)} si ottiene l'operando a 16 bit ottenuto sommando al displacement \texttt{0x002A3A2B} il contenuto di EDI, modulo $2^32$.
	\item \textbf{Bimodificato senza displacement}, ad esempio \texttt{OPCODEW (\%EBX, \%EDI)}, che dipende sia da EBX che da EDI. Si può anche includere una scala: \texttt{OPCODEW (\%EBX, \%EDI, 8)}.
\item \textbf{Bimodificato con displacement}, come prima ma con displacement: \texttt{OPCODEB 0x002F9000 (\%EBX, \%EDI)}, ovvero l'indirizzo dato da base in EBX + indice in EDI + l'offset modulo $2^32$. Si può avere anche negativo: \texttt{OPCODEB -0x9000 (\%EBX, \%EDI)}, dove si sottrae l'offset invece di sommarlo.
\end{itemize}

Notare che senza il \$ i numeri in formato esadecimale sono interpretati automaticamente come indirizzi.

\item
Si possono \textbf{indirizzare le porte I/O}, come prima in uno solo dei due operandi. 
Questo si fa con le istruzioni specifiche IN e OUT.
In particolare si ha indirizzamento di tipo:

\begin{itemize}
	\item \textbf{Diretto}, solo per indirizzi $ < 256 $, in quanto nel formato macchina ci sono 8 bit.
		Ad esempio \texttt{IN 0x001A, \%AL} o \texttt{OUT \%AL, 0x003A}.
	\item \textbf{Indiretto con registro puntatore}, usando come registro puntatore soltanto DX.
		Ad esempio \texttt{IN (\%DX), \%AX} o \texttt{OUT \%AL, (\%DX)}.
\end{itemize}

\end{itemize}

\subsection{Panoramica sulle istruzioni}
Abbiamo diviso le istruzioni in \textbf{operative} e \textbf{di controllo}.
Possiamo fare ulteriori suddivisioni:

\begin{itemize}
	\item \textbf{Operative:}
		\begin{itemize}
			\item Di trasferimento;
			\item Aritmetiche;
			\item Di traslazione/rotazione:
			\item Logiche.
		\end{itemize}
	\item \textbf{Di controllo:}
		\begin{itemize}
			\item Di salto;
			\item Di gestione di sottoprogrammi.
		\end{itemize}
\end{itemize}

Conviene definire formato, funzionamento, comportamento sui flag e modalità di indirizzamento ammesse per gli operandi di ogni operazione, in quanto l'assembler non è \textbf{ortogonale}, ergo ci sono particolari restrizioni su \textit{quali} operandi e modalità di indirizzamento possono essere combinate.

\end{document}


\documentclass[a4paper,11pt]{article}
\usepackage[a4paper, margin=8em]{geometry}

% usa i pacchetti per la scrittura in italiano
\usepackage[french,italian]{babel}
\usepackage[T1]{fontenc}
\usepackage[utf8]{inputenc}
\frenchspacing 

% usa i pacchetti per la formattazione matematica
\usepackage{amsmath, amssymb, amsthm, amsfonts}

% usa altri pacchetti
\usepackage{gensymb}
\usepackage{hyperref}
\usepackage{standalone}

% imposta il titolo
\title{Appunti Reti Logiche}
\author{Luca Seggiani}
\date{2024}

% imposta lo stile
% usa helvetica
\usepackage[scaled]{helvet}
% usa palatino
\usepackage{palatino}
% usa un font monospazio guardabile
\usepackage{lmodern}

\renewcommand{\rmdefault}{ppl}
\renewcommand{\sfdefault}{phv}
\renewcommand{\ttdefault}{lmtt}

% disponi il titolo
\makeatletter
\renewcommand{\maketitle} {
	\begin{center} 
		\begin{minipage}[t]{.8\textwidth}
			\textsf{\huge\bfseries \@title} 
		\end{minipage}%
		\begin{minipage}[t]{.2\textwidth}
			\raggedleft \vspace{-1.65em}
			\textsf{\small \@author} \vfill
			\textsf{\small \@date}
		\end{minipage}
		\par
	\end{center}

	\thispagestyle{empty}
	\pagestyle{fancy}
}
\makeatother

% disponi teoremi
\usepackage{tcolorbox}
\newtcolorbox[auto counter, number within=section]{theorem}[2][]{%
	colback=blue!10, 
	colframe=blue!40!black, 
	sharp corners=northwest,
	fonttitle=\sffamily\bfseries, 
	title=Teorema~\thetcbcounter: #2, 
	#1
}

% disponi definizioni
\newtcolorbox[auto counter, number within=section]{definition}[2][]{%
	colback=red!10,
	colframe=red!40!black,
	sharp corners=northwest,
	fonttitle=\sffamily\bfseries,
	title=Definizione~\thetcbcounter: #2,
	#1
}

% disponi codice
\usepackage{listings}
\usepackage[table]{xcolor}

\lstdefinestyle{codestyle}{
		backgroundcolor=\color{black!5}, 
		commentstyle=\color{codegreen},
		keywordstyle=\bfseries\color{magenta},
		numberstyle=\sffamily\tiny\color{black!60},
		stringstyle=\color{green!50!black},
		basicstyle=\ttfamily\footnotesize,
		breakatwhitespace=false,         
		breaklines=true,                 
		captionpos=b,                    
		keepspaces=true,                 
		numbers=left,                    
		numbersep=5pt,                  
		showspaces=false,                
		showstringspaces=false,
		showtabs=false,                  
		tabsize=2
}

\lstdefinestyle{shellstyle}{
		backgroundcolor=\color{black!5}, 
		basicstyle=\ttfamily\footnotesize\color{black}, 
		commentstyle=\color{black}, 
		keywordstyle=\color{black},
		numberstyle=\color{black!5},
		stringstyle=\color{black}, 
		showspaces=false,
		showstringspaces=false, 
		showtabs=false, 
		tabsize=2, 
		numbers=none, 
		breaklines=true
}

\lstdefinelanguage{javascript}{
	keywords={typeof, new, true, false, catch, function, return, null, catch, switch, var, if, in, while, do, else, case, break},
	keywordstyle=\color{blue}\bfseries,
	ndkeywords={class, export, boolean, throw, implements, import, this},
	ndkeywordstyle=\color{darkgray}\bfseries,
	identifierstyle=\color{black},
	sensitive=false,
	comment=[l]{//},
	morecomment=[s]{/*}{*/},
	commentstyle=\color{purple}\ttfamily,
	stringstyle=\color{red}\ttfamily,
	morestring=[b]',
	morestring=[b]"
}

% disponi sezioni
\usepackage{titlesec}

\titleformat{\section}
	{\sffamily\Large\bfseries} 
	{\thesection}{1em}{} 
\titleformat{\subsection}
	{\sffamily\large\bfseries}   
	{\thesubsection}{1em}{} 
\titleformat{\subsubsection}
	{\sffamily\normalsize\bfseries} 
	{\thesubsubsection}{1em}{}

% tikz
\usepackage{tikz}

% float
\usepackage{float}

% grafici
\usepackage{pgfplots}
\pgfplotsset{width=10cm,compat=1.9}

% disponi alberi
\usepackage{forest}

\forestset{
	rectstyle/.style={
		for tree={rectangle,draw,font=\large\sffamily}
	},
	roundstyle/.style={
		for tree={circle,draw,font=\large}
	}
}

% disponi algoritmi
\usepackage{algorithm}
\usepackage{algorithmic}
\makeatletter
\renewcommand{\ALG@name}{Algoritmo}
\makeatother

% disponi numeri di pagina
\usepackage{fancyhdr}
\fancyhf{} 
\fancyfoot[L]{\sffamily{\thepage}}

\makeatletter
\fancyhead[L]{\raisebox{1ex}[0pt][0pt]{\sffamily{\@title \ \@date}}} 
\fancyhead[R]{\raisebox{1ex}[0pt][0pt]{\sffamily{\@author}}}
\makeatother

\begin{document}
% sezione (data)
\section{Lezione del 26-09-24}

% stili pagina
\thispagestyle{empty}
\pagestyle{fancy}

% testo
\subsection{Istruzioni di trasferimento}
Le istruzioni di trasferimento spostano memoria:
\begin{itemize}
	\item Dalla memoria a un registro;
	\item Da un registro a un registro;
	\item Dallo spazio I/O a un regsitro.
\end{itemize}

Non esistono altre possibilità, ergo non si può (per quanto interessa a noi) spostare da memoria a memoria.
In verità esistono alcune istruzioni nei processori di nuova generazione che ottimizzano operazioni di questo tipo, che verrano viste in seguito.
Sfruttando i registri, il trasferimento da memoria a memoria si fa attraverso un registro, in due istruzioni.

Nessuna istruzione di trasferimento modifica i flag.

\subsubsection{MOVE}
\begin{itemize}
	\item \textbf{Formato:} \texttt{MOV source, destination}
	\item \textbf{Azione:} sostituisce l'operando destinatario con una copia dell'operando sorgente.
	\item \textbf{Flag:} nessuno.

		\begin{table}[h!]
			\center \rowcolors{2}{white}{black!10}
			\begin{tabular} { c | p{5cm} }
				\bfseries Operandi & \bfseries Esempi \\
				\hline 
				Memoria, Registro Generale & \texttt{MOV 0x00002000, \%EDX} \\
				Registro Generale, Memoria & \texttt{MOV \%CL, 0x12AB1024} \\
				Registro Generale, Registro Generale & \texttt{MOV \%AX, \%DX} \\
				Immediato, Memoria & \texttt{MOVB \$0x5B, (\%EDI)} \\ 
				Immediato, Registro generale & \texttt{MOV \$0x54A3, \%AX}
			\end{tabular}
		\end{table}
\end{itemize}

\subsubsection{LOAD EFFECTIVE ADDRESS}
\begin{itemize}
	\item \textbf{Formato:} \texttt{LEA source, destination}
	\item \textbf{Azione:} sostituisce l'operando destinatario con l'espressione indirizzo contenuta nell'operando sorgente.
	\item \textbf{Flag:} nessuno.

		\begin{table}[h!]
			\center \rowcolors{2}{white}{black!10}
			\begin{tabular} { c | p{7cm} }
				\bfseries Operandi & \bfseries Esempi \\
				\hline 
				Memoria, Registro Generale a 32 bit & \texttt{LEA 0x00002000, \%EDX} \\
																						& \texttt{LEA 0x00213AB1 (\%EAX,\%EBX,4), \%ECX}
			\end{tabular}
		\end{table}
\end{itemize}

A differenza di MOV, LEA calcola l'indirizzo della locazione di memoria cercata come $ \text{base} + \text{index} \times \text{scala} \pm \text{displacement} $, e carica quell'indirizzo nella destinazione, non il valore contenuto in esso.
Nel primo esempio, questo equivale alla MOV con indirizzamento immediato.
In altri casi permette di ricavare esplicitamente il valore ottenuto dall'indirizzamento complesso.

\subsubsection{EXCHANGE}
\begin{itemize}
	\item \textbf{Formato:} \texttt{XCHG source, destination}
	\item \textbf{Azione:} sostituisce l'operando destinatario con l'operando sorgente e viceversa. Questa operazione è l'unica che modifica il sorgente.
	\item \textbf{Flag:} nessuno.

		\begin{table}[h!]
			\center \rowcolors{2}{white}{black!10}
			\begin{tabular} { c | p{7cm} }
				\bfseries Operandi & \bfseries Esempi \\
				\hline 
				Memoria, Registro Generale & \texttt{XCHG 0x00002000, \%DX} \\
				Registro Generale, Memoria & \texttt{XCHG \%AL, 0x000A2003} \\
				Registro Generale, Registro Generale & \texttt{XCHG \%EAX, \%EDX}
			\end{tabular}
		\end{table}

		Grazie a quest'istruzione in assembler si possono scambiare due operandi con una sola istruzione (\textbf{non trasparenza} dei registri) \textbf{atomica}.
		Questo è particolarmente utile nel caso di esecuzione concorrente.
\end{itemize}

\subsubsection{INPUT}
\begin{itemize}
	\item \textbf{Formato:}
		\begin{itemize}
			\item \texttt{IN indirizzo, \%AL} (8 bit)
			\item \texttt{IN indirizzo, \%AX} (16 bit)
			\item \texttt{IN (\%DX), \%AX} (8 bit) 
			\item \texttt{IN (\%DX), \%Al} (16 bit)
		\end{itemize}
	\item \textbf{Azione:} sostituisce il contenuto del registro destinatario (AL 8 bit, AX 16 bit) con il contenuto di un adeguato numero di porte consecutive.
		L'indirizzo è specificato direttamente (per porte con indirizzo $<256$), o indirettamente usando il registro DX.
	\item \textbf{Flag:} nessuno.
\end{itemize}

\subsubsection{OUTPUT}
\begin{itemize}
	\item \textbf{Formato:}
		\begin{itemize}
			\item \texttt{OUT \%AL, indirizzo} (8 bit)
			\item \texttt{IN \%AX, indirizzo} (16 bit)
			\item \texttt{IN \%AX}, (\%DX) (8 bit) 
			\item \texttt{IN \%Al, (\%DX)} (16 bit)
		\end{itemize}
	\item \textbf{Azione:} copia il contenuto del registro sorgente (AL 8 bit, AX 16 bit) su un adeguato numero di porte consecutive.
		L'indirizzo è specificato direttamente (per porte con indirizzo $<256$), o indirettamente usando il registro DX.
	\item \textbf{Flag:} nessuno.
\end{itemize}

\subsubsection{Non ortogonalità INPUT/OUTPUT}
Le uniche due operazioni che gestiscono l'input e l'output possono trasferire solo dai o nei registri AL e AX, e indirizzare indirettamente la memoria puntando col registro DX.
Questo rende le operazioni non ortogonali: non si possono usare altri registri, ed eventuali operazioni vanno fatte nel processore,

\subsection{Pila}
La pila, o \textbf{stack}, è una regione di memoria gestita con politica Last In First Out (LIFO), essenziale al funzionamento del calcolatore.
Permette di annidare sottoprogrammi, funzionalità per cui l'assembler è organizzato.

Generalmente, la pila viene usata come segue per eseguire i sottoprogrammi:
\begin{itemize}
	\item Prima di saltare al sottoprogramma, si fa \textbf{PUSH} sulla pila dell'indirizzo di ritorno (e.g. l'indirizzo della prossima istruzione);
	\item Si esegue il sottoprogramma;
	\item Al termine del sottoprogramma, si fa \textbf{POP} dalla pila del prossimo indirizzo.
\end{itemize}

Più sottoprogrammi possono chiamarsi a vicenda (annidarsi), ponendosi su livelli via via superiori della pila.
Al termine della sua esecuzione, ogni sottoprogramma tornerà all'indirizzo di ripresa del sottoprogramma precedente, finché tutti i sottoprogrammi non termineranno l'esecuzione.

Il registro \textbf{ESP} punta al top della pila, ergo non va usato per altri scopi.
Va però inizializzato prima che parta il programma.
Si deve inoltre notare che la pila in assembler si estende \textit{verso il basso}: aggiungere alla pila significa decrementare ESP, e rimuovere dalla pila significa incrementare ESP.
I frame successivi della pila si vanno a disporre via via sotto (o "a sinistra") del frame corrente.

Per lavorare sulla pila si usano le istruzioni:

\subsubsection{PUSH}
\begin{itemize}
	\item \textbf{Formato:} \texttt{PUSH source}
	\item \textbf{Azione:} decrementa ESP e copia il sorgente nell'indirizzo puntato da ESP.
		Il sorgente deve essere a 16 bit o a 32 bit.
		Nello specifico, compie le seguenti azioni:
		\begin{itemize}
			\item Decrementa l'indirizzo contenuto nel registro ESP di 2 o 4;
			\item Memorizza una copia dell'operando sorgente nella word o long il cui indirizzo è contenuto in ESP.
		\end{itemize}
	\item \textbf{Flag:} nessuno.
\end{itemize}

		\begin{table}[h!]
			\center \rowcolors{2}{white}{black!10}
			\begin{tabular} { c | p{5cm} }
				\bfseries Operandi & \bfseries Esempi \\
				\hline 
				Memoria & \texttt{PUSHW 0x3214200A} \\ 
				Immediato & \texttt{PUSHL \$0x4871A000} \\ 
				Registro Generale & \texttt{PUSH \%BX}
			\end{tabular}
		\end{table}

\subsubsection{POP}
\begin{itemize}
	\item \textbf{Formato:} \texttt{POP destination}
	\item \textbf{Azione:} copia una word o un long dall'indirzzo puntato dall'ESP nel destinatario e incrementa ESP.
		Nello specifico compie le seguenti azioni:
		\begin{itemize}
			\item Sostituisce all'operando destinatario una copia del contenuto nella word o long il cui indirizzo è contenuto in ESP;
			\item Incrementa di due o quattro l'indirizzo contenuto in ESP, rimuovendo la word o il long copiato.
		\end{itemize}
	\item \textbf{Flag:} nessuno.
\end{itemize}
	
		\begin{table}[h!]
			\center \rowcolors{2}{white}{black!10}
			\begin{tabular} { c | p{5cm} }
				\bfseries Operandi & \bfseries Esempi \\
				\hline 
				Memoria & \texttt{POPW 0x02AB2000} \\ 
				Registro Generale & \texttt{POP \%BX}
			\end{tabular}
		\end{table}

\par\medskip

\noindent
\textsf{\textbf{Dati temporanei nella pila}} \\
Solitamente la pila viene usata per memorizzare dati temporanei, visto che i registri sono pochi e spesso hanno scopi diversi in momenti diversi. Ad esempio:

\begin{lstlisting}[style=codestyle]	
# sto usando %EAX, mi serve un dato da una porta
PUSH %EAX
IN 0x001A, %AL
...
POP %EAX # ritorno da dove ero
\end{lstlisting}

\subsubsection{PUSHAD}
\begin{itemize}
	\item \textbf{Formato:} \texttt{PUSHAD}
	\item \textbf{Azione:}: salva nella pila corrente una copia degli 8 registri generali a 32 bit, nell'ordine: EAX, ECX, EDX, EBX, ESP, EBP, ESI, EDI.
	\item \textbf{Flag:} nessuno.
\end{itemize}

\subsubsection{POPAD}
\begin{itemize}
	\item \textbf{Formato:} \texttt{POPAD}
	\item \textbf{Azione:}: copia dalla pila corrente gli 8 registri generali a 32 bit, nell'ordine: EAX, ECX, EDX, EBX, ESP, EBP, ESI, EDI. 
	\item \textbf{Flag:} nessuno.
\end{itemize}

\subsection{Istruzioni aritmetiche}
Molte operazioni aritmetiche di base non distinguono numeri naturali e numeri interi, distinzione che viene fatta solo per moltiplicazioni e divisioni.

Le operazioni possono modificare i flag, e in questo caso i flag da controllare dipenderanno dal tipo di numeri su cui si è fatta l'operazione (informazione nota soltanto al programmatore).

Abbiamo quindi che un'operazione aritmetica si svolge seguendo i passi:
\begin{itemize}
	\item Si esegue l'operazione;
	\item Si controllano i flag interessati (OF, SF e ZF sugli interi, CF e ZF sui naturali) per verificarne l'esito.
\end{itemize}

Vediamo quindi le operazioni aritmetiche:

\subsubsection{ADD}
\begin{itemize}
	\item \textbf{Formato:} \texttt{ADD source, destination}
	\item \textbf{Azione:} modifica l'operando destinatario sommandovi l'operando sorgente.
		Il risultato è consistente sia che si interpretino i numeri come naturali, che come interi.
	\item \textbf{Flag:} attiva CF se, interpretando i numeri come naturali, si è verificato un riporto; attiva OF se, interpretando gli operandi come interi, si è verificato un traboccamento.
		Inoltre attiva opportunamente ZF e SF se il numero è rispettivamente zero o negativo (in complemento a 2).
\end{itemize}

		\begin{table}[h!]
			\center \rowcolors{2}{white}{black!10}
			\begin{tabular} { c | p{5cm} }
				\bfseries Operandi & \bfseries Esempi \\
				\hline
				Memoria, Registro Generale & \texttt{ADD 0x00002000, \%EDX} \\ 
				Registro Generale, Memoria & \texttt{ADD \%CL, 0x12AB1024} \\ 
				Registro Generale, Registro Generale & \texttt{ADD \%AX, \%DX} \\ 
				Immediato, Memoria & \texttt{ADDB \$0x5B, (\%EDI)} \\ 
				Immediato, Registro Generale & \texttt{ADD \$0x54A3, \%AX}
			\end{tabular}
		\end{table}

\par\medskip
\noindent
\textbf{\textsf{Funzionamento della ADD}} \\
Il passo elementare di una somma consiste nel sommare due addendi (propriamente due cifre degli addendi) e un riporto entrante per produrre:
	\begin{itemize}
		\item Una cifra;
		\item Un riporto uscente (cioè il riporto entrante per il prossimo passo).
	\end{itemize}
L'ultimo riporto, se non entra in memoria, attiva il carry flag (CF).

L'operazione di somma ha lo stesso effetto sia su naturali che su interi in complemento a 2: la differenza sta nel controllo dell'attivazione dei flag.
Il carry flag non ha infatti alcun significato nella somma fra interi: dobbiamo controllare l'OF.

In generale, si ha overflow (OF) quando il risultato esce dall'intervallo di rappresentabilità.
Si può capire se si è verificato un overflow controllando i segni degli operandi:
\begin{itemize}
	\item \textbf{Segni discordi:} non c'é overflow;
	\item \textbf{Segni concordi:} il risultato è concorde se è concorde con gli operandi.
\end{itemize}
La ADD imposta quindi OF secondo queste regole.
Il ZF viene poi impostato se il risultato è fatto da tutti zeri, e il SF viene impostato se il MSB è uno.

\subsubsection{INCREMENT}
\begin{itemize}
	\item \textbf{Formato:} \texttt{INC destination}
	\item \textbf{Azione:} equivale all'istruzione \texttt{ADD \$1, destination}. 
	\item \textbf{Flag:} modifica tutti i flag di ADD tranne CF (il riporto).
\end{itemize}

		\begin{table}[H]
			\center \rowcolors{2}{white}{black!10}
			\begin{tabular} { c | c }
				\bfseries Operandi & \bfseries Esempi \\
				\hline 
				Memoria & \texttt{INCB (\%ESI)} \\
				Registro Generale & \texttt{INC \%CX}
			\end{tabular}
		\end{table}

Quest'istruzione è più compatta di ADD, e storicamente era anche più veloce.
Questo deriva dal fatto che la circuiteria che implementava l'incremento era più efficiente di quella che implementa le somme.

\subsubsection{SUBTRACT}
\begin{itemize}
	\item \textbf{Formato:} \texttt{SUB source, destination}
	\item \textbf{Azione:} modifica l'operando destinatario sottraendovi l'operando sorgente. 
		Il risultato è consistente sia che si interpretino i numeri come naturali, che come interi.
	\item \textbf{Flag:} attiva CF se, interpretando i numeri come naturali, si è verificato un riporto; attiva OF se, interpretando gli operandi come interi, si è verificato un traboccamento.
\end{itemize}

		\begin{table}[h!]
			\center \rowcolors{2}{white}{black!10}
			\begin{tabular} { c | p{5cm} }
				\bfseries Operandi & \bfseries Esempi \\
				\hline
				Memoria, Registro Generale & \texttt{SUB 0x00002000, \%EDX} \\ 
				Registro Generale, Memoria & \texttt{SUB \%CL, 0x12AB1024} \\ 
				Registro Generale, Registro Generale & \texttt{SUB \%AX, \%DX} \\ 
				Immediato, Memoria & \texttt{SUBB \$0x5B, (\%EDI)} \\ 
				Immediato, Registro Generale & \texttt{SUB \$0x54A3, \%AX}
			\end{tabular}
		\end{table}

\par\medskip
\noindent
\textbf{\textsf{Funzionamento della SUBTRACT}} \\
Il passo elementare della sottrazione è effettivamente il contrario di quello della somma: si sottraggono il sottraendo e un prestito entrante al minuendo, producendo:
\begin{itemize}
	\item Una cifra;
	\item Un prestito uscente.
\end{itemize}

Il carry flag (CF) memorizza il prestito.
Se alla fine dell'operazione il CF è impostato, significa che il risultato è un numero intero.

Questo funziona anche sugli interi: in questo caso, come prima, non si controlla il CF, ma l'OF, che conterrà la seguente informazione:
\begin{itemize}
	\item La differenza di numeri concordi è sempre rappresentabile;
	\item La differenza di numeri discordi è rappresentabile solo se il risultato ha il segno del minuendo.
\end{itemize}

Il ZF e il SF vengono attivati secondo le regole già note.

\subsubsection{DECREMENT}
\begin{itemize}
	\item \textbf{Formato:} \texttt{DEC destination}
	\item \textbf{Azione:} equivale all'istruzione \texttt{SUB \$1, destination}. 
	\item \textbf{Flag:} modifica tutti i flag di SUBTRACT tranne CF (il prestito).
\end{itemize}

		\begin{table}[h!]
			\center \rowcolors{2}{white}{black!10}
			\begin{tabular} { c | c }
				\bfseries Operandi & \bfseries Esempi \\
				\hline 
				Memoria & \texttt{DECB (\%EDI)} \\
				Registro Generale & \texttt{DEC \%CX}
			\end{tabular}
		\end{table}

\subsubsection{ADD WITH CARRY}
\begin{itemize}
	\item \textbf{Formato:} \texttt{ADC source, destination}
	\item \textbf{Azione:} modifica l'operando destinatario sommandovi sia l'operando sorgente sia il contenuto del flag CF.
	\item \textbf{Flag:} modifica tutti i flag come ADD. 
\end{itemize}

\begin{table}[h!]
			\center \rowcolors{2}{white}{black!10}
			\begin{tabular} { c | p{5cm} }
				\bfseries Operandi & \bfseries Esempi \\
				\hline
				Memoria, Registro Generale & \texttt{ADC 0x00002000, \%EDX} \\ 
				Registro Generale, Memoria & \texttt{ADC \%CL, 0x12AB1024} \\ 
				Registro Generale, Registro Generale & \texttt{ADC \%AX, \%DX} \\ 
				Immediato, Memoria & \texttt{ADCB \$0x5B, (\%EDI)} \\ 
				Immediato, Registro Generale & \texttt{ADC \$0x54A3, \%AX}
			\end{tabular}
		\end{table}

Quest'istruzione è utile per effettuare somme di numeri più grandi di 32 bit.
In questo caso si:
\begin{itemize}
	\item Effettua la somma dei 32 bit meno significativi con ADD;
	\item Sommano i successivi 32 bit con ADC portandosi quindi dietro il carry.
\end{itemize}

\subsubsection{SUBTRACT WITH BORROW}
\begin{itemize}
	\item \textbf{Formato:} \texttt{SBB source, destination}
	\item \textbf{Azione:} modifica l'operando destinatario sottraendovi sia l'operando sorgente sia il contenuto del flag CF.
	\item \textbf{Flag:} modifica tutti i flag come SUBTRACT. 
\end{itemize}

		\begin{table}[H]
		\center \rowcolors{2}{white}{black!10}
			\begin{tabular} { c | p{5cm} }
				\bfseries Operandi & \bfseries Esempi \\
				\hline
				Memoria, Registro Generale & \texttt{SBB 0x00002000, \%EDX} \\ 
				Registro Generale, Memoria & \texttt{SBB \%CL, 0x12AB1024} \\ 
				Registro Generale, Registro Generale & \texttt{SBB \%AX, \%DX} \\ 
				Immediato, Memoria & \texttt{SBBB \$0x255B, (\%EDI)} \\ 
				Immediato, Registro Generale & \texttt{SBB \$0x54A3, \%AX}
			\end{tabular}
		\end{table}

Come ormai dovrebbe essere chiaro, è la duale dell'ADC, e si usa per effettuare sottrazioni di numeri più grandi di 32 bit.

\subsubsection{NEGATE}
\begin{itemize}
	\item \textbf{Formato:} \texttt{NEG destination}
	\item \textbf{Azione:} interpreta l'operando destinatario come un numero intero e lo sostituisce con il suo opposto in complemento a 2. 
	\item \textbf{Flag:} quando l'operazione non è possibile (l'intervallo di rappresentabilità degli interi in complemento a 2 non è simmetrico) imposta il flag OF.
		Imposta inoltre il flag CF quando l'operando è diverso da zero, e tutti gli altri flag in base a nullità e segno del risultato. 
\end{itemize}

		\begin{table}[H]
		\center \rowcolors{2}{white}{black!10}
			\begin{tabular} { c | p{5cm} }
				\bfseries Operandi & \bfseries Esempi \\
				\hline
				Memoria & \texttt{NEGB (\%EDI)} \\ 
				Registro Generale & \texttt{NEG \%CX}
			\end{tabular}
		\end{table}


\par\medskip
\noindent
\textbf{\textsf{Funzionamento della NEGATE}} \\
L'opposto di un numero $X$ in complemento a due è:
$$
-X = \bar{X} + 1
$$

Si ricordi che questo ha senso \textit{solamente} se il numero è rappresentato in complemento a due.

\subsubsection{COMPARE}
\begin{itemize}
	\item \textbf{Formato:} \texttt{CMP source, destination}
	\item \textbf{Azione:} verifica se l'operando destinatario è maggiore, uguale o minore dell'operando sorgente, sia interpretando gli operandi come naturali che come interi, e aggiorna i flag di conseguenza.
		Più propriamente, la compare si comporta come la SUB, ma senza sovrascrivere nessuno degli operandi.
	\item \textbf{Flag:} come la SUB. 
\end{itemize}


		\begin{table}[H]
		\center \rowcolors{2}{white}{black!10}
			\begin{tabular} { c | p{5cm} }
				\bfseries Operandi & \bfseries Esempi \\
				\hline
				Memoria, Registro Generale & \texttt{CMP 0x00002000, \%EDX} \\ 
				Registro Generale, Memoria & \texttt{CMP \%CL, 0x12AB1024} \\ 
				Registro Generale, Registro Generale & \texttt{CMP \%AX, \%DX} \\ 
				Immediato, Memoria & \texttt{CMPB \$0x255B, (\%EDI)} \\ 
				Immediato, Registro Generale & \texttt{CMP \$0x54A3, \%AX}
			\end{tabular}
		\end{table}

\subsubsection{Funzionamento della COMPARE}
Solitamente la CMP si usa nei salti condizionati come:
\begin{lstlisting}[style=codestyle]	
CMP %AX, %BX
JCOND # salto condizionato
\end{lstlisting}
\noindent
Ciò che fa la CMP è effettivamente creare un'oggetto temporaneo:
$$
\text{tmp} = \text{dest} - \text{source}
$$
che viene poi rimosso.

I flag restano però aggiornati, e questo valore può essere interpretato correttamente dalla JE per effettuare un salto condizionale.


\subsection{Moltiplicazioni}
Le moltiplicazioni, a differenza delle somme e delle differenze, sono diverse fra naturali ed interi.
Bisogna inoltre notare che le dimensioni il risultato della somma di un numero a $n$ cifre sta su $n$ o $n+1$ cifre, mentre il prodotto di due numeri a $n$ cifre sta su $2n$ cifre.
In altre parole, il numero di bit necessari a memorizzare il risultato non è più confrontabile con quello degli operatori.


\subsubsection{MULTIPLY}
\begin{itemize}
	\item \textbf{Formato:} \texttt{MUL source}
	\item \textbf{Azione:} considera l'operando sorgente come un moltiplicando, l'operando destinatario (implicito) come un moltiplicatore, e effettua la moltiplicazione assumendo i numeri naturali. Nello specifico:
	\begin{itemize}
		\item Sorgente a 8 bit, si ha $\text{AX} = \text{AL} \times \text{source}$;
		\item Sorgente a 16 bit, si ha $\text{DX}\_\text{AX} = \text{AX} \times \text{source}$;
		\item Sorgente a 32 bit, si ha $\text{EDX}\_\text{EAX} = \text{EAX} \times \text{source}$.
	\end{itemize}
	\item \textbf{Flag:} imposta CF e OF se il risultato non sta nel numero di bit di source. SF e ZF sono indefiniti.
\end{itemize}

		\begin{table}[H]
		\center \rowcolors{2}{white}{black!10}
			\begin{tabular} { c | p{5cm} }
				\bfseries Operandi & \bfseries Esempi \\
				\hline
				Memoria & \texttt{MULB (\%ESI)} \\ 
				Registro Generale & \texttt{MUL \%ECX}
			\end{tabular}
		\end{table}

\subsubsection{INTEGER MULTIPLY}
\begin{itemize}
	\item \textbf{Formato:} \texttt{MUL source}
	\item \textbf{Azione:} considera l'operando sorgente come un moltiplicando, l'operando destinatario (implicito) come un moltiplicatore, e effettua la moltiplicazione assumendo i numeri interi. Nello specifico:
	\begin{itemize}
		\item Sorgente a 8 bit, si ha $\text{AX} = \text{AL} \times \text{source}$;
		\item Sorgente a 16 bit, si ha $\text{DX}\_\text{AX} = \text{AX} \times \text{source}$;
		\item Sorgente a 32 bit, si ha $\text{EDX}\_\text{EAX} = \text{EAX} \times \text{source}$.
	\end{itemize}
	\item \textbf{Flag:} li imposta tutti, ma non è attendibile.
\end{itemize}

		\begin{table}[H]
		\center \rowcolors{2}{white}{black!10}
			\begin{tabular} { c | p{5cm} }
				\bfseries Operandi & \bfseries Esempi \\
				\hline
				Memoria & \texttt{IMULB (\%ESI)} \\ 
				Registro Generale & \texttt{IMUL \%ECX}
			\end{tabular}
		\end{table}

\par\medskip
\noindent
\textbf{\textsf{Funzionamento delle MULTIPLY e INTEGER MULTIPLY}} \\
Queste operazioni hanno sia un operando che il destinatario impliciti, in base al tipo dell'operando fornito.
Questo deriva dal fatto che il risultato di una moltiplicazione raramente sta nello stesso numero di bit dei fattori.
Di preciso, abbiamo visto i 3 tipi di moltiplicazione concessi:
\begin{itemize}
	\item Sorgente a 8 bit, si ha $\text{AX} = \text{AL} \times \text{source}$;
	\item Sorgente a 16 bit, si ha $\text{DX}\_\text{AX} = \text{AX} \times \text{source}$;
	\item Sorgente a 32 bit, si ha $\text{EDX}\_\text{EAX} = \text{EAX} \times \text{source}$.
\end{itemize}

La differenza fra le prime due operazioni e l'ultima, in particolare con sorgente a 16 bit, che usa una due registri da 16 bit separati, ha principalmente motivi storici (il registro EAX è stato introdotto dopo).

Si può rimettere il valore dai due registri a 16 bit in un registro a 32 bit attraverso la pila:
\begin{lstlisting}[style=codestyle]	
PUSH \%DX
PUSH \%AX
POP \%EAX
\end{lstlisting}

\end{document}


\documentclass[a4paper,11pt]{article}
\usepackage[a4paper, margin=8em]{geometry}

% usa i pacchetti per la scrittura in italiano
\usepackage[french,italian]{babel}
\usepackage[T1]{fontenc}
\usepackage[utf8]{inputenc}
\frenchspacing 

% usa i pacchetti per la formattazione matematica
\usepackage{amsmath, amssymb, amsthm, amsfonts}

% usa altri pacchetti
\usepackage{gensymb}
\usepackage{hyperref}
\usepackage{standalone}

% imposta il titolo
\title{Appunti Reti Logiche}
\author{Luca Seggiani}
\date{2024}

% imposta lo stile
% usa helvetica
\usepackage[scaled]{helvet}
% usa palatino
\usepackage{palatino}
% usa un font monospazio guardabile
\usepackage{lmodern}

\renewcommand{\rmdefault}{ppl}
\renewcommand{\sfdefault}{phv}
\renewcommand{\ttdefault}{lmtt}

% disponi il titolo
\makeatletter
\renewcommand{\maketitle} {
	\begin{center} 
		\begin{minipage}[t]{.8\textwidth}
			\textsf{\huge\bfseries \@title} 
		\end{minipage}%
		\begin{minipage}[t]{.2\textwidth}
			\raggedleft \vspace{-1.65em}
			\textsf{\small \@author} \vfill
			\textsf{\small \@date}
		\end{minipage}
		\par
	\end{center}

	\thispagestyle{empty}
	\pagestyle{fancy}
}
\makeatother

% disponi teoremi
\usepackage{tcolorbox}
\newtcolorbox[auto counter, number within=section]{theorem}[2][]{%
	colback=blue!10, 
	colframe=blue!40!black, 
	sharp corners=northwest,
	fonttitle=\sffamily\bfseries, 
	title=Teorema~\thetcbcounter: #2, 
	#1
}

% disponi definizioni
\newtcolorbox[auto counter, number within=section]{definition}[2][]{%
	colback=red!10,
	colframe=red!40!black,
	sharp corners=northwest,
	fonttitle=\sffamily\bfseries,
	title=Definizione~\thetcbcounter: #2,
	#1
}

% disponi codice
\usepackage{listings}
\usepackage[table]{xcolor}

\definecolor{codegreen}{rgb}{0,0.6,0}
\definecolor{codegray}{rgb}{0.5,0.5,0.5}
\definecolor{codepurple}{rgb}{0.58,0,0.82}
\definecolor{backcolour}{rgb}{0.95,0.95,0.92}

\lstdefinestyle{codestyle}{
		backgroundcolor=\color{black!5}, 
		commentstyle=\color{codegreen},
		keywordstyle=\bfseries\color{magenta},
		numberstyle=\sffamily\tiny\color{black!60},
		stringstyle=\color{green!50!black},
		basicstyle=\ttfamily\footnotesize,
		breakatwhitespace=false,         
		breaklines=true,                 
		captionpos=b,                    
		keepspaces=true,                 
		numbers=left,                    
		numbersep=5pt,                  
		showspaces=false,                
		showstringspaces=false,
		showtabs=false,                  
		tabsize=2
}

\lstdefinestyle{shellstyle}{
		backgroundcolor=\color{black!5}, 
		basicstyle=\ttfamily\footnotesize\color{black}, 
		commentstyle=\color{black}, 
		keywordstyle=\color{black},
		numberstyle=\color{black!5},
		stringstyle=\color{black}, 
		showspaces=false,
		showstringspaces=false, 
		showtabs=false, 
		tabsize=2, 
		numbers=none, 
		breaklines=true
}


\lstdefinelanguage{assembler}{
  keywords={AAA, AAD, AAM, AAS, ADC, ADCB, ADCW, ADCL, ADD, ADDB, ADDW, ADDL, AND, ANDB, ANDW, ANDL,
        ARPL, BOUND, BSF, BSFL, BSFW, BSR, BSRL, BSRW, BSWAP, BT, BTC, BTCB, BTCW, BTCL, BTR, 
        BTRB, BTRW, BTRL, BTS, BTSB, BTSW, BTSL, CALL, CBW, CDQ, CLC, CLD, CLI, CLTS, CMC, CMP,
        CMPB, CMPW, CMPL, CMPS, CMPSB, CMPSD, CMPSW, CMPXCHG, CMPXCHGB, CMPXCHGW, CMPXCHGL,
        CMPXCHG8B, CPUID, CWDE, DAA, DAS, DEC, DECB, DECW, DECL, DIV, DIVB, DIVW, DIVL, ENTER,
        HLT, IDIV, IDIVB, IDIVW, IDIVL, IMUL, IMULB, IMULW, IMULL, IN, INB, INW, INL, INC, INCB,
        INCW, INCL, INS, INSB, INSD, INSW, INT, INT3, INTO, INVD, INVLPG, IRET, IRETD, JA, JAE,
        JB, JBE, JC, JCXZ, JE, JECXZ, JG, JGE, JL, JLE, JMP, JNA, JNAE, JNB, JNBE, JNC, JNE, JNG,
        JNGE, JNL, JNLE, JNO, JNP, JNS, JNZ, JO, JP, JPE, JPO, JS, JZ, LAHF, LAR, LCALL, LDS,
        LEA, LEAVE, LES, LFS, LGDT, LGS, LIDT, LMSW, LOCK, LODSB, LODSD, LODSW, LOOP, LOOPE,
        LOOPNE, LSL, LSS, LTR, MOV, MOVB, MOVW, MOVL, MOVSB, MOVSD, MOVSW, MOVSX, MOVSXB,
        MOVSXW, MOVSXL, MOVZX, MOVZXB, MOVZXW, MOVZXL, MUL, MULB, MULW, MULL, NEG, NEGB, NEGW,
        NEGL, NOP, NOT, NOTB, NOTW, NOTL, OR, ORB, ORW, ORL, OUT, OUTB, OUTW, OUTL, OUTSB, OUTSD,
        OUTSW, POP, POPA, POPAD, POPF, POPFD, PUSH, PUSHA, PUSHAD, PUSHF, PUSHFD, RCL, RCLB, RCLW,
        RCLL, RCR, RCRB, RCRW, RCRL, RDMSR, RDPMC, RDTSC, REP, REPE, REPNE, RET, ROL, ROLB, ROLW,
        ROLL, ROR, RORB, RORW, RORL, SAHF, SAL, SALB, SALW, SALL, SAR, SARB, SARW, SARL, SBB,
        SBBB, SBBW, SBBL, SCASB, SCASD, SCASW, SETA, SETAE, SETB, SETBE, SETC, SETE, SETG, SETGE,
        SETL, SETLE, SETNA, SETNAE, SETNB, SETNBE, SETNC, SETNE, SETNG, SETNGE, SETNL, SETNLE,
        SETNO, SETNP, SETNS, SETNZ, SETO, SETP, SETPE, SETPO, SETS, SETZ, SGDT, SHL, SHLB, SHLW,
        SHLL, SHLD, SHR, SHRB, SHRW, SHRL, SHRD, SIDT, SLDT, SMSW, STC, STD, STI, STOSB, STOSD,
        STOSW, STR, SUB, SUBB, SUBW, SUBL, TEST, TESTB, TESTW, TESTL, VERR, VERW, WAIT, WBINVD,
        XADD, XADDB, XADDW, XADDL, XCHG, XCHGB, XCHGW, XCHGL, XLAT, XLATB, XOR, XORB, XORW, XORL},
  keywordstyle=\color{blue}\bfseries,
  ndkeywordstyle=\color{darkgray}\bfseries,
  identifierstyle=\color{black},
  sensitive=false,
  comment=[l]{\#},
  morecomment=[s]{/*}{*/},
  commentstyle=\color{purple}\ttfamily,
  stringstyle=\color{red}\ttfamily,
  morestring=[b]',
  morestring=[b]"
}

\lstset{language=assembler, style=codestyle}

% disponi sezioni
\usepackage{titlesec}

\titleformat{\section}
	{\sffamily\Large\bfseries} 
	{\thesection}{1em}{} 
\titleformat{\subsection}
	{\sffamily\large\bfseries}   
	{\thesubsection}{1em}{} 
\titleformat{\subsubsection}
	{\sffamily\normalsize\bfseries} 
	{\thesubsubsection}{1em}{}

% tikz
\usepackage{tikz}

% float
\usepackage{float}

% grafici
\usepackage{pgfplots}
\pgfplotsset{width=10cm,compat=1.9}

% disponi alberi
\usepackage{forest}

\forestset{
	rectstyle/.style={
		for tree={rectangle,draw,font=\large\sffamily}
	},
	roundstyle/.style={
		for tree={circle,draw,font=\large}
	}
}

% disponi algoritmi
\usepackage{algorithm}
\usepackage{algorithmic}
\makeatletter
\renewcommand{\ALG@name}{Algoritmo}
\makeatother

% disponi numeri di pagina
\usepackage{fancyhdr}
\fancyhf{} 
\fancyfoot[L]{\sffamily{\thepage}}

\makeatletter
\fancyhead[L]{\raisebox{1ex}[0pt][0pt]{\sffamily{\@title \ \@date}}} 
\fancyhead[R]{\raisebox{1ex}[0pt][0pt]{\sffamily{\@author}}}
\makeatother

\begin{document}
% sezione (data)
\section{Lezione del 27-09-24}

% stili pagina
\thispagestyle{empty}
\pagestyle{fancy}

% testo
\subsection{Divisioni}
La divisone è l'operazione più complessa fra le 4 operazioni aritmetiche fondamentali.
I risultati, di base, sono due: \textbf{quoziente} e \textbf{resto}.
Inoltre, l'operazione non è ben definita quando il divisore vale 0.

Facciamo innanzitutto delle considerazioni di dimensione dei risultati:
$$
X / Y \rightarrow (Q, R), \quad
0 \leq R \leq Y - 1, \quad
0 \leq Q \leq X
$$

In assembler, si assume il quoziente e il resto stiano sulla metà dei bit che rappresentano il dividendo.
Bisogna fare attenzione in quanto questo non è sempre il caso.

\subsubsection{DIVIDE}
\begin{itemize}
	\item \textbf{Formato:} \lstinline|DIV source|
	\item \textbf{Azione:} considera l'operando sorgente come un divisore, l'operando destinatario (implicito) come un dividendo, e effettua la divisione assumendo i numeri naturali. Nello specifico:
	\begin{itemize}
	\item Sorgente a 8 bit, si ha $\text{AL} = \text{AX} \div \text{source}$, e $ \text{AH} = \text{AX} \mod \text{source} $;
	\item Sorgente a 16 bit, si ha $\text{AX} = \text{DX\_AX} \div \text{source}$, e $ \text{DX} = \text{DX\_AX} \mod \text{source} $;
	\item Sorgente a 32 bit, si ha $\text{EAX} = \text{EDX\_EAX} \div \text{source}$, e $ \text{EDX} = \text{EDX\_EAX} \mod \text{source} $;
	\end{itemize}
		Nel caso il quoziente non sia esprimibile su un numero di bit pari a quello del divisore, allora si genera un'eccezione interna, che mette in esecuzione un sottoprogramma.
		Da lì in poi i risultati generati non sono più attendibili
	\item \textbf{Flag:} imposta tutti i bit, ma non è attendibile. 
\end{itemize}

		\begin{table}[H]
		\center \rowcolors{2}{white}{black!10}
			\begin{tabular} { c | p{10cm} }
				\bfseries Operandi & \bfseries Esempi \\
				\hline
				Memoria & \lstinline!DIVB (\%ESI)	\# AX destinazione implicita! \\ 
				Registro Generale & \lstinline|DIV \%ECX	\# EDX\_EAX destinazione implicita|
			\end{tabular}
		\end{table}

Attenzione: la destinazione implicita non è quella che va a contenere il risultato, ma quella che contiene il dividendo.
Negli esempi, le destinazioni quoziente resto sono rispettivamente AL e AH, EAX e EDX.

\subsubsection{INTEGER DIVIDE}
\begin{itemize}
	\item \textbf{Formato:} \lstinline|IMUL source|
	\item \textbf{Azione:} considera l'operando sorgente come un divisore, l'operando destinatario (implicito) come un dividendo, e effettua la divisione assumendo i numeri interi. Nello specifico:
	\begin{itemize}
	\item Sorgente a 8 bit, si ha $\text{AL} = \text{AX} / \text{source}$, e $ \text{AH} = \text{AX} \mod \text{source} $;
	\item Sorgente a 16 bit, si ha $\text{AX} = \text{DX\_AX} / \text{source}$, e $ \text{DX} = \text{DX\_AX} \mod \text{source} $;
	\item Sorgente a 32 bit, si ha $\text{EAX} = \text{EDX\_EAX} / \text{source}$, e $ \text{EDX} = \text{EDX\_EAX} \mod \text{source} $;
	\end{itemize}
	\item \textbf{Flag:} li imposta tutti, ma non è attendibile.
\end{itemize}

		\begin{table}[H]
		\center \rowcolors{2}{white}{black!10}
			\begin{tabular} { c | p{10cm} }
				\bfseries Operandi & \bfseries Esempi \\
				\hline
				Memoria & \lstinline|IDIVB (\%ESI)	\# AX destinazione implicita| \\ 
				Registro Generale & \lstinline|IDIV \%ECX	\# EDX\_EAX destinazione implicita|
			\end{tabular}
		\end{table}

Bisogna stare attenti ai segni della divisione intera.
Nella divisione intera il resto ha sempre il segno del dividendo, ed è minore in modulo del divisore.
Ciò significa che il quoziente si approssima sempre all'intero più vicino allo zero (\textit{per troncamento}).
Ad esempio, $-7 \ \mathrm{idiv} \ 3 = -2, -1$ e $7 \ \mathrm{idiv} \ -3 = -2, +1$.

\par\medskip
\noindent
\textbf{\textsf{Funzionamento delle DIVIDE e INTEGER DIVIDE}} \\
Esistono quindi, come per le moltiplicazioni, tre tipi di divisione, con operando e destinatario impliciti:
\begin{itemize}
	\item Sorgente a 8 bit, si ha $\text{AL} = \text{AX} / \text{source}$, e $ \text{AH} = \text{AX} \mod \text{source} $;
	\item Sorgente a 16 bit, si ha $\text{AX} = \text{DX\_AX} / \text{source}$, e $ \text{DX} = \text{DX\_AX} \mod \text{source} $;
	\item Sorgente a 32 bit, si ha $\text{EAX} = \text{EDX\_EAX} / \text{source}$, e $ \text{EDX} = \text{EDX\_EAX} \mod \text{source} $;
\end{itemize}

In tabella questo significa:

\begin{table}[h!]
	\center \rowcolors{2}{white}{black!10}
	\begin{tabular} { c | c | c | c | c }
		\bfseries Dim. sorgente (divisore) & \bfseries Dim. dividendo & \bfseries Dividendo & \bfseries Quoziente & \bfseries Resto \\ 
		\hline 
		8 bit & 16 bit & AX & AL & AH \\ 
		16 bit & 32 bit & DX\_AX & AX & DX \\ 
		32 bit & 64 bit & EDX\_EAX & EAX & EDX
	\end{tabular}
\end{table}

Se il quoziente non sta nel numero di bit previsto, viene sollevata un'eccezione, e il programma va in HALT.
Bisogna quindi decidere quali versioni usare tenendo conto delle dimensioni dei possibili quoziente.
Questo è importante in quanto non è cosi raro avere divisioni dove il quoziente non sta nella metà dei bit del dividendo, ad esempio:

\begin{lstlisting}[language=assembler,style=codestyle]	
MOV $3, %CL
MOV $15000, %AX
DIV %CL	# come metto 5000 su una locazione da 8 bit?
\end{lstlisting}

per risolvere il problema, dobbiamo costringere il processore ad usare un altro tipo di divisione, quindi:
\begin{lstlisting}[language=assembler,style=codestyle]	
MOV $3, %CX
MOV $15000, %AX
MOV $0, %DX	# devo ripulire DX, verra' usato il dividendo DX_AX 
DIV %CX	# il risultato va in AX, tutto bene 
\end{lstlisting}

\subsection{Note conclusive su moltiplicazioni e divisioni}
Dobbiamo quindi ricordarci, riguardo a moltiplicazioni e divisioni, di:
\begin{itemize}
	\item Scegliere con cura la versione che usiamo (sopratutto nel caso di divisioni dove il quoziente potrebbe non stare nella metà del numero di bit del dividendo);
	\item Azzerare di azzerare i registri DX o EDX prima della divisione, se è a più di 8 bit;
	\item Ricordare che il contenuto di DX o EDX viene modificato per operazioni su più di 8 bit.
\end{itemize}

\subsection{Estensione di campo}
Attraverso l'estensione di campo si rappresenta lo stesso numero su più cifre.
Questo è banale sui naturali (si aggiunge uno zero), ma più complicato per gli interi.
In questo caso si estende con il bit più significativo (quello di segno).

\subsubsection{CONVERT BYTE TO WORD}
\begin{itemize}
	\item \textbf{Formato:} \lstinline|CBX|
	\item \textbf{Azione:} interpreta il contenuto di AL come un numero intero a 8 bit, la rappresenta su 16 bit e quindi lo memorizza in AX.
	\item \textbf{Flag:} nessuno.
\end{itemize}

\subsubsection{CONVERT WORD TO DOUBLEWORD}
\begin{itemize}
	\item \textbf{Formato:} \lstinline|CWDE|
	\item \textbf{Azione:} interpreta il contenuto di AX come un numero intero a 16 bit, la rappresenta su 32 bit e quindi lo memorizza in EAX.
	\item \textbf{Flag:} nessuno.
\end{itemize}

Poniamo ad esempio di voler sommare due interi, uno in AX e l'altro in EBX:
\begin{lstlisting}[language=assembler,style=codestyle]	
MOV $-5, %AX
MOV $100000, $EBX
CWDE
ADD %EAX, %EBX
\end{lstlisting}

\subsection{Istruzioni di traslazione e rotazione}
Queste istruzioni variano l'ordine dei bit in un operando destinatario.
Hanno due formati: \lstinline|OCPODE source, destination| o \lstinline|OPCODE destination|.

Quando si specifica un sorgente, esso rappresenta il numero di iterazioni per cui si ripete l'operazione.
Il sorgente può essere ad indirizzamento immediato o essere il registro CL.
Inoltre, deve essere $\leq 31$ (sarebbe inutile fare $\geq32$ trasformazioni di 32 bit).
Quando è omesso, il sorgente vale di default 1.

\subsubsection{SHIFT LOGICAL LEFT}
\begin{itemize}
	\item \textbf{Formato:} \lstinline|SHL source, destination|
	\item \textbf{Azione:} interpreta l'operando sorgente come un naturale $n$, e per $n$ iterazioni:
		\begin{itemize}
			\item Sostituisce il bit in CF con il MSB;
			\item Sostituisce ogni bit (tranne il LSB) con il bit immediatamente a destra  (il meno significativo);
			\item Sostituisce il LSB con 0.
		\end{itemize}
	\item \textbf{Flag:} nessuno.
\end{itemize}

		\begin{table}[H]
		\center \rowcolors{2}{white}{black!10}
			\begin{tabular} { c | p{5cm} }
				\bfseries Operandi & \bfseries Esempi \\
				\hline
				Immediato, Registro Generale & \lstinline|SHL \$1, \%EAX| \\
				Immediato, Memoria & \lstinline|SHLB \$7, 0x00002000| \\
				Registro CL, Registro Generale & \lstinline|SHL \%CL, \%EAX| \\
				Registro CL, Memoria & \lstinline|SHLL \%CL, (\%EDI)| \\
				Memoria & \lstinline|SHLL (\%EDI)| \\ 
				Registro Generale & \lstinline|SHL \%AX|
			\end{tabular}
		\end{table}

La SHL è utile per effettuare moltiplicazioni per 2 (shift a sinistra in binario significa $\times 2$), tranne nei casi in cui il prodotto non sta sul numero di bit del destinatario.

Per questo si controlla il CF, facendo però attenzione che per $n$ iterazioni (date dal sorgente) vengono effettuati $n$ sovrascrizioni del CF.
Ergo, se la moltiplicazione fallisce, non sappiamo \textit{quando} fallisce.

\subsubsection{SHIFT ARITHMETIC LEFT}
\begin{itemize}
	\item \textbf{Formato:} \lstinline|SAL source, destination|
	\item \textbf{Azione:} è identica alla SHL. 
		Quindi equivale a moltiplicare per $2^\text{source}$.
	\item \textbf{Flag:} nessuno.
\end{itemize}

Esiste come duale della SAR, ma in questo caso non deve fare nulla di diverso dalla SHL.

\subsubsection{SHIFT LOGICAL RIGHT}
\begin{itemize}
	\item \textbf{Formato:} \lstinline|SHR source, destination|
	\item \textbf{Azione:} interpreta l'operando sorgente come un naturale $n$, e per $n$ iterazioni:
		\begin{itemize}
			\item Sostituisce il bit in CF con il LSB;
			\item Sostituisce ogni bit (tranne il MSB) con il bit immediatamente a sinistra (il più significativo);
			\item Sostituisce il MSB con 0.
		\end{itemize}
	\item \textbf{Flag:} nessuno.
\end{itemize}

		\begin{table}[H]
		\center \rowcolors{2}{white}{black!10}
			\begin{tabular} { c | p{5cm} }
				\bfseries Operandi & \bfseries Esempi \\
				\hline
				Immediato, Registro Generale & \lstinline|SHR \$1, \%EAX| \\
				Immediato, Memoria & \lstinline|SHRB \$7, 0x00002000| \\
				Registro CL, Registro Generale & \lstinline|SHR \%CL, \%EAX| \\
				Registro CL, Memoria & \lstinline|SHRL \%CL, (\%EDI)| \\
				Memoria & \lstinline|SHRL (\%EDI)| \\ 
				Registro Generale & \lstinline|SHR \%AX|
			\end{tabular}
		\end{table}

La SHR, come la SHL, è utile per effettuare divisioni per 2 (shift a destra in binario significa $\div 2$), concessa approssimazione del bit perso, tranne nei casi in cui il numero è un intero (lo 0 al MSB corrompe il segno). 
Per questo motivo si definisce la:

\subsubsection{SHIFT ARITHMETIC RIGHT}
\begin{itemize}
	\item \textbf{Formato:} \lstinline|SAR source, destination|
	\item \textbf{Azione:} è identica alla SHR, ma non sostituisce il MSB con 0, lasciandolo tale.
		Questo equivale a dividere per $2^\text{source}$.
	\item \textbf{Flag:} nessuno.
\end{itemize}

La SAR ci permette di dividere velocemente interi per 2, come avremmo fatto sui naturali con la SHR.

\subsubsection{Divisioni intere}
Le IDIV e SAR approssimano diversamente: la IDIV approssima per troncamento, mentre la SAR approssima sempre a sinistra.
Quindi, IDIV e SAR danno lo stesso quoziente solo quando il dividendo è positivo, o il resto nullo.

\subsection{Istruzioni di rotazione}
Le istruzioni di rotazione ruotano i bit, cioè effettuano uno shift con rientro dei bit in uscita dal lato opposto, con la possibilità di includere o meno CF nella rotazione.

\subsubsection{ROTATE LEFT}
\begin{itemize}
	\item \textbf{Formato:} \lstinline|ROL source, destination|
	\item \textbf{Azione:} interpreta l'operando sorgente come un naturale $n$, e per $n$ iterazioni ruota verso sinistra senza usare il carry.
	\item \textbf{Flag:} nessuno.
\end{itemize}

		\begin{table}[H]
		\center \rowcolors{2}{white}{black!10}
			\begin{tabular} { c | p{5cm} }
				\bfseries Operandi & \bfseries Esempi \\
				\hline
				Immediato, Registro Generale & \lstinline|ROL \$1, \%EAX| \\
				Immediato, Memoria & \lstinline|ROLB \$7, 0x00002000| \\
				Registro CL, Registro Generale & \lstinline|ROL \%CL, \%EAX| \\
				Registro CL, Memoria & \lstinline|ROLL \%CL, (\%EDI)| \\
				Memoria & \lstinline|ROLL (\%EDI)| \\ 
				Registro Generale & \lstinline|ROL \%AX|
			\end{tabular}
		\end{table}

\subsubsection{ROTATE RIGHT}
\begin{itemize}
	\item \textbf{Formato:} \lstinline|ROR source, destination|
	\item \textbf{Azione:} interpreta l'operando sorgente come un naturale $n$, e per $n$ iterazioni ruota verso destra senza usare il carry.
	\item \textbf{Flag:} nessuno.
\end{itemize}

		\begin{table}[H]
		\center \rowcolors{2}{white}{black!10}
			\begin{tabular} { c | p{5cm} }
				\bfseries Operandi & \bfseries Esempi \\
				\hline
				Immediato, Registro Generale & \lstinline|ROR \$1, \%EAX| \\
				Immediato, Memoria & \lstinline|RORB \$7, 0x00002000| \\
				Registro CL, Registro Generale & \lstinline|ROR \%CL, \%EAX| \\
				Registro CL, Memoria & \lstinline|RORL \%CL, (\%EDI)| \\
				Memoria & \lstinline|RORL (\%EDI)| \\ 
				Registro Generale & \lstinline|ROR \%AX|
			\end{tabular}
		\end{table}

\subsubsection{ROTATE CARRY LEFT}
\begin{itemize}
	\item \textbf{Formato:} \lstinline|RCL source, destination|
	\item \textbf{Azione:} interpreta l'operando sorgente come un naturale $n$, e per $n$ iterazioni ruota verso sinistra usando il carry.
	\item \textbf{Flag:} imposta il carry assumendolo a sinistra del MSB.
\end{itemize}

		\begin{table}[H]
		\center \rowcolors{2}{white}{black!10}
			\begin{tabular} { c | p{5cm} }
				\bfseries Operandi & \bfseries Esempi \\
				\hline
				Immediato, Registro Generale & \lstinline|RCL \$1, \%EAX| \\
				Immediato, Memoria & \lstinline|RCLB \$7, 0x00002000| \\
				Registro CL, Registro Generale & \lstinline|RCL \%CL, \%EAX| \\
				Registro CL, Memoria & \lstinline|RCLL \%CL, (\%EDI)| \\
				Memoria & \lstinline|RCLL (\%EDI)| \\ 
				Registro Generale & \lstinline|RCL \%AX|
			\end{tabular}
		\end{table}

\subsubsection{ROTATE CARRY RIGHT}
\begin{itemize}
	\item \textbf{Formato:} \lstinline|RCR source, destination|
	\item \textbf{Azione:} interpreta l'operando sorgente come un naturale $n$, e per $n$ iterazioni ruota verso destra usando il carry.
	\item \textbf{Flag:} imposta il carry assumendolo a destra del LSB.
\end{itemize}

		\begin{table}[H]
		\center \rowcolors{2}{white}{black!10}
			\begin{tabular} { c | p{5cm} }
				\bfseries Operandi & \bfseries Esempi \\
				\hline
				Immediato, Registro Generale & \lstinline|RCR \$1, \%EAX| \\
				Immediato, Memoria & \lstinline|RCRB \$7, 0x00002000| \\
				Registro CL, Registro Generale & \lstinline|RCR \%CL, \%EAX| \\
				Registro CL, Memoria & \lstinline|RCRL \%CL, (\%EDI)| \\
				Memoria & \lstinline|RCRL (\%EDI)| \\ 
				Registro Generale & \lstinline|RCR \%AX|
			\end{tabular}
		\end{table}

\subsection{Istruzioni logiche}
Queste istruzioni applicano gli operatori dell'algebra di Boole, e solitamente modificano flag.

\subsubsection{NOT}
\begin{itemize}
	\item \textbf{Formato:} \lstinline|NOT destination|
	\item \textbf{Azione:} modifica il destinatario applicandogli il NOT bit a bit. 
	\item \textbf{Flag:} nessuno. 
\end{itemize}

		\begin{table}[H]
		\center \rowcolors{2}{white}{black!10}
			\begin{tabular} { c | p{5cm} }
				\bfseries Operandi & \bfseries Esempi \\
				\hline
				Memoria & \lstinline|NOTL (\%ESI)| \\ 
				Registro Generale & \lstinline|NOT \%CX| 
			\end{tabular}
		\end{table}

\subsubsection{AND}
\begin{itemize}
	\item \textbf{Formato:} \lstinline|AND source, destination|
	\item \textbf{Azione:} modifica il destinatario applicando l'AND bit a bit degli operandi. 
	\item \textbf{Flag:} modifica tutti i flag (annulla CF e OF).
\end{itemize}

		\begin{table}[H]
		\center \rowcolors{2}{white}{black!10}
			\begin{tabular} { c | p{5cm} }
				\bfseries Operandi & \bfseries Esempi \\
				\hline
				Memoria, Registro Generale & \lstinline|AND 0x00002000, \%EDX| \\ 
				Registro Generale, Memoria & \lstinline|AND \%CL, 0x12AB1024| \\ 
				Registro Generale, Registro Generale & \lstinline|AND \%AX, \%DX| \\ 
				Immediato, Memoria & \lstinline|AND 5x5B, (\%EDI)| \\ 
				Immediato, Registro Generale & \lstinline|AND \$0x45AB54A3, \%EAX|
			\end{tabular}
		\end{table}

\subsubsection{OR}
\begin{itemize}
	\item \textbf{Formato:} \lstinline|OR source, destination|
	\item \textbf{Azione:} modifica il destinatario applicando l'OR bit a bit degli operandi. 
	\item \textbf{Flag:} modifica tutti i flag (annulla CF e OF).
\end{itemize}

		\begin{table}[H]
		\center \rowcolors{2}{white}{black!10}
			\begin{tabular} { c | p{5cm} }
				\bfseries Operandi & \bfseries Esempi \\
				\hline
				Memoria, Registro Generale & \lstinline|OR 0x00002000, \%EDX| \\ 
				Registro Generale, Memoria & \lstinline|OR \%CL, 0x12AB1024| \\ 
				Registro Generale, Registro Generale & \lstinline|OR \%AX, \%DX| \\ 
				Immediato, Memoria & \lstinline|OR 5x5B, (\%EDI)| \\ 
				Immediato, Registro Generale & \lstinline|OR \$0x45AB54A3, \%EAX|
			\end{tabular}
		\end{table}

\subsubsection{XOR}
\begin{itemize}
	\item \textbf{Formato:} \lstinline|XOR source, destination|
	\item \textbf{Azione:} modifica il destinatario applicando l'OR bit a bit degli operandi. 
	\item \textbf{Flag:} modifica tutti i flag (annulla CF e OF).
\end{itemize}

		\begin{table}[H]
		\center \rowcolors{2}{white}{black!10}
			\begin{tabular} { c | p{5cm} }
				\bfseries Operandi & \bfseries Esempi \\
				\hline
				Memoria, Registro Generale & \lstinline|XOR 0x00002000, \%EDX| \\ 
				Registro Generale, Memoria & \lstinline|XOR \%CL, 0x12AB1024| \\ 
				Registro Generale, Registro Generale & \lstinline|XOR \%AX, \%DX| \\ 
				Immediato, Memoria & \lstinline|XOR 5x5B, (\%EDI)| \\ 
				Immediato, Registro Generale & \lstinline|XOR \$0x45AB54A3, \%EAX|
			\end{tabular}
		\end{table}

\subsubsection{Uso delle istruzioni logiche}
Le istruzioni logiche vengono usate per operare su singoli bit degli operandi, usando uno specifico operatore sorgente immediato detto maschera (\textbf{bitmask}).
Nello specifico:
\begin{itemize}
	\item \textbf{AND:} 
		\begin{itemize}
			\item si usa per testare singoli bit di un operando.
			Ad esempio, si può implementare un salto condizionale se il quinto bit di AL vale zero:
			\begin{lstlisting}[language=assembler,style=codestyle]	
AND $0x20, %AL	# 0x20 = 00100000
JZ # vale zero
\end{lstlisting} 
			\item si usa per resettare singoli bit di un operando.
			Ad esempio, si può resettare il sesto bit di BH:
			\begin{lstlisting}[language=assembler,style=codestyle]	
AND $0xBF, $BH	# 0xBF = 10111111
\end{lstlisting}
			\item si usa per l'estensione di operandi \textit{naturali}.
				Ad esempio, si possono sommare due numeri naturali, di cui uno in AL e l'altro in EBX:
				\begin{lstlisting}[language=assembler,style=codestyle]	
MOV $5, $AL
MOV $100000, %EBX
AND $0x000000FF, $EAX
ADD %EAX, %EBX	
\end{lstlisting}
		\end{itemize} 
	\item \textbf{OR:} si usa per settare singoli bit di un operando.
		Ad esempio, si può settare il quarto bit di CL:
		\begin{lstlisting}[language=assembler,style=codestyle]	
OR $0x10, %CL	# =x10 = 00010000
\end{lstlisting}
	\item \textbf{XOR:}
		\begin{itemize}
			\item si usa per invertire singoli bit.
		Ad esempio, si può invertire il quinto bit del registro AH:
		\begin{lstlisting}[language=assembler,style=codestyle]	
XOR $0x20, %AH	# 0x20 = 00100000
\end{lstlisting}
	\item si usa per resettare registri.
		Ad esempio, si può resettare EAX come:
		\begin{lstlisting}[language=assembler,style=codestyle]	
XOR %EAX, %EAX	# equivale a dire MOV $0, %EAX, ma occupa 
								# 1 byte invece di 5
\end{lstlisting}
		\end{itemize}
\end{itemize}

\subsection{Istruzioni di controllo}
Le istruzioni di controllo permettono di alterare il flusso del programma, che altrimenti scorrerebbe normalmente in sequenza (le istruzioni vengono eseguite come vengono lette in memoria).

Conosciamo il ciclo fetch-execute: il processore carica un'istruzione, incrementa EIP, e la esegue.
Alcune istruzioni alterano il valore di EIP, implementando quindi alterazioni del flusso di esecuzione:
\begin{itemize}
	\item \textbf{Istruzioni di salto:} JMP, Jcon;
	\item \textbf{Istruzioni di gestione sottoprgrammi}: CALL, RET.
\end{itemize}

\subsubsection{JUMP}
\begin{itemize}
	\item \textbf{Formato:} \lstinline|JMP  \%EIP +/- displacement|, \lstinline|JMP *extended\_register|, \lstinline|JMP *memory|
	\item \textbf{Azione:} calcola un'indrizzo di salto e lo immette nel registro EIP. 
	\item \textbf{Flag:} nessuno. 
\end{itemize}

Solitamente le istruzioni di salto si riferiscono ad un nome simbolico, ed è quindi compito dell'assemblatore ricondurre la sintassi ad una delle forme sopra riportate.

\subsubsection{JUMP if CONDITION MET}
\begin{itemize}
	\item \textbf{Formato:} \lstinline|Jcon \%EIP +/- displacement|
	\item \textbf{Azione:} esamina il contenuto dei flag.
		Se da questo esame risulta che la condizione \textit{con} è soddisfatta, si comporta come \lstinline|JMP \%EIP +/- displacement|, altrimenti non fa nulla.
	\item \textbf{Flag:} nessuno. 
\end{itemize}

I prossimi paragrafi riguardano tutti i di condizione supportati.

\subsubsection{Condizioni sui flag}
Esistono le seguenti condizioni sui singoli flag:

\begin{table}[h!]
	\center \rowcolors{1}{white}{black!5}
	\begin{tabular} { c  p{10cm} }
		\bfseries Condizione & \bfseries Funzionamento \\
		\hline 
		JZ & Jump If Zero, la condizione è soddisfatta se ZF è impostato, ergo se il risultato dell'istruzione precedente è stato 0. \\ 
		JNZ & Jump If Not Zero, la condizione è soddisfatta se ZF non è impostato, ergo se il risultato dell'istruzione precedente non è stato 0. \\ 
		JC & Jump if Carry, la condizione è soddisfatta se CF è impostato. \\
		JNC & Jump if No Carry, la condizione è soddisfatta se CF non è impostato. \\ 
		JO & Jump if Overflow, la condizione è soddisfatta se OF è impostato. \\
		JNO & Jump if No Overflow, la condizione è soddisfatta se OF non è impostato. \\ 
		JS & Jump if Sign, la condizione è soddisfatta se SF è impostato. \\
		JNS & Jump if No Sign, la condizione è soddisfatta se SF non è impostato. \\ 
	\end{tabular}
\end{table}

\par\medskip
\noindent
\textbf{\textsf{Esempi}} \\
\begin{itemize}
	\item 
\begin{lstlisting}[language=assembler,style=codestyle]	
ADD %AX, %BX
JC ...
# continua
\end{lstlisting}
Se la somma dei contenuti di AX e BX presi come naturali non è rappresentabile su 16 bit, salta.

	\item 
\begin{lstlisting}[language=assembler,style=codestyle]	
ADD %AX, %BX
JO ...
# continua
\end{lstlisting}
Se la somma dei contenuti di AX e BX presi come interi non è rappresentabile su 16 bit, salta.

	\item 
\begin{lstlisting}[language=assembler,style=codestyle]	
SUB %AL, %BL
JS ...
# continua
\end{lstlisting}
Se la somma differenza dei contenuti di BL ed AL (in quest'ordine) presi come interi è negativa, salta.
\end{itemize}

\subsubsection{Condizioni sui naturali}
Esistono le seguenti condizioni sui confronti fra naturali:

\begin{table}[h!]
	\center \rowcolors{2}{white}{black!5}
	\begin{tabular} { c  p{10cm} }
		\bfseries Condizione & \bfseries Funzionamento \\
		\hline 
		JE & Jump if Equal, la condizione è soddisfatta se ZF contiene 1, cioè dopo CMP su due numeri uguali. \\
		JNE & Jump if Not Equal, la condizione è soddisfatta se ZF contiene 0, cioè dopo CMP su due numeri non uguali. \\ 
		JA & Jump if Above, la condizione è soddisfatta se CF contiene 0 e ZF contiene 0, cioè dopo CMP su un destinatario maggiore del sorgente. \\
		JAE & Jump if Above or Equal, la condizione è soddisfatta se CF contiene 0, cioè dopo CMP su un destinatario maggiore o uguale del sorgente. \\ 
		JB & Jump if Below, la condizione è soddisfatta se CF contiene 1, cioè dopo CMP su un destinatario minore del sorgente. \\
		JBE & Jump if Below or Equal, la condizione è soddisfatta se CF contiene 1 o ZF contiene 1, cioè dopo CMP su un destinatario minore o uguale del sorgente. \\ 
	\end{tabular}
\end{table}

Tutte queste condizioni seguono sempre una CMP, che aggiorna i flag in modo da permettere il confronto.
I risultati dei confronti possono sempre evincersi dai flag.

\par\medskip
\noindent
\textbf{\textsf{Esempi}} \\
\begin{itemize}
	\item 
\begin{lstlisting}[language=assembler,style=codestyle]	
CMP %AX, %BX
JAE ...
# continua
\end{lstlisting}
Se BX è maggiore o uguale di AX, presi come naturali, salta.

	\item 
\begin{lstlisting}[language=assembler,style=codestyle]	
CMP %EDX, %ECX
JB ...
# continua
\end{lstlisting}
Se ECX è minore stretto di EDX, presi come naturali, salta.
\end{itemize}

\subsubsection{Condizioni sugli interi}
Esistono le seguenti condizioni sui confronti fra interi:

\begin{table}[h!]
	\center \rowcolors{2}{white}{black!5}
	\begin{tabular} { c  p{10cm} }
		\bfseries Condizione & \bfseries Funzionamento \\
		\hline 
		JE & Jump if Equal, la condizione è soddisfatta se ZF contiene 1, cioè dopo CMP su due numeri uguali. \\
		JNE & Jump if Not Equal, la condizione è soddisfatta se ZF contiene 0, cioè dopo CMP su due numeri non uguali. \\ 
		JG & Jump if Greater, la condizione è soddisfatta se ZF contiene 0 e se SF è uguale a OF, cioè dopo CMP su un destinatario maggiore del sorgente. \\
		JGE & Jump if Greater or Equal, la condizione è soddisfatta se SF è uguale a OF, cioè dopo CMP su un destinatario maggiore o uguale del sorgente. \\ 
		JL & Jump if Less, la condizione è soddisfatta se SF è diverso da OF, cioè dopo CMP su un destinatario minore del sorgente. \\
		JLE & Jump if Less or Equal, la condizione è soddisfatta se ZF contiene 1 o se Sf è diverso da OF, cioè dopo CMP su un destinatario minore o uguale del sorgente. \\ 
	\end{tabular}
\end{table}

Come prima, queste operazioni seguono sempre una CMP ed evincono il risultato del confronto dai flag.

\par\medskip
\noindent
\textbf{\textsf{Esempi}} \\
\begin{itemize}
	\item 
\begin{lstlisting}[language=assembler,style=codestyle]	
CMP %AX, %BX
JGE ...
# continua
\end{lstlisting}
Se BX è maggiore o uguale di AX, presi come interi, salta.

	\item 
\begin{lstlisting}[language=assembler,style=codestyle]	
CMP %EDX, %ECX
JL ...
# continua
\end{lstlisting}
Se ECX è minore stretto di EDX, presi come interi, salta.
\end{itemize}
\end{document}


\documentclass[a4paper,11pt]{article}
\usepackage[a4paper, margin=8em]{geometry}

% usa i pacchetti per la scrittura in italiano
\usepackage[french,italian]{babel}
\usepackage[T1]{fontenc}
\usepackage[utf8]{inputenc}
\frenchspacing 

% usa i pacchetti per la formattazione matematica
\usepackage{amsmath, amssymb, amsthm, amsfonts}

% usa altri pacchetti
\usepackage{gensymb}
\usepackage{hyperref}
\usepackage{standalone}

% imposta il titolo
\title{Appunti Reti Logiche}
\author{Luca Seggiani}
\date{2024}

% imposta lo stile
% usa helvetica
\usepackage[scaled]{helvet}
% usa palatino
\usepackage{palatino}
% usa un font monospazio guardabile
\usepackage{lmodern}

\renewcommand{\rmdefault}{ppl}
\renewcommand{\sfdefault}{phv}
\renewcommand{\ttdefault}{lmtt}

% disponi il titolo
\makeatletter
\renewcommand{\maketitle} {
	\begin{center} 
		\begin{minipage}[t]{.8\textwidth}
			\textsf{\huge\bfseries \@title} 
		\end{minipage}%
		\begin{minipage}[t]{.2\textwidth}
			\raggedleft \vspace{-1.65em}
			\textsf{\small \@author} \vfill
			\textsf{\small \@date}
		\end{minipage}
		\par
	\end{center}

	\thispagestyle{empty}
	\pagestyle{fancy}
}
\makeatother

% disponi teoremi
\usepackage{tcolorbox}
\newtcolorbox[auto counter, number within=section]{theorem}[2][]{%
	colback=blue!10, 
	colframe=blue!40!black, 
	sharp corners=northwest,
	fonttitle=\sffamily\bfseries, 
	title=Teorema~\thetcbcounter: #2, 
	#1
}

% disponi definizioni
\newtcolorbox[auto counter, number within=section]{definition}[2][]{%
	colback=red!10,
	colframe=red!40!black,
	sharp corners=northwest,
	fonttitle=\sffamily\bfseries,
	title=Definizione~\thetcbcounter: #2,
	#1
}

% disponi codice
\usepackage{listings}
\usepackage[table]{xcolor}

\definecolor{codegreen}{rgb}{0,0.6,0}
\definecolor{codegray}{rgb}{0.5,0.5,0.5}
\definecolor{codepurple}{rgb}{0.58,0,0.82}
\definecolor{backcolour}{rgb}{0.95,0.95,0.92}

\lstdefinestyle{codestyle}{
		backgroundcolor=\color{black!5}, 
		commentstyle=\color{codegreen},
		keywordstyle=\bfseries\color{magenta},
		numberstyle=\sffamily\tiny\color{black!60},
		stringstyle=\color{green!50!black},
		basicstyle=\ttfamily\footnotesize,
		breakatwhitespace=false,         
		breaklines=true,                 
		captionpos=b,                    
		keepspaces=true,                 
		numbers=left,                    
		numbersep=5pt,                  
		showspaces=false,                
		showstringspaces=false,
		showtabs=false,                  
		tabsize=2
}

\lstdefinestyle{shellstyle}{
		backgroundcolor=\color{black!5}, 
		basicstyle=\ttfamily\footnotesize\color{black}, 
		commentstyle=\color{black}, 
		keywordstyle=\color{black},
		numberstyle=\color{black!5},
		stringstyle=\color{black}, 
		showspaces=false,
		showstringspaces=false, 
		showtabs=false, 
		tabsize=2, 
		numbers=none, 
		breaklines=true
}


\lstdefinelanguage{assembler}{
  keywords={AAA, AAD, AAM, AAS, ADC, ADCB, ADCW, ADCL, ADD, ADDB, ADDW, ADDL, AND, ANDB, ANDW, ANDL,
        ARPL, BOUND, BSF, BSFL, BSFW, BSR, BSRL, BSRW, BSWAP, BT, BTC, BTCB, BTCW, BTCL, BTR, 
        BTRB, BTRW, BTRL, BTS, BTSB, BTSW, BTSL, CALL, CBW, CDQ, CLC, CLD, CLI, CLTS, CMC, CMP,
        CMPB, CMPW, CMPL, CMPS, CMPSB, CMPSD, CMPSW, CMPXCHG, CMPXCHGB, CMPXCHGW, CMPXCHGL,
        CMPXCHG8B, CPUID, CWDE, DAA, DAS, DEC, DECB, DECW, DECL, DIV, DIVB, DIVW, DIVL, ENTER,
        HLT, IDIV, IDIVB, IDIVW, IDIVL, IMUL, IMULB, IMULW, IMULL, IN, INB, INW, INL, INC, INCB,
        INCW, INCL, INS, INSB, INSD, INSW, INT, INT3, INTO, INVD, INVLPG, IRET, IRETD, JA, JAE,
        JB, JBE, JC, JCXZ, JE, JECXZ, JG, JGE, JL, JLE, JMP, JNA, JNAE, JNB, JNBE, JNC, JNE, JNG,
        JNGE, JNL, JNLE, JNO, JNP, JNS, JNZ, JO, JP, JPE, JPO, JS, JZ, LAHF, LAR, LCALL, LDS,
        LEA, LEAVE, LES, LFS, LGDT, LGS, LIDT, LMSW, LOCK, LODSB, LODSD, LODSW, LOOP, LOOPE,
        LOOPNE, LSL, LSS, LTR, MOV, MOVB, MOVW, MOVL, MOVSB, MOVSD, MOVSW, MOVSX, MOVSXB,
        MOVSXW, MOVSXL, MOVZX, MOVZXB, MOVZXW, MOVZXL, MUL, MULB, MULW, MULL, NEG, NEGB, NEGW,
        NEGL, NOP, NOT, NOTB, NOTW, NOTL, OR, ORB, ORW, ORL, OUT, OUTB, OUTW, OUTL, OUTSB, OUTSD,
        OUTSW, POP, POPL, POPW, POPB, POPA, POPAD, POPF, POPFD, PUSH, PUSHL, PUSHW, PUSHB, PUSHA, 
				PUSHAD, PUSHF, PUSHFD, RCL, RCLB, RCLW,
        RCLL, RCR, RCRB, RCRW, RCRL, RDMSR, RDPMC, RDTSC, REP, REPE, REPNE, RET, ROL, ROLB, ROLW,
        ROLL, ROR, RORB, RORW, RORL, SAHF, SAL, SALB, SALW, SALL, SAR, SARB, SARW, SARL, SBB,
        SBBB, SBBW, SBBL, SCASB, SCASD, SCASW, SETA, SETAE, SETB, SETBE, SETC, SETE, SETG, SETGE,
        SETL, SETLE, SETNA, SETNAE, SETNB, SETNBE, SETNC, SETNE, SETNG, SETNGE, SETNL, SETNLE,
        SETNO, SETNP, SETNS, SETNZ, SETO, SETP, SETPE, SETPO, SETS, SETZ, SGDT, SHL, SHLB, SHLW,
        SHLL, SHLD, SHR, SHRB, SHRW, SHRL, SHRD, SIDT, SLDT, SMSW, STC, STD, STI, STOSB, STOSD,
        STOSW, STR, SUB, SUBB, SUBW, SUBL, TEST, TESTB, TESTW, TESTL, VERR, VERW, WAIT, WBINVD,
        XADD, XADDB, XADDW, XADDL, XCHG, XCHGB, XCHGW, XCHGL, XLAT, XLATB, XOR, XORB, XORW, XORL},
  keywordstyle=\color{blue}\bfseries,
  ndkeywordstyle=\color{darkgray}\bfseries,
  identifierstyle=\color{black},
  sensitive=false,
  comment=[l]{\#},
  morecomment=[s]{/*}{*/},
  commentstyle=\color{purple}\ttfamily,
  stringstyle=\color{red}\ttfamily,
  morestring=[b]',
  morestring=[b]"
}

\lstset{language=assembler, style=codestyle}

% disponi sezioni
\usepackage{titlesec}

\titleformat{\section}
	{\sffamily\Large\bfseries} 
	{\thesection}{1em}{} 
\titleformat{\subsection}
	{\sffamily\large\bfseries}   
	{\thesubsection}{1em}{} 
\titleformat{\subsubsection}
	{\sffamily\normalsize\bfseries} 
	{\thesubsubsection}{1em}{}

% tikz
\usepackage{tikz}

% float
\usepackage{float}

% grafici
\usepackage{pgfplots}
\pgfplotsset{width=10cm,compat=1.9}

% disponi alberi
\usepackage{forest}

\forestset{
	rectstyle/.style={
		for tree={rectangle,draw,font=\large\sffamily}
	},
	roundstyle/.style={
		for tree={circle,draw,font=\large}
	}
}

% disponi algoritmi
\usepackage{algorithm}
\usepackage{algorithmic}
\makeatletter
\renewcommand{\ALG@name}{Algoritmo}
\makeatother

% disponi numeri di pagina
\usepackage{fancyhdr}
\fancyhf{} 
\fancyfoot[L]{\sffamily{\thepage}}

\makeatletter
\fancyhead[L]{\raisebox{1ex}[0pt][0pt]{\sffamily{\@title \ \@date}}} 
\fancyhead[R]{\raisebox{1ex}[0pt][0pt]{\sffamily{\@author}}}
\makeatother

\begin{document}
% sezione (data)
\section{Lezione del 01-10-24}

% stili pagina
\thispagestyle{empty}
\pagestyle{fancy}

% testo
\subsection{Istruzioni per sottoprogrammi}
Nei sottoprogrammi vengono coninvolte due istruzioni \lstinline|CALL|, e \lstinline|RET|.
Entrambe si riferiscono alla pila.


\subsubsection{CALL}
\begin{itemize}
	\item \textbf{Formato:} \lstinline|CALL %EIP +/- $displacement|, \lstinline|CALL *extended_register|, \lstinline|CALL *memory| 
	\item \textbf{Azione:} effettua la chiamata di un sottoprogramma, ovvero:
		\begin{itemize}
			\item Salva il valore corrente di EIP nella pila;
			\item Modifica EIP come farebbe JMP.
		\end{itemize}
	\item \textbf{Flag:} nessuno.
\end{itemize}

		\begin{table}[H]
		\center \rowcolors{2}{white}{black!10}
			\begin{tabular} { c | p{5cm} }
				\bfseries Operandi & \bfseries Esempi \\
				\hline
				Displacement & \lstinline|CALL 0x00400010| \\ 
				Registro & \lstinline|CALL *%EAX| \\ 
				Memoria & \lstinline|CALL *0x00400010|
			\end{tabular}
		\end{table}

\subsubsection{RET}
\begin{itemize}
	\item \textbf{Formato:} \lstinline|RET|
	\item \textbf{Azione:} ritorna da un sottoprogramma, ovvero:
		\begin{itemize}
			\item Rimuove un long dalla pila;
			\item Lo inserisce in EIP.
		\end{itemize}
	\item \textbf{Flag:} nessuno.
\end{itemize}

\par\medskip 
Esistono poi altre istruzioni di controllo, ovvero:

\subsubsection{NOP}
\begin{itemize}
	\item \textbf{Formato:} \lstinline|NOP|
	\item \textbf{Azione:} è l'istruzione nulla. 
	\item \textbf{Flag:} nessuno.
\end{itemize}

\subsubsection{HLT}
\begin{itemize}
	\item \textbf{Formato:} \lstinline|HLT|
	\item \textbf{Azione:} arresta l'esecuzione fino al prossimo interrupt. 
	\item \textbf{Flag:} nessuno.
\end{itemize}

\subsubsection{HCF}
\begin{itemize}
	\item \textbf{Formato:} \lstinline|HCF|
	\item \textbf{Azione:} arresta l'esecuzione e causa l'autocombustione spontanea del processore. 
	\item \textbf{Flag:} nessuno.
\end{itemize}

\subsection{Istruzioni privilegiate}
Il codice in assembler può girare secondo due modalità sul sistema:
\begin{itemize}
	\item \textbf{Sistema:} con accesso totale a tutte le istruzioni;
	\item \textbf{Utente:} senza l'accesso ad alcune istruzioni dette privilegiate.
\end{itemize}

Tra le istruzioni privilegiate ci sono \lstinline|HLT|, \lstinline|IN| e \lstinline|OUT|.
La \lstinline|HLT| non è un grande problema, ma lo sono \lstinline|IN| e \lstinline|OUT|.
Per ottenere input e output dal sistema, adoperiamo quindi determinati sottoprogrammi di servizio atti a fornire esattamente queste informazioni.

L'uso di sottoprogrammi di servizio per l'input/output è dovuto al fatto che le interfacce sono sistemi complessi, facili da portare in stato inconsistente, mentre i sottoprogrammi si assicurano di farne un corretto uso.

\subsection{Struttura di un programma assembler}
Vediamo adesso come strutturare un programma assembler scritto nell'ambiente GAS (Gnu Assembler).
Un programma assembler è diviso in due sezioni
\begin{itemize}
	\item \textbf{Sezione dati:} qui si dichiarano le variabili, ergo nomi simbolici per indirizzi di memoria che contengono i dati del programma;
	\item \textbf{Sezione codice:} istruzioni.


In un programma abbiamo bisogno di:
\end{itemize}


\begin{itemize}
	\item \textbf{Istruzioni}, viste finora;
	\item \textbf{Direttive}, necessarie all'assemblaggio e alla dichiarazione di variabili.
\end{itemize}

Ad esempio, potremo avere:
\begin{lstlisting}	
.GLOBAL _main

.DATA
...

.TEXT
_main:	NOP
...
				RET
\end{lstlisting}

Le linee che iniziano col punto sono direttive, le altre istruzioni.
Una riga qualsiasi del codice è fatta come:
\begin{lstlisting}	
nome:	OPCODE operandi # commento [\CR]
\end{lstlisting}
dove abbiamo una label, l'istruzione e un commento.

Tutto qui può mancare, tranne il ritorno carrello.
Tutte le righe, inclusa l'ultima, vanno terminate.
Inoltre, l'ultima riga dovrebbe essere una RET, che restituisce l'esecuzione al chiamante (qui l'ambiente).

Conviene iniziare il programma con una NOP, per assicurarsi che in fase di inizializzazione esso non faccia effettivamente nulla.

Vediamo ad esempio il programma visto prima per il conteggio degli uni, reso in questa struttura:
\begin{lstlisting} 
.GLOBAL _main
.DATA
dato:				.LONG 0x0F0F0101
conteggio:	.BYTE 0x00

.TEXT
_main:			NOP
						MOVB $0x00, %CL
						MOVL dato, %EAX
comp:				CMPL $0x00, %EAX
						JE fine
						SHRL %EAX
						ADCB $0x00, %CL
						JMP comp
fine:				MOVB %CL, conteggio
						RET
\end{lstlisting}

\subsubsection{Direttive}
Tutte le direttive iniziano con il carattere punto.
Esse sono:
\begin{itemize}
	\item \textbf{Dichiarazione di variabili:}
		Variabili dichiarate di seguito sono sempre consecutive in memoria. Si ha, di base:
		\begin{itemize}
			\item \lstinline|.BYTE|: riserva 1 byte;
			\item \lstinline|.WORD|: riserva 2 byte;
			\item \lstinline|.LONG|: riserva 4 byte.
		\end{itemize}
	\textsf{\textbf{Esempi}}
\begin{lstlisting}	
var0:	.WORD									# scalare, 2 byte, valore 0x0000 
														#	(considerato brutto, non inizializzare
														#	 si fa con .FILL)
var1: .BYTE 0x30						# scalare, 1 byte, valore 0x30
var2: .BYTE 0x30,0x31				# vettore, 2 componenti da 1 byte, 
														#	valore 0x30 e 0x31
var3:	.WORD 0x1020, 0x32AB	# vettore, 2 componenti da 2 byte, 
														#	valore 0x1020e 0x32AB
var4: .LONG var3+2					# scalare, 4 byte, valore 0xAB
\end{lstlisting}
\par\smallskip 
Esistono altri modi di inizializzare variabili particolari:
\begin{itemize}
	\item \lstinline|.FILL numero, dim, espressione|: dichiara \lstinline|numero| variabili di lunghezza \lstinline|dim| e le inizializza ad espression (0 di default).
		Dim può essere 1, 2 o 4.
	\item \textbf{ASCII}: si può usare la codifica ASCII fra single tick ', coi caratteri speciali dopo sequenze di escape, per indicare singoli byte. Ad esempio:
\begin{lstlisting}	
var5: .BYTE 'S', 'o', 'n', 'n', 'o'				# vettore, 4 componenti 
																					# da 1 byte
var6: .BYTE 0x53, 0x6F, 0x6E, 0x6E, 0x6F	# vettore, 4 componenti 
																					# da 1 byte
var7: .ASCII "Stea"												# vettore, 4 componenti
																					# da 1 byte
var8: .ASCIZ "Stea"												# vettore, 5 componenti 
																					# da 1 byte (include il 
																					# terminatore)
	\end{lstlisting} 
\end{itemize}
\item \textbf{Altre direttive:}
	\begin{itemize}
		\item \lstinline|.INCLUDE "path"|: include un sorgente nel presente file, prima dell'assemblamento;
		\item \lstinline|.SET nome, espressione|: serve a creare \textbf{costanti simboliche}. 
			Tali costanti hanno nome \lstinline|nome| e valore \lstinline|espressione|. Ad esempio:
\begin{lstlisting}	
.SET dimensione, 4
.SET n_iter, (100 * dimensione)
...
MOV $n_iter, %CX	# e' accesso immediato
\end{lstlisting}
	\end{itemize}
\end{itemize}

\subsection{Costanti numeriche}
Possiamo indicare costanti numeriche attraverso le seguenti convenzioni:
\begin{itemize}
	\item \textbf{Naturali:} non hanno segno, e vengono convertite nella loro rappresentazione in base 2;
	\item \textbf{Intere:} hanno un segno + o - davanti, e vengono convertite nella loro rappresentazione in complemento a 2.
\end{itemize}

Inoltre possiamo scrivere costanti in base 2, 8, 10 e 16 attraverso i prefissi \lstinline|0b|, \lstinline|0|, nessun prefisso e \lstinline|0x|.

Le variabili, quando non sono della dimensione giusta, vengono solitamente troncate (con avviso dall'assemblatore) o estese (senza avvisi dall'assemblatore).

\subsection{Controllo di flusso}
I costrutti di flusso a cui siamo abituati vengono implementati attraverso istruzioni di salto.
Conviene comunque ragionare in costrutti ad alto livello, e limitarsi a tradurli in assembler.
Da qui in puoi useremo una sintassi pseudo-C per indicare questi costrutti ad alto livello.

\subsubsection{If-then-else}
Prendiamo la sintassi:
\begin{lstlisting}[language=C++, style=codestyle]	
if(%AX < variabile) {
	//ramo if
	...
} else {
	//ramo else
	...
}
//prosegui
...
\end{lstlisting}
potremo tradurla in due modi:
\begin{itemize}
	\item Invertendo i rami then e else:
\begin{lstlisting}	
					CMP variabile, %AX
					JB ramothen
ramoelse:	... # ramo else
					JMP segue
ramothen: ... # ramo then
segue:		# prosegui
					...
\end{lstlisting}
	\item Invertendo la condizione:
\begin{lstlisting}[language=C++, style=codestyle]	
					CMP variabile, %AX
					JAE ramoelse
ramothen:	... # ramo then
					JMP segue
ramoelse: ... # ramo else
segue:		... # prosegui
\end{lstlisting}
\end{itemize}

\subsubsection{Ciclo for}
Prendiamo:
\begin{lstlisting}[language=C++, style=codestyle]	
for(int i = 0; i < variabile; i++) {
	//iter
	...
}
//prosegui
...
\end{lstlisting}
si rende attraverso il registro CX, come:
\begin{lstlisting}	
				MOV $0, %CX
ciclo:	CMP var, %CX
				JE segue
				...	# iter
				INC %CX
				JMP ciclo
segue:	... # prosegui
\end{lstlisting}

\subsubsection{Ciclo do-while}
Prendiamo infine:
\begin{lstlisting}[language=C++, style=codestyle]	
do {
	//iter
	...
} while(AX < var)
//prosegui
...
\end{lstlisting}
si rende come:
\begin{lstlisting}	
ciclo:	... # iter
				CMP var, %AX
				JB ciclo
				... # prosegui
\end{lstlisting}

\subsubsection{Un piatto di spaghetti}
In assembler ci è concesso fare ciò che non è permesso da linguaggi strutturati come il C o il Pascal.
In questi linguaggi, un costrutto ha un solo punto di ingresso e un solo punto di uscita.

In assembler, invece, possiamo saltare fuori e dentro cicli e costrutti quando e dove vogliamo, ed è il programmatore che deve pensare a cosa il programma sta effettivamente facendo. Ad esempio, nessuno ci vieta di dire:
\begin{lstlisting}	
ciclo: 	... # inizio ciclo
			 	...
label1:	... # meta' ciclo
				CMP var, %AX
				JB ciclo
				...
				JMP label1 # salto dentro un ciclo a meta' esecuzione?
\end{lstlisting}
\par\medskip
In assembler abbiamo a disposizione un'istruzione dedicata per i loop, che è:

\subsubsection{LOOP}
\begin{itemize}
	\item \textbf{Formato:} \lstinline|LOOP destination|
	\item \textbf{Azione:} decrementa ECX e salta alla destinazione se ECX $\neq0$. ECX va inizializzato al numero di iterazioni desiderate, e non va toccato durante il ciclo. 
	\item \textbf{Flag:} nessuno.
\end{itemize}

Si nota che la LOOP decrementa sempre ECX, quindi si applica difficilmente a cicli FOR dove vogliamo che la variabile di controllo incrementi, e ci serve che il suo valore nel corpo del ciclo. Si noti la differenza nei due esempi:

\begin{minipage}{0.45\textwidth}
\begin{lstlisting}[language=C++, style=codestyle]	
for(int i = var; i > 0; i--) {
	//iter (usa i)
}
\end{lstlisting}
diventa:
\begin{lstlisting}
				MOV var, %ECX
ciclo:	... # iter
				LOOP ciclo
\end{lstlisting}
\end{minipage}%
\hfill % This adds horizontal space between the two minipages
\begin{minipage}{0.45\textwidth}
\begin{lstlisting}[language=C++, style=codestyle]	
for(int i = 0; i < var; i++) {
	//iter (usa i)
}
\end{lstlisting}
diventa:
\begin{lstlisting}
				MOV $0, %EBX # usa EBX
ciclo:	... # iter
				INC EBX
				CMP var, %EBX
				JE ciclo
\end{lstlisting}
\end{minipage}

\subsubsection{LOOP condizionali}
Esistono versioni condizionali della LOOP, che sono \lstinline|LOOPE| e \lstinline|LOOPNE|, simili alle Jump condizionali. In questo caso, oltre al registro ECX, si verifica la condizione e nel caso si salta. Ad esempio:

\begin{lstlisting}	
				MOV $10, %ECX
ciclo: 	CMP src, dest
				LOOPcond ciclo
\end{lstlisting}

\par\smallskip
Queste istruzioni non sono indispensabili, in quanto possono essere rimpiazzate facilmente dalla \lstinline|CMP| unita ad un Jump condizionale.

\subsection{Passaggio di argomenti a sottoprogrammi}
Le \lstinline|CALL| e \lstinline|RET| prima definite non fornisicono modi per passare parametri ai sottoprogrammi, o restituire valori ai chiamanti.

Dobbiamo quindi stabilire delle convenzioni, scegliendo se:
\begin{itemize}
	\item Usare locazioni di memoria condivise;
	\item Usare registri;
	\item Usare la pila (che non verrà visto nel corso).
\end{itemize}

In assembler non esiste il concetto di visibilità o variabili locali, tutta la memoria è indirizzabile a qualsiasi livello.
Comunque, quando si scrive un sottoprogramma, bisogna specificare i parametri di ingresso e di uscita con un'opportuno commento, come:
\begin{lstlisting}	
# sottoprogramma "sottoprog", [descrizione]
# ingresso: %AX, [descrizione]
#					  %EBX, [descrizione]
# uscita:	  CAX, [descrizione]

sottoprog: 	...
						MOV ..., %CX # preparo il ritorno
						RET
\end{lstlisting} 

adesso potremo usare il sottoprogramma come:
\begin{lstlisting}	
MOV ..., %AX # preparo i parametri
MOV ..., %EBX
CALL sottoprog # chiamo
MOV %CX, var # var contiene il ritorno
\end{lstlisting}

\end{document}


\documentclass[a4paper,11pt]{article}
\usepackage[a4paper, margin=8em]{geometry}

% usa i pacchetti per la scrittura in italiano
\usepackage[french,italian]{babel}
\usepackage[T1]{fontenc}
\usepackage[utf8]{inputenc}
\frenchspacing 

% usa i pacchetti per la formattazione matematica
\usepackage{amsmath, amssymb, amsthm, amsfonts}

% usa altri pacchetti
\usepackage{gensymb}
\usepackage{hyperref}
\usepackage{standalone}

% imposta il titolo
\title{Appunti Reti Logiche}
\author{Luca Seggiani}
\date{2024}

% imposta lo stile
% usa helvetica
\usepackage[scaled]{helvet}
% usa palatino
\usepackage{palatino}
% usa un font monospazio guardabile
\usepackage{lmodern}

\renewcommand{\rmdefault}{ppl}
\renewcommand{\sfdefault}{phv}
\renewcommand{\ttdefault}{lmtt}

% disponi il titolo
\makeatletter
\renewcommand{\maketitle} {
	\begin{center} 
		\begin{minipage}[t]{.8\textwidth}
			\textsf{\huge\bfseries \@title} 
		\end{minipage}%
		\begin{minipage}[t]{.2\textwidth}
			\raggedleft \vspace{-1.65em}
			\textsf{\small \@author} \vfill
			\textsf{\small \@date}
		\end{minipage}
		\par
	\end{center}

	\thispagestyle{empty}
	\pagestyle{fancy}
}
\makeatother

% disponi teoremi
\usepackage{tcolorbox}
\newtcolorbox[auto counter, number within=section]{theorem}[2][]{%
	colback=blue!10, 
	colframe=blue!40!black, 
	sharp corners=northwest,
	fonttitle=\sffamily\bfseries, 
	title=Teorema~\thetcbcounter: #2, 
	#1
}

% disponi definizioni
\newtcolorbox[auto counter, number within=section]{definition}[2][]{%
	colback=red!10,
	colframe=red!40!black,
	sharp corners=northwest,
	fonttitle=\sffamily\bfseries,
	title=Definizione~\thetcbcounter: #2,
	#1
}

% disponi codice
\usepackage{listings}
\usepackage[table]{xcolor}

\definecolor{codegreen}{rgb}{0,0.6,0}
\definecolor{codegray}{rgb}{0.5,0.5,0.5}
\definecolor{codepurple}{rgb}{0.58,0,0.82}
\definecolor{backcolour}{rgb}{0.95,0.95,0.92}

\lstdefinestyle{codestyle}{
		backgroundcolor=\color{black!5}, 
		commentstyle=\color{codegreen},
		keywordstyle=\bfseries\color{magenta},
		numberstyle=\sffamily\tiny\color{black!60},
		stringstyle=\color{green!50!black},
		basicstyle=\ttfamily\footnotesize,
		breakatwhitespace=false,         
		breaklines=true,                 
		captionpos=b,                    
		keepspaces=true,                 
		numbers=left,                    
		numbersep=5pt,                  
		showspaces=false,                
		showstringspaces=false,
		showtabs=false,                  
		tabsize=2
}

\lstdefinestyle{shellstyle}{
		backgroundcolor=\color{black!5}, 
		basicstyle=\ttfamily\footnotesize\color{black}, 
		commentstyle=\color{black}, 
		keywordstyle=\color{black},
		numberstyle=\color{black!5},
		stringstyle=\color{black}, 
		showspaces=false,
		showstringspaces=false, 
		showtabs=false, 
		tabsize=2, 
		numbers=none, 
		breaklines=true
}


\lstdefinelanguage{assembler}{ 
  keywords={AAA, AAD, AAM, AAS, ADC, ADCB, ADCW, ADCL, ADD, ADDB, ADDW, ADDL, AND, ANDB, ANDW, ANDL,
        ARPL, BOUND, BSF, BSFL, BSFW, BSR, BSRL, BSRW, BSWAP, BT, BTC, BTCB, BTCW, BTCL, BTR, 
        BTRB, BTRW, BTRL, BTS, BTSB, BTSW, BTSL, CALL, CBW, CDQ, CLC, CLD, CLI, CLTS, CMC, CMP,
        CMPB, CMPW, CMPL, CMPS, CMPSB, CMPSD, CMPSW, CMPXCHG, CMPXCHGB, CMPXCHGW, CMPXCHGL,
        CMPXCHG8B, CPUID, CWDE, DAA, DAS, DEC, DECB, DECW, DECL, DIV, DIVB, DIVW, DIVL, ENTER,
        HLT, IDIV, IDIVB, IDIVW, IDIVL, IMUL, IMULB, IMULW, IMULL, IN, INB, INW, INL, INC, INCB,
        INCW, INCL, INS, INSB, INSD, INSW, INT, INT3, INTO, INVD, INVLPG, IRET, IRETD, JA, JAE,
        JB, JBE, JC, JCXZ, JE, JECXZ, JG, JGE, JL, JLE, JMP, JNA, JNAE, JNB, JNBE, JNC, JNE, JNG,
        JNGE, JNL, JNLE, JNO, JNP, JNS, JNZ, JO, JP, JPE, JPO, JS, JZ, LAHF, LAR, LCALL, LDS,
        LEA, LEAVE, LES, LFS, LGDT, LGS, LIDT, LMSW, LOCK, LODSB, LODSD, LODSW, LOOP, LOOPE,
        LOOPNE, LSL, LSS, LTR, MOV, MOVB, MOVW, MOVL, MOVSB, MOVSD, MOVSW, MOVSX, MOVSXB,
        MOVSXW, MOVSXL, MOVZX, MOVZXB, MOVZXW, MOVZXL, MUL, MULB, MULW, MULL, NEG, NEGB, NEGW,
        NEGL, NOP, NOT, NOTB, NOTW, NOTL, OR, ORB, ORW, ORL, OUT, OUTB, OUTW, OUTL, OUTSB, OUTSD,
        OUTSW, POP, POPL, POPW, POPB, POPA, POPAD, POPF, POPFD, PUSH, PUSHL, PUSHW, PUSHB, PUSHA, 
				PUSHAD, PUSHF, PUSHFD, RCL, RCLB, RCLW, MOVSL, MOVSB, MOVSW, STOSL, STOSB, STOSW, LODSB, LODSW,
				LODSL, INSB, INSW, INSL, OUTSB, OUTSL, OUTSW
        RCLL, RCR, RCRB, RCRW, RCRL, RDMSR, RDPMC, RDTSC, REP, REPE, REPNE, RET, ROL, ROLB, ROLW,
        ROLL, ROR, RORB, RORW, RORL, SAHF, SAL, SALB, SALW, SALL, SAR, SARB, SARW, SARL, SBB,
        SBBB, SBBW, SBBL, SCASB, SCASD, SCASW, SETA, SETAE, SETB, SETBE, SETC, SETE, SETG, SETGE,
        SETL, SETLE, SETNA, SETNAE, SETNB, SETNBE, SETNC, SETNE, SETNG, SETNGE, SETNL, SETNLE,
        SETNO, SETNP, SETNS, SETNZ, SETO, SETP, SETPE, SETPO, SETS, SETZ, SGDT, SHL, SHLB, SHLW,
        SHLL, SHLD, SHR, SHRB, SHRW, SHRL, SHRD, SIDT, SLDT, SMSW, STC, STD, STI, STOSB, STOSD,
        STOSW, STR, SUB, SUBB, SUBW, SUBL, TEST, TESTB, TESTW, TESTL, VERR, VERW, WAIT, WBINVD,
        XADD, XADDB, XADDW, XADDL, XCHG, XCHGB, XCHGW, XCHGL, XLAT, XLATB, XOR, XORB, XORW, XORL},
  keywordstyle=\color{blue}\bfseries,
  ndkeywordstyle=\color{darkgray}\bfseries,
  identifierstyle=\color{black},
  sensitive=false,
  comment=[l]{\#},
  morecomment=[s]{/*}{*/},
  commentstyle=\color{purple}\ttfamily,
  stringstyle=\color{red}\ttfamily,
  morestring=[b]',
  morestring=[b]"
}

\lstset{language=assembler, style=codestyle}

% disponi sezioni
\usepackage{titlesec}

\titleformat{\section}
	{\sffamily\Large\bfseries} 
	{\thesection}{1em}{} 
\titleformat{\subsection}
	{\sffamily\large\bfseries}   
	{\thesubsection}{1em}{} 
\titleformat{\subsubsection}
	{\sffamily\normalsize\bfseries} 
	{\thesubsubsection}{1em}{}

% tikz
\usepackage{tikz}

% float
\usepackage{float}

% grafici
\usepackage{pgfplots}
\pgfplotsset{width=10cm,compat=1.9}

% disponi alberi
\usepackage{forest}

\forestset{
	rectstyle/.style={
		for tree={rectangle,draw,font=\large\sffamily}
	},
	roundstyle/.style={
		for tree={circle,draw,font=\large}
	}
}

% disponi algoritmi
\usepackage{algorithm}
\usepackage{algorithmic}
\makeatletter
\renewcommand{\ALG@name}{Algoritmo}
\makeatother

% disponi numeri di pagina
\usepackage{fancyhdr}
\fancyhf{} 
\fancyfoot[L]{\sffamily{\thepage}}

\makeatletter
\fancyhead[L]{\raisebox{1ex}[0pt][0pt]{\sffamily{\@title \ \@date}}} 
\fancyhead[R]{\raisebox{1ex}[0pt][0pt]{\sffamily{\@author}}}
\makeatother

\begin{document}
% sezione (data)
\section{Lezione del 02-10-24}

% stili pagina
\thispagestyle{empty}
\pagestyle{fancy}

% testo
\subsection{Effetti collaterali}
I sottoprogrammi non dovrebbero avere effetti collaterali, ergo dovrebbero lasciare i registri come li trovano.
Per fare ciò, si sfrutta la pila per immagazzinare i loro valori precedenti:
\begin{lstlisting}	
sottoprog:	PUSH ... # fai push dei registri
						PUSH ...
						...	# esegui il sottoprogramma
						MOV ..., %CX

						POP ... # riprendi i resisti
						POP ...
						RET
\end{lstlisting}

Sono fondamentali due linee guida:
\begin{itemize}
	\item Bisogna stare attenti ad operazioni come \lstinline|IDIV| e \lstinline|IMUL|, che sporcano registri come EDX implictamente;
	\item Bisogna far corrispondere una \lstinline|POP| ad ogni \lstinline|PUSH|, altrimenti si lascia la pila in uno stato inconsistente per il prossimo \lstinline|RET|.
\end{itemize}

\subsection{Sottoprogramma principale}
Il \lstinline|_main| va in esecuzione come un sottoprogramma, ergo deve terminare con una \lstinline|RET| e lasciare in EAX un valore di ritorno (0 significa tutto ok, $\neq0$ significa codice di errore).
Per quanto ci riguarda, basterà scrivere \lstinline|XOR %EAX, %EAX|.

\subsection{Dichiarazione dello stack}
Lo stack esiste se viene:
\begin{enumerate}
	\item Dichiarato con una direttiva;
	\item Inizializzato con il registro ESP.
\end{enumerate}

Dichiarare significa allocare abbastanza memoria, e inizializzare significa impostare ESP alla cella successiva al fondo dello stack (si ricorda che lo stack si evolve verso sinistra). Ad esempio, potremo avere:

\begin{lstlisting}	
.DATA
...
mystack:	.FILL 1024, 4 #dichiarazione stack
.SET			initial_esp, (mystack + 1024*4)

.TEXT
_main:		NOP
					MOV $initial_esp, %ESP	# inizializzazione stack
\end{lstlisting}

Lo stack può essere grande a piacere del programmatore.
Nel nostro ambiente (ma non in generale) possiamo omettere la dichiarazione.

La pila può essere anche usata per il passaggio dei documanti (è il metodo che usano i compilatori). 
Questo risulta difficile da fare a mano, e quindi è sconsigliato per programmi più semplici.

\subsection{Sottoprogrammi di Input/Output}
In assembler non esistono istruzioni di ingresso e uscita (tranne le \lstinline|IN| e \lstinline|OUT|, che però sappiamo essere privilegiate).
Si usano quindi i servizi del sistema (DOS), ovvero sottoprogrammi scritti da altri che girano in modalità sistema.
Questi servizi sono molto primitivi: permettono l'uscita di singoli caratteri.
Esistono quindi sottoprogrammi (leggermente) più sofisticati per l'output di numeri, ecc...

\subsubsection{I/O tastiera e video}
Le informazioni che entrano ed escono da interfacce sono solo codifice ASCII di singoli caratteri.
Infatti in assembler non esiste il concetto di I/O tipato di variabili.

Ricevere il numero 32 significa ottenere i caratteri '3' e '2', mentre stamparlo significa inviare i caratteri '3' e '2'.
Questo chiaramente sui decimale si traduce in moltiplicazioni per 10 (in entrata) e divisioni per 10 con resto (in uscita) atte ad ottenere queste cifre.

\subsubsection{I/O di caratteri e stringhe}
Nel corso si userà il file di utilità \lstinline|.INCLUDE "./files/utility.s"|.
Questo file mette a disposizione alcuni sottoprogrammi fra cui:
\begin{itemize}
	\item \textbf{inchar:} mette in AL la codifica ASCII del tasto premuto;
	\item \textbf{outchar:} mette sul video la codifica ascii contenuta in AL;
	\item \textbf{newline:} stampa \lstinline|0x0D| (Carriage Return) e \lstinline|0x0A| (Line Feed), ergo va a capo;
	\item \textbf{pauseN:} mette in pausa il programma e stampa a video:
\begin{lstlisting}	
Checkpoint number N. Press any key to continue
\end{lstlisting}
dove N deve essere una cifra decimale.
\end{itemize}

Sopra questi sottoprogrammi sono state scritte routine più complesse:
\begin{itemize}
	
	\item \textbf{inline:} 
	\begin{itemize}
		\item \textbf{Descrizione:} porta una stringa di massimo 80 caratteri in un buffer di memoria, digitando con eco su video.
		\item \textbf{Parametri di ingresso:} 
			\begin{itemize}
				\item EBX: indirizzo di memoria del buffer;
				\item CX: numero di caratteri da leggere (massimo 80, una linea).
			\end{itemize}
	\end{itemize}
		Questo programma legge effettivamente 78 caratteri utili, in quanto gli ultimi 2 sono obbligatoriamente il nuova linea.
		Il programma inoltre gestisce la pressione dei tasti invio (finisci di ottenere caratteri) e backspace (cancella caratteri).
	
	\item \textbf{outline, outmess:}
		\begin{itemize}
		\item \textbf{Descrizione:} stampa a video massimo 80 caratteri da un buffer di memoria. Si ferma prima se trova un carattere di ritorno carrello, andando anche a capo. 
		\item \textbf{Parametri di ingresso:} 
			\begin{itemize}
				\item EBX: indirizzo di memoria del buffer;
			\end{itemize}
		\end{itemize}
		
	\item \textbf{inbyte, inword, inlong:}
		\begin{itemize}
		\item \textbf{Descrizione:} prelevano da tastiera (con eco sul video) 2, 4 o 8 caratteri.
			Interpretano tale sequenza di caratteri come un numero esadecimale a 2, 4 o 8 cifre.
			Ignorano tutti gli altri caratteri.
		\item \textbf{Parametri di ingresso:} 
			\begin{itemize}
				\item AL, AX, o EAX: il numero esadecimale digitato.
			\end{itemize}
		\end{itemize}

	\item \textbf{outbyte, outword, outlong:}
		\begin{itemize}
		\item \textbf{Descrizione:} stampano a video 2, 4 o 8 caratteri, corrispondenti a cifre esadecimali.
		\item \textbf{Parametri di ingresso:} 
			\begin{itemize}
				\item AL, AX, o EAX: il numero esadecimale da stampare.
			\end{itemize}
		\end{itemize}


	\item \textbf{indecimal\_byte, indecimal\_word, indecimal\_long:}
		\begin{itemize}
		\item \textbf{Descrizione:} prelevano da tastiera (con eco sul video) fino a 3, 5 o 10 cifre decimali.
			Interpretano tale sequenza di caratteri come un numero decimale.
		\item \textbf{Parametri di ingresso:} 
			\begin{itemize}
				\item AL, AX, o EAX: il numero decimale digitato.
			\end{itemize}
		\end{itemize}
	Se il numero decimale è troppo grande viene troncato.
	Inoltre si può usare invio per dare ingresso a meno cifre.

	
	\item \textbf{outdecimal\_byte, outdecimal\_word, outdecimal\_long:}
		\begin{itemize}
		\item \textbf{Descrizione:} stampano a video caratteri corrispondenti a cifre decimali.
		\item \textbf{Parametri di ingresso:} 
			\begin{itemize}
				\item AL, AX, o EAX: il numero decimale da stampare.
			\end{itemize}
		\end{itemize}

\end{itemize}

\subsection{Manipolazione di stringhe e vettori}
In assembler non esistono tipi di dati né strutture dati.
Si supporta però il concetto di vettore: si dichiarano vettori di variabili di una certa dimensione, e si indirizzano i loro elementi attraverso l'indirizzamento complesso ($\text{displacement} + \text{base} + \text{indice} \times \text{scala}$).

In verità esistono istruzioni stringa, che servono a copiare interi buffer di memoria, che sfruttano i registri ESI e EDI.
Ad esempio, copiare un vettore a mano significherebbe:
\begin{lstlisting}	
vett_sorg:	.FILL 1000,4
vett_dest:	.FILL 1000,4

						MOV $1000, %ECX
						LEA vett_sorg, %ESI
						LEA vett_dest, %EDI
ciclo:			MOV (%ESI), %EAX
						MOV %EAX , (%EDI)
						ADD $4, %ESI
						ADD $4, %EDI
						LOOP ciclo
\end{lstlisting}
ma abbiamo la possibilità di scrivere la stessa cosa come:
\begin{lstlisting}	
vett_sorg:	.FILL 1000,4
vett_dest:	.FILL 1000,4

						MOV $1000, %ECX
						LEA vett_sorg, %ESI
						LEA vett_dest, %EDI
						REP MOVSL
\end{lstlisting}
dove l'istruzione \lstinline|REP MOVSL| indica ripetizione (prefisso \lstinline|REP|), di movimento da stringa a stringa su long (\lstinline|MOVSL|) finché ECX $\neq 0$.

\subsubsection{Direction Flag}
Esiste un'altro bit utile nel registro dei flag: il Direction Flag, o DF.
Si imposta con le istruzioni:
\begin{itemize}
	\item \textbf{\textsf{STD}}: \textsf{SET DIRECTION FLAG}, la imposta ad 1;
	\item \textbf{\textsf{CLD}}: \textsf{CLEAR DIRECTION FLAG}, la imposta a  0;
\end{itemize}

Si usa questo flag per dare indicazioni alla prossima istruzione:

\subsubsection{MOVE DATA FROM STRING TO STRING (with REPEAT)}
\begin{itemize}
	\item \textbf{Formato:} \lstinline|MOVSsuf|, \lstinline|REP MOVSsuf| 
	\item \textbf{Azione:} copia il numero di byte indicato dal suffisso \textit{suf} dall'indirizzo di memoria puntato da ESI all'indirizzo di memoria puntato da EDI.
		Successivamente, SE DF è 1, sottrae da ESI e EDI il numero di byte indicati da \textit{suf}, altrimenti li somma.

		Se si include il prefisso, le operazioni vengono ripetute decrementando ECX (come per \lstinline|LOOP|).
	\item \textbf{Flag:} nessuno.
\end{itemize}
\par\medskip
Esistono poi altre istruzioni di stringa, fra cui:

\subsubsection{LOAD DATA FROM STRING}
\begin{itemize}
	\item \textbf{Formato:} \lstinline|LODSsuf| 
	\item \textbf{Azione:} copia in AL, AX, oppure EAX, il contenuto della memoria all'indirizzo puntato da ESI. Successivamente incrementa o decrementa ESI di 1, 2 o 4 a seconda di DF.
	\item \textbf{Flag:} nessuno.
\end{itemize}

\subsubsection{STORE DATA TO STRING}
\begin{itemize}
	\item \textbf{Formato:} \lstinline|LODSsuf| 
	\item \textbf{Azione:} copia il registro AL, AX, oppure EAX, in memoria all'indirizzo puntato da EDI. Successivamente incrementa o decrementa EDI di 1, 2 o 4 a seconda di DF.
	\item \textbf{Flag:} nessuno.
\end{itemize}

Si dovrebbe essere notato che ESI sta per sorgente, ed EDI per destinatario.
Vediamo quindi degli esempi:
\par\smallskip
\begin{minipage}[t]{0.45\textwidth}
Copia un vettore da una parte all'altra, eseguendo un'operazione su tutti i suoi elementi:
\begin{lstlisting}	
				MOV $1000, %CX
				LEA buffer_src, %ESI
				LEA buffer_dst, %EDI
				CLD
ciclo:	LODSL
				...	#modifica %EAX
				STOSL
				LOOP ciclo
\end{lstlisting}

\end{minipage}%
\hfill % This adds horizontal space between the two minipages
\begin{minipage}[t]{0.45\textwidth}
Riempi un buffer in memoria di zeri:
\begin{lstlisting}	
MOV $1000, %ECX
LEA buffer, %EDI
XOR %EAX, %EAX
CLD
REP STOSL
\end{lstlisting}
\end{minipage}

\subsubsection{Istruzioni stringa per l'I/O}
Esistono delle istruzioni stringa di ingresso e uscita: 


\subsubsection{INSERT STRING}
\begin{itemize}
	\item \textbf{Formato:} \lstinline|INSsuf| 
	\item \textbf{Azione:} fa ingresso di 1, 2 o 4 byte dalla porta di I/O il cui offset è contenuto in DX. L'operando viene inserito in memoria a partire dall'indirizzo contenuto in EDI.
		Successivamente incrementa o decrementa EDI di 1, 2, o 4 a seconda di DF.
	\item \textbf{Flag:} nessuno.
\end{itemize}

\subsubsection{OUTPUT STRING}
\begin{itemize}
	\item \textbf{Formato:} \lstinline|INSsuf| 
	\item \textbf{Azione:} fa uscita di 1, 2 o 4 byte dall'indirizzo di memoria contenuto in EDI. L'operando viene inserito nella porta di I/O il cui offset è contenuto in DX.
		Successivamente incrementa o decrementa ESI di 1, 2, o 4 a seconda di DF. 

	\item \textbf{Flag:} nessuno.
\end{itemize}

\subsubsection{Istruzioni di confronto su stringhe}
Vediamo infine alcune istruzioni per effettuare confronti su e fra stringhe:

\subsubsection{COMPARE STRINGS}
\begin{itemize}
	\item \textbf{Formato:} \lstinline|CMPSsuf| 
	\item \textbf{Azione:} confronta il valore delle locazioni (singole, doppie o quadruple) indicate da ESI (sorgente) ed EDI (destinatario). 
		Successivamente incrementa o decrementa ESI di 1, 2, o 4 a seconda di DF. 

	\item \textbf{Flag:} nessuno.
\end{itemize}

\subsubsection{SCAN STRING}
\begin{itemize}
	\item \textbf{Formato:} \lstinline|SCASsuf| 
	\item \textbf{Azione:} confronta il contenuto del registro AL, AX o EAX con la locazione (singola, doppia o quadrupla) di memoria indirizzata da EDI. L'algoritmo di confronto è lo stesso di CMP. 
		Successivamente incrementa o decrementa ESI di 1, 2, o 4 a seconda di DF. 

	\item \textbf{Flag:} nessuno.
\end{itemize}

Quest'espressione si usa per trovare valori noti dentro un vettore con, DF $=0$ che cerca la prima occorrenza, e DF $=1$ che cerca l'ultima. Ad esempio, poniamo di voler trovare il primo elemento differente fra due vettori:
\begin{lstlisting}	
arrayl: .WORD 1, 2, 3, 4, 5, 6, 7, 8, 9, 10
array2: .WORD 1, 2, 3, 4, 7, 6, 7, 8, 9, 10

CLD
LEA array 1, %ESI
LEA array2, %EDI
MOV $10, %ECX
REPE CMPSW
\end{lstlisting}

dove si noti che alla fine del ciclo EDI e ESI puntano all'elemento successivo. 

\subsubsection{Prefissi di ripetizione}
Vediamo nel dettaglio il prefisso \lstinline|REP|, e le sue varianti \lstinline|REPE| e \lstinline|REPNE|.
Bisogna ricordare che questi prefissi si applicano ad istruzioni, non a blocchi di codice.
\begin{itemize}
	\item \lstinline|REP|: si può usare con \lstinline|MOVS|, \lstinline|LODS|, \lstinline|STOS|, \lstinline|INS| e \lstinline|OUTS|, anche se l'utilizzo con \lstinline|LODS| è privo di senso (almeno che non si voglia ottenere l'ultimo elemento...).
	\item \lstinline|REPE| e \lstinline|REPNE|: si può usare con \lstinline|CMPS| e \lstinline|SCAS|, ed effettua al massimo ECX ripetizioni, finché la condizione specificata è vera.
\end{itemize}

\subsubsection{Perchè due direzioni?}
L'uso di due direzioni di scorrimento di stringhe attraverso il flag DF è utile, sopratutto nel caso si debbano fare traslazioni del vettore (copia di buffer \textbf{parzialmente sovrapposti}).
Infatti, cercando si spostare il vettore a destra spostandoci verso destra, finiremo per copiare sempre gli stessi dati.

\subsection{Note sull'efficienza}
Un compilatore ottimizza il codice in alto livello per il sistema su cui quel codice dovrà girare.
Un assemblatore, invece, traduce le istruzioni una per una.

\subsubsection{Tempo di esecuzione di un processo}
Un processo è un programma in esecuzione con dei dati.
In questo, dipende dai dati, dallo stato del sistema, e da cosa sta facendo il processore (chi lo sta usando?).
Questo rende il calcolatore una macchina poco prevedibile, e il tempo di esecuzione del processo difficile da calcolare a priori. Di base, infatti:
\begin{itemize}
	\item Il clock non va a velocità costante;
	\item Il vostro processo non necessariamente gira su un solo core;
	\item Altri meccanismi introducono variabilità considerevoli:
		\begin{itemize}
			\item Memorie cache;
			\item Code di prefetch;
			\item Esecuzione in pipeline: eseguire un'istruzione significa fare fetch dell'istruzione, recuperare l'OPCODE, il sorgente, scrivere sul destinatario, ecc... conviene eseguire queste operazioni in pipeline, cioè eseguendo in parallelo più istruzioni possibili contemporaneamente;
			\item Esecuzione non sequenziale: il processore non esegue necessariamente il codice nell'ordine in cui è scritto: se possibile, modifica l'ordine in modo dal caricare in modo più efficiente possibile la pipeline;
			\item Branch prediction: quando si esegue in pipeline, le istruzioni condizionali creano forti bottleneck di prestazioni. Per ovviare a questo problema, il processore cerca di predire il tipo della prossima istruzione, pagando un prezzo nel caso si sbagli, ma ottenendo un significativo incremento di velocità nel caso abbia successo.
		\end{itemize}
\end{itemize}

\subsubsection{Lunghezza delle istruzioni e tempo di fetch}
Il numero di byte occupati da un'istruzione dipende dall'OPCODE e dal tipo di indirizzamento.
Se gli operandi sono \textbf{registri}, le istruzioni stanno normalmente su 1 byte; gli operandi \textbf{immediati} devono essere codificati (in 1, 2 o 4 byte); i \textbf{displacement} occupano 4 byte.

La lunghezza delle istruzioni, oltre alle dimensioni dei file binari, influenza anche il tempo di fetch delle stesse, e va quindi tenuto in considerazione.

\subsubsection{Tempo di esecuzione delle istruzioni}
Il tempo di esecuzione delle istruzioni dipende molto dall'architettura specifica del processore (anche in processori della stessa famiglia).

Abbiamo che le istruzioni ALU (escluse MUL e DIV) costano poco, su O(1) cicli di clock. Le MUL e DIV costano sui O(10) cicli di clock, e per questo vengono tradotte in procedure alternative (attraverso LEA o le istruzioni di shift) dai compilatori attraverso apposite tablle di corrispondenza.

Le operazioni più costose sono quelle dellA FPU (Floating Point Unit), che richiedono sulle O(100) istruzioni.

Anche le istruzioni condizionali sono molto costose, ma per i motivi visti prima che rallentano le pipeline.


\end{document}



\documentclass[a4paper,11pt]{article}
\usepackage[a4paper, margin=8em]{geometry}

% usa i pacchetti per la scrittura in italiano
\usepackage[french,italian]{babel}
\usepackage[T1]{fontenc}
\usepackage[utf8]{inputenc}
\frenchspacing 

% usa i pacchetti per la formattazione matematica
\usepackage{amsmath, amssymb, amsthm, amsfonts}

% usa altri pacchetti
\usepackage{gensymb}
\usepackage{hyperref}
\usepackage{standalone}

% imposta il titolo
\title{Appunti Reti Logiche}
\author{Luca Seggiani}
\date{2024}

% imposta lo stile
% usa helvetica
\usepackage[scaled]{helvet}
% usa palatino
\usepackage{palatino}
% usa un font monospazio guardabile
\usepackage{lmodern}

\renewcommand{\rmdefault}{ppl}
\renewcommand{\sfdefault}{phv}
\renewcommand{\ttdefault}{lmtt}

% disponi il titolo
\makeatletter
\renewcommand{\maketitle} {
	\begin{center} 
		\begin{minipage}[t]{.8\textwidth}
			\textsf{\huge\bfseries \@title} 
		\end{minipage}%
		\begin{minipage}[t]{.2\textwidth}
			\raggedleft \vspace{-1.65em}
			\textsf{\small \@author} \vfill
			\textsf{\small \@date}
		\end{minipage}
		\par
	\end{center}

	\thispagestyle{empty}
	\pagestyle{fancy}
}
\makeatother

% disponi teoremi
\usepackage{tcolorbox}
\newtcolorbox[auto counter, number within=section]{theorem}[2][]{%
	colback=blue!10, 
	colframe=blue!40!black, 
	sharp corners=northwest,
	fonttitle=\sffamily\bfseries, 
	title=Teorema~\thetcbcounter: #2, 
	#1
}

% disponi definizioni
\newtcolorbox[auto counter, number within=section]{definition}[2][]{%
	colback=red!10,
	colframe=red!40!black,
	sharp corners=northwest,
	fonttitle=\sffamily\bfseries,
	title=Definizione~\thetcbcounter: #2,
	#1
}

% disponi codice
\usepackage{listings}
\usepackage[table]{xcolor}

\definecolor{codegreen}{rgb}{0,0.6,0}
\definecolor{codegray}{rgb}{0.5,0.5,0.5}
\definecolor{codepurple}{rgb}{0.58,0,0.82}
\definecolor{backcolour}{rgb}{0.95,0.95,0.92}

\lstdefinestyle{codestyle}{
		backgroundcolor=\color{black!5}, 
		commentstyle=\color{codegreen},
		keywordstyle=\bfseries\color{magenta},
		numberstyle=\sffamily\tiny\color{black!60},
		stringstyle=\color{green!50!black},
		basicstyle=\ttfamily\footnotesize,
		breakatwhitespace=false,         
		breaklines=true,                 
		captionpos=b,                    
		keepspaces=true,                 
		numbers=left,                    
		numbersep=5pt,                  
		showspaces=false,                
		showstringspaces=false,
		showtabs=false,                  
		tabsize=2
}

\lstdefinestyle{shellstyle}{
		backgroundcolor=\color{black!5}, 
		basicstyle=\ttfamily\footnotesize\color{black}, 
		commentstyle=\color{black}, 
		keywordstyle=\color{black},
		numberstyle=\color{black!5},
		stringstyle=\color{black}, 
		showspaces=false,
		showstringspaces=false, 
		showtabs=false, 
		tabsize=2, 
		numbers=none, 
		breaklines=true
}


\lstdefinelanguage{assembler}{ 
  keywords={AAA, AAD, AAM, AAS, ADC, ADCB, ADCW, ADCL, ADD, ADDB, ADDW, ADDL, AND, ANDB, ANDW, ANDL,
        ARPL, BOUND, BSF, BSFL, BSFW, BSR, BSRL, BSRW, BSWAP, BT, BTC, BTCB, BTCW, BTCL, BTR, 
        BTRB, BTRW, BTRL, BTS, BTSB, BTSW, BTSL, CALL, CBW, CDQ, CLC, CLD, CLI, CLTS, CMC, CMP,
        CMPB, CMPW, CMPL, CMPS, CMPSB, CMPSD, CMPSW, CMPXCHG, CMPXCHGB, CMPXCHGW, CMPXCHGL,
        CMPXCHG8B, CPUID, CWDE, DAA, DAS, DEC, DECB, DECW, DECL, DIV, DIVB, DIVW, DIVL, ENTER,
        HLT, IDIV, IDIVB, IDIVW, IDIVL, IMUL, IMULB, IMULW, IMULL, IN, INB, INW, INL, INC, INCB,
        INCW, INCL, INS, INSB, INSD, INSW, INT, INT3, INTO, INVD, INVLPG, IRET, IRETD, JA, JAE,
        JB, JBE, JC, JCXZ, JE, JECXZ, JG, JGE, JL, JLE, JMP, JNA, JNAE, JNB, JNBE, JNC, JNE, JNG,
        JNGE, JNL, JNLE, JNO, JNP, JNS, JNZ, JO, JP, JPE, JPO, JS, JZ, LAHF, LAR, LCALL, LDS,
        LEA, LEAVE, LES, LFS, LGDT, LGS, LIDT, LMSW, LOCK, LODSB, LODSD, LODSW, LOOP, LOOPE,
        LOOPNE, LSL, LSS, LTR, MOV, MOVB, MOVW, MOVL, MOVSB, MOVSD, MOVSW, MOVSX, MOVSXB,
        MOVSXW, MOVSXL, MOVZX, MOVZXB, MOVZXW, MOVZXL, MUL, MULB, MULW, MULL, NEG, NEGB, NEGW,
        NEGL, NOP, NOT, NOTB, NOTW, NOTL, OR, ORB, ORW, ORL, OUT, OUTB, OUTW, OUTL, OUTSB, OUTSD,
        OUTSW, POP, POPL, POPW, POPB, POPA, POPAD, POPF, POPFD, PUSH, PUSHL, PUSHW, PUSHB, PUSHA, 
				PUSHAD, PUSHF, PUSHFD, RCL, RCLB, RCLW, MOVSL, MOVSB, MOVSW, STOSL, STOSB, STOSW, LODSB, LODSW,
				LODSL, INSB, INSW, INSL, OUTSB, OUTSL, OUTSW
        RCLL, RCR, RCRB, RCRW, RCRL, RDMSR, RDPMC, RDTSC, REP, REPE, REPNE, RET, ROL, ROLB, ROLW,
        ROLL, ROR, RORB, RORW, RORL, SAHF, SAL, SALB, SALW, SALL, SAR, SARB, SARW, SARL, SBB,
        SBBB, SBBW, SBBL, SCASB, SCASD, SCASW, SETA, SETAE, SETB, SETBE, SETC, SETE, SETG, SETGE,
        SETL, SETLE, SETNA, SETNAE, SETNB, SETNBE, SETNC, SETNE, SETNG, SETNGE, SETNL, SETNLE,
        SETNO, SETNP, SETNS, SETNZ, SETO, SETP, SETPE, SETPO, SETS, SETZ, SGDT, SHL, SHLB, SHLW,
        SHLL, SHLD, SHR, SHRB, SHRW, SHRL, SHRD, SIDT, SLDT, SMSW, STC, STD, STI, STOSB, STOSD,
        STOSW, STR, SUB, SUBB, SUBW, SUBL, TEST, TESTB, TESTW, TESTL, VERR, VERW, WAIT, WBINVD,
        XADD, XADDB, XADDW, XADDL, XCHG, XCHGB, XCHGW, XCHGL, XLAT, XLATB, XOR, XORB, XORW, XORL},
  keywordstyle=\color{blue}\bfseries,
  ndkeywordstyle=\color{darkgray}\bfseries,
  identifierstyle=\color{black},
  sensitive=false,
  comment=[l]{\#},
  morecomment=[s]{/*}{*/},
  commentstyle=\color{purple}\ttfamily,
  stringstyle=\color{red}\ttfamily,
  morestring=[b]',
  morestring=[b]"
}

\lstset{language=assembler, style=codestyle}

% disponi sezioni
\usepackage{titlesec}

\titleformat{\section}
	{\sffamily\Large\bfseries} 
	{\thesection}{1em}{} 
\titleformat{\subsection}
	{\sffamily\large\bfseries}   
	{\thesubsection}{1em}{} 
\titleformat{\subsubsection}
	{\sffamily\normalsize\bfseries} 
	{\thesubsubsection}{1em}{}

% tikz
\usepackage{tikz}

% float
\usepackage{float}

% grafici
\usepackage{pgfplots}
\pgfplotsset{width=10cm,compat=1.9}

% disponi alberi
\usepackage{forest}

\forestset{
	rectstyle/.style={
		for tree={rectangle,draw,font=\large\sffamily}
	},
	roundstyle/.style={
		for tree={circle,draw,font=\large}
	}
}

% disponi algoritmi
\usepackage{algorithm}
\usepackage{algorithmic}
\makeatletter
\renewcommand{\ALG@name}{Algoritmo}
\makeatother

% disponi numeri di pagina
\usepackage{fancyhdr}
\fancyhf{} 
\fancyfoot[L]{\sffamily{\thepage}}

\makeatletter
\fancyhead[L]{\raisebox{1ex}[0pt][0pt]{\sffamily{\@title \ \@date}}} 
\fancyhead[R]{\raisebox{1ex}[0pt][0pt]{\sffamily{\@author}}}
\makeatother

\begin{document}
% sezione (data)
\section{Lezione del 03-10-24}

% stili pagina
\thispagestyle{empty}
\pagestyle{fancy}

% testo
\subsection{Assembler a 64 bit}
Finora abbiamo studiato il linguaggio assembler a 32 bit (registri estesi EAX, EBX, ecc...).
Vediamo adesso alcune caratteristiche dell'assembler a 64 bit.

Nei processori a 64 bit Intel-AMD x86 abbiamo 16 registri generali a 64 bit, con prefisso R, e che quindi si indicano come RAX, RBX, ecc...
Di questi si può indirizzare la parte estesa dei 32 bit meno significativi (EAX), i 16 bit meno significativi (AX), e gli 8 bit meno significativi (AL).
Per RAX, RBX, RCX e RDX si possono inoltre indirizzare gli 8 bit precedenti ad AL, BL, CL e DL usando AH, BH, CH e DH, ma questo è sconsigliato in quanto ci sono diverse limitazioni (non sono compatibili col prefisso REX).

Una lista completa dei registri genrali è la seguente, inclusi i nomi dei sottoregistri di dimensione minore:

\begin{table}[h!]
	\center \rowcolors{2}{white}{black!10}
	\begin{tabular} { c || c | c | c | c }
		\bfseries 64 bit & \bfseries 32 bit & \bfseries 16 bit & \bfseries 8 bit & \bfseries 8 bit (legacy) \\
		\hline 
		RAX & EAX & AX & AL & AH \\
		RBX & EBX & BX & BL & BH \\
		RCX & ECX & CX & CL & CH \\
		RDX & EDX & DX & DL & DH \\
		RSP & ESP & SP & SPL & \\
		RBP & EBP & BP & BPL & \\
		RSI & ESI & SI & SIL & \\
		RDI & EDI & DI & DIL & \\
		R8 &R8D &R8W &R8B & \\
		R9 &R9D &R9W &R9B & \\
		R10&R10D&R10W&R10B & \\
		R11&R11D&R11W&R11B & \\
		R12&R12D&R12W&R12B & \\
		R13&R13D&R13W&R13B & \\
		R14&R14D&R14W&R14B & \\
		R15&R15D&R15W&R15B & \\
	\end{tabular}
\end{table}

Ricordiamo poi i registri RIP, l'instruction pointer, e RFLAGS che è il registro dei flag.

\subsubsection{Spazio indirizzabile}
Tecnicamente con architettura a 64 bit si potrebbero indirizzare $2^{64}$ byte distinti, ma i processori moderni permettono di indirizzarne solo $2^{48} = 256 \ \mathrm{TiB}$, con alcuni modelli più recenti che arrivano a $2^{57}= 128 \ \mathrm{PiB}$.
I 48 (o 57) bit occupati sono i meno significativi, e i restanti 16 (o 7) devono avere il valore del bit più significativo utilizzato.
Questo significa che sono indirizzabili effettivamente due porzioni contigue ma separate fra di loro di memoria:
\begin{table}[h!]
	\center \rowcolors{2}{white}{black!10}
	\begin{tabular} { c | p{4cm} | p{4cm} }
		& \bfseries 48 bit & \bfseries 57 bit \\
		\hline
		\bfseries Regione alta & \texttt{0000 0000 0000 0000} \texttt{0000 7fff ffff ffff} & \texttt{0000 0000 0000 0000} \texttt{01ff ffff ffff ffff} \\
		\bfseries Regione bassa & \texttt{ffff 8000 0000 0000} \texttt{ffff ffff ffff ffff} & \texttt{fe00 0000 0000 0000} \texttt{ffff ffff ffff ffff} \\
	\end{tabular}
\end{table}

Lo spazio I/O, infine, è di $2^{16} = 64 \ \mathrm{KiB}$ locazioni.

\subsubsection{Istruzioni}
Le operazioni possono possono usare 1, 2, 4 o 8 byte per un operando (rispettivamente Byte, Word, Long e Quad).

Notiamo che non possiamo usare displacement o operandi immediati a 64 bit: siamo limitati a 32 bit.
Per ovviare a questo problema esiste una versione alternativa della \lstinline|MOV|:

\subsubsection{MOVABS}
\begin{itemize}
	\item \textbf{Formato:} \lstinline|MOVABS $const, destination|
	\item \textbf{Azione:} porta una costante a 64 bit (che ci permette di scrivere) in un indirizzo generale.
	\item \textbf{Flag:} nessuno.
\end{itemize}

		\begin{table}[H]
		\center \rowcolors{2}{white}{black!10}
			\begin{tabular} { c | p{5cm} }
				\bfseries Operandi & \bfseries Esempi \\
				\hline
				Immediato & \lstinline|MOVABS $0xffff8105402300ef, %RBX| \\ 
				Memoria & \lstinline|CALL 0x00ef0b2a, %RAX| \\ 
				Registro & \lstinline|CALL %RAX, 0x00ef0b2a|
			\end{tabular}
		\end{table}

\par\smallskip 
In generale, in assember a 64 bit si usano registri con valori base di 64 bit, e poi si indirizza con displacement a 32 bit, che in complemento a 2 concedono $\pm 2^{32}$, ergo $\pm 2 \mathrm{GB}$ di memoria indirizzabile rispetto alla base.

\subsection{Reti logiche}
Una rete logica è un modello astratto di un sistema fisico, costituito da dispositivi tra loro interconnessi.
Le informazioni vengono codificate da questi dispositivi attraverso fenomeni fisici che si presentano in due aspetti distinti (corrente forte / corrente debole, tensione forte / tensione debole, magnetizzazione / non magnetizzazione, ecc...).

\subsubsection{Caratterizzazione di rete logica}
Una rete logica è caratterizzata da:
\begin{itemize}
	\item Un'insieme di $N$ variabili di ingresso. Il loro valore all'istante temporale $t$ si chiama stato di ingresso. L'insieme di tutti i $2^N$ stati di ingresso si indicherà come $X.X = \{ x_{N-1} x_{N-2} ... x_1 x_0 \}$. 
	\item Un'insieme di $M$ variabili di uscita. Il loro valore all'istante temporale $t$ si chiama stato di uscita. L'insieme di tutti i $2^M$ stati di uscita si indicherà come $Z.Z = \{ x_{M-1} x_{M-2} ... x_1 x_0 \}$. 
	\item Una legge di evoluzione che determina come le uscite si evolvono in funzione degli ingressi.
\end{itemize}

Possiamo classificare le reti logiche in base a 2 criteri riguardanti l'evoluzione nel tempo:
\begin{itemize}
	\item \textbf{Presenza/assenza di memoria}:
		\begin{itemize}
			\item \textbf{Reti combinatorie:} analoghe a funzioni matematiche, le loro uscite dipendono solo dai loro ingressi in un qualsiasi istanti $t$;
			\item \textbf{Reti sequenziali:} lo stato di uscita dipende dalla storia degli ingressi precedenti, ergo sono reti con memoria.
		\end{itemize}
	\item \textbf{Temporizzazione della legge di evoluzione:}
		\begin{itemize}
			\item \textbf{Reti asincrone:} l'aggiornamento delle uscite avviene costantemente nel tempo;
			\item \textbf{Reti sincronizzate:} l'aggiornamento delle uscite avviene ad istanti di sincronizzazione discreti nel tempo.
		\end{itemize}
\end{itemize}

I modelli sono ortogonali, ergo possiamo avere qualsiasi delle 4 combinazioni di queste caratteristiche:
\begin{itemize}
	\item Reti combinatorie (si considerano le sincronizzate come caso particolare);
	\item Reti sequenziali asincrone;
	\item Reti sequenziali sincronizzate.
\end{itemize}

Quindi in sostanza una rete logica comunica con l'esterno attraverso variabili logiche (0 e 1).
L'interpretazione di questi messaggi è una convenzione del progettista, programatore, ecc...

Usiamo le reti logiche per modellizzare circuiti elettronici all'interno del calcolatore, che codificano le informazioni in tensione.
Notiamo quindi che una rete logica fisica ha, oltre agli ingressi e alle uscite, i collegamenti ai terminali positivi e negativi di un generatore di tensione, che noi ignoreremo. 

\subsection{Transizione dei segnali}
Una variabile logica (per noi il voltaggio su un circuito) può settarsi (andare a 1), restare settato per tempi paragonabili a $\Delta T$, e resettarsi (andare a 0) in un qualsiasi momento temporale $t$:

\begin{center}
\begin{tikzpicture}
    \begin{axis}[
        xlabel={$t$},
        ylabel={$V$},
        xmin=0, xmax=14,
        ymin=-1, ymax=6,
        grid=major,
        domain=0:14,
				xtick={5,8},
				ytick={0,5},
				xticklabels={$t_1$,$t_2$},
				yticklabels={$0$, $V_{max}$},
        samples=100,
        legend pos=south west,
    		width=14cm,
				height=7cm
			]
    \addplot[blue, thick] {5 * (x >= 5) * (x <=  8)};
    \end{axis}
\end{tikzpicture}
\end{center}

In un sistema fisico reale, durante la transizione c'è un periodo di indecisione in cui il voltaggio sale o scende fisicamente fino al valore necessario, sotto l'atto di una qualche potenza. Vediamo il grafico a $\Delta t << \Delta T$:

\begin{center}
\begin{tikzpicture}
    \begin{axis}[
        xlabel={$t$},
        ylabel={$V$},
        xmin=4.8, xmax=5.2,
        ymin=-1, ymax=6,
        grid=major,
        domain=0:14,
				xtick={5,8},
				ytick={0,5},
				xticklabels={$t_1$,$t_2$},
				yticklabels={$0$, $V_{max}$},
        samples=100,
        legend pos=south west,
    		width=14cm,
				height=7cm
			]
    \addplot[blue, thick] {5 * (x >= 5) * (x <=  8)};
    \end{axis}
\end{tikzpicture}
\end{center}

Decidiamo di ignorare questo problema, in quanto abbiamo visto che il $\Delta t$ di transizione è molto più piccolo del $\Delta t$ di stasi delle variabili.

Il problema si presenta nel caso si parli di \textbf{contemporaneità}. 
Supponiamo di avere una rete logica con due ingressi $x_0$ e $x_1$ e un'uscita $z_0$.
Abbiamo che prima dell'istante $t_1$ lo stato di ingresso è $(1,0)$, e che subito dopo lo stesso stato è $(0, 1)$.
Nell'istante di transizione non abbiamo la sicurezza che le singole transizioni delle due variabili della rete avvengano contemporaneamente:

\begin{tikzpicture}
    \begin{axis}[
        xlabel={$t$},
        ylabel={$V$},
        xmin=0, xmax=8,
        ymin=-1, ymax=6,
        grid=major,
        domain=0:8,
        samples=100,
				xtick={3},
				ytick={0,5},
				xticklabels={$t_1$},
				yticklabels={$0$, $V_{max}$},
        legend pos=south west,
    		width=14cm,
				height=7cm
			]
    \addplot[blue, thick] {5 * (x >= 3)};
    \addplot[red, thick] {5 * (x >= 3.5) + 0.25};
    \end{axis}
\end{tikzpicture}

Questa considerazione è importante nel caso delle reti logiche asincrone, dove considerare le transizioni come contemporanee potrebbe portare alla comparsa di stati di uscita spuri, e nelle reti sequenziali, dove potrebbe portare ad evoluzioni imprevedibili del sistema.

\subsection{Il linguaggio Verilog}
Per descrivere le reti logiche fa comodo adottare una \textbf{notazione testuale}.
Per reti semplici useremo disegni o espressioni algebriche: per reti complesse introduciamo un \textbf{linguaggio di descrizione hardware}, il \textbf{Verilog}.
Questo linguaggio è più \textbf{compatto}, e può essere \textbf{interpretato} automaticamente da una macchina, permettendoci di effettuare prove (e realizzare \textbf{diagrammi di temporizzazione}).

% cosa voglio sapere del verilog? 
% le basi (cos'è un modulo, cos'è un testbench, ecc...)
% BENE gli assegnamenti continui, procedurali, bloccanti, non bloccanti, ecc...

Non si riporteranno appunti riguardanti operatori e sintassi particolarmente specifiche del Verilog, in quanto esistono testi sicuramente più utili e approfonditi: procederemo principalmente per esempi, esplicitando quando si rende necessario particolarità del linguaggio.

\subsubsection{Struttura di una sintesi Verilog}
Il linguaggio Verilog descrive \textbf{moduli}.
Un modulo è formato da un insieme di \textbf{input} e \textbf{output}, e da una \textbf{struttura interna} che descrive la legge di evoluzione degli output in funzione degli input.
Ad esempio, una rete basilare potrebbe essere:
\begin{lstlisting}[language=verilog, style=codestyle]	
module rete(x, z);
	input x;
	output z;
	wire y;
	assign y = x;
	assign z = y;
endmodule
\end{lstlisting}

Nell'esempio, si definisce un modulo \lstinline|rete|, formato da un input $x$ e un output $z$. 
La realizzazione interna della rete è formata da un filo $y$ a cui sono connessi sia l'input che l'output.
Il funzionamento della rete è quindi semplicemente quello di replicare il suo ingresso.

In particolare, diciamo che la parola chiave \lstinline|assign| rapprenta un \textbf{assegnamento continuo}.
Più avanti vedremo i diversi tipi di assegnamento e le differenze fra di loro.

\subsection{Reti combinatorie}
Il primo tipo di reti logiche che andiamo a studiare sono le \textbf{reti combinatorie}.
Una rete combinatoria è caratterizzata da:
\begin{itemize}
	\item Un'insieme di $N$ variabili logiche di ingresso;
	\item Un'insieme di $M$ variabili logiche di uscita;
	\item Una descrizione funzionale $F: X \rightarrow Z$ che mappa stati di ingresso a stati di uscita;
	\item Una legge di evoluzione nel tempo che adegua $F(X)$ allo stato di ingresso $X$ continuamente.
\end{itemize}

\subsubsection{Tempo di attraversamento}
Il tempo di attraversamento (o di accesso) è una caratteristica di tutte le reti logiche asincrone: è il tempo necessario perché la rete si "accorga" della variazione degli ingressi e aggiorni di conseguenza le sue uscite.

Questo tempo è solitamente non nullo, ed è quindi necessario attendere che la rete arrivi a \textbf{regime} prima di valutare le uscite.
Questo vincolo prende il nome di \textbf{pilotaggio in modo fondamentale}: si dice che è una rete è pilotata in modo fondamentale quando chi la pilota aspetta sempre che essa arrivi a regime prima di valutare le sue uscite.
\end{document}


\documentclass[a4paper,11pt]{article}
\usepackage[a4paper, margin=8em]{geometry}

% usa i pacchetti per la scrittura in italiano
\usepackage[french,italian]{babel}
\usepackage[T1]{fontenc}
\usepackage[utf8]{inputenc}
\frenchspacing 

% usa i pacchetti per la formattazione matematica
\usepackage{amsmath, amssymb, amsthm, amsfonts}

% usa altri pacchetti
\usepackage{gensymb}
\usepackage{hyperref}
\usepackage{standalone}

\usepackage{colortbl}

% circuiti
\usepackage{circuitikz}
\usetikzlibrary{babel}

% imposta il titolo
\title{Appunti Reti Logiche}
\author{Luca Seggiani}
\date{2024}

% imposta lo stile
% usa helvetica
\usepackage[scaled]{helvet}
% usa palatino
\usepackage{palatino}
% usa un font monospazio guardabile
\usepackage{lmodern}

\renewcommand{\rmdefault}{ppl}
\renewcommand{\sfdefault}{phv}
\renewcommand{\ttdefault}{lmtt}

% disponi il titolo
\makeatletter
\renewcommand{\maketitle} {
	\begin{center} 
		\begin{minipage}[t]{.8\textwidth}
			\textsf{\huge\bfseries \@title} 
		\end{minipage}%
		\begin{minipage}[t]{.2\textwidth}
			\raggedleft \vspace{-1.65em}
			\textsf{\small \@author} \vfill
			\textsf{\small \@date}
		\end{minipage}
		\par
	\end{center}

	\thispagestyle{empty}
	\pagestyle{fancy}
}
\makeatother

% disponi teoremi
\usepackage{tcolorbox}
\newtcolorbox[auto counter, number within=section]{theorem}[2][]{%
	colback=blue!10, 
	colframe=blue!40!black, 
	sharp corners=northwest,
	fonttitle=\sffamily\bfseries, 
	title=Teorema~\thetcbcounter: #2, 
	#1
}

% disponi definizioni
\newtcolorbox[auto counter, number within=section]{definition}[2][]{%
	colback=red!10,
	colframe=red!40!black,
	sharp corners=northwest,
	fonttitle=\sffamily\bfseries,
	title=Definizione~\thetcbcounter: #2,
	#1
}

% disponi codice
\usepackage{listings}
\usepackage[table]{xcolor}

\definecolor{codegreen}{rgb}{0,0.6,0}
\definecolor{codegray}{rgb}{0.5,0.5,0.5}
\definecolor{codepurple}{rgb}{0.58,0,0.82}
\definecolor{backcolour}{rgb}{0.95,0.95,0.92}

\lstdefinestyle{codestyle}{
		backgroundcolor=\color{black!5}, 
		commentstyle=\color{codegreen},
		keywordstyle=\bfseries\color{magenta},
		numberstyle=\sffamily\tiny\color{black!60},
		stringstyle=\color{green!50!black},
		basicstyle=\ttfamily\footnotesize,
		breakatwhitespace=false,         
		breaklines=true,                 
		captionpos=b,                    
		keepspaces=true,                 
		numbers=left,                    
		numbersep=5pt,                  
		showspaces=false,                
		showstringspaces=false,
		showtabs=false,                  
		tabsize=2
}

\lstdefinestyle{shellstyle}{
		backgroundcolor=\color{black!5}, 
		basicstyle=\ttfamily\footnotesize\color{black}, 
		commentstyle=\color{black}, 
		keywordstyle=\color{black},
		numberstyle=\color{black!5},
		stringstyle=\color{black}, 
		showspaces=false,
		showstringspaces=false, 
		showtabs=false, 
		tabsize=2, 
		numbers=none, 
		breaklines=true
}


\lstdefinelanguage{assembler}{ 
  keywords={AAA, AAD, AAM, AAS, ADC, ADCB, ADCW, ADCL, ADD, ADDB, ADDW, ADDL, AND, ANDB, ANDW, ANDL,
        ARPL, BOUND, BSF, BSFL, BSFW, BSR, BSRL, BSRW, BSWAP, BT, BTC, BTCB, BTCW, BTCL, BTR, 
        BTRB, BTRW, BTRL, BTS, BTSB, BTSW, BTSL, CALL, CBW, CDQ, CLC, CLD, CLI, CLTS, CMC, CMP,
        CMPB, CMPW, CMPL, CMPS, CMPSB, CMPSD, CMPSW, CMPXCHG, CMPXCHGB, CMPXCHGW, CMPXCHGL,
        CMPXCHG8B, CPUID, CWDE, DAA, DAS, DEC, DECB, DECW, DECL, DIV, DIVB, DIVW, DIVL, ENTER,
        HLT, IDIV, IDIVB, IDIVW, IDIVL, IMUL, IMULB, IMULW, IMULL, IN, INB, INW, INL, INC, INCB,
        INCW, INCL, INS, INSB, INSD, INSW, INT, INT3, INTO, INVD, INVLPG, IRET, IRETD, JA, JAE,
        JB, JBE, JC, JCXZ, JE, JECXZ, JG, JGE, JL, JLE, JMP, JNA, JNAE, JNB, JNBE, JNC, JNE, JNG,
        JNGE, JNL, JNLE, JNO, JNP, JNS, JNZ, JO, JP, JPE, JPO, JS, JZ, LAHF, LAR, LCALL, LDS,
        LEA, LEAVE, LES, LFS, LGDT, LGS, LIDT, LMSW, LOCK, LODSB, LODSD, LODSW, LOOP, LOOPE,
        LOOPNE, LSL, LSS, LTR, MOV, MOVB, MOVW, MOVL, MOVSB, MOVSD, MOVSW, MOVSX, MOVSXB,
        MOVSXW, MOVSXL, MOVZX, MOVZXB, MOVZXW, MOVZXL, MUL, MULB, MULW, MULL, NEG, NEGB, NEGW,
        NEGL, NOP, NOT, NOTB, NOTW, NOTL, OR, ORB, ORW, ORL, OUT, OUTB, OUTW, OUTL, OUTSB, OUTSD,
        OUTSW, POP, POPL, POPW, POPB, POPA, POPAD, POPF, POPFD, PUSH, PUSHL, PUSHW, PUSHB, PUSHA, 
				PUSHAD, PUSHF, PUSHFD, RCL, RCLB, RCLW, MOVSL, MOVSB, MOVSW, STOSL, STOSB, STOSW, LODSB, LODSW,
				LODSL, INSB, INSW, INSL, OUTSB, OUTSL, OUTSW
        RCLL, RCR, RCRB, RCRW, RCRL, RDMSR, RDPMC, RDTSC, REP, REPE, REPNE, RET, ROL, ROLB, ROLW,
        ROLL, ROR, RORB, RORW, RORL, SAHF, SAL, SALB, SALW, SALL, SAR, SARB, SARW, SARL, SBB,
        SBBB, SBBW, SBBL, SCASB, SCASD, SCASW, SETA, SETAE, SETB, SETBE, SETC, SETE, SETG, SETGE,
        SETL, SETLE, SETNA, SETNAE, SETNB, SETNBE, SETNC, SETNE, SETNG, SETNGE, SETNL, SETNLE,
        SETNO, SETNP, SETNS, SETNZ, SETO, SETP, SETPE, SETPO, SETS, SETZ, SGDT, SHL, SHLB, SHLW,
        SHLL, SHLD, SHR, SHRB, SHRW, SHRL, SHRD, SIDT, SLDT, SMSW, STC, STD, STI, STOSB, STOSD,
        STOSW, STR, SUB, SUBB, SUBW, SUBL, TEST, TESTB, TESTW, TESTL, VERR, VERW, WAIT, WBINVD,
        XADD, XADDB, XADDW, XADDL, XCHG, XCHGB, XCHGW, XCHGL, XLAT, XLATB, XOR, XORB, XORW, XORL},
  keywordstyle=\color{blue}\bfseries,
  ndkeywordstyle=\color{darkgray}\bfseries,
  identifierstyle=\color{black},
  sensitive=false,
  comment=[l]{\#},
  morecomment=[s]{/*}{*/},
  commentstyle=\color{purple}\ttfamily,
  stringstyle=\color{red}\ttfamily,
  morestring=[b]',
  morestring=[b]"
}

\lstset{language=assembler, style=codestyle}

% disponi sezioni
\usepackage{titlesec}

\titleformat{\section}
	{\sffamily\Large\bfseries} 
	{\thesection}{1em}{} 
\titleformat{\subsection}
	{\sffamily\large\bfseries}   
	{\thesubsection}{1em}{} 
\titleformat{\subsubsection}
	{\sffamily\normalsize\bfseries} 
	{\thesubsubsection}{1em}{}

% tikz
\usepackage{tikz}

% float
\usepackage{float}

% grafici
\usepackage{pgfplots}
\pgfplotsset{width=10cm,compat=1.9}

% disponi alberi
\usepackage{forest}

\forestset{
	rectstyle/.style={
		for tree={rectangle,draw,font=\large\sffamily}
	},
	roundstyle/.style={
		for tree={circle,draw,font=\large}
	}
}

% disponi algoritmi
\usepackage{algorithm}
\usepackage{algorithmic}
\makeatletter
\renewcommand{\ALG@name}{Algoritmo}
\makeatother

% disponi numeri di pagina
\usepackage{fancyhdr}
\fancyhf{} 
\fancyfoot[L]{\sffamily{\thepage}}

\makeatletter
\fancyhead[L]{\raisebox{1ex}[0pt][0pt]{\sffamily{\@title \ \@date}}} 
\fancyhead[R]{\raisebox{1ex}[0pt][0pt]{\sffamily{\@author}}}
\makeatother

\begin{document}
% sezione (data)
\section{Lezione del 08-10-24}

% stili pagina
\thispagestyle{empty}
\pagestyle{fancy}

% testo
\subsection{Descrizione funzionale}
La caratteristica più importante di una rete combinatoria è la funzione $F$, cioé la descrizione funzionale.
Esistono più modi per esprimere questa funzione:
\begin{itemize}
	\item A parole;
	\item Usando notazioni testuali (e..g. il Verilog);
	\item Attraverso \textbf{tabelle di verità}.
		In una tabella di verità contiene due insiemi di colonne: gli ingressi e le uscite.
		Ogni riga mostra una configurazione di stati di ingresso e il corrispondente stato d'uscita. Ad esempio:
	\begin{table}[H]
		\center 
		\begin{tabular} { c  c  c | c c }
			$x_2$ & $x_1$ & $x_0$ & $z_1$ & $z_0$ \\ 
			\hline 
			$0$ & $0$ & $0$ & $0$ & $0$ \\
			$0$ & $0$ & $1$ & $-$ & $1$ \\
			$0$ & $1$ & $0$ & $1$ & $0$ \\
			...
		\end{tabular}
	\end{table}
	Si dice che la variabile di uscita \textbf{riconosce} particolari stati quando si attiva in presenza di essi.
	Inoltre, i trattini indicano stati \textbf{non specificati}, in inglese DC, \textit{don't care}.
	Questi non equivalgono alla fascia di indeterminazione, ma a uno dei due stati accettati, anche se non è importante quale.
	I \textit{don't care} vanno conservati, e non fissati a variabili come $0$ o $1$, in quanto è importante mantenere il funzionamento interno delle reti il più semplice possibile. 
\end{itemize}

\subsubsection{Descrizione e sintesi}
Una \textbf{descrizione} di una rete deve essere formale, in modo che si possa capire esattamente cosa fa quella rete.
La \textbf{sintesi} di una rete è il progetto stesso di realizzazione della rete, cioè quali componenti combinare in quale modo, ecc...
Prima si fa la descrizione, e poi la sintesi.

Notiamo una proprietà fondamentale: ogni rete combinatoria di $N$ ingressi e $M$ uscite può essere realizzata interconnettendo $M$ reti combinatorie ad $N$ ingressi ed una uscita.
Questo ci permette di trattare tutte le reti con reti con una sola uscita.

\subsection{Reti a 0 ingressi}
Le reti a 0 ingresso di uscita si chiamano \textbf{generatori di costante}, e rappresentano un caso degenere.
Si indicano come: 

\begin{center}
	\begin{circuitikz}
		\draw[->] (0.5,0) to (1,0);
    \draw (0,0) node[draw, rectangle, minimum width = 1cm, minimum height = 1cm] {1};

		\draw[->] (2.5,0) to (3,0);
    \draw (2,0) node[draw, rectangle, minimum width = 1cm, minimum height = 1cm] {0};
	\end{circuitikz}
\end{center}

La loro uscita chiaramente vale $1$ o $0$ costante.
Fisicamente, i generatori di costante si realizzano collegando resistori in serie al VCC (genera $1$) o a massa (genera $0$), ergo:

\begin{center}
	\begin{circuitikz}
		\draw[->] (1,0) to (1.5,0);
    \draw (0,0) node[draw, rectangle, minimum width = 2cm, minimum height = 2cm] {};

		\draw (0,-1) node[below] {1};
		
		\draw (0, 1) to[ american voltage source, l=VCC, transform shape, scale=0.5] (0,0);
		\draw (0,0) to [ R, transform shape, scale=0.5] (2,0);

	\end{circuitikz}
	\hspace{1cm}
	\begin{circuitikz}
		\draw[->] (1,0) to (1.5,0);
    \draw (0,0) node[draw, rectangle, minimum width = 2cm, minimum height = 2cm] {};

		\draw (0,-1) node[below] {0};

		\draw (0,0) to [ R, transform shape, scale=0.5] (2,0);
		\draw (0, 0) node[ground] {};

	\end{circuitikz}
\end{center}

\subsection{Reti a 1 ingresso}
\subsubsection{Invertitore}
L'invertitore è una rete descritta dalla tabella di verità:

\begin{table}[H]
	\center 
	\begin{tabular} { c | c }
		$x$ & $z$ \\ 
		\hline 
		$0$ & $1$ \\
		$1$ & $0$ \\
	\end{tabular}
\end{table}

e indicata come:

\begin{center}
	\begin{circuitikz}
			\draw
			(0,0) node[not port] (mynot) {};
	\end{circuitikz}
\end{center}

Essenzialmente nega il suo ingresso.

\subsubsection{Elemento neutro}
L'elemento neutro, detto anche \textit{buffer}, è una rete descritta dalla tabella di verità:

\begin{table}[H]
	\center 
	\begin{tabular} { c | c }
		$x$ & $z$ \\ 
		\hline 
		$0$ & $0$ \\
		$1$ & $1$ \\
	\end{tabular}
\end{table}

e indicata come:

\begin{center}
	\begin{circuitikz}
			\draw
			(0,0) node[buffer port] (mynot) {};
	\end{circuitikz}
\end{center}

Lascia il suo ingresso invariato.
Può avere un utilità come rete di rallentamento, in quanto, inevitabilmente, si perde tempo per attraversarla (pensa alla NOP).
Questo è utile per le temporizzazioni delle reti.

Inoltre, dal punto di vista elettrico, l'elemento neutro ha anche un utilità per la \textbf{rigenerazione} dei segnali.
Infatti, essendo collegato a massa e al VCC, può prendere segnali scadenti (vicini alla fascia di indeterminazione) e trasformarli in segnali di buona qualità (vicini al fondoscala).
Questa proprietà, veramente, è comune a tutte le reti logiche, ma l'elemento neutro è l'unico che non ha altri effetti collaterali.

\subsubsection{Reti costanti}
Si possono interpretare i generatori di costante come reti ad un ingresso degeneri.
Effettivamente, restano tali a se stesse, in quanto gli ingressi sono ignorati.
Le loro tabelle di verità sono:

\begin{center}
\begin{minipage}[t]{0.2\textwidth} % Left half of the page
	Generatore di 1:
\begin{table}[H]
	\center 
	\begin{tabular} { c | c }
		$x$ & $z$ \\ 
		\hline 
		$0$ & $1$ \\
		$1$ & $1$ \\
	\end{tabular}
\end{table}
\end{minipage}%
\hspace{2cm}
\begin{minipage}[t]{0.2\textwidth} % Right half of the page
	Generatore di 0:
\begin{table}[H]
	\center 
	\begin{tabular} { c | c }
		$x$ & $z$ \\ 
		\hline 
		$0$ & $0$ \\
		$1$ & $0$ \\
	\end{tabular}
\end{table}
\end{minipage}
\end{center}

\subsection{Reti a 2 ingressi}
La prima domanda da porsi quando si parla di reti a 2 (come $N$) ingressi, è quante reti possiamo creare in tutto.
Su $N$ ingressi, la tabella di verità avrà $2^N$ righe.
Le configurazioni possibili di $0$ e $1$ su $2^N$ righe sono $2^{2^N}$.
Ergo, nel caso $N=2$, abbiamo $2^{2^2} = 16$ possibili combinazioni, che sono:

\begin{table}[H]
	\center 
	\begin{tabular} { c  c | c >{\columncolor{green!40!white}}c c c 
										c c >{\columncolor{red!40!white}}c >{\columncolor{blue!40!white}}c
										>{\columncolor{purple!40!white}}c >{\columncolor{orange!40!white}}c c c 
										c c >{\columncolor{cyan!40!white}}c c }
		$x_1$ & $x_0$ & $z^0$ & $z^1$ & $z^2$ & $z^3$ & $z^4$ & $z^5$ & $z^6$ & $z^7$ & $z^8$ & $z^9$ & $z^{10}$ & $z^{11}$ & $z^{12}$ & $z^{13}$ & $z^{14}$ & $z^{15}$ \\ 
		\hline 
		0 & 0 & 0 & 0 & 0 & 0 & 0 & 0 & 0 & 0 & 1 & 1 & 1 & 1 & 1 & 1 & 1 & 1 \\  
		0 & 1 & 0 & 0 & 0 & 0 & 1 & 1 & 1 & 1 & 0 & 0 & 0 & 0 & 1 & 1 & 1 & 1 \\ 
		1 & 0 & 0 & 0 & 1 & 1 & 0 & 0 & 1 & 1 & 0 & 0 & 1 & 1 & 0 & 0 & 1 & 1 \\ 
		1 & 1 & 0 & 1 & 0 & 1 & 0 & 1 & 0 & 1 & 0 & 1 & 0 & 1 & 0 & 1 & 0 & 1 \\ 
	\end{tabular}
\end{table}

Ad alcune di queste corrispondono nomi speciali.
Vediamole nel dettaglio:

\subsubsection{Porta AND}
La porta AND, indicata in \color{green!50!black} $z^1$ \color{black}, corrisponde al $\wedge$ logico, ergo $z = 1 \Leftrightarrow x_0 = x_1 = 1$.
Si indica come:

\begin{center}
	\begin{circuitikz}
			\draw
			(0,0) node[and port] (mynot) {};
	\end{circuitikz}
\end{center}

e ha tabella di verità:
\begin{table}[H]
	\center
	\begin{tabular} { c  c | c }
		$x_1$ & $x_0$ & $z$ \\ 
		\hline 
		0 & 0 & 0 \\ 
		0 & 1 & 0 \\ 
		1 & 0 & 0 \\ 
		1 & 1 & 1 \\
	\end{tabular}
\end{table}

\subsubsection{Porta XOR}
La porta XOR, indicata in \color{red!50!black} $z^6$ \color{black}, corrisponde all'\textit{aut} logico, cioè esclusivo, ergo $z = 1 \Leftrightarrow x_0 \neq x_1$.
Si indica come:

\begin{center}
	\begin{circuitikz}
			\draw
			(0,0) node[xor port] (mynot) {};
	\end{circuitikz}
\end{center}

e ha tabella di verità:
\begin{table}[H]
	\center
	\begin{tabular} { c  c | c }
		$x_1$ & $x_0$ & $z$ \\ 
		\hline 
		0 & 0 & 0 \\ 
		0 & 1 & 1 \\ 
		1 & 0 & 1 \\ 
		1 & 1 & 0 \\
	\end{tabular}
\end{table}

\subsubsection{Porta OR}
La porta OR, indicata in \color{blue!50!black} $z^7$ \color{black}, corrisponde al $\lor$ logico, ergo $z = 0 \Leftrightarrow x_0 = x_1 = 0$.
Si indica come:

\begin{center}
	\begin{circuitikz}
			\draw
			(0,0) node[or port] (mynot) {};
	\end{circuitikz}
\end{center}

e ha tabella di verità:
\begin{table}[H]
	\center
	\begin{tabular} { c  c | c }
		$x_1$ & $x_0$ & $z$ \\ 
		\hline 
		0 & 0 & 0 \\ 
		0 & 1 & 1 \\ 
		1 & 0 & 1 \\ 
		1 & 1 & 1 \\
	\end{tabular}
\end{table}

\subsubsection{Porta NOR}
La porta NOR, indicata in \color{purple!50!black} $z^8$ \color{black}, corrisponde alla negazione dell'$\lor$ logico, ergo $z = 1 \Leftrightarrow x_0 = x_1 = 0$.
Si indica come:

\begin{center}
	\begin{circuitikz}
			\draw
			(0,0) node[nor port] (mynot) {};
	\end{circuitikz}
\end{center}

e ha tabella di verità:
\begin{table}[H]
	\center
	\begin{tabular} { c  c | c }
		$x_1$ & $x_0$ & $z$ \\ 
		\hline 
		0 & 0 & 1 \\ 
		0 & 1 & 0 \\ 
		1 & 0 & 0 \\ 
		1 & 1 & 0 \\
	\end{tabular}
\end{table}

\subsubsection{Porta XNOR}
La porta XNOR, indicata in \color{orange!50!black} $z^9$ \color{black}, corrisponde alla negazione dell'\textit{aut} logico, ergo $z = 1 \Leftrightarrow x_0 = x_1$.
Si indica come:

\begin{center}
	\begin{circuitikz}
			\draw
			(0,0) node[xnor port] (mynot) {};
	\end{circuitikz}
\end{center}

e ha tabella di verità:
\begin{table}[H]
	\center
	\begin{tabular} { c  c | c }
		$x_1$ & $x_0$ & $z$ \\ 
		\hline 
		0 & 0 & 1 \\ 
		0 & 1 & 0 \\ 
		1 & 0 & 0 \\ 
		1 & 1 & 1 \\
	\end{tabular}
\end{table}

\subsubsection{Porta NAND}
La porta NAND, indicata in \color{cyan!50!black} $z^14$ \color{black}, corrisponde alla negazione dell'$\wedge$ logico, ergo $z = 0 \Leftrightarrow x_0 = x_1 = 1$.
Si indica come:

\begin{center}
	\begin{circuitikz}
			\draw
			(0,0) node[nand port] (mynot) {};
	\end{circuitikz} 
\end{center}

e ha tabella di verità:
\begin{table}[H]
	\center
	\begin{tabular} { c  c | c }
		$x_1$ & $x_0$ & $z$ \\ 
		\hline 
		0 & 0 & 1 \\ 
		0 & 1 & 1 \\ 
		1 & 0 & 1 \\ 
		1 & 1 & 0 \\
	\end{tabular}
\end{table}

\par\smallskip
Si dovrebbe essere notato che un pallino finale indica negazione.
A volte si usa solo questa notazione, invece di tutta la porta NOT.

\subsubsection{Casi degeneri}
Alcuni casi speciali della tabella delle possibili reti a due porte sono degeneri: abbiamo due generatori di costante ($z^0$ e $z^{15}$), due elementi neutri, rispettivamente su $x_1$ e $x_0$ ($z^3$ e $z_5$), e due inversori sugli stessi ingressi ($z_{10}$ e $z_{12}$).

\subsection{AND e OR a più ingressi}
Posso pensare di estendere AND e OR ad $N$ ingressi:
\begin{itemize}
	\item \textbf{AND a $N$ ingressi:} l'uscita vale $1$ se tutti gli $N$ ingressi valgono $1$;
	\item \textbf{OR a $N$ ingressi:} l'uscita vale $1$ se almeno un'ingresso vale $1$;
\end{itemize}

Questo può essere realizzato concatenando più porte logiche dello stesso tipo, come segue:

\begin{center}
	\begin{circuitikz} 
		\node (short) at (-1.5, 1.28) {}; 
		\draw (0,0) node[and port] (myand2) {}
		(2,1) node[and port] (myand3) {}
		(short) -- (myand3.in 1)
		(myand2.out) -- (myand3.in 2);
	\end{circuitikz} 
\end{center}

La dimostrazione è semplice dalla tabella di verità, o dalle proprietà degli operatori logici.

Una nota va fatta sulle combinazioni di più di 3 ingressi, infatti una rete del genere è sconveniente:

\begin{center}
	\begin{circuitikz} 
		\node (short1) at (-3.5, 1.28) {}; 
		\node (short2) at (-3.5, 0.28) {}; 
		\draw (0,0) node[and port] (myand1) {}
		(-2,-1) node[and port] (myand2) {}
		(2,1) node[and port] (myand3) {}
		(short1) -- (myand3.in 1)
		(short2) -- (myand1.in 1)
		(myand2.out) -- (myand1.in 2)
		(myand1.out) -- (myand3.in 2);
	\end{circuitikz} 
\end{center}


in quanto il segnale deve attraversare al massimo 3 livelli di logica, mentre disponendo le porte come:

\begin{center}
	\begin{circuitikz} \draw
			(0,2) node[and port] (myand1) {}
			(0,0) node[and port] (myand2) {}
			(2,1) node[and port] (myand3) {}
			(myand1.out) -- (myand3.in 1)
			(myand2.out) -- (myand3.in 2);
	\end{circuitikz} 
\end{center}

il segnale dovrà attraversare al massimo 2 livelli di logica.

Conviene quindi disporre gli $N$ ingressi e le relative porte come un'albero binario bilanciato, in modo da minimizzare gli attraversamenti di livelli di logica.
Si noti che questo discorso vale per AND e OR: non per NAND, NOR, XOR o XNOR. 

Possiamo osservare velocemente cosa accade se si collegano queste porte fra di loro:
\begin{itemize}
	\item \textbf{NAND:} un singolo NAND può formare un NOT quando i suoi ingressi sono uniti insieme.
		Se si mettono 2 NAND in serie (a \textit{cascata}) in questo modo, si ottiene di nuovo un AND;
	\item \textbf{NOR:} un singolo NOR può formare un NAND nello stesso modo del NAND.
		Se si mettono 2 NOR a cascata, si ottiene di nuovo un NOR;
	\item \textbf{XOR:} con $\geq 2$ XOR, si crea effettivamente un controllore di parità, ergo una rete che si attiva quando un numero dispari dei suoi ingressi sono accesi;
	\item \textbf{XNOR:} con $\geq 2$ XNOR, si ha l'opposto che con gli XOR: si crea una rete che si attiva quando un numero pari dei suoi ingressi sono accesi.
\end{itemize}

Queste porte si indicano solitamente come con gli input su unica orizzontale, che risulta più compatto.

\subsection{Algebra di Boole}
L'algebra di Boole adopera gli operatori logici conosciuti, applicati ad elementi del campo binario $GF(2) = \{0 , 1\}$

Vediamo questi operatori:
\begin{itemize}
	\item \textbf{Complemento logico:} si indica come $\overline{x}$, oppure $!x$ o $/x$. 
		Si definisce come: $$ \overline{0} = 1, \quad \bar{1} = 0 $$
	\item\ \textbf{Somma logica:} si indica con $x + y$, e ha tabella di verità:
	\begin{table}[H]
		\center
		\begin{tabular} { c  c | c }
			$x$ & $y$ & $ x + y $ \\ 
			\hline 
			0 & 0 & 0 \\ 
			0 & 1 & 1 \\ 
			1 & 0 & 1 \\ 
			1 & 1 & 1 \\
		\end{tabular}
	\end{table}
		cioè equivale all'OR.
	\item \textbf{Prodotto logico:} si indica con $x \cdot y$, e ha tabella di verità:
	\begin{table}[H]
		\center
		\begin{tabular} { c  c | c }
			$x$ & $y$ & $ x \cdot y $ \\ 
			\hline 
			0 & 0 & 0 \\ 
			0 & 1 & 0 \\ 
			1 & 0 & 0 \\ 
			1 & 1 & 1 \\
		\end{tabular}
	\end{table}
		cioè equivale all'AND.

\end{itemize}

\par\smallskip

Su questi operatori valgono le proprietà:
\begin{enumerate}
	\item \textbf{Involutiva del complemento:} $\overline{\bar{x}} = x$;
	\item \textbf{Commutativa della somma e del prodotto:} $ x + y = y + x, \quad x \cdot y = y \cdot x$;
	\item \textbf{Associativa della somma:} $ x + y + z = (x + y) + z = x + (y + z)$;
	\item \textbf{Associativa del prodotto:} $ x \cdot y \cdot z = (x \cdot y) \cdot z = x \cdot (y \cdot z)$;
	\item \textbf{Distributiva della somma rispetto al prodotto:} $ x \cdot (y + z) = (x \cdot y) + (x \cdot z) $;
	\item \textbf{Distributiva del prodotto rispetto alla somma:} $ x + (y \cdot z) = (x + y) \cdot (x + z) $. Bisogna fare attenzione in quanto questa non vale in $\mathbb{R}$;
	\item \textbf{Complementazione:} $ x \cdot \overline{x} = 0, \quad x + \bar{x} = 1 $;
	\item \textbf{Unione e intersezione:} $ x + 0 = x, \quad x + 1 = 1 $, cioè $0$ è l'elemento neutro e $1$ l'elemento assorbente della somma (non lo è in $\mathbb{R}$); \\
																				$ x \cdot 0 = 0, \quad x \cdot 1 = x $, cioè $1$ è l'elemento neutro e $0$ l'elemento assorbente del prodotto;
	\item \textbf{Idempotenza:} $x + x = x$, \quad $x \cdot x = x$, altra che non vale in $\mathbb{R}$;
	\item \textbf{Leggi di De Morgan:} $\overline{x \cdot x} = \overline{x} + \bar{x}$ e $\overline{x + x} = \bar{x} \cdot \bar{x}$.
\end{enumerate}

\subsubsection{Teoremi di De Morgan}
Le leggi di De Morgan comuni della logica si estendono ad $N$ variabili come:
\begin{enumerate}
	\item $\overline{x_0 \cdot x_1 \cdot ... \cdot x_n} = \overline{x}_0 + \bar{x}_1 + ... + \bar{x}_n$
	\item $\overline{x_0 + x_1 + ... + x_{n}} = \overline{x}_0 \cdot \bar{x}_1 \cdot ... \cdot \bar{x}_n$
\end{enumerate}

\noindent
\textbf{\textsf{Dimostrazione per induzione}} \\
Richiamiamo le basi dell'induzione:
\begin{itemize}
	\item Si dimostra che una proprietà vale per un certo numero $n_0$ (passo base);
	\item Si dimostra che se vale per un certo $n \geq n_0$, allora vale anche per $n + 1$.
\end{itemize}

Partiamo con le dimostrazioni classiche ottenute con le tabelle di verità:

\begin{table}[H]
	\center
	\begin{tabular} { c  c | c | c | c | c | c }
		$x$ & $y$ & $ x \cdot y $ & $\overline{x \cdot y}$ & $\overline{x}$ & $\bar{y}$ & $\bar{x} + \bar{y}$ \\ 
		\hline 
		0 & 0 & 0 & 1 & 1 & 1 & 1 \\  
		0 & 1 & 0 & 1 & 1 & 0 & 1 \\ 
		1 & 0 & 0 & 1 & 0 & 1 & 1 \\ 
		1 & 1 & 1 & 0 & 0 & 1 & 0
	\end{tabular}
\end{table}

che ci portano a $n_0 = 2$.
Posso quindi porre l'ipotesi:

$$
\overline{x_0 \cdot ... \cdot x_{n-1}} = \overline{x}_0 + ... + \bar{x}_{n-1}
$$

e la tesi:

$$
\overline{x_0 \cdot ... \cdot x_{n-1} \cdot x_n} = \overline{x}_0 + ... + \bar{x}_{n-1} + \cdot x_n
$$

A questo punto faccio il passo induttivo, sfruttando l'associatività del prodotto (o della somma), e quindi riscrivendo la tesi come:

$$
\overline{\alpha \cdot x_n}, \quad \alpha = x_0 + ... x_{n-1} 
$$
dove notiamo la variabile introdotta $\alpha$, se complementata, rispetta:
$$
\overline{\alpha} = \overline{x_0 \cdot ... \cdot x_{n-1}} = \bar{x}_0 + ... + \bar{x}_{n-1}
$$
dall'ipotesi.

Possiamo quindi svolgere il passaggio:
$$
\overline{\alpha \cdot x_n} = \overline{\alpha} + \bar{x}_n =  \bar{x}_0 + ... + \bar{x}_{n-1} + \bar{x}_n
$$
che conferma la tesi.

\subsubsection{Algebra di Boole e reti combinatorie}
Esiste una corrispondenza fra l'algebra di Boole e le reti combinatorie.
In particolare, si ha che:
\begin{itemize}
	\item \textbf{Data una rete combinatoria}, (comunque complessa), è sempre possibile trovare un'espressione booleana che mette in relazione ogni sua uscita con gli ingressi (in verità un'espressione per ogni uscita);
	\item \textbf{Data un'espressione booleana}, p sempre possibile sintetizzare una rete combinatoria (ad un'uscita) in cui la relazione tra ingresso ed uscita data è dall'espressione.
\end{itemize}

Si noti che, effettivamente, espressioni logiche equivalenti $\Leftrightarrow$ reti logiche che svolgono lo stesso compito, ma non per questo l'equivalenza è totale: ci conviene creare reti che usano meno componenti possibili, in quanto queste le rende più affidabili, più economiche e meno dispendiose di energia.
Le proprietà dell'algebra di Boole possono quindi essere usate per ridurre il numero di porte logiche, attraverso un processo che chiameremo \textbf{minimizzazione}.

\end{document}


\documentclass[a4paper,11pt]{article}
\usepackage[a4paper, margin=8em]{geometry}

% usa i pacchetti per la scrittura in italiano
\usepackage[french,italian]{babel}
\usepackage[T1]{fontenc}
\usepackage[utf8]{inputenc}
\frenchspacing 

% usa i pacchetti per la formattazione matematica
\usepackage{amsmath, amssymb, amsthm, amsfonts}

% usa altri pacchetti
\usepackage{gensymb}
\usepackage{hyperref}
\usepackage{standalone}

\usepackage{colortbl}

% imposta il titolo
\title{Appunti Reti Logiche}
\author{Luca Seggiani}
\date{2024}

% imposta lo stile
% usa helvetica
\usepackage[scaled]{helvet}
% usa palatino
\usepackage{palatino}
% usa un font monospazio guardabile
\usepackage{lmodern}

\renewcommand{\rmdefault}{ppl}
\renewcommand{\sfdefault}{phv}
\renewcommand{\ttdefault}{lmtt}

% circuiti
\usepackage{circuitikz}
\usetikzlibrary{babel}

% disponi il titolo
\makeatletter
\renewcommand{\maketitle} {
	\begin{center} 
		\begin{minipage}[t]{.8\textwidth}
			\textsf{\huge\bfseries \@title} 
		\end{minipage}%
		\begin{minipage}[t]{.2\textwidth}
			\raggedleft \vspace{-1.65em}
			\textsf{\small \@author} \vfill
			\textsf{\small \@date}
		\end{minipage}
		\par
	\end{center}

	\thispagestyle{empty}
	\pagestyle{fancy}
}
\makeatother

% disponi teoremi
\usepackage{tcolorbox}
\newtcolorbox[auto counter, number within=section]{theorem}[2][]{%
	colback=blue!10, 
	colframe=blue!40!black, 
	sharp corners=northwest,
	fonttitle=\sffamily\bfseries, 
	title=Teorema~\thetcbcounter: #2, 
	#1
}

% disponi definizioni
\newtcolorbox[auto counter, number within=section]{definition}[2][]{%
	colback=red!10,
	colframe=red!40!black,
	sharp corners=northwest,
	fonttitle=\sffamily\bfseries,
	title=Definizione~\thetcbcounter: #2,
	#1
}

% disponi codice
\usepackage{listings}
\usepackage[table]{xcolor}

\definecolor{codegreen}{rgb}{0,0.6,0}
\definecolor{codegray}{rgb}{0.5,0.5,0.5}
\definecolor{codepurple}{rgb}{0.58,0,0.82}
\definecolor{backcolour}{rgb}{0.95,0.95,0.92}

\lstdefinestyle{codestyle}{
		backgroundcolor=\color{black!5}, 
		commentstyle=\color{codegreen},
		keywordstyle=\bfseries\color{magenta},
		numberstyle=\sffamily\tiny\color{black!60},
		stringstyle=\color{green!50!black},
		basicstyle=\ttfamily\footnotesize,
		breakatwhitespace=false,         
		breaklines=true,                 
		captionpos=b,                    
		keepspaces=true,                 
		numbers=left,                    
		numbersep=5pt,                  
		showspaces=false,                
		showstringspaces=false,
		showtabs=false,                  
		tabsize=2
}

\lstdefinestyle{shellstyle}{
		backgroundcolor=\color{black!5}, 
		basicstyle=\ttfamily\footnotesize\color{black}, 
		commentstyle=\color{black}, 
		keywordstyle=\color{black},
		numberstyle=\color{black!5},
		stringstyle=\color{black}, 
		showspaces=false,
		showstringspaces=false, 
		showtabs=false, 
		tabsize=2, 
		numbers=none, 
		breaklines=true
}


\lstdefinelanguage{assembler}{ 
  keywords={AAA, AAD, AAM, AAS, ADC, ADCB, ADCW, ADCL, ADD, ADDB, ADDW, ADDL, AND, ANDB, ANDW, ANDL,
        ARPL, BOUND, BSF, BSFL, BSFW, BSR, BSRL, BSRW, BSWAP, BT, BTC, BTCB, BTCW, BTCL, BTR, 
        BTRB, BTRW, BTRL, BTS, BTSB, BTSW, BTSL, CALL, CBW, CDQ, CLC, CLD, CLI, CLTS, CMC, CMP,
        CMPB, CMPW, CMPL, CMPS, CMPSB, CMPSD, CMPSW, CMPXCHG, CMPXCHGB, CMPXCHGW, CMPXCHGL,
        CMPXCHG8B, CPUID, CWDE, DAA, DAS, DEC, DECB, DECW, DECL, DIV, DIVB, DIVW, DIVL, ENTER,
        HLT, IDIV, IDIVB, IDIVW, IDIVL, IMUL, IMULB, IMULW, IMULL, IN, INB, INW, INL, INC, INCB,
        INCW, INCL, INS, INSB, INSD, INSW, INT, INT3, INTO, INVD, INVLPG, IRET, IRETD, JA, JAE,
        JB, JBE, JC, JCXZ, JE, JECXZ, JG, JGE, JL, JLE, JMP, JNA, JNAE, JNB, JNBE, JNC, JNE, JNG,
        JNGE, JNL, JNLE, JNO, JNP, JNS, JNZ, JO, JP, JPE, JPO, JS, JZ, LAHF, LAR, LCALL, LDS,
        LEA, LEAVE, LES, LFS, LGDT, LGS, LIDT, LMSW, LOCK, LODSB, LODSD, LODSW, LOOP, LOOPE,
        LOOPNE, LSL, LSS, LTR, MOV, MOVB, MOVW, MOVL, MOVSB, MOVSD, MOVSW, MOVSX, MOVSXB,
        MOVSXW, MOVSXL, MOVZX, MOVZXB, MOVZXW, MOVZXL, MUL, MULB, MULW, MULL, NEG, NEGB, NEGW,
        NEGL, NOP, NOT, NOTB, NOTW, NOTL, OR, ORB, ORW, ORL, OUT, OUTB, OUTW, OUTL, OUTSB, OUTSD,
        OUTSW, POP, POPL, POPW, POPB, POPA, POPAD, POPF, POPFD, PUSH, PUSHL, PUSHW, PUSHB, PUSHA, 
				PUSHAD, PUSHF, PUSHFD, RCL, RCLB, RCLW, MOVSL, MOVSB, MOVSW, STOSL, STOSB, STOSW, LODSB, LODSW,
				LODSL, INSB, INSW, INSL, OUTSB, OUTSL, OUTSW
        RCLL, RCR, RCRB, RCRW, RCRL, RDMSR, RDPMC, RDTSC, REP, REPE, REPNE, RET, ROL, ROLB, ROLW,
        ROLL, ROR, RORB, RORW, RORL, SAHF, SAL, SALB, SALW, SALL, SAR, SARB, SARW, SARL, SBB,
        SBBB, SBBW, SBBL, SCASB, SCASD, SCASW, SETA, SETAE, SETB, SETBE, SETC, SETE, SETG, SETGE,
        SETL, SETLE, SETNA, SETNAE, SETNB, SETNBE, SETNC, SETNE, SETNG, SETNGE, SETNL, SETNLE,
        SETNO, SETNP, SETNS, SETNZ, SETO, SETP, SETPE, SETPO, SETS, SETZ, SGDT, SHL, SHLB, SHLW,
        SHLL, SHLD, SHR, SHRB, SHRW, SHRL, SHRD, SIDT, SLDT, SMSW, STC, STD, STI, STOSB, STOSD,
        STOSW, STR, SUB, SUBB, SUBW, SUBL, TEST, TESTB, TESTW, TESTL, VERR, VERW, WAIT, WBINVD,
        XADD, XADDB, XADDW, XADDL, XCHG, XCHGB, XCHGW, XCHGL, XLAT, XLATB, XOR, XORB, XORW, XORL},
  keywordstyle=\color{blue}\bfseries,
  ndkeywordstyle=\color{darkgray}\bfseries,
  identifierstyle=\color{black},
  sensitive=false,
  comment=[l]{\#},
  morecomment=[s]{/*}{*/},
  commentstyle=\color{purple}\ttfamily,
  stringstyle=\color{red}\ttfamily,
  morestring=[b]',
  morestring=[b]"
}

\lstset{language=assembler, style=codestyle}

% disponi sezioni
\usepackage{titlesec}

\titleformat{\section}
	{\sffamily\Large\bfseries} 
	{\thesection}{1em}{} 
\titleformat{\subsection}
	{\sffamily\large\bfseries}   
	{\thesubsection}{1em}{} 
\titleformat{\subsubsection}
	{\sffamily\normalsize\bfseries} 
	{\thesubsubsection}{1em}{}

% tikz
\usepackage{tikz}

% float
\usepackage{float}

% grafici
\usepackage{pgfplots}
\pgfplotsset{width=10cm,compat=1.9}

% disponi alberi
\usepackage{forest}

\forestset{
	rectstyle/.style={
		for tree={rectangle,draw,font=\large\sffamily}
	},
	roundstyle/.style={
		for tree={circle,draw,font=\large}
	}
}

% disponi algoritmi
\usepackage{algorithm}
\usepackage{algorithmic}
\makeatletter
\renewcommand{\ALG@name}{Algoritmo}
\makeatother

% disponi numeri di pagina
\usepackage{fancyhdr}
\fancyhf{} 
\fancyfoot[L]{\sffamily{\thepage}}

\makeatletter
\fancyhead[L]{\raisebox{1ex}[0pt][0pt]{\sffamily{\@title \ \@date}}} 
\fancyhead[R]{\raisebox{1ex}[0pt][0pt]{\sffamily{\@author}}}
\makeatother

\begin{document}
% sezione (data)
\section{Lezione del 09-10-24}

% stili pagina
\thispagestyle{empty}
\pagestyle{fancy}

% testo
\subsection{Decoder}
Un decoder è una rete con $N$ ingressi e $p$ uscite con $p = 2^N$.
Si indica come:

\begin{center}
	\begin{circuitikz}
		\node[trapezium, trapezium angle=60, minimum height=1cm, minimum width=2cm, draw] (decoder) at (0,0) {};
		\node (xn) at (-0.5,1.5) {$x_{N-1}$};
		\node (x) at (0.5,1.5) {$x_0$};
		
		\draw (xn) -- (-0.5, 0.5);
		\draw (x) -- (0.5, 0.5);

		\node (zn) at (-1,-1.5) {$z_{p-1}$};
		\node (z) at (1,-1.5) {$z_0$};
		\node (j) at(0, -1.5) {$z_j$};

		\draw (zn) -- (-1, -0.5);
		\draw (z) -- (1, -0.5);
		\draw (j) -- (0, -0.5);

		\node at (0.11, 1.45) {$...$};
		\node at (-0.4, -1.55) {$...$};
		\node at (0.5, -1.55) {$...$};

	\end{circuitikz}
\end{center}

La sua legge di corrispondenza stabilisce che ogni uscita riconosce uno ed un solo stato di ingresso, in particolare l'uscita $j$-esima ($z_j$) riconosce lo stato di ingresso i cui bit sono la codifica di $j$ in base 2, cioè:
$$
(x_{n-1}, ..., x_0)_{2} = j
$$

Ad esempio, un decoder da 2 a 4 ha tabella di verità:

\begin{table}[h!]
	\center 
	\begin{tabular} { c c | c c c c }
		$x_1$ &$x_0$ &$z_0$ &$z_1$ &$z_2$ &$z_3$ \\
	\hline 
	0 & 0 & 1 & 0 & 0 & 0 \\
	0 & 1 & 0 & 1 & 0 & 0 \\
	1 & 0 & 0 & 0 & 1 & 0 \\
	1 & 1 & 0 & 0 & 0 & 1 \\
	\end{tabular}
\end{table}
che equivale alla codifica \textit{one-hot} del binario in ingresso (cioè ogni numero codificato da $n$ bit viene mandato al $j$-esimo di $p$ output che corrispondono uno ad uno ai numeri rappresentabili).

Vediamo di passare da questa descrizione ad una sintesi della rete. Abbiamo che:
\[
	\begin{cases}
			
z_3 = x_1 \cdot x_0 \\ 
z_2 = x_1 \cdot \overline{x}_0 \\ 
z_1 = \overline{x_1} \cdot x_0 \\ 
z_0 = \overline{x_1} \cdot \overline{x_0} \\ 
	\end{cases}
\]
cioè ogni "indice" del decoder corrisponde al prodotto dei due ingressi opportunamente negati: l'ultima uscità avra tutti i bit attivi (sarebbe $2^N -1$ considerando numeri naturali), ergo prende il prodotto di tutti gli ingressi.
Di contro, la prima uscita ($0$) avrà tutti i bit disattivi, quindi prenderà il prodotto di tutti gli ingressi negati.
Gli altri numeri vengono indirizzati prendendo il prodotto e complementando i bit che quel particolare numero si aspetterebbe come $0$.
Notiamo che, sebbene si abbiano 4 negazioni, nella rete fisica conviene negare gli input in entrata risparmiando 2 invertitori.

Per le figure, rimandiamo a \url{https://github.com/Guray00/IngegneriaInformatica/blob/master/SECONDO%20ANNO/I%20SEMESTRE/Reti%20Logiche/Diapositive%20OCR/Reti%20combinatorie%20ocr.pdf}.

Generalizziamo quindi questa struttura a decoder da $N$ a $2^N$, applicando quanto detto prima. Si avrà:
\[
	\begin{cases}
		z_0 = \overline{x_{N-1}} \cdot \overline{x_{N-2}} \cdot ... \cdot \overline{x_1} \cdot \overline{x_0}	\\
		z_1 = \overline{x_{N-1}} \cdot \overline{x_{N-2}} \cdot ... \cdot \overline{x_1} \cdot x_0	\\
		... \\ 
		z_{p-2} = x_{N-1} \cdot x_{N-2} \cdot ... \cdot x_1 \cdot \overline{x_0} \\
		z_{p-1} = x_{N-1} \cdot x_{N-2} \cdot ... \cdot x_1 \cdot x_0	\\
	\end{cases}
\]

Vediamo quindi le codifiche in Verilog di decoder a diversi valori di $N$.
Si definisce innanzitutto il caso banale di $N = 1$, che finora non è stato trattato.
Questo servirà a definire, in maniera gerarchica (ma come vedremo imperfetta), decoder più complessi:

\lstinputlisting[language=verilog, style=codestyle]{../verilog/10-09/decoders/b1to2_decoder.v}

Possiamo quindi definire il decoder da 2 a 4 visto prima:

\lstinputlisting[language=verilog, style=codestyle]{../verilog/10-09/decoders/b2to4_decoder.v}

un decoder da 3 a 8:

\lstinputlisting[language=verilog, style=codestyle]{../verilog/10-09/decoders/b3to8_decoder.v}

e infine, ad evidenziare quanto velocemente esplode il numero di termini (cioè esponenzialmente), un decoder da 4 a 16:

\lstinputlisting[language=verilog, style=codestyle]{../verilog/10-09/decoders/b4to16_decoder.v}

\subsubsection{Decoder con enabler}
Il problema dei decoder come appena descritti è che sono poco agili nell'espansione: non si possono costruire, come avevamo visto per i gli AND o gli OR, reti di più decoder combinati, a meno di non ridursi a ritrovare quelli che sono effettivamente i mintermini della tabella di verità (come si nota dagli esempi).
Introduciamo per questo motivo il decoder con \textbf{enabler}:

\begin{center}
	\begin{circuitikz}
		\node[trapezium, trapezium angle=60, minimum height=1cm, minimum width=2cm, draw] (decoder) at (0,0) {};
		\node (xn) at (-0.5,1.5) {$x_{N-1}$};
		\node (x) at (0.5,1.5) {$x_0$};
		
		\draw (xn) -- (-0.5, 0.5);
		\draw (x) -- (0.5, 0.5);

		\node (zn) at (-1,-1.5) {$z_{p-1}$};
		\node (z) at (1,-1.5) {$z_0$};
		\node (j) at(0, -1.5) {$z_j$};

		\draw (zn) -- (-1, -0.5);
		\draw (z) -- (1, -0.5);
		\draw (j) -- (0, -0.5);

		\node (e) at(-2, 0) {$e$};
		\draw (e) -- (-0.79,0);

		\node at (0.11, 1.45) {$...$};
		\node at (-0.4, -1.55) {$...$};
		\node at (0.5, -1.55) {$...$};

	\end{circuitikz}
\end{center}

Questi decoder hanno $N + 1$ ingressi, cioè quelli normali più l'enabler, che ha il compito di "accendere" il decoder stesso.
Fisicamente, potremmo semplicemente inserire il decoder $e$ come ingresso aggiuntivo agli AND già predisposti, per avere che:
\[
	z_i =
	\begin{cases}
			y_i \quad e = 1 \\
			0 \quad \ e = 0
	\end{cases}
\]
e quindi:
\[
	\begin{cases}
		z_0 = e \cdot \overline{x_{N-1}} \cdot \overline{x_{N-2}} \cdot ... \cdot \overline{x_1} \cdot \overline{x_0}	\\
		z_1 = e \cdot \overline{x_{N-1}} \cdot \overline{x_{N-2}} \cdot ... \cdot \overline{x_1} \cdot x_0	\\
		... \\ 
		z_{p-2} = e \cdot x_{N-1} \cdot x_{N-2} \cdot ... \cdot x_1 \cdot \overline{x_0}	\\
		z_{p-1} = e \cdot x_{N-1} \cdot x_{N-2} \cdot ... \cdot x_1 \cdot x_0	\\
	\end{cases}
\]

Adesso basta accorgersi che reti di decoder con $N > 2$ possono crearsi concatenando decoder a decoder, cioè usando un decoder con i bit più significativi in entrata per generare l'enabler di $N$ nuovi decoder, i quali ricevono i bit meno significativi in entrata. 

Ad esempio, se vogliamo creare un decoder \texttt{4to16} a partire da decoder \texttt{2to4}, useremo 4 decoder, con gli stessi input ($x_0$ e $x_1$), abilitati da un quinto decoder con input $x_2$ e $x_3$.

Vediamo un'esempio pratico, dato dalle implementazioni in Verilog degli stessi decoder visti prima, ma stavolta dotati di enabler (e posti in cascata, nelle sintesi gerarchiche, attraverso tali enabler).
Si inizia col decoder 1 a 2:

\lstinputlisting[language=verilog, style=codestyle]{../verilog/10-09/enb_decoders/b1to2_enb_decoder.v}

Possiamo quindi definire il decoder da 2 a 4:

\lstinputlisting[language=verilog, style=codestyle]{../verilog/10-09/enb_decoders/b2to4_enb_decoder.v}

un decoder da 3 a 8:

\lstinputlisting[language=verilog, style=codestyle]{../verilog/10-09/enb_decoders/b3to8_enb_decoder.v}

e infine, di cui notiamo la sintesi gerarchica molto più immediata rispetto al caso senza enabler, un decoder da 4 a 16:

\lstinputlisting[language=verilog, style=codestyle]{../verilog/10-09/enb_decoders/b4to16_enb_decoder.v}

\subsection{Demultiplexer}
Il demultiplexer è una rete con $N+1$ ingressi e $p = 2^N$ uscite:

\begin{center}
	\begin{circuitikz}
		\node[rectangle, minimum height=2cm, minimum width=2cm, draw] (multiplex) at (0,0) {};
		\node (x) at (0,1.5) {$x$};	
		\draw (x) -- (0, 1);

		\node (zn) at (-1,-1.5) {$z_{p-1}$};
		\node (z) at (1,-1.5) {$z_0$};
		\node (j) at(0, -1.5) {$z_j$};

		\draw (zn) -- (-1, -1);
		\draw (z) -- (1, -1);
		\draw (j) -- (0, -1);

		\node (bn) at(-2, 0.5) {$b_{N-1}$};
		\draw (bn) -- (-1,0.5);

		\node (b) at(-2, -0.5) {$b_0$};
		\draw (b) -- (-1,-0.5);
		
		\node at (-0.4, -1.55) {$...$};
		\node at (0.5, -1.55) {$...$};
		\node at (-2, 0) {$...$};

		\draw[dashed] (0, 1) -- (-1, -1);
		\draw[dashed] (0, 1) -- (1, -1);
	\end{circuitikz}
\end{center}

Chiamiamo $x$ la \textbf{variabile da commutare}, e le altre \textbf{variabili di comando} ($b$).
La $j$-esima uscita insegue la variabile da commutare se e solo se:
$$
(b_{n-1}, ..., b_0)_{2} = j
$$
altrimenti vale 0.
Questo significa che il demultiplexer invia il suo input, $x$, all'output $z_j$ tale che i controlli $b_{N-1} ... b_0$ sono la codifica binaria di $j$.

Il multiplexer, fisicamente, è identico ad un decoder con enabler: si fa la parte di decoding con il:
\[
	\begin{cases}
		z_0 = \overline{x_{N-1}} \cdot \overline{x_{N-2}} \cdot ... \cdot \overline{x_1} \cdot \overline{x_0}	\\
		z_1 = \overline{x_{N-1}} \cdot \overline{x_{N-2}} \cdot ... \cdot \overline{x_1} \cdot x_0	\\
		... \\ 
		z_{p-2} = x_{N-1} \cdot x_{N-2} \cdot ... \cdot x_1 \cdot \overline{x_0}	\\
		z_{p-1} = x_{N-1} \cdot x_{N-2} \cdot ... \cdot x_1 \cdot x_0	\\
	\end{cases}
\]
di prima, e si moltiplica per $x$ per ottenere il comportamento desiderato:
\[
	\begin{cases}
		z_0 = x \cdot \overline{x_{N-1}} \cdot \overline{x_{N-2}} \cdot ... \cdot \overline{x}_1 \cdot \overline{x_0}	\\
		z_1 = x \cdot \overline{x_{N-1}} \cdot \overline{x_{N-2}} \cdot ... \cdot \overline{x_1} \cdot x_0	\\
		... \\ 
		z_{p-2} = x \cdot x_{N-1} \cdot x_{N-2} \cdot ... \cdot x_1 \cdot \overline{x_0}	\\
		z_{p-1} = x \cdot x_{N-1} \cdot x_{N-2} \cdot ... \cdot x_1 \cdot x_0	\\
	\end{cases}
\]

Con $x = e$ questo è un decoder con enabler $x$.

Vediamo infatti l'implementazione in Verilog di un demultiplexer da 1 a 2:

\lstinputlisting[language=verilog, style=codestyle]{../verilog/10-09/demuxers/b1to2_demuxer.v}

e come se ne può ricavare uno da 1 a 4:

\lstinputlisting[language=verilog, style=codestyle]{../verilog/10-09/demuxers/b1to4_demuxer.v}

\subsection{Multiplexer}
Il multiplexer è il duale del demultiplexer: una rete con $N + 2^N$ ingressi e $1$ uscita:

\begin{center}
	\begin{circuitikz}
		\node[rectangle, minimum height=2cm, minimum width=2cm, draw] (multiplex) at (0,0) {};
		\node (x) at (0,-1.5) {$x$};	
		\draw (x) -- (0, -1);

		\node (zn) at (-1,1.5) {$z_{p-1}$};
		\node (z) at (1,1.5) {$z_0$};
		\node (j) at(0, 1.5) {$z_j$};

		\draw (zn) -- (-1, 1);
		\draw (z) -- (1, 1);
		\draw (j) -- (0, 1);

		\node (bn) at(-2, 0.5) {$b_{N-1}$};
		\draw (bn) -- (-1,0.5);

		\node (b) at(-2, -0.5) {$b_0$};
		\draw (b) -- (-1,-0.5);
		
		\node at (-0.4, 1.45) {$...$};
		\node at (0.5, 1.45) {$...$};
		\node at (-2, 0) {$...$};

		\draw[dashed] (-1, 1) -- (0, -1);
		\draw[dashed] (1, 1) -- (0, -1);
	\end{circuitikz}
\end{center}

Gli ingressi $b_i$ si chiamano variabili di comando, e selezionano l'ingresso connesso all'uscita come:
$$
z = x_i \Leftrightarrow (b_{N-1}, ..., b_1, b_0) = i
$$

Abbiamo detto che il multiplexer è il duale del demultiplexer: se quest'ultimo prendeva un segnale $x$ e lo inviava al $j$-esimo output sulla base della codifica di $j$ ottenuta alle variabili di controllo, il multiplexer prende il $j$-esimo ingresso, secondo gli stessi canoni, e lo invia alla linea $x$ di uscita.

Alla base della sintesi di un multiplexer sta un decoder: infatti, abbiamo che quest'ultimo seleziona uno solo (\textit{one-hot}) degli output, che possiamo moltiplicare (mettiamo una AND) per l'ingresso corrispondente.
Visto che solo uno degli output in uscita dagli AND è attivo in un dato momento, possiamo ricombinare il segnale finlle con un unico grande OR.

Come prima, possiamo eliminare gli AND in cascata dal decoder connettendoli agli AND già contenuti in esso.

Otteniamo quindi la descrizione algebrica (si noti che adesso abbiamo fatto sintesi $\rightarrow$ descrizione, mentre fino a questo punto avevamo fatto l'operazione inversa, descrizione $\rightarrow$ sintesi):

\[
	\begin{aligned}
		z = x_0 \cdot \overline{b_{N-1}} \cdot \overline{b_{N-2}} \cdot ... \cdot \overline{b_1} \cdot \overline{b_0}	+ \\
		x_1 \cdot \overline{b_{N-1}} \cdot \overline{b_{N-2}} \cdot ... \cdot \overline{b_1} \cdot b_0	+\\
		... + \\
		x_{p-2} \cdot b_{N-1} \cdot b_{N-2} \cdot ... \cdot b_1 \cdot \overline{b_0}	+\\
		x_{p-1} \cdot b_{N-1} \cdot b_{N-2} \cdot ... \cdot b_1 \cdot b_0	\\
	\end{aligned}
\]

Notiamo che il multiplexer è una rete a 2 livelli di logica: il segnale passerà al massimo da una AND e una OR.
Le NOT sugli ingressi non si contano, in quanto in una rete fisica le variabili di comando proverranno da registri, che forniscono già una versione negata del loro output senza bisogno di ulteriori inversori.

Vediamo quindi un'implementazione in Verilog di un multiplexer da 2 a 1:

\lstinputlisting[language=verilog, style=codestyle]{../verilog/10-09/muxers/b2to1_muxer.v}

e come se ne può ricavare uno da 4 a 1:

\lstinputlisting[language=verilog, style=codestyle]{../verilog/10-09/muxers/b4to1_muxer.v}

\subsubsection{Multiplexer come rete combinatoria universale}
Dimostriamo il seguente teorema:
\begin{theorem}{Multiplexer come rete combinatoria universale}	
Un multiplexer con $N$ variabili di comando è in grado di realizzare qualunque legge combinatoria ad $N$ ingressi ed un uscita, connettendo i $2^N$ ingressi a generatori di costante.
\end{theorem}

Abbiamo che:
\begin{itemize}
	\item Un multiplexer si ricava con porte AND, OR e NOT a due livelli di logica; 
	\item Un multiplexer realizza qualsiasi rete combinatoria ad un'uscita;
	\item una rete a più uscite può essere scomposta in più reti con le uscite messe "in parellelo".
\end{itemize}
Allora qualsiasi rete combinatoria può essere creata combinando AND, OR e NOT su due livelli di logica.

Inoltre, si può dimostrare che per qualsiasi tabella di verità ad $N$ ingressi, si può trovare una rete che la implementa tramite un multiplexer a $N-1$ variabili di comando, e al più porte NOT.

\subsection{Modello strutturale universale per reti combinatorie}
Vediamo adesso un modo per sintetizzare una rete logica ad $N$ ingressi ed $M$ uscite a partire da una tabella di verità.
Si prende prima di tutto un decoder con $N$ ingressi, e si creano $M$ linee parallele alle $2^N$ (che è anche il numero delle righe della tabella di verità) linee di uscita del decoder.
Si combinano quindi queste linee di uscita attraverso OR su ogni intersezione che corrisponde ad una certa cella della tabella di verità.

\subsubsection{Riduzione dei costi}
Definiamo informalmente il costo come ridotto quando si usano meno porte logiche.
Troviamo quindi un modo per ridurre il costo della rete creata.
Avremo che, inizialmente, tutte le uscite si presentano in una forma canonica \textbf{SP}, che sta per Somma di Prodotti, del tipo:
$$ 
z_j = x_{n-1} \cdot ... \cdot x_0 + ... + x_{n-1} \cdot ... \cdot x_0
$$
con la possibilità di complementare qualsiasi $x$. 
Questa forma equivale effettivamente a una forma normale disgiuntiva.

Possiamo quindi usare le proprietà dell'algebra di Boole per raggruppare e semplificare i termini.
Vogliamo un algoritmo che ci permetta di eseguire questi passaggi in modo ordinato, e ci porti sempre alla soluzione ottimale.

\end{document}


\documentclass[a4paper,11pt]{article}
\usepackage[a4paper, margin=8em]{geometry}

% usa i pacchetti per la scrittura in italiano
\usepackage[french,italian]{babel}
\usepackage[T1]{fontenc}
\usepackage[utf8]{inputenc}
\frenchspacing 

% usa i pacchetti per la formattazione matematica
\usepackage{amsmath, amssymb, amsthm, amsfonts}

% usa altri pacchetti
\usepackage{gensymb}
\usepackage{hyperref}
\usepackage{standalone}

\usepackage{colortbl}

\usepackage{xstring}
\usepackage{karnaugh-map}

% imposta il titolo
\title{Appunti Reti Logiche}
\author{Luca Seggiani}
\date{2024}

% imposta lo stile
% usa helvetica
\usepackage[scaled]{helvet}
% usa palatino
\usepackage{palatino}
% usa un font monospazio guardabile
\usepackage{lmodern}

\renewcommand{\rmdefault}{ppl}
\renewcommand{\sfdefault}{phv}
\renewcommand{\ttdefault}{lmtt}

% circuiti
\usepackage{circuitikz}
\usetikzlibrary{babel}

% disponi il titolo
\makeatletter
\renewcommand{\maketitle} {
	\begin{center} 
		\begin{minipage}[t]{.8\textwidth}
			\textsf{\huge\bfseries \@title} 
		\end{minipage}%
		\begin{minipage}[t]{.2\textwidth}
			\raggedleft \vspace{-1.65em}
			\textsf{\small \@author} \vfill
			\textsf{\small \@date}
		\end{minipage}
		\par
	\end{center}

	\thispagestyle{empty}
	\pagestyle{fancy}
}
\makeatother

% disponi teoremi
\usepackage{tcolorbox}
\newtcolorbox[auto counter, number within=section]{theorem}[2][]{%
	colback=blue!10, 
	colframe=blue!40!black, 
	sharp corners=northwest,
	fonttitle=\sffamily\bfseries, 
	title=Teorema~\thetcbcounter: #2, 
	#1
}

% disponi definizioni
\newtcolorbox[auto counter, number within=section]{definition}[2][]{%
	colback=red!10,
	colframe=red!40!black,
	sharp corners=northwest,
	fonttitle=\sffamily\bfseries,
	title=Definizione~\thetcbcounter: #2,
	#1
}

% disponi codice
\usepackage{listings}
\usepackage[table]{xcolor}

\definecolor{codegreen}{rgb}{0,0.6,0}
\definecolor{codegray}{rgb}{0.5,0.5,0.5}
\definecolor{codepurple}{rgb}{0.58,0,0.82}
\definecolor{backcolour}{rgb}{0.95,0.95,0.92}

\lstdefinestyle{codestyle}{
		backgroundcolor=\color{black!5}, 
		commentstyle=\color{codegreen},
		keywordstyle=\bfseries\color{magenta},
		numberstyle=\sffamily\tiny\color{black!60},
		stringstyle=\color{green!50!black},
		basicstyle=\ttfamily\footnotesize,
		breakatwhitespace=false,         
		breaklines=true,                 
		captionpos=b,                    
		keepspaces=true,                 
		numbers=left,                    
		numbersep=5pt,                  
		showspaces=false,                
		showstringspaces=false,
		showtabs=false,                  
		tabsize=2
}

\lstdefinestyle{shellstyle}{
		backgroundcolor=\color{black!5}, 
		basicstyle=\ttfamily\footnotesize\color{black}, 
		commentstyle=\color{black}, 
		keywordstyle=\color{black},
		numberstyle=\color{black!5},
		stringstyle=\color{black}, 
		showspaces=false,
		showstringspaces=false, 
		showtabs=false, 
		tabsize=2, 
		numbers=none, 
		breaklines=true
}


\lstdefinelanguage{assembler}{ 
  keywords={AAA, AAD, AAM, AAS, ADC, ADCB, ADCW, ADCL, ADD, ADDB, ADDW, ADDL, AND, ANDB, ANDW, ANDL,
        ARPL, BOUND, BSF, BSFL, BSFW, BSR, BSRL, BSRW, BSWAP, BT, BTC, BTCB, BTCW, BTCL, BTR, 
        BTRB, BTRW, BTRL, BTS, BTSB, BTSW, BTSL, CALL, CBW, CDQ, CLC, CLD, CLI, CLTS, CMC, CMP,
        CMPB, CMPW, CMPL, CMPS, CMPSB, CMPSD, CMPSW, CMPXCHG, CMPXCHGB, CMPXCHGW, CMPXCHGL,
        CMPXCHG8B, CPUID, CWDE, DAA, DAS, DEC, DECB, DECW, DECL, DIV, DIVB, DIVW, DIVL, ENTER,
        HLT, IDIV, IDIVB, IDIVW, IDIVL, IMUL, IMULB, IMULW, IMULL, IN, INB, INW, INL, INC, INCB,
        INCW, INCL, INS, INSB, INSD, INSW, INT, INT3, INTO, INVD, INVLPG, IRET, IRETD, JA, JAE,
        JB, JBE, JC, JCXZ, JE, JECXZ, JG, JGE, JL, JLE, JMP, JNA, JNAE, JNB, JNBE, JNC, JNE, JNG,
        JNGE, JNL, JNLE, JNO, JNP, JNS, JNZ, JO, JP, JPE, JPO, JS, JZ, LAHF, LAR, LCALL, LDS,
        LEA, LEAVE, LES, LFS, LGDT, LGS, LIDT, LMSW, LOCK, LODSB, LODSD, LODSW, LOOP, LOOPE,
        LOOPNE, LSL, LSS, LTR, MOV, MOVB, MOVW, MOVL, MOVSB, MOVSD, MOVSW, MOVSX, MOVSXB,
        MOVSXW, MOVSXL, MOVZX, MOVZXB, MOVZXW, MOVZXL, MUL, MULB, MULW, MULL, NEG, NEGB, NEGW,
        NEGL, NOP, NOT, NOTB, NOTW, NOTL, OR, ORB, ORW, ORL, OUT, OUTB, OUTW, OUTL, OUTSB, OUTSD,
        OUTSW, POP, POPL, POPW, POPB, POPA, POPAD, POPF, POPFD, PUSH, PUSHL, PUSHW, PUSHB, PUSHA, 
				PUSHAD, PUSHF, PUSHFD, RCL, RCLB, RCLW, MOVSL, MOVSB, MOVSW, STOSL, STOSB, STOSW, LODSB, LODSW,
				LODSL, INSB, INSW, INSL, OUTSB, OUTSL, OUTSW
        RCLL, RCR, RCRB, RCRW, RCRL, RDMSR, RDPMC, RDTSC, REP, REPE, REPNE, RET, ROL, ROLB, ROLW,
        ROLL, ROR, RORB, RORW, RORL, SAHF, SAL, SALB, SALW, SALL, SAR, SARB, SARW, SARL, SBB,
        SBBB, SBBW, SBBL, SCASB, SCASD, SCASW, SETA, SETAE, SETB, SETBE, SETC, SETE, SETG, SETGE,
        SETL, SETLE, SETNA, SETNAE, SETNB, SETNBE, SETNC, SETNE, SETNG, SETNGE, SETNL, SETNLE,
        SETNO, SETNP, SETNS, SETNZ, SETO, SETP, SETPE, SETPO, SETS, SETZ, SGDT, SHL, SHLB, SHLW,
        SHLL, SHLD, SHR, SHRB, SHRW, SHRL, SHRD, SIDT, SLDT, SMSW, STC, STD, STI, STOSB, STOSD,
        STOSW, STR, SUB, SUBB, SUBW, SUBL, TEST, TESTB, TESTW, TESTL, VERR, VERW, WAIT, WBINVD,
        XADD, XADDB, XADDW, XADDL, XCHG, XCHGB, XCHGW, XCHGL, XLAT, XLATB, XOR, XORB, XORW, XORL},
  keywordstyle=\color{blue}\bfseries,
  ndkeywordstyle=\color{darkgray}\bfseries,
  identifierstyle=\color{black},
  sensitive=false,
  comment=[l]{\#},
  morecomment=[s]{/*}{*/},
  commentstyle=\color{purple}\ttfamily,
  stringstyle=\color{red}\ttfamily,
  morestring=[b]',
  morestring=[b]"
}

\lstset{language=assembler, style=codestyle}

% disponi sezioni
\usepackage{titlesec}

\titleformat{\section}
	{\sffamily\Large\bfseries} 
	{\thesection}{1em}{} 
\titleformat{\subsection}
	{\sffamily\large\bfseries}   
	{\thesubsection}{1em}{} 
\titleformat{\subsubsection}
	{\sffamily\normalsize\bfseries} 
	{\thesubsubsection}{1em}{}

% tikz
\usepackage{tikz}

% float
\usepackage{float}

% grafici
\usepackage{pgfplots}
\pgfplotsset{width=10cm,compat=1.9}

% disponi alberi
\usepackage{forest}

\forestset{
	rectstyle/.style={
		for tree={rectangle,draw,font=\large\sffamily}
	},
	roundstyle/.style={
		for tree={circle,draw,font=\large}
	}
}

% disponi algoritmi
\usepackage{algorithm}
\usepackage{algorithmic}
\makeatletter
\renewcommand{\ALG@name}{Algoritmo}
\makeatother

% disponi numeri di pagina
\usepackage{fancyhdr}
\fancyhf{} 
\fancyfoot[L]{\sffamily{\thepage}}

\makeatletter
\fancyhead[L]{\raisebox{1ex}[0pt][0pt]{\sffamily{\@title \ \@date}}} 
\fancyhead[R]{\raisebox{1ex}[0pt][0pt]{\sffamily{\@author}}}
\makeatother

\begin{document}
% sezione (data)
\section{Lezione del 10-10-24}

% stili pagina
\thispagestyle{empty}
\pagestyle{fancy}

% testo
\subsection{Sintesi di reti in forma SP a costo minimo}
Esistono due criteri di costo per le reti:
\begin{itemize}
	\item \textbf{A porte:} ogni \textbf{porta} conta per un'unità di costo;
	\item \textbf{A diodi:} ogni \textbf{ingresso} conta per un'unità di costo.
\end{itemize}

Presentiamo un metoodo, applicabile a reti con'un uscita, che produce reti in forma SP a 2 livelli di logica in quanto, per una legge combinatoria F, si ha::
$$
\text{Sintesi di F a 2 L.L. in forma SP} \subset \text{Sintesi di F a 2 L.L.} \subset \text{Sintesi di F}
$$

\subsubsection{Espansione di Shannon}
Si può dimostrare il seguente risultato:
\begin{theorem}{Espansione di Shannon}
	Si può sempre scrivere qualunque legge combinatoria $f$ come somma di prodotti degli ingressi (diretti o negati).
\end{theorem}

Questo significa che, se ho una legge combinatoria $z = f(x_{N-1}, ..., x_0)$, posso dire:
\[
	\begin{aligned}
		z = f(0, ..., 0,  0) \cdot \overline{x_{N-1}} \cdot \overline{x_{N-2}} \cdot ... \cdot \overline{x_1} \cdot \overline{x_0}	+ \\
		f(0, ..., 0, 1) \cdot \overline{x_{N-1}} \cdot \overline{x_{N-2}} \cdot ... \cdot \overline{x_1} \cdot x_0	+\\
		... + \\
		f(1, ..., 1, 0) \cdot x_{N-1} \cdot x_{N-2} \cdot ... \cdot x_1 \cdot \overline{x_0}	+\\
		f(1, ..., 1, 1) \cdot x_{N-1} \cdot x_{N-2} \cdot ... \cdot x_1 \cdot x_0	\\
	\end{aligned}
\]
che equivale a quanto avevamo visto con la sintesi di reti combinatorie a $N$ ingressi con multiplexer a $N$ variabili di comando.

A questo punto possiamo ottenere la cosiddetta \textbf{forma canonica SP}, applicando le proprietà:
\[
	\begin{cases}
		1 \cdot \alpha = \alpha \\ 
		0 \cdot \alpha = 0 \\ 
		0 + \beta = \beta
	\end{cases}
\]
all'espansione di Shannon (sostanzialmente rimuoviamo tutti i termini a cui corrispondono uscite negate).
Della forma canonica SP possiamo dire che è:
\begin{itemize}
	\item \textbf{SP:} è fatta da somme e prodotti;
	\item \textbf{Canonica:} ogni prodotto ha come fattori tutti gli ingressi, diretti o negati;
	\item Ciascuno dei termini della somma si chiama \textbf{mintermine};
	\item Ogni mintermine corrisponde ad uno stato riconosciuto dalla rete.
\end{itemize}

L'insieme dei termini (mintermini) sommati fra di loro che otteniamo dall'espansione di Shannon prende il nome di \textbf{lista di mintermini}.

\subsection{Semplificazione della forma canonica SP}
Definiamo quindi un metodo per la semplificazione della lista dei mintermini.
Divideremo quest'operazione in due passaggi principali:

\begin{itemize}
	\item \textbf{Identificazione degli implicanti principali}: si ricava una lista di termini ricavati da quelli di partenza, e di dimensioni più piccole, che rappresentano la stessa legge combinatoria;
	\item \textbf{Eliminazione delle ridondanze}: si rimuovono gli implicanti che non portano informazioni utili alla legge combinatoria.
\end{itemize}

\subsubsection{Metodo di Quine-McCluskey}
Si presenta il metodo di Quine-McCluskey per l'identificazione degli implicanti principali.
Questo metodo prevede di:
\begin{itemize}
	\item \textbf{Fondere i mintermini} applicando \textbf{esaustivamente} la regola:
$$
\alpha x + \alpha \bar{x} = \alpha
$$
che possiamo dimostrare come:
$$
\alpha x + \alpha \bar{x} = \alpha (x + \bar{x}) = \alpha, \quad x + \bar{x} = 1
$$
alla lista dei mintermini.

Ripetiamo questo passaggio $N - 1$ volte per la dimensione $N$ dei termini, riducendo ogni volta la dimensione degli implicanti di 1.
Si ricava una forma SP, detta \textbf{lista di implicanti}.
	\item \textbf{Rimuovere i duplicati} dalla lista dei duplicanti, applicando l'altra regola:
$$
\alpha x + \alpha = \alpha 
$$
sugli implicanti che hanno elementi in comune.
\end{itemize}

Troviamo quindi quella che è detta \textbf{lista degli implicanti principali}.
Questa lista contiene meno elementi della forma canonica SP, non è ancora di costo minimo: potrebbe contenere ridondanze, cioè implicanti non necessari alla corretta modelizzazione della legge combinatoria.

Vediamo un modo per eliminare queste ridondanze.

\subsubsection{Liste di copertura ridondanti}
Una \textbf{lista di copertura} è una lista di implicanti, la cui somma è una forma SP per la funzione $f$.
La \textbf{lista di copertura non ridondante} è la lista che smette di essere una lista di copertura appena si toglie un elemento.

La lista dei mintermini è una lista non ridondante, mentre la lista degli implicanti principali può esserlo.

Si introduce quindi uno strumento per la visualizzazione di ridondanze.

\subsubsection{Mappe di Karnaugh}
Per una rete a $N$ ingressi la mappa di Karnaugh è una matrice di $2^N$ celle, dove le coordinate rappresentano gli ingressi, e gli elementi della matrice le uscite.
Sono diagrammi che tornano utili per rappresentare graficamente gli implicanti, ed eliminarne le ridonanze.
Vediamo, ad esempio, mappe con $N = 2$, $3$ e $4$:

\begin{center}
\begin{karnaugh-map}[2][2][1]
		\minterms{1,2}
		\maxterms{0,3}
\end{karnaugh-map}
\end{center}

\noindent
\begin{minipage}{0.45\textwidth}
	\begin{karnaugh-map}[4][2][1][$X_1X_0$][$X_2$]
			\minterms{3,4}
			\maxterms{0,1,6,7}
			\indeterminants{2,5}
	\end{karnaugh-map}
\end{minipage}%
\hfill
\begin{minipage}{0.45\textwidth}
\begin{karnaugh-map}
		\manualterms{0,0,0,0,0,0,0,0,0,0,0,0,0,0,0,0}
\end{karnaugh-map}
\end{minipage}


In una mappa di Karnaugh, celle \textbf{contigue} hanno coordinate \textbf{adiacenti}, e viceversa.
Oltre le 4 coordinate, per le mappe non possiamo più rappresentare queste mappe senza la terza dimensione.

Definiamo:
\begin{itemize}
	\item \textbf{Sottocubo di ordine 1:} una casella che contiene un 1, corrispondente quindi ad uno stato di ingresso riconosciuto dalla rete, si indica come SO1;
	\item \textbf{Coordinate} di un SO1: stato di ingresso corrispondente al sottocubo;
	\item \textbf{Adiacenza} fra SO1: due SO1 sono adiacenti se differiscono fra loro di una sola coordinata.
\end{itemize}

Vediamo, ad esempio, una mappa di Karnaugh con $N=2$, una serie di sottocubi di ordine 1 con la tabella associata:

\begin{center}
\noindent
\begin{minipage}{0.15\textwidth}
\begin{karnaugh-map}[2][2][1]
		\minterms{1,2}
		\maxterms{0,3}
		\implicant{1}{1}
		\implicant{2}{2}
\end{karnaugh-map}
\end{minipage}%
\hspace{3cm}
\begin{minipage}{0.15\textwidth}
	\begin{table}[H]
		\center \rowcolors{2}{white}{black!10}
		\begin{tabular} { c || c | c }
			& $x_1$ & $x_0$ \\ 
			\hline 
			\rowcolor{red!20!white} A & 0 & 1 \\
			\rowcolor{green!20!white} B & 1 & 0 \\
		\end{tabular}
	\end{table}
\end{minipage}
\end{center}

Notiamo come A corrisponde all'implicante $\overline{x_1}x_0$, e B all'implicante $x_1\overline{x_0}$.

Possiamo continuare:
\begin{itemize}
	\item \textbf{Sottocubo di ordine 2:} costituito da SO1 adiacenti, e si dice che \textbf{copre} i SO1 che lo formano. Si indica come SO2;
	\item \textbf{Sottocubo di ordine 4:} costituito da SO2 adiacenti, e si dice che \textbf{copre} i SO2 che lo formano. Si indica come SO4;
	\item \textbf{Sottocubo di ordine 8:} costituito da SO4 adiacenti, e si dice che \textbf{copre} i SO4 che lo formano. Si indica come SO8;
\end{itemize}

Vediamo un'ultimo esempio, con $N=4$: 

\begin{center}
\noindent
\begin{minipage}{0.3\textwidth}
\begin{karnaugh-map}
		\manualterms{0,0,0,0,0,0,0,0,0,0,0,0,0,0,0,0}
		\implicant{0}{1}
		\implicant{7}{14}
		\implicant{2}{10}
		\implicantcorner
\end{karnaugh-map}
\end{minipage}%
\hspace{3cm}
\begin{minipage}{0.3\textwidth}
	\begin{table}[H]
		\center \rowcolors{2}{white}{black!10}
		\begin{tabular} { c || c | c | c | c}
			& $x_3$ & $x_2$ & $x_1$ & $x_0$ \\ 
			\hline 
			\rowcolor{red!20!white} A & 0 & - & - & 1 \\
			\rowcolor{green!20!white} B & 1 & - & - & 1 \\
			\rowcolor{yellow!20!white} C & 1 & 0 & - & - \\
			\rowcolor{cyan!20!white} D & - & 0 & - & 0 \\
		\end{tabular}
	\end{table}
\end{minipage}
\end{center}

Notiamo dall'esempio che le mappe di Karnaugh rispettano il cosiddetti \textit{effetto pacman}: lo stesso implicante può esistere su lati opposti della mappa.
Il bisogno di rappresentare le adiacenze dà origine a questa particolarità, come determina l'ordine particolare delle attivazioni degli ingressi.
Inoltre, notiamo come i trattini nelle tabelle delle coordinate denotano che la variabile non influenza l'implicante, cioè rappresentano, in inglese, un \textit{don't care}.

\subsubsection{Ricerca delle liste di copertura non ridondanti}

Si dice che un sottocubo è \textbf{principale} quando non esiste nessun sottocubo più grande che lo copre completamente.
Si ha quindi che sottocubi e implicanti sono correlati: un sottocubo principale di ordine $p$ rappresenta un implicante principale di $N - \log_2(p)$ variabili.

Una \textbf{lista di copertura} è l'insieme (qualunque) di sottocubi che coprono tutti i SO1.
Una \textbf{lista di copertura non ridondante} è una lista di copertura che smette di essere tale quando si toglie un sottocubo.

Si presenta finalmente l'algoritmo:
\begin{algorithm}
\caption{per la ricerca dei sottocubi principali}
\begin{algorithmic}
	\STATE \textbf{Input:} i sottocubi di ordine più grande trovati sulla mappa 
	\STATE \textbf{Output:} i sottocubi principali della mappa
	\STATE \textsf{ciclo:}
	\STATE Considera tutti i sottocubi di ordine $p$ non interamente contenuti in sottocubi di ordine più grande, e
	segnali tutti: questi sono sicuramente principali
	\IF{l'insieme trovato finora basta a coprire tutta a mappa}
		\STATE L'algoritmo è terminato 
	\ELSE 
		\STATE Poni $p \leftarrow \frac{p}{2}$ e vai a \textsf{ciclo}
	\ENDIF 
\end{algorithmic}
\end{algorithm}

Ad esempio, nello scorso esempio, l'algoritmo rimuoverebbe l'implicante C.

\par\smallskip

Notiamo che alcuni sottocubi sono gli unici a coprire un dato sottocubo di ordine 1. In questo caso, si chiamano sottocubi \textbf{essenziali}, e costituiscono il \textbf{cuore} (\textit{core}) della mappa.

\end{document}



\documentclass[a4paper,11pt]{article}
\usepackage[a4paper, margin=8em]{geometry}

% usa i pacchetti per la scrittura in italiano
\usepackage[french,italian]{babel}
\usepackage[T1]{fontenc}
\usepackage[utf8]{inputenc}
\frenchspacing 

% usa i pacchetti per la formattazione matematica
\usepackage{amsmath, amssymb, amsthm, amsfonts}

% usa altri pacchetti
\usepackage{gensymb}
\usepackage{hyperref}
\usepackage{standalone}

\usepackage{colortbl}

\usepackage{xstring}
\usepackage{karnaugh-map}

% imposta il titolo
\title{Appunti Reti Logiche}
\author{Luca Seggiani}
\date{2024}

% imposta lo stile
% usa helvetica
\usepackage[scaled]{helvet}
% usa palatino
\usepackage{palatino}
% usa un font monospazio guardabile
\usepackage{lmodern}

\renewcommand{\rmdefault}{ppl}
\renewcommand{\sfdefault}{phv}
\renewcommand{\ttdefault}{lmtt}

% circuiti
\usepackage{circuitikz}
\usetikzlibrary{babel}

% disponi il titolo
\makeatletter
\renewcommand{\maketitle} {
	\begin{center} 
		\begin{minipage}[t]{.8\textwidth}
			\textsf{\huge\bfseries \@title} 
		\end{minipage}%
		\begin{minipage}[t]{.2\textwidth}
			\raggedleft \vspace{-1.65em}
			\textsf{\small \@author} \vfill
			\textsf{\small \@date}
		\end{minipage}
		\par
	\end{center}

	\thispagestyle{empty}
	\pagestyle{fancy}
}
\makeatother

% disponi teoremi
\usepackage{tcolorbox}
\newtcolorbox[auto counter, number within=section]{theorem}[2][]{%
	colback=blue!10, 
	colframe=blue!40!black, 
	sharp corners=northwest,
	fonttitle=\sffamily\bfseries, 
	title=Teorema~\thetcbcounter: #2, 
	#1
}

% disponi definizioni
\newtcolorbox[auto counter, number within=section]{definition}[2][]{%
	colback=red!10,
	colframe=red!40!black,
	sharp corners=northwest,
	fonttitle=\sffamily\bfseries,
	title=Definizione~\thetcbcounter: #2,
	#1
}

% disponi codice
\usepackage{listings}
\usepackage[table]{xcolor}

\definecolor{codegreen}{rgb}{0,0.6,0}
\definecolor{codegray}{rgb}{0.5,0.5,0.5}
\definecolor{codepurple}{rgb}{0.58,0,0.82}
\definecolor{backcolour}{rgb}{0.95,0.95,0.92}

\lstdefinestyle{codestyle}{
		backgroundcolor=\color{black!5}, 
		commentstyle=\color{codegreen},
		keywordstyle=\bfseries\color{magenta},
		numberstyle=\sffamily\tiny\color{black!60},
		stringstyle=\color{green!50!black},
		basicstyle=\ttfamily\footnotesize,
		breakatwhitespace=false,         
		breaklines=true,                 
		captionpos=b,                    
		keepspaces=true,                 
		numbers=left,                    
		numbersep=5pt,                  
		showspaces=false,                
		showstringspaces=false,
		showtabs=false,                  
		tabsize=2
}

\lstdefinestyle{shellstyle}{
		backgroundcolor=\color{black!5}, 
		basicstyle=\ttfamily\footnotesize\color{black}, 
		commentstyle=\color{black}, 
		keywordstyle=\color{black},
		numberstyle=\color{black!5},
		stringstyle=\color{black}, 
		showspaces=false,
		showstringspaces=false, 
		showtabs=false, 
		tabsize=2, 
		numbers=none, 
		breaklines=true
}


\lstdefinelanguage{assembler}{ 
  keywords={AAA, AAD, AAM, AAS, ADC, ADCB, ADCW, ADCL, ADD, ADDB, ADDW, ADDL, AND, ANDB, ANDW, ANDL,
        ARPL, BOUND, BSF, BSFL, BSFW, BSR, BSRL, BSRW, BSWAP, BT, BTC, BTCB, BTCW, BTCL, BTR, 
        BTRB, BTRW, BTRL, BTS, BTSB, BTSW, BTSL, CALL, CBW, CDQ, CLC, CLD, CLI, CLTS, CMC, CMP,
        CMPB, CMPW, CMPL, CMPS, CMPSB, CMPSD, CMPSW, CMPXCHG, CMPXCHGB, CMPXCHGW, CMPXCHGL,
        CMPXCHG8B, CPUID, CWDE, DAA, DAS, DEC, DECB, DECW, DECL, DIV, DIVB, DIVW, DIVL, ENTER,
        HLT, IDIV, IDIVB, IDIVW, IDIVL, IMUL, IMULB, IMULW, IMULL, IN, INB, INW, INL, INC, INCB,
        INCW, INCL, INS, INSB, INSD, INSW, INT, INT3, INTO, INVD, INVLPG, IRET, IRETD, JA, JAE,
        JB, JBE, JC, JCXZ, JE, JECXZ, JG, JGE, JL, JLE, JMP, JNA, JNAE, JNB, JNBE, JNC, JNE, JNG,
        JNGE, JNL, JNLE, JNO, JNP, JNS, JNZ, JO, JP, JPE, JPO, JS, JZ, LAHF, LAR, LCALL, LDS,
        LEA, LEAVE, LES, LFS, LGDT, LGS, LIDT, LMSW, LOCK, LODSB, LODSD, LODSW, LOOP, LOOPE,
        LOOPNE, LSL, LSS, LTR, MOV, MOVB, MOVW, MOVL, MOVSB, MOVSD, MOVSW, MOVSX, MOVSXB,
        MOVSXW, MOVSXL, MOVZX, MOVZXB, MOVZXW, MOVZXL, MUL, MULB, MULW, MULL, NEG, NEGB, NEGW,
        NEGL, NOP, NOT, NOTB, NOTW, NOTL, OR, ORB, ORW, ORL, OUT, OUTB, OUTW, OUTL, OUTSB, OUTSD,
        OUTSW, POP, POPL, POPW, POPB, POPA, POPAD, POPF, POPFD, PUSH, PUSHL, PUSHW, PUSHB, PUSHA, 
				PUSHAD, PUSHF, PUSHFD, RCL, RCLB, RCLW, MOVSL, MOVSB, MOVSW, STOSL, STOSB, STOSW, LODSB, LODSW,
				LODSL, INSB, INSW, INSL, OUTSB, OUTSL, OUTSW
        RCLL, RCR, RCRB, RCRW, RCRL, RDMSR, RDPMC, RDTSC, REP, REPE, REPNE, RET, ROL, ROLB, ROLW,
        ROLL, ROR, RORB, RORW, RORL, SAHF, SAL, SALB, SALW, SALL, SAR, SARB, SARW, SARL, SBB,
        SBBB, SBBW, SBBL, SCASB, SCASD, SCASW, SETA, SETAE, SETB, SETBE, SETC, SETE, SETG, SETGE,
        SETL, SETLE, SETNA, SETNAE, SETNB, SETNBE, SETNC, SETNE, SETNG, SETNGE, SETNL, SETNLE,
        SETNO, SETNP, SETNS, SETNZ, SETO, SETP, SETPE, SETPO, SETS, SETZ, SGDT, SHL, SHLB, SHLW,
        SHLL, SHLD, SHR, SHRB, SHRW, SHRL, SHRD, SIDT, SLDT, SMSW, STC, STD, STI, STOSB, STOSD,
        STOSW, STR, SUB, SUBB, SUBW, SUBL, TEST, TESTB, TESTW, TESTL, VERR, VERW, WAIT, WBINVD,
        XADD, XADDB, XADDW, XADDL, XCHG, XCHGB, XCHGW, XCHGL, XLAT, XLATB, XOR, XORB, XORW, XORL},
  keywordstyle=\color{blue}\bfseries,
  ndkeywordstyle=\color{darkgray}\bfseries,
  identifierstyle=\color{black},
  sensitive=false,
  comment=[l]{\#},
  morecomment=[s]{/*}{*/},
  commentstyle=\color{purple}\ttfamily,
  stringstyle=\color{red}\ttfamily,
  morestring=[b]',
  morestring=[b]"
}

\lstset{language=assembler, style=codestyle}

% disponi sezioni
\usepackage{titlesec}

\titleformat{\section}
	{\sffamily\Large\bfseries} 
	{\thesection}{1em}{} 
\titleformat{\subsection}
	{\sffamily\large\bfseries}   
	{\thesubsection}{1em}{} 
\titleformat{\subsubsection}
	{\sffamily\normalsize\bfseries} 
	{\thesubsubsection}{1em}{}

% tikz
\usepackage{tikz}

% float
\usepackage{float}

% grafici
\usepackage{pgfplots}
\pgfplotsset{width=10cm,compat=1.9}

% disponi alberi
\usepackage{forest}

\forestset{
	rectstyle/.style={
		for tree={rectangle,draw,font=\large\sffamily}
	},
	roundstyle/.style={
		for tree={circle,draw,font=\large}
	}
}

% disponi algoritmi
\usepackage{algorithm}
\usepackage{algorithmic}
\makeatletter
\renewcommand{\ALG@name}{Algoritmo}
\makeatother

% disponi numeri di pagina
\usepackage{fancyhdr}
\fancyhf{} 
\fancyfoot[L]{\sffamily{\thepage}}

\makeatletter
\fancyhead[L]{\raisebox{1ex}[0pt][0pt]{\sffamily{\@title \ \@date}}} 
\fancyhead[R]{\raisebox{1ex}[0pt][0pt]{\sffamily{\@author}}}
\makeatother

\begin{document}
% sezione (data)
\section{Lezione del 11-10-24}

% stili pagina
\thispagestyle{empty}
\pagestyle{fancy}

% testo
\subsection{Sintesi di leggi non completamente specificate}
Applichiamo quanto abbiamo detto sulla sintesi di reti in forma SP a costo minimo, nel caso particolare in cui la legge non è completamente specificata (\textit{don't care}).

Prendiamo in esempio un decodificatore BCD a 7 segmenti, simile a quello che si potrebbe trovare ad accendere le tracce di un display a cristalli liquidi.

\begin{center}
	\begin{circuitikz}

		\node at (-3.3, 1.5) {$x_3$};
		\draw (-3, 1.5) -> (-1.3, 1.5);
		\node at (-3.3, 0.5) {$x_2$};
		\draw (-3, 0.5) -> (-1.3, 0.5);
		\node at (-3.3, -0.5) {$x_1$};
		\draw (-3, -0.5) -> (-1.3, -0.5);
		\node at (-3.3, -1.5) {$x_0$};
		\draw (-3, -1.5) -> (-1.3, -1.5);

		\node at (3.3, 1.5) {$z_G$};
		\draw (1.3, 1.5) -> (3, 1.5);
		\node at (3.3, 1) {$z_F$};
		\draw (1.3, 1) -> (3, 1);
		\node at (3.3, 0.5) {$z_E$};
		\draw (1.3, 0.5) -> (3, 0.5);
		\node at (3.3, 0) {$z_D$};
		\draw (1.3, 0) -> (3, 0);
		\node at (3.3, -0.5) {$z_C$};
		\draw (1.3, -0.5) -> (3, -0.5);
		\node at (3.3, -1) {$z_B$};
		\draw (1.3, -1) -> (3, -1);
		\node at (3.3, -1.5) {$z_A$};
		\draw (1.3, -1.5) -> (3, -1.5);

		\node[draw, rectangle, minimum width=2cm, minimum height=4cm] (decoder) at (0, 0) {BCD Decoder};
	\end{circuitikz}
\end{center}

Questo componente ha 4 variabili di ingresso, interpretate come una cifra in base 2 $j$, e 7 uscite, che specificano quali tracce accendere per visualizzare la cifra ottenuta in base 10.
Si ha che con 4 variabili di ingresso si indirizzano 16 possibili configurazioni dei segmenti, quando ne abbiamo bisogno solo 10 (una per ogni cifra decimale).
Le configurazioni di ingresso scartate si dicono quindi non specificate.

La tabella di verità della rete sarà quindi:

\begin{table}[H]
	\center 
	\begin{tabular} { c | c c c c | c c c c c c c c }
		$j$ & $x_3$ & $x_2$ & $x_1$ & $x_0$ & $z_G$ & $z_F$ & $z_E$ & $z_D$ & $z_C$ & $z_B$ & $z_A$ & \bfseries Display \\
		\hline											%	
		$0$ & $0$ & $0$ & $0$ & $0$ & $0$ & $1$ & $1$ & $1$ & $1$ & $1$ & $1$ &		
\begin{tabular}{l}
\begin{circuitikz}
\draw (0,0) node[seven segment bits=1111110 dot off box off](segment){}; %give your node a name
\node [font=\tiny] at (segment.a) {A}; %you can then refer to them as this
\node [font=\tiny] at (segment.b) {B};
\node [font=\tiny] at (segment.c) {C};
\node [font=\tiny] at (segment.d) {D};
\node [font=\tiny] at (segment.e) {E};
\node [font=\tiny] at (segment.f) {F};
\node [font=\tiny] at (segment.g) {G};
\end{circuitikz}
\end{tabular}
		\\
		$1$ & $0$ & $0$ & $0$ & $1$ & $0$ & $0$ & $0$ & $0$ & $1$ & $1$ & $0$ &
\begin{tabular}{l}
\begin{circuitikz}
\draw (0,0) node[seven segment bits=0110000 dot off box off](segment){}; %give your node a name
\node [font=\tiny] at (segment.a) {A}; %you can then refer to them as this
\node [font=\tiny] at (segment.b) {B};
\node [font=\tiny] at (segment.c) {C};
\node [font=\tiny] at (segment.d) {D};
\node [font=\tiny] at (segment.e) {E};
\node [font=\tiny] at (segment.f) {F};
\node [font=\tiny] at (segment.g) {G};
\end{circuitikz}
\end{tabular}
		\\
		$2$ & $0$ & $0$ & $1$ & $0$ & $1$ & $0$ & $1$ & $1$ & $0$ & $1$ & $1$ &
\begin{tabular}{l}
\begin{circuitikz}
\draw (0,0) node[seven segment bits=1101101 dot off box off](segment){}; %give your node a name
\node [font=\tiny] at (segment.a) {A}; %you can then refer to them as this
\node [font=\tiny] at (segment.b) {B};
\node [font=\tiny] at (segment.c) {C};
\node [font=\tiny] at (segment.d) {D};
\node [font=\tiny] at (segment.e) {E};
\node [font=\tiny] at (segment.f) {F};
\node [font=\tiny] at (segment.g) {G};
\end{circuitikz}
\end{tabular}
		\\
		$3$ & $0$ & $0$ & $1$ & $1$ & $1$ & $0$ & $0$ & $1$ & $1$ & $1$ & $1$ &
\begin{tabular}{l}
\begin{circuitikz}
\draw (0,0) node[seven segment bits=1111001 dot off box off](segment){}; %give your node a name
\node [font=\tiny] at (segment.a) {A}; %you can then refer to them as this
\node [font=\tiny] at (segment.b) {B};
\node [font=\tiny] at (segment.c) {C};
\node [font=\tiny] at (segment.d) {D};
\node [font=\tiny] at (segment.e) {E};
\node [font=\tiny] at (segment.f) {F};
\node [font=\tiny] at (segment.g) {G};
\end{circuitikz}
\end{tabular}
		\\

		$4$ & $0$ & $1$ & $0$ & $0$ & $1$ & $1$ & $0$ & $0$ & $1$ & $1$ & $0$ &
\begin{tabular}{l}
\begin{circuitikz}
\draw (0,0) node[seven segment bits=0110011 dot off box off](segment){}; %give your node a name
\node [font=\tiny] at (segment.a) {A}; %you can then refer to them as this
\node [font=\tiny] at (segment.b) {B};
\node [font=\tiny] at (segment.c) {C};
\node [font=\tiny] at (segment.d) {D};
\node [font=\tiny] at (segment.e) {E};
\node [font=\tiny] at (segment.f) {F};
\node [font=\tiny] at (segment.g) {G};
\end{circuitikz}
\end{tabular}
		\\
		$5$ & $0$ & $1$ & $0$ & $1$ & $1$ & $1$ & $0$ & $1$ & $1$ & $0$ & $1$ &
\begin{tabular}{l}
\begin{circuitikz}
\draw (0,0) node[seven segment bits=1011011 dot off box off](segment){}; %give your node a name
\node [font=\tiny] at (segment.a) {A}; %you can then refer to them as this
\node [font=\tiny] at (segment.b) {B};
\node [font=\tiny] at (segment.c) {C};
\node [font=\tiny] at (segment.d) {D};
\node [font=\tiny] at (segment.e) {E};
\node [font=\tiny] at (segment.f) {F};
\node [font=\tiny] at (segment.g) {G};
\end{circuitikz}
\end{tabular}
		\\
		$6$ & $0$ & $1$ & $1$ & $0$ & $1$ & $1$ & $1$ & $1$ & $1$ & $0$ & $1$ &
\begin{tabular}{l}
\begin{circuitikz}
\draw (0,0) node[seven segment bits=1011111 dot off box off](segment){}; %give your node a name
\node [font=\tiny] at (segment.a) {A}; %you can then refer to them as this
\node [font=\tiny] at (segment.b) {B};
\node [font=\tiny] at (segment.c) {C};
\node [font=\tiny] at (segment.d) {D};
\node [font=\tiny] at (segment.e) {E};
\node [font=\tiny] at (segment.f) {F};
\node [font=\tiny] at (segment.g) {G};
\end{circuitikz}
\end{tabular}
		\\
		$7$ & $0$ & $1$ & $1$ & $1$ & $0$ & $0$ & $0$ & $0$ & $1$ & $1$ & $1$ &
\begin{tabular}{l}
\begin{circuitikz}
\draw (0,0) node[seven segment bits=1110000 dot off box off](segment){}; %give your node a name
\node [font=\tiny] at (segment.a) {A}; %you can then refer to them as this
\node [font=\tiny] at (segment.b) {B};
\node [font=\tiny] at (segment.c) {C};
\node [font=\tiny] at (segment.d) {D};
\node [font=\tiny] at (segment.e) {E};
\node [font=\tiny] at (segment.f) {F};
\node [font=\tiny] at (segment.g) {G};
\end{circuitikz}
\end{tabular}
		\\
	
		$8$ & $1$ & $0$ & $0$ & $0$ & $1$ & $1$ & $1$ & $1$ & $1$ & $1$ & $1$ &
\begin{tabular}{l}
\begin{circuitikz}
\draw (0,0) node[seven segment bits=1111111 dot off box off](segment){}; %give your node a name
\node [font=\tiny] at (segment.a) {A}; %you can then refer to them as this
\node [font=\tiny] at (segment.b) {B};
\node [font=\tiny] at (segment.c) {C};
\node [font=\tiny] at (segment.d) {D};
\node [font=\tiny] at (segment.e) {E};
\node [font=\tiny] at (segment.f) {F};
\node [font=\tiny] at (segment.g) {G};
\end{circuitikz}
\end{tabular}
		\\
		$9$ & $1$ & $0$ & $0$ & $1$ & $1$ & $1$ & $0$ & $1$ & $1$ & $1$ & $1$ &
\begin{tabular}{l}
\begin{circuitikz}
\draw (0,0) node[seven segment bits=1111011 dot off box off](segment){}; %give your node a name
\node [font=\tiny] at (segment.a) {A}; %you can then refer to them as this
\node [font=\tiny] at (segment.b) {B};
\node [font=\tiny] at (segment.c) {C};
\node [font=\tiny] at (segment.d) {D};
\node [font=\tiny] at (segment.e) {E};
\node [font=\tiny] at (segment.f) {F};
\node [font=\tiny] at (segment.g) {G};
\end{circuitikz}
\end{tabular}
		\\
		$10$ & $1$ & $0$ & $1$ & $0$ & $-$ & $-$ & $-$ & $-$ & $-$ & $-$ & $-$ \\
		$11$ & $1$ & $0$ & $1$ & $1$ & $-$ & $-$ & $-$ & $-$ & $-$ & $-$ & $-$ \\

		$12$ & $1$ & $1$ & $0$ & $0$ & $-$ & $-$ & $-$ & $-$ & $-$ & $-$ & $-$ \\
		$13$ & $1$ & $1$ & $0$ & $1$ & $-$ & $-$ & $-$ & $-$ & $-$ & $-$ & $-$ \\
		$14$ & $1$ & $1$ & $1$ & $0$ & $-$ & $-$ & $-$ & $-$ & $-$ & $-$ & $-$ \\
		$15$ & $1$ & $1$ & $1$ & $1$ & $-$ & $-$ & $-$ & $-$ & $-$ & $-$ & $-$ \\
	\end{tabular}
\end{table}

Visto che vogliamo sintetizzare reti su uscite singole, prendiamo la tabella di verità della rete sull'uscita $z_E$ (le altre uscite richiederanno procedimenti simili):

\begin{table}[H]
	\center 
	\begin{tabular} { c | c c c c | c }
		$j$ & $x_3$ & $x_2$ & $x_1$ & $x_0$ & $z_E$ \\
		\hline
		$0$ & $0$ & $0$ & $0$ & $0$ & $1$ \\
		$1$ & $0$ & $0$ & $0$ & $1$ & $0$ \\
		$2$ & $0$ & $0$ & $1$ & $0$ & $1$ \\
		$3$ & $0$ & $0$ & $1$ & $1$ & $0$ \\

		$4$ & $0$ & $1$ & $0$ & $0$ & $0$ \\
		$5$ & $0$ & $1$ & $0$ & $1$ & $0$ \\
		$6$ & $0$ & $1$ & $1$ & $0$ & $1$ \\
		$7$ & $0$ & $1$ & $1$ & $1$ & $0$ \\
	
		$8$ & $1$ & $0$ & $0$ & $0$ & $1$ \\
		$9$ & $1$ & $0$ & $0$ & $1$ & $0$ \\
		$10$ & $1$ & $0$ & $1$ & $0$ & $-$ \\
		$11$ & $1$ & $0$ & $1$ & $1$ & $-$ \\

		$12$ & $1$ & $1$ & $0$ & $0$ & $-$ \\
		$13$ & $1$ & $1$ & $0$ & $1$ & $-$ \\
		$14$ & $1$ & $1$ & $1$ & $0$ & $-$ \\
		$15$ & $1$ & $1$ & $1$ & $1$ & $-$ \\
	\end{tabular}
\end{table}

Disegnamo quindi la mappa di Karnaugh.
Quando si disegnano mappe di Karnaugh con elementi indeterminati, questi si interpretano diversamente a seconda che si stiano cercando i sottocubi principali, o che si stiano classificando:
\begin{itemize}
	\item \textbf{Ricerca dei sottocubi principali:} si prendono come 1. Questo ci permette di prendere i sottocubi più grandi possibili nella ricerca dei primali (è irrilevante se si vanno a impostare uscite non specificate a 1). \par\smallskip
		Si trova quindi:
\begin{center}
\noindent
\begin{minipage}{0.3\textwidth}
\begin{karnaugh-map}
		\manualterms{1, 0, 1, 0, 0, 0, 1, 0, 1, 0, -, -, -, -, -, -}
		\implicant{15}{10}
		\implicant{2}{10}
		\implicant{12}{14}
		\implicantcorner
		\implicantedge{12}{8}{14}{10}
\end{karnaugh-map}
\end{minipage}%
\hspace{3cm}
\begin{minipage}{0.3\textwidth}
	\begin{table}[H]
		\center \rowcolors{2}{white}{black!10}
		\begin{tabular} { c || c | c | c | c}
			& $x_3$ & $x_2$ & $x_1$ & $x_0$ \\ 
			\hline 
			\rowcolor{red!20!white} A & 1 & - & 1 & - \\
			\rowcolor{green!20!white} B & - & - & 1 & 0 \\
			\rowcolor{yellow!20!white} C & 1 & 1 & - & - \\
			\rowcolor{cyan!20!white} D & - & 0 & - & 0 \\
			\rowcolor{blue!20!white} E & 1 & - & - & 0 \\
		\end{tabular}
	\end{table}
\end{minipage}
\end{center}

	\item \textbf{Classificazione dei sottocubi principali:} si prendono come 0. Così si evita di conservare implicanti che siano rilevanti su uscite non specificate (sarebbe inutile). \par\smallskip
		Si ha quindi che i sottocubi A e C prendono solo SO1 indeterminati, ergo si scartano.
		Restano B e D essenziali, ed E a questo punto eliminabile, in quanto è già compreso in questi.
\end{itemize}

La sintesi completa dela rete è allora:
$$ z_E = x_1 \overline{x_0} + \overline{x_2}\overline{x_0} $$

\subsection{Sintesi in forma PS}
Abbiamo usato finora la forma SP (somma di prodotti).
Esiste la duale, ovvero la forma PS (prodotto di somme).
Per trovare questa forma, esiste un metodo parallelo a quello studiato per la SP, dove si parte dal considerare i maxtermini invece che dei mintermini, cioè scegliendo sottocubi negli elementi che valgono 0 della mappa di Karnaugh.

Non considereremo questo metodo, ma un'alternativa più veloce:

\begin{algorithm}
\caption{per la sintesi in forma PS}
\begin{algorithmic}
	\STATE \textbf{Input:} una legge combinatoria $F$
	\STATE \textbf{Output:} la sintesi in forma PS di $F$
	\STATE Si ricava $\overline{F}$ complementando $F$
	\STATE Si reallizza una sintesi SP della legge $\overline{F}$
	\STATE Si ottiene una sintesi di $F$ aggiungendo un invertitore in uscita alla rete SP che sintetizza $\overline{F}$
	\STATE Si applicano i teoremi di de Morgan, da destra verso sinistra
\end{algorithmic}
\end{algorithm}

Algebricamente, l'ultimo passaggio significa scrivere $\overline{F}$ in forma SP:
$$
\overline{z} = P_1 + ... + P_k
$$
dove $P_i$ sono prodotti di variabili di ingresso, e applicare de Morgan come:
$$
z = \overline{\overline{z}} = \overline{P_1 + ... + P_k} = \overline{P_1} \cdot ... \cdot \overline{P_k}
$$

A questo punto si applica di nuovo de Morgan, come:
$$
\overline{P_i} = \overline{\prod x_j} = \sum \overline{x_j}
$$

\subsubsection{Dualità fra forme SP e PS}
Con il procedimento presentato abbiamo che se $\overline{F}$ è in forma canonica SP, allora $F$ è in forma canonica PS.
Se la sintesi SP di $\overline{F}$ costa $C$, allora la sintesi PS di $F$ costa C.
Quindi se la sintesi SP di $\overline{F}$ è a costo minimo fra tutte le possibili sintesi SP, lo è anche la sintesi PS di $F$ fra tutte le possibili sintesi PS.
Se fosse il contrario, applicando de Morgan più volte avrei sintesi di costo sempre minore, violando la dualità.

A questo punto sappiamo effettuare la sintesi a costo minimo in forma SP di una qualsiasi legge $F$, e ponendo di sintetizzare prima $\overline{F}$ in forma SP, sappiamo anche trovare la sintesi a costo minimo in forma PS della stessa legge.
Non possiamo determinare con sicurezza quale fra queste due sintesi ha costo minimo in generale, quindi bisogna controllare per forza la tabella della verità.

Troviamo ad esempio la sintesi in forma PS del BCD a 7 segmenti visto prima.
Si ha che la negazine di $F$, su $z_E$, è:

\begin{table}[H]
	\center 
	\begin{tabular} { c | c c c c | c | c }
		$j$ & $x_3$ & $x_2$ & $x_1$ & $x_0$ & $z_E$ & $\overline{z_E}$ \\
		\hline
		$0$ & $0$ & $0$ & $0$ & $0$ & $1$ & $0$ \\
		$1$ & $0$ & $0$ & $0$ & $1$ & $0$ & $1$ \\
		$2$ & $0$ & $0$ & $1$ & $0$ & $1$ & $0$ \\
		$3$ & $0$ & $0$ & $1$ & $1$ & $0$ & $1$ \\

		$4$ & $0$ & $1$ & $0$ & $0$ & $0$ & $1$ \\
		$5$ & $0$ & $1$ & $0$ & $1$ & $0$ & $1$ \\
		$6$ & $0$ & $1$ & $1$ & $0$ & $1$ & $0$ \\
		$7$ & $0$ & $1$ & $1$ & $1$ & $0$ & $1$ \\
	
		$8$ & $1$ & $0$ & $0$ & $0$ & $1$ & $0$ \\
		$9$ & $1$ & $0$ & $0$ & $1$ & $0$ & $1$ \\
		$10$ & $1$ & $0$ & $1$ & $0$ & $-$ & $-$ \\
		$11$ & $1$ & $0$ & $1$ & $1$ & $-$ & $-$ \\

		$12$ & $1$ & $1$ & $0$ & $0$ & $-$ & $-$ \\
		$13$ & $1$ & $1$ & $0$ & $1$ & $-$ & $-$ \\
		$14$ & $1$ & $1$ & $1$ & $0$ & $-$ & $-$ \\
		$15$ & $1$ & $1$ & $1$ & $1$ & $-$ & $-$ \\
	\end{tabular}
\end{table}

Ricaviamo quindi la mappa di Karnaugh:

\begin{center}
\noindent
\begin{minipage}{0.3\textwidth}
\begin{karnaugh-map}
		\manualterms{0, 1, 0, 1, 1, 1, 0, 1, 0, 1, -, -, -, -, -, -}
		\implicant{1}{11}
		\implicant{12}{14}
		\implicant{15}{10}
		\implicant{4}{13}
\end{karnaugh-map}
\end{minipage}%
\hspace{3cm}
\begin{minipage}{0.3\textwidth}
	\begin{table}[H]
		\center \rowcolors{2}{white}{black!10}
		\begin{tabular} { c || c | c | c | c}
			& $x_3$ & $x_2$ & $x_1$ & $x_0$ \\ 
			\hline 
			\rowcolor{red!20!white} A & - & - & - & 1 \\
			\rowcolor{green!20!white} B & 1 & 1 & - & - \\
			\rowcolor{yellow!20!white} C & 1 & - & 1 & - \\
			\rowcolor{cyan!20!white} D & - & 1 & 0 & - \\
		\end{tabular}
	\end{table}
\end{minipage}
\end{center}

Si ha che B e C sono inutili, in quanto comprendono solo indeterminati.
Restano allora A e D, entrambi essenziali, ergo la sintesi SP di $\overline{F}$ è:
$$ \overline{F} = \overline{z_E} = x_0 + x_2 \overline{x_1} $$
che neghiamo per ottenere nuovamente $F$:
$$ F = z_E = \overline{ x_0 + x_2 \overline{x_1} } $$

A questo punto si può applicare de Morgan, prima sulla somma e poi sul prodotto a destra, per ottenere:
$$ = \overline{x_0} \cdot \overline{x_2 \overline{x1}} = \overline{x_0} \cdot \left( \overline{x_2} + x_1 \right)$$

Cioè è la sintesi di $z_E$ in forma PS, che notiamo essere meno costosa della sintesi in forma SP, di due porte logiche in meno. 

\end{document}


\documentclass[a4paper,11pt]{article}
\usepackage[a4paper, margin=8em]{geometry}

% usa i pacchetti per la scrittura in italiano
\usepackage[french,italian]{babel}
\usepackage[T1]{fontenc}
\usepackage[utf8]{inputenc}
\frenchspacing 

% usa i pacchetti per la formattazione matematica
\usepackage{amsmath, amssymb, amsthm, amsfonts}

% usa altri pacchetti
\usepackage{gensymb}
\usepackage{hyperref}
\usepackage{standalone}

\usepackage{colortbl}

\usepackage{xstring}
\usepackage{karnaugh-map}

% imposta il titolo
\title{Appunti Reti Logiche}
\author{Luca Seggiani}
\date{2024}

% imposta lo stile
% usa helvetica
\usepackage[scaled]{helvet}
% usa palatino
\usepackage{palatino}
% usa un font monospazio guardabile
\usepackage{lmodern}

\renewcommand{\rmdefault}{ppl}
\renewcommand{\sfdefault}{phv}
\renewcommand{\ttdefault}{lmtt}

% circuiti
\usepackage{circuitikz}
\usetikzlibrary{babel}

% disponi il titolo
\makeatletter
\renewcommand{\maketitle} {
	\begin{center} 
		\begin{minipage}[t]{.8\textwidth}
			\textsf{\huge\bfseries \@title} 
		\end{minipage}%
		\begin{minipage}[t]{.2\textwidth}
			\raggedleft \vspace{-1.65em}
			\textsf{\small \@author} \vfill
			\textsf{\small \@date}
		\end{minipage}
		\par
	\end{center}

	\thispagestyle{empty}
	\pagestyle{fancy}
}
\makeatother

% disponi teoremi
\usepackage{tcolorbox}
\newtcolorbox[auto counter, number within=section]{theorem}[2][]{%
	colback=blue!10, 
	colframe=blue!40!black, 
	sharp corners=northwest,
	fonttitle=\sffamily\bfseries, 
	title=Teorema~\thetcbcounter: #2, 
	#1
}

% disponi definizioni
\newtcolorbox[auto counter, number within=section]{definition}[2][]{%
	colback=red!10,
	colframe=red!40!black,
	sharp corners=northwest,
	fonttitle=\sffamily\bfseries,
	title=Definizione~\thetcbcounter: #2,
	#1
}

% disponi codice
\usepackage{listings}
\usepackage[table]{xcolor}

\definecolor{codegreen}{rgb}{0,0.6,0}
\definecolor{codegray}{rgb}{0.5,0.5,0.5}
\definecolor{codepurple}{rgb}{0.58,0,0.82}
\definecolor{backcolour}{rgb}{0.95,0.95,0.92}

\lstdefinestyle{codestyle}{
		backgroundcolor=\color{black!5}, 
		commentstyle=\color{codegreen},
		keywordstyle=\bfseries\color{magenta},
		numberstyle=\sffamily\tiny\color{black!60},
		stringstyle=\color{green!50!black},
		basicstyle=\ttfamily\footnotesize,
		breakatwhitespace=false,         
		breaklines=true,                 
		captionpos=b,                    
		keepspaces=true,                 
		numbers=left,                    
		numbersep=5pt,                  
		showspaces=false,                
		showstringspaces=false,
		showtabs=false,                  
		tabsize=2
}

\lstdefinestyle{shellstyle}{
		backgroundcolor=\color{black!5}, 
		basicstyle=\ttfamily\footnotesize\color{black}, 
		commentstyle=\color{black}, 
		keywordstyle=\color{black},
		numberstyle=\color{black!5},
		stringstyle=\color{black}, 
		showspaces=false,
		showstringspaces=false, 
		showtabs=false, 
		tabsize=2, 
		numbers=none, 
		breaklines=true
}


\lstdefinelanguage{assembler}{ 
  keywords={AAA, AAD, AAM, AAS, ADC, ADCB, ADCW, ADCL, ADD, ADDB, ADDW, ADDL, AND, ANDB, ANDW, ANDL,
        ARPL, BOUND, BSF, BSFL, BSFW, BSR, BSRL, BSRW, BSWAP, BT, BTC, BTCB, BTCW, BTCL, BTR, 
        BTRB, BTRW, BTRL, BTS, BTSB, BTSW, BTSL, CALL, CBW, CDQ, CLC, CLD, CLI, CLTS, CMC, CMP,
        CMPB, CMPW, CMPL, CMPS, CMPSB, CMPSD, CMPSW, CMPXCHG, CMPXCHGB, CMPXCHGW, CMPXCHGL,
        CMPXCHG8B, CPUID, CWDE, DAA, DAS, DEC, DECB, DECW, DECL, DIV, DIVB, DIVW, DIVL, ENTER,
        HLT, IDIV, IDIVB, IDIVW, IDIVL, IMUL, IMULB, IMULW, IMULL, IN, INB, INW, INL, INC, INCB,
        INCW, INCL, INS, INSB, INSD, INSW, INT, INT3, INTO, INVD, INVLPG, IRET, IRETD, JA, JAE,
        JB, JBE, JC, JCXZ, JE, JECXZ, JG, JGE, JL, JLE, JMP, JNA, JNAE, JNB, JNBE, JNC, JNE, JNG,
        JNGE, JNL, JNLE, JNO, JNP, JNS, JNZ, JO, JP, JPE, JPO, JS, JZ, LAHF, LAR, LCALL, LDS,
        LEA, LEAVE, LES, LFS, LGDT, LGS, LIDT, LMSW, LOCK, LODSB, LODSD, LODSW, LOOP, LOOPE,
        LOOPNE, LSL, LSS, LTR, MOV, MOVB, MOVW, MOVL, MOVSB, MOVSD, MOVSW, MOVSX, MOVSXB,
        MOVSXW, MOVSXL, MOVZX, MOVZXB, MOVZXW, MOVZXL, MUL, MULB, MULW, MULL, NEG, NEGB, NEGW,
        NEGL, NOP, NOT, NOTB, NOTW, NOTL, OR, ORB, ORW, ORL, OUT, OUTB, OUTW, OUTL, OUTSB, OUTSD,
        OUTSW, POP, POPL, POPW, POPB, POPA, POPAD, POPF, POPFD, PUSH, PUSHL, PUSHW, PUSHB, PUSHA, 
				PUSHAD, PUSHF, PUSHFD, RCL, RCLB, RCLW, MOVSL, MOVSB, MOVSW, STOSL, STOSB, STOSW, LODSB, LODSW,
				LODSL, INSB, INSW, INSL, OUTSB, OUTSL, OUTSW
        RCLL, RCR, RCRB, RCRW, RCRL, RDMSR, RDPMC, RDTSC, REP, REPE, REPNE, RET, ROL, ROLB, ROLW,
        ROLL, ROR, RORB, RORW, RORL, SAHF, SAL, SALB, SALW, SALL, SAR, SARB, SARW, SARL, SBB,
        SBBB, SBBW, SBBL, SCASB, SCASD, SCASW, SETA, SETAE, SETB, SETBE, SETC, SETE, SETG, SETGE,
        SETL, SETLE, SETNA, SETNAE, SETNB, SETNBE, SETNC, SETNE, SETNG, SETNGE, SETNL, SETNLE,
        SETNO, SETNP, SETNS, SETNZ, SETO, SETP, SETPE, SETPO, SETS, SETZ, SGDT, SHL, SHLB, SHLW,
        SHLL, SHLD, SHR, SHRB, SHRW, SHRL, SHRD, SIDT, SLDT, SMSW, STC, STD, STI, STOSB, STOSD,
        STOSW, STR, SUB, SUBB, SUBW, SUBL, TEST, TESTB, TESTW, TESTL, VERR, VERW, WAIT, WBINVD,
        XADD, XADDB, XADDW, XADDL, XCHG, XCHGB, XCHGW, XCHGL, XLAT, XLATB, XOR, XORB, XORW, XORL},
  keywordstyle=\color{blue}\bfseries,
  ndkeywordstyle=\color{darkgray}\bfseries,
  identifierstyle=\color{black},
  sensitive=false,
  comment=[l]{\#},
  morecomment=[s]{/*}{*/},
  commentstyle=\color{purple}\ttfamily,
  stringstyle=\color{red}\ttfamily,
  morestring=[b]',
  morestring=[b]"
}

\lstset{language=assembler, style=codestyle}

% disponi sezioni
\usepackage{titlesec}

\titleformat{\section}
	{\sffamily\Large\bfseries} 
	{\thesection}{1em}{} 
\titleformat{\subsection}
	{\sffamily\large\bfseries}   
	{\thesubsection}{1em}{} 
\titleformat{\subsubsection}
	{\sffamily\normalsize\bfseries} 
	{\thesubsubsection}{1em}{}

% tikz
\usepackage{tikz}

% float
\usepackage{float}

% grafici
\usepackage{pgfplots}
\pgfplotsset{width=10cm,compat=1.9}

% disponi alberi
\usepackage{forest}

\forestset{
	rectstyle/.style={
		for tree={rectangle,draw,font=\large\sffamily}
	},
	roundstyle/.style={
		for tree={circle,draw,font=\large}
	}
}

% disponi algoritmi
\usepackage{algorithm}
\usepackage{algorithmic}
\makeatletter
\renewcommand{\ALG@name}{Algoritmo}
\makeatother

% disponi numeri di pagina
\usepackage{fancyhdr}
\fancyhf{} 
\fancyfoot[L]{\sffamily{\thepage}}

\makeatletter
\fancyhead[L]{\raisebox{1ex}[0pt][0pt]{\sffamily{\@title \ \@date}}} 
\fancyhead[R]{\raisebox{1ex}[0pt][0pt]{\sffamily{\@author}}}
\makeatother

\begin{document}
% sezione (data)
\section{Lezione del 15-10-24}

% stili pagina
\thispagestyle{empty}
\pagestyle{fancy}

% testo
\subsection{Porte logiche universali}
Si dice che NAND e NOR sono \textbf{porte logiche universali}.
Si possono realizzare AND, OR e NOT usando solo porte NAND o solo porte NOR.

Algebricamente, questo significa fare:
\begin{table}[h!]
	\center \rowcolors{2}{white}{black!10}
	\begin{tabular} { c || c | c }
		\bfseries Porta & \bfseries Realizzazione NAND & \bfseries Realizzazione NOR \\
		\hline 
		NOT & $ x = x \cdot x \Rightarrow \overline{x} = \overline{x \cdot x} $ & $ x = x + x \Rightarrow \overline{x} = \overline{x + x} $ \\
		AND & $ x \cdot y = \overline{\left( \overline{x \cdot y} \right)} $ & $ x \cdot y = \overline{\overline{x} + \overline{y}} \ $ (de Morgan) \\
		OR & $ x + y = \overline{\overline{x} \cdot \overline{y}} \ $ (de Morgan) & $ x + y = \overline{\left( \overline{x + y} \right)} $ \\
	\end{tabular}
\end{table}

cioè collegare porte logiche fisiche nelle seguenti configurazioni:
\begin{table}[h!]
	\center 
	\begin{tabular} { c || c | c }
		\bfseries Porta & \bfseries Realizzazione NAND & \bfseries Realizzazione NOR \\
		\hline 
		NOT &
\begin{tabular}{l}
\begin{circuitikz}
	\draw (0,0) node[nand port, number inputs=2] (myNAND) {};
	
	\draw (myNAND.in 1) node[left] {} -- ++(-0.5,0) |- (myNAND.in 2);

	\node at (-2.2, 0) {$x$}; 
	\draw (myNAND.out) -- ++(0.5,0) node[right] {$z$};
\end{circuitikz}
\end{tabular}
		& 
\begin{tabular}{l}
\begin{circuitikz}
	\draw (0,0) node[nor port, number inputs=2] (myNAND) {};
	
	\draw (myNAND.in 1) node[left] {} -- ++(-0.5,0) |- (myNAND.in 2);

	\node at (-2.2, 0) {$x$}; 
	\draw (myNAND.out) -- ++(0.5,0) node[right] {$z$};
\end{circuitikz}
\end{tabular}
		\\
		\hline
		AND & 
\begin{tabular}{l}
\begin{circuitikz}
	\draw (-2.04,0) node[nand port, number inputs=2] (myNAND1) {};
	\draw (0,0) node[nand port, number inputs=2] (myNAND) {};
	
	\draw (myNAND.in 1) node[left] {} -- ++(-0.5,0) |- (myNAND.in 2);

	\node at (-3.7, 0.3) {$x$}; 
	\node at (-3.7, -0.3) {$y$}; 
	\draw (myNAND.out) -- ++(0.5,0) node[right] {$z$};
\end{circuitikz}
\end{tabular}
		& 
\begin{tabular}{l}
\begin{circuitikz}
	\draw (-2.04,0.75) node[nor port, number inputs=2] (myNAND1) {};
	\draw (-2.04,-0.75) node[nor port, number inputs=2] (myNAND2) {};
	\draw (0,0) node[nor port, number inputs=2] (myNAND) {};
	
	\draw (myNAND1.in 1) node[left] {} -- ++(-0.5,0) |- (myNAND1.in 2);
	\draw (myNAND2.in 1) node[left] {} -- ++(-0.5,0) |- (myNAND2.in 2);

	\draw (myNAND.in 1) node[left] {} -- (myNAND1.out);
	\draw (myNAND.in 2) node[left] {} -- (myNAND2.out);

	\node at (-4.2, 0.75) {$x$}; 
	\node at (-4.2, -0.75) {$y$}; 
	\draw (myNAND.out) -- ++(0.5,0) node[right] {$z$};
\end{circuitikz}
\end{tabular}
		\\
		\hline
		OR  
		&
\begin{tabular}{l}
\begin{circuitikz}
	\draw (-2.04,0.75) node[nand port, number inputs=2] (myNAND1) {};
	\draw (-2.04,-0.75) node[nand port, number inputs=2] (myNAND2) {};
	\draw (0,0) node[nand port, number inputs=2] (myNAND) {};
	
	\draw (myNAND1.in 1) node[left] {} -- ++(-0.5,0) |- (myNAND1.in 2);
	\draw (myNAND2.in 1) node[left] {} -- ++(-0.5,0) |- (myNAND2.in 2);

	\draw (myNAND.in 1) node[left] {} -- (myNAND1.out);
	\draw (myNAND.in 2) node[left] {} -- (myNAND2.out);

	\node at (-4.2, 0.75) {$x$}; 
	\node at (-4.2, -0.75) {$y$}; 
	\draw (myNAND.out) -- ++(0.5,0) node[right] {$z$};
\end{circuitikz}
\end{tabular}
		&
\begin{tabular}{l}
\begin{circuitikz}
	\draw (-2.04,0) node[nor port, number inputs=2] (myNAND1) {};
	\draw (0,0) node[nor port, number inputs=2] (myNAND) {};
	
	\draw (myNAND.in 1) node[left] {} -- ++(-0.5,0) |- (myNAND.in 2);

	\node at (-3.7, 0.3) {$x$}; 
	\node at (-3.7, -0.3) {$y$}; 
	\draw (myNAND.out) -- ++(0.5,0) node[right] {$z$};
\end{circuitikz}
\end{tabular}
	\end{tabular}
\end{table}

Potremmo sembrare che, se si poteva realizzare qualsiasi rete combinatoria con AND, OR e NOT su 2 livelli di logica, usando solo NAND o NOR dovremmo accontentarci di 4 livelli di logica (AND e OR richiedono di per sé una rete a 2 livelli di logica).

In verità, le porte NAND e NOR permettono di creare circuiti logici con gli stessi livelli di logica delle porte AND, OR e NOT.

\subsubsection{Sintesi a porte NAND}
Vediamo quindi il seguente algoritmo per la sintesi di un circuito con sole porte NAND:
\begin{algorithm}
\caption{sintesi a porte NAND}
\begin{algorithmic}
	\STATE \textbf{Input:} un circuito in forma SP 
	\STATE \textbf{Output:} una sintesi a porte NAND
	\STATE Si sostituisce la porta OR con il suo equivalente a NAND
	\STATE Si sostituisce ciascun AND con il suo equivalente a NAND
	\STATE Si eliminano le coppie di NOT interne a cascata
\end{algorithmic}
\end{algorithm}

Ignoriamo i NOT sull'ingresso, in quanto abbiamo visto sono effettivamente gratuiti.
Abbiamo quindi che, rimuovendo le coppie NOT interni (creati da coppie di NAND con gli stessi input) ritorniamo in una forma a 2 livelli di logica.

Dal punto di vista algebrico si ha:
$$
z = P_1 + P_2 + ... + P_k = \overline{\overline{ P_1 + P_2 + ... + P_k }} = \overline{ \overline{P_1} \cdot \overline{P_2} \cdot ... \cdot \overline{P_k}}
$$
dove il complemento superiore è l'ultima porta NAND (quella che sostituisce l'OR), e i singoli $P_i$ complementati sono singole porte NAND ($P_i$ è un prodotto, quindi $\overline{P_i}$ è una porta NAND).

\subsubsection{Sintesi a porte NOR}
Vediamo poi l'algoritmo per la sintesi di un circuito con sole porte NOR:
\begin{algorithm}
\caption{sintesi a porte NOR}
\begin{algorithmic}
	\STATE \textbf{Input:} un circuito in forma PS 
	\STATE \textbf{Output:} una sintesi a porte NOR
	\STATE Si sostituisce la porta AND con il suo equivalente a NOR
	\STATE Si sostituisce ciascun OR con il suo equivalente a NOR
	\STATE Si eliminano le coppie di NOT interne a cascata
\end{algorithmic}
\end{algorithm}

Anche qui ignoriamo i NOT sull'ingresso, per gli stessi motivi di prima, e rimuovendo le coppie NOT interni (creati da coppie di NAND con gli stessi input) ritorniamo nuovamente in una forma a 2 livelli di logica.

Dal punto di vista algebrico si ha:
$$
z = S_1 \cdot S_2 \cdot ... \cdot S_k = \overline{\overline{ S_1 \cdot S_2 \cdot ... \cdot S_k }} = \overline{ \overline{S_1} + \overline{S_2} + ... + \overline{S_k}}
$$
dove il complemento superiore è l'ultima porta NOR (quella che sostituisce l'AND), e i singoli $S_i$ complementati sono singole porte NOR ($S_i$ è una somma, quindi $\overline{S_i}$ è una porta NOR).

\subsection{Porte tri-state}
Fa comodo poter connettere insieme le uscite delle reti usando bus condivisi, cioè linee di ingresso-uscita.
Abbiamo che l'uscita di una rete, dal punto di vista di una rete, corrisponde a un interrutture fra il Vcc (1 logico) o la massa (0 logico), cioé:

\begin{center}
\begin{circuitikz}
	\draw[->] (0,0.2) to[R, l=$R_1$] (0,2) node[above]{Vcc};
	\draw (0,-0.2) to[R, l_=$R_2$] (0,-2) node[ground]{};
	\draw[->] (0.2,0) -- (2, 0);

	\draw[thick] (0, 0.2) -- (0.2, 0);

	\node at(2,0.2) {$z$}; 

\end{circuitikz}
\end{center}

dove $R_1$ e $R_0$ sono incognite.

Quando vado a collegare più uscite sulla stessa linea possono crearsi più situazioni:

\begin{itemize} 
	\item \textbf{1 logici:}
		se ho due 1 logici, cioe due generatori di potenziale a Vcc, connessi sulla stessa linea, ho che la tensione sulla linea è sempre Vcc, quindi tutto ok:

\begin{center}
	\begin{circuitikz}
		\begin{scope}[shift={(0,3)}]
		\draw[->] (1,0) to (1.5,0);
    \draw (0,0) node[draw, rectangle, minimum width = 2cm, minimum height = 2cm] {};

		\draw (0,-1) node[below] {1};
		
		\draw (0, 1) to[ american voltage source, l=VCC, transform shape, scale=0.5] (0,0);
		\draw (0,0) to [ R, transform shape, scale=0.5] (2,0);
		\end{scope}	

		\begin{scope}[shift={(0,0)}]
		\draw[->] (1,0) to (1.5,0);
    \draw (0,0) node[draw, rectangle, minimum width = 2cm, minimum height = 2cm] {};

		\draw (0,-1) node[below] {1};
		
		\draw (0, 1) to[ american voltage source, l=VCC, transform shape, scale=0.5] (0,0);
		\draw (0,0) to [ R, transform shape, scale=0.5] (2,0);
		\end{scope}	

		\draw (1.5, 4) -- (1.5, -1);

	\end{circuitikz}
\end{center}
	\item \textbf{0 logici:}
		allo stesso tempo, se ho due 0 logici, quindi due collegamenti a massa, sulla linea si avrà tensione nulla:
\begin{center}
	\begin{circuitikz}
		\begin{scope}[shift={(0,3)}]
		\draw[->] (1,0) to (1.5,0);
    \draw (0,0) node[draw, rectangle, minimum width = 2cm, minimum height = 2cm] {};

		\draw (0,-1) node[below] {0};

		\draw (0,0) to [ R, transform shape, scale=0.5] (2,0);
		\draw (0, 0) node[ground] {};
		\end{scope}	

		\begin{scope}[shift={(0,0)}]
		\draw[->] (1,0) to (1.5,0);
    \draw (0,0) node[draw, rectangle, minimum width = 2cm, minimum height = 2cm] {};

		\draw (0,-1) node[below] {0};

		\draw (0,0) to [ R, transform shape, scale=0.5] (2,0);
		\draw (0, 0) node[ground] {};
		\end{scope}	

		\draw (1.5, 4) -- (1.5, -1);

	\end{circuitikz}
\end{center}

	\item \textbf{0 e 1 logici:}
		se collego un 1 logico e uno 0 logico alla stessa linea, ottengo effettivamente un partitore di tensione:
\begin{center}
	\begin{circuitikz}
		\begin{scope}[shift={(0,3)}]
		\draw[->] (1,0) to (1.5,0);
    \draw (0,0) node[draw, rectangle, minimum width = 2cm, minimum height = 2cm] {};

		\draw (0,-1) node[below] {1};
		
		\draw (0, 1) to[ american voltage source, l=VCC, transform shape, scale=0.5] (0,0);
		\draw (0,0) to [ R, transform shape, scale=0.5] (2,0);
		\end{scope}	

		\begin{scope}[shift={(0,0)}]
		\draw[->] (1,0) to (1.5,0);
    \draw (0,0) node[draw, rectangle, minimum width = 2cm, minimum height = 2cm] {};

		\draw (0,-1) node[below] {0};

		\draw (0,0) to [ R, transform shape, scale=0.5] (2,0);
		\draw (0, 0) node[ground] {};
		\end{scope}	

		\draw (1.5, 4) -- (1.5, -1);
		\node[circ] at(1.5, 1.5) {};
		\node[right] at(1.5, 1.5) {$V_l$};

	\end{circuitikz}
\end{center}
		da cui ricavo:
		$$
		V_l = \frac{R_0}{R_1 + R_0}
		$$

		Notiamo sopratutto che se $R_0$ e $R_1$ sono molto piccoli, otteniamo correnti $I$ molto grandi, che significa componenti bruciati.

\end{itemize}

Per risolvere il problema dato da 0 e 1 logici connessi sulla stessa linea, usiamo specifici apparecchi detti \textbf{porte tri-state}, che sono capaci di disconnettere fisicamente un'uscita da una linea condivisa.
Si rappresentano come:

\begin{center}
\begin{circuitikz}
    % Draw a tristate buffer
    \draw (0,0) node[buffer, anchor=in] (myBuffer) {};
    
    % Draw inputs and outputs
    \draw (myBuffer.in) -- ++(-1,0) node[left] {$x$}; % Input A
    \draw (myBuffer.out) -- ++(1,0) node[right] {$z$}; % Output Y
    \draw (0.5,-0.49) -- (0.5, -1) -- (-1,-1) node[left] {$b$}; % Enable input
\end{circuitikz}
\end{center}

dove $x$ è l'ingresso, $z$ l'uscita, e $b$ l'enabler.
A $b=1$ la porta si comporta come un'elemento neutro, mentre a $b=0$ offre un'alta impedenza, effettivamente scollegando l'uscita.
La tabella di verità corrispondente sarà:

\begin{table}[h!]
	\center 
	\begin{tabular} { c c | c }
		$b$ & $x$ & $z$ \\
		\hline 
		1 & 0 & 0 \\ 
		1 & 1 & 1 \\ 
		0 & - & Hi-Z
	\end{tabular}
\end{table}

Notiamo che il valore Hi-Z (alta impedenza) non è un valore logico: ciò che esce da una porta in stato Hi-Z viene interpretato come un filo staccato dal resto della rete.
Ogni porta logica gestisce poi questa situazione secondo le sue specifiche di realizzazione, restando comunque attaccata sia a Vcc che a massa, e quindi non in uno stato HiZ.

\subsubsection{Multiplexer decodificati}
Un componente realizzato attraverso le porte tri-state è il multiplexer decodificato:

\begin{center}
\begin{circuitikz}
    % Draw a tristate buffer
    \draw (0,0) node[buffer, anchor=in] (myBuffer) {};
    
    % Draw inputs and outputs
		\draw (myBuffer.in) -- ++(-1,0) node[left] {$x_{N-1}$}; % Input A
    \draw (myBuffer.out) -- ++(1,0) node[right] {}; % Output Y
		\draw (0.5,-0.49) -- (0.5, -1) -- (-1,-1) node[left] {$b_{N-1}$}; % Enable input

		\node at(0.5,-1.75) {$...$};

    % Draw a tristate buffer
    \draw (0,-3) node[buffer, anchor=in] (myBuffer1) {};
    
    % Draw inputs and outputs
    \draw (myBuffer1.in) -- ++(-1,0) node[left] {$x_0$}; % Input A
    \draw (myBuffer1.out) -- ++(1,0) node[right] {}; % Output Y
    \draw (0.5,-3.49) -- (0.5, -4) -- (-1,-4) node[left] {$b_0$}; % Enable input

    \draw (2.4,0) -- (2.4, -3); 
		\draw (2.4, -1.5) -- (3, -1.5) node[right] {$z$};

\end{circuitikz}
\end{center}

Questo componente offre alta indipendeza a tutte le variabili $x_i$ in ingresso tranne una, quella all'indice $j$, selezionata attraverso una variabile di comando $b_j$.

\subsubsection{Linea di ingresso/uscita}
Si usano le porte tri-state per permettere a componenti di comunicare su linee di ingresso/uscita, ad esempio con la memoria.
In questo caso, si biforca la linea, ammettendo la linea in entrata così com'è, e mettendo una porta tri-state nella linea in uscita.

\begin{center}
\begin{circuitikz}
    \draw (-2,0.1) node[draw, rectangle, minimum width = 2cm, minimum height = 4cm] {};
		\node at (-2, 0.1) {$S$};

		\node at (-0.75, -1.2) {$b$};

		\draw[<-] (-1,1.5) -- (2.4, 1.5);

		% Draw a tristate buffer
    \draw (0,0) node[buffer, anchor=in] (myBuffer) {};
    
    % Draw inputs and outputs
		\draw[<-] (myBuffer.in) -- ++(-1,0) {}; % Input A
		\draw (myBuffer.out) -- ++(1,0) -- (2.4, 1.5); % Output Y
		\draw (0.5,-0.49) -- (0.5, -1) -- (-1,-1) ; % Enable input

		\draw (2.4,0.75) -- (3, 0.75) node[right] {$d$};
\end{circuitikz}
\end{center}

Così, quando il componente $S$ vuole comunicare con l'esterno, imposta l'enabler $b$ a 1 e mette sulla linea $d$ ciò che vuole comunicare. Altrimenti tiene $b$ a 0 e ascolta ciò che arriva su $d$.

Se il componente $S$ comunica con un altro componente $T$, questi dovranno impostare alternativamente i loro enabler $b_S$ e $b_T$ a 1 e 0, scambiandosi messaggi sulla linea $d$.

\subsection{Circuiti di ritardo e formatori di impulso}
A volte bisogna trattare di \textbf{segnali}.
In questo caso si usano gli elementi neutri $\Delta$ (realizzati spesso con numeri pari di invertitori), che sappiamo porre un \textbf{ritardo simmetrico} agli ingressi, dove simmetrico significa identico sulle transizioni $0 \rightarrow 1$ e $1 \rightarrow 0$.

Potrebbe essere utile avere circuiti con ritardi \textbf{asimmetrici}, cioè variabili sulle transizioni $0 \rightarrow 1$ e $1 \rightarrow 0$.
Indichiamo questi componenti come $\Delta^+$.

\subsubsection{Circuito di ritardo sul fronte di discesa}
Collegando un neutro $\Delta$ assieme al segnale stesso ad una porta AND, si ottiene un'andamento del tipo:

\begin{center}
	\begin{minipage}{0.2\textwidth}  % Adjust width as needed
		\begin{circuitikz}
			\node at(0,0) {$x$};
			
			\draw (0.5, 0) -- (1, 0);
			\draw (1,0.5) -- (1, -0.5);
			\draw  (1, 0.5) -- (1.96, 0.5);

    	\draw (2.25, 0.5) node[draw, rectangle, minimum width = 0.4cm, minimum height = 0.75cm] {$\Delta$};

			\draw[->] (2.54, 0.5) -- (3.4, 0.5);
			\draw[->] (1, -0.5) -- (3.4, -0.5);
			
    	\draw (4,0) node[draw, rectangle, minimum width = 1cm, minimum height = 1.5cm] {AND};
			\draw (4.58, 0) -- (5, 0);

			\node at(5.2,0) {$z$};
		\end{circuitikz}
	\end{minipage}
	\hspace{2cm}  % Horizontal space between the two components
	\begin{minipage}{0.6\textwidth}  % Adjust width as needed
		\begin{tikzpicture}
    \begin{axis}[
        xmin=4, xmax=9,
        ymin=-1, ymax=6,
        grid=major,
        domain=4:9,
        xtick={5,6,8},
        ytick={0,5},
        xticklabels={$t_1$, $t_1 + \Delta$,$t_2$},
        yticklabels={$0$, $V_{max}$},
        samples=100,
        legend pos=north west, % Position of the legend
        width=9cm,
        height=7cm
    ]
    % Blue plot with legend entry
    \addplot[blue, thick] {5 * (x >= 5) * (x <=  8)}; 
    \addlegendentry{$x$} % Legend entry for the blue plot
    
    % Red plot with legend entry
    \addplot[red, thick] {5 * (x >= 6) * (x <= 8) + 0.15};
    \addlegendentry{$z$} % Legend entry for the red plot
    \end{axis}
\end{tikzpicture}
	\end{minipage}
\end{center}

Nello specifico, transizionando da $1 \rightarrow 0$, si ha che il primo ingresso che va a 0 porta a 0 l'uscita. C'è un ritardo piccolo da parte della porta AND.
Quando invece si tranziziona da $0 \rightarrow 1$, si ha che il secondo ingresso che va a 1 (quello che passa da $\Delta$) porta a 1 l'uscita. C'è un ritardo grande da parte del $\Delta$ e della porta AND.

\subsubsection{Circuito di ritardo sul fronte di salita}
Allo stesso modo, collegando un neutro $\Delta$ assieme al segnale stesso ad una porta OR, si ottiene un'andamento del tipo:

\begin{center}
	\begin{minipage}{0.2\textwidth}  % Adjust width as needed
		\begin{circuitikz}
			\node at(0,0) {$x$};
			
			\draw (0.5, 0) -- (1, 0);
			\draw (1,0.5) -- (1, -0.5);
			\draw  (1, 0.5) -- (1.96, 0.5);

    	\draw (2.25, 0.5) node[draw, rectangle, minimum width = 0.4cm, minimum height = 0.75cm] {$\Delta$};

			\draw[->] (2.54, 0.5) -- (3.4, 0.5);
			\draw[->] (1, -0.5) -- (3.4, -0.5);
			
    	\draw (4,0) node[draw, rectangle, minimum width = 1cm, minimum height = 1.5cm] {OR};
			\draw (4.58, 0) -- (5, 0);

			\node at(5.2,0) {$z$};
		\end{circuitikz}
	\end{minipage}
	\hspace{2cm}  % Horizontal space between the two components
	\begin{minipage}{0.6\textwidth}  % Adjust width as needed
		\begin{tikzpicture}
    \begin{axis}[
        xmin=4, xmax=9,
        ymin=-1, ymax=6,
        grid=major,
        domain=4:9,
        xtick={5,7,8},
        ytick={0,5},
        xticklabels={$t_1$, $t_2$, $t_2 + \Delta$},
        yticklabels={$0$, $V_{max}$},
        samples=100,
        legend pos=north west, % Position of the legend
        width=9cm,
        height=7cm
    ]
    % Blue plot with legend entry
    \addplot[blue, thick] {5 * (x >= 5) * (x <=  7)}; 
    \addlegendentry{$x$} % Legend entry for the blue plot
    
    % Red plot with legend entry
    \addplot[red, thick] {5 * (x >= 5) * (x <= 8) + 0.15};
    \addlegendentry{$z$} % Legend entry for the red plot
    \end{axis}
\end{tikzpicture}
	\end{minipage}
\end{center}

Nello specifico, transizionando da $0 \rightarrow 1$, si ha che il primo ingresso che va a 1 porta a 1 l'uscita. C'è un ritardo piccolo da parte della porta OR.
Quando invece si tranziziona da $1 \rightarrow 0$, si ha che il secondo ingresso che va a 0 (quello che passa da $\Delta$) porta a 0 l'uscita. C'è un ritardo grande da parte del $\Delta$ e della porta OR.

\subsubsection{Formatore di impulso sul fronte di salita}
I formatori di impulso sono reti combinatorie che generano in uscita un \textbf{impulso} di durata nota.
Si indicano con $P^+$.

Si crea un formatore di impulso sul fronte di salita collegando la negazione di un $\Delta$ e il segnale stesso ad una porta AND, cioé:

\begin{center}
	\begin{minipage}{0.2\textwidth}  % Adjust width as needed
		\begin{circuitikz}
			\node at(0,0) {$x$};
			
			\draw (0.5, 0) -- (1, 0);
			\draw (1,0.5) -- (1, -0.5);
			\draw  (1, 0.5) -- (1.2, 0.5);

    	\draw (1.5, 0.5) node[draw, rectangle, minimum width = 0.4cm, minimum height = 0.75cm] {$\Delta$};
    	\draw (2.5, 0.5) node[draw, rectangle, minimum width = 0.4cm, minimum height = 0.75cm] {NOT};

			\draw (1.78, 0.5) -- (1.94, 0.5);

			\draw[->] (3.05, 0.5) -- (3.4, 0.5);
			\draw[->] (1, -0.5) -- (3.4, -0.5);
			
    	\draw (4,0) node[draw, rectangle, minimum width = 1cm, minimum height = 1.5cm] {AND};
			\draw (4.58, 0) -- (5, 0);

			\node at(5.2,0) {$z$};
		\end{circuitikz}
	\end{minipage}
	\hspace{2cm}  % Horizontal space between the two components
	\begin{minipage}{0.6\textwidth}  % Adjust width as needed
		\begin{tikzpicture}
    \begin{axis}[
        xmin=4, xmax=9,
        ymin=-1, ymax=6,
        grid=major,
        domain=4:9,
        xtick={5,6,8},
        ytick={0,5},
        xticklabels={$t_1$, $t_1 + \Delta$, $t_2$},
        yticklabels={$0$, $V_{max}$},
        samples=100,
        legend pos=north west, % Position of the legend
        width=9cm,
        height=7cm
    ]
    % Blue plot with legend entry
    \addplot[blue, thick] {5 * (x >= 5) * (x <=  8)}; 
    \addlegendentry{$x$} % Legend entry for the blue plot
    
    % Red plot with legend entry
    \addplot[red, thick] {5 * (x >= 5) * (x <= 6) + 0.15};
    \addlegendentry{$z$} % Legend entry for the red plot
    \end{axis}
\end{tikzpicture}
	\end{minipage}
\end{center}

Nello specifico, transizionando da $0 \rightarrow 1$, si ha che il segnale va a 1, attivando la AND (l'ingresso dalla NOT era già attivo). 
Dopo il ritardo $\Delta$, NOT torna a 0, e quindi l'uscita della AND va a 0.
Si ha quindi un'impulso di durata del ritardo $\Delta$.
Transizionando da $1 \rightarrow 0$, invece, si ha che il segnale "ancora" istantaneamente l'uscita della AND a zero, ergo non si hanno altri artefatti.

\subsubsection{Formatore di impulso sul fronte di discesa}
Si crea un formatore di impulso sul fronte di discesa collegando la negazione di un $\Delta$ e il segnale stesso ad una porta NOR, cioé:

\begin{center}
	\begin{minipage}{0.2\textwidth}  % Adjust width as needed
		\begin{circuitikz}
			\node at(0,0) {$x$};
			
			\draw (0.5, 0) -- (1, 0);
			\draw (1,0.5) -- (1, -0.5);
			\draw  (1, 0.5) -- (1.2, 0.5);

    	\draw (1.5, 0.5) node[draw, rectangle, minimum width = 0.4cm, minimum height = 0.75cm] {$\Delta$};
    	\draw (2.5, 0.5) node[draw, rectangle, minimum width = 0.4cm, minimum height = 0.75cm] {NOT};

			\draw (1.78, 0.5) -- (1.94, 0.5);

			\draw[->] (3.05, 0.5) -- (3.4, 0.5);
			\draw[->] (1, -0.5) -- (3.4, -0.5);
			
    	\draw (4,0) node[draw, rectangle, minimum width = 1cm, minimum height = 1.5cm] {NOR};
			\draw (4.58, 0) -- (5, 0);

			\node at(5.2,0) {$z$};
		\end{circuitikz}
	\end{minipage}
	\hspace{2cm}  % Horizontal space between the two components
	\begin{minipage}{0.6\textwidth}  % Adjust width as needed
		\begin{tikzpicture}
    \begin{axis}[
        xmin=4, xmax=9,
        ymin=-1, ymax=6,
        grid=major,
        domain=4:9,
        xtick={5,7,8},
        ytick={0,5},
        xticklabels={$t_1$, $t_2$, $t_2 + \Delta$},
        yticklabels={$0$, $V_{max}$},
        samples=100,
        legend pos=north west, % Position of the legend
        width=9cm,
        height=7cm
    ]
    % Blue plot with legend entry
    \addplot[blue, thick] {5 * (x >= 5) * (x <=  7)}; 
    \addlegendentry{$x$} % Legend entry for the blue plot
    
    % Red plot with legend entry
    \addplot[red, thick] {5 * (x >= 7) * (x <= 8) + 0.15};
    \addlegendentry{$z$} % Legend entry for the red plot
    \end{axis}
\end{tikzpicture}
	\end{minipage}
\end{center}

Nello specifico, transizionando da $0 \rightarrow 1$, si ha che il segnale va a 1,ergo la NOR resta a 0 (l'ingresso dalla NOT era già attivo, e due ingressi attivi sono sempre 0 della NOR). 
Dopo il ritardo $\Delta$, l'uscita della NOT torna a 0, così che quando si stacca il segnale, per una durata $\Delta$ entrambe le linee in entrata alla NOR vanno a 0, e quindi questa va a 1.
Dopo il ritardo $\Delta$, NOT torna a 0, e quindi l'uscita della AND va a 0.
Si ha quindi, ancora una volta, un'impulso di durata del ritardo $\Delta$.

\end{document}


\documentclass[a4paper,11pt]{article}
\usepackage[a4paper, margin=8em]{geometry}

% usa i pacchetti per la scrittura in italiano
\usepackage[french,italian]{babel}
\usepackage[T1]{fontenc}
\usepackage[utf8]{inputenc}
\frenchspacing 

% usa i pacchetti per la formattazione matematica
\usepackage{amsmath, amssymb, amsthm, amsfonts}

% usa altri pacchetti
\usepackage{gensymb}
\usepackage{hyperref}
\usepackage{standalone}

\usepackage{colortbl}

\usepackage{xstring}
\usepackage{karnaugh-map}

% imposta il titolo
\title{Appunti Reti Logiche}
\author{Luca Seggiani}
\date{2024}

% imposta lo stile
% usa helvetica
\usepackage[scaled]{helvet}
% usa palatino
\usepackage{palatino}
% usa un font monospazio guardabile
\usepackage{lmodern}

\renewcommand{\rmdefault}{ppl}
\renewcommand{\sfdefault}{phv}
\renewcommand{\ttdefault}{lmtt}

% circuiti
\usepackage{circuitikz}
\usetikzlibrary{babel}

% disponi il titolo
\makeatletter
\renewcommand{\maketitle} {
	\begin{center} 
		\begin{minipage}[t]{.8\textwidth}
			\textsf{\huge\bfseries \@title} 
		\end{minipage}%
		\begin{minipage}[t]{.2\textwidth}
			\raggedleft \vspace{-1.65em}
			\textsf{\small \@author} \vfill
			\textsf{\small \@date}
		\end{minipage}
		\par
	\end{center}

	\thispagestyle{empty}
	\pagestyle{fancy}
}
\makeatother

% disponi teoremi
\usepackage{tcolorbox}
\newtcolorbox[auto counter, number within=section]{theorem}[2][]{%
	colback=blue!10, 
	colframe=blue!40!black, 
	sharp corners=northwest,
	fonttitle=\sffamily\bfseries, 
	title=Teorema~\thetcbcounter: #2, 
	#1
}

% disponi definizioni
\newtcolorbox[auto counter, number within=section]{definition}[2][]{%
	colback=red!10,
	colframe=red!40!black,
	sharp corners=northwest,
	fonttitle=\sffamily\bfseries,
	title=Definizione~\thetcbcounter: #2,
	#1
}

% disponi codice
\usepackage{listings}
\usepackage[table]{xcolor}

\definecolor{codegreen}{rgb}{0,0.6,0}
\definecolor{codegray}{rgb}{0.5,0.5,0.5}
\definecolor{codepurple}{rgb}{0.58,0,0.82}
\definecolor{backcolour}{rgb}{0.95,0.95,0.92}

\lstdefinestyle{codestyle}{
		backgroundcolor=\color{black!5}, 
		commentstyle=\color{codegreen},
		keywordstyle=\bfseries\color{magenta},
		numberstyle=\sffamily\tiny\color{black!60},
		stringstyle=\color{green!50!black},
		basicstyle=\ttfamily\footnotesize,
		breakatwhitespace=false,         
		breaklines=true,                 
		captionpos=b,                    
		keepspaces=true,                 
		numbers=left,                    
		numbersep=5pt,                  
		showspaces=false,                
		showstringspaces=false,
		showtabs=false,                  
		tabsize=2
}

\lstdefinestyle{shellstyle}{
		backgroundcolor=\color{black!5}, 
		basicstyle=\ttfamily\footnotesize\color{black}, 
		commentstyle=\color{black}, 
		keywordstyle=\color{black},
		numberstyle=\color{black!5},
		stringstyle=\color{black}, 
		showspaces=false,
		showstringspaces=false, 
		showtabs=false, 
		tabsize=2, 
		numbers=none, 
		breaklines=true
}


\lstdefinelanguage{assembler}{ 
  keywords={AAA, AAD, AAM, AAS, ADC, ADCB, ADCW, ADCL, ADD, ADDB, ADDW, ADDL, AND, ANDB, ANDW, ANDL,
        ARPL, BOUND, BSF, BSFL, BSFW, BSR, BSRL, BSRW, BSWAP, BT, BTC, BTCB, BTCW, BTCL, BTR, 
        BTRB, BTRW, BTRL, BTS, BTSB, BTSW, BTSL, CALL, CBW, CDQ, CLC, CLD, CLI, CLTS, CMC, CMP,
        CMPB, CMPW, CMPL, CMPS, CMPSB, CMPSD, CMPSW, CMPXCHG, CMPXCHGB, CMPXCHGW, CMPXCHGL,
        CMPXCHG8B, CPUID, CWDE, DAA, DAS, DEC, DECB, DECW, DECL, DIV, DIVB, DIVW, DIVL, ENTER,
        HLT, IDIV, IDIVB, IDIVW, IDIVL, IMUL, IMULB, IMULW, IMULL, IN, INB, INW, INL, INC, INCB,
        INCW, INCL, INS, INSB, INSD, INSW, INT, INT3, INTO, INVD, INVLPG, IRET, IRETD, JA, JAE,
        JB, JBE, JC, JCXZ, JE, JECXZ, JG, JGE, JL, JLE, JMP, JNA, JNAE, JNB, JNBE, JNC, JNE, JNG,
        JNGE, JNL, JNLE, JNO, JNP, JNS, JNZ, JO, JP, JPE, JPO, JS, JZ, LAHF, LAR, LCALL, LDS,
        LEA, LEAVE, LES, LFS, LGDT, LGS, LIDT, LMSW, LOCK, LODSB, LODSD, LODSW, LOOP, LOOPE,
        LOOPNE, LSL, LSS, LTR, MOV, MOVB, MOVW, MOVL, MOVSB, MOVSD, MOVSW, MOVSX, MOVSXB,
        MOVSXW, MOVSXL, MOVZX, MOVZXB, MOVZXW, MOVZXL, MUL, MULB, MULW, MULL, NEG, NEGB, NEGW,
        NEGL, NOP, NOT, NOTB, NOTW, NOTL, OR, ORB, ORW, ORL, OUT, OUTB, OUTW, OUTL, OUTSB, OUTSD,
        OUTSW, POP, POPL, POPW, POPB, POPA, POPAD, POPF, POPFD, PUSH, PUSHL, PUSHW, PUSHB, PUSHA, 
				PUSHAD, PUSHF, PUSHFD, RCL, RCLB, RCLW, MOVSL, MOVSB, MOVSW, STOSL, STOSB, STOSW, LODSB, LODSW,
				LODSL, INSB, INSW, INSL, OUTSB, OUTSL, OUTSW
        RCLL, RCR, RCRB, RCRW, RCRL, RDMSR, RDPMC, RDTSC, REP, REPE, REPNE, RET, ROL, ROLB, ROLW,
        ROLL, ROR, RORB, RORW, RORL, SAHF, SAL, SALB, SALW, SALL, SAR, SARB, SARW, SARL, SBB,
        SBBB, SBBW, SBBL, SCASB, SCASD, SCASW, SETA, SETAE, SETB, SETBE, SETC, SETE, SETG, SETGE,
        SETL, SETLE, SETNA, SETNAE, SETNB, SETNBE, SETNC, SETNE, SETNG, SETNGE, SETNL, SETNLE,
        SETNO, SETNP, SETNS, SETNZ, SETO, SETP, SETPE, SETPO, SETS, SETZ, SGDT, SHL, SHLB, SHLW,
        SHLL, SHLD, SHR, SHRB, SHRW, SHRL, SHRD, SIDT, SLDT, SMSW, STC, STD, STI, STOSB, STOSD,
        STOSW, STR, SUB, SUBB, SUBW, SUBL, TEST, TESTB, TESTW, TESTL, VERR, VERW, WAIT, WBINVD,
        XADD, XADDB, XADDW, XADDL, XCHG, XCHGB, XCHGW, XCHGL, XLAT, XLATB, XOR, XORB, XORW, XORL},
  keywordstyle=\color{blue}\bfseries,
  ndkeywordstyle=\color{darkgray}\bfseries,
  identifierstyle=\color{black},
  sensitive=false,
  comment=[l]{\#},
  morecomment=[s]{/*}{*/},
  commentstyle=\color{purple}\ttfamily,
  stringstyle=\color{red}\ttfamily,
  morestring=[b]',
  morestring=[b]"
}

\lstset{language=assembler, style=codestyle}

% disponi sezioni
\usepackage{titlesec}

\titleformat{\section}
	{\sffamily\Large\bfseries} 
	{\thesection}{1em}{} 
\titleformat{\subsection}
	{\sffamily\large\bfseries}   
	{\thesubsection}{1em}{} 
\titleformat{\subsubsection}
	{\sffamily\normalsize\bfseries} 
	{\thesubsubsection}{1em}{}

% tikz
\usepackage{tikz}

% float
\usepackage{float}

% grafici
\usepackage{pgfplots}
\pgfplotsset{width=10cm,compat=1.9}

% disponi alberi
\usepackage{forest}

\forestset{
	rectstyle/.style={
		for tree={rectangle,draw,font=\large\sffamily}
	},
	roundstyle/.style={
		for tree={circle,draw,font=\large}
	}
}

% disponi algoritmi
\usepackage{algorithm}
\usepackage{algorithmic}
\makeatletter
\renewcommand{\ALG@name}{Algoritmo}
\makeatother

% disponi numeri di pagina
\usepackage{fancyhdr}
\fancyhf{} 
\fancyfoot[L]{\sffamily{\thepage}}

\makeatletter
\fancyhead[L]{\raisebox{1ex}[0pt][0pt]{\sffamily{\@title \ \@date}}} 
\fancyhead[R]{\raisebox{1ex}[0pt][0pt]{\sffamily{\@author}}}
\makeatother

\begin{document}
% sezione (data)
\section{Lezione del 17-10-24}

% stili pagina
\thispagestyle{empty}
\pagestyle{fancy}

% testo
\subsection{Rappresentazione dei numeri naturali}
Riprendiamo l'argomento della rappresentazione dei numeri naturali (e in seguito anche degli interi) in modo da sintetizzare reti combinatorie che svolgano operazioni su tali insiemi numerici.

Noi esseri umani rappresentiamo i numeri naturali attraverso una notazione posizionale, ovvero come:
\begin{itemize}
	\item Un numero $\beta \geq 2$, detto \textbf{base di rappresentazione}. Nel caso del sistema decimale, $\beta = 10$.
	\item Un'insieme di $\beta$ simboli, detti \textbf{cifre}, a ciascuno dei quali è associato un numero naturale $\in [0, \beta - 1]$.
	\item Una \textbf{legge di rappresentazione} che fa corrispondere ad ogni sequenza di cifre un numero naturale.
\end{itemize}

\subsubsection{Notazione posizionale}
Dato un numero $A \in \mathbb{N}$, lo possiamo rappresentare in base $\beta$ attraverso una sequenza di cifre:
$$ A \equiv (a_{n-1}a_{n-2} ... a_1 a_0)_\beta, \quad 0 \leq a_i \leq \beta - 1, \quad 0 \leq i \leq n - 1 $$
dove la legge di rappresentazione è:
$$
A = \sum_{i=0}^{n-1} a_i \cdot \beta^i
$$

Nella rappresentazione di un numero naturale, una cifra contribuisce a determinare il numero in modo differente a seconda della propria posizione nella sequenza.

Normalmente usiamo il sistema decimale, con $\beta = 10$. Nell'informatica, ci interessiamo al sistema binario, con $\beta = 2$.

\subsection{Teoremi della divisione con resto}
Dimostriamo due teoremi che formulano effettivamente la divisione sugli insiemi dei naturali e degli interi, e che ci torneranno utili nella conversione fra basi diverse e in generale nell'aritmetica in base $\beta$. 
\subsubsection{Numeri naturali}
Nel caso naturale, il teorema è il seguente:
\begin{theorem}{della divisone con resto sui numeri naturali}
	Dato $x \in \mathbb{N}, \beta \in \mathbb{N}, \beta > 0$, esiste ed è unica la coppia di numeri $q,r$ con:
	\begin{itemize}
		\item $q \in \mathbb{N}$;
		\item $r \in \mathbb{N}, \ 0 \leq r < \beta$;
	\end{itemize}
	tale che $x = q \cdot \beta + r$.
\end{theorem}
$q$ prende il nome di \textbf{quoziente} e $r$ di \textbf{resto}.

\par\medskip
\noindent
\textbf{\textsf{Dimostrazione}} \\
Per avere una definizione effettiva di $q$ ed $r$ pensiamo di partizionare  $\mathbb{N}$ in intervalli:
$$
[n \cdot \beta, (n+1) \cdot \beta[ \ , \quad n \in \mathbb{N}
$$
Questi partiranno da $[0, \beta]$, e così via a coprire tutto $N$:
$$
\bigcup_{n \in \mathbb{N}} [n \cdot \beta, (n+1) \cdot \beta[ \ \equiv \mathbb{N}
$$

Potremo quindi usare queste partizioni per definire:
$$ \forall x \in \mathbb{N}, \beta \in \mathbb{N}, \quad \exists ! \ q : x \in [ q \cdot \beta, (q+1) \cdot \beta [ \ , \quad \exists ! \ r \ : r = x - q \cdot \beta $$
con le solite ipotesi $q \in \mathbb{N}$ e $r \in \mathbb{N}$ con $0 < r < \beta$.

Vogliamo dimostrare che (1) questo è sempre possibile e (2) $q$ e $r$ sono unici.
\begin{enumerate}
	\item Questo viene direttamente dal fatto che l'unione di tutte le partizioni dà $\mathbb{N}$ (come sopra), e per ogni partizione, visto che ciascuna di esse è di dimensione $\beta$, $r \in [0, \beta[$ basta a coprire tutti i naturali ivi compresi.
	\item Assumiamo per assurdo che esistano due possibili rappresentazioni in quoziente e resto, cioè:
		$$
		\exists q_1, r_1 \implies x = q_1 \cdot \beta + r_1
		$$
		$$
		\exists q_2, r_2 \implies x = q_2 \cdot \beta + r_2
		$$
		Sarà allora vero che:
		$$
		x = x \Rightarrow  q_1 \cdot \beta + r_1 = 	q_2 \cdot \beta + r_2, \quad (q_1 - q_2) \cdot \beta = r_2 - r_1 $$
		e dall'ipotesi $0 < r < \beta$:
		$$ -\beta < (r_2 - r_1) < \beta \Rightarrow -\beta < (q_1 - q_2) \cdot \beta < \beta $$
		Cioè dividendo per $\beta$:
		$$ -1 < (q_1 - q_2) < 1 $$
		che in $\mathbb{N}$ significa $q_1 - q_2 = 0$, cioè necessariamente $q_1 = q_2$. 
\end{enumerate}
$\hfill\square$

\subsubsection{Numeri interi}
Il teorema nel caso naturale non è valido così com'è nell'insieme dei numeri interi, in quanto esistono infinite coppie $q, r \in \mathbb{Z}$ tali per cui $x = q \cdot \beta + r$.
Si decide di adottare una regola sui segni di $q$ ed $r$, in particolare la seguente:
\begin{itemize}
	\item $q$ è positivo se $x$ e $\beta$ sono concordi in segno, negativo altrimenti. Nel caso di $\beta \geq 0$, come sarà per le basi che prenderemo, questo significa che $q$ è concorde a $x$;
	\item $r$ è sempre $\in [0,\beta[$, e inoltre è concorde a $x$. 
\end{itemize}

Queste restrizioni ci permetteranno di riformulare il teorema:
\begin{theorem}{della divisone con resto sui numeri naturali}
	Dato $x \in \mathbb{Z}, \beta \in \mathbb{Z}, \beta > 0$, esiste ed è unica la coppia di numeri $q,r$ con:
	\begin{itemize}
		\item $q \in \mathbb{Z}$, $
			\begin{cases}
				q \geq 0, \quad \text{se $x$ e $\beta$ concordi} \\
				q < 0, \quad \text{altrimenti}
			\end{cases}
		$
		\item $r \in \mathbb{N}, \ 0 \leq |r| < \beta$, $r$ concorde a $x$;
		\item $|x| = |q| \cdot |\beta| + |r|$
	\end{itemize}
	tale che $x = q \cdot \beta + r$.
\end{theorem}

\par\medskip
\noindent
\textbf{\textsf{Dimostrazione}} \\
La dimostrazione del teorema si basa sul mostrare, attraverso l'ausilio della funzione segno:
$$
s(x) = 
	\begin{cases}
			1, \quad x \geq 0 \\ 
			-1, \quad x < 0
	\end{cases}
$$
che il teorema con le condizioni riportate equivale al caso naturale già dimostrato.

Si ha quindi che:
$$
x = q \cdot \beta + r \Rightarrow s(x) |x| = s(q) |q| \cdot s(\beta) |\beta| + s(r) |r|
$$
Dalle ipotesi, $s(q) = s(x) \cdot s(\beta)$ e $s(r) = s(x)$, e quindi:
$$
s(x) |x| = s(x) s(\beta) |q| \cdot s(\beta) |\beta| + s(x) |r|
$$
il quale, notando che $s(\beta) \cdot s(\beta) = 1$ e dividendo per $s(x)$, diventa:
$$
|x| = |q| \cdot |\beta| + |r|
$$
Ergo, se le ipotesi sono soddisfatte, il teorema si riduce al caso naturale, e inoltre proprio per le ipotesi la rappresentazione quoziente resto è unica.
$\hfill\square$

\subsubsection{Interpretazione come divisione}
Notiamo che sia nel caso naturale che nel caso intero il teorema della divisione con resto ricalca la comune divisione naturale e intera con resto, cioè:
$$
q = \left\lfloor \frac{x}{\beta} \right\rfloor, \quad r = |x|_\beta, \quad x = q \cdot \beta + r = \left\lfloor \frac{x}{\beta} \right\rfloor \cdot \beta + |x|_\beta
$$

Inoltre se la divisione è tra naturali, anche $q$ è naturale, cioè la divisione con resto è \textbf{chiusa} sui naturali: $x \in \mathbb{N} \Rightarrow q \in \mathbb{N}$.

\subsubsection{Proprietà dell'operatore modulo}
Abbiamo usato l'operatore modulo ($|x|_\beta$). Vediamone alcune proprietà, dato $\alpha \in \mathbb{N}^+$:

\begin{enumerate}
	\item $| x + k \cdot \alpha |_\alpha = |x|_\alpha, \quad k \in \mathbb{Z} $ \\
	Questo da:
	$$
	x = \left\lfloor \frac{x}{\alpha} \right\rfloor \cdot \alpha + |x|_\alpha, \quad x + k \cdot \alpha = \left( \left\lfloor \frac{x}{\alpha} \right\rfloor  + k \right) \cdot \alpha + |x|_\alpha
	$$
	chiamiamo $x' = x + k \cdot \alpha$:
	$$
	x' = \left\lfloor \frac{x'}{\alpha} \right\rfloor \cdot \alpha + |x'|_\alpha = \left\lfloor \frac{x}{\alpha} + k \right\rfloor \cdot \alpha + |x'|_\alpha = \left( \left\lfloor \frac{x}{\alpha}  \right\rfloor + k \right) \cdot \alpha + |x'|_\alpha
	$$
	Dove il passaggio $\left\lfloor \frac{x}{\alpha} + k \right\rfloor = \left\lfloor \frac{x}{\alpha}  \right\rfloor + k $ è concesso da $ k \in \mathbb{Z}$.
	Notiamo che le ultime due espresioni ricavate si equivalgono, ergo dev'essere vero che $|x'|_\alpha = |x|_\alpha$, da cui la tesi.

\item $ |x  + y|_\alpha = \left| |x|_\alpha + |y|_\alpha \right|_\alpha $ \\ 
	Questo da:
	$$ 
	| x + y |_\alpha = \left| \left\lfloor \frac{x}{\alpha} \right\rfloor \cdot \alpha + |x|_\alpha + \left\lfloor \frac{y}{\alpha} \right\rfloor \cdot \alpha + |y|_\alpha \right|_\alpha = \left| \left( \left\lfloor \frac{x}{\alpha} \right\rfloor + \left\lfloor \frac{y}{\alpha} \right\rfloor \right) \cdot \alpha +  |x|_\alpha + |y|_\alpha \right|_\alpha 
	$$
	e l'applicazione della proprietà (1), da cui la tesi.

\item $ | x \cdot y |_\alpha = \left| |x|_\alpha \cdot |y|_\alpha \right|_\alpha $ \\ 
	Questo da:
	$$
| x \cdot y |_\alpha = \left| \left( \left\lfloor \frac{x}{\beta} \right\rfloor \cdot \beta + |x|_\beta \right) \left( \left\lfloor \frac{y}{\beta} \right\rfloor \cdot \beta + |y|_\beta \right)\right| 
	$$
	$$
 = \left| \left\lfloor \frac{x}{\alpha} \right\rfloor \left\lfloor \frac{y}{\alpha} \right\rfloor \alpha^2 + \left( \left\lfloor \frac{x}{\alpha} \right\rfloor |y|_\alpha +  \left\lfloor \frac{y}{\alpha} \right\rfloor |y|_\alpha\right) \alpha + |x|_\alpha \cdot |y|_\alpha \right| 
	$$
	Chiamiamo allora: 
	$$
	\left\lfloor \frac{x}{\alpha} \right\rfloor \left\lfloor \frac{y}{\alpha} \right\rfloor \alpha + \left( \left\lfloor \frac{x}{\alpha} \right\rfloor |y|_\alpha +  \left\lfloor \frac{y}{\alpha} \right\rfloor |y|_\alpha\right) = k
	$$
	da cui:
	$$
	|x \cdot y|_\alpha = | k \cdot \alpha + |x|_\alpha \cdot |y|_\alpha |_\alpha
	$$
	che ancora applicando la proprietà (1) dà la tesi.
\end{enumerate}

\subsubsection{Algoritmo delle divisioni successive}
Possiamo usare il teorema della divisione con resto iterativamente per trovare la sequenza di cifre che rappresentano $A$ in base $\beta$:
\begin{algorithm}[H]
\caption{delle divisioni successive}
\begin{algorithmic}
	\STATE \textbf{Input:} un naturale $A$ e una base $\beta$ % input
	\STATE \textbf{Output:} la rappresentazione di $A$ in base $\beta$ (in ordine inverso) % output
	% body
	\STATE$i \gets 1$
	\STATE$q_0 \gets A$
	\WHILE{$q_i \neq 0$}
		\STATE $q_{i-1} \gets \alpha_{i-1} + \beta \cdot q_i$
		\STATE $i \gets i + 1$
	\ENDWHILE
\end{algorithmic}
\end{algorithm}

Dimostriamo la correttezza dell'algoritmo: si ha che eseguendo i passaggi ricaviamo una forma:
$$ 
A = a_0 + \beta \cdot q_1 = a_0 + \beta \cdot (a_1 + \beta (a_2 + \beta \cdot (...))) = a_0 + a_1 \cdot \beta + a_2 \cdot \beta^2 + ...
$$
e quindi:
$$ 
A = \sum_{i=0}^{n-1} a_i \cdot \beta^i
$$
che è per definizione la rappresentazione di $A$ in base $\beta$.
Inoltre, il teorema della divisione con resto garantisce che la $n$-upla di cifre trovata è \textbf{unica}.

Questo algoritmo non è altro che la formalizzazione del DIV-MOD visto in precedenza.

\subsubsection{Rappresentazione su un numero finito di cifre}
Con $n$ cifre in base $\beta$, sappiamo che potremo formulare $\beta^n$ sequenze differenti quindi rappresentare al massimo il numero $\beta^n - 1$, cioè quello dove tutte le cifre hanno valore massimo, $\beta - 1$.

Ciò si dimostra da:
$$
A = \sum_{i=0}^{n-1} (\beta - 1) \cdot \beta^i = \sum_{i=0}^{n-1} \beta^{i + 1} - \sum_{i=0}^{n-1} \beta^i = \sum_{i=1}^{n} \beta^{i} - \sum_{i=0}^{n-1} \beta^i  = \beta^n - 1 $$

Il numero di cifre necessario per rappresentare $A$ è il numero minimo $n$ per cui $\beta^n - 1 \geq A$, ergo:
$$
n = \log_\beta (\beta^n) \geq \log_\beta (A + 1) \rightarrow n = \lceil \log_\beta (A + 1) \rceil
$$

\subsection{Reti combinatorie per i numeri naturali}
Vogliamo cosotruire reti logiche che elaborino numeri naturali rappresentati in una data base $\beta$, generalmente $\beta = 2$.
Si useranno \textbf{reti combinatorie}, dove lo \textit{stato di uscita} è il \textbf{risultato} e lo \textit{stato di ingresso} sono gli \textbf{operandi}.

Per ogni operazione aritmetica di base daremo una descrizione \textbf{indipendente dalla base}, usando le proprietà della notazione posizionale per scomporre l'operazione in blocchi elementari.
In seguito, dettaglieremo le reti logiche che implementano questi blocchi elementari in base 2, attraverso le porte logiche già studiate.

Notiamo che spesso ci concentreremo più sulle \textbf{cifre} che sulle codifiche, indipendentemente dalla base.

\subsubsection{Complemento}
Dato $A = (a_{n-1} a_{n-2} ... a_1 a_0)_\beta$, in base $\beta$ su $n$ cifre, $0 \leq A \leq \beta^n$, definisco complemento di $a$ in base $\beta$ il numero:
$$
\overline{A} = \beta^n - 1 - A
$$

Si ha che il complemento di un numero a $n$ cifre sta su $n$ cifre,  e che:
$$
\overline{A} = \beta^n - 1 - A = \sum_{i=0}^{n-1} (\beta - 1) \beta^i - \sum_{i=0}^{n-1} \alpha_i \beta^i = \sum_{i = 0}^{n-1} (\beta - 1 - a_i)\beta^i
$$

$\beta - 1 -\alpha_i$ è una cifra in base $\beta$ in quanto compresa fra $0$ e $\beta - 1$.
Quindi, $\overline{A} = (\overline{a_{n-1}} \overline{a_{n-2}} ... \overline{a_{1}} \overline{a_{0}})_\beta$.

Questo significa che basta saper fare il complemento di una singola cifra per fare il complemento di un numero.
In base 2, questo significa usare una porta NOT:

\lstinputlisting[language=verilog, style=codestyle]{../verilog/10-17/complementers/b2_complementer.v}

In altre basi diventà più complicato, ad esempio in base 10 con codifica BCD avrò un circuito con 4 ingressi e 4 uscite, con tabella di verità:

\begin{table}[H]
	\center 
	\begin{tabular} { c  c  c  c | c  c  c  c }
		$x_3$ & $x_2$ & $x_1$ & $x_0$ & $z_3$ & $z_2$ & $z_1$ & $z_0$ \\ 
		\hline
		$0$ & $0$ & $0$ & $0$ & $1$ & $0$ & $0$ & $1$ \\
		$0$ & $0$ & $0$ & $1$ & $1$ & $0$ & $0$ & $0$ \\
		$0$ & $0$ & $1$ & $0$ & $0$ & $1$ & $1$ & $1$ \\
		$0$ & $0$ & $1$ & $1$ & $0$ & $1$ & $1$ & $0$ \\
		$0$ & $1$ & $0$ & $0$ & $0$ & $1$ & $0$ & $1$ \\
		$0$ & $1$ & $0$ & $1$ & $0$ & $1$ & $0$ & $0$ \\
		$0$ & $1$ & $1$ & $0$ & $0$ & $0$ & $1$ & $1$ \\
		$0$ & $1$ & $1$ & $1$ & $0$ & $0$ & $1$ & $0$ \\
		$1$ & $0$ & $0$ & $0$ & $0$ & $0$ & $0$ & $1$ \\
		$1$ & $0$ & $0$ & $1$ & $0$ & $0$ & $0$ & $0$ \\
		$1$ & $0$ & $1$ & $0$ & $-$ & $-$ & $-$ & $-$ \\
		$1$ & $0$ & $1$ & $1$ & $-$ & $-$ & $-$ & $-$ \\
		$1$ & $1$ & $0$ & $0$ & $-$ & $-$ & $-$ & $-$ \\
		$1$ & $1$ & $0$ & $1$ & $-$ & $-$ & $-$ & $-$ \\
		$1$ & $1$ & $1$ & $0$ & $-$ & $-$ & $-$ & $-$ \\
		$1$ & $1$ & $1$ & $1$ & $-$ & $-$ & $-$ & $-$ \\
	\end{tabular}
\end{table}

che dopo una sintesi dà:
\[
	\begin{cases}
		z_3	= \overline{x_3}\overline{x_2}\overline{x_1} \\ 
		z_2 = \overline{x_3}\overline{x_2}x_1 + x_2\overline{x_1} \\ 
		z_1 = x_1 \\ 
		z_0 = \overline{x_0}
	\end{cases}
\]

e il seguente codice Verilog:

\lstinputlisting[language=verilog, style=codestyle]{../verilog/10-17/complementers/b10_complementer.v}

\end{document}


\documentclass[a4paper,11pt]{article}
\usepackage[a4paper, margin=8em]{geometry}

% usa i pacchetti per la scrittura in italiano
\usepackage[french,italian]{babel}
\usepackage[T1]{fontenc}
\usepackage[utf8]{inputenc}
\frenchspacing 

% usa i pacchetti per la formattazione matematica
\usepackage{amsmath, amssymb, amsthm, amsfonts}

% usa altri pacchetti
\usepackage{gensymb}
\usepackage{hyperref}
\usepackage{standalone}

\usepackage{colortbl}

\usepackage{xstring}
\usepackage{karnaugh-map}

% imposta il titolo
\title{Appunti Reti Logiche}
\author{Luca Seggiani}
\date{2024}

% imposta lo stile
% usa helvetica
\usepackage[scaled]{helvet}
% usa palatino
\usepackage{palatino}
% usa un font monospazio guardabile
\usepackage{lmodern}

\renewcommand{\rmdefault}{ppl}
\renewcommand{\sfdefault}{phv}
\renewcommand{\ttdefault}{lmtt}

% circuiti
\usepackage{circuitikz}
\usetikzlibrary{babel}

% disponi il titolo
\makeatletter
\renewcommand{\maketitle} {
	\begin{center} 
		\begin{minipage}[t]{.8\textwidth}
			\textsf{\huge\bfseries \@title} 
		\end{minipage}%
		\begin{minipage}[t]{.2\textwidth}
			\raggedleft \vspace{-1.65em}
			\textsf{\small \@author} \vfill
			\textsf{\small \@date}
		\end{minipage}
		\par
	\end{center}

	\thispagestyle{empty}
	\pagestyle{fancy}
}
\makeatother

% disponi teoremi
\usepackage{tcolorbox}
\newtcolorbox[auto counter, number within=section]{theorem}[2][]{%
	colback=blue!10, 
	colframe=blue!40!black, 
	sharp corners=northwest,
	fonttitle=\sffamily\bfseries, 
	title=Teorema~\thetcbcounter: #2, 
	#1
}

% disponi definizioni
\newtcolorbox[auto counter, number within=section]{definition}[2][]{%
	colback=red!10,
	colframe=red!40!black,
	sharp corners=northwest,
	fonttitle=\sffamily\bfseries,
	title=Definizione~\thetcbcounter: #2,
	#1
}

% disponi codice
\usepackage{listings}
\usepackage[table]{xcolor}

\definecolor{codegreen}{rgb}{0,0.6,0}
\definecolor{codegray}{rgb}{0.5,0.5,0.5}
\definecolor{codepurple}{rgb}{0.58,0,0.82}
\definecolor{backcolour}{rgb}{0.95,0.95,0.92}

\lstdefinestyle{codestyle}{
		backgroundcolor=\color{black!5}, 
		commentstyle=\color{codegreen},
		keywordstyle=\bfseries\color{magenta},
		numberstyle=\sffamily\tiny\color{black!60},
		stringstyle=\color{green!50!black},
		basicstyle=\ttfamily\footnotesize,
		breakatwhitespace=false,         
		breaklines=true,                 
		captionpos=b,                    
		keepspaces=true,                 
		numbers=left,                    
		numbersep=5pt,                  
		showspaces=false,                
		showstringspaces=false,
		showtabs=false,                  
		tabsize=2
}

\lstdefinestyle{shellstyle}{
		backgroundcolor=\color{black!5}, 
		basicstyle=\ttfamily\footnotesize\color{black}, 
		commentstyle=\color{black}, 
		keywordstyle=\color{black},
		numberstyle=\color{black!5},
		stringstyle=\color{black}, 
		showspaces=false,
		showstringspaces=false, 
		showtabs=false, 
		tabsize=2, 
		numbers=none, 
		breaklines=true
}


\lstdefinelanguage{assembler}{ 
  keywords={AAA, AAD, AAM, AAS, ADC, ADCB, ADCW, ADCL, ADD, ADDB, ADDW, ADDL, AND, ANDB, ANDW, ANDL,
        ARPL, BOUND, BSF, BSFL, BSFW, BSR, BSRL, BSRW, BSWAP, BT, BTC, BTCB, BTCW, BTCL, BTR, 
        BTRB, BTRW, BTRL, BTS, BTSB, BTSW, BTSL, CALL, CBW, CDQ, CLC, CLD, CLI, CLTS, CMC, CMP,
        CMPB, CMPW, CMPL, CMPS, CMPSB, CMPSD, CMPSW, CMPXCHG, CMPXCHGB, CMPXCHGW, CMPXCHGL,
        CMPXCHG8B, CPUID, CWDE, DAA, DAS, DEC, DECB, DECW, DECL, DIV, DIVB, DIVW, DIVL, ENTER,
        HLT, IDIV, IDIVB, IDIVW, IDIVL, IMUL, IMULB, IMULW, IMULL, IN, INB, INW, INL, INC, INCB,
        INCW, INCL, INS, INSB, INSD, INSW, INT, INT3, INTO, INVD, INVLPG, IRET, IRETD, JA, JAE,
        JB, JBE, JC, JCXZ, JE, JECXZ, JG, JGE, JL, JLE, JMP, JNA, JNAE, JNB, JNBE, JNC, JNE, JNG,
        JNGE, JNL, JNLE, JNO, JNP, JNS, JNZ, JO, JP, JPE, JPO, JS, JZ, LAHF, LAR, LCALL, LDS,
        LEA, LEAVE, LES, LFS, LGDT, LGS, LIDT, LMSW, LOCK, LODSB, LODSD, LODSW, LOOP, LOOPE,
        LOOPNE, LSL, LSS, LTR, MOV, MOVB, MOVW, MOVL, MOVSB, MOVSD, MOVSW, MOVSX, MOVSXB,
        MOVSXW, MOVSXL, MOVZX, MOVZXB, MOVZXW, MOVZXL, MUL, MULB, MULW, MULL, NEG, NEGB, NEGW,
        NEGL, NOP, NOT, NOTB, NOTW, NOTL, OR, ORB, ORW, ORL, OUT, OUTB, OUTW, OUTL, OUTSB, OUTSD,
        OUTSW, POP, POPL, POPW, POPB, POPA, POPAD, POPF, POPFD, PUSH, PUSHL, PUSHW, PUSHB, PUSHA, 
				PUSHAD, PUSHF, PUSHFD, RCL, RCLB, RCLW, MOVSL, MOVSB, MOVSW, STOSL, STOSB, STOSW, LODSB, LODSW,
				LODSL, INSB, INSW, INSL, OUTSB, OUTSL, OUTSW
        RCLL, RCR, RCRB, RCRW, RCRL, RDMSR, RDPMC, RDTSC, REP, REPE, REPNE, RET, ROL, ROLB, ROLW,
        ROLL, ROR, RORB, RORW, RORL, SAHF, SAL, SALB, SALW, SALL, SAR, SARB, SARW, SARL, SBB,
        SBBB, SBBW, SBBL, SCASB, SCASD, SCASW, SETA, SETAE, SETB, SETBE, SETC, SETE, SETG, SETGE,
        SETL, SETLE, SETNA, SETNAE, SETNB, SETNBE, SETNC, SETNE, SETNG, SETNGE, SETNL, SETNLE,
        SETNO, SETNP, SETNS, SETNZ, SETO, SETP, SETPE, SETPO, SETS, SETZ, SGDT, SHL, SHLB, SHLW,
        SHLL, SHLD, SHR, SHRB, SHRW, SHRL, SHRD, SIDT, SLDT, SMSW, STC, STD, STI, STOSB, STOSD,
        STOSW, STR, SUB, SUBB, SUBW, SUBL, TEST, TESTB, TESTW, TESTL, VERR, VERW, WAIT, WBINVD,
        XADD, XADDB, XADDW, XADDL, XCHG, XCHGB, XCHGW, XCHGL, XLAT, XLATB, XOR, XORB, XORW, XORL},
  keywordstyle=\color{blue}\bfseries,
  ndkeywordstyle=\color{darkgray}\bfseries,
  identifierstyle=\color{black},
  sensitive=false,
  comment=[l]{\#},
  morecomment=[s]{/*}{*/},
  commentstyle=\color{purple}\ttfamily,
  stringstyle=\color{red}\ttfamily,
  morestring=[b]',
  morestring=[b]"
}

\lstset{language=verilog, style=codestyle}

% disponi sezioni
\usepackage{titlesec}

\titleformat{\section}
	{\sffamily\Large\bfseries} 
	{\thesection}{1em}{} 
\titleformat{\subsection}
	{\sffamily\large\bfseries}   
	{\thesubsection}{1em}{} 
\titleformat{\subsubsection}
	{\sffamily\normalsize\bfseries} 
	{\thesubsubsection}{1em}{}

% tikz
\usepackage{tikz}

% float
\usepackage{float}

% grafici
\usepackage{pgfplots}
\pgfplotsset{width=10cm,compat=1.9}

% disponi alberi
\usepackage{forest}

\forestset{
	rectstyle/.style={
		for tree={rectangle,draw,font=\large\sffamily}
	},
	roundstyle/.style={
		for tree={circle,draw,font=\large}
	}
}

% disponi algoritmi
\usepackage{algorithm}
\usepackage{algorithmic}
\makeatletter
\renewcommand{\ALG@name}{Algoritmo}
\makeatother

% disponi numeri di pagina
\usepackage{fancyhdr}
\fancyhf{} 
\fancyfoot[L]{\sffamily{\thepage}}

\makeatletter
\fancyhead[L]{\raisebox{1ex}[0pt][0pt]{\sffamily{\@title \ \@date}}} 
\fancyhead[R]{\raisebox{1ex}[0pt][0pt]{\sffamily{\@author}}}
\makeatother

\begin{document}
% sezione (data)
\section{Lezione del 22-10-24}

% stili pagina
\thispagestyle{empty}
\pagestyle{fancy}

% testo
\subsection{Operazioni a costo nullo}
\subsubsection{Moltiplicazioni e divisioni per potenze di base}
Moltiplicare e dividere per potenze della base $\beta$ significa semplicemente aggiungere o togliere zeri, ergo si tratta di operazioni a \textbf{costo nullo}.
Se le operazioni sono a costo nullo, è molto probabile che le reti che le implementano siano \textbf{prive di logica}.

\begin{itemize}
	\item \textbf{Moltiplicazione:} effettivamente, la rete che implementa una moltiplicazione per $\beta$ sposta gli input $x_{n - 1}, ..., x_0$ "su", attraverso una mappa:
\begin{center}
	\begin{tikzpicture}
		\node at(0,0) {$x_0$};
		\node at(0,0.5) {$x_1$};
		\node at(0,1) {$...$};
		\node at(0,1.5) {$x_{n-2}$};
		\node at(0,2) {$x_{n-1}$};

		\draw[->] (0.5,0) -- (2.5,0.5);
		\draw[->] (0.5,0.5) -- (2.5,1);
		\draw[->] (0.5,1.5) -- (2.5,2);
		\draw[->] (0.5,2) -- (2.5,2.5);

		\node at(3,0) {$0$};
		\node at(3,0.5) {$y_1$};
		\node at(3,1) {$y_2$};
		\node at(3,1.5) {$...$};
		\node at(3,2) {$y_{n-1}$};
		\node at(3,2.5) {$y_n$};
	\end{tikzpicture}
\end{center}
assegnando $y_0$ ad un generatore di zero.

Per reti che moltiplicano per multipli $\beta \cdot k$, generalizzeremo la stessa cosa come:
\[
	\begin{cases}
		y_j = x_{j - k}, \quad k \leq j \leq n - 1 + k \\ 
		y_j = 0, \quad 0 \leq j \leq k-1 \\ 
	\end{cases}
\]
ottenendo quindi la mappa:
\begin{center}
	\begin{tikzpicture}
		\node at(0,0) {$x_0$};
		\node at(0,0.5) {$x_1$};
		\node at(0,1) {$...$};
		\node at(0,1.5) {$x_{n-2}$};
		\node at(0,2) {$x_{n-1}$};

		\draw[->] (0.5,0) -- (2.5,1.5);
		\draw[->] (0.5,0.5) -- (2.5,2);
		\draw[->] (0.5,1.5) -- (2.5,3);
		\draw[->] (0.5,2) -- (2.5,3.5);

    \draw[decorate,decoration={brace,amplitude=10pt,mirror}] (3.5,0) -- (3.5, 1)
        node[midway,right=10pt]{$k$};

		\node at(3,0) {$0$};
		\node at(3,0.5) {$...$};
		\node at(3,1) {$0$};
		\node at(3,1.5) {$y_k$};
		\node at(3,2) {$y_{k+1}$};
		\node at(3,2.5) {$...$};
		\node at(3.25,3) {$y_{n-2+k}$};
		\node at(3.25,3.5) {$y_{n-1+k}$};
	\end{tikzpicture}
\end{center}

\item \textbf{Quoziente:} allo stesso modo, si ha che si vogliono spostare gli input "giù", ovvero applicare:
\[
	\begin{cases}
y_j = x_{j + k}, \quad k \leq j \leq n - 1 -k \\ 
	\end{cases}
\]
che per $k=1$ è:
\begin{center}
	\begin{tikzpicture}
		\node at(0,0) {$x_0$};
		\node at(0,0.5) {$x_1$};
		\node at(0,1) {$x_2$};
		\node at(0,1.5) {$...$};
		\node at(0,2) {$x_{n-2}$};
		\node at(0,2.5) {$x_{n-1}$};

		\draw[->] (0.5,0.5) -- (2.5,0);
		\draw[->] (0.5,1) -- (2.5,0.5);
		\draw[->] (0.5,2) -- (2.5,1.5);
		\draw[->] (0.5,2.5) -- (2.5,2);

		\node at(3,0) {$y_0$};
		\node at(3,0.5) {$y_1$};
		\node at(3,1) {$...$};
		\node at(3,1.5) {$y_{n-3}$};
		\node at(3,2) {$y_{n-2}$};
	\end{tikzpicture}
\end{center}
e per $k$ arbitrari è:
\begin{center}
	\begin{tikzpicture}
		\node at(0,0) {$x_0$};
		\node at(0,0.5) {$...$};
		\node at(0,1) {$x_{k-1}$};
		\node at(0,1.5) {$x_k$};
		\node at(0,2) {$x_{k+1}$};
		\node at(0,2.5) {$...$};
		\node at(0,3) {$x_{n-2}$};
		\node at(0,3.5) {$x_{n-1}$};

		\draw[->] (0.5,1.5) -- (2.5,0);
		\draw[->] (0.5,2) -- (2.5,0.5);
		\draw[->] (0.5,3) -- (2.5,1.5);
		\draw[->] (0.5,3.5) -- (2.5,2);

    \draw[decorate,decoration={brace,amplitude=10pt,mirror}] (-0.5,1) -- (-0.5,0)
        node[midway,left=10pt]{$k$};

		\node at(3,0) {$y_0$};
		\node at(3,0.5) {$y_1$};
		\node at(3,1) {$...$};
		\node at(3.25,1.5) {$y_{n-2-k}$};
		\node at(3.25,2) {$y_{n-1-k}$};
	\end{tikzpicture}
\end{center}
dove i primi $k$ elementi di $x$ vengono trascurati (\textbf{troncamento}).

\item \textbf{Resto:} il resto significa semplicemente "tagliare" tutti gli ingressi prima di $x_k$, ergo:
	\[
		\begin{cases}
			y_j = x_j, \quad 0 \leq j \leq k- 1	
		\end{cases}
	\]
\end{itemize}
secondo la mappa:
\begin{center}
	\begin{tikzpicture}
		\node at(0,0) {$x_0$};
		\node at(0,0.5) {$x_1$};
		\node at(0,1) {$...$};
		\node at(0,1.5) {$x_{k-2}$};
		\node at(0,2) {$x_{k-1}$};
		\node at(0,2.5) {$...$};
		\node at(0,3) {$x_{n-1}$};

		\draw[->] (0.5,0) -- (2.5,0);
		\draw[->] (0.5,0.5) -- (2.5,0.5);
		\draw[->] (0.5,1.5) -- (2.5,1.5);
		\draw[->] (0.5,2) -- (2.5,2);

		\node at(3,0) {$y_0$};
		\node at(3,0.5) {$y_1$};
		\node at(3,1) {$...$};
		\node at(3,1.5) {$y_{k-2}$};
		\node at(3,2) {$y_{k-1}$};
	\end{tikzpicture}
\end{center}

\subsubsection{Concatenamento}
Concatenare in $X$ due numeri $Y$ e $Z$ a $k$ e $n-k$ cifre significa dire:
$$ 
X = Z \cdot \beta^k + Y
$$

Anche questa è un'operazione a complessità nulla, in quanto significa prendere le cifre di $Y$ e $Z$:
\[
	\begin{cases}
		x_j = y_j, \quad 0 \leq j \leq k - 1\\ 	
		x_j = z_j, \quad k \leq j \leq n - 1 \\ 	
	\end{cases}
\]
secondo la mappa:
\begin{center}
	\begin{tikzpicture}
		\node at(0,0) {$y_0$};
		\node at(0,0.5) {$...$};
		\node at(0,1) {$y_{k-1}$};
		\node at(0,1.5) {$z_0$};
		\node at(0,2) {$...$};
		\node at(0,2.5) {$z_{n-k-1}$};

		\draw[->] (0.5,0) -- (2.5,0);
		\draw[->] (0.5,1) -- (2.5,1);
		\draw[->] (0.5,1.5) -- (2.5,1.5);
		\draw[->] (0.75,2.5) -- (2.5,2.5);

    \draw[decorate,decoration={brace,amplitude=10pt,mirror}] (-0.5,1) -- (-0.5,0)
        node[midway,left=10pt]{$y$};

    \draw[decorate,decoration={brace,amplitude=10pt,mirror}] (-0.75,2.5) -- (-0.75,1.5)
        node[midway,left=10pt]{$z$};
		
		\node at(3,0) {$y_0$};
		\node at(3,0.5) {$...$};
		\node at(3,1) {$y_{k-1}$};
		\node at(3,1.5) {$y_k$};
		\node at(3,2) {$...$};
		\node at(3,2.5) {$y_n$};

	\end{tikzpicture}
\end{center}

\subsubsection{Estensione di campo}
L'estensione di campo è l'operazione con cui rappresentiamo un naturale su $n$ cifre su un numero maggiore di cifre.
Per i naturali dobbiamo trivialmente aggiungere zero a sinistra della MSD, mentre vedremo che per l'aritmetica intera dovremmo replicare la MSD sulle cifre aggiunte per mantenere il segno corretto.

Abbiamo quindi che per un numero $x = (x_{n-1}, ..., x_0)$ su $n$ cifre vogliamo trovare l'esteso $x' = (x_{n-1+k}, ..., x_0)$ su $n + k$ cifre, cioè il numero tale per cui:
\[
	\begin{cases}
		x'_j = x_j, \quad 0 \leq j \leq n - 1 \\ 
		x_j = 0, \quad n \leq j \leq n - 1 + k
	\end{cases}
\]
cioè che rispetta la mappa:
\begin{center}
	\begin{tikzpicture}
		\node at(0,0) {$x_0$};
		\node at(0,0.5) {$x_1$};
		\node at(0,1) {$...$};
		\node at(0,1.5) {$x_{n-2}$};
		\node at(0,2) {$x_{n-1}$};

		\draw[->] (0.5,0) -- (2.5,0);
		\draw[->] (0.5,0.5) -- (2.5,0.5);
		\draw[->] (0.5,1.5) -- (2.5,1.5);
		\draw[->] (0.5,2) -- (2.5,2);

    \draw[decorate,decoration={brace,amplitude=10pt,mirror}] (3.75,2.5) -- (3.75,3.5)
        node[midway,right=10pt]{$k$};

		\node at(3,0) {$y_0$};
		\node at(3,0.5) {$y_1$};
		\node at(3,1) {$...$};
		\node at(3,1.5) {$y_{n-2}$};
		\node at(3,2) {$y_{n-1}$};
		\node at(3,2.5) {$0_n$};
		\node at(3,3) {$...$};
		\node at(3,3.5) {$0_{n-1+k}$};
	\end{tikzpicture}
\end{center}

\subsection{Addizione}
La somma, sostanzialmente, consiste nel:
\begin{enumerate}
	\item Sommare le coppie di cifre di pari posizione, singolarmente, dalla LSD alla MSD e tenendo conto dell'eventuale \textbf{riporto entrante};
	\item Se la somma di cifre non è rappresentabile su una singola cifra, usare il \textbf{riporto uscente} per la coppia di cifre successive.
\end{enumerate}
Abbiamo che il riporto è sempre $\in \{0, 1\}$, e che per la prima coppia di cifre possiamo assumerlo $= 0$.
Ad ogni passaggio, quindi, applichiamo una funzione:
$$
(a_i, b_i, c_{in}) \rightarrow (s_i, c_{out})
$$

Inoltre, si ha che l'algoritmo non dipende dalla base $\beta$, ma solamente dalla \textbf{notazione posizionale}.

\subsubsection{Dimensioni di somme}
Avevamo quindi che, dati $X$, $Y$ in base $\beta$ su $n$ cifre,  cioè $X, Y \in [0, \beta^n - 1$], con $C_{in} \in [0, 1]$, volevamo calcolare:
$$ Z = X + Y + C_{in} $$
ovvero trovare il cosiddetto \textbf{full adder}.
Possiamo dimostrare che il numero di cifre su cui sta il risultato è:
$$ 0 \leq X + Y + C_{in} \leq 2\beta^n - 1 \leq \beta^{n+1} - 1 $$
dove la cifra $n+1$ è compresa in $Z_{n+1} \in [0, 1]$, cioè rappresenta il riporto uscente di $X + Y$. 

\par\smallskip

Possiamo quindi affermare con sicurezza che la somma fra due naturali espressi in base $\beta$ su $n$ cifre più un'eventuale riporto entrante $C_{in}$ produce un naturale che è sempre rappresentabile su $n+1$ cifre in base $\beta$, delle quali la $n+1$-esima cifra è il riporto uscente, e può valere soltanto $0$ o $1$.

Quello che vogliamo è un circuito sommatore in base $\beta$ a $n$ cifre che prenda le cifre di due naturali $X$ e $Y$ su $n$ cifre e un riporto entrante $C_{in}$ (un bit), e restituisca un'altro naturale $Z$, sempre su $n$ cifre e un riporto uscente $C_{out}$ (sempre un bit). 

Nel caso uno dei numeri abbia $m > n$ cifre, si estende il numero su $n$ cifre fino a $m$ (aggiungendo $n - m$ zeri in testa), e poi si somma.
Se si vuole poi che la somma sia \textit{sempre} rappresentabile, bisogna usare un sommatore ad $n+1$ cifre, ed estenedere gli ingressi su $n+1$ cifre.
In questo caso l'ultimo riporto sarà sempre zero.

\subsubsection{Ripple carry e full adder}
Creare circuiti per $2n + 1$ ingressi può essere complicato, quindi si preferisce adottare un approccio \textbf{modulare}, dove si scompone ogni somma su una singola coppia di cifre, purchè:
\begin{itemize}
	\item Le somme vengano eseguite dalla LSD alla MSD;
	\item Il rapporto si \textbf{propaghi} (in inglese \textit{ripple}) da una cifra alla successiva.
\end{itemize}
Chiamiamo quindi ogni sommatore su due cifre (una di $X$ e una di $Y$) \textbf{full adder}, e il montaggio in cui li disponiamo a \textbf{ripple carry} (\textit{propagazione dei resti}).

\subsubsection{Full adder in base 2}
In base 2, un full adder è un circuito con 3 ingressi ($x_i$, $y_i$ e $c_{in}$) e 2 uscite ($s_i$ e $c_{out}$).
Abbiamo che la rete dovrebbe avere tabella di verità:

\begin{table}[H]
	\center 
	\begin{tabular} { c  c  c | c  c }
		$x_i$ & $y_i$ & $c_{in}$ & $s_{i}$ & $c_{out}$ \\ 
		\hline
		0 & 0 & 0 & 0 & 0 \\ 
		0 & 0 & 1 & 1 & 0 \\ 
		0 & 1 & 0 & 1 & 0 \\ 
		0 & 1 & 1 & 0 & 1 \\ 
		1 & 0 & 0 & 1 & 0 \\ 
		1 & 0 & 1 & 0 & 1 \\ 
		1 & 1 & 0 & 0 & 1 \\ 
		1 & 1 & 1 & 1 & 1 
	\end{tabular}
\end{table}

Potremmo adesso applicare Karnaugh, ma notiamo che $s_i$ vale 1 quando la somma degli ingressi è dispari, cioè si può dire che è lo XOR in cascata di $x_i$, $y_i$ e $c_{in}$. 
Allo stesso modo, il $c_{out}$ non sarà altro che un circuito SP standard, che prende gli AND di ogni coppia di ingressi e li passa attraverso un OR.
Abbiamo che il full adder è una rete a 2 livelli di logica.

Si ha allora l'implementazione in Verilog del full adder in base 2:

\lstinputlisting[language=verilog, style=codestyle]{../verilog/10-22/adders/b2_adder.v}

A questo punto si potranno concatenare, ad esempio, 4 full adder per creare un adder a più bit:

\lstinputlisting[language=verilog, style=codestyle]{../verilog/10-22/adders/n4_b2_adder.v}

\subsubsection{Incrementatore}
Abbiamo che in assembler potevamo distinguere fra le operazioni \lstinline|ADD $1, %al| e \lstinline|INC %al|.
Possiamo fare l'assunzione che almeno uno degli ingressi di un full adder sia sempre zero per realizzare un \textbf{half adder} o \textit{incrementatore}: ad esempio, prendiamo $y_i = 0$. 
In questo caso, $c_{out}$ potrà essere prodotto con un solo livello di logica, cioè attraverso l'AND fra $x_i$ e $c_{in}$.
Allo stesso modo, potremo ridurre $s_i$ ad un solo XOR fra $x_i$ e $c_{in}$.
Riportiamo una sintesi a tabella di verità e a porte logiche in Verilog:

\lstinputlisting[language=verilog, style=codestyle]{../verilog/10-22/halfadders/b2_halfadder.v}

Analogamente, potremmo pensare di sintetizzare \textbf{incrementatori} (stavolta nel vero senso della parola, e non \textit{half adder}, con cui corrispondevano in base 2 (detto questo, si sono comunque chiamati impropriamente i moduli Verilog \lstinline|b3_halfadder| e \lstinline|b10_halfadder| per congruenza con le altre definizioni)) in base 3: 

\lstinputlisting[language=verilog, style=codestyle]{../verilog/10-22/halfadders/b3_halfadder.v}

e in base 10: 

\lstinputlisting[language=verilog, style=codestyle]{../verilog/10-22/halfadders/b10_halfadder.v}

Le sintesi SP (con anche gate XOR aggiunti a discrezione) si ricavano direttamente dagli statement \lstinline|assign| delle descrizioni riportate.

\subsubsection{Parallelizzazione della somma}
Facciamo delle considerazioni sulle prestazioni: se in un full adder ogni input arriva in tempo $t$, dopo 2 livelli di logica il $c_{in}$ del prossimo full adder arriverà a $t + 2$.
Quindi il risultato di quel full adder uscirà a $t + 4$ e così via.
Si ha che per $n$ full adder concatenati, ergo $n$ cifre, l'$n-1$-esima cifra viene computata in tempo $t + 2n$.
Questo ci dice che la somma è si \textbf{scomponibile}, ma non \textbf{parallelizzabile}.

In verità, negli anni, sono state sviluppate architetture che implementano il \textbf{carry lookahead}, cioè implementano su due livelli di logica, con 5 ingressi, un "precalcolo" del carry a qualche $t + 4$, cioè ogni due full adder (i 5 ingressi sono il primo carry e i 2 + 2 ingressi dei 2 full adder).
Questo pressapoco raddoppia la velocità di calcolo delle somme su $n$ cifre, passando quindi da $t + 2n$ a $\approx t + n$.

Possiamo ricavare le formule di precalcolo prendendo l'implementazione di un adder data prima.
Avevamo che il bit di riporto di un full adder che calcolava $x + y$ era dato da:
$$
C_i = G_i + P_i C_{in}
$$
dove $G_i = x_i \cdot y_i$ e $P_i = x_i \oplus y_i$ vengono detti \textbf{generazione} e \textbf{propagazione} di carry alla cifra $i$. Intuitivamente, infatti, questi rappresentano se un riporto viene \textit{generato} o \textit{propagato} alla cifra $i$.

Avremo allora le formule estese, facendo le dovute sostituzioni:
$$
C_0 = G_0 + P_0 \cdot C_{in}
$$
$$
C_1 = G_1 + P_1 \cdot G_0 + P_1 \cdot P_0 \cdot C_{in}
$$
$$
C_2 = G_2 + P_2 \cdot G_1 + P_2 \cdot P_1 \cdot G_0 + P_2 \cdot P_1 \cdot P_0 \cdot C_{in}
$$
$$
C_3 = G_3 + P_3 \cdot G_2 + P_3 \cdot P_2 \cdot G_1 + P_3 \cdot P_2 \cdot P_1 \cdot G_0 + P_3 \cdot P_2 \cdot P_1 \cdot P_0 \cdot C_{in}
$$

Questi riporti verranno propagati ai corrispettivi full adder, che non dovranno più aspettare la propagazione sul lato sinistro da parte dei full adder precedenti.
Un esempio di un adder a 4 bit dotato di CLA implementato in Verilog è il seguente:

\lstinputlisting[language=verilog, style=codestyle]{../verilog/10-22/adders/n4_b2_cla_adder.v}

\subsection{Sottrazione}
L'algoritmo di sottrazione consiste nell'applicare un'algoritmo analogo alla somma ma con prestiti al contrario, cioè nel:
\begin{enumerate}
	\item Sotrarre le coppie di cifre di pari posizione, singolarmente, dalla LSD alla MSD;
	\item Se la somma di cifre non è rappresentabile su una singola cifra, generare un \textbf{prestito} (\textit{borrow}) per la coppia di cifre successive.
\end{enumerate}

Si ha anche qui che il prestito è sempre $\in [0, 1]$.
Inoltre, anche questo algoritmo non dipende dalla base $\beta$, ma solo dalla notazione posizionale.

\subsubsection{Dimensioni di sottrazioni}
Abbiamo quindi due naturali $X$ e $Y$ in base $\beta$ su $n$ cifre, quindi tali che $X, Y \in [0, \beta^n - 1]$, e un bit $B_{in}$ con $0 \leq b_{in} \leq 1$.
Voglio calcolare il naturale:
$$
Z = X - Y - b_{in}
$$
ammesso che questo naturale \textit{esista!}.
Questo perché i naturali non sono chiusi rispetto ala sottrazione, cioè:
$$ - \beta^n \leq X-Y-b_{in} \leq \beta^n -1$$
potrei avere $Z \in \mathbb{Z}$.

\subsubsection{Rappresentabilità}
Dico quindi che, dal teorema della divisione con resto, posso scrivere $Z$ come quoziente e resto di una divisione per $\beta^n$:
$$ Z = -b_{out} \cdot \beta^n + D = X - Y - b_{in} $$
definito:
$$
-b_{out} = \left\lfloor \frac{X-Y-b_{in}}{\beta^n} \right\rfloor, \quad D = |X-Y-b_{in}|_{\beta^n}
$$
dove noto che $b_{out} \in \{0 ,1\}$ indipendentemente da $\beta$ (si comporta come il \textit{carry} della somma). 

Posso quindi scrivere $Y$ come il suo complemento, noto che:
$$ 
Y + \overline{Y} = \beta^n - 1, \quad Y = \beta^n - 1 - \overline{Y}
$$
da cui sostituendo:
$$
(1 - b_{out}) \cdot \beta^n + D = X + \overline{Y} + (1-b_{in}) \equiv \overline{b_{out}} \cdot \beta^n + D = X + \overline{Y} + \overline{b_{in}}
$$
dove si complementano i bit $b_{in}$ e $b_{out}$.
Chiamiamo:
\[
	\begin{cases}
		\overline{b_{out}} = c_{out} \\ 
		\overline{b_{in}} = c_{in}
	\end{cases}
\]

Otteniamo che l'equazione finale è sostanzialmente quella di un sommatore:
$$ 
\overline{b_{out}} \cdot \beta^n + D = X + \overline{Y} + \overline{b_{in}} \equiv c_{out} \cdot \beta^n + D = X + \overline{Y} + c_{in}
$$
dove la differenza fra $X$ e $Y$ meno un prestito entrante, se naturale, può essere ottenuta se sommo $X$ ad $\overline{Y}$, più un'eventuale riporto entrante ottenuto complementando il prestito entrante.
Se a questo punto il riporto uscente di $\overline{b_{out}}$ vale 1, si ha che la differenza è un naturale pari a $D$, altrimenti non è rappresentabile.

\subsubsection{Comparazione di numeri naturali}
Dati due \textbf{naturali} $X$ e $Y$, si possono usare i sottrattori per comparare i loro valori, cioè per ottenere $x < Y$.
Per fare ciò, si calcola $X - Y$ e si guarda il prestito uscente: se $b_{out} = 1$, allora $X<Y$, altrimenti viceversa.

Per controllare l'uguaglianza, invece, si prende $b_{out}$ e $D$: se $b_{out} = 1$ (differenza rappresentabile) e $D=0$, allora $X = Y$, altrimenti viceversa.

\subsection{Moltiplicazione}
Dati $X$ e $C$ naturali in base $\beta$ su $n$ cifre, cioè $X,C \in [0, \beta^n - 1]$, e $Y$ naturale in base $\beta$ su $m$ cifre, cioè $Y \in [0, \beta^m - 1]$, vogliamo calcolare:
$$
P = X \cdot Y + C 
$$

\subsubsection{Dimensioni di prodotti}
Si ha che, da quanto detto prima:
$$
P = X \cdot Y + C \leq (\beta^n - 1) \cdot (\beta^m - 1) + (\beta^n - 1) = \beta^m \cdot (\beta^n - 1) < \beta^{n+m} - 1 
$$
cioè il risultato sta su $n+m$ cifre.

\subsubsection{Algoritmo di moltiplicazione}
La moltiplicazione fra naturali si effettua come segue:

\begin{enumerate}
	\item Si motliplica $X$ per tutte le cifre di $Y$, iterativamente;
	\item Moltiplicando, si generano \textbf{risultati parziali}, che vengono disposti a partire dalla cifra per cui stiamo moltiplicando, per quanto ci riguarda si tratta di una moltiplicazione per $\beta^k$;
	\item I risultati parziali vengono sommati fra di loro con riporto.
\end{enumerate}

Diverse architetture implementano diversi algoritmi di moltiplicazione, ma l'idea fondamentale è quella di creare risultati parziali e sommarli fra di loro.
Un modo particolarmente efficiente di fare moltiplicazioni è quello di:
\begin{enumerate}
	\item Moltiplicare un numero ad $n$ cifre per un numero ad una sola cifra;
	\item Sommare gli $m$ addendi, opportunamente traslati, per ottenere il risultato finale.
\end{enumerate}

Possiamo sfruttare il fatto che la somma è \textbf{associativa}, e che la cifra $i$-esima del prodotto, con $0 \leq i \leq n - 1$, è determinata univocamente dai prodotti parziali $j \leq i$, ergo possiamo sommare i risultati parziali mentre si svolgono le moltiplicazioni. 
Quest'ultima differenza è la più sostanziale dalla classica moltiplicazione \textit{"in colonna"} insegnata a scuola.

Si va quindi a definire una rete detta \textbf{moltiplicatore con addizionatore}, che:
\begin{enumerate}
	\item Moltiplica $X$ per una cifra di $Y$, sommando un termine $C$ inizialmente nullo, che viene poi impostato alle cifre più significative del risultato parziale trovato.
		La LSD, invece, viene assegnata direttamente alla posizione corrispondente nel risultato finale.
	\item Infine, concatena tutti le cifre ottenute come LSD nel risultato finale.
\end{enumerate}

In questo modo possiamo fare solo moltiplicazioni su $n \times 1$ cifra e somme su due addendi su $n+1$ cifre.

\subsubsection{Moltiplicatore con addizionatore in base 2}
Vediamo quindi come realizzare un moltiplicatore con addizionatore $n \times 1$ in base 2, cioè un moltiplicatore con addizionatore ad una cifra.

Vorremmo il risultato, piuttosto triviale in $\beta = 2$:
$$
P_i = y_i \cdot X + C =
	\begin{cases}
		(0 \, +) \, C, \quad y_i = 0 \\ 
		X + C, \quad \, \, y_i = 1
	\end{cases}
$$

Possiamo effettuare la selezione su $y_i$ attraverso quello che è effettivamente un \textbf{multiplexer}.
Quello che facciamo quindi è collegare un muliplexer fra $X$ e $0$ con variabile di controllo $y_i$ a un ingresso di un full adder, e $C$ all'altro ingresso.
L'ingresso $C_{in}$ del full adder varrà 0, mentre l'uscita $C_{out}$ verrà concatenata alla somma $S$.

Abbiamo che in base 2 un multiplexer a due ingressi, con uno di questi negato, è effettivamente una porta AND fra l'ingresso non nullo e la variabile di controllo.
Sostituiamo quindi il multiplexer con un AND a $n$ fra $X$ e $y_i$.

\subsubsection{Richiamo all'assembler}
Avevamo visto che in assembler la moltiplicazione aveva un solo operando esplicito, mentre l'altro era implicito su AL, AX o EAX.
Il risultato veniva poi concatenato in AX, DX\_AX o EDX\_EAX.
Questo rispetta la logica vista finora:partendo da fattori su $n$ e $m$ bit, con $n = m$, si arriva ad un risultato rappresentabile su $n + m = 2n$ bit, cioè $8 + 8 = 16$ bit (AL $\rightarrow$ AX), $16 + 16 = 32$ bit (AX $\rightarrow$ DX\_AX) e $16 + 16 = 32$ bit (EAX $\rightarrow$ EDX\_EAX)

\subsubsection{Convertitori di base}
Vediamo come realizzare un convertitore da 2 cifre, $x_1$ e $x_0$ in codifica BCD, alla codifica binaria.
Due cifre rappresentano al massimo 99, che in binario sta su 7 bit.
Ergo vogliamo un circuito con 8 bit di ingresso (4 bit + 4 bit degli ingressi BCD) e 7 bit di uscita.
Abbiamo che, banalmente, la conversione si effettua come:
$$
y = 10 \cdot x_1 + x_0 
$$
Questo si può realizzare con un moltiplicatore con addizionatore con $X=x_1$, $Y=10$ e $C=x_0$.
Abbiamo che il risultato è su 8 bit, di cui sappiamo però possiamo ridurre il campo a 7.
\par\smallskip 
Un circuito più efficiente può essere realizzato usando solo somme e shift, infatti abbiamo che:
$$ y = 10 \cdot x_1 + x_ 0 = 8 \cdot x_1 + 2 \cdot x_1 + x_0 $$
che appare migliore dal punto di vista della realizzazione in aritmetica binaria (8 e 2 sono $2^3$ e $2^1$).
Abbiamo quindi che possiamo usare i moltiplicatori per $b^k$, e ottenere un circuito con lo stesso comportamento.

Per la precisione, prendiamo $8 \cdot x_1$ e troviamo che si estende fino a 7 bit.
Prendiamo poi  $2 \cdot x_1 + x_0$ e vediamo che la somma si rappresenta su 5 bit.
Sommando i 7 bit di $8 \cdot x_1$ ai 5 di $2 \cdot x_1 + x_0$ abbiamo un risultato sempre su 7 bit.
\end{document}


\documentclass[a4paper,11pt]{article}
\usepackage[a4paper, margin=8em]{geometry}

% usa i pacchetti per la scrittura in italiano
\usepackage[french,italian]{babel}
\usepackage[T1]{fontenc}
\usepackage[utf8]{inputenc}
\frenchspacing 

% usa i pacchetti per la formattazione matematica
\usepackage{amsmath, amssymb, amsthm, amsfonts}

% usa altri pacchetti
\usepackage{gensymb}
\usepackage{hyperref}
\usepackage{standalone}

\usepackage{colortbl}

\usepackage{xstring}
\usepackage{karnaugh-map}

% imposta il titolo
\title{Appunti Reti Logiche}
\author{Luca Seggiani}
\date{2024}

% imposta lo stile
% usa helvetica
\usepackage[scaled]{helvet}
% usa palatino
\usepackage{palatino}
% usa un font monospazio guardabile
\usepackage{courier}

\usepackage{euler}

\renewcommand{\rmdefault}{ppl}
\renewcommand{\sfdefault}{phv}
\renewcommand{\ttdefault}{lmtt}

% circuiti
\usepackage{circuitikz}
\usetikzlibrary{babel}

% disponi il titolo
\makeatletter
\renewcommand{\maketitle} {
	\begin{center} 
		\begin{minipage}[t]{.8\textwidth}
			\textsf{\huge\bfseries \@title} 
		\end{minipage}%
		\begin{minipage}[t]{.2\textwidth}
			\raggedleft \vspace{-1.65em}
			\textsf{\small \@author} \vfill
			\textsf{\small \@date}
		\end{minipage}
		\par
	\end{center}

	\thispagestyle{empty}
	\pagestyle{fancy}
}
\makeatother

% disponi teoremi
\usepackage{tcolorbox}
\newtcolorbox[auto counter, number within=section]{theorem}[2][]{%
	colback=blue!10, 
	colframe=blue!40!black, 
	sharp corners=northwest,
	fonttitle=\sffamily\bfseries, 
	title=Teorema~\thetcbcounter: #2, 
	#1
}

% disponi definizioni
\newtcolorbox[auto counter, number within=section]{definition}[2][]{%
	colback=red!10,
	colframe=red!40!black,
	sharp corners=northwest,
	fonttitle=\sffamily\bfseries,
	title=Definizione~\thetcbcounter: #2,
	#1
}

% disponi codice
\usepackage{listings}
\usepackage[table]{xcolor}

\definecolor{codegreen}{rgb}{0,0.6,0}
\definecolor{codegray}{rgb}{0.5,0.5,0.5}
\definecolor{codepurple}{rgb}{0.58,0,0.82}
\definecolor{backcolour}{rgb}{0.95,0.95,0.92}

\lstdefinestyle{codestyle}{
		backgroundcolor=\color{black!5}, 
		commentstyle=\color{codegreen},
		keywordstyle=\bfseries\color{magenta},
		numberstyle=\sffamily\tiny\color{black!60},
		stringstyle=\color{green!50!black},
		basicstyle=\ttfamily\footnotesize,
		breakatwhitespace=false,         
		breaklines=true,                 
		captionpos=b,                    
		keepspaces=true,                 
		numbers=left,                    
		numbersep=5pt,                  
		showspaces=false,                
		showstringspaces=false,
		showtabs=false,                  
		tabsize=2
}

\lstdefinestyle{shellstyle}{
		backgroundcolor=\color{black!5}, 
		basicstyle=\ttfamily\footnotesize\color{black}, 
		commentstyle=\color{black}, 
		keywordstyle=\color{black},
		numberstyle=\color{black!5},
		stringstyle=\color{black}, 
		showspaces=false,
		showstringspaces=false, 
		showtabs=false, 
		tabsize=2, 
		numbers=none, 
		breaklines=true
}


\lstdefinelanguage{assembler}{ 
  keywords={AAA, AAD, AAM, AAS, ADC, ADCB, ADCW, ADCL, ADD, ADDB, ADDW, ADDL, AND, ANDB, ANDW, ANDL,
        ARPL, BOUND, BSF, BSFL, BSFW, BSR, BSRL, BSRW, BSWAP, BT, BTC, BTCB, BTCW, BTCL, BTR, 
        BTRB, BTRW, BTRL, BTS, BTSB, BTSW, BTSL, CALL, CBW, CDQ, CLC, CLD, CLI, CLTS, CMC, CMP,
        CMPB, CMPW, CMPL, CMPS, CMPSB, CMPSD, CMPSW, CMPXCHG, CMPXCHGB, CMPXCHGW, CMPXCHGL,
        CMPXCHG8B, CPUID, CWDE, DAA, DAS, DEC, DECB, DECW, DECL, DIV, DIVB, DIVW, DIVL, ENTER,
        HLT, IDIV, IDIVB, IDIVW, IDIVL, IMUL, IMULB, IMULW, IMULL, IN, INB, INW, INL, INC, INCB,
        INCW, INCL, INS, INSB, INSD, INSW, INT, INT3, INTO, INVD, INVLPG, IRET, IRETD, JA, JAE,
        JB, JBE, JC, JCXZ, JE, JECXZ, JG, JGE, JL, JLE, JMP, JNA, JNAE, JNB, JNBE, JNC, JNE, JNG,
        JNGE, JNL, JNLE, JNO, JNP, JNS, JNZ, JO, JP, JPE, JPO, JS, JZ, LAHF, LAR, LCALL, LDS,
        LEA, LEAVE, LES, LFS, LGDT, LGS, LIDT, LMSW, LOCK, LODSB, LODSD, LODSW, LOOP, LOOPE,
        LOOPNE, LSL, LSS, LTR, MOV, MOVB, MOVW, MOVL, MOVSB, MOVSD, MOVSW, MOVSX, MOVSXB,
        MOVSXW, MOVSXL, MOVZX, MOVZXB, MOVZXW, MOVZXL, MUL, MULB, MULW, MULL, NEG, NEGB, NEGW,
        NEGL, NOP, NOT, NOTB, NOTW, NOTL, OR, ORB, ORW, ORL, OUT, OUTB, OUTW, OUTL, OUTSB, OUTSD,
        OUTSW, POP, POPL, POPW, POPB, POPA, POPAD, POPF, POPFD, PUSH, PUSHL, PUSHW, PUSHB, PUSHA, 
				PUSHAD, PUSHF, PUSHFD, RCL, RCLB, RCLW, MOVSL, MOVSB, MOVSW, STOSL, STOSB, STOSW, LODSB, LODSW,
				LODSL, INSB, INSW, INSL, OUTSB, OUTSL, OUTSW
        RCLL, RCR, RCRB, RCRW, RCRL, RDMSR, RDPMC, RDTSC, REP, REPE, REPNE, RET, ROL, ROLB, ROLW,
        ROLL, ROR, RORB, RORW, RORL, SAHF, SAL, SALB, SALW, SALL, SAR, SARB, SARW, SARL, SBB,
        SBBB, SBBW, SBBL, SCASB, SCASD, SCASW, SETA, SETAE, SETB, SETBE, SETC, SETE, SETG, SETGE,
        SETL, SETLE, SETNA, SETNAE, SETNB, SETNBE, SETNC, SETNE, SETNG, SETNGE, SETNL, SETNLE,
        SETNO, SETNP, SETNS, SETNZ, SETO, SETP, SETPE, SETPO, SETS, SETZ, SGDT, SHL, SHLB, SHLW,
        SHLL, SHLD, SHR, SHRB, SHRW, SHRL, SHRD, SIDT, SLDT, SMSW, STC, STD, STI, STOSB, STOSD,
        STOSW, STR, SUB, SUBB, SUBW, SUBL, TEST, TESTB, TESTW, TESTL, VERR, VERW, WAIT, WBINVD,
        XADD, XADDB, XADDW, XADDL, XCHG, XCHGB, XCHGW, XCHGL, XLAT, XLATB, XOR, XORB, XORW, XORL},
  keywordstyle=\color{blue}\bfseries,
  ndkeywordstyle=\color{darkgray}\bfseries,
  identifierstyle=\color{black},
  sensitive=false,
  comment=[l]{\#},
  morecomment=[s]{/*}{*/},
  commentstyle=\color{purple}\ttfamily,
  stringstyle=\color{red}\ttfamily,
  morestring=[b]',
  morestring=[b]"
}

\lstset{language=assembler, style=codestyle}

% disponi sezioni
\usepackage{titlesec}

\titleformat{\section}
	{\sffamily\Large\bfseries} 
	{\thesection}{1em}{} 
\titleformat{\subsection}
	{\sffamily\large\bfseries}   
	{\thesubsection}{1em}{} 
\titleformat{\subsubsection}
	{\sffamily\normalsize\bfseries} 
	{\thesubsubsection}{1em}{}

% tikz
\usepackage{tikz}

% float
\usepackage{float}

% grafici
\usepackage{pgfplots}
\pgfplotsset{width=10cm,compat=1.9}

% disponi alberi
\usepackage{forest}

\forestset{
	rectstyle/.style={
		for tree={rectangle,draw,font=\large\sffamily}
	},
	roundstyle/.style={
		for tree={circle,draw,font=\large}
	}
}

% disponi algoritmi
\usepackage{algorithm}
\usepackage{algorithmic}
\makeatletter
\renewcommand{\ALG@name}{Algoritmo}
\makeatother

% disponi numeri di pagina
\usepackage{fancyhdr}
\fancyhf{} 
\fancyfoot[L]{\sffamily{\thepage}}

\makeatletter
\fancyhead[L]{\raisebox{1ex}[0pt][0pt]{\sffamily{\@title \ \@date}}} 
\fancyhead[R]{\raisebox{1ex}[0pt][0pt]{\sffamily{\@author}}}
\makeatother

\begin{document}
% sezione (data)
\section{Lezione del 23-10-24}

% stili pagina
\thispagestyle{empty}
\pagestyle{fancy}

% testo
\subsection{Divisione}
Siano dati $X$, un naturale in base $\beta$ su $n+m$ cifre, detto \textbf{dividendo}, con $0 \leq X \leq \beta^{n+m}-1$,
e $Y$, un naturale in base $\beta$ su $m$ cifre, detto \textbf{divisore}, con $0 \leq Y \leq \beta^m -1$.
Vogliamo calcolare i due numeri $Q$ ed $R$ tali che:
$$ X = Q \cdot Y + R$$

Abbiamo che, con $|Y=0|$, la divisione non è fattibile, quindi avremo bisogno di un uscita di \textbf{non fattibilità} \lstinline|no_div|.

\subsubsection{Dimensioni di resti e quozienti}
Assumendo $Y > 0$, si ha che $Q$ sta su $n+m$ cifre (caso peggiore $Y = 1$), mentre $R$ sta su $m$ cifre, in quanto $0 \leq R \leq Y$ dalle proprietà della divisione.
Scelgo, per ragioni tecniche, che il quoziente dovrà stare su $n$ cifre, quindi impongo $Q \leq \beta^n -1$.
Nel caso non si possa rappresentare $Q$, quindi, userò sempre la stessa uscita \lstinline|no_div| di prima.

La decisione fatta riguardo a $Q$ implica che:
$$ 
X = Q \cdot Y + R \leq (\beta^n - 1) \cdot Y + (Y - 1) = \beta^n \cdot Y - 1 \Rightarrow X < \beta^n \cdot Y
$$

L'ipotesi potrebbe sembrare limitante, ma visto che si può ricavare $n$ che soddisfi la disuguaglianza, possiamo eseguire qualsiasi divisione poste \textbf{estensioni} del dividendo e \textbf{riserve} di cifre (cioè più delle strettamente necessarie) per il quoziente.

Solo nel caso il numero di cifre $n$, $m$ sia dato dal problema, cioè quando si lavora su \textbf{campi finiti}, l'ipotesi è restrittiva.

L'obiettivo è quello di progettare circuiti che eseguano questa divisione su campi di dimensioni prestabilite: dovremmo ricordare questa proprietà nello sviluppo e del circuito (e noteremo ha un significato specifico), e quando andiamo ad utilizzarlo, cioè quando si scrivono istruzioni assembly che ordinano divisioni fra numeri su registri di dimensione diversa.

\subsubsection{Modulo divisore}
Vogliamo quindi realizzare un circuito che:
\begin{enumerate}
	\item Verifichi la \textbf{fattibilità} della divisione nelle ipotesi date;
	\item Se il quoziente sta su $n$ cifre, lo restituisca, altrimenti restituisca \lstinline|no_div|.
\end{enumerate}

La divisione viene svolta, tradizionalmente, prendendo un sottoinsieme delle $n$ cifre più significative del dividendo, tali per cui possiamo trovare quante volta il divisore sta nel sottoinsieme.
Visto che non possiamo riscalare il numero di cifre prese dal divisore una volta assemblato il circuito, abbiamo bisogno di un numero minimo di cifre da prendere ogni volta, per essere sicuri di poter eseguire la divisione. 
Formalmente, quindi, prendo il minimo numero di cifre più significative di $X$ per ottenere un $X' \in [Y, \beta \cdot Y[$
In questo $m$ cifre possono non bastare (potremmo avere che le $m$ cifre più significative di $X$ sono $< Y$), mentre $m+1$ bastano sempre (purchè $X$ non abbia zeri in testa).

Si calcolano quindi i quozienti e i resti \textbf{parziali}, $q$ e $R'$, dalla divisione di $X'$ e $Y$.
Si ha che $q$ sta su una sola cifra, perchè $X' < \beta \cdot Y$ dall'ipotesi.

Calcolo quindi il nuovo dividendo $X'$ concatenando $R'$ con la cifra più significativa non ancora utilizzata di $X$.
Il nuovo dividendo, date le ipotesi, è ancora $< \beta \cdot Y$:
$$
R' \leq Y - 1, \quad \beta \cdot R' + (\beta - 1) \leq \beta \cdot  Y - \beta + \beta + 1 = \beta \cdot Y
$$

Si itera fino ad esaurimento delle cifre del dividendo.
A questo punto il \textbf{quoziente} è ottenuto dal concatenamento dei quozienti parziali, e il resto è l'ultimo resto parziale.

Abbiamo che l'unica divisione effettiva è quella di $m+1$ per $m$ cifre, mentre tutte le altre sono effettivamente scomposizioni, quindi circuiti di logica a costo nullo.
Resta quindi da calcolare solo il flag di non fattibilità \lstinline|no_div|: questo deriva naturalmente da quanto avevamo detto riguardo alle dimensioni del dividendo:
$$
X < \beta^n \cdot Y
$$

Inoltre, vogliamo impostare \lstinline|no_div| anche nel caso $Y$ sia nullo, per ovvi motivi.

\subsubsection{Divisione nei processori Intel x86}
Abbiamo visto come nei processori Intel x86, abbiamo a disposizione tre versioni della divisione:

\begin{table}[h!]
	\center \rowcolors{2}{white}{black!10}
	\begin{tabular} { c | c | c | c | c }
		\bfseries Dim. sorgente (divisore) & \bfseries Dim. dividendo & \bfseries Dividendo & \bfseries Quoziente & \bfseries Resto \\ 
		\hline 
		8 bit & 16 bit & AX & AL & AH \\ 
		16 bit & 32 bit & DX\_AX & AX & DX \\ 
		32 bit & 64 bit & EDX\_EAX & EAX & EDX
	\end{tabular}
\end{table}

Si ha che la \lstinline|DIV| ammette dividendo su $2n$ bit e divisore su $n$ bit, con $n = 8, \, 16, \, 32$, e richiede che il quoziente stia su $n$ bit (altrimenti genera un'eccezione).
Questo è quello che si otterrebbe ponendo $n = m$.

\subsubsection{Divisione elementare in base 2}
Resta quindi da capire come effettuare la divisione elementare fra un numero a $m+1$ cifre e un altro a $m$ cifre, sotto l'ipotesi $X \leq 2Y = 2^1 \cdot Y$ (siamo in $\beta=2$).

Abbiamo che $Q$ può valere 0 o 1. Vale 0 se il divisore $Y$ è maggiore del dividendo $X$, 1 altrimenti.
$R$, invece, è uguale al dividendo $X$ se questo è minore del divisore $Y$, altrimenti è uguale a $X - Y$:
$$
Q=
\begin{cases}
	0, \quad X < Y \\ 
	1, \quad X \geq Y
\end{cases}, \quad 
R =
\begin{cases}
	X, \quad X < Y \\ 
	X-Y, \quad X \geq Y
\end{cases}
$$

Per rappresentare questo sistema ci serve un comparatore fra $X$ e $Y$. 
Lo realizziamo con un sottrattore (di cui bisognavamo comunque per il calcolo di $X-Y$), quindi mandando $Y$ complementato (ed opportunamente esteso) al secondo input di un sommatore, ed $X$ al primo. 
Il sommatore ha $C_{in} = 0$.

Fuori dal sommatore, avremo $X-Y$ come risultato, e $b_{out}$ come discrimnante per $X < Y$.
Mandiamo quindi $X$ e $X-Y$ agli ingressi di un multiplexer con variabile di controllo $C_{out}$ dal sommatore,
cioè discriminiamo fra $X$ e $X-Y$ sulla base di quanto restituito dal comparatore.

A questo punto si ha che $b_{out}$ rappresenta $Q$, mentre l'uscita del multiplexer è $R$.

Vediamo quindi l'implementazione del singolo modulo divisore:

\lstinputlisting[language=verilog, style=codestyle]{../verilog/10-23/dividers/n3by2_b2_divider.v}

e come questi si possono combinare a formare un modulo divisore completo a con dividendo a 4 e divisore a 2 cifre binarie:

\lstinputlisting[language=verilog, style=codestyle]{../verilog/10-23/dividers/n4by2_b2_divider.v}

dove si nota che il modulo \lstinline|n3_b2_subtractor| è un sottrattore a 3 cifre realizzato analogamente a quanto mostrato alla lezione precedente (in ogni caso, un'implementazione è reperibile nella cartella \lstinline|/verilog| della repository contenente gli appunti).
\end{document}


\documentclass[a4paper,11pt]{article}
\usepackage[a4paper, margin=8em]{geometry}

% usa i pacchetti per la scrittura in italiano
\usepackage[french,italian]{babel}
\usepackage[T1]{fontenc}
\usepackage[utf8]{inputenc}
\frenchspacing 

% usa i pacchetti per la formattazione matematica
\usepackage{amsmath, amssymb, amsthm, amsfonts}

% usa altri pacchetti
\usepackage{gensymb}
\usepackage{hyperref}
\usepackage{standalone}

\usepackage{colortbl}

\usepackage{xstring}
\usepackage{karnaugh-map}

% imposta il titolo
\title{Appunti Reti Logiche}
\author{Luca Seggiani}
\date{2024}

% imposta lo stile
% usa helvetica
\usepackage[scaled]{helvet}
% usa palatino
\usepackage{palatino}
% usa un font monospazio guardabile
\usepackage{lmodern}

\renewcommand{\rmdefault}{ppl}
\renewcommand{\sfdefault}{phv}
\renewcommand{\ttdefault}{lmtt}

% circuiti
\usepackage{circuitikz}
\usetikzlibrary{babel}

% disponi il titolo
\makeatletter
\renewcommand{\maketitle} {
	\begin{center} 
		\begin{minipage}[t]{.8\textwidth}
			\textsf{\huge\bfseries \@title} 
		\end{minipage}%
		\begin{minipage}[t]{.2\textwidth}
			\raggedleft \vspace{-1.65em}
			\textsf{\small \@author} \vfill
			\textsf{\small \@date}
		\end{minipage}
		\par
	\end{center}

	\thispagestyle{empty}
	\pagestyle{fancy}
}
\makeatother

% disponi teoremi
\usepackage{tcolorbox}
\newtcolorbox[auto counter, number within=section]{theorem}[2][]{%
	colback=blue!10, 
	colframe=blue!40!black, 
	sharp corners=northwest,
	fonttitle=\sffamily\bfseries, 
	title=Teorema~\thetcbcounter: #2, 
	#1
}

% disponi definizioni
\newtcolorbox[auto counter, number within=section]{definition}[2][]{%
	colback=red!10,
	colframe=red!40!black,
	sharp corners=northwest,
	fonttitle=\sffamily\bfseries,
	title=Definizione~\thetcbcounter: #2,
	#1
}

% disponi codice
\usepackage{listings}
\usepackage[table]{xcolor}

\definecolor{codegreen}{rgb}{0,0.6,0}
\definecolor{codegray}{rgb}{0.5,0.5,0.5}
\definecolor{codepurple}{rgb}{0.58,0,0.82}
\definecolor{backcolour}{rgb}{0.95,0.95,0.92}

\lstdefinestyle{codestyle}{
		backgroundcolor=\color{black!5}, 
		commentstyle=\color{codegreen},
		keywordstyle=\bfseries\color{magenta},
		numberstyle=\sffamily\tiny\color{black!60},
		stringstyle=\color{green!50!black},
		basicstyle=\ttfamily\footnotesize,
		breakatwhitespace=false,         
		breaklines=true,                 
		captionpos=b,                    
		keepspaces=true,                 
		numbers=left,                    
		numbersep=5pt,                  
		showspaces=false,                
		showstringspaces=false,
		showtabs=false,                  
		tabsize=2
}

\lstdefinestyle{shellstyle}{
		backgroundcolor=\color{black!5}, 
		basicstyle=\ttfamily\footnotesize\color{black}, 
		commentstyle=\color{black}, 
		keywordstyle=\color{black},
		numberstyle=\color{black!5},
		stringstyle=\color{black}, 
		showspaces=false,
		showstringspaces=false, 
		showtabs=false, 
		tabsize=2, 
		numbers=none, 
		breaklines=true
}


\lstdefinelanguage{assembler}{ 
  keywords={AAA, AAD, AAM, AAS, ADC, ADCB, ADCW, ADCL, ADD, ADDB, ADDW, ADDL, AND, ANDB, ANDW, ANDL,
        ARPL, BOUND, BSF, BSFL, BSFW, BSR, BSRL, BSRW, BSWAP, BT, BTC, BTCB, BTCW, BTCL, BTR, 
        BTRB, BTRW, BTRL, BTS, BTSB, BTSW, BTSL, CALL, CBW, CDQ, CLC, CLD, CLI, CLTS, CMC, CMP,
        CMPB, CMPW, CMPL, CMPS, CMPSB, CMPSD, CMPSW, CMPXCHG, CMPXCHGB, CMPXCHGW, CMPXCHGL,
        CMPXCHG8B, CPUID, CWDE, DAA, DAS, DEC, DECB, DECW, DECL, DIV, DIVB, DIVW, DIVL, ENTER,
        HLT, IDIV, IDIVB, IDIVW, IDIVL, IMUL, IMULB, IMULW, IMULL, IN, INB, INW, INL, INC, INCB,
        INCW, INCL, INS, INSB, INSD, INSW, INT, INT3, INTO, INVD, INVLPG, IRET, IRETD, JA, JAE,
        JB, JBE, JC, JCXZ, JE, JECXZ, JG, JGE, JL, JLE, JMP, JNA, JNAE, JNB, JNBE, JNC, JNE, JNG,
        JNGE, JNL, JNLE, JNO, JNP, JNS, JNZ, JO, JP, JPE, JPO, JS, JZ, LAHF, LAR, LCALL, LDS,
        LEA, LEAVE, LES, LFS, LGDT, LGS, LIDT, LMSW, LOCK, LODSB, LODSD, LODSW, LOOP, LOOPE,
        LOOPNE, LSL, LSS, LTR, MOV, MOVB, MOVW, MOVL, MOVSB, MOVSD, MOVSW, MOVSX, MOVSXB,
        MOVSXW, MOVSXL, MOVZX, MOVZXB, MOVZXW, MOVZXL, MUL, MULB, MULW, MULL, NEG, NEGB, NEGW,
        NEGL, NOP, NOT, NOTB, NOTW, NOTL, OR, ORB, ORW, ORL, OUT, OUTB, OUTW, OUTL, OUTSB, OUTSD,
        OUTSW, POP, POPL, POPW, POPB, POPA, POPAD, POPF, POPFD, PUSH, PUSHL, PUSHW, PUSHB, PUSHA, 
				PUSHAD, PUSHF, PUSHFD, RCL, RCLB, RCLW, MOVSL, MOVSB, MOVSW, STOSL, STOSB, STOSW, LODSB, LODSW,
				LODSL, INSB, INSW, INSL, OUTSB, OUTSL, OUTSW
        RCLL, RCR, RCRB, RCRW, RCRL, RDMSR, RDPMC, RDTSC, REP, REPE, REPNE, RET, ROL, ROLB, ROLW,
        ROLL, ROR, RORB, RORW, RORL, SAHF, SAL, SALB, SALW, SALL, SAR, SARB, SARW, SARL, SBB,
        SBBB, SBBW, SBBL, SCASB, SCASD, SCASW, SETA, SETAE, SETB, SETBE, SETC, SETE, SETG, SETGE,
        SETL, SETLE, SETNA, SETNAE, SETNB, SETNBE, SETNC, SETNE, SETNG, SETNGE, SETNL, SETNLE,
        SETNO, SETNP, SETNS, SETNZ, SETO, SETP, SETPE, SETPO, SETS, SETZ, SGDT, SHL, SHLB, SHLW,
        SHLL, SHLD, SHR, SHRB, SHRW, SHRL, SHRD, SIDT, SLDT, SMSW, STC, STD, STI, STOSB, STOSD,
        STOSW, STR, SUB, SUBB, SUBW, SUBL, TEST, TESTB, TESTW, TESTL, VERR, VERW, WAIT, WBINVD,
        XADD, XADDB, XADDW, XADDL, XCHG, XCHGB, XCHGW, XCHGL, XLAT, XLATB, XOR, XORB, XORW, XORL},
  keywordstyle=\color{blue}\bfseries,
  ndkeywordstyle=\color{darkgray}\bfseries,
  identifierstyle=\color{black},
  sensitive=false,
  comment=[l]{\#},
  morecomment=[s]{/*}{*/},
  commentstyle=\color{purple}\ttfamily,
  stringstyle=\color{red}\ttfamily,
  morestring=[b]',
  morestring=[b]"
}

\lstset{language=assembler, style=codestyle}

% disponi sezioni
\usepackage{titlesec}

\titleformat{\section}
	{\sffamily\Large\bfseries} 
	{\thesection}{1em}{} 
\titleformat{\subsection}
	{\sffamily\large\bfseries}   
	{\thesubsection}{1em}{} 
\titleformat{\subsubsection}
	{\sffamily\normalsize\bfseries} 
	{\thesubsubsection}{1em}{}

% tikz
\usepackage{tikz}

% float
\usepackage{float}

% grafici
\usepackage{pgfplots}
\pgfplotsset{width=10cm,compat=1.9}

% disponi alberi
\usepackage{forest}

\forestset{
	rectstyle/.style={
		for tree={rectangle,draw,font=\large\sffamily}
	},
	roundstyle/.style={
		for tree={circle,draw,font=\large}
	}
}

% disponi algoritmi
\usepackage{algorithm}
\usepackage{algorithmic}
\makeatletter
\renewcommand{\ALG@name}{Algoritmo}
\makeatother

% disponi numeri di pagina
\usepackage{fancyhdr}
\fancyhf{} 
\fancyfoot[L]{\sffamily{\thepage}}

\makeatletter
\fancyhead[L]{\raisebox{1ex}[0pt][0pt]{\sffamily{\@title \ \@date}}} 
\fancyhead[R]{\raisebox{1ex}[0pt][0pt]{\sffamily{\@author}}}
\makeatother

\begin{document}
% sezione (data)
\section{Lezione del 24-10-24}

% stili pagina
\thispagestyle{empty}
\pagestyle{fancy}

% testo
\subsubsection{Rappresentazione dei numeri interi}
Vogliamo rappresentare numeri interi su $n$ cifre.
Finora avevamo definito la legge di rappresentazione:
$$
A = \sum_{i=0}^{n-1} a_i \cdot \beta^i
$$
per i numeri naturali.

Nel sistema decimale, usiamo solutamente la rappresentazione \textbf{modulo e segno}, cioè usiamo un segno prefisso ($+$ o $-$), e poi indichiamo il modulo come un naturale.

In base 2, invece, decidiamo di sfruttare la seguente proprietà: preso un insieme di $\beta^n$ numeri interi, posso sempre trovare una legge biunivoca che gli fa corrispondere un insieme di $\beta^n$ numeri naturali.

Definisco quindi una legge $L: \mathbb{Z} \rightarrow \mathbb{N}$, e chiamo $A$ un numero naturale (che indicheremo in Maiuscolo da qui in avanti) e $a$ un numero intero (che indicheremo in minuscolo da qui in avanti).
Si ha quindi che:
$$
A = L(a) \Leftrightarrow a = L^{-1}(A)
$$
e cioè:
$$ 
a \leftrightarrow^L A \equiv (a_{n-1}a_{n-2}...a_1a_0)_\beta
$$
cioè con la stessa sequenza di cifre indichiamo sia un naturale che il corrispondente intero.
Notiamo che la $a$ minuscola qui significa \textit{cifra}, che è indifferentemente di $A$ o di $a$ (numeri naturali e interi).

Scegliendo leggi $L$ valide possiamo otttenere dei significativi vantaggi implementativi: ad esempio potremmo definire una legge che permette di usare la stessa circuiteria per le operazioni aritmetiche sia sui naturali che sugli interi.

Il \textbf{dominio} di $L$ dovrà essere contiguo, cioè un'intervallo, magari il più simmetrico possibile rispetto allo zero.
Questo è possibile solo se $\beta$ è dispari.
Nel caso di $\beta$ pari, come sarà nel nostro caso di interesse $\beta = 2$, dovremmo prendere un numero "in più" a destra o a sinistra.
Nel sistema adottato (sarà il complemento a 2) prendiamo il numero a sinistra, cioè quello negativo, ergo avremo l'\textbf{intervallo di rappresentabilità}:
$$
\left[ -\frac{\beta^n}{2}, \frac{\beta^n}{2} - 1 \right]
$$

Notiamo che da qui in poi assumeremo di lavorare in $\beta$ pari, in quanto in caso contrario dovremmo usare la seguente notazione:
$$
\left\lfloor -\frac{\beta^n}{2} \right\rfloor, \quad \left\lceil -\frac{\beta^n}{2} \right\rceil, \quad ...
$$
che appesantirebbe la trattazione, cosa inutile in quanto abbiamo stabilito che il nostro interesse finale è trovare metodi che si applichino a $\beta = 2$.

Abbiamo infatti che, posto $\beta=2$, l'intervallo riportato precedentemente rappresenta quello a cui siamo abituati per la rappresentabilità dei numeri interi in complemento a 2:
$$
\left[ -2^{n-1}, 2^{n-1} - 1 \right]
$$

\subsubsection{Rappresentazione in traslazione}
Una possibile legge di rappresentazione è data da:
$$
L(a) = A = a + \frac{\beta^n}{2}
$$
chiamiamo $\frac{\beta^n}{2}$ \textbf{fattore di polarizzazione}.
Questa rappresentazione è utile, \textbf{monotona} ($a < b \Leftrightarrow A < B$), e viene usata nei convertitori analogico/digitale e digitale/analogico, dove viene chiamata \textit{binario bipolare}.
Inoltre, ricordiamo che rappresenta l'esponente nei numeri reali in virgola mobile secondo lo standard IEEE 754.

Su un grafico dove le ordinate rappresentano $A$ e le ascisse $a$, abbiamo la mappa:
\begin{center}
	\begin{tikzpicture} [scale=0.9]
    \begin{axis}[
        axis lines=middle,
        xlabel={$a$},
        ylabel={$A$},
				xtick={-0.5,0.5},
				ytick={0,0.5,1},
				xticklabels={$-\frac{\beta^n}{2}$, $\frac{\beta^n}{2} - 1$},
				yticklabels={$0$, $\frac{\beta}{2}$, $\beta^n - 1$},
				axis line style = {-}, % Use this line to remove arrows
				] 


		\addplot[domain=-0.5:0.5, black, thick] {x+0.5};

    \end{axis}
\end{tikzpicture}
\end{center}

\subsubsection{Complemento alla radice}
Definiamo la legge, che abbiamo solitamente chiamato \textbf{complemento a 2}:
\[
		L(a) = A =	
	\begin{cases}
		a, \quad \quad \quad  0 \leq a < \frac{\beta^n}{2} \\ 
		\beta^n + a, \quad -\frac{\beta^n}{2} \leq a < 0
	\end{cases}
\]

Con questa legge perdiamo la \textbf{monotoneità}.
Ciò nonostante, è la legge usata di norma dal processore.

Ricordiamo che graficamente si può rendere con il cosiddetto diagramma a farfalla:
\begin{center}
	\begin{tikzpicture} [scale=0.9]
    \begin{axis}[
        axis lines=middle,
        xlabel={$a$},
        ylabel={$A$},
				xtick={-0.5,0.5},
				ytick={0,0.5,1},
				xticklabels={$-\frac{\beta^n}{2}$, $\frac{\beta^n}{2} - 1$},
				yticklabels={$0$, $\frac{\beta}{2}$, $\beta^n - 1$},
				axis line style = {-}, % Use this line to remove arrows
				] 


		\addplot[domain=-0.5:0, black, thick] {x+1};
		\addplot[domain=0:0.5, black, thick] {x};

    \end{axis}
\end{tikzpicture}
\end{center}

\subsubsection{Modulo e segno}
Possiamo sempre usare la legge di rappresentazione in modulo e segno, cioè $(s, M) \leftrightarrow a$:
\[
	s=
	\begin{cases}
		0, \quad a \geq 0 \\ 
		1, \quad a < 0
	\end{cases} 
\]
$$
M = \mathrm{abs}(a)
$$
Notiamo che noi intendiamo, per modulo e segno, una rappresentazione deve $n$ bit rappresentano il modulo, e l'$n$-esimo bit rappresenta il segno, per un totale di $n+1$ bit.
Non stiamo quindi mettendo in relazione intervalli di naturali con intervalli di interi, ma intervalli di naturali complementati da un bit in più di segno, con intervalli di interi.
Abbiamo quindi che questo tipo di rappresentazione non ricade nella categoria definita prima.
Ricordiamo comunque che viene applicata per rappresentare il segno dei numeri reali reali in virgola mobile secondo lo standard IEEE 754.

\subsubsection{Determinazione del segno}
Posso determinare il segno di un numero intero $a$ dalle cifre della sua rappresentazione $A$:
\[
	\begin{cases}
		a \geq 0 \Leftrightarrow 0 \leq A < \frac{\beta^n}{2} \\	
		a < 0 \Leftrightarrow \frac{\beta^n}{2} \leq A \ \beta^n	
	\end{cases}
\]

Facciamo le solite considerazioni:

\begin{itemize}
	\item Il \textbf{massimo numero rappresentabile} è $\frac{\beta^n}{2} - 1$, che in CR ha rappresentazione $\left( \frac{\beta}{2} - 1, \beta, ..., \beta \right)_\beta$;
	\item Il \textbf{minimo numero rappresentabile} è $\frac{\beta^n}{2}$, che in CR ha rappresentazione $\left( \frac{\beta}{2}, 0, ..., 0 \right)_\beta$;
	\item Lo \textbf{0} coincide in $A$ e $a$, ergo vale 0 e ha rappresentazione in CR $\left( 0, ..., 0 \right)_\beta$;
	\item Il \textbf{-1} è $\beta^n-1$, che in CR ha rappresentazione $\left( \beta, ..., \beta \right)_\beta$.
\end{itemize}

Quindi, per capire se la rappresentazione $A$ è un numero naturale maggiore o minore di $\frac{\beta^n}{2}$, che equivale a capire se l'intero che rappresenta è maggiore o minore di zero, basta guardare la cifra più significativa:
\[
	\begin{cases}
		a_{n-1} < \frac{\beta}{2} \Leftrightarrow 0 \leq A < \frac{\beta^n}{2} \\ 	
		a_{n-1} \geq \frac{\beta}{2} \Leftrightarrow \frac{\beta^n}{2} \leq A < \beta^n \\ 	
	\end{cases}
\]

Questo ci permette di riscrivere $L$ nella forma più elegante:
\[
		L(a) = A =	
	\begin{cases}
		a, \quad \quad \quad  a_{n-1} < \frac{\beta}{2} \\ 
		\beta^n + a, \quad a_{n-1} \geq \frac{\beta}{2}
	\end{cases}
\]


\subsubsection{Legge inversa del CR}
Possiamo ottenere per sostituzione la legge inversa della legge di rappresentazione CR:
\[
		L(a) = A =	
	\begin{cases}
		a, \quad \quad \quad  0 \leq a < \frac{\beta^n}{2} \\ 
		\beta^n + a, \quad -\frac{\beta^n}{2} \leq a < 0
	\end{cases}, \quad 
		L^{-1}(A) = a =
	\begin{cases}
		A, \quad \quad \quad 0 \leq A < \frac{\beta^n}{2} \\ 
		A - \beta^n, \quad \frac{\beta^n}{2} \leq A < \beta^n 
	\end{cases} 
\]

Che possiamo riscrivere nel modo più elegante:
\[
	L^{-1}(A) = a =
	\begin{cases}			
		A, \quad \quad \quad \quad a_{n-1} < \frac{\beta}{2} \\ 
		-\left( \overline{A} + 1 \right), \quad a_{n-1} \geq \frac{\beta}{2} 
	\end{cases}
\]
usando quanto detto sulla MSD e quanto conoscevamo sui complementi (ancora, è sostanzialmente un complemento a 2).

In particolare, il cambio delle disequazioni viene fatto note le propietà sul MSD.
La trasformazione $A - \beta^n = -\left( \overline{A} + 1 \right)$ si ricava invece dalla proprietà fondamentale:
$$
A + \overline{A} = \beta^n - 1
$$
con semplici passaggi algebrici.

\subsubsection{Forma alternativa del CR}
Possiamo usare la forma più concisa (ma anche più pericolosa) delle legge di rappresentazoine CR:
\[
	L(a) = A = |a|_{\beta^n}, \quad \text{se} \, -\frac{\beta^n}{2} \leq a < \frac{\beta^n}{2} - 1
\]

Abbiamo che:
\begin{enumerate}
	\item Se $a \geq 0$, allora è anche $< \beta^n$, quindi $A = a = |a|_{\beta^n}$;
	\item Se $a < 0$, allora è compreso in $\left[ -\frac{\beta^n}{2}, 0 \right[$, quindi diviso $\beta^n$ dà quoziente $-1$, cioè dal teorema della divisione con resto, $a = -1\cdot \beta^n + |a|_{\beta^n}$.
			Avevamo dalla legge di rappresentazione in complemento a radice che, sotto questa ipotesi, volevamo esattamente $\beta^n + a$, ergo si ottiene ugualmente $A = |a|_{\beta^n}$.
\end{enumerate}

Possiamo quindi interpretare il complemento a radice come una rappresentazione modulare:
\begin{center}
	\begin{tikzpicture} [scale=0.9]
    \begin{axis}[
        axis lines=middle,
        xlabel={$a$},
        ylabel={$A$},
				xtick={-1, -0.5,0.5, 1},
				ytick={0,0.5,1},
				xticklabels={$-\beta^n$, $-\frac{\beta^n}{2}$, $\frac{\beta^n}{2} - 1$, $\beta^n$},
				yticklabels={$0$, $\frac{\beta}{2}$, $\beta^n - 1$},
				axis line style = {-}, % Use this line to remove arrows
				height=5cm,
				width=14cm
				] 

		\addplot[domain=-3:-2, black, thick] {x+3};
		\addplot[domain=-2:-1, black, thick] {x+2};
		\addplot[domain=-1:0, black, thick] {x+1};
		\addplot[domain=0:1, black, thick] {x};
		\addplot[domain=1:2, black, thick] {x-1};
		\addplot[domain=2:3, black, thick] {x-2};
		\addplot[domain=2:3, black, thick] {x-2};

    \end{axis}
\end{tikzpicture}
\end{center}

Occorre fare attenzione in quanto questo è vero solo nel caso $a$ sia \textbf{rappresentabile}, cioè se rispetta: $ -\frac{\beta^n}{2} \leq a < \frac{\beta^n}{2} - 1 $. 
In caso contrario, come è chiaro dal grafico, si potrebbero avere le stesse rappresentazioni per interi diversi fra di loro (cioè essenzialmente un overflow).

\subsection{Operazioni su interi in CR}
Vogliamo progettare circuiti che lavorano sulle rappresentazioni, come avevamo fatto per i naturali.
Ricordiamo che la rappresentazione vale per un naturale $A$ o un intero $a$ a seconda di quanto deciso dal programmatore, e nient'altro.

\subsubsection{Valore assoluto}
Vogliamo trovare il valore assoluto di un numero intero $B = \mathrm{abs}(a)$, con: 
$$ a \in \left[ -\frac{\beta^n}{2}, \frac{\beta^n}{2} - 1 \right] \Rightarrow B \in \left[0, \frac{\beta^n}{2}\right]
$$
Si ha che $B$ è un numero naturale rappresentabile su $n$ cifre.
Sappiamo che:
\[
	\mathrm{abs}(a) =
	\begin{cases}
			a, \quad a \geq 0 \\ 
			-a, \quad \, a < 0 
	\end{cases}
\] 

Posso quindi ottenere $\mathrm{abs}(a)$ complementando la rappresentazione nel range di valori che so essere negativo (cioè quando $a_{n-1} \geq \frac{\beta}{2}$):
\[
	B = \mathrm{abs}(a) =
	\begin{cases}
		A, \quad \quad a_{n-1} < \frac{\beta}{2} \\ 
		\overline{A} +  1, \quad a_{n-1} \geq \frac{\beta}{2}
	\end{cases}
\]

Logicamente, questo sarà rappresentato da un multiplexer che discrimina fra $A$ e $\overline{A} - 1$.
La variabile di comando sara datà dal $b_{out}$ di un comparatore fra $a_{n-1}$ e $\frac{\beta}{2}$

In base 2 questo è notevolmente più semplice: avrò che basta prendere $a_{n-1}$ come variabile di comando.
Si potrà quindi complementare con uno XOR fra le cifre di A e $a_{n-1}$, e usare $a_{n-1}$ anche come ingresso di un incrementatore, con le cifre di $A$ all'altro ingresso.

\subsection{Conversione da CR a MS}
Vediamo come convertire un numero in CR nella rappresentazione modulo e segno.
Prima di tutto notiamo una discrepanza nell'intervallo di rappresentabilità:
$$
i_{MS} = [-\beta^n + 1, \beta^n - 1] \not\Leftrightarrow i_{CR} = \left[-\frac{\beta^n}{2}, \frac{\beta^n}{2}\right]
$$
Abbiamo però che $i_{CR} \subset i_{MS}$, tolto il bit di segno, quindi l'operazione è sempre fattibile, calcolando l'assoluto e stabilendo:
$$
\mathrm{sgn}(a) = 
	\begin{cases}
		0, \quad a_{n-1} < \frac{\beta}{2} \Leftrightarrow a_{n-1} = 0 \\ 
		1, \quad a_{n-1} \geq \frac{\beta}{2} \Leftrightarrow a_{n-1} = 1 \\ 
	\end{cases}
$$
dove si è riportato il valore di $a_{n-1}$ in base $\beta = 2$.

\subsection{Calcolo dell'opposto}
Vediamo come trovare l'opposto di un numero in CR, quindi dato $A \leftrightarrow a$, $B \leftrightarrow b$ tale che $ b = -a$.
Questo' operazione non è sempre possibile, a causa dell'asimmetria dell'intervallo di rappresentabilità in CR $\left[-\frac{\beta^n}{2}, \frac{\beta^n}{2}\right]$: avremo che il numero in $-\frac{\beta^n}{2}$, negativo, non ha opposto positivo rappresentabile.

Avremo quindi bisogno di un flag di overflow, diciamo \lstinline|ow|.
Le due uscite, l'opposto e \lstinline|ow|, andranno quindi calcolate separatamente.
Assumendo \lstinline|ow|$=0$, si ha:
$$
B = |-a|_{\beta^n} = \left||-1|_{\beta^n} \cdot |a|_{\beta^n} \right|_{\beta^n} = |(\beta^n - 1) \cdot A|_{\beta^n}
$$
$$
= |\beta^n \cdot A - A |_{\beta^n} = |-A|_{\beta^n} = |-\beta^n + 1 + \overline{A}|_{\beta^n} = |1 + \overline{A}|_{\beta^n}
$$
cioè si ritrova sostanzialmente la legge di rappresentazione inversa $L^{-1}$.

Sappiamo di poter implementare questa legge con una negazione di tutte le cifre, seguita da un incremento di 1.
Nel caso precedente, avevamo usato lo XOR in quanto volevamo che la negazione fosse condizionale (pilotata dal bit di segno).
In questo caso vogliamo negare sempre, quindi basta una porta NOT.

L'\lstinline|ow| viene invece impostato sulla base di un AND fra le cifre più significative $a_{n-1}$ del numero non negato e $b_{n-1}$ del numero negato.

\subsubsection{Richiamo all'assembly}
Si ricorda che in assembly avevamo l'istruzione \lstinline|NEG|, che interpretava una sequenza di bit come un numero intero, e ne calcolava l'opposto se possibile, impostando il flag OF altrimenti.

\subsection{Estensione di campo per gli interi}
Avevamo detto che l'estensione di campo per gli interi richiedeva logica.
Possiamo infatti ricavare, per via algebrica il valore dell'intero con l'$n$-esima cifra aggiunta, $A_{EST}$:

$$
A_{EST} = 
	\begin{cases}
		a, \quad \quad \quad \quad 0 \leq a < \frac{\beta^n}{2} \\ 
		\beta^{n+1} + a, \quad -\frac{\beta^n}{2} \leq a < 0
	\end{cases}
	=
	\begin{cases}
		A, \quad \quad \quad \quad \quad \quad \quad 0 \leq a < \frac{\beta^n}{2} \\ 
		\beta^{n} \cdot (\beta - 1) + a, \quad -\frac{\beta^n}{2} \leq a < 0
	\end{cases}
$$
da cui troviamo:
$$
	=
	\begin{cases}
		0 \cdot \beta^n + A,  \quad \quad \quad \quad 0 \leq a < \frac{\beta^n}{2} \\ 
		\beta^{n} \cdot (\beta - 1) + a, \quad -\frac{\beta^n}{2} \leq a < 0
	\end{cases}
$$

Notiamo che i termini che moltiplicano $\beta^n$ sono quelli della cifra che vogliamo aggiungere.
Possiamo quindi definire la cifra aggiunta:
$$
a_n =
	\begin{cases}
		0, \quad \quad a_{n-1} < \frac{\beta^n}{2} \\ 
		\beta -1 \quad a_{n-1} \geq \frac{\beta^n}{2}
	\end{cases}
$$

Graficamente, possiamo pensare all'estensione di campo come una traslazione verso l'alto del lato sinistro dell'intervallo di ordinate, cioè quello che rappresenta gli interi negativi.
Più propriamente, se avevamo dato la rappresentazione $|a|_{\beta^n}$ per $A$, adesso dobbiamo prendere $|a|_{\beta^{n+1}}$ che graficamente dà:
\begin{center}
	\begin{tikzpicture} [scale=0.9]
    \begin{axis}[
        axis lines=middle,
        xlabel={$a$},
        ylabel={$A$},
				xtick={-2, -1, -0.5,0.5, 1, 2},
				ytick={0,0.5,1},
				xticklabels={$-\beta^{n+1}$, $-\beta^n$, $-\frac{\beta^n}{2}$, $\frac{\beta^n}{2} - 1$, $\beta^n$, $\beta^{n+1}$},
				yticklabels={$0$, $\frac{\beta}{2}$, $\beta^n - 1$},
				axis line style = {-}, % Use this line to remove arrows
				height=5cm,
				width=14cm
				] 

		\addplot[domain=-3:-2, red, thick] {x+3};
		\addplot[domain=-2:-1, red, thick] {x+2};
		\addplot[domain=-1:0, red, thick] {x+1};
		\addplot[domain=0:1, red, thick] {x};
		\addplot[domain=1:2, red, thick] {x-1};
		\addplot[domain=2:3, red, thick] {x-2};
		\addplot[domain=2:3, red, thick] {x-2};

		\addplot[domain=-2:0, blue, thick] {x+2};
		\addplot[domain=0:2, blue, thick] {x};
    \end{axis}
\end{tikzpicture}
\end{center}
da cui si nota ancora meglio che il lato per interi positivi resta tale, mentre il lato negativo trasla in alto (in rosso si ha $|a|_{\beta^n}$, e in blu $|a|_{\beta^{n+1}}$).

Dal punto di vista della base 2, questo tipo di estensore può essere realizzato semplicemente replicando la $n-1$-esima cifra.

\end{document}


\documentclass[a4paper,11pt]{article}
\usepackage[a4paper, margin=8em]{geometry}

% usa i pacchetti per la scrittura in italiano
\usepackage[french,italian]{babel}
\usepackage[T1]{fontenc}
\usepackage[utf8]{inputenc}
\frenchspacing 

% usa i pacchetti per la formattazione matematica
\usepackage{amsmath, amssymb, amsthm, amsfonts}

% usa altri pacchetti
\usepackage{gensymb}
\usepackage{hyperref}
\usepackage{standalone}

\usepackage{colortbl}

\usepackage{xstring}
\usepackage{karnaugh-map}

% imposta il titolo
\title{Appunti Reti Logiche}
\author{Luca Seggiani}
\date{2024}

% imposta lo stile
% usa helvetica
\usepackage[scaled]{helvet}
% usa palatino
\usepackage{palatino}
% usa un font monospazio guardabile
\usepackage{lmodern}

\renewcommand{\rmdefault}{ppl}
\renewcommand{\sfdefault}{phv}
\renewcommand{\ttdefault}{lmtt}

% circuiti
\usepackage{circuitikz}
\usetikzlibrary{babel}

% disponi il titolo
\makeatletter
\renewcommand{\maketitle} {
	\begin{center} 
		\begin{minipage}[t]{.8\textwidth}
			\textsf{\huge\bfseries \@title} 
		\end{minipage}%
		\begin{minipage}[t]{.2\textwidth}
			\raggedleft \vspace{-1.65em}
			\textsf{\small \@author} \vfill
			\textsf{\small \@date}
		\end{minipage}
		\par
	\end{center}

	\thispagestyle{empty}
	\pagestyle{fancy}
}
\makeatother

% disponi teoremi
\usepackage{tcolorbox}
\newtcolorbox[auto counter, number within=section]{theorem}[2][]{%
	colback=blue!10, 
	colframe=blue!40!black, 
	sharp corners=northwest,
	fonttitle=\sffamily\bfseries, 
	title=Teorema~\thetcbcounter: #2, 
	#1
}

% disponi definizioni
\newtcolorbox[auto counter, number within=section]{definition}[2][]{%
	colback=red!10,
	colframe=red!40!black,
	sharp corners=northwest,
	fonttitle=\sffamily\bfseries,
	title=Definizione~\thetcbcounter: #2,
	#1
}

% disponi codice
\usepackage{listings}
\usepackage[table]{xcolor}

\definecolor{codegreen}{rgb}{0,0.6,0}
\definecolor{codegray}{rgb}{0.5,0.5,0.5}
\definecolor{codepurple}{rgb}{0.58,0,0.82}
\definecolor{backcolour}{rgb}{0.95,0.95,0.92}

\lstdefinestyle{codestyle}{
		backgroundcolor=\color{black!5}, 
		commentstyle=\color{codegreen},
		keywordstyle=\bfseries\color{magenta},
		numberstyle=\sffamily\tiny\color{black!60},
		stringstyle=\color{green!50!black},
		basicstyle=\ttfamily\footnotesize,
		breakatwhitespace=false,         
		breaklines=true,                 
		captionpos=b,                    
		keepspaces=true,                 
		numbers=left,                    
		numbersep=5pt,                  
		showspaces=false,                
		showstringspaces=false,
		showtabs=false,                  
		tabsize=2
}

\lstdefinestyle{shellstyle}{
		backgroundcolor=\color{black!5}, 
		basicstyle=\ttfamily\footnotesize\color{black}, 
		commentstyle=\color{black}, 
		keywordstyle=\color{black},
		numberstyle=\color{black!5},
		stringstyle=\color{black}, 
		showspaces=false,
		showstringspaces=false, 
		showtabs=false, 
		tabsize=2, 
		numbers=none, 
		breaklines=true
}


\lstdefinelanguage{assembler}{ 
  keywords={AAA, AAD, AAM, AAS, ADC, ADCB, ADCW, ADCL, ADD, ADDB, ADDW, ADDL, AND, ANDB, ANDW, ANDL,
        ARPL, BOUND, BSF, BSFL, BSFW, BSR, BSRL, BSRW, BSWAP, BT, BTC, BTCB, BTCW, BTCL, BTR, 
        BTRB, BTRW, BTRL, BTS, BTSB, BTSW, BTSL, CALL, CBW, CDQ, CLC, CLD, CLI, CLTS, CMC, CMP,
        CMPB, CMPW, CMPL, CMPS, CMPSB, CMPSD, CMPSW, CMPXCHG, CMPXCHGB, CMPXCHGW, CMPXCHGL,
        CMPXCHG8B, CPUID, CWDE, DAA, DAS, DEC, DECB, DECW, DECL, DIV, DIVB, DIVW, DIVL, ENTER,
        HLT, IDIV, IDIVB, IDIVW, IDIVL, IMUL, IMULB, IMULW, IMULL, IN, INB, INW, INL, INC, INCB,
        INCW, INCL, INS, INSB, INSD, INSW, INT, INT3, INTO, INVD, INVLPG, IRET, IRETD, JA, JAE,
        JB, JBE, JC, JCXZ, JE, JECXZ, JG, JGE, JL, JLE, JMP, JNA, JNAE, JNB, JNBE, JNC, JNE, JNG,
        JNGE, JNL, JNLE, JNO, JNP, JNS, JNZ, JO, JP, JPE, JPO, JS, JZ, LAHF, LAR, LCALL, LDS,
        LEA, LEAVE, LES, LFS, LGDT, LGS, LIDT, LMSW, LOCK, LODSB, LODSD, LODSW, LOOP, LOOPE,
        LOOPNE, LSL, LSS, LTR, MOV, MOVB, MOVW, MOVL, MOVSB, MOVSD, MOVSW, MOVSX, MOVSXB,
        MOVSXW, MOVSXL, MOVZX, MOVZXB, MOVZXW, MOVZXL, MUL, MULB, MULW, MULL, NEG, NEGB, NEGW,
        NEGL, NOP, NOT, NOTB, NOTW, NOTL, OR, ORB, ORW, ORL, OUT, OUTB, OUTW, OUTL, OUTSB, OUTSD,
        OUTSW, POP, POPL, POPW, POPB, POPA, POPAD, POPF, POPFD, PUSH, PUSHL, PUSHW, PUSHB, PUSHA, 
				PUSHAD, PUSHF, PUSHFD, RCL, RCLB, RCLW, MOVSL, MOVSB, MOVSW, STOSL, STOSB, STOSW, LODSB, LODSW,
				LODSL, INSB, INSW, INSL, OUTSB, OUTSL, OUTSW
        RCLL, RCR, RCRB, RCRW, RCRL, RDMSR, RDPMC, RDTSC, REP, REPE, REPNE, RET, ROL, ROLB, ROLW,
        ROLL, ROR, RORB, RORW, RORL, SAHF, SAL, SALB, SALW, SALL, SAR, SARB, SARW, SARL, SBB,
        SBBB, SBBW, SBBL, SCASB, SCASD, SCASW, SETA, SETAE, SETB, SETBE, SETC, SETE, SETG, SETGE,
        SETL, SETLE, SETNA, SETNAE, SETNB, SETNBE, SETNC, SETNE, SETNG, SETNGE, SETNL, SETNLE,
        SETNO, SETNP, SETNS, SETNZ, SETO, SETP, SETPE, SETPO, SETS, SETZ, SGDT, SHL, SHLB, SHLW,
        SHLL, SHLD, SHR, SHRB, SHRW, SHRL, SHRD, SIDT, SLDT, SMSW, STC, STD, STI, STOSB, STOSD,
        STOSW, STR, SUB, SUBB, SUBW, SUBL, TEST, TESTB, TESTW, TESTL, VERR, VERW, WAIT, WBINVD,
        XADD, XADDB, XADDW, XADDL, XCHG, XCHGB, XCHGW, XCHGL, XLAT, XLATB, XOR, XORB, XORW, XORL},
  keywordstyle=\color{blue}\bfseries,
  ndkeywordstyle=\color{darkgray}\bfseries,
  identifierstyle=\color{black},
  sensitive=false,
  comment=[l]{\#},
  morecomment=[s]{/*}{*/},
  commentstyle=\color{purple}\ttfamily,
  stringstyle=\color{red}\ttfamily,
  morestring=[b]',
  morestring=[b]"
}

\lstset{language=assembler, style=codestyle}

% disponi sezioni
\usepackage{titlesec}

\titleformat{\section}
	{\sffamily\Large\bfseries} 
	{\thesection}{1em}{} 
\titleformat{\subsection}
	{\sffamily\large\bfseries}   
	{\thesubsection}{1em}{} 
\titleformat{\subsubsection}
	{\sffamily\normalsize\bfseries} 
	{\thesubsubsection}{1em}{}

% tikz
\usepackage{tikz}

% float
\usepackage{float}

% grafici
\usepackage{pgfplots}
\pgfplotsset{width=10cm,compat=1.9}

% disponi alberi
\usepackage{forest}

\forestset{
	rectstyle/.style={
		for tree={rectangle,draw,font=\large\sffamily}
	},
	roundstyle/.style={
		for tree={circle,draw,font=\large}
	}
}

% disponi algoritmi
\usepackage{algorithm}
\usepackage{algorithmic}
\makeatletter
\renewcommand{\ALG@name}{Algoritmo}
\makeatother

% disponi numeri di pagina
\usepackage{fancyhdr}
\fancyhf{} 
\fancyfoot[L]{\sffamily{\thepage}}

\makeatletter
\fancyhead[L]{\raisebox{1ex}[0pt][0pt]{\sffamily{\@title \ \@date}}} 
\fancyhead[R]{\raisebox{1ex}[0pt][0pt]{\sffamily{\@author}}}
\makeatother

\begin{document}
% sezione (data)
\section{Lezione del 29-10-24}

% stili pagina
\thispagestyle{empty}
\pagestyle{fancy}

% testo
\subsection{Riduzione di campo di interi}
Vogliamo creare un circuito che passa dalla rappresentazione $A$ su $n+1$ cifre di un numero intero $a$, ad un $A^{RID}$ su $n$ cifre, che rappresenta sempre $a$. 
Chiaramente questo non è sempre possibile, e vale soltanto se:
$$
a \in \left[ -\frac{\beta^n}{2}, \frac{\beta^n}{2} - 1 \right] \subset \left[ -\frac{\beta^{n+1}}{2}, \frac{\beta^{n+1}}{2} - 1 \right]
$$

Avremo quindi bisogno di un flag di overflow \lstinline|ow|, per indicare la non rappresentabilità.
Verifichiamo da quanto visto sulle estensioni di campo, che i numeri che rispettano tale proprietà sono i tali per cui $\text{MSD} = 0$ e la cifra successiva $a_{n-1} < \frac{\beta}{2}$, e i tali per cui $\text{MSD} = \beta - 1$ e la cifra successiva $a_{n-1} \geq \frac{\beta}{2}$. Quindi:

$$
\mathtt{ow} = 0 \Leftrightarrow \left( a_n = 0 \wedge a_{n-1} < \frac{\beta}{2} \right) \vee \left( a_n = \beta - 1 \wedge a_{n-1} \geq \frac{\beta}{2} \right)
$$

Abbiamo sul grafico a farfalla adattato all'estensione su $n+1$ bit, che queste regole isolano le due sezioni del campo di numeri estesi ( $\left[ -\frac{\beta^{n+1}}{2}, \frac{\beta^{n+1}}{2} - 1 \right]$ ) che hanno riscontro nel campo ridotto ( $\left[ -\frac{\beta^n}{2}, \frac{\beta^n}{2} - 1 \right]$ ).

Quindi, in questo caso, il numero $a$ è rappresentabile su $n$ cifre, e si può calcolare il ridotto $A^{RID}$ semplicemente rimuovendo l'ultima cifra, cioè calcolando:
$$
A^{RID} = |A|_{\beta^n}
$$

Chiamiamo il circuito che riconosce la non riducibilità \textbf{circuito di overflow}.
In base 2, si ha rispetto alle cifre che $a_{n-1} < \frac{\beta}{2}$ vale se $a_{n-1} = 0$, e viceversa $a_{n-1} \geq \frac{\beta}{2}$ vale se $a_{n-1} = 1$, cioè un numero non è rappresentabile su $n-1$ bit se le sue due cifre più significative sono uguali. In questo modo il circuito si traduce in un confronto fra le due cifre più significative $a_n$ e $a_{n-1}$, che si fa con uno XOR.

In Verilog, questo si traduce come:

\lstinputlisting[language=verilog, style=codestyle]{../verilog/10-29/field_reducers/b2_field_reducer.v}

Per completezza, vediamo il circuito equivalente per la base 10 in codifica BCD, realizzato con un comparatore a 4 cifre binarie:

\lstinputlisting[language=verilog, style=codestyle]{../verilog/10-29/field_reducers/b10_field_reducer.v}

\subsubsection{Moltiplicazione di interi per potenza della base}
Vediamo come si realizza un moltiplicatore per $b = \beta \cdot a$, dato $A = \left(a{n-1} a_{n-2} ... a_0 \right)$ rappresentante $a$ su $n$ cifre, $B$ rappresentante $b$ su $n+1$ cifre.

Vogliamo chiederci prima di tutto se $b$ è sempre rapprsentabile da $B$ su $n+1$ cifre. Questo è vero, in quanto si può dimostrare che:
$$
B = \beta \cdot A
$$
Questo viene da:
$$
L: \quad B =
\begin{cases}
	b = \beta \cdot a, \quad 0 \leq a < \frac{\beta^n}{2} \\ 
	\beta^{n+1} + b = \beta^{n+1} + \beta \cdot a = \beta \cdot \left( \beta^n + a \right), \quad -\frac{\beta^n}{2} \leq a < 0
\end{cases}
$$
dove si nota che $a$ e $\beta^n + a$ valgono $A$ nei rispettivi campi di esistenza.
Si applica quindi quanto conoscevamo sulle moltiplicazioni per potenze di base su naturali, e il prodotto sta su $n+1$ cifre.

Per prodotti con potenze ulteriori della base, diciamo $\beta^k$, si ha che:
$$
b = \beta^k \cdot a \equiv B = \beta^k \cdot A
$$
e quindi il risultato starà su $n + k$ cifre.

\subsubsection{Divisione per potenza della base}
Vogliamo fare l'operazione equivalente per le divisioni, cioè dato $A$ rappresentante $a$ su $n+1$ cifre, trovare $B$ rappresentante $b$ su $n$ cifre tale per cui $b = \left\lfloor \frac{a}{\beta} \right\rfloor$.

Possiamo dimostrare, come prima, che:
$$
B = \left\lfloor \frac{A}{\beta} \right\rfloor
$$

Per fare ciò, approfittiamo della proprietà vista sul complemento a radice che ci permette di rappresentare $B$ come $|b|_{\beta^n}$:
$$
B = \left| \left\lfloor \frac{a}{\beta} \right\rfloor \right|_{\beta^n} = \left| \left\lfloor \frac{ \lfloor a/\beta^{n+1} \rfloor \cdot \beta^{n+1} + |a|_{\beta^n+1} }{\beta} \right\rfloor \right|_{\beta^n} = \left| \left\lfloor \lfloor a / \beta^{n+1} \rfloor \cdot \beta^n + \frac{|a|_{\beta^{n+1}}}{\beta} \right\rfloor  \right|_{\beta^n}
$$
$$
\left| \left\lfloor \frac{|a|_{\beta^{n+1}}}{\beta} \right\rfloor  \right|_{\beta^n}
= \left| \left\lfloor \frac{A}{\beta} \right\rfloor \right| = \left\lfloor \frac{A}{\beta} \right\rfloor
$$

Abbiamo quindi che possiamo sfruttare quanto avevamo detto sulla divisione per potenze di basi su naturali, e il quoziente sta su $n$ cifre.

Per divisioni con potenze ulteriori della base, diciamo $\beta^k$, si ha che:
$$
b = \left\lfloor \frac{a}{\beta^k} \right\rfloor \equiv B = \left\lfloor \frac{A}{\beta^k} \right\rfloor 
$$
e quindi il risultato starà su $n - k$ cifre (o $A$ dovrà stare su $n + k$ cifre rispetto a $B$, solita cosa).

\subsubsection{Note sugli shift logico e aritmetico}
Abbiamo visto come sono state definiti operazioni diverse per lo shift logico (SH) e aritmetico (SA) in linguaggio assembly.
Abbiamo visto adesso, però, che moltiplicazione e divisione per la base si fanno allo stesso modo sia su interi che su naturali.

Possiamo dire che, nel caso dello shift a sinistra, effettivamente le operazioni eseguite dal calcolatore sono uguali sia nel caso di SHL che SAR.
Per quanto riguarda lo shift a destra, invece, dobbiamo renderci conto che la $n-1$-esima cifra (quella che avevamo escluso dicendo che $A$ su $n+1$ cifre va in $B$ su $n$ cifre) resterà comunque nella locazione di memoria, cioè non si possono ridimensionare le locazioni.
C'è quindi una differenza sul modo in cui si popola l'$n-1$-esimo bit entrante: lo shift aritmetico SAL ripete il MSD (cioè estende l'intero su $n$ cifre) e lo shift logico SHL introduce sempre zeri (ergo perde i segni nel caso di $a$ ngativi).

\subsection{Somma di interi}
Dati $A$ e $B$ in base $\beta$ su $n$ cifre, rappresentanti rispettivamente gli interi $a$ e $b$, vogliamo calcolare $S$ su $n$ cifre tale $S$ rappresenta $s$ e $s = a + b$.
Abbiamo, che la somma potrebbe uscire dall'intervallo di rappresentabilità su $n$ cifre, in quanto sta su:
$$
-\beta^n \leq s \leq \beta^n -2
$$
e starebbe al massimo su $n + 1$ cifre:
$$
\left[ -\beta^n, \beta^n -2 \right] \subset \left[ -\frac{\beta^{n+1}}{2}, \frac{\beta^n+1}{2} - 1 \right]
$$

Abbiamo quindi bisogno di flag di overflow, \lstinline|ow|.

Quando $s$ è invece rappresentabile su $n$ cifre, si ha che:
$$
S = |s|_{\beta^n} = |a+b|_{\beta^n} = \left| |a|_{\beta^n} + |b|_{\beta^n} \right|_{\beta^n} = | A + B |_{\beta^n}
$$

Questa è la proprietà fondamentale per cui si usa il complemento alla radice, e lo passiamo dimostrare da quanto già dimostrato sulle proprietà dell'operatore modulo.

Si può quindi usare un sommatore (che è quindi indifferente per naturali e interi), e l'unico problema resta determinare il flag \lstinline|ow|.

L'unica cosa che dovremo aggiungere è un modo per calcolare il flag \lstinline|ow|.
Per adesso abbiamo dal sommatore l'uscita $C_{out}$, cioè il riporto della somma: questa non basta da sola a verificare la rappresentabilità del risultato.
Notiamo che la somma è sempre rappresentabile estendendo gli ingressi a $n+1$ bit e riducendo in uscita. Se la riduzione è possibile, ergo le ultime due cifre più significative sono diverse, allora la somma è rappresentabile.

In una base arbitraria, per fare ciò devo effettivamente fare lo XOR delle due cifre, mentre in binario posso sfruttare le proprietà dello XOR, ricordando che internamente al full adder, la cifra in uscita $s$ non è altro che $a \oplus b \oplus c$. Si ha quindi:
$$
\mathtt{ow} = s_n \oplus s_{n-1} = (a_{n} \oplus b_{n} \oplus c_{n}) \oplus (a_{n-1} \oplus b_{n-1} \oplus c_{n-1}) = 0 \oplus c_n \oplus c_{n-1} = c_n \oplus c_{n-1}
$$

Possiamo quindi ricavare il flag \lstinline|ow| confrontando il $C^{n}_{out}$ con l'$n-1$-esimo $C^{n-1}_{out}$: a bit uguali sia \lstinline|ow| falso, e viceversa, cioè si usa un singolo XOR

Si ha quindi che la stessa circuiteria esegue somme sia fra interi che fra naturali.
In assembly, avevamo visto che la ADD esegue le stesse operazioni, ed è compito del programmatore controllare i flag di carry o di overflow a seconda di ciò che era andato a sommare (interi $\rightarrow$ overflow, naturali $\rightarrow$ carry).

Possiamo quindi aggiornare l'implementazione Verilog del sommatore a 4 cifre binarie per tenere conto di un flag di overflow, come segue:

\lstinputlisting[language=verilog, style=codestyle]{../verilog/10-29/adders_i/n4_b2_adder_i.v}

\subsection{Sottrazione di interi}
La sottrazione fra interi è analoga alla somma: abbiamo sempre due $A$ e $B$ in base $\beta$ su $n$ cifre, e vogliamo trovare $D$ sempre su $n$ cifre tale per cui fra $a$, $b$ e $d$ rappresentati vale $a - b = d$.
Si ha, prendendo il complemento a radice:
$$  
D = |d|_{\beta^n} = |a-b|_{\beta^n} = \left| a - b \right|_{\beta^{n}} = \left| |a|_{\beta^n} - |b|_{\beta^n} \right| = |A - B|_{\beta^n} = |A + \overline{B} + 1|_{\beta^n}
$$

Come prima, abbiamo che il flag \lstinline|ow| è dato dallo XOR degli ultimi due prestiti (prima erano riporti).
Questo si dimostra analogamente a prima, prendendo il bit esteso:
$$
D^{EST} = |d|_{\beta^{n+1}} = |a-b|_{\beta^{n+1}} = \left| |a|_{\beta^{n+1}} - |b|_{\beta^{n+1}} \right|_{\beta^{n+1}} 
$$
$$
= |A^{EST} - B^{EST}|_{\beta^{n+1}} = |A^{EST} + \overline{B^{EST}} -1|_{\beta^{n+1}}
$$
cioè si ha che sull'$n+1$-esimo bit la differenza è uguale prendendo le estensioni degli ingressi su $n+1$ bit, ergo la rappresentabilità è data dalla riducibilità del risultato su $n$ bit, e quindi come prima dallo XOR sugli ultimi due prestiti.

In Verilog, posso aggiornare il sottrattore a 4 cifre binarie come segue:

\lstinputlisting[language=verilog, style=codestyle]{../verilog/10-29/subtractors_i/n4_b2_subtractor_i.v}

\subsection{Comparazione di numeri interi}
Notiamo che c'è una differenza fra la comparazione fra interi e quella fra naturali.
Per quanto riguarda l'uguaglianza $a = b$, abbiamo effettivamente la stessa cosa dei naturali.

Invece, per la minoranza $a < b$, non possiamo più controllare i prestiti uscenti.
Dobbiamo quindi guardare il segno del risultato della sottrazione, che deve quindi poter essere svolta: si estende su $n+1$ cifre e si controlla la $n$ esima cifra del risultato: questa varrà da $\mathrm{sgn}(a-b)$, e quindi da flag di minoranza per $a < b$.
Se non si fosse esteso su $n+1$ cifre, non avremmo potuto essere sicuri di non aver scartato eventuali valori negativi (negli intervalli di non rappresentabilità).

Possiamo quindi definire un comparatore per interi in Verilog:

\lstinputlisting[language=verilog, style=codestyle]{../verilog/10-29/subtractors_i/n4_b2_integer_comparator.v}

notando che il modulo \lstinline|n5_b2_subtractor| non è altro che un sottrattore a 5 cifre binarie, che come sempre è implementato nel codice Verilog annesso alla lezione (directory \lstinline|/verilog|).

\subsection{Moltiplicazione e divisione di interi}
Moltiplicazioni e divisioni di interi riescono più facili se prima si converte in rappresentazione modulo e segno: i moduli si moltiplicano o dividono come naturali, e il segno viene determinato dai segni degli operandi attraverso la comune algebra alternante ($+ \cdot + = +$, $+ \cdot - = -$, $- \cdot - = +$).

Ricordiamo di aver già visto un circuito di conversione da CR a MS.
Ci manca quindi il circuito di conversione opposto:

\subsubsection{Conversone da MS a CR}
Vogliamo una rete che prende in ingresso il valore assoluto su $n$ cifre ed il segno della rappresentazione di un numero intero, e produce un uscita la sua rappresentazione in complemento alla radice su $n$ cifre.
Quest'operazione non è sempre possibile: abbiamo $-(\beta^n - 1) \leq a \leq \beta^n - 1$ in ingresso e $-\frac{\beta^n}{2} \leq a \leq \frac{\beta^n}{2} - 1$ in uscita.

Se l'operazione è fattibile, avremo che:
$$
A = |a|_{\beta^n} =
\begin{cases}
	|ABS_a|_{\beta^n}, \quad a \geq 0 \\ 
	|-ABS_a|_{\beta^n}, \quad a < 0 \\ 
\end{cases} = 
\begin{cases}
	|ABS_a|_{\beta^n}, \quad a \geq 0 \\ 
	|\overline{ABS_a} + 1|_{\beta^n}, \quad a < 0 \\ 
\end{cases} 
$$

Quindi si usa un multiplexer, con il segno della rappresentazione MS a variabile di controllo, che distingue fra la rappresentazione stessa $ABS_a$ e il suo complemento (calcolato con un circuito di inversione e incremento).

Per quanto riguarda l'overflow, abbiamo invece che \lstinline|ow| è impostato in due casi:
\begin{itemize}
	\item Siamo fuori dal campo di rappresentabilità: questo si verifica quando $\mathrm{abs}(a) > \frac{\beta^N}{2}$, cioè si è passati oltre agli $n-1$ bit su cui dobbiamo ridurre il modulo;
	\item Si è sull'unico valore positivo che non ha rappresentazione con $a = \frac{\beta^n}{2}$ e il bit di segno vale 0, cioè $a$ è uguale al massimo rappresentabile $\frac{\beta^n}{2} - 1 + 1$. 
\end{itemize}

Questo si sintetizza nella regola, espressa attraverso l'operatore ternario:
$$
\mathtt{ow} = \left( \left( \mathrm{abs}(a) > \frac{\beta^n}{2} \right) \vee \left( \left( \mathrm{abs}(a) = \frac{\beta^n}{2} \right) \wedge \left( \mathrm{sgn}(a) = 1 \right) \right) \right) \ ? \ 1 : 0
$$

Occorre stare attenti fra la funzione segno e il valore del bit di segno, in quanto vale, come per la convenzione solita sul bit di segno:
\[
	\begin{cases}			
\mathrm{sgn}(a) = 1 \Rightarrow \mathtt{sgn} = 0 \\
\mathrm{sgn}(a) = -1 \Rightarrow \mathtt{sgn} = 1
	\end{cases}
\]

In Verilog questa rete si presenta come la duale del \lstinline|n4_c2_ms_converter|, e si usa il modulo negatore sintetizzato prima:

\lstinputlisting[language=verilog, style=codestyle]{../verilog/10-24/c2_ms_converters/n4_ms_c2_converter.v}

\par\medskip

Riassumiamo brevemente come si svolge la conversione fra complemento a radice e rappresentazione modulo e segno.
Abbiamo sostanzialmente che, salvo il caso della conversione da MS a CR dove si può incappare in non rappresentabilità, la conversione da CR a MS e viceversa si fa sempre con un multiplexer che distingue fra la rappresentazione $A$ presa così com'è e il suo complemento calcolato con inversione incremento. 
In particolare:
\begin{itemize}
	\item Nel caso \textbf{CR a MS}, si prende $A$ se per la cifra più significativa $a_{n-1}$ vale $a_{n-1} \geq \frac{\beta}{2}$, altrimenti il complemento $\overline{A}$: questo significa che il multiplexer è pilotato dalla MSD;
	\item Nel caso \textbf{MS a CR}, si prende $A$ se il bit di segno non è impostato, e il complemento $\overline{A}$ altrimenti: questo significa che il multiplexer è pilotato dal bit di segno.
\end{itemize}

\subsubsection{Moltiplicazione}
Per svolgere la moltiplicazione vogliamo quindi trasformare due ingressi $A$ e $B$, rappresentanti gli interi $a$ e $b$ su $n$ e $m$ bit, nella loro rappresentazione MS come:
$$
a \Rightarrow \mathrm{sgn}(a), \mathrm{abs}(a), \quad 
a \Rightarrow \mathrm{sgn}(a), \mathrm{abs}(a), 
$$
dove i segni stanno su un bit e i moduli su $n$ e $m$ bit.

Abbiamo che la moltiplicazione dei moduli di $A$ e $B$ è sempre rappresentabile, in quanto:
$$
\mathrm{sgn}(A) \cdot \mathrm{sgn}(B) \leq \frac{\beta^n}{2} \cdot \frac{\beta^m}{2} = \frac{\beta^{n+m}}{2}
$$

Osserviamo quindi che il prodotto intero $p$ è:
$$
p =
\begin{cases}
	\mathrm{abs}(a) \cdot \mathrm{abs}(b), \quad \mathrm{sgn}(a) = \mathrm{sgn}(b) \\ 
	-\mathrm{abs}(a) \cdot \mathrm{abs}(b), \quad \mathrm{sgn}(a) \neq \mathrm{sgn}(b)
\end{cases}
$$
ergo:
\[
	\begin{cases}
		\mathrm{abs}(p) = \mathrm{abs}(a) \cdot \mathrm{abs}(b) \\ 	
		\mathrm{sgn}(p) = \mathrm{sgn}(a) \cdot \mathrm{sgn}(b)
	\end{cases}
\]
cioè quanto avevamo detto sulla rappresentazione modulo e segno per i prodotti. 

Si ha quindi che il moltiplicatore fra interi si realizza convertendo gli ingressi da CR a MS, mandando i valori assoluti ad un moltiplicatore per naturali, e ricavando il segno del successivo convertitore da MS a CR (che ci darà il risultato) da uno XOR fra i segni degli MS in ingresso.
L'overflow non è considerato in quanto non potra mai verificarsi (da sopra).

In Verilog, quindi, usiamo le definizioni date precedentemente di convertitori fra modulo e segno e complemento a 2:

\lstinputlisting[language=verilog, style=codestyle]{../verilog/10-29/integer_multipliers/n4by4_b2_integer_multiplier.v}

Il modulo \lstinline|n8_ms_c2_converter|, in particolare, è un convertitore da modulo e segno a complemento a 2 su 8 cifre binarie (si è data, nella cartella \lstinline|/verilog|, sia un implementazione a 4 che a 8 cifre).

\subsubsection{Divisione}
Vogliamo calcolare, dati due naturali $A$ e $B$ rappresentanti gli interi $a$ e $b$ su $n$ e $m$ cifre, il quoziente $Q$ e il resto $R$, rispettivamente su $n$ e $m$ cifre, tali che:
$$
a = q \cdot b + r
$$

Per svolgere questa divisione abbiamo bisogno di una riformulazione del teorema della divisione con resto che funzioni sull'anello $\mathbb{Z}$, in quanto adesso la semplice $a = q \cdot b + r$ con $r < b$ ammette infiniti valori di $r$ (che può essere negativo).
Decidiamo quindi di imporre:
\begin{itemize}
	\item Il quoziente $q$ è positivo se i segni di $a$ e $b$ sono concordi, e negativo viceversa;
	\item $r$ e $b$ sono uguali in segno.
\end{itemize}

Da qui si ha la proprietà più importante, cioè:
\[
	\begin{cases}
		\mathrm{abs}(r) < \mathrm{abs}(b) \\ 
		\mathrm{sgn}(r) = \mathrm{sgn}(b)
	\end{cases}
\]

Si verifica che questo significa che vogliamo i risultati che ci aspettiamo dalla comune divisione fra interi.

Abbiamo quindi, pensando in modulo e segno, che:
$$
a = q \cdot b + r 
$$
diventa:
$$
\mathrm{sgn}(a) \cdot \mathrm{abs}(a) = q \cdot \mathrm{sgn}(b) \cdot \mathrm{abs}(b) + \mathrm{sgn}(r) \cdot \mathrm{abs}(r)
$$
Ma se avevamo $\mathrm{sgn}(a) = \mathrm{sgn}(r)$, allora:
$$
\mathrm{abs}(a) = \left( q \cdot \mathrm{sgn}(b) \cdot \mathrm{sgn}(a) \right) \cdot \mathrm{abs}(b) + \mathrm{abs}(r)
$$

Si nota quindi che $ q \cdot \mathrm{sgn}(b) \cdot \mathrm{sgn}(a) $ è semplicemente $\mathrm{abs}(q)$, ergo si può rendere la divisione fra interi come la divisione fra i moduli di quegli interi, prendendo il segno separatamente (che è quello che avevamo fatto per la moltiplicazione).

Resta da trovare il valore del flag di non fattibilità \lstinline|no_div|.
Abbiamo, dal divisore fra naturali, che la rappresentabilità è data da:
$$
\mathrm{abs}(a) < \beta^n \cdot \mathrm{abs}(q)
$$

Questa condizione non basta, in quanto non si è ancora assicurato che l'intero in uscita sia rappresentabile su $n$ cifre.
Si prende quindi anche il flag di overflow \lstinline|ow|, ricavato dalla conversione finale da MS a CR, e si mette a OR con il flag \lstinline|no_div| che abbiamo ricavato dal circuito divisore.

Tutto questo si traduce nell'ultima rete puramente combinatoria che vedremo, ovvero:

\lstinputlisting[language=verilog, style=codestyle]{../verilog/10-29/integer_dividers/n4by2_b2_integer_divider.v}

\end{document}


\documentclass[a4paper,11pt]{article}
\usepackage[a4paper, margin=8em]{geometry}

% usa i pacchetti per la scrittura in italiano
\usepackage[french,italian]{babel}
\usepackage[T1]{fontenc}
\usepackage[utf8]{inputenc}
\frenchspacing 

% usa i pacchetti per la formattazione matematica
\usepackage{amsmath, amssymb, amsthm, amsfonts}

% usa altri pacchetti
\usepackage{gensymb}
\usepackage{hyperref}
\usepackage{standalone}

\usepackage{colortbl}

\usepackage{xstring}
\usepackage{karnaugh-map}

% imposta il titolo
\title{Appunti Reti Logiche}
\author{Luca Seggiani}
\date{2024}

% imposta lo stile
% usa helvetica
\usepackage[scaled]{helvet}
% usa palatino
\usepackage{palatino}
% usa un font monospazio guardabile
\usepackage{lmodern}

\renewcommand{\rmdefault}{ppl}
\renewcommand{\sfdefault}{phv}
\renewcommand{\ttdefault}{lmtt}

% circuiti
\usepackage{circuitikz}
\usetikzlibrary{babel}

% testo cerchiato
\newcommand*\circled[1]{\tikz[baseline=(char.base)]{
            \node[shape=circle,draw,inner sep=2pt] (char) {#1};}}

% disponi il titolo
\makeatletter
\renewcommand{\maketitle} {
	\begin{center} 
		\begin{minipage}[t]{.8\textwidth}
			\textsf{\huge\bfseries \@title} 
		\end{minipage}%
		\begin{minipage}[t]{.2\textwidth}
			\raggedleft \vspace{-1.65em}
			\textsf{\small \@author} \vfill
			\textsf{\small \@date}
		\end{minipage}
		\par
	\end{center}

	\thispagestyle{empty}
	\pagestyle{fancy}
}
\makeatother

% disponi teoremi
\usepackage{tcolorbox}
\newtcolorbox[auto counter, number within=section]{theorem}[2][]{%
	colback=blue!10, 
	colframe=blue!40!black, 
	sharp corners=northwest,
	fonttitle=\sffamily\bfseries, 
	title=Teorema~\thetcbcounter: #2, 
	#1
}

% disponi definizioni
\newtcolorbox[auto counter, number within=section]{definition}[2][]{%
	colback=red!10,
	colframe=red!40!black,
	sharp corners=northwest,
	fonttitle=\sffamily\bfseries,
	title=Definizione~\thetcbcounter: #2,
	#1
}

% disponi codice
\usepackage{listings}
\usepackage[table]{xcolor}

\definecolor{codegreen}{rgb}{0,0.6,0}
\definecolor{codegray}{rgb}{0.5,0.5,0.5}
\definecolor{codepurple}{rgb}{0.58,0,0.82}
\definecolor{backcolour}{rgb}{0.95,0.95,0.92}

\lstdefinestyle{codestyle}{
		backgroundcolor=\color{black!5}, 
		commentstyle=\color{codegreen},
		keywordstyle=\bfseries\color{magenta},
		numberstyle=\sffamily\tiny\color{black!60},
		stringstyle=\color{green!50!black},
		basicstyle=\ttfamily\footnotesize,
		breakatwhitespace=false,         
		breaklines=true,                 
		captionpos=b,                    
		keepspaces=true,                 
		numbers=left,                    
		numbersep=5pt,                  
		showspaces=false,                
		showstringspaces=false,
		showtabs=false,                  
		tabsize=2
}

\lstdefinestyle{shellstyle}{
		backgroundcolor=\color{black!5}, 
		basicstyle=\ttfamily\footnotesize\color{black}, 
		commentstyle=\color{black}, 
		keywordstyle=\color{black},
		numberstyle=\color{black!5},
		stringstyle=\color{black}, 
		showspaces=false,
		showstringspaces=false, 
		showtabs=false, 
		tabsize=2, 
		numbers=none, 
		breaklines=true
}


\lstdefinelanguage{assembler}{ 
  keywords={AAA, AAD, AAM, AAS, ADC, ADCB, ADCW, ADCL, ADD, ADDB, ADDW, ADDL, AND, ANDB, ANDW, ANDL,
        ARPL, BOUND, BSF, BSFL, BSFW, BSR, BSRL, BSRW, BSWAP, BT, BTC, BTCB, BTCW, BTCL, BTR, 
        BTRB, BTRW, BTRL, BTS, BTSB, BTSW, BTSL, CALL, CBW, CDQ, CLC, CLD, CLI, CLTS, CMC, CMP,
        CMPB, CMPW, CMPL, CMPS, CMPSB, CMPSD, CMPSW, CMPXCHG, CMPXCHGB, CMPXCHGW, CMPXCHGL,
        CMPXCHG8B, CPUID, CWDE, DAA, DAS, DEC, DECB, DECW, DECL, DIV, DIVB, DIVW, DIVL, ENTER,
        HLT, IDIV, IDIVB, IDIVW, IDIVL, IMUL, IMULB, IMULW, IMULL, IN, INB, INW, INL, INC, INCB,
        INCW, INCL, INS, INSB, INSD, INSW, INT, INT3, INTO, INVD, INVLPG, IRET, IRETD, JA, JAE,
        JB, JBE, JC, JCXZ, JE, JECXZ, JG, JGE, JL, JLE, JMP, JNA, JNAE, JNB, JNBE, JNC, JNE, JNG,
        JNGE, JNL, JNLE, JNO, JNP, JNS, JNZ, JO, JP, JPE, JPO, JS, JZ, LAHF, LAR, LCALL, LDS,
        LEA, LEAVE, LES, LFS, LGDT, LGS, LIDT, LMSW, LOCK, LODSB, LODSD, LODSW, LOOP, LOOPE,
        LOOPNE, LSL, LSS, LTR, MOV, MOVB, MOVW, MOVL, MOVSB, MOVSD, MOVSW, MOVSX, MOVSXB,
        MOVSXW, MOVSXL, MOVZX, MOVZXB, MOVZXW, MOVZXL, MUL, MULB, MULW, MULL, NEG, NEGB, NEGW,
        NEGL, NOP, NOT, NOTB, NOTW, NOTL, OR, ORB, ORW, ORL, OUT, OUTB, OUTW, OUTL, OUTSB, OUTSD,
        OUTSW, POP, POPL, POPW, POPB, POPA, POPAD, POPF, POPFD, PUSH, PUSHL, PUSHW, PUSHB, PUSHA, 
				PUSHAD, PUSHF, PUSHFD, RCL, RCLB, RCLW, MOVSL, MOVSB, MOVSW, STOSL, STOSB, STOSW, LODSB, LODSW,
				LODSL, INSB, INSW, INSL, OUTSB, OUTSL, OUTSW
        RCLL, RCR, RCRB, RCRW, RCRL, RDMSR, RDPMC, RDTSC, REP, REPE, REPNE, RET, ROL, ROLB, ROLW,
        ROLL, ROR, RORB, RORW, RORL, SAHF, SAL, SALB, SALW, SALL, SAR, SARB, SARW, SARL, SBB,
        SBBB, SBBW, SBBL, SCASB, SCASD, SCASW, SETA, SETAE, SETB, SETBE, SETC, SETE, SETG, SETGE,
        SETL, SETLE, SETNA, SETNAE, SETNB, SETNBE, SETNC, SETNE, SETNG, SETNGE, SETNL, SETNLE,
        SETNO, SETNP, SETNS, SETNZ, SETO, SETP, SETPE, SETPO, SETS, SETZ, SGDT, SHL, SHLB, SHLW,
        SHLL, SHLD, SHR, SHRB, SHRW, SHRL, SHRD, SIDT, SLDT, SMSW, STC, STD, STI, STOSB, STOSD,
        STOSW, STR, SUB, SUBB, SUBW, SUBL, TEST, TESTB, TESTW, TESTL, VERR, VERW, WAIT, WBINVD,
        XADD, XADDB, XADDW, XADDL, XCHG, XCHGB, XCHGW, XCHGL, XLAT, XLATB, XOR, XORB, XORW, XORL},
  keywordstyle=\color{blue}\bfseries,
  ndkeywordstyle=\color{darkgray}\bfseries,
  identifierstyle=\color{black},
  sensitive=false,
  comment=[l]{\#},
  morecomment=[s]{/*}{*/},
  commentstyle=\color{purple}\ttfamily,
  stringstyle=\color{red}\ttfamily,
  morestring=[b]',
  morestring=[b]"
}

\lstset{language=assembler, style=codestyle}

% disponi sezioni
\usepackage{titlesec}

\titleformat{\section}
	{\sffamily\Large\bfseries} 
	{\thesection}{1em}{} 
\titleformat{\subsection}
	{\sffamily\large\bfseries}   
	{\thesubsection}{1em}{} 
\titleformat{\subsubsection}
	{\sffamily\normalsize\bfseries} 
	{\thesubsubsection}{1em}{}

% tikz
\usepackage{tikz}

% float
\usepackage{float}

% grafici
\usepackage{pgfplots}
\pgfplotsset{width=10cm,compat=1.9}

% disponi alberi
\usepackage{forest}

\forestset{
	rectstyle/.style={
		for tree={rectangle,draw,font=\large\sffamily}
	},
	roundstyle/.style={
		for tree={circle,draw,font=\large}
	}
}

% disponi algoritmi
\usepackage{algorithm}
\usepackage{algorithmic}
\makeatletter
\renewcommand{\ALG@name}{Algoritmo}
\makeatother

% disponi numeri di pagina
\usepackage{fancyhdr}
\fancyhf{} 
\fancyfoot[L]{\sffamily{\thepage}}

\makeatletter
\fancyhead[L]{\raisebox{1ex}[0pt][0pt]{\sffamily{\@title \ \@date}}} 
\fancyhead[R]{\raisebox{1ex}[0pt][0pt]{\sffamily{\@author}}}
\makeatother

\begin{document}
% sezione (data)
\section{Lezione del 30-10-24}

% stili pagina
\thispagestyle{empty}
\pagestyle{fancy}

% testo
\subsection{La funzione di memoria}
Finora abbiamo visto reti combinatorie, cioè \textbf{reti prive di memoria}, dove lo stato di uscita ad un istante dipende solo dallo stato di ingresso corrente.
Nelle \textbf{reti sequenziali}, invece, l'uscita dipende dalla sequenza degli stati di ingresso visti dalla rete fino a quel momento. 
Questa memoria si implementa attraverso \textbf{anelli di retroazione}.

Prendiamo ad esempio un buffer con un anello di retrazione, cioè una linea che porta la sua uscita al suo ingresso, e che estrae in uscita $q$.

Questo potrà quindi esistere in due situazioni di stabilità:
\begin{itemize}
	\item L'uscita vale 0, e va in ingresso al buffer, dove si \textbf{rigenera} (o si \textit{autosostiene});
	\item L'uscita vale 1, e va in ingresso al buffer, dove ancora una volta si rigenera e mantiene il suo valore.
\end{itemize}

La presenza del buffer è fondamentale: mantiene l'uscita $q$ a 0 e 1, e sopratutto ci assicura di poter associare a quel punto della rete uno stato logico.

Il problema di una rete di questo genere è che è fondamentalmente inutile: non si può controllare lo stato di stabilità del buffer, a quindi non si possono immagazzinare bit diversi a tempi diversi.

\subsubsection{Uscita negata}
Realizziamo allora il nostro buffer, sostituendolo cone due porte NOR disposte come invertitori (quindi un doppio invertitore, che equivale al buffer).
Si ha che fra le due porte NOR abbiamo il valore complementato del buffer, cioè 1 a 0 e 0 a 1.
Possiamo quindi dotare la rete di un'ulteriore uscita $q_N$, che equivale appunto alla negazione di $q$.
Per questo motivo, avevamo detto, nella valutazione dei livelli di logica si ignorano le porte NOT: solitamente abbiamo già un valore negato a disposizione dai registri.

\subsubsection{Stato all'accensione}
Ora, se all'accensione $q$ e $q_N$ sono discordi, la rete si troverà già in uno degli stati stabili, e lì restera.
In caso contrario, se sono concordi, teoricamente ciascuna delle due uscite dovrebbe oscillare all'infinito, con un periodo pari al doppio del tempo di risposta delle porte (in due passaggi si completa un ciclo, cioè l'ingresso della prima porta torna al neutro).
Nella pratica, la rete si stabilizza, in quanto il tempo delle porte sarà necessariamente diverso e quindi si creerà prima o poi una condizione analoga a prima, dove le uscite sono discordi e la rete resta stabile. 

\subsubsection{Latch SR}
Vediamo quini come rendere pilotabile lo stato del circuito.
Introduciamo due ingressi negli input (finora duplicati) delle porte NOR: in entrata alla prima porta avremo il comando S, per SET, e in entranta alla seconda porta avremo il comando R, per RESET.
Questi ingressi sono \textit{attivi alti}, cioè i comandi S e R vengono dati quando le rispettive entrate sono in tensione.
Chiamiamo questa rete \textbf{latch SR}, a volte impropriamente detta \textit{flip-flop SR}.

Vediamo il funzionamento della rete nei diversi casi di attivazione degli ingressi:
\begin{itemize}
	\item $S=1$, $R=0$: si ha che la prima NOR ha un ingresso 1, ergo mette l'uscita a 0. Quindi, la seconda NOR ha un ingresso 0, ergo mette l'uscita a 1. Ci troviamo nella configurazione stabile $q=1$, $q_N = 0$, cioè abbiamo memorizzato un bit. 
	\item $S=0$, $R=1$: si ha che la seconda NOR ha un ingresso 1, ergo mette l'uscita a 0. Quindi, la prima NOR ha un ingresso 0, quindi mette l'uscita a 1. Ci troviamo nella configurazione stabile $q=0$, $q_N = 1$, cioè abbiamo resettato un bit.
	\item $S=0$, $R=0$: l'uscita della prima NOR vale 0 se $q=1$, e 1 se $q=0$, quindi $q_N$ dà semplicemente $\bar{q}$ e viceversa, e la rete conserva il valore che aveva precedentemente. Questo comportamento rende la rete \textbf{sequenziale}: nello stato di \textbf{conservazione}, cioè quello a ingressi disattivati, si ha che la rete rimane nello stato stabile $S_0$ o $S_1$ nel quale si era portata in un momento precedente nella sequenza di stati.
		Si può anche dire che la rete \textbf{ricorda} l'ultimo SET o RESET ricevuto.
		Comunque, è una rete \textbf{asincrona}, in quanto l'uscita si aggiorna subito rispetto agli ingressi (e non in sincronia ad un clock).
	\item $S=1$, $R=1$: semanticamente, questa istruzione non ha molto significato. In uno stato di pilotaggio corretto, diciamo che questo stato \textbf{non è permesso}.
		Se si venisse a verificare, avremmo che alla prima porta un entrata è 1, e quindi l'uscita è 0. Alla seconda porta, quindi, un'uscita sarà 1, e avremo di nuovo uscita 0.
		Forzeremmo quindi la rete in uno stato $q=0$, $q_N=0$, che non significa nulla dal punto di vista della rappresentazione in bit della memoria.
\end{itemize}

In Verilog, possiamo descrivere il latch SR come segue:

\lstinputlisting[language=verilog, style=codestyle]{../verilog/10-30/sr_latches/sr_latch.v}

\subsubsection{Tabella di applicazione}
Per descrivere il comportamento delle reti con memoria usiamo le \textbf{tabelle di applicazione}.
Queste rappresentano, a sinistra, il valore attuale della variabile e il valore successivo che si vuole questa assuma, e a destra il comando necessario perchè l'uscita passi dal valore attuale a quello successivo.
Nel caso del latch SR, si ha che questa è:
\begin{table}[h!]
	\center 
	\begin{tabular} { c c | c c }
		\bfseries $q$ & \bfseries $q'$ & \bfseries $s$ & \bfseries $r$ \\
		\hline 
		0 & 0 & 0 & - \\ 
		0 & 1 & 1 & 0 \\ 
		1 & 0 & 0 & 1 \\ 
		1 & 1 & - & 0 
	\end{tabular}
\end{table}

\subsubsection{Regole di pilotaggio}
Avevamo visto le regole per le reti combinatorie:
\begin{itemize}
	\item Siamo in \textbf{pilotaggio in modo fondamentale}: si cambiano gli ingressi solo quando la rete è a regime;
	\item Gli stati di ingresso consecutivi devono essere adiacenti (per evitare \textit{race condition}).
\end{itemize}

Vogliamo definire una serie di regole simili per le reti sequenziali.
Abbiamo che la regola di \textbf{pilotaggio in modo fondamentale} va rispettata comunque: la rete avrà un certo \textbf{tempo di attraversamento} di cui tenere conto.
Anche la seconda regola, degli \textbf{stati di ingressi consecutivi adiacenti}, è fondamentale: se non viene rispettata, si possono presentare in ingresso stati transitori spuri, e l'evoluzione delle uscite diventa imprevedebile.

Nel latch SR, però, vale che questa legge può essere violata: cioè il latch SR è robusto nei confronti di pilotaggi scorretti.
Questo è il punto di forza che lo rende la rete alla base dei registri e di tutti gli elementi di memoria.

\subsubsection{Lo stato iniziale}
Abbiamo detto che l'SR è l'elemento alla base dei circuiti di memoria.
Un SR può contenere informazioni, che corrispondono allo \textit{stato} $S_0$ o $S_1$ in cui si trova.
Si ha, però, che all'accensione il bit contenuto nell'SR è \textbf{casuale} (da quello che avevamo visto dalle modalità di pilotaggio).
All'accensione di un calcolatore, si ha che alcuni elementi possono avere un valore casuale (ad esempio la RAM), altri no (ad esmpio l'instruction pointer).
Si definisce quindi una \textbf{fase di reset}, distinta dalla \textbf{fase operativa}, cioè quella di operatività standard.
Nella fase di reset si inizializzano gli elementi di memoria: notiamo che questo reset non corrisponde al comando R, di RESET, che diamo ai latch.
In generale, quindi, non è vero che gli elementi di memoria contengono tutti zero all'accensione del calcolatore.

Vediamo quindi il circuito:
\begin{center}
	\begin{circuitikz}
		\draw (0, 0) node[spdt, xscale=-1] (on) {};	
		\draw (0, -2) node[spdt, xscale=-1] (reset) {};
		\draw (on.out 2) node[ground] {};
		\draw (reset.out 2) node[ground] {};
		\node[anchor=east] at (on.out 1) {Vcc};

		\draw (on.in) -- (2, 0)
			to [ resistor, l = $R$ ] (2, -2)
			to [ capacitor, l = $C$] (2, -3);
		\draw (2, -3) node[ground] {};
		\draw (4, -2) node [ schmitt ] (schm) {};
		\draw (reset.in) -- (schm.in);
		\node[anchor=west] at (4.75, -2) {/reset};

		\node at (0, 0.75) {ON/OFF};
		\node at (0, -1.25) {reset};
	\end{circuitikz}
\end{center}
Abbiamo che la circuiteria di trigger è realizzata attraverso un circuito RC, fra l'interruttore ON/OFF e il pulsante di reset, con $\tau = R \cdot C \approx 0 \, \mathrm{\mu s}$, dove si collega il nodo fra R e C ad un trigger di Schmitt.
Il trigger di Schmitt effettivamente "quantizza" la tensione, cioè scatta ad un valore 1 di tensione solamente quando la tensione in entranta è maggiore di una certa soglia.
Abbiamo quindi che, spostando l'interruttore nella posizione ON, il circuito raggiunge il regime in un tempo $\approx \tau$, e quindi il trigger va a 1 in un tempo $\approx \tau$. Lo stesso quando si preme il pulsante di reset, il capacitore C si scarica e dobbiamo riportare nuovamente il circuito a regime, per cui abbiamo un istante $\approx \tau$ dove il trigger è a 0. 

Abbiamo che l'uscita di questa rete va ad un ingresso detto /reset (che ricordiamo è distinto dai singoli reset dei latch SR), che è \textit{attivo basso}: cioè nella fase iniziale dell'accensione, e ad ogni pressione successiva del pulsante reset, si ha che dal trigger esce per un tempo $\approx \tau$ il comando di /reset.

Per implementare effettivamente il meccanismo di reset si dota il latch SR di due ingressi aggiuntivi: /preset e /preclear, entrambi attivi bassi.
Si distinguono quindi i seguenti casi:
\begin{itemize}
	\item $\mathrm{/preset} = \mathrm{/preclear} = 1$: la rete si comporta come un latch SR normale;
	\item $\mathrm{/preset} = 0$: la rete si trova nello stato $S_1$ (indipendentemente dallo stato di $s$ e $r$);
	\item $\mathrm{/preclear} = 0$: la rete si trova nello stato $S_0$ (indipendentemente dallo stato di $s$ e $r$);
	\item $\mathrm{/preset} = \mathrm{/preclear} = 0$: abbiamo, come nel caso già visto dei semplici ingressi Set e Reset, che questo stato non è permesso, e quindi non è interessante conoscere il funzionamento della rete in tale stato.
\end{itemize}

Abbiamo quindi che per inizializzare un latch SR a 1 si porta /preset a /reset, e /preclear al Vcc.
Viceversa, per inizializzare il latch a 0 si porta /preset al Vcc e /preclear a /reset.

Vogliamo quindi modificare la sintesi del latch SR: conviene unirlo ad una rete combinatoria, che ha per ingresso S, R, /preset e /preclear, e in uscita $z_s$ e $z_r$ (che andranno in ingresso al latch vero e proprio).
L'obiettivo di questa rete è di impostare i corrispondenti comandi di SET e RESET se uno fra /preset e /preclear è attivo basso, o di restituire S e R così come sono in caso entrambi siano alti. 

Abbiamo, dalla sintesi con le mappe di Karnaugh, riportando i valori in coppie $(z_s, z_r)$:

\begin{center}
	\begin{karnaugh-map}[4][4][1][/preclear][/preset][R][S]
		\manualterms{--, 10, 01, 00, --, 10, 01, 01, --, 10, 01, 10, --, 10, 01, 10}
	\end{karnaugh-map}
\end{center}

Si visualizzano i sottocubi nelle mappe presi separatamente $z_s$ e $z_r$:
\begin{itemize}
	\item $z_s$:
\begin{center}
\noindent
\begin{minipage}{0.3\textwidth}
	\begin{karnaugh-map}[4][4][1][/preclear][/preset][R][S]
		\manualterms{-, 1, 0, 0, -, 1, 0, 0, -, 1, 0, 1, -, 1, 0, 1}
		\implicant{0}{9}
		\implicant{13}{11}
	\end{karnaugh-map}
\end{minipage}%
\hspace{3cm}
\begin{minipage}{0.3\textwidth}
	\begin{table}[H]
		\center \rowcolors{2}{white}{black!10}
		\begin{tabular} { c || c | c | c | c}
			& S & R & /preset & /preclear \\ 
			\hline 
			\rowcolor{red!20!white} A & - & - & 0 & - \\
			\rowcolor{green!20!white} B & 1 & - & - & 1 \\
		\end{tabular}
	\end{table}
\end{minipage}
\end{center}
\item $s_r$:
\begin{center}
\noindent
\begin{minipage}{0.3\textwidth}
	\begin{karnaugh-map}[4][4][1][/preclear][/preset][R][S]
		\manualterms{-, 0, 1, 0, -, 0, 1, 1, -, 0, 1, 0, -, 0, 1, 1}
		\implicantedge{0}{8}{2}{10}
		\implicant{7}{14}
	\end{karnaugh-map}
\end{minipage}%
\hspace{3cm}
\begin{minipage}{0.3\textwidth}
	\begin{table}[H]
		\center \rowcolors{2}{white}{black!10}
		\begin{tabular} { c || c | c | c | c}
			& S & R & /preset & /preclear \\ 
			\hline 
			\rowcolor{red!20!white} A & - & - & - & 0 \\
			\rowcolor{green!20!white} B & - & 1 & 1 & - \\
		\end{tabular}
	\end{table}
\end{minipage}
\end{center}
\end{itemize}


Da cui si ricavano le due sintesi di $z_s$ e $z_r$:
\[
	\begin{cases}
		z_s = \overline{\mathrm{/preset}} + (\mathrm{/preclear} \cdot s) \\ 
		z_r = \overline{\mathrm/preclear} + (\mathrm{/preset} \cdot r)
	\end{cases}
\]

A questo punto, visto che il latch SR è realizzato a porte NOR, possiamo semplificare gli OR e i NOR in cascata: se assumiamo una NOR come una OR in serie ad una NOT, si ha che due OR equivalgono a una singola OR, ergo si possono mandare le uscite delle reti combinatorie appena sintetizzate direttamente ai NOR del latch SR, rimuovendo le OR che avremmo normalmento introdotto in una sintesi SP. 
Questo processo viene a volte detto \textit{compenetrazione}.

In Verilog, possiamo quindi descrivere il latch SR aggiornato con le entrate di $\mathrm{/preset}$ e $\mathrm{/preclear come segue:

\lstinputlisting[language=verilog, style=codestyle]{../verilog/10-30/sr_latches/sr_reset_latch.v}

\subsection{Tabelle e grafi di flusso}
Le reti sequenziali, pi+ spesso che con la tabella di applicazione, si descrivono usando \textbf{tabelle di flusso} e \textbf{grafi di flusso}.

\subsubsection{Tabelle di flusso}
Una tabella di flusso è una tabella che descrive come si evolvono lo stato interno e l'uscita al variare degli stati di ingresso.
Ad esempio, per un latch SR, ignorando /preclear e /preset:

\begin{table}[h!]
	\center 
	\begin{tabular} { c | c  c  c  c | c }
		& 00 & 01 & 11 & 10 & q \\ 
		\hline 
		$S_0$ & \circled{$S_0$} & \circled{$S_0$} & - & $s_1$ & 0 \\ 
		$S_1$ & \circled{$S_1$} & $s_0$ & - & \circled{$S_1$} & 1\\
	\end{tabular}
\end{table}

Si ha che nella tabella, le righe rappresentano gli \textbf{stati interni presenti} (SIP) e le colonne i possibili ingressi in entrata: all'intersezione fra uno stato e un ingresso si ha lo \textbf{stato interno successivo} (SIS).
Si indicano con la barra ($-$) gli stati non definiti (in questo caso \textit{non permessi}).
Inoltre, l'ultima colonna indica il valore effettivo delle uscite in ogni stato (qui si è riportato solo $q$, e non $q_N$).
Gli stati interni successivi cerchiati sono quelli che restano invariati dagli stati interni presenti precedenti: cioè, le coppie di stati interni presenti e ingressi che individuano uno stato interno successivo cerchiato sono coppie \textbf{stabili}.

\subsubsection{Grafi di flusso}
Un formalismo del tutto identico è quello del grafo di flusso: si prendono gli stati come nodi, e si disegnano archi (orientati) etichettati con gli stati di ingresso.
Gli archi uscenti da un nodo simboleggiano quindi i possibili ingressi di quello stato, e entrano nei nodi che rappresentano gli stati interni successivi.
Ad esempio, il grafo corrispondente alla tabella di flusso precedente è:
\begin{center}
	\begin{circuitikz}
		\node[draw, circle] (1) at (0, 0) {$S_0 / 0$};
		\node[draw, circle] (2) at (4, 0) {$S_1 / 1$};
		
		\draw[->] (1) to [out=45, in=135, looseness=1] (2);
		\node at (2,1.5) {sr = 10};

		\draw[->] (2) to [out=225, in=-45, looseness=1] (1);
		\node at (2,-1.5) {sr = 01};
		
		\draw[->] (1) to [out=70, in=110, looseness=5] (1);
		\node at (0, 1.5) {sr = 00};
		\draw[->] (2) to [out=70, in=110, looseness=5] (2);
		\node at (4, 1.5) {sr = 00};
	
		\draw[->] (1) -- (0, -1.2); 
		\node at (0, -1.5) {sr = 11};
		\draw[->] (2) -- (4, -1.2); 
		\node at (4, -1.5) {sr = 11};
	\end{circuitikz}
\end{center}

Notiamo come ad ogni nodo si può associare, separato dalla barra ($/$), l'uscita corrispondente a un dato stato interno presente.
Inoltre, gli stati non definiti vengono indicati con frecce non dirette verso alcun nodo.

\par\smallskip 

Questi strumenti sono utili per la descrizione e la verifica (in questo caso, sotto \textbf{ispezione}, \textbf{statica}) delle reti logiche.
Nel caso di reti sequenziali, poi, la verifica \textbf{dinamica} si fa attraverso un \textbf{diagramma di temporizzazione}, cioè un grafico temporale del valore logico di ogni variabile di interesse, creato seguendo i passaggi:
\begin{enumerate}
	\item Si decide uno stato iniziale;
	\item Si attribuiscono valori agli ingressi nel tempo;
	\item Si osserva l'evoluzione temporale della rete.
\end{enumerate}

\end{document}

\documentclass[a4paper,11pt]{article}
\usepackage[a4paper, margin=8em]{geometry}

% usa i pacchetti per la scrittura in italiano
\usepackage[french,italian]{babel}
\usepackage[T1]{fontenc}
\usepackage[utf8]{inputenc}
\frenchspacing 

% usa i pacchetti per la formattazione matematica
\usepackage{amsmath, amssymb, amsthm, amsfonts}

% usa altri pacchetti
\usepackage{gensymb}
\usepackage{hyperref}
\usepackage{standalone}

\usepackage{colortbl}

\usepackage{xstring}
\usepackage{karnaugh-map}

% imposta il titolo
\title{Appunti Reti Logiche}
\author{Luca Seggiani}
\date{2024}

% imposta lo stile
% usa helvetica
\usepackage[scaled]{helvet}
% usa palatino
\usepackage{palatino}
% usa un font monospazio guardabile
\usepackage{lmodern}

\renewcommand{\rmdefault}{ppl}
\renewcommand{\sfdefault}{phv}
\renewcommand{\ttdefault}{lmtt}

% circuiti
\usepackage{circuitikz}
\usetikzlibrary{babel}

% disponi il titolo
\makeatletter
\renewcommand{\maketitle} {
	\begin{center} 
		\begin{minipage}[t]{.8\textwidth}
			\textsf{\huge\bfseries \@title} 
		\end{minipage}%
		\begin{minipage}[t]{.2\textwidth}
			\raggedleft \vspace{-1.65em}
			\textsf{\small \@author} \vfill
			\textsf{\small \@date}
		\end{minipage}
		\par
	\end{center}

	\thispagestyle{empty}
	\pagestyle{fancy}
}
\makeatother

% disponi teoremi
\usepackage{tcolorbox}
\newtcolorbox[auto counter, number within=section]{theorem}[2][]{%
	colback=blue!10, 
	colframe=blue!40!black, 
	sharp corners=northwest,
	fonttitle=\sffamily\bfseries, 
	title=Teorema~\thetcbcounter: #2, 
	#1
}

% disponi definizioni
\newtcolorbox[auto counter, number within=section]{definition}[2][]{%
	colback=red!10,
	colframe=red!40!black,
	sharp corners=northwest,
	fonttitle=\sffamily\bfseries,
	title=Definizione~\thetcbcounter: #2,
	#1
}

% disponi codice
\usepackage{listings}
\usepackage[table]{xcolor}

\definecolor{codegreen}{rgb}{0,0.6,0}
\definecolor{codegray}{rgb}{0.5,0.5,0.5}
\definecolor{codepurple}{rgb}{0.58,0,0.82}
\definecolor{backcolour}{rgb}{0.95,0.95,0.92}

\lstdefinestyle{codestyle}{
		backgroundcolor=\color{black!5}, 
		commentstyle=\color{codegreen},
		keywordstyle=\bfseries\color{magenta},
		numberstyle=\sffamily\tiny\color{black!60},
		stringstyle=\color{green!50!black},
		basicstyle=\ttfamily\footnotesize,
		breakatwhitespace=false,         
		breaklines=true,                 
		captionpos=b,                    
		keepspaces=true,                 
		numbers=left,                    
		numbersep=5pt,                  
		showspaces=false,                
		showstringspaces=false,
		showtabs=false,                  
		tabsize=2
}

\lstdefinestyle{shellstyle}{
		backgroundcolor=\color{black!5}, 
		basicstyle=\ttfamily\footnotesize\color{black}, 
		commentstyle=\color{black}, 
		keywordstyle=\color{black},
		numberstyle=\color{black!5},
		stringstyle=\color{black}, 
		showspaces=false,
		showstringspaces=false, 
		showtabs=false, 
		tabsize=2, 
		numbers=none, 
		breaklines=true
}


\lstdefinelanguage{assembler}{ 
  keywords={AAA, AAD, AAM, AAS, ADC, ADCB, ADCW, ADCL, ADD, ADDB, ADDW, ADDL, AND, ANDB, ANDW, ANDL,
        ARPL, BOUND, BSF, BSFL, BSFW, BSR, BSRL, BSRW, BSWAP, BT, BTC, BTCB, BTCW, BTCL, BTR, 
        BTRB, BTRW, BTRL, BTS, BTSB, BTSW, BTSL, CALL, CBW, CDQ, CLC, CLD, CLI, CLTS, CMC, CMP,
        CMPB, CMPW, CMPL, CMPS, CMPSB, CMPSD, CMPSW, CMPXCHG, CMPXCHGB, CMPXCHGW, CMPXCHGL,
        CMPXCHG8B, CPUID, CWDE, DAA, DAS, DEC, DECB, DECW, DECL, DIV, DIVB, DIVW, DIVL, ENTER,
        HLT, IDIV, IDIVB, IDIVW, IDIVL, IMUL, IMULB, IMULW, IMULL, IN, INB, INW, INL, INC, INCB,
        INCW, INCL, INS, INSB, INSD, INSW, INT, INT3, INTO, INVD, INVLPG, IRET, IRETD, JA, JAE,
        JB, JBE, JC, JCXZ, JE, JECXZ, JG, JGE, JL, JLE, JMP, JNA, JNAE, JNB, JNBE, JNC, JNE, JNG,
        JNGE, JNL, JNLE, JNO, JNP, JNS, JNZ, JO, JP, JPE, JPO, JS, JZ, LAHF, LAR, LCALL, LDS,
        LEA, LEAVE, LES, LFS, LGDT, LGS, LIDT, LMSW, LOCK, LODSB, LODSD, LODSW, LOOP, LOOPE,
        LOOPNE, LSL, LSS, LTR, MOV, MOVB, MOVW, MOVL, MOVSB, MOVSD, MOVSW, MOVSX, MOVSXB,
        MOVSXW, MOVSXL, MOVZX, MOVZXB, MOVZXW, MOVZXL, MUL, MULB, MULW, MULL, NEG, NEGB, NEGW,
        NEGL, NOP, NOT, NOTB, NOTW, NOTL, OR, ORB, ORW, ORL, OUT, OUTB, OUTW, OUTL, OUTSB, OUTSD,
        OUTSW, POP, POPL, POPW, POPB, POPA, POPAD, POPF, POPFD, PUSH, PUSHL, PUSHW, PUSHB, PUSHA, 
				PUSHAD, PUSHF, PUSHFD, RCL, RCLB, RCLW, MOVSL, MOVSB, MOVSW, STOSL, STOSB, STOSW, LODSB, LODSW,
				LODSL, INSB, INSW, INSL, OUTSB, OUTSL, OUTSW
        RCLL, RCR, RCRB, RCRW, RCRL, RDMSR, RDPMC, RDTSC, REP, REPE, REPNE, RET, ROL, ROLB, ROLW,
        ROLL, ROR, RORB, RORW, RORL, SAHF, SAL, SALB, SALW, SALL, SAR, SARB, SARW, SARL, SBB,
        SBBB, SBBW, SBBL, SCASB, SCASD, SCASW, SETA, SETAE, SETB, SETBE, SETC, SETE, SETG, SETGE,
        SETL, SETLE, SETNA, SETNAE, SETNB, SETNBE, SETNC, SETNE, SETNG, SETNGE, SETNL, SETNLE,
        SETNO, SETNP, SETNS, SETNZ, SETO, SETP, SETPE, SETPO, SETS, SETZ, SGDT, SHL, SHLB, SHLW,
        SHLL, SHLD, SHR, SHRB, SHRW, SHRL, SHRD, SIDT, SLDT, SMSW, STC, STD, STI, STOSB, STOSD,
        STOSW, STR, SUB, SUBB, SUBW, SUBL, TEST, TESTB, TESTW, TESTL, VERR, VERW, WAIT, WBINVD,
        XADD, XADDB, XADDW, XADDL, XCHG, XCHGB, XCHGW, XCHGL, XLAT, XLATB, XOR, XORB, XORW, XORL},
  keywordstyle=\color{blue}\bfseries,
  ndkeywordstyle=\color{darkgray}\bfseries,
  identifierstyle=\color{black},
  sensitive=false,
  comment=[l]{\#},
  morecomment=[s]{/*}{*/},
  commentstyle=\color{purple}\ttfamily,
  stringstyle=\color{red}\ttfamily,
  morestring=[b]',
  morestring=[b]"
}

\lstset{language=assembler, style=codestyle}

% disponi sezioni
\usepackage{titlesec}

\titleformat{\section}
	{\sffamily\Large\bfseries} 
	{\thesection}{1em}{} 
\titleformat{\subsection}
	{\sffamily\large\bfseries}   
	{\thesubsection}{1em}{} 
\titleformat{\subsubsection}
	{\sffamily\normalsize\bfseries} 
	{\thesubsubsection}{1em}{}

% tikz
\usepackage{tikz}

% float
\usepackage{float}

% grafici
\usepackage{pgfplots}
\pgfplotsset{width=10cm,compat=1.9}

% disponi alberi
\usepackage{forest}

\forestset{
	rectstyle/.style={
		for tree={rectangle,draw,font=\large\sffamily}
	},
	roundstyle/.style={
		for tree={circle,draw,font=\large}
	}
}

% disponi algoritmi
\usepackage{algorithm}
\usepackage{algorithmic}
\makeatletter
\renewcommand{\ALG@name}{Algoritmo}
\makeatother

% disponi numeri di pagina
\usepackage{fancyhdr}
\fancyhf{} 
\fancyfoot[L]{\sffamily{\thepage}}

\makeatletter
\fancyhead[L]{\raisebox{1ex}[0pt][0pt]{\sffamily{\@title \ \@date}}} 
\fancyhead[R]{\raisebox{1ex}[0pt][0pt]{\sffamily{\@author}}}
\makeatother

\begin{document}
% sezione (data)
\section{Lezione del 05-11-24}

% stili pagina
\thispagestyle{empty}
\pagestyle{fancy}

% testo
\subsection{D-Latch trasparente}
Introduciamo una nuova rete sequenziale dotata di due ingressi, $d$ (data) e $c$ (control), e un'uscita $q$.
Il D-latch memorizza il bit in $d$ quando $c$ (\textbf{trasparenza}) vale 1.
Quando $c$ vale 0, invece, si dice che è in \textbf{conservazione}, ergo memorizza l'ultimo valore che $d$ ha assunto quando $c$ valeva 1.

La tabella di flusso di questa rete è la seguente, assunti in quest'ordine $c$ e $d$:
\begin{table}[h!]
	\center 
	\begin{tabular} { c | c  c  c  c | c }
		& 00 & 01 & 10 & 11 & q \\ 
		\hline 
		$S_0$ & \circled{$S_0$} & \circled{$S_0$} & \circled{$S_0$} & $S_1$ & 0 \\ 
		$S_1$ & \circled{$S_1$} & \circled{$S_1$} & $S_0$ & \circled{$S_1$} & 1\\
	\end{tabular}
\end{table}

cioè quando si è in conservazione, qualsiasi valore di $d$ viene ignorato e si memorizza il valore passato.
Quando si è in trasparenza, invece, $q$ si asegua a $d$.

Si può realizzare un D-latch attraverso un latch SR, con in ingresso una certa rete combinatoria.
Quello che vogliamo fare è portare $d$ e $c$ in $s$ e $r$, attraverso la tabella di verità:
\begin{table}[H]
	\center 
	\begin{tabular} { c  c | c  c }
		$c$ & $d$ & $s$ & $r$ \\
		\hline
		0 & 0 & 0 & 0 \\ 
		0 & 1 & 0 & 0 \\ 
		1 & 0 & 0 & 1 \\ 
		1 & 1 & 1 & 0
	\end{tabular}
\end{table}

Questo si sintetizza in $s = c \cdot d$ e $r = c \cdot \overline{d}$.
Si ha che le porte AND che rappresentano le congiunzioni in questa rete combinatoria possono collassare con le porte AND che formavano la rete combinatoria del latch SR che permetteva preset e preclear.

\subsubsection{Pilotaggio del D-Latch}
Nel pilotaggio del D-latch, dobbiamo assicurarci che $d$ sia costante a cavallo della transizione di $c$ da 1 a 0, in quanto potremmo finire per memorizzare dati ignoti (l'ultima cosa che il D-latch ha "visto" prima del reset di $c$).
I tempi per cui $d$ deve essere costante, rispettivamente \textbf{prima} e \textbf{dopo} della transizione di $c$, si dicono $T_{setup}$ e $T_{hold}$, e sono dati di progetto.

\subsubsection{Trasparenza}
Quando il D-latch è in \textbf{trasparenza}. il suo ingresso è direttamente connesso, in \textbf{senso logico} (ci sono comunque ritardi nella logica delle reti), all'uscita.
Per questo motivo, se $q$ e $d$ sono collegati in \textbf{retroazione negativa} (un feedback loop negato), si ha che con $c=1$ abbiamo oscillazioni incontrollate, e che con $c=0$ in $q$ (cioè lo stato interno) resta un valore casuale (l'ultimo rilevato durante le oscillazioni casuali prima che $c$ sia transito a 0).

Questo significa che il D-latch è una rete \textbf{trasparente}, cioè \textit{la sua uscita cambia mentre la rete è sensibile alle variazioni di ingresso}.
Questo significa che non possiamo memorizzare niente che sia funzione dell'uscita (saremmo nel caso della retroazine negativa di prima).

Poniamo di voler eseguire un'istruzione semplice come \lstinline|INC %AX|. 
A livello hardware, questo significà connettere un registro (quindi una serie di D-latch) ad una rete combinatoria per l'incremento (probabilmente un half adder), e quindi l'uscita di questa rete di nuovo al D-latch.
Quello che abbiamo essenzialmente creato è un ciclo di retrazione: il sistema devolverà velocemente in uno stato di oscillazione incontrollata.

\subsection{D flip-flop}
Il \textbf{D flip-flop} è una rete sequenziale \textbf{non trasparente} che si pone di risolvere i problemi di trasparenza del D-latch.
Quello che vedremo nel dettaglio è il \textbf{positive edge-triggered D flip-flop}, che è una rete che si comporta come segue, sulla base degli ingressi $d$ (data) e $p$, e l'uscita $q$: quando $p$ ha un fronte di salita, memorizza $d$, \textit{attendi} un determinato istante temporale e adegua l'uscita.

Possiamo concettualizzare il D flip-flop come composto, alla base, da un D-latch.
Mettiamo a $c$, invece dell'ingresso $p$, il \textbf{generatore di impulso} $P^+$ sul fronte di salita di $p$.
In uscita a $q$, poi, abbiamo un buffer $\Delta$, che introduce ritardo.
La proprietà fondamentale che desideriamo è:
$$
\Delta > P^+ 
$$
Questo significa che $q$ si adegua al valore campionato di $d$ soltanto \textit{dopo} che la rete ha smesso di essere sensibile a $d$.
È questa proprietà a rendere il D flip-flop una rete non trasparente.

\subsubsection{Pilotaggio del D flip-flop}
Innanzitutto, a cavallo del fronte di salita di $p$ l'ingresso $d$ deve rimanere costante, ergo si hanno gli stessi $_{setup}$ e $T_{hold}$ del D-latch.
Inoltre, si ha il ritardo di adeguamento dell'uscita, che denominiamo $T_{prop}$ (dall'inglese \textit{propagation}).
Qui la diseguaglianza d prima si traduce come:
$$
T_{prop} > T_{hold}
$$

\par\medskip 

Si che l'uscita di un D-FF non oscilla mai, a differenza di quella del D-Latch: l'adeguamento avviene in modo "secco", sul fronte di salita, e di lì in poi fino a reset e successivo set di $p$, l'uscita $q$ è in conservazione e ignora il comportamento di $d$.

\subsubsection{Sintesi Master-Slave di un D flip-flop}
Storicamente, un D flip-flop è stato realizzato attraverso un montaggio master/slave, attraverso due D-latch in cascata (di cui uno master, e l'altro chiaramente slave).
Si invia quindi l'ingresso $p$ allo slave, e il suo negato al master, e si fa passare la linea $d$ prima dal master, poi dalla sua uscita all'ingresso del slave, e poi al $q$ del D flip-flop.
Si ha che negli stati:
\begin{itemize}
	\item $p=0$: \textbf{master} e in \textit{trasparenza}, \textbf{slave} in \textit{conservazione};
	\item $p=1$: \textbf{master} in \textit{conservazione}, \textbf{slave} in \textit{trasparenza}.
\end{itemize}

Quando $p$ è a 0, lo slave è in conservazione, quindi la rete memorizza.
Nel frattempo il master è in trasparenza, quindi reagisce al valore in entrata di $d$.
Quando $p$ transisce a $1$, lo slavean automa in automaton theory  passa in trasparenza, e quindi risponde a quello che esce dal master, che invece si trova in conservazione del valore che aveva un'attimo prima della transizione.
Il risultato è un comportamento effettivamente analogo a quello della struttura a generatore di impulso e buffer vista prima.

Si possono avere problemi nel funzionamento transitorio dei due D-latch: per questo si agisce elettronicamente, sviluppando questi per commutare $c$ a valori di tensione diversi.
In particolare, vogliamo che in transizione di $p$ da 1 a 0 lo slave conservi il valore prima che il master passi a trasparenza, quindi che $c$ dello slave commuti prima di $c$ del master.

Nella pratica, infine, si ha che la sintesi reale di un D flip-flop è fatta a partire da un latch SR, prima del quale si dispone una rete sequenziale asincrona la cui sintesi è fuori dagli scopi del corso.

\subsection{Memorie RAM statiche}
Esistono due tipi principali di memoria:
\begin{itemize}
	\item \textbf{S-RAM}, costituite da matrici di D latch;
	\item \textbf{D-RAM}, realizzate in modo diverso, che per adesso ingnoreremo. 
\end{itemize}

Una riga di D latch rappresenta quindi una \textbf{locazione di memoria}, che può essere \textbf{letta} o \textbf{scritta} con apposite operazioni, strettamente \textbf{non simultanee}.

Una SRAM è presenta gli ingressi e le uscite:
\begin{itemize}
	\item \textbf{Ingressi di indirizzo:} in numero sufficiente per indirizzare tutte le celle di memoria. Ad esempio, con $2^{23}$ celle di 4 bit, 23 ingressi;
	\item \textbf{Ingressi/uscite di dati:} che andranno forchettati con porte \textbf{tri-state};
	\item \textbf{Memory read} e \textbf{memory write}, segnali attivi bassi;
	\item \textbf{Select}, segnale attivo basso che fa da \textbf{enabler}, in modo simile a quanto avevamo visto nei decoder.
\end{itemize}

Il comportamento che vogliamo dalla SRAM è il seguente:
\begin{table}[h!]
	\center \rowcolors{2}{white}{black!10}
	\begin{tabular} { c | c | c || c }
		\bfseries \lstinline|/s| & \bfseries \lstinline|/mr| & \bfseries \lstinline|/mw| & \bfseries Azione \\
		\hline 
		1 & - & - & Nulla \\ 
		0 & 1 & 1 & Nulla \\ 
		0 & 0 & 1 & Lettura in corso \\ 
		1 & 1 & 0 & Scrittura in corso \\ 
	\end{tabular}
\end{table}

\subsubsection{Temporizzazione delle RAM statiche}
Facciamo innanzitutto la divisione lettura/scrittura:
\begin{itemize}
	\item \textbf{Lettura:} per fare una lettura bisogna dare il comando (attivo basso) di memory read (\lstinline|/mr|), e impostare l'indirizzo di lettura.
Il comando di select (\lstinline|/s|) arriva in ritardo, e a quel punto, quando sia \lstinline|/s| che \lstinline|/mr| sono in conduzione, i multiplexer vanno a regime e si può fare una lettura sull'uscita dei dati.
Infine, quando \lstinline|/mr| torna a 1, i dati tornano ad alta impedenza, e l'indirizzo di lettura e la select possono assumere valori arbitrari.
	\item \textbf{Scrittura:} si ha che la scrittura è \textbf{distruttiva} (manda i D-latch in trasparenza). Bisogna quindi attendere che il select \lstinline|/s| e gli indirizzi siano stabili prima di abbassare \lstinline|mw| per dare il comando di scrittura (l'opposto di quanto avevamo fatto in lettura, qui vogliamo scrivere solo quando siamo sicuri di poterlo fare, ergo i multiplexer sono a regime).
		A questo punto, abbiamo che quando \lstinline|mw| torna alto dobbiamo assicurarci che i dati in scrittura siano fermi, in quanto i multiplexer riportano gli ingressi di controllo dei D-latch a 0 e l'indirizzo di lettura e la select possono, nuovamente, assumere valori arbitrari.
\end{itemize}

\subsubsection{Montaggio di banchi di memoria}
Vediamo come combinare più banchi di memoria per aumentare lo spazio di memoria indirizzabile.
\begin{itemize}
	\item \textbf{Montaggio in parallelo:} 
prendiamo in considerazione due banchi di memoria da $8 \text{M} \times 4$ bit, e vediamo come collegarli per formare un singolo banco di memoria da $8 \text{M} \times 8$ bit, quindi raddoppiando la dimensione delle locazioni.
\begin{itemize}
	\item Per quanto riguarda gli \textbf{indirizzi di lettura}, basta inviare l'indirizzo ad entrambi i banchi, da cui preleveremo la parte \textit{alta} e \textit{bassa} della locazione;
	\item Per quanto riguarda gli \textbf{ingressi/uscite di dati}, avremo che la combinazione delle linee sui due banchi, da 4 bit ciascuna, formano un singolo byte da 8 bit, ergo la locazione di memoria completa.
\end{itemize}
	\item \textbf{Montaggio in serie:}  
prendiamo in considerazione due banchi di memoria da $8 \text{M} \times 8$ bit, e vediamo come collegarli per formare un singolo banco di memoria da $16 \text{M} \times 8$ bit, quindi raddoppiamdo il numero di locazioni.
\begin{itemize}
	\item Per quanto riguarda gli \textbf{indirizzi di lettura}, discrimina dal MSB dell'indirizzo se selezionare dal primo o dal secondo banco, che faranno quindi da parte \textit{alta} e \textit{bassa} dello spazio di memoria indirizzabile. Facciamo questo attraverso l'ingresso di select \lstinline|/s|, che useremo per determinare altri due segnali di select \lstinline|/sl| e \lstinline|sh| (\textit{select low} e \textit{select high}), che a loro volta ci permettono di discriminare sulla base del MSD quale banco andiamo a selezionare (effettivamente rendere attivo); 
	\item Per quanto riguarda gli \textbf{ingressi/uscite di dati}, avremo che il banco attivo in un dato momento determina completamente le uscite. Potremmo pensare di dover inserire porte tri-state in uscita ai singoli banchi di memoria sulla linea di ingresso/uscita, ma questo non è necessario: le \lstinline|sl| e \lstinline|sh| sono mutualmente esclusive, e quindi non si verificherà mai il caso in cui le linee di uscita di entrambi i banchi sono in conduzione contemporaneamente.
\end{itemize}
\end{itemize}
\end{document}


\documentclass[a4paper,11pt]{article}
\usepackage[a4paper, margin=8em]{geometry}

% usa i pacchetti per la scrittura in italiano
\usepackage[french,italian]{babel}
\usepackage[T1]{fontenc}
\usepackage[utf8]{inputenc}
\frenchspacing 

% usa i pacchetti per la formattazione matematica
\usepackage{amsmath, amssymb, amsthm, amsfonts}

% usa altri pacchetti
\usepackage{gensymb}
\usepackage{hyperref}
\usepackage{standalone}

\usepackage{colortbl}

\usepackage{xstring}
\usepackage{karnaugh-map}

% imposta il titolo
\title{Appunti Reti Logiche}
\author{Luca Seggiani}
\date{2024}

% imposta lo stile
% usa helvetica
\usepackage[scaled]{helvet}
% usa palatino
\usepackage{palatino}
% usa un font monospazio guardabile
\usepackage{lmodern}

\renewcommand{\rmdefault}{ppl}
\renewcommand{\sfdefault}{phv}
\renewcommand{\ttdefault}{lmtt}

% circuiti
\usepackage{circuitikz}
\usetikzlibrary{babel}

% disponi il titolo
\makeatletter
\renewcommand{\maketitle} {
	\begin{center} 
		\begin{minipage}[t]{.8\textwidth}
			\textsf{\huge\bfseries \@title} 
		\end{minipage}%
		\begin{minipage}[t]{.2\textwidth}
			\raggedleft \vspace{-1.65em}
			\textsf{\small \@author} \vfill
			\textsf{\small \@date}
		\end{minipage}
		\par
	\end{center}

	\thispagestyle{empty}
	\pagestyle{fancy}
}
\makeatother

% disponi teoremi
\usepackage{tcolorbox}
\newtcolorbox[auto counter, number within=section]{theorem}[2][]{%
	colback=blue!10, 
	colframe=blue!40!black, 
	sharp corners=northwest,
	fonttitle=\sffamily\bfseries, 
	title=Teorema~\thetcbcounter: #2, 
	#1
}

% disponi definizioni
\newtcolorbox[auto counter, number within=section]{definition}[2][]{%
	colback=red!10,
	colframe=red!40!black,
	sharp corners=northwest,
	fonttitle=\sffamily\bfseries,
	title=Definizione~\thetcbcounter: #2,
	#1
}

% disponi codice
\usepackage{listings}
\usepackage[table]{xcolor}

\definecolor{codegreen}{rgb}{0,0.6,0}
\definecolor{codegray}{rgb}{0.5,0.5,0.5}
\definecolor{codepurple}{rgb}{0.58,0,0.82}
\definecolor{backcolour}{rgb}{0.95,0.95,0.92}

\lstdefinestyle{codestyle}{
		backgroundcolor=\color{black!5}, 
		commentstyle=\color{codegreen},
		keywordstyle=\bfseries\color{magenta},
		numberstyle=\sffamily\tiny\color{black!60},
		stringstyle=\color{green!50!black},
		basicstyle=\ttfamily\footnotesize,
		breakatwhitespace=false,         
		breaklines=true,                 
		captionpos=b,                    
		keepspaces=true,                 
		numbers=left,                    
		numbersep=5pt,                  
		showspaces=false,                
		showstringspaces=false,
		showtabs=false,                  
		tabsize=2
}

\lstdefinestyle{shellstyle}{
		backgroundcolor=\color{black!5}, 
		basicstyle=\ttfamily\footnotesize\color{black}, 
		commentstyle=\color{black}, 
		keywordstyle=\color{black},
		numberstyle=\color{black!5},
		stringstyle=\color{black}, 
		showspaces=false,
		showstringspaces=false, 
		showtabs=false, 
		tabsize=2, 
		numbers=none, 
		breaklines=true
}


\lstdefinelanguage{assembler}{ 
  keywords={AAA, AAD, AAM, AAS, ADC, ADCB, ADCW, ADCL, ADD, ADDB, ADDW, ADDL, AND, ANDB, ANDW, ANDL,
        ARPL, BOUND, BSF, BSFL, BSFW, BSR, BSRL, BSRW, BSWAP, BT, BTC, BTCB, BTCW, BTCL, BTR, 
        BTRB, BTRW, BTRL, BTS, BTSB, BTSW, BTSL, CALL, CBW, CDQ, CLC, CLD, CLI, CLTS, CMC, CMP,
        CMPB, CMPW, CMPL, CMPS, CMPSB, CMPSD, CMPSW, CMPXCHG, CMPXCHGB, CMPXCHGW, CMPXCHGL,
        CMPXCHG8B, CPUID, CWDE, DAA, DAS, DEC, DECB, DECW, DECL, DIV, DIVB, DIVW, DIVL, ENTER,
        HLT, IDIV, IDIVB, IDIVW, IDIVL, IMUL, IMULB, IMULW, IMULL, IN, INB, INW, INL, INC, INCB,
        INCW, INCL, INS, INSB, INSD, INSW, INT, INT3, INTO, INVD, INVLPG, IRET, IRETD, JA, JAE,
        JB, JBE, JC, JCXZ, JE, JECXZ, JG, JGE, JL, JLE, JMP, JNA, JNAE, JNB, JNBE, JNC, JNE, JNG,
        JNGE, JNL, JNLE, JNO, JNP, JNS, JNZ, JO, JP, JPE, JPO, JS, JZ, LAHF, LAR, LCALL, LDS,
        LEA, LEAVE, LES, LFS, LGDT, LGS, LIDT, LMSW, LOCK, LODSB, LODSD, LODSW, LOOP, LOOPE,
        LOOPNE, LSL, LSS, LTR, MOV, MOVB, MOVW, MOVL, MOVSB, MOVSD, MOVSW, MOVSX, MOVSXB,
        MOVSXW, MOVSXL, MOVZX, MOVZXB, MOVZXW, MOVZXL, MUL, MULB, MULW, MULL, NEG, NEGB, NEGW,
        NEGL, NOP, NOT, NOTB, NOTW, NOTL, OR, ORB, ORW, ORL, OUT, OUTB, OUTW, OUTL, OUTSB, OUTSD,
        OUTSW, POP, POPL, POPW, POPB, POPA, POPAD, POPF, POPFD, PUSH, PUSHL, PUSHW, PUSHB, PUSHA, 
				PUSHAD, PUSHF, PUSHFD, RCL, RCLB, RCLW, MOVSL, MOVSB, MOVSW, STOSL, STOSB, STOSW, LODSB, LODSW,
				LODSL, INSB, INSW, INSL, OUTSB, OUTSL, OUTSW
        RCLL, RCR, RCRB, RCRW, RCRL, RDMSR, RDPMC, RDTSC, REP, REPE, REPNE, RET, ROL, ROLB, ROLW,
        ROLL, ROR, RORB, RORW, RORL, SAHF, SAL, SALB, SALW, SALL, SAR, SARB, SARW, SARL, SBB,
        SBBB, SBBW, SBBL, SCASB, SCASD, SCASW, SETA, SETAE, SETB, SETBE, SETC, SETE, SETG, SETGE,
        SETL, SETLE, SETNA, SETNAE, SETNB, SETNBE, SETNC, SETNE, SETNG, SETNGE, SETNL, SETNLE,
        SETNO, SETNP, SETNS, SETNZ, SETO, SETP, SETPE, SETPO, SETS, SETZ, SGDT, SHL, SHLB, SHLW,
        SHLL, SHLD, SHR, SHRB, SHRW, SHRL, SHRD, SIDT, SLDT, SMSW, STC, STD, STI, STOSB, STOSD,
        STOSW, STR, SUB, SUBB, SUBW, SUBL, TEST, TESTB, TESTW, TESTL, VERR, VERW, WAIT, WBINVD,
        XADD, XADDB, XADDW, XADDL, XCHG, XCHGB, XCHGW, XCHGL, XLAT, XLATB, XOR, XORB, XORW, XORL},
  keywordstyle=\color{blue}\bfseries,
  ndkeywordstyle=\color{darkgray}\bfseries,
  identifierstyle=\color{black},
  sensitive=false,
  comment=[l]{\#},
  morecomment=[s]{/*}{*/},
  commentstyle=\color{purple}\ttfamily,
  stringstyle=\color{red}\ttfamily,
  morestring=[b]',
  morestring=[b]"
}

\lstset{language=assembler, style=codestyle}

% disponi sezioni
\usepackage{titlesec}

\titleformat{\section}
	{\sffamily\Large\bfseries} 
	{\thesection}{1em}{} 
\titleformat{\subsection}
	{\sffamily\large\bfseries}   
	{\thesubsection}{1em}{} 
\titleformat{\subsubsection}
	{\sffamily\normalsize\bfseries} 
	{\thesubsubsection}{1em}{}

% tikz
\usepackage{tikz}

% float
\usepackage{float}

% grafici
\usepackage{pgfplots}
\pgfplotsset{width=10cm,compat=1.9}

% disponi alberi
\usepackage{forest}

\forestset{
	rectstyle/.style={
		for tree={rectangle,draw,font=\large\sffamily}
	},
	roundstyle/.style={
		for tree={circle,draw,font=\large}
	}
}

% disponi algoritmi
\usepackage{algorithm}
\usepackage{algorithmic}
\makeatletter
\renewcommand{\ALG@name}{Algoritmo}
\makeatother

% disponi numeri di pagina
\usepackage{fancyhdr}
\fancyhf{} 
\fancyfoot[L]{\sffamily{\thepage}}

\makeatletter
\fancyhead[L]{\raisebox{1ex}[0pt][0pt]{\sffamily{\@title \ \@date}}} 
\fancyhead[R]{\raisebox{1ex}[0pt][0pt]{\sffamily{\@author}}}
\makeatother

\begin{document}
% sezione (data)
\section{Lezione del 06-11-24}

% stili pagina
\thispagestyle{empty}
\pagestyle{fancy}

% testo
\subsection{Collegamento al bus e maschere}
Resta da capire da dove vengono gli indirizzi di lettura usati nell'accesso alla SRAM.
Questi si trovano su un \textbf{bus indirizzi}, il cui valore è impostato dal processore ogni volta che vuole effettuare un accesso.
Ad esempio, per un modulo di RAM da $256 \text{M} \times 8$ bit, quindi con 28 ingressi, si avrà che bisogna di montare banchi di memoria a partire da \lstinline|0xE0000000| fino a \lstinline|0xEFFFFFFF|.

L'ingresso di select verrà quindi generato a partire dalla parte alta dell'indirizzo, fatto passare attraverso una certa maschera, facendo quindi corrispondere una certa impostazione dei bit più significativi (in questo caso 4) al segnale di select.
Visto che avevamo detto si parte da \lstinline|0xE0000000|, vogliamo \lstinline|0xE| = \lstinline|B1110|, quindi serve la rete combinatoria (maschera):
$$
\mathtt{/s} = \overline{a_{31}} + \overline{a_{30}} + \overline{a_{29}} + a_{28}
$$

Con un montaggio di questo tipo possiamo usare il select per scegliere quale banco di memoria RAM è associato a quale parte dello spazio indirizzabile.
Questo tipo di configurazione giustifica inoltre il fatto che il bit di select viene impostato con ritardo rispetto agli indirizzi: visto che deve essere calcolato sulla base di quest'ultimi, risentirà del ritardo della rete combinatoria che lo genera.

\subsection{Memorie ROM a sola lettura}
Le memorie a sola lettura, dette ROM, sono effettivamente reti combinatorie: l'uscita è costante qualunque siano gli stati passati.
Vengono montate nello spazio di memoria assieme alla RAM, e rappresentano la parte \textbf{non volatile} (persistente) dello spazio di memoria stesso (abbiamo visto come contengono le prime istruzioni eseguite dal processore).
La loro struttura interna ricalca quella della memoria RAM, privata della circuiteria necessaria alla lettura, e che usa generatori di costante al posto dei D-latch.

Una ROM può essere realizzata attraverso il MSU (\textit{Modello Strutturale Universale}), collegando in maniera ortogonale file di OR alle uscite di un decoder (come ci è concesso dall'espansione di Shannon), nei punti in cui vogliamo che un'uscita corrisponda a un certo ingresso.
Nella pratica, si usano più spesso porte NOR per ragioni elettroniche, e quindi si collegano le linee nel caso un uscita corrisponde alla negazione dell'ingresso.
Questo circuito viene solitamente stampato su un singolo chip di silicio, il cui costo fisso di progettazione è giustificato solo nel caso di produzione su larga scala.
Conviene quindi realizzare delle alternative, le \textbf{ROM programmabili}.

\subsubsection{ROM programmabili}
In una memoria ROM programmabile, le porte NOR sono tutte attaccate alle linee degli AND del decoder.
Si possono disabilitare selettivamente alcune di queste porte NOR per effettivamente \textbf{programmare} la memoria contenuta nella ROM.

Possiamo individuare delle categorie per queste reti:

\begin{itemize}
	\item Le \textbf{OTPROM} (\textit{One Time Programmable ROM}) vengono realizzate attraverso questa tecnologia, è la loro progammazione risulta quindi \textbf{distruttiva} (una volta programmato un bit non si può più rimuovere).
	\item  Una tecnologia più sofisticata è rappresentata dalle \textbf{EPROM} (\textit{Erasable programmable ROM}).
Queste vengono realizzate attraverso tranistor a field-effect.
La scrittura della EPROM può essere ripetuta sottoponendola a una luce ultravioletta, e quindi cancellando tutti i dati, per poi riscrivene altri.

Di una EPROM ci interessano:
\begin{itemize}
	\item \textbf{Endurance:} quante riscritture successive può supportare (solitamente dalle 10K alle 100K volte);
	\item \textbf{Data retention:} il periodo per cui si può fare affidamento sui dati contenuti in una EPROM (solitamente dai 10 ai 100 anni).
\end{itemize}
	\item Infine, le \textbf{EEPROM} (\textit{Electrically Erasable Programmable ROM}) permettono la riprogrammazione direttamente attraverso segnali elettrici, sul chip già montato nello spazio di memoria.
		Potremmo pensare che EEPROM e RAM sono effettivamente equivalente.
		Ci sono invece alcune differenze, che sono:
		\begin{itemize}
			\item L'EEPROM è persistente, mentre la RAM è volatile;
			\item Il numero di volte in cui si può riprogrammare una EEPROM è comunque limitato;
			\item Il tempo di riprogrammazione di una EEPROM è maggiore del tempo di lettura della RAM;
			\item Le tensioni che si usano nella programmazione di una EEPROM (12V-18V) sono maggiori dei 5V (o 3.3V) che richiede la RAM. 
		\end{itemize}
\end{itemize}

\subsection{Il linguaggio Verilog}
Per descrivere le reti logiche fa comodo adottare una \textbf{notazione testuale}.
Finora abbiamo usato disegni o espressioni algebriche: adesso introduciamo un \textbf{linguaggio di descrizione hardware}, il \textbf{Verilog}.
Questo linguaggio è più \textbf{compatto}, e può essere \textbf{interpretato} automaticamente da una macchina, permettendoci di effettuare prove (in modo simile a come avevamo introdotto coi diagrammi di temporizzazione, o addirittura con programmi come \lstinline|gtkwave| di realizzare veri e propri diagrammi di temporizzazione).

% cosa voglio sapere del verilog? 
% le basi (cos'è un modulo, cos'è un testbench, ecc...)
% BENE gli assegnamenti continui, procedurali, bloccanti, non bloccanti, ecc...

Non si riporteranno appunti riguardanti operatori e sintassi particolarmente specifiche del Verilog, in quanto esistono testi sicuramente più utili e approfonditi.
Parleremo invece della struttura di base di una sintesi in Verilog, con enfasi sugli \textbf{assegnamenti} disponibili.

\subsubsection{Struttura di una sintesi Verilog}
Il linguaggio Verilog descrive \textbf{moduli}.
Un modulo è formato da un insieme di \textbf{input} e \textbf{output}, e da una \textbf{struttura interna} che descrive la legge di evoluzione degli output in funzione degli input.
Ad esempio, il seguente frammento implementa un contatore in base 2 (che vedremo fra poco):
\begin{lstlisting}[language=verilog, style=codestyle]	
module b2_counter(eu, q, ei, clock, reset_);
  input clock, reset_;
  input ei;
  output eu, q;
  reg OUTR;
  assign q = OUTR;	
  wire a;
  assign {a, eu} = ({q, ei} == 'B00) ? 'B00:
                   ({q, ei} == 'B10) ? 'B10:
                   ({q, ei} == 'B01) ? 'B10:
                 /*({q, ei} == 'B11)*/ 'B01;
  always @(reset_ == 0) #1 OUTR <= 0;
  always @(posedge clock) if (reset_==1) #2 OUTR <= a;
endmodule
\end{lstlisting}

Si parte con la dichiarazione di un modulo \lstinline|b2_counter|, dove la sintassi "ad argomenti" indica le variabili che potremo usare come input o come output.
Definiamo poi input e output esplicitamente, notando che si può usare (e anzi è consigliata) una separazione logica delle variabili su più righe.
La parola chhiave \lstinline|reg| definisce poi un registro.

\subsubsection{Assegnamenti}
Una parte a noi particolarmente interessante del linguaggio è rappresentata dagli \textbf{assegnamenti}.
Abbiamo fatto 3 assegnamenti nelle ultime 3 istruzioni dell'esempio precedente, prima per implementare un contatore (con \lstinline|assign|), e in seguito specifcare il comportamento al reset e l'aggiornamento del registro (con \lstinline|always|).
Vediamo nel dettaglio tutte gli assegnamenti possibili e le loro differenze:
\begin{itemize}
	\item \textbf{Assegnamenti procedurali:} vengono dichiarati con \lstinline|initial| (per riferirsi a stati iniziali), o con \lstinline|always| (per riferirsi a stati qualsiasi durante la simulazione).
	In blocchi \lstinline|always|, si può specificare una condizione con il carattere \lstinline|@|, fra cui ad esempio il \lstinline|posedge| che ci permette di ricavare il positive edge di un segnale (il clock nell'esempio precedente).
	All'interno di assegnamenti procedurali abbiamo a disposizione più operatori:
	\begin{itemize}
		\item \textbf{Assegnamenti bloccanti} (=): vengono eseguiti strettamente nell'ordine in cui vengono incontrati (cosa a noi solitamente indesiderata, in quanto vogliamo modellizzare le inaccuratezze delle temporizzazioni, che non sono mai simultanee e non hanno, se non specificato, un ordine preciso);
		\item \textbf{Assegnamenti non bloccanti} (<=): vengono eseguiti in \textbf{parallelo}, cioè il valore non varia fino alla fine del blocco di assegnamento procedurale, e viene aggiornato con simultaneità al suo termine.
	\end{itemize}
\item \textbf{Assegnamenti continui:} legano \textbf{strutturalmente} una variabile ad un altra variabile, o a un costrutto \lstinline|case|, o ancora ad una sintassi ricavata dagli operatori ternari come quella nell'esempio precedente. Il valore legato viene aggiornato ad ogni aggiornamento del valore a cui è legato: vanno effettivamente intesi come fisicamente connessi.
\end{itemize}

Notiamo come negli assegnamenti in Verilog vale un concetto simile a quello di \lstinline|lvalue| e \lstinline|rvalue| nel C: si assegna sempre a \textit{sinistra}, sulla \textbf{LHS} (LeftHand Side) un valore a \textit{destra}, sulla \textbf{RHS} (RightHand Side).
Per fare un esempio, notiamo che non si può usare un \lstinline|wire| come LHS (infatti con un \lstinline|wire| ci aspetteremmo di usare una notazione strutturale, come quella data dagli assegnamenti continui).
Di contro, non si può avere aggiornamento continuo di registri (anche perché non avrebbe particolare significato logico).

Infine, notiamo che si può inserire, negli assegnamenti procedurali, un \textbf{ritardo}, indicato con il simbolo \lstinline|#| e misurato solitamente in secondi.

\end{document}


\documentclass[a4paper,11pt]{article}
\usepackage[a4paper, margin=8em]{geometry}

% usa i pacchetti per la scrittura in italiano
\usepackage[french,italian]{babel}
\usepackage[T1]{fontenc}
\usepackage[utf8]{inputenc}
\frenchspacing 

% usa i pacchetti per la formattazione matematica
\usepackage{amsmath, amssymb, amsthm, amsfonts}

% usa altri pacchetti
\usepackage{gensymb}
\usepackage{hyperref}
\usepackage{standalone}

\usepackage{colortbl}

\usepackage{xstring}
\usepackage{karnaugh-map}

% imposta il titolo
\title{Appunti Reti Logiche}
\author{Luca Seggiani}
\date{2024}

% imposta lo stile
% usa helvetica
\usepackage[scaled]{helvet}
% usa palatino
\usepackage{palatino}
% usa un font monospazio guardabile
\usepackage{lmodern}

\renewcommand{\rmdefault}{ppl}
\renewcommand{\sfdefault}{phv}
\renewcommand{\ttdefault}{lmtt}

% circuiti
\usepackage{circuitikz}
\usetikzlibrary{babel}

% disponi il titolo
\makeatletter
\renewcommand{\maketitle} {
	\begin{center} 
		\begin{minipage}[t]{.8\textwidth}
			\textsf{\huge\bfseries \@title} 
		\end{minipage}%
		\begin{minipage}[t]{.2\textwidth}
			\raggedleft \vspace{-1.65em}
			\textsf{\small \@author} \vfill
			\textsf{\small \@date}
		\end{minipage}
		\par
	\end{center}

	\thispagestyle{empty}
	\pagestyle{fancy}
}
\makeatother

% disponi teoremi
\usepackage{tcolorbox}
\newtcolorbox[auto counter, number within=section]{theorem}[2][]{%
	colback=blue!10, 
	colframe=blue!40!black, 
	sharp corners=northwest,
	fonttitle=\sffamily\bfseries, 
	title=Teorema~\thetcbcounter: #2, 
	#1
}

% disponi definizioni
\newtcolorbox[auto counter, number within=section]{definition}[2][]{%
	colback=red!10,
	colframe=red!40!black,
	sharp corners=northwest,
	fonttitle=\sffamily\bfseries,
	title=Definizione~\thetcbcounter: #2,
	#1
}

% disponi codice
\usepackage{listings}
\usepackage[table]{xcolor}

\definecolor{codegreen}{rgb}{0,0.6,0}
\definecolor{codegray}{rgb}{0.5,0.5,0.5}
\definecolor{codepurple}{rgb}{0.58,0,0.82}
\definecolor{backcolour}{rgb}{0.95,0.95,0.92}

\lstdefinestyle{codestyle}{
		backgroundcolor=\color{black!5}, 
		commentstyle=\color{codegreen},
		keywordstyle=\bfseries\color{magenta},
		numberstyle=\sffamily\tiny\color{black!60},
		stringstyle=\color{green!50!black},
		basicstyle=\ttfamily\footnotesize,
		breakatwhitespace=false,         
		breaklines=true,                 
		captionpos=b,                    
		keepspaces=true,                 
		numbers=left,                    
		numbersep=5pt,                  
		showspaces=false,                
		showstringspaces=false,
		showtabs=false,                  
		tabsize=2
}

\lstdefinestyle{shellstyle}{
		backgroundcolor=\color{black!5}, 
		basicstyle=\ttfamily\footnotesize\color{black}, 
		commentstyle=\color{black}, 
		keywordstyle=\color{black},
		numberstyle=\color{black!5},
		stringstyle=\color{black}, 
		showspaces=false,
		showstringspaces=false, 
		showtabs=false, 
		tabsize=2, 
		numbers=none, 
		breaklines=true
}


\lstdefinelanguage{assembler}{ 
  keywords={AAA, AAD, AAM, AAS, ADC, ADCB, ADCW, ADCL, ADD, ADDB, ADDW, ADDL, AND, ANDB, ANDW, ANDL,
        ARPL, BOUND, BSF, BSFL, BSFW, BSR, BSRL, BSRW, BSWAP, BT, BTC, BTCB, BTCW, BTCL, BTR, 
        BTRB, BTRW, BTRL, BTS, BTSB, BTSW, BTSL, CALL, CBW, CDQ, CLC, CLD, CLI, CLTS, CMC, CMP,
        CMPB, CMPW, CMPL, CMPS, CMPSB, CMPSD, CMPSW, CMPXCHG, CMPXCHGB, CMPXCHGW, CMPXCHGL,
        CMPXCHG8B, CPUID, CWDE, DAA, DAS, DEC, DECB, DECW, DECL, DIV, DIVB, DIVW, DIVL, ENTER,
        HLT, IDIV, IDIVB, IDIVW, IDIVL, IMUL, IMULB, IMULW, IMULL, IN, INB, INW, INL, INC, INCB,
        INCW, INCL, INS, INSB, INSD, INSW, INT, INT3, INTO, INVD, INVLPG, IRET, IRETD, JA, JAE,
        JB, JBE, JC, JCXZ, JE, JECXZ, JG, JGE, JL, JLE, JMP, JNA, JNAE, JNB, JNBE, JNC, JNE, JNG,
        JNGE, JNL, JNLE, JNO, JNP, JNS, JNZ, JO, JP, JPE, JPO, JS, JZ, LAHF, LAR, LCALL, LDS,
        LEA, LEAVE, LES, LFS, LGDT, LGS, LIDT, LMSW, LOCK, LODSB, LODSD, LODSW, LOOP, LOOPE,
        LOOPNE, LSL, LSS, LTR, MOV, MOVB, MOVW, MOVL, MOVSB, MOVSD, MOVSW, MOVSX, MOVSXB,
        MOVSXW, MOVSXL, MOVZX, MOVZXB, MOVZXW, MOVZXL, MUL, MULB, MULW, MULL, NEG, NEGB, NEGW,
        NEGL, NOP, NOT, NOTB, NOTW, NOTL, OR, ORB, ORW, ORL, OUT, OUTB, OUTW, OUTL, OUTSB, OUTSD,
        OUTSW, POP, POPL, POPW, POPB, POPA, POPAD, POPF, POPFD, PUSH, PUSHL, PUSHW, PUSHB, PUSHA, 
				PUSHAD, PUSHF, PUSHFD, RCL, RCLB, RCLW, MOVSL, MOVSB, MOVSW, STOSL, STOSB, STOSW, LODSB, LODSW,
				LODSL, INSB, INSW, INSL, OUTSB, OUTSL, OUTSW
        RCLL, RCR, RCRB, RCRW, RCRL, RDMSR, RDPMC, RDTSC, REP, REPE, REPNE, RET, ROL, ROLB, ROLW,
        ROLL, ROR, RORB, RORW, RORL, SAHF, SAL, SALB, SALW, SALL, SAR, SARB, SARW, SARL, SBB,
        SBBB, SBBW, SBBL, SCASB, SCASD, SCASW, SETA, SETAE, SETB, SETBE, SETC, SETE, SETG, SETGE,
        SETL, SETLE, SETNA, SETNAE, SETNB, SETNBE, SETNC, SETNE, SETNG, SETNGE, SETNL, SETNLE,
        SETNO, SETNP, SETNS, SETNZ, SETO, SETP, SETPE, SETPO, SETS, SETZ, SGDT, SHL, SHLB, SHLW,
        SHLL, SHLD, SHR, SHRB, SHRW, SHRL, SHRD, SIDT, SLDT, SMSW, STC, STD, STI, STOSB, STOSD,
        STOSW, STR, SUB, SUBB, SUBW, SUBL, TEST, TESTB, TESTW, TESTL, VERR, VERW, WAIT, WBINVD,
        XADD, XADDB, XADDW, XADDL, XCHG, XCHGB, XCHGW, XCHGL, XLAT, XLATB, XOR, XORB, XORW, XORL},
  keywordstyle=\color{blue}\bfseries,
  ndkeywordstyle=\color{darkgray}\bfseries,
  identifierstyle=\color{black},
  sensitive=false,
  comment=[l]{\#},
  morecomment=[s]{/*}{*/},
  commentstyle=\color{purple}\ttfamily,
  stringstyle=\color{red}\ttfamily,
  morestring=[b]',
  morestring=[b]"
}

\lstset{language=assembler, style=codestyle}

% disponi sezioni
\usepackage{titlesec}

\titleformat{\section}
	{\sffamily\Large\bfseries} 
	{\thesection}{1em}{} 
\titleformat{\subsection}
	{\sffamily\large\bfseries}   
	{\thesubsection}{1em}{} 
\titleformat{\subsubsection}
	{\sffamily\normalsize\bfseries} 
	{\thesubsubsection}{1em}{}

% tikz
\usepackage{tikz}

% float
\usepackage{float}

% grafici
\usepackage{pgfplots}
\pgfplotsset{width=10cm,compat=1.9}

% disponi alberi
\usepackage{forest}

\forestset{
	rectstyle/.style={
		for tree={rectangle,draw,font=\large\sffamily}
	},
	roundstyle/.style={
		for tree={circle,draw,font=\large}
	}
}

% disponi algoritmi
\usepackage{algorithm}
\usepackage{algorithmic}
\makeatletter
\renewcommand{\ALG@name}{Algoritmo}
\makeatother

% disponi numeri di pagina
\usepackage{fancyhdr}
\fancyhf{} 
\fancyfoot[L]{\sffamily{\thepage}}

\makeatletter
\fancyhead[L]{\raisebox{1ex}[0pt][0pt]{\sffamily{\@title \ \@date}}} 
\fancyhead[R]{\raisebox{1ex}[0pt][0pt]{\sffamily{\@author}}}
\makeatother

\begin{document}
% sezione (data)
\section{Lezione del 07-11-24}

% stili pagina
\thispagestyle{empty}
\pagestyle{fancy}

% testo
\subsection{Reti sequenziali sincronizzate}
Le reti sequenziali sincronizzate (RSS), a differenza della asincrone (RSA), non si aggiornano per la sola variazione degli ingressi, ma per l'arrivo di un determinato segnale periodico, che chiamiamo \textbf{clock}.

Il clock è un segnale con forma d'onda periodica, di frequenza $1 \over T$ periodo, e \textit{duty cicle} (ciclo di lavoro) $\tau \over T$ intorno al 50\%.
Solitamente l'evento di sincronia delle reti sequenziali sincronizzate è il \textbf{fronte di salita} del clock.

\subsection{Registri}
Un registro a $W$ bit è una collezione di $W$ D flip-flop positive edge-triggered, che hanno:
\begin{itemize}
	\item $W$ ingressi $d_i$ e $W$ uscite $q_i$ separate (in verità ricordiamo che troviamo sempre $q$ e $\overline{q}$ negata, noi riporteremo solo la prima per semplicità);
	\item Un ingresso $p$ in parallelo a tutti gli ingressi $p_i$ dei singoli D flip-flop.
\end{itemize}

Si ha che $p$ funge da \textbf{segnale di sincronizzazione} (effettivamente il nostro \textit{clock}).
Consideriamo quindi le variabili di entrata e di uscita di un registro come due singole variabili a più bit, $d_{W-1}\_d_0$ e $q_{W-1}\_q_0$.

\subsubsection{Pilotaggio di registri}
Per il corretto pilotaggio di un registro gli ingressi $d_i$ devono essere stabili intorno al fronte di salita del clock, per un tempo $T_{setup}$ prima e $T_{hold}$ dopo il fronte stesso.
L'uscita cambia dopo, come avevamo visto per i D flip-flop, un tempo $T_{prop} > T_{hold}$.

Tutto cio che accade in ingresso fra due istanti di sincronizzazione è irrilevante e non viene memorizzato.

\par\smallskip

Il registro \textit{memorizza} lo stato di ingresso al \textbf{fronte di salita}.
Gli stati di ingresso fra due fronti di salita adiacenti possono essere identici, adiacenti o non adiacenti: è irrilevante in quanto, come abbiamo detto, l'aggiornamento accade soltanto nelle condizioni di stabilità intorno al fronte di salita del clock.

Dopo il fronte di salita, le uscite cambiano il loro valore dopo $T_{prop}$.

\par\medskip

Possiamo quindi aggiornare la nostra definizione di RSS come \textit{collezione di registri e reti combinatorie}, montati arbitrariamente, purchè non ci siano anelli di retroazione di reti combinatorie (costituirebbero reti sequenziali asincrone).
I registri hanno tutti lo stesso clock in comune, e possono formare anelli, in quanto abbiamo visto dal loro pilotaggio, questo non genera problemi.

\subsubsection{Regole di pilotaggio per RSS}
Dato l'$i$-esimo fronte di salita del clock al tempo $t_i$, lo stato di ingresso ai registri dovrà essere stabile, dalle loro regole di pilotaggio, nell'intervallo $[t_i - T_{setup}, t_i + T_{hold}]$.
Non potrò quindi scegliere periodi $T$ del clock piccoli a piacere: dovrò lasciare tempo ai registri di produrre nuovi valori (in tempo $T_{prop}$), e alle reti combinatorie di elaborare tali valori coi loro tempi di ritardo interni,e quindi di propagarsi nuovamente fino ai registri.

Definiamo, nello specifico, i ritardi:
\begin{itemize}
	\item $T_{in\_to\_reg}$: il tempo di attraversamento massimo della catena di sole reti combinatorie che da uno degli ingressi della rete all'ingresso di un registro;
	\item $T_{reg\_to\_reg}$: il tempo di attravarsamento massimo fra l'uscita e l'ingresso di un registro;
	\item $T_{in\_to\_out}$: il tempo di attraversamento massimo fra un ingresso e un uscita dell'intera rete;
	\item $T_{reg\_to\_out}$: il tempo di attraversamento massimo fra l'uscita di un registro e un uscita della rete.
\end{itemize}

Dobbiamo introdurre poi i tempi $T_{a\_monte}$ e $T_{a\_valle}$, cioè i tempi necessari all'utente della rete per, rispettivamente, \textbf{modificare} gli ingressi e \textbf{leggere} le uscite.
Questi formano due ulteriori vincoli di pilotaggio in ingresso e in uscita.

Queste variabili di temporizzazione daranno vita ad un sistema di 4 diseguaglianze. Vediamole nel dettaglio:
\begin{itemize}
	\item $T \geq T_{hold} + T_{a\_monte} + T_{in\_to\_reg} + T_{setup}$ \\ 
		Questa diseguaglianza assicura che un registro abbia tempo $T_{hold}$ di immagazinare il valore dello scorso ciclo, l'utente esterno abbia tempo $T_{a\_monte}$ di modificare l'ingresso della rete, e che questo ingresso abbia tempo di arrivare ai registri $T_{in\_to\_reg}$ prima del tempo di setup $T_{setup}$ degli stessi, che sappiamo essere necessario perchè al clock seguente i registri memorizzino effettivamente il valore (dopo $T_{hold}$, e lo replichino dopo $T_{prop}$);
	\item $T \geq T_{prop} + T_{reg\_to\_reg} + T_{setup}$ \\ 
		Questa diseguaglianza assicura che il valore generato dai registri possa propagarsi dopo $T_{prop}$, arrivare ai registri stessi in tempo $T_{reg\_to\_reg}$ per il $T_{setup}$ necessario perche lo memorizzino.
		In sostanza, è come la precedente ma riferita alle uscite dei registri anziché dell'utente;
	\item $T \geq T_{hold} + T_{a\_monte} + T_{in\_to\_out} + T_{a\_valle}$ \\
		Questa diseguaglianza assicura che la rete abbia tempo di aggiornarsi dopo un ingresso dell'utente ($T_{a\_monte}$), e restituire il risultato per un tempo che basti all'utente per leggere l'uscita ($T_{a\_valle}$).
		Nello specifico, sappiamo che l'utente non proverà a modificare gli ingressi della rete prima del $T_{hold}$ necessario ad aggiornare i registri al positive edge del clock, e quindi impiegherà un tempo $T_{a\_monte}$ per farlo.
		A questo punto, il segnale di uscità dovra viaggiare almeno dall'ingresso all'uscita, quindi si dovrà aspettare un tempo $T_{in\_to\_out}$, e infine resterà il tempo $T_{a\_valle}$ perchè l'utente abbia modo di effettuare la lettura.
		Notiamo che questa legge si rende necessaria in quanto un aggiornamento degli ingressi può comportare un aggiornamento delle uscite \textit{prima} che i registri ne rispondano.
		In altre parole, reti di questo tipo non sono automaticamente \textbf{trasparenti};
	\item $T \geq T_{prop} + T_{reg\_to\_out} + T_{a\_valle}$ \\ 
		Quest'ultima diseguaglianza assicura che la rete abbia tempo di aggiornare le sue uscite, e quindi farle leggere all'utente ($T_{a\_valle}$), a memorizzazione effettuata dei registri.
		Nello specifico, i registri otterrano il valore al ciclo corrente nel tempo compreso fra $T_{setup}$ e $T_{hold}$ centrato sul positive edge dello scorso clock, e quindi si adegueranno dopo un tempo $T_{prop}$ rispetto al positive edge stesso.
		Di qui in poi dovremo aspettare un tempo $T_{reg\_to\_out}$ perchè questo valore attraversi la rete fino alle uscite, e infine il tempo $T_{a\_valle}$ perchè l'utente abbia modo di effettuare la lettura.
		Questa legge si rende necessaria, al contrario della precedente, sia per reti \textit{trasparenti} che per reti \textbf{non trasparenti}, e anzi vedremo che reti non trasparenti saranno proprio i registri a fornire le uscite. 
\end{itemize}

Possiamo quindi porre il sistema completo: 
\[
	\begin{cases}
		T \geq T_{hold} + T_{a\_monte} + T_{in\_to\_reg} + T_{setup} \\ 
		T \geq T_{prop} + T_{reg\_to\_reg} + T_{setup} \\ 
		T \geq T_{hold} + T_{a\_monte} + T_{in\_to\_out} + T_{a\_valle} \\
		T \geq T_{prop} + T_{reg\_to\_out} + T_{a\_valle}
	\end{cases}
\]

Dove, riassumendo, le prime due condizioni garantiscono che lo stato delle variabili di ingresso resti stabile negli intervalli $(-T_{setup}, T_{hold})$ centrati sui positive edge di ogni clock; la prima e la terza tengono conto del mondo esterno \textit{a monte}, quindi in fase di scrittura; la seconda e la quarta tengono conto del mondo esterno \textit{a valle}, quindi in fase di lettura. 

In verità, avremo altri due ritardi di cui tenere conto:
\begin{itemize}
	\item $T_{sfas}$: il \textbf{massimo sfasamento} fra due clock.
		Visto che questo viene portato a elementi diversi, a qualche registro arrivera prima e a qualche registro arrivera dopo;
	\item $T_{reg}$: se un registro è formato da $W$ > 1 bit, questi non cambieranno tutti contemporaneamente: dovremmo aggiungere $T_{prop} + T_{reg} = T'_{prop}$. A questo punto, però, possiamo considerare solo $T_{prop} \leftarrow T'_{prop}$ e ignorare $T_{reg}$.
\end{itemize}

\subsubsection{Anticipazioni sui modelli di Moore e di Mealy ritardato}
Potremmo voler determinare qual'è la più vincolante fra le diseguaglianze riportate prima.
Questa, chiaramente, è quella che copre il percorso più lungo, cioè la terza.
Se decidiamo di vietare il percorso che copre, cioè quello diretto fra ingressi e uscite, otteniamo il cosiddetto \textbf{modello di Moore}: cioè, un modello di RSS dove non si ammettono reti combinatorie che collegano gli ingressi direttamente alle uscite.

Un'altro vincolo che potremmo voler rilassare è il quarto, nel cosiddetto \textbf{modello di Mealy ritardato}.
Questo equivale a prelevare le uscite direttamente dalle uscite dei registri, cioè a eliminare il tempo $T_{reg\_to\_out}$.

\subsection{Contatori}
Un contatore è una RSS il cui stato di uscita può essere visto come un \textbf{numero naturale} ad $n$ cifre in base $\beta$.
Ad ogni clock, il contatore \textbf{incrementa} o \textbf{decrementa}.

Abbiamo che si può realizzare un contatore collegando un modulo sommatore a $n$ cifre a un registro a $n$ cifre.
L'uscita del registro viene collegata in anello di retrazione a uno degli ingressi del sommatore.
Impostando il $C_{in}$ del sommatore a 1, e il suo secondo ingresso ad un'array di $n$ generatori di costante 0, si ha un contatore \textbf{incrementatore}, cioè che incrementa il suo valore ad ogni ciclo di clock.
	L'equivalente \textbf{decrementatore} si può creare usando un sottrattore a $n$ cifre invece di un sommatore.veri

Si può creare un contatore con ingresso di abilitazione (sostanzialmente una \textbf{variabile di controllo}), cioè che incrementa o decrementa solo se è alto un certo bit di controllo, collegando tale bit al carry (o al borrow) del sommatore (sottrattore).

\subsection{Contatore a una cifra in base 2}
Vediamo quindi come realizzare un contatore a una cifra in base $\beta = 2$.
Se l'intenzione è di creare un contatore per $N$ cifre in codifica binaria, questo rappresenterà l'elemento fondamentale (che andremo a combinare nei prossimi paragrafi, attraverso catene di \textbf{ripple carry}).

Avremo bisogno di un input, $e_i$ oltre al clock e al reset, che rappresenterà il riporto entrante dell'incrementatore (che come abbiamo visto può fungere da variabile di controllo, in quanto lasciare $e_i$ a 0 significa sommare 0 a un numero, quindi lasciarlo invariato),
Prenderemo poi due uscite: $q$, cioè l'uscita vera e propria dal registro del contatore (che chiameremo \lstinline|OUTR|), e $e_u$, il riporto uscente.
Nel caso di un contatore a una cifra in base 2, il riporto uscente si riduce al valore dell'AND fra $q$ ed $e_i$, ergo se $q$ è alto e introduciamo un riporto entrante $e_i$, andremo al di fuori della rappresentazione possibile su un bit e dovremo passare il riporto (che possiamo anche qui intendere come segnale di controllo) al prossimo contatore della catena.
L'uscita dell'incrementatore (chiamiamola $a$) andrà messa in \lstinline|OUTR| ad ogni aggiornamento, e verrà aggiornata dai valori di $q$, cioè l'uscita stessa di \lstinline|OUTR| (ciclo di retroazione) e $e_i$, attraverso la logica dell'incrementatore che riportiamo sotto forma di tabella di verità:

\begin{table}[H]
	\center 
	\begin{tabular} { c c | c c }
		$q$ & $e_i$ & $a$ & $e_u$ \\ 
		\hline
		0 & 0 & 0 & 0 \\ 
		0 & 1 & 1 & 0 \\ 
		1 & 0 & 1 & 0 \\ 
		1 & 1 & 0 & 1 \\
	\end{tabular}
\end{table}

Notiamo che non abbiamo mai l'uscita $1, 1$ su $a, e_u$, in quanto uno degli ingressi del sommatore che forma l'incrementatore è "fissato" a 0.
Possiamo quindi dare la sintesi in Verilog:

\lstinputlisting[language=verilog, style=codestyle]{../verilog/11-07/b2_counter.v}
sfruttando la definizione già data di incrementatore (cioè un \textit{half adder}).

\subsubection{Contatore a una cifra in base 3}
Con procedimenti simili a quelli dell'esempio precedente si può ricavare un contatore a una cifra in base $\beta = 3$.
In questo caso la tabella di verità dell'incrementatore sarà:
\begin{table}[H]
	\center 
	\begin{tabular} { c c c | c c c }
		$q_1$ & $q_0$ & $e_i$ & $a_1$ & $a_0$ & $e_u$ \\ 
		\hline
		0 & 0 & 0 & 0 & 0 & 0 \\ 
		0 & 0 & 1 & 0 & 1 & 0 \\ 
		0 & 1 & 0 & 0 & 1 & 0 \\ 
		0 & 1 & 1 & 1 & 0 & 0 \\
		1 & 0 & 0 & 1 & 0 & 0 \\ 
		1 & 0 & 1 & 0 & 0 & 1 \\ 
		1 & 1 & 0 & - & - & - \\ 
		1 & 1 & 1 & - & - & -
	\end{tabular}
\end{table}

e la sintesi sarà:
\lstinputlisting[language=verilog, style=codestyle]{../verilog/11-07/b2_counter.v}
sfruttando la definizione già data di incrementatore in base 3.

\subsubection{Contatore a una cifra in base 10}
Infine, vediamo come si può ricavare un contatore a una cifra in base $\beta = 10$.
In questo caso la tabella di verità dell'incrementatore sarà:

\begin{table}[H]
	\center 
	\begin{tabular} { c c  c  c  c | c  c  c  c  c }
		$x_3$ & $x_2$ & $x_1$ & $x_0$ & $e_i$ & $z_3$ & $z_2$ & $z_1$ & $z_0$ & $e_u$ \\ 
		\hline
		0 & 0 & 0 & 0 & 0 & 0 & 0 & 0 & 0 & 0 \\
		0 & 0 & 0 & 0 & 1 & 0 & 0 & 0 & 1 & 0 \\
		0 & 0 & 0 & 1 & 0 & 0 & 0 & 0 & 1 & 0 \\
		0 & 0 & 0 & 1 & 1 & 0 & 0 & 1 & 0 & 0 \\
		0 & 0 & 1 & 0 & 0 & 0 & 0 & 1 & 0 & 0 \\
		0 & 0 & 1 & 0 & 1 & 0 & 0 & 1 & 1 & 0 \\
		0 & 0 & 1 & 1 & 0 & 0 & 0 & 1 & 1 & 0 \\
		0 & 0 & 1 & 1 & 1 & 0 & 1 & 0 & 0 & 0 \\
		0 & 1 & 0 & 0 & 0 & 0 & 1 & 0 & 0 & 0 \\
		0 & 1 & 0 & 0 & 1 & 0 & 1 & 0 & 1 & 0 \\
		0 & 1 & 0 & 1 & 0 & 0 & 1 & 0 & 1 & 0 \\
		0 & 1 & 0 & 1 & 1 & 0 & 1 & 1 & 0 & 0 \\
		0 & 1 & 1 & 0 & 0 & 0 & 1 & 1 & 0 & 0 \\
		0 & 1 & 1 & 0 & 1 & 0 & 1 & 1 & 1 & 0 \\
		0 & 1 & 1 & 1 & 0 & 0 & 1 & 1 & 1 & 0 \\
		0 & 1 & 1 & 1 & 1 & 1 & 0 & 0 & 0 & 0 \\
		1 & 0 & 0 & 0 & 0 & 1 & 0 & 0 & 0 & 0 \\
		1 & 0 & 0 & 0 & 1 & 1 & 0 & 0 & 1 & 0 \\
		1 & 0 & 0 & 1 & 0 & 1 & 0 & 0 & 1 & 0 \\
		1 & 0 & 0 & 1 & 1 & 0 & 0 & 0 & 0 & 1 \\
		1 & 0 & 1 & 0 & 0 & - & - & - & - & - \\
		ecc...
	\end{tabular}
\end{table}

e la sintesi sarà:
\lstinputlisting[language=verilog, style=codestyle]{../verilog/11-07/b10_counter.v}
sfruttando la definizione già data di incrementatore in base 10.

\subsubsection{Scomposizione in moduli di contatori}
Un contatore può essere scomposto, in qualsiasi base, in una serie di contatori ad una cifra collegati a \textbf{catena di riporti} (\textit{ripple carry}).
In questo caso il registro è dato dalla combinazione di $n$ registri, uno per ogni cifra (e quindi per ogni contatore), tutti sincronizzati sullo stesso clock.
All'attivarsi del riporto di uno dei contatori, il successivo nella catena si attiva, e così via in un processo del tutto identico al propagarsi delle somme nei sommatori a più cifre.
Si riportano le descrizioni in Verilog di sommatori a quattro cifre, in base $\beta = 2$:

\lstinputlisting[language=verilog, style=codestyle]{../verilog/11-07/n4_b2_counter.v}

in base $\beta = 3$:

\lstinputlisting[language=verilog, style=codestyle]{../verilog/11-07/n4_b3_counter.v}

e infine in base $\beta = 10$:

\lstinputlisting[language=verilog, style=codestyle]{../verilog/11-07/n4_b10_counter.v}

\end{document}


\documentclass[a4paper,11pt]{article}
\usepackage[a4paper, margin=8em]{geometry}

% usa i pacchetti per la scrittura in italiano
\usepackage[french,italian]{babel}
\usepackage[T1]{fontenc}
\usepackage[utf8]{inputenc}
\frenchspacing 

% usa i pacchetti per la formattazione matematica
\usepackage{amsmath, amssymb, amsthm, amsfonts}

% usa altri pacchetti
\usepackage{gensymb}
\usepackage{hyperref}
\usepackage{standalone}

\usepackage{colortbl}

\usepackage{xstring}
\usepackage{karnaugh-map}

% imposta il titolo
\title{Appunti Reti Logiche}
\author{Luca Seggiani}
\date{2024}

% imposta lo stile
% usa helvetica
\usepackage[scaled]{helvet}
% usa palatino
\usepackage{palatino}
% usa un font monospazio guardabile
\usepackage{lmodern}

\renewcommand{\rmdefault}{ppl}
\renewcommand{\sfdefault}{phv}
\renewcommand{\ttdefault}{lmtt}

% circuiti
\usepackage{circuitikz}
\usetikzlibrary{babel}

% disponi il titolo
\makeatletter
\renewcommand{\maketitle} {
	\begin{center} 
		\begin{minipage}[t]{.8\textwidth}
			\textsf{\huge\bfseries \@title} 
		\end{minipage}%
		\begin{minipage}[t]{.2\textwidth}
			\raggedleft \vspace{-1.65em}
			\textsf{\small \@author} \vfill
			\textsf{\small \@date}
		\end{minipage}
		\par
	\end{center}

	\thispagestyle{empty}
	\pagestyle{fancy}
}
\makeatother

% disponi teoremi
\usepackage{tcolorbox}
\newtcolorbox[auto counter, number within=section]{theorem}[2][]{%
	colback=blue!10, 
	colframe=blue!40!black, 
	sharp corners=northwest,
	fonttitle=\sffamily\bfseries, 
	title=Teorema~\thetcbcounter: #2, 
	#1
}

% disponi definizioni
\newtcolorbox[auto counter, number within=section]{definition}[2][]{%
	colback=red!10,
	colframe=red!40!black,
	sharp corners=northwest,
	fonttitle=\sffamily\bfseries,
	title=Definizione~\thetcbcounter: #2,
	#1
}

% disponi codice
\usepackage{listings}
\usepackage[table]{xcolor}

\definecolor{codegreen}{rgb}{0,0.6,0}
\definecolor{codegray}{rgb}{0.5,0.5,0.5}
\definecolor{codepurple}{rgb}{0.58,0,0.82}
\definecolor{backcolour}{rgb}{0.95,0.95,0.92}

\lstdefinestyle{codestyle}{
		backgroundcolor=\color{black!5}, 
		commentstyle=\color{codegreen},
		keywordstyle=\bfseries\color{magenta},
		numberstyle=\sffamily\tiny\color{black!60},
		stringstyle=\color{green!50!black},
		basicstyle=\ttfamily\footnotesize,
		breakatwhitespace=false,         
		breaklines=true,                 
		captionpos=b,                    
		keepspaces=true,                 
		numbers=left,                    
		numbersep=5pt,                  
		showspaces=false,                
		showstringspaces=false,
		showtabs=false,                  
		tabsize=2
}

\lstdefinestyle{shellstyle}{
		backgroundcolor=\color{black!5}, 
		basicstyle=\ttfamily\footnotesize\color{black}, 
		commentstyle=\color{black}, 
		keywordstyle=\color{black},
		numberstyle=\color{black!5},
		stringstyle=\color{black}, 
		showspaces=false,
		showstringspaces=false, 
		showtabs=false, 
		tabsize=2, 
		numbers=none, 
		breaklines=true
}


\lstdefinelanguage{assembler}{ 
  keywords={AAA, AAD, AAM, AAS, ADC, ADCB, ADCW, ADCL, ADD, ADDB, ADDW, ADDL, AND, ANDB, ANDW, ANDL,
        ARPL, BOUND, BSF, BSFL, BSFW, BSR, BSRL, BSRW, BSWAP, BT, BTC, BTCB, BTCW, BTCL, BTR, 
        BTRB, BTRW, BTRL, BTS, BTSB, BTSW, BTSL, CALL, CBW, CDQ, CLC, CLD, CLI, CLTS, CMC, CMP,
        CMPB, CMPW, CMPL, CMPS, CMPSB, CMPSD, CMPSW, CMPXCHG, CMPXCHGB, CMPXCHGW, CMPXCHGL,
        CMPXCHG8B, CPUID, CWDE, DAA, DAS, DEC, DECB, DECW, DECL, DIV, DIVB, DIVW, DIVL, ENTER,
        HLT, IDIV, IDIVB, IDIVW, IDIVL, IMUL, IMULB, IMULW, IMULL, IN, INB, INW, INL, INC, INCB,
        INCW, INCL, INS, INSB, INSD, INSW, INT, INT3, INTO, INVD, INVLPG, IRET, IRETD, JA, JAE,
        JB, JBE, JC, JCXZ, JE, JECXZ, JG, JGE, JL, JLE, JMP, JNA, JNAE, JNB, JNBE, JNC, JNE, JNG,
        JNGE, JNL, JNLE, JNO, JNP, JNS, JNZ, JO, JP, JPE, JPO, JS, JZ, LAHF, LAR, LCALL, LDS,
        LEA, LEAVE, LES, LFS, LGDT, LGS, LIDT, LMSW, LOCK, LODSB, LODSD, LODSW, LOOP, LOOPE,
        LOOPNE, LSL, LSS, LTR, MOV, MOVB, MOVW, MOVL, MOVSB, MOVSD, MOVSW, MOVSX, MOVSXB,
        MOVSXW, MOVSXL, MOVZX, MOVZXB, MOVZXW, MOVZXL, MUL, MULB, MULW, MULL, NEG, NEGB, NEGW,
        NEGL, NOP, NOT, NOTB, NOTW, NOTL, OR, ORB, ORW, ORL, OUT, OUTB, OUTW, OUTL, OUTSB, OUTSD,
        OUTSW, POP, POPL, POPW, POPB, POPA, POPAD, POPF, POPFD, PUSH, PUSHL, PUSHW, PUSHB, PUSHA, 
				PUSHAD, PUSHF, PUSHFD, RCL, RCLB, RCLW, MOVSL, MOVSB, MOVSW, STOSL, STOSB, STOSW, LODSB, LODSW,
				LODSL, INSB, INSW, INSL, OUTSB, OUTSL, OUTSW
        RCLL, RCR, RCRB, RCRW, RCRL, RDMSR, RDPMC, RDTSC, REP, REPE, REPNE, RET, ROL, ROLB, ROLW,
        ROLL, ROR, RORB, RORW, RORL, SAHF, SAL, SALB, SALW, SALL, SAR, SARB, SARW, SARL, SBB,
        SBBB, SBBW, SBBL, SCASB, SCASD, SCASW, SETA, SETAE, SETB, SETBE, SETC, SETE, SETG, SETGE,
        SETL, SETLE, SETNA, SETNAE, SETNB, SETNBE, SETNC, SETNE, SETNG, SETNGE, SETNL, SETNLE,
        SETNO, SETNP, SETNS, SETNZ, SETO, SETP, SETPE, SETPO, SETS, SETZ, SGDT, SHL, SHLB, SHLW,
        SHLL, SHLD, SHR, SHRB, SHRW, SHRL, SHRD, SIDT, SLDT, SMSW, STC, STD, STI, STOSB, STOSD,
        STOSW, STR, SUB, SUBB, SUBW, SUBL, TEST, TESTB, TESTW, TESTL, VERR, VERW, WAIT, WBINVD,
        XADD, XADDB, XADDW, XADDL, XCHG, XCHGB, XCHGW, XCHGL, XLAT, XLATB, XOR, XORB, XORW, XORL},
  keywordstyle=\color{blue}\bfseries,
  ndkeywordstyle=\color{darkgray}\bfseries,
  identifierstyle=\color{black},
  sensitive=false,
  comment=[l]{\#},
  morecomment=[s]{/*}{*/},
  commentstyle=\color{purple}\ttfamily,
  stringstyle=\color{red}\ttfamily,
  morestring=[b]',
  morestring=[b]"
}

\lstset{language=assembler, style=codestyle}

% disponi sezioni
\usepackage{titlesec}

\titleformat{\section}
	{\sffamily\Large\bfseries} 
	{\thesection}{1em}{} 
\titleformat{\subsection}
	{\sffamily\large\bfseries}   
	{\thesubsection}{1em}{} 
\titleformat{\subsubsection}
	{\sffamily\normalsize\bfseries} 
	{\thesubsubsection}{1em}{}

% tikz
\usepackage{tikz}

% float
\usepackage{float}

% grafici
\usepackage{pgfplots}
\pgfplotsset{width=10cm,compat=1.9}

% disponi alberi
\usepackage{forest}

\forestset{
	rectstyle/.style={
		for tree={rectangle,draw,font=\large\sffamily}
	},
	roundstyle/.style={
		for tree={circle,draw,font=\large}
	}
}

% disponi algoritmi
\usepackage{algorithm}
\usepackage{algorithmic}
\makeatletter
\renewcommand{\ALG@name}{Algoritmo}
\makeatother

% disponi numeri di pagina
\usepackage{fancyhdr}
\fancyhf{} 
\fancyfoot[L]{\sffamily{\thepage}}

\makeatletter
\fancyhead[L]{\raisebox{1ex}[0pt][0pt]{\sffamily{\@title \ \@date}}} 
\fancyhead[R]{\raisebox{1ex}[0pt][0pt]{\sffamily{\@author}}}
\makeatother

\begin{document}
% sezione (data)
\section{Lezione del 08-11-24}

% stili pagina
\thispagestyle{empty}
\pagestyle{fancy}

% testo
\subsection{Contatori e divisione in frequenza}
Si ha che i contatori \textit{contano} i cicli di clock a cui sono sottoposti (cioè incrementano per ogni ciclo).
Possiamo usare un contatore per \textbf{dividere} la frequenza del clock per un dato valore.
Ad esempio, il MSB di un contatore in base 3 va 3 volte più lento del clock che lo pilota.
In generale, si ha che per un contatore a $N$ cifre in base 2 che riceve clock a periodo $T$, l'MSB è a periodo $2^N \cdot T$.

Si potrebbe pensare di usare l'uscita di riporto del contatore in MSB come uscita divisa del clock: questo non è raccomandabile in quanto l'uscita di riporto è un \textbf{uscita combinatoria}, che non è né \textbf{stabile} né a \textbf{temporizzazione certa} (a differenza dell'uscita di un registro).

\subsection{Registro multifunzionale}
Un registro multifunzionale è una rete che, all'arrivo del clock, memorizza nello stesso registro una tra $K$ funzioni combinatorie possibili, scelte impostando un certo numero di variabili di comando $W = \log_2 K$.

L'implementazione effettiva del registro è data da un multiplexer da $0$ a $K-1$ reti combinatorie, dove $W$ è la variabile di comando, la cui uscita viene inviata a un certo registro (che spedisce poi la sua uscita in retroazione alle reti combinatorie funzionali, e cosi via).

In Verilog, un'implementazione di esempio (non funzionante in quanto non completamente specificata) può essere la seguente:

\lstinputlisting[language=verilog, style=codestyle]{../verilog/11-08/multifunction_registers/n4_multifunction_register.v}

Fra le casisistiche di utilizzo dei registri multifunzionali c'è quella dei \textbf{registri di shift}, cioè registri che sulla base di una variabile di comando memorizzano la prossima variabile in ingresso, o spostano la variabile immagazzinata verso destra o sinistra (implementando effettivamente le \lstinline|SHR| e \lstinline|SHL| dell'assembler).
Vediamo entrambe le casistiche in Verilog, prima lo shift a destra:

\lstinputlisting[language=verilog, style=codestyle]{../verilog/11-08/multifunction_registers/n4_rshift_register.v}

e poi il (perlopiù analogo) shift a sinistra:

\lstinputlisting[language=verilog, style=codestyle]{../verilog/11-08/multifunction_registers/n4_lshift_register.v}

\subsection{Modello di reti sequenziali sincronizzate}
Definiamo tre modelli per le RSS (che abbiamo già introdotto parlando delle regole di pilotaggio):

\subsubsection{Modello di Moore}
Una RSS di Moore è definita a partire da:
\begin{itemize}
	\item Un insieme di $N$ variabili logiche di ingressi:
	\item Un insieme di $M$ variabili logiche di uscita;
	\item Un \textbf{meccanismo di marcatura}, che a ogni istante marca uno \textbf{stato interno presente}, scelto fra $K$ finito stati interni $S \equiv \{ S_0, ..., S_{K-1} \}$;
	\item Una legge di evoluzione nel tempo $A: X \times S \rightarrow S$, che mappa una coppia, data da un $X$ stato di ingresso e un elemento $s \in S$ stato interno, ad un nuovo stato interno (diciamo $s' \in S$);
	\item Una legge di evoluzione nel tempo $B:S \rightarrow Z$, che mappa uno stato interno $s \in S$ a uno stato di uscita $Z$.
\end{itemize}

La rete riceve \textbf{segnali di sincronizzazione}, come ad esempio le transizioni da 0 a 1 del segnale di clock (avevamo detto il \textit{leading edge}).
La legge di temporizzazione di una RSS di Moore è quindi la seguente: dato un elemento $s \in S$, stato interno marcato ad un certo istante, e dato $X$ ingresso ad un certo istante immediatamente precedente l'arrivo di un segnale di sincronizzazione:

\begin{enumerate}
	\item Si individua il nuovo stato interno da marcare $s' = A(X, s)$;
	\item Si aspetta $T_{prop}$ dopo l'arrivo del segnale di sincronizzazione;
	\item Si promuove $s'$ al rango di \textbf{stato interno marcato}.
\end{enumerate}

Lo stato interno marcato viene memorizzato in un apposito registro, detto \textbf{STAR} (da \textit{STAtus Register}).
Questo viene implementato con una batteria di D flip-flop non trasparenti.

In una RSS di Moore si hanno quindi i seguenti vincoli di pilotaggio:
\[
	\begin{cases}
		T \geq T_{hold} + T_{a\_monte} + T_A + T_{setup} \\ 
		T \geq T_{prop} + T_A + T_{setup} \\ 
		T \geq T_{prop} + T_Z + T_{a\_valle}
	\end{cases}
\]

che riguardano rispettivamente i tempi ingresso-STAR, STAR-STAR e STAR-uscita.

\subsubsection{Flip-flop JK}
Un'esempio di RSS di Moore è il flip-flop JK, che valuta due ingressi $j$ e $k$ e si comporta come segue:
\begin{table}[h!]
	\center \rowcolors{2}{white}{black!10}
	\begin{tabular} { c | c | c }
		$j$ & $k$ & Azione \\ 
		\hline 
		0 & 0 & Conserva \\ 
		1 & 0 & Setta \\ 
		0 & 1 & Resetta \\ 
		1 & 1 & Commuta
	\end{tabular}
\end{table}

Un modo di vedere questa rete è come un registro multifunzionale ad un bit, con tabella di applicazione:
\begin{table}[h!]
	\center 
	\begin{tabular} { c  c | c  c }
		$q$ & $q'$ & $j$ & $k$ \\ 
		\hline 
		0 & 0 & 0 & - \\ 
		0 & 1 & 1 & - \\ 
		1 & 0 & - & 1 \\ 
		1 & 1 & - & 0
	\end{tabular}
\end{table}

Vediamone la sintesi: visto che conosciamo soltanto la sintesi di reti combinatorie attraverso le mappe di Karnaugh, basterà sintetizzare le due reti combinatorie, dalla definizione di rete di Moore. \textbf{RCA} e \textbf{RCB} che implementano le funzioni $A$ e $B$.
Il registro STAR conterrà a questo punto lo stato interno, che nel caso del flip-flop JK ridurrà RCB a un cortocircuito per ogni uscita del registro.

In Verilog, l'implementazione dopo la sintesi come rete di Moore risulterà la seguente:

\lstinputlisting[language=verilog, style=codestyle]{../verilog/11-08/jk_flip_flops/jk_flip_flop.v}

Si nota inoltre che i moduli con definizioni di circuiti di reset (che finora abbiamo relegato al blocco \lstinline|initial| si trovano nella cartella \lstinline|/verilog|, assieme al resto del codice).

\end{document}


\documentclass[a4paper,11pt]{article}
\usepackage[a4paper, margin=8em]{geometry}

% usa i pacchetti per la scrittura in italiano
\usepackage[french,italian]{babel}
\usepackage[T1]{fontenc}
\usepackage[utf8]{inputenc}
\frenchspacing 

% usa i pacchetti per la formattazione matematica
\usepackage{amsmath, amssymb, amsthm, amsfonts}

% usa altri pacchetti
\usepackage{gensymb}
\usepackage{hyperref}
\usepackage{standalone}

\usepackage{colortbl}

\usepackage{xstring}
\usepackage{karnaugh-map}

% imposta il titolo
\title{Appunti Reti Logiche}
\author{Luca Seggiani}
\date{2024}

% imposta lo stile
% usa helvetica
\usepackage[scaled]{helvet}
% usa palatino
\usepackage{palatino}
% usa un font monospazio guardabile
\usepackage{lmodern}

\renewcommand{\rmdefault}{ppl}
\renewcommand{\sfdefault}{phv}
\renewcommand{\ttdefault}{lmtt}

% circuiti
\usepackage{circuitikz}
\usetikzlibrary{babel}

% disponi il titolo
\makeatletter
\renewcommand{\maketitle} {
	\begin{center} 
		\begin{minipage}[t]{.8\textwidth}
			\textsf{\huge\bfseries \@title} 
		\end{minipage}%
		\begin{minipage}[t]{.2\textwidth}
			\raggedleft \vspace{-1.65em}
			\textsf{\small \@author} \vfill
			\textsf{\small \@date}
		\end{minipage}
		\par
	\end{center}

	\thispagestyle{empty}
	\pagestyle{fancy}
}
\makeatother

% disponi teoremi
\usepackage{tcolorbox}
\newtcolorbox[auto counter, number within=section]{theorem}[2][]{%
	colback=blue!10, 
	colframe=blue!40!black, 
	sharp corners=northwest,
	fonttitle=\sffamily\bfseries, 
	title=Teorema~\thetcbcounter: #2, 
	#1
}

% disponi definizioni
\newtcolorbox[auto counter, number within=section]{definition}[2][]{%
	colback=red!10,
	colframe=red!40!black,
	sharp corners=northwest,
	fonttitle=\sffamily\bfseries,
	title=Definizione~\thetcbcounter: #2,
	#1
}

% disponi codice
\usepackage{listings}
\usepackage[table]{xcolor}

\definecolor{codegreen}{rgb}{0,0.6,0}
\definecolor{codegray}{rgb}{0.5,0.5,0.5}
\definecolor{codepurple}{rgb}{0.58,0,0.82}
\definecolor{backcolour}{rgb}{0.95,0.95,0.92}

\lstdefinestyle{codestyle}{
		backgroundcolor=\color{black!5}, 
		commentstyle=\color{codegreen},
		keywordstyle=\bfseries\color{magenta},
		numberstyle=\sffamily\tiny\color{black!60},
		stringstyle=\color{green!50!black},
		basicstyle=\ttfamily\footnotesize,
		breakatwhitespace=false,         
		breaklines=true,                 
		captionpos=b,                    
		keepspaces=true,                 
		numbers=left,                    
		numbersep=5pt,                  
		showspaces=false,                
		showstringspaces=false,
		showtabs=false,                  
		tabsize=2
}

\lstdefinestyle{shellstyle}{
		backgroundcolor=\color{black!5}, 
		basicstyle=\ttfamily\footnotesize\color{black}, 
		commentstyle=\color{black}, 
		keywordstyle=\color{black},
		numberstyle=\color{black!5},
		stringstyle=\color{black}, 
		showspaces=false,
		showstringspaces=false, 
		showtabs=false, 
		tabsize=2, 
		numbers=none, 
		breaklines=true
}


\lstdefinelanguage{assembler}{ 
  keywords={AAA, AAD, AAM, AAS, ADC, ADCB, ADCW, ADCL, ADD, ADDB, ADDW, ADDL, AND, ANDB, ANDW, ANDL,
        ARPL, BOUND, BSF, BSFL, BSFW, BSR, BSRL, BSRW, BSWAP, BT, BTC, BTCB, BTCW, BTCL, BTR, 
        BTRB, BTRW, BTRL, BTS, BTSB, BTSW, BTSL, CALL, CBW, CDQ, CLC, CLD, CLI, CLTS, CMC, CMP,
        CMPB, CMPW, CMPL, CMPS, CMPSB, CMPSD, CMPSW, CMPXCHG, CMPXCHGB, CMPXCHGW, CMPXCHGL,
        CMPXCHG8B, CPUID, CWDE, DAA, DAS, DEC, DECB, DECW, DECL, DIV, DIVB, DIVW, DIVL, ENTER,
        HLT, IDIV, IDIVB, IDIVW, IDIVL, IMUL, IMULB, IMULW, IMULL, IN, INB, INW, INL, INC, INCB,
        INCW, INCL, INS, INSB, INSD, INSW, INT, INT3, INTO, INVD, INVLPG, IRET, IRETD, JA, JAE,
        JB, JBE, JC, JCXZ, JE, JECXZ, JG, JGE, JL, JLE, JMP, JNA, JNAE, JNB, JNBE, JNC, JNE, JNG,
        JNGE, JNL, JNLE, JNO, JNP, JNS, JNZ, JO, JP, JPE, JPO, JS, JZ, LAHF, LAR, LCALL, LDS,
        LEA, LEAVE, LES, LFS, LGDT, LGS, LIDT, LMSW, LOCK, LODSB, LODSD, LODSW, LOOP, LOOPE,
        LOOPNE, LSL, LSS, LTR, MOV, MOVB, MOVW, MOVL, MOVSB, MOVSD, MOVSW, MOVSX, MOVSXB,
        MOVSXW, MOVSXL, MOVZX, MOVZXB, MOVZXW, MOVZXL, MUL, MULB, MULW, MULL, NEG, NEGB, NEGW,
        NEGL, NOP, NOT, NOTB, NOTW, NOTL, OR, ORB, ORW, ORL, OUT, OUTB, OUTW, OUTL, OUTSB, OUTSD,
        OUTSW, POP, POPL, POPW, POPB, POPA, POPAD, POPF, POPFD, PUSH, PUSHL, PUSHW, PUSHB, PUSHA, 
				PUSHAD, PUSHF, PUSHFD, RCL, RCLB, RCLW, MOVSL, MOVSB, MOVSW, STOSL, STOSB, STOSW, LODSB, LODSW,
				LODSL, INSB, INSW, INSL, OUTSB, OUTSL, OUTSW
        RCLL, RCR, RCRB, RCRW, RCRL, RDMSR, RDPMC, RDTSC, REP, REPE, REPNE, RET, ROL, ROLB, ROLW,
        ROLL, ROR, RORB, RORW, RORL, SAHF, SAL, SALB, SALW, SALL, SAR, SARB, SARW, SARL, SBB,
        SBBB, SBBW, SBBL, SCASB, SCASD, SCASW, SETA, SETAE, SETB, SETBE, SETC, SETE, SETG, SETGE,
        SETL, SETLE, SETNA, SETNAE, SETNB, SETNBE, SETNC, SETNE, SETNG, SETNGE, SETNL, SETNLE,
        SETNO, SETNP, SETNS, SETNZ, SETO, SETP, SETPE, SETPO, SETS, SETZ, SGDT, SHL, SHLB, SHLW,
        SHLL, SHLD, SHR, SHRB, SHRW, SHRL, SHRD, SIDT, SLDT, SMSW, STC, STD, STI, STOSB, STOSD,
        STOSW, STR, SUB, SUBB, SUBW, SUBL, TEST, TESTB, TESTW, TESTL, VERR, VERW, WAIT, WBINVD,
        XADD, XADDB, XADDW, XADDL, XCHG, XCHGB, XCHGW, XCHGL, XLAT, XLATB, XOR, XORB, XORW, XORL},
  keywordstyle=\color{blue}\bfseries,
  ndkeywordstyle=\color{darkgray}\bfseries,
  identifierstyle=\color{black},
  sensitive=false,
  comment=[l]{\#},
  morecomment=[s]{/*}{*/},
  commentstyle=\color{purple}\ttfamily,
  stringstyle=\color{red}\ttfamily,
  morestring=[b]',
  morestring=[b]"
}

\lstset{language=assembler, style=codestyle}

% disponi sezioni
\usepackage{titlesec}

\titleformat{\section}
	{\sffamily\Large\bfseries} 
	{\thesection}{1em}{} 
\titleformat{\subsection}
	{\sffamily\large\bfseries}   
	{\thesubsection}{1em}{} 
\titleformat{\subsubsection}
	{\sffamily\normalsize\bfseries} 
	{\thesubsubsection}{1em}{}

% tikz
\usepackage{tikz}

% float
\usepackage{float}

% grafici
\usepackage{pgfplots}
\pgfplotsset{width=10cm,compat=1.9}

% disponi alberi
\usepackage{forest}

\forestset{
	rectstyle/.style={
		for tree={rectangle,draw,font=\large\sffamily}
	},
	roundstyle/.style={
		for tree={circle,draw,font=\large}
	}
}

% disponi algoritmi
\usepackage{algorithm}
\usepackage{algorithmic}
\makeatletter
\renewcommand{\ALG@name}{Algoritmo}
\makeatother

% disponi numeri di pagina
\usepackage{fancyhdr}
\fancyhf{} 
\fancyfoot[L]{\sffamily{\thepage}}

\makeatletter
\fancyhead[L]{\raisebox{1ex}[0pt][0pt]{\sffamily{\@title \ \@date}}} 
\fancyhead[R]{\raisebox{1ex}[0pt][0pt]{\sffamily{\@author}}}
\makeatother

\begin{document}
% sezione (data)
\section{Lezione del 12-11-24}

% stili pagina
\thispagestyle{empty}
\pagestyle{fancy}

% testo
\subsection{Riconoscitore di sequenze}
Un riconoscitore di sequenze è una rete sequenziale sincronizzata a $N$ ingressi ed un uscita.
Questa rete si evolve secondo la legge seguente: se si presenta, in sequenza, una sequenza di stati di ingresso voluta, l'uscita vale 1, 0 altrimenti.

Ogni stato di ingresso deve essere presentato prima del prossimo ciclo di clock, e per $n$ stati di ingresso avremo bisogno di $n$ cicli di clock per leggerli tutti.
Inoltre, se un valore permane per più di un ciclo di clock, si considera questa una ripetizione.

Si ha che qesta è una \textbf{rete con memoria}: deve ricordare ad ogni stato di ingresso la sequenza degli stati di ingresso \textbf{corretti} e \textbf{consecutivi} visti finora, cioè $K + 1$ stati (compreso lo stato finale con l'uscita a 1) per sequenze di $K$ stati.

# la sintesi è un esercizio per casa

\subsection{Modello di Mealy}
Nel modello di Moore avevamo detto che l'uscità è funzione soltanto dello stato interno precedente: $B : S \rightarrow Z$.
Nelle reti di mealy, la legge $B$ è più generale, e dipende anche dagli ingressi: $B : X \times S \rightarrow Z$.

Vediamo che le reti RCA e RCB, a questo punto, possono essere espresse come un unica grande rete RC con cicli di retroazione dal registro STAR.
Possiamo quindi riformuare le diseguaglianze di temporizzazione come:
\[
	\begin{cases}
		T \geq T_{hold} + T_{a\_monte} + T_{RC} + T_{setup} \\ 
		T \geq T_{prop} + T_{RC} + T_{setup} \\ 
		T \geq T_{hold} + T_{a\_monte} + T_{RC} + T_{a\_valle} \\
		T \geq T_{prop} + T_{RC} + T_{a\_valle}
	\end{cases}
\]

dove i tempi di attraversamento da ingresso a registro, da registro a registro e da registro a uscita sono sostituiti dal tempo di attraversamento di RC # come mai?

Notiamo che al variare dell'ingresso, una rete di Mealy può produrre una nuova uscita \textit{prima dell'aggiornamento del clock}.
Questo rende le reti di Mealy \textbf{non trasparenti}: gli ingressi sono connessi direttamente alle uscite (in senso logico), ergo cicli di retroazione possono creare oscillazioni incontrollate.
Prendiamo ad esempio il contatore visto precedentemente: è effettivamente una \textbf{rete di Mealy} rispetto alle uscite di riporto $eu$ e $ei$. 
L'uscita $q$, invece, che è collegata direttamente al registro, non è trasparente (la chiamiamo \textit{uscita di Moore}).
Notiamo quindi che basta un'\textit{uscita di Mealy} a rendere una rete una rete di Mealy.

\end{document}


\documentclass[a4paper,11pt]{article}
\usepackage[a4paper, margin=8em]{geometry}

% usa i pacchetti per la scrittura in italiano
\usepackage[french,italian]{babel}
\usepackage[T1]{fontenc}
\usepackage[utf8]{inputenc}
\frenchspacing 

% usa i pacchetti per la formattazione matematica
\usepackage{amsmath, amssymb, amsthm, amsfonts}

% usa altri pacchetti
\usepackage{gensymb}
\usepackage{hyperref}
\usepackage{standalone}

\usepackage{colortbl}

\usepackage{xstring}
\usepackage{karnaugh-map}

% imposta il titolo
\title{Appunti Reti Logiche}
\author{Luca Seggiani}
\date{2024}

% imposta lo stile
% usa helvetica
\usepackage[scaled]{helvet}
% usa palatino
\usepackage{palatino}
% usa un font monospazio guardabile
\usepackage{lmodern}

\renewcommand{\rmdefault}{ppl}
\renewcommand{\sfdefault}{phv}
\renewcommand{\ttdefault}{lmtt}

% circuiti
\usepackage{circuitikz}
\usetikzlibrary{babel}

% disponi il titolo
\makeatletter
\renewcommand{\maketitle} {
	\begin{center} 
		\begin{minipage}[t]{.8\textwidth}
			\textsf{\huge\bfseries \@title} 
		\end{minipage}%
		\begin{minipage}[t]{.2\textwidth}
			\raggedleft \vspace{-1.65em}
			\textsf{\small \@author} \vfill
			\textsf{\small \@date}
		\end{minipage}
		\par
	\end{center}

	\thispagestyle{empty}
	\pagestyle{fancy}
}
\makeatother

% disponi teoremi
\usepackage{tcolorbox}
\newtcolorbox[auto counter, number within=section]{theorem}[2][]{%
	colback=blue!10, 
	colframe=blue!40!black, 
	sharp corners=northwest,
	fonttitle=\sffamily\bfseries, 
	title=Teorema~\thetcbcounter: #2, 
	#1
}

% disponi definizioni
\newtcolorbox[auto counter, number within=section]{definition}[2][]{%
	colback=red!10,
	colframe=red!40!black,
	sharp corners=northwest,
	fonttitle=\sffamily\bfseries,
	title=Definizione~\thetcbcounter: #2,
	#1
}

% disponi codice
\usepackage{listings}
\usepackage[table]{xcolor}

\definecolor{codegreen}{rgb}{0,0.6,0}
\definecolor{codegray}{rgb}{0.5,0.5,0.5}
\definecolor{codepurple}{rgb}{0.58,0,0.82}
\definecolor{backcolour}{rgb}{0.95,0.95,0.92}

\lstdefinestyle{codestyle}{
		backgroundcolor=\color{black!5}, 
		commentstyle=\color{codegreen},
		keywordstyle=\bfseries\color{magenta},
		numberstyle=\sffamily\tiny\color{black!60},
		stringstyle=\color{green!50!black},
		basicstyle=\ttfamily\footnotesize,
		breakatwhitespace=false,         
		breaklines=true,                 
		captionpos=b,                    
		keepspaces=true,                 
		numbers=left,                    
		numbersep=5pt,                  
		showspaces=false,                
		showstringspaces=false,
		showtabs=false,                  
		tabsize=2
}

\lstdefinestyle{shellstyle}{
		backgroundcolor=\color{black!5}, 
		basicstyle=\ttfamily\footnotesize\color{black}, 
		commentstyle=\color{black}, 
		keywordstyle=\color{black},
		numberstyle=\color{black!5},
		stringstyle=\color{black}, 
		showspaces=false,
		showstringspaces=false, 
		showtabs=false, 
		tabsize=2, 
		numbers=none, 
		breaklines=true
}


\lstdefinelanguage{assembler}{ 
  keywords={AAA, AAD, AAM, AAS, ADC, ADCB, ADCW, ADCL, ADD, ADDB, ADDW, ADDL, AND, ANDB, ANDW, ANDL,
        ARPL, BOUND, BSF, BSFL, BSFW, BSR, BSRL, BSRW, BSWAP, BT, BTC, BTCB, BTCW, BTCL, BTR, 
        BTRB, BTRW, BTRL, BTS, BTSB, BTSW, BTSL, CALL, CBW, CDQ, CLC, CLD, CLI, CLTS, CMC, CMP,
        CMPB, CMPW, CMPL, CMPS, CMPSB, CMPSD, CMPSW, CMPXCHG, CMPXCHGB, CMPXCHGW, CMPXCHGL,
        CMPXCHG8B, CPUID, CWDE, DAA, DAS, DEC, DECB, DECW, DECL, DIV, DIVB, DIVW, DIVL, ENTER,
        HLT, IDIV, IDIVB, IDIVW, IDIVL, IMUL, IMULB, IMULW, IMULL, IN, INB, INW, INL, INC, INCB,
        INCW, INCL, INS, INSB, INSD, INSW, INT, INT3, INTO, INVD, INVLPG, IRET, IRETD, JA, JAE,
        JB, JBE, JC, JCXZ, JE, JECXZ, JG, JGE, JL, JLE, JMP, JNA, JNAE, JNB, JNBE, JNC, JNE, JNG,
        JNGE, JNL, JNLE, JNO, JNP, JNS, JNZ, JO, JP, JPE, JPO, JS, JZ, LAHF, LAR, LCALL, LDS,
        LEA, LEAVE, LES, LFS, LGDT, LGS, LIDT, LMSW, LOCK, LODSB, LODSD, LODSW, LOOP, LOOPE,
        LOOPNE, LSL, LSS, LTR, MOV, MOVB, MOVW, MOVL, MOVSB, MOVSD, MOVSW, MOVSX, MOVSXB,
        MOVSXW, MOVSXL, MOVZX, MOVZXB, MOVZXW, MOVZXL, MUL, MULB, MULW, MULL, NEG, NEGB, NEGW,
        NEGL, NOP, NOT, NOTB, NOTW, NOTL, OR, ORB, ORW, ORL, OUT, OUTB, OUTW, OUTL, OUTSB, OUTSD,
        OUTSW, POP, POPL, POPW, POPB, POPA, POPAD, POPF, POPFD, PUSH, PUSHL, PUSHW, PUSHB, PUSHA, 
				PUSHAD, PUSHF, PUSHFD, RCL, RCLB, RCLW, MOVSL, MOVSB, MOVSW, STOSL, STOSB, STOSW, LODSB, LODSW,
				LODSL, INSB, INSW, INSL, OUTSB, OUTSL, OUTSW
        RCLL, RCR, RCRB, RCRW, RCRL, RDMSR, RDPMC, RDTSC, REP, REPE, REPNE, RET, ROL, ROLB, ROLW,
        ROLL, ROR, RORB, RORW, RORL, SAHF, SAL, SALB, SALW, SALL, SAR, SARB, SARW, SARL, SBB,
        SBBB, SBBW, SBBL, SCASB, SCASD, SCASW, SETA, SETAE, SETB, SETBE, SETC, SETE, SETG, SETGE,
        SETL, SETLE, SETNA, SETNAE, SETNB, SETNBE, SETNC, SETNE, SETNG, SETNGE, SETNL, SETNLE,
        SETNO, SETNP, SETNS, SETNZ, SETO, SETP, SETPE, SETPO, SETS, SETZ, SGDT, SHL, SHLB, SHLW,
        SHLL, SHLD, SHR, SHRB, SHRW, SHRL, SHRD, SIDT, SLDT, SMSW, STC, STD, STI, STOSB, STOSD,
        STOSW, STR, SUB, SUBB, SUBW, SUBL, TEST, TESTB, TESTW, TESTL, VERR, VERW, WAIT, WBINVD,
        XADD, XADDB, XADDW, XADDL, XCHG, XCHGB, XCHGW, XCHGL, XLAT, XLATB, XOR, XORB, XORW, XORL},
  keywordstyle=\color{blue}\bfseries,
  ndkeywordstyle=\color{darkgray}\bfseries,
  identifierstyle=\color{black},
  sensitive=false,
  comment=[l]{\#},
  morecomment=[s]{/*}{*/},
  commentstyle=\color{purple}\ttfamily,
  stringstyle=\color{red}\ttfamily,
  morestring=[b]',
  morestring=[b]"
}

\lstset{language=assembler, style=codestyle}

% disponi sezioni
\usepackage{titlesec}

\titleformat{\section}
	{\sffamily\Large\bfseries} 
	{\thesection}{1em}{} 
\titleformat{\subsection}
	{\sffamily\large\bfseries}   
	{\thesubsection}{1em}{} 
\titleformat{\subsubsection}
	{\sffamily\normalsize\bfseries} 
	{\thesubsubsection}{1em}{}

% tikz
\usepackage{tikz}

% float
\usepackage{float}

% grafici
\usepackage{pgfplots}
\pgfplotsset{width=10cm,compat=1.9}

% disponi alberi
\usepackage{forest}

\forestset{
	rectstyle/.style={
		for tree={rectangle,draw,font=\large\sffamily}
	},
	roundstyle/.style={
		for tree={circle,draw,font=\large}
	}
}

% disponi algoritmi
\usepackage{algorithm}
\usepackage{algorithmic}
\makeatletter
\renewcommand{\ALG@name}{Algoritmo}
\makeatother

% disponi numeri di pagina
\usepackage{fancyhdr}
\fancyhf{} 
\fancyfoot[L]{\sffamily{\thepage}}

\makeatletter
\fancyhead[L]{\raisebox{1ex}[0pt][0pt]{\sffamily{\@title \ \@date}}} 
\fancyhead[R]{\raisebox{1ex}[0pt][0pt]{\sffamily{\@author}}}
\makeatother

\begin{document}
% sezione (data)
\section{Lezione del 14-11-24}

% stili pagina
\thispagestyle{empty}
\pagestyle{fancy}

% testo
\subsection{Riconoscitore di sequenze}
Vediamo come sintetizzare un circuito \textbf{riconoscitore di sequenze}, sia come rete di Moore che come rete di Mealy:

\subsubsection{Sintesi in rete di Moore}
# fallo

\subsubsection{Sintesi in rete di Mealy}
# fallo

\subsection{Confronto fra Moore e Mealy}
Abbiamo che le reti di Moore hanno leggi $B$ meno flessibili delle reti di Mealy, e quindi in una rete di Mealy si hanno meno stati interni che in una rete di Moore.
Si potrebbe quindi pensare che una rete di Mealy può esprimere funzioni che una rete di Moore non può rappresentare.
Questo è falso, in quanto si può dimostrare che Moore e Mealy hanno la \textbf{stessa potenza descrittiva}: per una rete di Moore, si può ricavare l'equivalente di Mealy, e viceversa.

Tra le altre differenze che possiamo notare, si ha che il clock di una rete di Moore al pari di una rete di Mealy deve essere più veloce, e sopratutto che una rete di Mealy si aggiorna \textit{al pari} con gli ingressi, cioè è una rete \textbf{trasparente}.

Un anello di retroazione fra due reti di Mealy può infatti creare un \textbf{anello combinatorio}, che sappiamo essere suscettibile a oscillazioni incontrollate.
Di contro, fra due reti di Moore incontreremo sempre un registro, ergo non avrmo problemi di formazione di anelli combinatori.

\subsection{Modello di Mealy ritardato}
Creiamo una cosiddetta rete di \textbf{Mealy ritardato} prendendo una rete di Mealy e introducendo un ulteriore registro, \textbf{OUTR}, in uscita.
Le uscite, come nelle reti di Moore, non sono più trasparenti e variano all'arrivo del clock dopo un tempo $T_{prop}$.

# rimetti in ordine da Mealy

\subsubsection{Temporizzazione del modello di Mealy ritardato}
# riporta

\subsection{Assegnamenti procedurali}
# leggi max chiedi foto simone

In Verilog possiamo descrivere il comportamento di una rete di Mealy attraverso i cosiddetti \textbf{assegnamenti procedurali}.
Notiamo la temporizzazione di una forma del tipo:
\begin{lstlisting}[language=verilog, style=codestyle]	
s0: begin STAR<=S1; OUTR<=STAR; end
\end{lstlisting}
Le istruzioni contenute nel blocco \lstinline|begin| (...) \lstinline|end| accadono \textbf{contemporaneamente}, e il fatto che STAR sia a sinistra nel primo assegnamento e OUTR sia sinistra nel secondo indica che questi avvengono \textit{dopo il clock}, cioè \textit{prima del clock} si legge il valore (S1 o STAR nell'esempio) e soltanto dopo si scrive effettivamente sul registro.

# finisci slide


\subsection{Reti sequenziali complesse}
I modelli concettuali che abbiamo visto finora (Moore, Mealy e Mealy ritardato) riescono a sintetizzare solo reti molto semplici.
Prendiamo ad esempio il modello di Mealy ritardato.

Vogliamo creare una rete che conta, modulo 16, il numero di sequenze corrette 00, 01, 10 ricevute in ingresso.
Quindi, ogni volta che viene registrata una sequenza corretta, la rete incrementa di 1 l'uscita, rappresentata su 4 bit.
Abbiamo quindi \textbf{2 ingressi}, \textbf{4 uscite}, e 3 stati interni per 16 stati di OUTR, cioè 48 stati interni totali.

Modellizzare questa rete con un tale numero di ingressi risulta chiaramente molto laborioso: un apporccio migliore sarebbe creare una rete di Mealy ritardato che riconosce una sola sequenza, e mandarla in input a un contatore a 4 bit.
# potresti fare disegnini però è laborioso pure quello

\textit{Esplodendo} una rete siffatta troviamo un modello formato da una rete di Mealy ritardato (cioè da una rete combinatoria \textit{di riconoscimento} nel caso del riconoscitore di sequenza, annessa ai registri STAR e OUTR), e da un contatore (formato a sua volta da una rete combinatoria che implementa la logica del contatore e un altro registro, che chiameremo \textbf{COUNT}).
L'uscita di questa rete sarà formata dal registro COUNT, e potremo inoltre racchiudere le due reti combinatorie in un unica RC totale.

La rete così ottenuta non rispetta il modello di Mealy ritardato: ha più di due registri, e sopratutto fa rientrare nella RC totale più di un registro uscente (finora era stato STAR).
Troviamo che questo è molto comodo: introducendo \textbf{registri operativi} abbiamo a disposizione locazioni di memoria che supportano sia uscite che computazioni intermedie.

Possiamo quindi distinguere i registri in due categorie:
\begin{itemize}
	\item \textbf{Registri di stato:} simili a quelli che abbiamo già visto, cioò che rappresentano lo stato interno della rete;
	\item \textbf{Registri operativi:} che contengono sia \textit{valori di uscita} che \textit{valori intermedi} (o \textit{computazioni intermedie}, insomma risultati utili al ricavo dell'uscita della rete). 
\end{itemize}

\end{document}


\documentclass[a4paper,11pt]{article}
\usepackage[a4paper, margin=8em]{geometry}

% usa i pacchetti per la scrittura in italiano
\usepackage[french,italian]{babel}
\usepackage[T1]{fontenc}
\usepackage[utf8]{inputenc}
\frenchspacing 

% usa i pacchetti per la formattazione matematica
\usepackage{amsmath, amssymb, amsthm, amsfonts}

% usa altri pacchetti
\usepackage{gensymb}
\usepackage{hyperref}
\usepackage{standalone}

\usepackage{colortbl}

\usepackage{xstring}
\usepackage{karnaugh-map}

% imposta il titolo
\title{Appunti Reti Logiche}
\author{Luca Seggiani}
\date{2024}

% imposta lo stile
% usa helvetica
\usepackage[scaled]{helvet}
% usa palatino
\usepackage{palatino}
% usa un font monospazio guardabile
\usepackage{lmodern}

\renewcommand{\rmdefault}{ppl}
\renewcommand{\sfdefault}{phv}
\renewcommand{\ttdefault}{lmtt}

% circuiti
\usepackage{circuitikz}
\usetikzlibrary{babel}

% testo cerchiato
\newcommand*\circled[1]{\tikz[baseline=(char.base)]{
            \node[shape=circle,draw,inner sep=2pt] (char) {#1};}}

% disponi il titolo
\makeatletter
\renewcommand{\maketitle} {
	\begin{center} 
		\begin{minipage}[t]{.8\textwidth}
			\textsf{\huge\bfseries \@title} 
		\end{minipage}%
		\begin{minipage}[t]{.2\textwidth}
			\raggedleft \vspace{-1.65em}
			\textsf{\small \@author} \vfill
			\textsf{\small \@date}
		\end{minipage}
		\par
	\end{center}

	\thispagestyle{empty}
	\pagestyle{fancy}
}
\makeatother

% disponi teoremi
\usepackage{tcolorbox}
\newtcolorbox[auto counter, number within=section]{theorem}[2][]{%
	colback=blue!10, 
	colframe=blue!40!black, 
	sharp corners=northwest,
	fonttitle=\sffamily\bfseries, 
	title=Teorema~\thetcbcounter: #2, 
	#1
}

% disponi definizioni
\newtcolorbox[auto counter, number within=section]{definition}[2][]{%
	colback=red!10,
	colframe=red!40!black,
	sharp corners=northwest,
	fonttitle=\sffamily\bfseries,
	title=Definizione~\thetcbcounter: #2,
	#1
}

% disponi codice
\usepackage{listings}
\usepackage[table]{xcolor}

\definecolor{codegreen}{rgb}{0,0.6,0}
\definecolor{codegray}{rgb}{0.5,0.5,0.5}
\definecolor{codepurple}{rgb}{0.58,0,0.82}
\definecolor{backcolour}{rgb}{0.95,0.95,0.92}

\lstdefinestyle{codestyle}{
		backgroundcolor=\color{black!5}, 
		commentstyle=\color{codegreen},
		keywordstyle=\bfseries\color{magenta},
		numberstyle=\sffamily\tiny\color{black!60},
		stringstyle=\color{green!50!black},
		basicstyle=\ttfamily\footnotesize,
		breakatwhitespace=false,         
		breaklines=true,                 
		captionpos=b,                    
		keepspaces=true,                 
		numbers=left,                    
		numbersep=5pt,                  
		showspaces=false,                
		showstringspaces=false,
		showtabs=false,                  
		tabsize=2
}

\lstdefinestyle{shellstyle}{
		backgroundcolor=\color{black!5}, 
		basicstyle=\ttfamily\footnotesize\color{black}, 
		commentstyle=\color{black}, 
		keywordstyle=\color{black},
		numberstyle=\color{black!5},
		stringstyle=\color{black}, 
		showspaces=false,
		showstringspaces=false, 
		showtabs=false, 
		tabsize=2, 
		numbers=none, 
		breaklines=true
}


\lstdefinelanguage{assembler}{ 
  keywords={AAA, AAD, AAM, AAS, ADC, ADCB, ADCW, ADCL, ADD, ADDB, ADDW, ADDL, AND, ANDB, ANDW, ANDL,
        ARPL, BOUND, BSF, BSFL, BSFW, BSR, BSRL, BSRW, BSWAP, BT, BTC, BTCB, BTCW, BTCL, BTR, 
        BTRB, BTRW, BTRL, BTS, BTSB, BTSW, BTSL, CALL, CBW, CDQ, CLC, CLD, CLI, CLTS, CMC, CMP,
        CMPB, CMPW, CMPL, CMPS, CMPSB, CMPSD, CMPSW, CMPXCHG, CMPXCHGB, CMPXCHGW, CMPXCHGL,
        CMPXCHG8B, CPUID, CWDE, DAA, DAS, DEC, DECB, DECW, DECL, DIV, DIVB, DIVW, DIVL, ENTER,
        HLT, IDIV, IDIVB, IDIVW, IDIVL, IMUL, IMULB, IMULW, IMULL, IN, INB, INW, INL, INC, INCB,
        INCW, INCL, INS, INSB, INSD, INSW, INT, INT3, INTO, INVD, INVLPG, IRET, IRETD, JA, JAE,
        JB, JBE, JC, JCXZ, JE, JECXZ, JG, JGE, JL, JLE, JMP, JNA, JNAE, JNB, JNBE, JNC, JNE, JNG,
        JNGE, JNL, JNLE, JNO, JNP, JNS, JNZ, JO, JP, JPE, JPO, JS, JZ, LAHF, LAR, LCALL, LDS,
        LEA, LEAVE, LES, LFS, LGDT, LGS, LIDT, LMSW, LOCK, LODSB, LODSD, LODSW, LOOP, LOOPE,
        LOOPNE, LSL, LSS, LTR, MOV, MOVB, MOVW, MOVL, MOVSB, MOVSD, MOVSW, MOVSX, MOVSXB,
        MOVSXW, MOVSXL, MOVZX, MOVZXB, MOVZXW, MOVZXL, MUL, MULB, MULW, MULL, NEG, NEGB, NEGW,
        NEGL, NOP, NOT, NOTB, NOTW, NOTL, OR, ORB, ORW, ORL, OUT, OUTB, OUTW, OUTL, OUTSB, OUTSD,
        OUTSW, POP, POPL, POPW, POPB, POPA, POPAD, POPF, POPFD, PUSH, PUSHL, PUSHW, PUSHB, PUSHA, 
				PUSHAD, PUSHF, PUSHFD, RCL, RCLB, RCLW, MOVSL, MOVSB, MOVSW, STOSL, STOSB, STOSW, LODSB, LODSW,
				LODSL, INSB, INSW, INSL, OUTSB, OUTSL, OUTSW
        RCLL, RCR, RCRB, RCRW, RCRL, RDMSR, RDPMC, RDTSC, REP, REPE, REPNE, RET, ROL, ROLB, ROLW,
        ROLL, ROR, RORB, RORW, RORL, SAHF, SAL, SALB, SALW, SALL, SAR, SARB, SARW, SARL, SBB,
        SBBB, SBBW, SBBL, SCASB, SCASD, SCASW, SETA, SETAE, SETB, SETBE, SETC, SETE, SETG, SETGE,
        SETL, SETLE, SETNA, SETNAE, SETNB, SETNBE, SETNC, SETNE, SETNG, SETNGE, SETNL, SETNLE,
        SETNO, SETNP, SETNS, SETNZ, SETO, SETP, SETPE, SETPO, SETS, SETZ, SGDT, SHL, SHLB, SHLW,
        SHLL, SHLD, SHR, SHRB, SHRW, SHRL, SHRD, SIDT, SLDT, SMSW, STC, STD, STI, STOSB, STOSD,
        STOSW, STR, SUB, SUBB, SUBW, SUBL, TEST, TESTB, TESTW, TESTL, VERR, VERW, WAIT, WBINVD,
        XADD, XADDB, XADDW, XADDL, XCHG, XCHGB, XCHGW, XCHGL, XLAT, XLATB, XOR, XORB, XORW, XORL},
  keywordstyle=\color{blue}\bfseries,
  ndkeywordstyle=\color{darkgray}\bfseries,
  identifierstyle=\color{black},
  sensitive=false,
  comment=[l]{\#},
  morecomment=[s]{/*}{*/},
  commentstyle=\color{purple}\ttfamily,
  stringstyle=\color{red}\ttfamily,
  morestring=[b]',
  morestring=[b]"
}

\lstset{language=assembler, style=codestyle}

% disponi sezioni
\usepackage{titlesec}

\titleformat{\section}
	{\sffamily\Large\bfseries} 
	{\thesection}{1em}{} 
\titleformat{\subsection}
	{\sffamily\large\bfseries}   
	{\thesubsection}{1em}{} 
\titleformat{\subsubsection}
	{\sffamily\normalsize\bfseries} 
	{\thesubsubsection}{1em}{}

% tikz
\usepackage{tikz}

% float
\usepackage{float}

% grafici
\usepackage{pgfplots}
\pgfplotsset{width=10cm,compat=1.9}

% disponi alberi
\usepackage{forest}

\forestset{
	rectstyle/.style={
		for tree={rectangle,draw,font=\large\sffamily}
	},
	roundstyle/.style={
		for tree={circle,draw,font=\large}
	}
}

% disponi algoritmi
\usepackage{algorithm}
\usepackage{algorithmic}
\makeatletter
\renewcommand{\ALG@name}{Algoritmo}
\makeatother

% disponi numeri di pagina
\usepackage{fancyhdr}
\fancyhf{} 
\fancyfoot[L]{\sffamily{\thepage}}

\makeatletter
\fancyhead[L]{\raisebox{1ex}[0pt][0pt]{\sffamily{\@title \ \@date}}} 
\fancyhead[R]{\raisebox{1ex}[0pt][0pt]{\sffamily{\@author}}}
\makeatother

\begin{document}
% sezione (data)
\section{Lezione del 19-11-24}

% stili pagina
\thispagestyle{empty}
\pagestyle{fancy}

% testo
\subsubsection{Microprogrammazione}
Avevamo visto il concetto basilare di \textbf{rete sequenziale sincronizzata complessa}.
La sintesi di reti di questo tipo prende il nome di \textbf{microprogrammazione}. 
Bisogna notare che la parola \textit{programmazione} qui è piuttosto fuorviante: l'idea è comunque quella di dare \textbf{descrizioni} di hardware.

Il Verilog comprende un sottoinsieme di linguaggio adibito esattamente agli scopi della microprogrammazione, cioè un \textbf{linguaggio di trasmissione tra registri}.
In ogni statement includiamo:
\begin{itemize}
	\item $\mu$-istruzioni, cioè assegnamenti a registri operativi;
	\item $\mu$-salti, cioè assegnament al registro STAR, che possono essere a una o più vie.
\end{itemize}
Notiamo che qui il $\mu$ significa semplicemente "hardware".
Si possono ommettere le $\mu$-istruzioni relative a \textit{registri operativi}, che in questo caso implicano conservazione.
Per quanto riguarda il registro STAR, invece, l'omissione del $\mu$-salto implicherebbe un salto incondizionato (altrimenti avremmo \lstinline|STAR<=STAR|, che porterebbe a un \textbf{deadlock}).
Nella pratica, non ometteremo mai l'aggiornamento di STAR, in quanto porta facilmente a errori, o comunque a codice poco chiaro.

\subsubsection{Temporizzazione di reti complesse}
Le temporizzazioni di una rete complessa sono le stesse delle rety di Mealy ritardato: i percorsi sono gli stessi ($T_{in\_to\_reg}$, ecc...) e preleviamo le uscite direttamente dai registri.

\subsection{Handshake e temporizzazione delle uscite}
Solitamente le reti sequenziali sincronizzate comunicano con altre reti sequenziali sincronizzate, idealmente con cicli di clock mutualmente sincronizzati (così da doverci preoccupare solo dei tempi di lettura e scrittura $T_{a\_monte}$ e $T_{a\_valle}$).
Nel caso le reti che consideriamo siano invece mutuamente asincrone fra di loro, cioè abbiano clock discordi che non si allineano mai (necessariamente), dovremmo adottare soluzioni diverse. 

\par\smallskip

Poniamo una situazione di esempio: una rete, detta \textbf{produttore}, mette su una linea un numero rappresentato su 8 bit.
Chiamiamo questa linea \textit{linea di trasmissione}.
Un altra rete, detta \textbf{consumatore}, riceve questo numero e tiene la sua uscita alta per il numero di cicli indicato dal numero ricevuto.
Visto che le due reti vedono solamente i rispettivi input e output, come facciamo in modo che il produttore sappia quando il consumatore ha letto il numero con successo, e viceversa che il consumatore sappia quando c'è un nuovo numero da leggere?
Il problema si risolve introducendo \textbf{linee di handshake} (dall'inglese per \textit{stretta di mano}).

Doteremo quindi il produttore di una linea di uscita \lstinline|/dav| (\textit{data valid}), e il consumatore di una linea di uscita \lstinline|/rfd| (\textit{ready for data}).
La linea \lstinline|/dav| segnala che ci sono nuovi dati sulla linea di uscita del produttore, mentre la linea \lstinline|/rdf| segnala che il consumatore ha letto ciò che era sulla linea di trasmissione ed è pronto a ricevere nuovi dati.
Entrambe le variabili sono attive basse.

Facciamo una nota sulla circuiteria di \textbf{reset}: la linea \lstinline|/reset| sarà infatti presente e comune alle due reti.
A \lstinline|/reset| bassa, quindi, possiamo mettere \lstinline|/dav| e \lstinline|/rfd| a 1 (per segnalare che linea di trasmissione non è pronta e il consumatore non è pronto a leggerla).

In seguito, vorremo effettuare una trasmissione vera e propria di dati.
Innanzitutto, il produttore metterà un numero sulla linea di trasmissione.
In seguito, metterà \lstinline|/dav| bassa per segnalare che i dati sulla linea di trasmissione sono pronti.
A questo punto, il consumatore dovrà rilevare il \lstinline|/dav|, leggere i dati sulla linea di trasmissione e mettere il suo \lstinline|/rfd| basso.
Questo segnalerà, per la rete produttore, che il consumatore \textbf{ha letto}
 i dati con successo ed è pronto ad un nuovo ciclo di trasmissione.
Da qui in in poi, il consumatore non potrà più aspettarsi che i dati sulla linea di trasmissione siano validi: in qualsiasi momento il produttore potrebbe aggiornarli e rialzare \lstinline|/dav| (o viceversa, rialzare \lstinline|/dav| e poi scrivere sulla linea, ciò che importa è che il consumatore non ha più nulla da leggere fino a un nuovo ciclo di abbassamento del \lstinline|/dav|).
Quando \lstinline|/dav| si alza, quindi, anche il consumatore riporterà il suo \lstinline|/rfd| alto, e ci troveremo nella situazione di partenza (cioè \lstinline|/dav| e \lstinline|/rfd| alti).
Dobbiamo stare attenti al fatto che \lstinline|/rfd| torna alto \textbf{dopo} che lo fa \lstinline|/dav|: altrimenti il produttore potrebbe perdersi la doppia transizione di \lstinline|/rfd|, e finiremmo in uno stato di deadlock.
Possiamo riassumere quest'ultima affermazione come segue: una corretta sincronizzazione delle reti avviene \textbf{solamente} se si \textbf{alternano} le transizioni delle linee di handshake (a ogni aggiornamento del produttore segue un aggiornamento del consumatore, e così via, senza che nessuno faccia doppi aggiornamenti "di testa propria").

Veniamo quindi all'implementazione pratica della rete consumatore.
Vogliamo che questa mantenga un uscita bassa per un numero di cicli fornito sulla linea di trasmissione.
Questo si realizza con un \textbf{contatore}.

\end{document}


\documentclass[a4paper,11pt]{article}
\usepackage[a4paper, margin=8em]{geometry}

% usa i pacchetti per la scrittura in italiano
\usepackage[french,italian]{babel}
\usepackage[T1]{fontenc}
\usepackage[utf8]{inputenc}
\frenchspacing 

% usa i pacchetti per la formattazione matematica
\usepackage{amsmath, amssymb, amsthm, amsfonts}

% usa altri pacchetti
\usepackage{gensymb}
\usepackage{hyperref}
\usepackage{standalone}

\usepackage{colortbl}

\usepackage{xstring}
\usepackage{karnaugh-map}

% imposta il titolo
\title{Appunti Reti Logiche}
\author{Luca Seggiani}
\date{2024}

% imposta lo stile
% usa helvetica
\usepackage[scaled]{helvet}
% usa palatino
\usepackage{palatino}
% usa un font monospazio guardabile
\usepackage{lmodern}

\renewcommand{\rmdefault}{ppl}
\renewcommand{\sfdefault}{phv}
\renewcommand{\ttdefault}{lmtt}

% circuiti
\usepackage{circuitikz}
\usetikzlibrary{babel}

% testo cerchiato
\newcommand*\circled[1]{\tikz[baseline=(char.base)]{
            \node[shape=circle,draw,inner sep=2pt] (char) {#1};}}

% disponi il titolo
\makeatletter
\renewcommand{\maketitle} {
	\begin{center} 
		\begin{minipage}[t]{.8\textwidth}
			\textsf{\huge\bfseries \@title} 
		\end{minipage}%
		\begin{minipage}[t]{.2\textwidth}
			\raggedleft \vspace{-1.65em}
			\textsf{\small \@author} \vfill
			\textsf{\small \@date}
		\end{minipage}
		\par
	\end{center}

	\thispagestyle{empty}
	\pagestyle{fancy}
}
\makeatother

% disponi teoremi
\usepackage{tcolorbox}
\newtcolorbox[auto counter, number within=section]{theorem}[2][]{%
	colback=blue!10, 
	colframe=blue!40!black, 
	sharp corners=northwest,
	fonttitle=\sffamily\bfseries, 
	title=Teorema~\thetcbcounter: #2, 
	#1
}

% disponi definizioni
\newtcolorbox[auto counter, number within=section]{definition}[2][]{%
	colback=red!10,
	colframe=red!40!black,
	sharp corners=northwest,
	fonttitle=\sffamily\bfseries,
	title=Definizione~\thetcbcounter: #2,
	#1
}

% disponi codice
\usepackage{listings}
\usepackage[table]{xcolor}

\definecolor{codegreen}{rgb}{0,0.6,0}
\definecolor{codegray}{rgb}{0.5,0.5,0.5}
\definecolor{codepurple}{rgb}{0.58,0,0.82}
\definecolor{backcolour}{rgb}{0.95,0.95,0.92}

\lstdefinestyle{codestyle}{
		backgroundcolor=\color{black!5}, 
		commentstyle=\color{codegreen},
		keywordstyle=\bfseries\color{magenta},
		numberstyle=\sffamily\tiny\color{black!60},
		stringstyle=\color{green!50!black},
		basicstyle=\ttfamily\footnotesize,
		breakatwhitespace=false,         
		breaklines=true,                 
		captionpos=b,                    
		keepspaces=true,                 
		numbers=left,                    
		numbersep=5pt,                  
		showspaces=false,                
		showstringspaces=false,
		showtabs=false,                  
		tabsize=2
}

\lstdefinestyle{shellstyle}{
		backgroundcolor=\color{black!5}, 
		basicstyle=\ttfamily\footnotesize\color{black}, 
		commentstyle=\color{black}, 
		keywordstyle=\color{black},
		numberstyle=\color{black!5},
		stringstyle=\color{black}, 
		showspaces=false,
		showstringspaces=false, 
		showtabs=false, 
		tabsize=2, 
		numbers=none, 
		breaklines=true
}


\lstdefinelanguage{assembler}{ 
  keywords={AAA, AAD, AAM, AAS, ADC, ADCB, ADCW, ADCL, ADD, ADDB, ADDW, ADDL, AND, ANDB, ANDW, ANDL,
        ARPL, BOUND, BSF, BSFL, BSFW, BSR, BSRL, BSRW, BSWAP, BT, BTC, BTCB, BTCW, BTCL, BTR, 
        BTRB, BTRW, BTRL, BTS, BTSB, BTSW, BTSL, CALL, CBW, CDQ, CLC, CLD, CLI, CLTS, CMC, CMP,
        CMPB, CMPW, CMPL, CMPS, CMPSB, CMPSD, CMPSW, CMPXCHG, CMPXCHGB, CMPXCHGW, CMPXCHGL,
        CMPXCHG8B, CPUID, CWDE, DAA, DAS, DEC, DECB, DECW, DECL, DIV, DIVB, DIVW, DIVL, ENTER,
        HLT, IDIV, IDIVB, IDIVW, IDIVL, IMUL, IMULB, IMULW, IMULL, IN, INB, INW, INL, INC, INCB,
        INCW, INCL, INS, INSB, INSD, INSW, INT, INT3, INTO, INVD, INVLPG, IRET, IRETD, JA, JAE,
        JB, JBE, JC, JCXZ, JE, JECXZ, JG, JGE, JL, JLE, JMP, JNA, JNAE, JNB, JNBE, JNC, JNE, JNG,
        JNGE, JNL, JNLE, JNO, JNP, JNS, JNZ, JO, JP, JPE, JPO, JS, JZ, LAHF, LAR, LCALL, LDS,
        LEA, LEAVE, LES, LFS, LGDT, LGS, LIDT, LMSW, LOCK, LODSB, LODSD, LODSW, LOOP, LOOPE,
        LOOPNE, LSL, LSS, LTR, MOV, MOVB, MOVW, MOVL, MOVSB, MOVSD, MOVSW, MOVSX, MOVSXB,
        MOVSXW, MOVSXL, MOVZX, MOVZXB, MOVZXW, MOVZXL, MUL, MULB, MULW, MULL, NEG, NEGB, NEGW,
        NEGL, NOP, NOT, NOTB, NOTW, NOTL, OR, ORB, ORW, ORL, OUT, OUTB, OUTW, OUTL, OUTSB, OUTSD,
        OUTSW, POP, POPL, POPW, POPB, POPA, POPAD, POPF, POPFD, PUSH, PUSHL, PUSHW, PUSHB, PUSHA, 
				PUSHAD, PUSHF, PUSHFD, RCL, RCLB, RCLW, MOVSL, MOVSB, MOVSW, STOSL, STOSB, STOSW, LODSB, LODSW,
				LODSL, INSB, INSW, INSL, OUTSB, OUTSL, OUTSW
        RCLL, RCR, RCRB, RCRW, RCRL, RDMSR, RDPMC, RDTSC, REP, REPE, REPNE, RET, ROL, ROLB, ROLW,
        ROLL, ROR, RORB, RORW, RORL, SAHF, SAL, SALB, SALW, SALL, SAR, SARB, SARW, SARL, SBB,
        SBBB, SBBW, SBBL, SCASB, SCASD, SCASW, SETA, SETAE, SETB, SETBE, SETC, SETE, SETG, SETGE,
        SETL, SETLE, SETNA, SETNAE, SETNB, SETNBE, SETNC, SETNE, SETNG, SETNGE, SETNL, SETNLE,
        SETNO, SETNP, SETNS, SETNZ, SETO, SETP, SETPE, SETPO, SETS, SETZ, SGDT, SHL, SHLB, SHLW,
        SHLL, SHLD, SHR, SHRB, SHRW, SHRL, SHRD, SIDT, SLDT, SMSW, STC, STD, STI, STOSB, STOSD,
        STOSW, STR, SUB, SUBB, SUBW, SUBL, TEST, TESTB, TESTW, TESTL, VERR, VERW, WAIT, WBINVD,
        XADD, XADDB, XADDW, XADDL, XCHG, XCHGB, XCHGW, XCHGL, XLAT, XLATB, XOR, XORB, XORW, XORL},
  keywordstyle=\color{blue}\bfseries,
  ndkeywordstyle=\color{darkgray}\bfseries,
  identifierstyle=\color{black},
  sensitive=false,
  comment=[l]{\#},
  morecomment=[s]{/*}{*/},
  commentstyle=\color{purple}\ttfamily,
  stringstyle=\color{red}\ttfamily,
  morestring=[b]',
  morestring=[b]"
}

\lstset{language=assembler, style=codestyle}

% disponi sezioni
\usepackage{titlesec}

\titleformat{\section}
	{\sffamily\Large\bfseries} 
	{\thesection}{1em}{} 
\titleformat{\subsection}
	{\sffamily\large\bfseries}   
	{\thesubsection}{1em}{} 
\titleformat{\subsubsection}
	{\sffamily\normalsize\bfseries} 
	{\thesubsubsection}{1em}{}

% tikz
\usepackage{tikz}

% float
\usepackage{float}

% grafici
\usepackage{pgfplots}
\pgfplotsset{width=10cm,compat=1.9}

% disponi alberi
\usepackage{forest}

\forestset{
	rectstyle/.style={
		for tree={rectangle,draw,font=\large\sffamily}
	},
	roundstyle/.style={
		for tree={circle,draw,font=\large}
	}
}

% disponi algoritmi
\usepackage{algorithm}
\usepackage{algorithmic}
\makeatletter
\renewcommand{\ALG@name}{Algoritmo}
\makeatother

% disponi numeri di pagina
\usepackage{fancyhdr}
\fancyhf{} 
\fancyfoot[L]{\sffamily{\thepage}}

\makeatletter
\fancyhead[L]{\raisebox{1ex}[0pt][0pt]{\sffamily{\@title \ \@date}}} 
\fancyhead[R]{\raisebox{1ex}[0pt][0pt]{\sffamily{\@author}}}
\makeatother

\begin{document}
% sezione (data)
\section{Lezione del 21-11-24}

% stili pagina
\thispagestyle{empty}
\pagestyle{fancy}

% testo
\subsection{Handhake soc-eoc}
Abbiamo visto l'handshake dav-rfd.
Vediamo adesso un altro tipo di handshake, detto \textbf{handshake soc-eoc} (\textit{Start Of Computation, End Of Computation}).
Nell'handshake soc-eoc, è il consumatore a fare la prima mossa, cioè a segnalare al produttore che bisogna di un nuovo dato.

\par\smallskip 
La situazione di riposo è quella dove \lstinline|soc| è 0 e \lstinline|eoc| è 1.
In questo caso, l'ultimo dato è già sulla linea di trasmissione, quindi possiamo leggerlo liberamente.
Quando il consumatore richiede un nuovo dato, mette \lstinline|soc| a 1.
A questo punto il produttore metterà \lstinline|eoc| a 0: questo signnifica che la computazione è in corso e non possiamo leggere dalla linea di trasmissione.
Una volta rilevato \lstinline|eoc| a 0, il consumatore dovrà rimettere \lstinline|soc| a 0, per rispettare l'alternanza dei segnali di handshake.
Al rialzarsi di \lstinline|eoc|, quindi, il nuovo dato sarà pronto e potremo leggerlo dalla linea di trasmissione, \lstinline|soc| sarà a 0 da prima e ci troveremo quindi nuovamente nella situazione di riposo.

Viste su un grafico, le variabili \lstinline|soc| e \lstinline|eoc| assumono, durante un ciclo di handshake valido, i seguenti valori: 

\begin{center}
	\begin{tikzpicture}
		\begin{axis}[
				xmin=4, xmax=9,
				ymin=-1, ymax=18,
				grid=major,
				domain=4:9,
				xtick={},
				ytick={},
				xticklabels={},
				yticklabels={},
				samples=100,
				legend pos=north west, % Position of the legend
				width=15cm,
				height=7cm
		]
		% Blue plot with legend entry
		\addplot[blue, thick] {7 + 5 * (x >= 5) * (x <=  7)}; 
		\addlegendentry{soc} % Legend entry for the blue plot
		
		% Red plot with legend entry
		\addplot[red, thick] {5 - 5 * (x >= 6) * (x <=  8)};
		\addlegendentry{eoc} % Legend entry for the red plot
		\end{axis}
	\end{tikzpicture}
\end{center}

\subsection{Sintesi di RSS complesse}
Finora abbiamo visto descrizioni scritte attraverso il \textbf{linguaggio di trasferimento fra registri} del Verilog.
Adesso dobbiamo vedere di come \textit{sintetizzare} effettivamente una rete complessa, attraverso \textbf{circuiti elementari} noti.
Notiamo che non parleremo di \textit{sintesi ottime}, in quanto questo non è un problema che adesso ci poniamo.

Vediamo quindi come operiamo nella pratica: dovremo operare scomposizione in \textbf{parte operativa} e \textbf{parte di controllo}:

\begin{itemize}
	\item \textbf{Parte operativa:} contiene la logica necessaria all'interfacciamento col mondo estero e alla produzione di stati di ingresso per i \textbf{registri operativi};
	\item \textbf{Parte di controllo:} si occupa di mantenere aggiornato lo stato interno.
\end{itemize}
Queste due reti hanno lo stesso clock, e comunicano fra di loro attraverso due gruppi di variabili: quelle di \textbf{comando} (PC $\rightarrow$ PO), e quelle di \textbf{condizionamento} (PO $\rightarrow$ PC).

Per realizzare la \textbf{parte operativa} dobbiamo pensare ai registri operativi come \textbf{registri multifunzionali}.
Per ogni registro isoliamo le $\mu$-operazioni diverse, una per ogni stato, che dovremo effettuare in modo da ricavare il suo ingresso, e le inseriamo in una rete combinatoria con in coda multiplexer guidati da variabili di comando (che sono sia variabili di controllo dei multiplexer e variabili di comando della parte operativa). 

Dalla parte operativa genereremo le variabili di condizionamento, attraverso un suo sottoinsieme detto \textbf{rete combinatoria di condizionamento}, che useremo per sintetizzare la parte di controllo.
La rete combinatoria di condizionamento prende come ingressi le variabili di ingresso della RSS e lo stato dei registri operativi (che in generale sono visibili alla parte operativa).
Notiamo che la parte operativi rappresenta effettivamente una rete di Mealy, in quanto alcune uscite escono da registi (regisri multifunzionali), e altre vengono direttamente da reti combinatorie connesse agli ingressi (variabili di condizionamento).

La \textbf{parte di controllo} si occupa invece di gestire, sostanzialmente, il registro STAR.
Prende in ingresso le variabili di condizionamento appena generate, e restituisce in uscita le variabili di comando dei multiplexer della parte operativa.
Questo si può implementare effettivamente come un RSS di Moore, dove le uscite dei registri rappresentano le variabili di comando stesse.

Notiamo quindi di aver ricondotto l'RSS a due reti sequenziali, una di Moore e una di Mealy, in retroazione fra di loro.
Questo, come abbiamo visto, si può fare in quanto una di loro è di Moore.

\end{document}



\documentclass[a4paper,11pt]{article}
\usepackage[a4paper, margin=8em]{geometry}

% usa i pacchetti per la scrittura in italiano
\usepackage[french,italian]{babel}
\usepackage[T1]{fontenc}
\usepackage[utf8]{inputenc}
\frenchspacing 

% usa i pacchetti per la formattazione matematica
\usepackage{amsmath, amssymb, amsthm, amsfonts}

% usa altri pacchetti
\usepackage{gensymb}
\usepackage{hyperref}
\usepackage{standalone}

\usepackage{colortbl}

\usepackage{xstring}
\usepackage{karnaugh-map}

% imposta il titolo
\title{Appunti Reti Logiche}
\author{Luca Seggiani}
\date{2024}

% imposta lo stile
% usa helvetica
\usepackage[scaled]{helvet}
% usa palatino
\usepackage{palatino}
% usa un font monospazio guardabile
\usepackage{lmodern}

\renewcommand{\rmdefault}{ppl}
\renewcommand{\sfdefault}{phv}
\renewcommand{\ttdefault}{lmtt}

% circuiti
\usepackage{circuitikz}
\usetikzlibrary{babel}

% testo cerchiato
\newcommand*\circled[1]{\tikz[baseline=(char.base)]{
            \node[shape=circle,draw,inner sep=2pt] (char) {#1};}}

% disponi il titolo
\makeatletter
\renewcommand{\maketitle} {
	\begin{center} 
		\begin{minipage}[t]{.8\textwidth}
			\textsf{\huge\bfseries \@title} 
		\end{minipage}%
		\begin{minipage}[t]{.2\textwidth}
			\raggedleft \vspace{-1.65em}
			\textsf{\small \@author} \vfill
			\textsf{\small \@date}
		\end{minipage}
		\par
	\end{center}

	\thispagestyle{empty}
	\pagestyle{fancy}
}
\makeatother

% disponi teoremi
\usepackage{tcolorbox}
\newtcolorbox[auto counter, number within=section]{theorem}[2][]{%
	colback=blue!10, 
	colframe=blue!40!black, 
	sharp corners=northwest,
	fonttitle=\sffamily\bfseries, 
	title=Teorema~\thetcbcounter: #2, 
	#1
}

% disponi definizioni
\newtcolorbox[auto counter, number within=section]{definition}[2][]{%
	colback=red!10,
	colframe=red!40!black,
	sharp corners=northwest,
	fonttitle=\sffamily\bfseries,
	title=Definizione~\thetcbcounter: #2,
	#1
}

% disponi codice
\usepackage{listings}
\usepackage[table]{xcolor}

\definecolor{codegreen}{rgb}{0,0.6,0}
\definecolor{codegray}{rgb}{0.5,0.5,0.5}
\definecolor{codepurple}{rgb}{0.58,0,0.82}
\definecolor{backcolour}{rgb}{0.95,0.95,0.92}

\lstdefinestyle{codestyle}{
		backgroundcolor=\color{black!5}, 
		commentstyle=\color{codegreen},
		keywordstyle=\bfseries\color{magenta},
		numberstyle=\sffamily\tiny\color{black!60},
		stringstyle=\color{green!50!black},
		basicstyle=\ttfamily\footnotesize,
		breakatwhitespace=false,         
		breaklines=true,                 
		captionpos=b,                    
		keepspaces=true,                 
		numbers=left,                    
		numbersep=5pt,                  
		showspaces=false,                
		showstringspaces=false,
		showtabs=false,                  
		tabsize=2
}

\lstdefinestyle{shellstyle}{
		backgroundcolor=\color{black!5}, 
		basicstyle=\ttfamily\footnotesize\color{black}, 
		commentstyle=\color{black}, 
		keywordstyle=\color{black},
		numberstyle=\color{black!5},
		stringstyle=\color{black}, 
		showspaces=false,
		showstringspaces=false, 
		showtabs=false, 
		tabsize=2, 
		numbers=none, 
		breaklines=true
}


\lstdefinelanguage{assembler}{ 
  keywords={AAA, AAD, AAM, AAS, ADC, ADCB, ADCW, ADCL, ADD, ADDB, ADDW, ADDL, AND, ANDB, ANDW, ANDL,
        ARPL, BOUND, BSF, BSFL, BSFW, BSR, BSRL, BSRW, BSWAP, BT, BTC, BTCB, BTCW, BTCL, BTR, 
        BTRB, BTRW, BTRL, BTS, BTSB, BTSW, BTSL, CALL, CBW, CDQ, CLC, CLD, CLI, CLTS, CMC, CMP,
        CMPB, CMPW, CMPL, CMPS, CMPSB, CMPSD, CMPSW, CMPXCHG, CMPXCHGB, CMPXCHGW, CMPXCHGL,
        CMPXCHG8B, CPUID, CWDE, DAA, DAS, DEC, DECB, DECW, DECL, DIV, DIVB, DIVW, DIVL, ENTER,
        HLT, IDIV, IDIVB, IDIVW, IDIVL, IMUL, IMULB, IMULW, IMULL, IN, INB, INW, INL, INC, INCB,
        INCW, INCL, INS, INSB, INSD, INSW, INT, INT3, INTO, INVD, INVLPG, IRET, IRETD, JA, JAE,
        JB, JBE, JC, JCXZ, JE, JECXZ, JG, JGE, JL, JLE, JMP, JNA, JNAE, JNB, JNBE, JNC, JNE, JNG,
        JNGE, JNL, JNLE, JNO, JNP, JNS, JNZ, JO, JP, JPE, JPO, JS, JZ, LAHF, LAR, LCALL, LDS,
        LEA, LEAVE, LES, LFS, LGDT, LGS, LIDT, LMSW, LOCK, LODSB, LODSD, LODSW, LOOP, LOOPE,
        LOOPNE, LSL, LSS, LTR, MOV, MOVB, MOVW, MOVL, MOVSB, MOVSD, MOVSW, MOVSX, MOVSXB,
        MOVSXW, MOVSXL, MOVZX, MOVZXB, MOVZXW, MOVZXL, MUL, MULB, MULW, MULL, NEG, NEGB, NEGW,
        NEGL, NOP, NOT, NOTB, NOTW, NOTL, OR, ORB, ORW, ORL, OUT, OUTB, OUTW, OUTL, OUTSB, OUTSD,
        OUTSW, POP, POPL, POPW, POPB, POPA, POPAD, POPF, POPFD, PUSH, PUSHL, PUSHW, PUSHB, PUSHA, 
				PUSHAD, PUSHF, PUSHFD, RCL, RCLB, RCLW, MOVSL, MOVSB, MOVSW, STOSL, STOSB, STOSW, LODSB, LODSW,
				LODSL, INSB, INSW, INSL, OUTSB, OUTSL, OUTSW
        RCLL, RCR, RCRB, RCRW, RCRL, RDMSR, RDPMC, RDTSC, REP, REPE, REPNE, RET, ROL, ROLB, ROLW,
        ROLL, ROR, RORB, RORW, RORL, SAHF, SAL, SALB, SALW, SALL, SAR, SARB, SARW, SARL, SBB,
        SBBB, SBBW, SBBL, SCASB, SCASD, SCASW, SETA, SETAE, SETB, SETBE, SETC, SETE, SETG, SETGE,
        SETL, SETLE, SETNA, SETNAE, SETNB, SETNBE, SETNC, SETNE, SETNG, SETNGE, SETNL, SETNLE,
        SETNO, SETNP, SETNS, SETNZ, SETO, SETP, SETPE, SETPO, SETS, SETZ, SGDT, SHL, SHLB, SHLW,
        SHLL, SHLD, SHR, SHRB, SHRW, SHRL, SHRD, SIDT, SLDT, SMSW, STC, STD, STI, STOSB, STOSD,
        STOSW, STR, SUB, SUBB, SUBW, SUBL, TEST, TESTB, TESTW, TESTL, VERR, VERW, WAIT, WBINVD,
        XADD, XADDB, XADDW, XADDL, XCHG, XCHGB, XCHGW, XCHGL, XLAT, XLATB, XOR, XORB, XORW, XORL},
  keywordstyle=\color{blue}\bfseries,
  ndkeywordstyle=\color{darkgray}\bfseries,
  identifierstyle=\color{black},
  sensitive=false,
  comment=[l]{\#},
  morecomment=[s]{/*}{*/},
  commentstyle=\color{purple}\ttfamily,
  stringstyle=\color{red}\ttfamily,
  morestring=[b]',
  morestring=[b]"
}

\lstset{language=assembler, style=codestyle}

% disponi sezioni
\usepackage{titlesec}

\titleformat{\section}
	{\sffamily\Large\bfseries} 
	{\thesection}{1em}{} 
\titleformat{\subsection}
	{\sffamily\large\bfseries}   
	{\thesubsection}{1em}{} 
\titleformat{\subsubsection}
	{\sffamily\normalsize\bfseries} 
	{\thesubsubsection}{1em}{}

% tikz
\usepackage{tikz}

% float
\usepackage{float}

% grafici
\usepackage{pgfplots}
\pgfplotsset{width=10cm,compat=1.9}

% disponi alberi
\usepackage{forest}

\forestset{
	rectstyle/.style={
		for tree={rectangle,draw,font=\large\sffamily}
	},
	roundstyle/.style={
		for tree={circle,draw,font=\large}
	}
}

% disponi algoritmi
\usepackage{algorithm}
\usepackage{algorithmic}
\makeatletter
\renewcommand{\ALG@name}{Algoritmo}
\makeatother

% disponi numeri di pagina
\usepackage{fancyhdr}
\fancyhf{} 
\fancyfoot[L]{\sffamily{\thepage}}

\makeatletter
\fancyhead[L]{\raisebox{1ex}[0pt][0pt]{\sffamily{\@title \ \@date}}} 
\fancyhead[R]{\raisebox{1ex}[0pt][0pt]{\sffamily{\@author}}}
\makeatother

\begin{document}
% sezione (data)
\section{Lezione del 22-11-24}

% stili pagina
\thispagestyle{empty}
\pagestyle{fancy}

% testo
\subsection{Microprogrammazione della parte di controllo}
Le tecniche di microprogrammazione ci permettono di sintetizzare la parte di controllo di reti complesse.
In particolare, associata una codifica ad ogni stato del registro STAR, e chiamata questa codifica per ogni STAR $\boldsymbol{\mu}$\textbf{-indirizzo} dello stato, possiamo creare una tabella:
\begin{table}[H]
	\center 
	\begin{tabular} { c | c  c  c  c }
		& \multicolumn{4}{c}{$\boldsymbol{\mu}$\bfseries -istruzione} \\ 
		\hline
		$\boldsymbol{\mu}$\bfseries-{indirizzo} & 
		$\boldsymbol{\mu}$\bfseries-{codice} & 
		$c_{eff}$ & 
		$\boldsymbol{\mu}$\bfseries-{indirizzo T} & 
		$\boldsymbol{\mu}$\bfseries-{indirizzo F} \\
		\hline 
		00 & 00 & 1 & 00 & 01 \\ 
		01 & 11 & 0 & 10 & 01 \\ 
		...
	\end{tabular}
\end{table}

dove si associa ad ogni $\mu$-indirizzo, quindi ad ogni stato, una $\boldsymbol{\mu}$\textbf{-istruzione}: intendiamo una $\mu$-istruzione come un insieme di variabili di comando associate a quello stato (il $\boldsymbol{\mu}$\textbf{-codice}), una \textbf{variabile efficiente} $c_{eff}$ sulla base della quale si effettuano i $\mu$-salti, e due $\mu$-indirizzi, $T$ e $F$, che determinano il salto successivo sulla base del valore, rispettivamente vero o falso, della $c_{eff}$.

A partire da una tabella del genere, possiamo sintetizzare la PC secondo due modalità:
\begin{itemize}
	\item \textbf{Modello basato sui} $\boldsymbol{\mu}$\textbf{-indirizzi:}
		# mi sa non riesco ora
	\item \textbf{Modello basato sulle} $\boldsymbol{\mu}$\textbf{-istruzioni:}
\end{itemize}

\subsubsection{Reintrodurre i salti a più vie}
Nel caso di salti a più di due vie, si dovranno considerare più condizioni in cicli di clock differenti.
# fai esempio
Questo diventa poco efficiente quando i salti sono a un elevato numero di vie, in quanto per $n$ possibili $\mu$-salti si perdono $\sim n$ cicli di clock.

Nella pratica, i processori sono spesso progettati per compiere salti a due vie, tranne che in due casi particolari:
\begin{itemize}
	\item All'\textbf{inizio} della fase di fetch, cioè quando si legge il \textbf{formato} dell'opcode, dove si dovrà saltare a un blocco $\mu$-codice diverso a seconda della posizione degli operandi # fai esempio assembler.
		Si perderanno quindi $\sim f$ cicli per $f$ formati possibili delle istruzioni;
	\item Alla \textbf{fine} della fase di fetch, cioè quando si determina il salto al blocco di $\mu$-codice che gestisce la \textbf{fase di esecuzione} dell'istruzione.
		Si perderanno quindi $\sim i$ cicli per $i$ istruzioni possibili.
\end{itemize}
Una soluzione al problema dei salti a più vie e data quindi dal \textbf{Multiway Jump Register}.

\subsubsection{Multiway Jump Register}
L'MJR non è un gruppo punk americano ma un \textbf{registro operativo} destinato a contenere indirizzi di salto.
Generiamo l'ingresso del MJR attraverso la parte operativa della sintesi, e lo utilizziamo nella parte di controllo.

Per codificare la presenza del MJR, nella ROM della sintesi della parte di controllo dovremmo introdurre una nuova uscita, il $\boldsymbol{\mu}$\textbf{-tipo}.
Il valore del $\mu$-tipo determina il tipo di salto che vorremo eseguire: $\mu$-tipo a 0 significherà salto standard a 2 vie, e $\mu$-tipo a 1 significherà salto basato sul MJR.

# parla del b\_k di sovrascrittura del MJR

\subsection{Sottoliste}
Talvolta può convenire strutturare una descrizione di RSS con sottoliste simili a \textbf{sottoprogrammi}.
Porzioni di $\mu$-programma diverse potranno quindi essere raggiunte da stati di partenza diversi, che riporteranno allo stato di partenza stesso al termine della loro esecuzione attraverso un processo simile a quello delle CALL e RET viste sull'assembly (# integra).
Questo può essere implementato nella pratica, inserendo il $\mu$-indirizzo successivo all'esecuzione della sottolista nel MJR, cioè impostando $b_k$ a 0 per quello stato, e inserendo quindi il $\mu$-indirizzo dell'inizio della sottolista in STAR.
A questo punto la rete di controllo "eseguirà" il $\mu$-codice ed effettuerà i $\mu$-salti specificati dalla sottolista fino al passo finale, che rimetterà MJR in STAR, e quindi riprenderà l'esecuzione dal $\mu$-indirizzo memorizzato prima della "chiamata" della sottolista.

Due limitazioni di questo approccio sono che MJR diventa inutilizzabile durante l'esecuzione della sottolista, e sopratutto che un singolo MJR ci permette un solo livello di annidamento di sottoliste.
Per avere più livelli avremo bisogno di una \textbf{pila di MJR}, che però non è trattata in questo corso.

\subsection{Struttura del calcolatore}
Siamo arrivati ora a a poter descrivere in Verilog un \textbf{sistema completo} di:
\begin{itemize}
	\item Processore;
	\item Memoria;
	\item Interfacce;
	\item Dispositivi di I/O
\end{itemize}
collegati fra di loro attraverso una rete di interconnessione.

All'interno del \textbf{sottosistema di ingresso/uscita} distinguiamo \textbf{interfacce} e \textbf{dispositivi}.
Gli ultimi si occupano effettivamente di ottenere codifiche di dati dal mondo esterno, mentre le prime gestiscono i dispositivi in modo che questi possano colloquiare col processore.
Le interfacce contengono un piccolo numero di \textbf{registri di interfaccia} su cui il processore può leggere o scrivere.

La \textbf{memoria principale} sarà formata in larga parte da memoria RAM, e conterra in ogni istante le \textbf{istruzioni} e i \textbf{dati} che questo elabora.
Una parte della memoria principale dovrà essere implementata attraverso memoria ROM, in quanto c'è da risolvere il problema dello stato di avvio del processore introducendo dati predefiniti che vengono puntati per primi dall'istruction pointer. (# credo)
Il modello che andremo a studiare poi sarà dotato di memoria video, che conterrà le immagini visualizzate sullo schermo, e sarà anch'essa in diretta comunicazione col processore.

Il \textbf{processore} eseguirà il ciclo \textbf{fetch-execute}, prelevando dala memoria principale \textbf{istruzioni operative} e \textbf{istruzioni di controllo}.
Dovrà partire in una determinata configurazione dei registri, ottenuta collegando opportunamente piedini di \lstinline|/preset| e \lstinline|/preclear| alla linea di \lstinline|/reset|, in modo da inizializzare (come detto prima) l'instruction pointer a puntare ad una locazione di memoria nota che lanci un determinato programma in memoria, detto \textbf{bootstrapper}.

Per quanto ci riguarda, il calcolatore sarà formato da una serie di RSS, e il processore potrà essere sintetizzato attraverso la separazione PO/PC.

\subsubsection{Memoria}
La nostra memoria sarà formata da uno spazio lineare di $2^{24}$ lozazioni di memoria da un byte, per un totale di 16 MB indirizzati su 24 bit (3 byte).
Lo spazio di I/O sarà formato da uno spazio lineare di $2^{16}$ locazioni di memoria da un byte, per un totale di 64 B indirizzati su 16 bit (2 byte).

\subsubsection{Processore}
Il processore sarà dotato di 3 tipi di registri:
\begin{itemize}
	\item \textbf{Registri accumulatore:} AH e AL, da 8 bit ciascuno;
	\item \textbf{Registro dei flag:} 8 bit con 4 significativi: CF (0), ZF (1), SF (2), OF (3);
	\item \textbf{Registri puntatore:} da 24 bit (3 byte) ciascuno (devono contenere indirizzi di memoria), sono:
		\begin{itemize}
			\item \textbf{IP}: l'instruction pointer;
			\item \textbf{SP}: lo stack pointer;
			\item \textbf{DP}: il data pointer, che come vedremo contiene le locazioni degli operandi di istruzioni.
		\end{itemize}
\end{itemize}

Come avevamo visto, non programmeremo il nostro processore attraverso il linguaggio macchina, ma con un linguaggio assembler che codifica le istruzioni macchina, nella forma già vista:
\begin{lstlisting}	
OPCODE source, destination
\end{lstlisting}

Questo linguaggio sarà simile a quello già studiato, cioè dei processori Intel x86.
La differenza sarà che avremo come problema il dover effettivamente codificare ciò che scriviamo in assembler in istruzioni in linguaggio macchina da fornire al processore (adesso non stiamo solo \textit{programmando}, ma anche \textit{progettando} il processore).

\subsubsection{Modalità di indirizzamento}
Avevamo visto le seguenti modalità di indirizzamento per le istruzioni:
\begin{itemize}
	\item Di registro;
	\item Immediato;
	\item Di memoria;
	\item Delle porte di I/O # riempi
\end{itemize}

# listone istruzioni

\end{document}


\documentclass[a4paper,11pt]{article}
\usepackage[a4paper, margin=8em]{geometry}

% usa i pacchetti per la scrittura in italiano
\usepackage[french,italian]{babel}
\usepackage[T1]{fontenc}
\usepackage[utf8]{inputenc}
\frenchspacing 

% usa i pacchetti per la formattazione matematica
\usepackage{amsmath, amssymb, amsthm, amsfonts}

% usa altri pacchetti
\usepackage{gensymb}
\usepackage{hyperref}
\usepackage{standalone}

\usepackage{colortbl}

\usepackage{xstring}
\usepackage{karnaugh-map}

% imposta il titolo
\title{Appunti Reti Logiche}
\author{Luca Seggiani}
\date{2024}

% imposta lo stile
% usa helvetica
\usepackage[scaled]{helvet}
% usa palatino
\usepackage{palatino}
% usa un font monospazio guardabile
\usepackage{lmodern}

\renewcommand{\rmdefault}{ppl}
\renewcommand{\sfdefault}{phv}
\renewcommand{\ttdefault}{lmtt}

% circuiti
\usepackage{circuitikz}
\usetikzlibrary{babel}

% testo cerchiato
\newcommand*\circled[1]{\tikz[baseline=(char.base)]{
            \node[shape=circle,draw,inner sep=2pt] (char) {#1};}}

% disponi il titolo
\makeatletter
\renewcommand{\maketitle} {
	\begin{center} 
		\begin{minipage}[t]{.8\textwidth}
			\textsf{\huge\bfseries \@title} 
		\end{minipage}%
		\begin{minipage}[t]{.2\textwidth}
			\raggedleft \vspace{-1.65em}
			\textsf{\small \@author} \vfill
			\textsf{\small \@date}
		\end{minipage}
		\par
	\end{center}

	\thispagestyle{empty}
	\pagestyle{fancy}
}
\makeatother

% disponi teoremi
\usepackage{tcolorbox}
\newtcolorbox[auto counter, number within=section]{theorem}[2][]{%
	colback=blue!10, 
	colframe=blue!40!black, 
	sharp corners=northwest,
	fonttitle=\sffamily\bfseries, 
	title=Teorema~\thetcbcounter: #2, 
	#1
}

% disponi definizioni
\newtcolorbox[auto counter, number within=section]{definition}[2][]{%
	colback=red!10,
	colframe=red!40!black,
	sharp corners=northwest,
	fonttitle=\sffamily\bfseries,
	title=Definizione~\thetcbcounter: #2,
	#1
}

% disponi codice
\usepackage{listings}
\usepackage[table]{xcolor}

\definecolor{codegreen}{rgb}{0,0.6,0}
\definecolor{codegray}{rgb}{0.5,0.5,0.5}
\definecolor{codepurple}{rgb}{0.58,0,0.82}
\definecolor{backcolour}{rgb}{0.95,0.95,0.92}

\lstdefinestyle{codestyle}{
		backgroundcolor=\color{black!5}, 
		commentstyle=\color{codegreen},
		keywordstyle=\bfseries\color{magenta},
		numberstyle=\sffamily\tiny\color{black!60},
		stringstyle=\color{green!50!black},
		basicstyle=\ttfamily\footnotesize,
		breakatwhitespace=false,         
		breaklines=true,                 
		captionpos=b,                    
		keepspaces=true,                 
		numbers=left,                    
		numbersep=5pt,                  
		showspaces=false,                
		showstringspaces=false,
		showtabs=false,                  
		tabsize=2
}

\lstdefinestyle{shellstyle}{
		backgroundcolor=\color{black!5}, 
		basicstyle=\ttfamily\footnotesize\color{black}, 
		commentstyle=\color{black}, 
		keywordstyle=\color{black},
		numberstyle=\color{black!5},
		stringstyle=\color{black}, 
		showspaces=false,
		showstringspaces=false, 
		showtabs=false, 
		tabsize=2, 
		numbers=none, 
		breaklines=true
}


\lstdefinelanguage{assembler}{ 
  keywords={AAA, AAD, AAM, AAS, ADC, ADCB, ADCW, ADCL, ADD, ADDB, ADDW, ADDL, AND, ANDB, ANDW, ANDL,
        ARPL, BOUND, BSF, BSFL, BSFW, BSR, BSRL, BSRW, BSWAP, BT, BTC, BTCB, BTCW, BTCL, BTR, 
        BTRB, BTRW, BTRL, BTS, BTSB, BTSW, BTSL, CALL, CBW, CDQ, CLC, CLD, CLI, CLTS, CMC, CMP,
        CMPB, CMPW, CMPL, CMPS, CMPSB, CMPSD, CMPSW, CMPXCHG, CMPXCHGB, CMPXCHGW, CMPXCHGL,
        CMPXCHG8B, CPUID, CWDE, DAA, DAS, DEC, DECB, DECW, DECL, DIV, DIVB, DIVW, DIVL, ENTER,
        HLT, IDIV, IDIVB, IDIVW, IDIVL, IMUL, IMULB, IMULW, IMULL, IN, INB, INW, INL, INC, INCB,
        INCW, INCL, INS, INSB, INSD, INSW, INT, INT3, INTO, INVD, INVLPG, IRET, IRETD, JA, JAE,
        JB, JBE, JC, JCXZ, JE, JECXZ, JG, JGE, JL, JLE, JMP, JNA, JNAE, JNB, JNBE, JNC, JNE, JNG,
        JNGE, JNL, JNLE, JNO, JNP, JNS, JNZ, JO, JP, JPE, JPO, JS, JZ, LAHF, LAR, LCALL, LDS,
        LEA, LEAVE, LES, LFS, LGDT, LGS, LIDT, LMSW, LOCK, LODSB, LODSD, LODSW, LOOP, LOOPE,
        LOOPNE, LSL, LSS, LTR, MOV, MOVB, MOVW, MOVL, MOVSB, MOVSD, MOVSW, MOVSX, MOVSXB,
        MOVSXW, MOVSXL, MOVZX, MOVZXB, MOVZXW, MOVZXL, MUL, MULB, MULW, MULL, NEG, NEGB, NEGW,
        NEGL, NOP, NOT, NOTB, NOTW, NOTL, OR, ORB, ORW, ORL, OUT, OUTB, OUTW, OUTL, OUTSB, OUTSD,
        OUTSW, POP, POPL, POPW, POPB, POPA, POPAD, POPF, POPFD, PUSH, PUSHL, PUSHW, PUSHB, PUSHA, 
				PUSHAD, PUSHF, PUSHFD, RCL, RCLB, RCLW, MOVSL, MOVSB, MOVSW, STOSL, STOSB, STOSW, LODSB, LODSW,
				LODSL, INSB, INSW, INSL, OUTSB, OUTSL, OUTSW
        RCLL, RCR, RCRB, RCRW, RCRL, RDMSR, RDPMC, RDTSC, REP, REPE, REPNE, RET, ROL, ROLB, ROLW,
        ROLL, ROR, RORB, RORW, RORL, SAHF, SAL, SALB, SALW, SALL, SAR, SARB, SARW, SARL, SBB,
        SBBB, SBBW, SBBL, SCASB, SCASD, SCASW, SETA, SETAE, SETB, SETBE, SETC, SETE, SETG, SETGE,
        SETL, SETLE, SETNA, SETNAE, SETNB, SETNBE, SETNC, SETNE, SETNG, SETNGE, SETNL, SETNLE,
        SETNO, SETNP, SETNS, SETNZ, SETO, SETP, SETPE, SETPO, SETS, SETZ, SGDT, SHL, SHLB, SHLW,
        SHLL, SHLD, SHR, SHRB, SHRW, SHRL, SHRD, SIDT, SLDT, SMSW, STC, STD, STI, STOSB, STOSD,
        STOSW, STR, SUB, SUBB, SUBW, SUBL, TEST, TESTB, TESTW, TESTL, VERR, VERW, WAIT, WBINVD,
        XADD, XADDB, XADDW, XADDL, XCHG, XCHGB, XCHGW, XCHGL, XLAT, XLATB, XOR, XORB, XORW, XORL},
  keywordstyle=\color{blue}\bfseries,
  ndkeywordstyle=\color{darkgray}\bfseries,
  identifierstyle=\color{black},
  sensitive=false,
  comment=[l]{\#},
  morecomment=[s]{/*}{*/},
  commentstyle=\color{purple}\ttfamily,
  stringstyle=\color{red}\ttfamily,
  morestring=[b]',
  morestring=[b]"
}

\lstset{language=assembler, style=codestyle}

% disponi sezioni
\usepackage{titlesec}

\titleformat{\section}
	{\sffamily\Large\bfseries} 
	{\thesection}{1em}{} 
\titleformat{\subsection}
	{\sffamily\large\bfseries}   
	{\thesubsection}{1em}{} 
\titleformat{\subsubsection}
	{\sffamily\normalsize\bfseries} 
	{\thesubsubsection}{1em}{}

% tikz
\usepackage{tikz}

% float
\usepackage{float}

% grafici
\usepackage{pgfplots}
\pgfplotsset{width=10cm,compat=1.9}

% disponi alberi
\usepackage{forest}

\forestset{
	rectstyle/.style={
		for tree={rectangle,draw,font=\large\sffamily}
	},
	roundstyle/.style={
		for tree={circle,draw,font=\large}
	}
}

% disponi algoritmi
\usepackage{algorithm}
\usepackage{algorithmic}
\makeatletter
\renewcommand{\ALG@name}{Algoritmo}
\makeatother

% disponi numeri di pagina
\usepackage{fancyhdr}
\fancyhf{} 
\fancyfoot[L]{\sffamily{\thepage}}

\makeatletter
\fancyhead[L]{\raisebox{1ex}[0pt][0pt]{\sffamily{\@title \ \@date}}} 
\fancyhead[R]{\raisebox{1ex}[0pt][0pt]{\sffamily{\@author}}}
\makeatother

\begin{document}
% sezione (data)
\section{Lezione del 26-11-24}

% stili pagina
\thispagestyle{empty}
\pagestyle{fancy}

% testo
\subsection{Dall'assembler al linguaggio macchina}
Abbiamo visto come un processore si programma attraverso il linguaggio \textbf{assembler}, che è più distante dai dettagli di implementazione e più vicino al linguaggio umano, quindi al programmatore.
Vediamo quindi il processo di conversione dall'assembler al linguaggio macchina "\textit{assemblaggio}".

\subsubsection{Fetch degli operandi}
Il problema principale del processore è indiduare gli operandi in base al tipo di indirizzamento dell'istruzione.
Ad esempio:
\begin{itemize}
	\item \lstinline|MOV AH, AL| contiene operandi che si trovano già nei registri;
	\item \lstinline|MOV $0x10, AL| contiene operandi che vanno letti in memoria assieme all'istruzione stessa;
	\item \lstinline|MOV 0x00FF, AL| contiene operandi che vanno letti in memoria, in locazioni diverse da quella dell'istruzione stessa.
\end{itemize}

Durante la fase di fetch, quindi, il processore deve procurarsi gli operandi necessari ad eseguire la prossima istruzione, che questi siano nei registri o in memoria.
Questa operazione sarà comune a ogni istruzione, cioè le modalità di indirizzamento degli operandi saranno identiche che si parli di una \lstinline|ADD| come di una \lstinline|MOV|, ecc... (finchè le istruzioni appartengono allo stesso \textit{formato}, vedi sotto).
Dal punto di vista pratico, questo significa che la parte di \textbf{fetch} potrà essere messa \textit{in comune}, mentre la fase di \textbf{esecuzione} sarà \textit{specifica} ad ogni istruzione.

\subsubsection{Formato delle istruzioni macchina}
I primi 3 bit del codice operativo di un istruzione rappresentano il cosiddetto \textbf{formato} dell'istruzione, ovvero le modalità secondo cui si dovrebbero ricavare gli operandi in fase di fetch:
\begin{itemize}
	\item \textbf{Formato F0} (000): sono istruzioni per le quali il processore non deve compiere nessuna azione, in quanto accade alternativamente che:
		\begin{itemize}
			\item Gli operandi sono registri;
			\item Le istruzioni non hanno operandi.
		\end{itemize}
		ergo una volta letto l'opcode puntato dall'instruction pointer, si può proseguire all'esecuzione;
	\item \textbf{Formato F1} (001): raggruppa le istruzioni non classificabili negli altri formati, fra cui:
		\begin{itemize}
			\item Istruzioni I/O, con indirizzo a 16 bit sia sorgente (\lstinline|IN|) che destinatario (\lstinline|OUT|);
			\item \lstinline|MOV| con uno dei registri a 24 bit (DP o SP), sia sorgente che destinatario.
		\end{itemize}
		Le operazioni in formato F1 si limitano in fase di fetch a prelevare l'OPCODE (fetch \textit{scarna}).
		Gli operandi vengono gestiti successivamente in fase di esecuzione.
		Questa soluzione è sì poco elegante, ma risulta più agevole dal punto di vista dell'implementazione.
	\item \textbf{Formato F2} (010): l'operando \textbf{sorgente} si trova in memoria, indirizzato attraverso DP.
		In questi casi tutta l'istruzione sta su un byte, ma l'operando sorgente va prelevato tramite un'ulteriore lettura in memoria di un byte;
	\item \textbf{Formato F3} (011): l'operando \textbf{destinatario} si trova in memoria, indirizzato attraverso DP.
	Notiamo che in questo caso l'operando destinatario non va \textit{prelevato} dal processore, ma deve essere \textit{sovrascritto}, cosa che accade durante la fase di esecuzione.
	Non ci sono quindi letture ulteriori in fase di fetch;
\item \textbf{Formato F4} (100): l'operando \textbf{sorgente} è indirizzato in modo \textbf{immediato}, e sta su 8 bit.
	In questo caso l'istruzione è lunga 2 byte, e nella fase di fetch si dovranno leggere due byte in memoria consecutivi puntati dal registro IP (e non DP come nel caso precedente);

\item \textbf{Formato F5} (101): l'operando \textbf{sorgente} si trova in memoria con indirizzamento \textbf{diretto}.
	Visto che lo spazio di memoria è a 24 bit, un istruzione sarà lunga 4 byte: 1 di OPCODE e 3 di indirizzo.
	In fase di fetch si dovrà quindi:
	\begin{itemize}
		\item Leggere i 4 byte consecutivi dell'istruzione puntati dal registro IP per ottenere istruzione e indirizzo dell'operando sorgente;
		\item Leggere l'operando sorgente stesso, in un'altra locazione di memoria.
	\end{itemize}
\item \textbf{Formato F6} (110): l'operando \textbf{destinatario} si trova in memoria con indirizzamento \textbf{diretto}. Di nuovo, il processore dovrà leggere 4 byte consecutivi puntati dall'IP, ma a questo punto non ci saranno \textit{letture} in memoria, bensì \textit{scritture} sull'indirizzo prelevato del destinatario, in fase di \textbf{esecuzione};
\item \textbf{Formato F7} (111): raggruppa le istruzioni di controllo seguite da indirizzo (quindi, ad esempio, non la \lstinline|RET|, ma le varie \lstinline|CALL|, \lstinline|JMP|, salti condizionali ecc..).
	In questo caso, l'indirizzo viene specificato come prima su 3 byte, e bisogna nuovamente leggere 4 byte consecutivi puntati dall'indirizzo in IP.
\end{itemize}

\subsection{Architettura hardware del calcolatore}
Vediamo quindi come viene implementato dal punto di vista fisico il calcolatore.
Innanzitutto consideriamo un \textbf{bus}, formato da:
\begin{itemize}
\item \textbf{Linee di indirizzo}, nel nostro esempio 24 per 24 bit, di uscita per il processore e ingresso per gli spazi di memoria e I/O, notando che nel salto dal bus allo spazio di I/O si perdono gli 8 bit più significativi (si passa da 24 a 16).
Non servono "forchette" con porte tri-state in quanto il processore è sempre l'unico a scrivere sulle linee;
\item \textbf{Linee dati}, nel nostro esempio 8 per 8 bit, usate per leggere e scrivere byte di memoria.
	In questo caso il processore potrebbe leggere (dalla memoria o dallo spazio di I/O) o scrivere (sempre nella memoria o nello spazio di I/O), ergo potrebbero esserci conflitti di pilotaggio dei dati.
	Si adottano quindi le porte tri-state, disposte come abbiamo visto a forchetta.
\item \textbf{Linee di controllo}, tutte attive basse, che sono nel nostro caso:
	\begin{itemize}
		\item \lstinline|/mr| e \lstinline|mw|, cioe memory read e memory write per la memoria;
		\item \lstinline|ior| e \lstinline|iow|, cioè I/O read e I/O write per lo spazio di I/O
	\end{itemize}
\item \textbf{Clock} e \textbf{reset}.
\end{itemize}

Notiamo come nel bus non figura la linea di selezione \lstinline|/s| per gli spazi di memoria, in quanto questo viene generato attraverso una maschera dalla memoria stessa sulla base degli indirizzo di lettura, cioè avrà il solo scopo di selezionare diversi banchi di memoria, a \textit{livello} di memoria.

\subsubsection{Spazio di memoria}
Abbiamo quindi che lo spazio di memoria è implementato, su 16 MB, in parte con tecnologia RAM, in parte con EPROM (che contiene il programma bootstrap), e in parte con memoria video dedicata (che non vedremo affatto).
Diciamo di avere 64 KB di memoria EPROM e 64 KB di memoria video, disposte come segue:

\begin{table}[h!]
	\center \rowcolors{2}{white}{black!10}
	\begin{tabular} { c | c | c }
		\bfseries Regione & \bfseries Indirizzi & \bfseries Num. locazioni \\
		\hline 
		RAM 1 & \lstinline|0x000000| & 589.824 byte \\
					& \lstinline|0x09FFFF| & \\	
		\hline
		M. Video & \lstinline|0x0A0000| & 65.536 byte \\
						 & \lstinline|0x0AFFFF| & \\	
		\hline
		RAM 2 & \lstinline|0x0B0000| & 16.056.320 byte \\
					& \lstinline|0xFEFFFF| & \\	
		\hline
		EPROM & \lstinline|0xFF0000| & 65.536 byte \\
					& \lstinline|0xFFFFFF| & \\
	\end{tabular}
\end{table}

Notiamo come in effetti solo 16.646.144 delle $2^{24} =$ 16.777.216 locazioni da un byte indirizzabili su 24 bit siano disponibili in memoria RAM.  

Possiamo combinare queste memorie attraverso, come abbiamo visto prima, linee di select generate attraverso opportune maschere:
\begin{lstlisting}[language=verilog, style=codestyle]	
wire sRAM_, sMV_, sEPROM_;
assign {sRAM_, sMV_, sEPROM_} = (a23_a16 = 'H0A) ? 'B101:
																(a23_a16 = 'HFF) ? 'B110:
																/* 	don't care	*/ 'B011;
\end{lstlisting}
e indirizzare quindi i moduli corrispondenti ad ogni indirizzo.

\subsubsection{Spazio di I/O}
Lo spazio di I/O è realizzato fisicamente attraverso \textbf{interfacce}, che sono elementi di raccordo tra il bus e i dispositivi I/O.
Dal lato dispositivo, queste sono implementate in una maniera che "risponde" al dispositivo.
Dal lato bus, invece, sono tutte uguali, cioè presentano le entrate di selezione, I/O read e I/O write, eventuali \textbf{indirizzi interni}, che servono a discriminare più \textbf{porte}, e le linee di entrata/uscita di un byte di dati, cioè, tranne che per gli indirizzi interni, le stesse linee fornite da una RAM.

Notiamo che in un interfaccia una porta può operare o in \textbf{sola lettura}, o in \textbf{sola scrittura}.
Ad esempio, non potremo scrivere nell'interfaccia di una tastiera, e non potremo leggere nell'interfaccia di un monitor.

Potremmo chiederci come mai implementare interfacce per ogni dispositivo, e non connetterli direttamente al bus.
Ci sono principalmente due ragioni:
\begin{itemize}
	\item Diversi dispositivi hanno \textbf{diverse velocità}: spesso di molti ordini di grandezza, e comunque molto più lente del processore;
	\item Diversi dispositivi hanno \textbf{diverse modalità di trasferimento}: a volta un bit alla volta (\textbf{seriale}), a volte in gruppi di bit (\textbf{parallelo}).
\end{itemize}

Implementare interfaccie ci permette quindi di \textbf{standardizzare} l'input e l'output del calcolatore, rendendo le temporizzazioni e le modalità di trasferimento \textbf{omogenee}.

\subsubsection{Processore}
Possiamo quindi individuare, nel processore, tutti i registri effettivamente necessari:
\begin{itemize}
	\item Registri visibili al programmatore, cioè:
		\begin{itemize}
			\item \textbf{AH};
			\item \textbf{AL};
			\item \textbf{IP};
			\item \textbf{DP};
			\item \textbf{SP};
			\item \textbf{F} (Flags);
		\end{itemize}
	\item Registri di supporto a uscite, fra cui:
		\begin{itemize}
			\item \textbf{D7\_D0} e \textbf{DIR} (supporta le forchette sulla linea dati);
			\item \textbf{A23\_A0};
			\item \textbf{/MR};
			\item \textbf{/MW};
			\item \textbf{/IOR};
			\item \textbf{/MW};
		\end{itemize}
		Dove notiamo la particolarità dei registri che supportano le linee dati, che saranno innanzitutto \textit{forchettati}, cioè fatti passare attraverso una porta tri-state controllata da un enabler generato dal registro DIR (per \textit{direzione}).
		DIR sarà a 0 di default, cioè scollegherà il registro dati di uscita dal bus, e lo porremo a 1 solo nel caso in cui dovremmo scrivere, attraverso tale registro, sul bus.
	\item I registri \textbf{STAR} e \textbf{MJR};
	\item Registri di \textbf{OPCODE}, \textbf{SOURCE} e \textbf{DEST\_ADDR} necessari alla fase di fetch;
	\item Registri di appoggio per operazioni, che saranno \textbf{APP3}, \textbf{APP2}, \textbf{APP1}, \textbf{APP0} (4 registri di appoggio) e \textbf{NUMLOC} (contiene il numero di locazioni a cui accedere quando si effettuano letture in memoria).
\end{itemize}

\subsubsection{Ciclo di fetch-execute}
Vediamo quindi i dettagli del ciclo fetch-execute.
\begin{itemize}
	\item La fase iniziale è quella di \textbf{reset}, dove si inizializzano i registri:
		\begin{itemize}
			\item F verrà inizializzato a 0;
			\item IP otterrà il valore del primo indirizzo dello spazio di memoria dove si trova il bootstrapper;
			\item STAR sarà inizializzato al primo stato;
			\item DIR verrà inizializzato a 0;
			\item /MR, /MW, /IOR e /IOW verranno inizializzati a 1 (attivi bassi);
		\end{itemize}
	\item Poi si passa alla fase di \textbf{fetch}, dove si prelevano istruzioni e operandi. In ordine:
		\begin{itemize}
			\item Il processore preleva un byte dalla memoria, all'indirizzo indicato in IP;
			\item Incrementa IP, modulo $2^{24}$;
			\item Controlla che il byte prelevato corrisponda all'OPCODE di una delle istruzioni che conosce.
				In caso contrario, si va in stato di blocco;
			\item Carica il byte prelevato in OPCODE;
			\item Controlla il formato dell'OPCODE in modo da definire le modalità di indirizzamento.
				A questo punto si ramifica, effettuando le letture necessarie in memoria come specificato qualche paragrafo fa. Nello specifico, si dovrannno eseguire le seguenti operazioni:
				\begin{itemize}
			\item F0: non si fa nulla;
			\item F1: gli operandi verranno raccolti in fase execute (per ora non si fa nulla).
					\item F2, F4, F5: operando sorgente da 8 bit in SOURCE
						\begin{itemize}
							\item F2: sorgente in memoria con indirizzamento indiretto: si fa accesso in memoria all'indirizzo contenuto in DP;
							\item F4: sorgente immediato: si fa un accesso all'indirizzo contenuto in IP. Bisogna incrementare IP di 1;
							\item F6: sorgente in mem. con indirizzamento diretto: si fanno due accessi in memoria, i 3 byte dell'indirizzo in IP e il byte all'indirizzo appena trovato. Bisogna incrementare IP di 3.
						\end{itemize}
				\item F3, F6, F7: indirizzo dell'operando destinatario da 24 bit in DEST\_ADDR
				\begin{itemize}
					\item F3: indirizzo destinatario in memoria: si copia l'indirizzo da DP;
					\item F6, F7: indirizzo destinatario in memoria: si fa accesso in memoria ai 3 byte puntati da IP. Bisogna incrementare IP di 3; 
				\end{itemize}
				\end{itemize}
			\item Come ultima cosa, si guarda al contenuto di OPCODE per iniziare l'esecuzione dell'istruzione desiderata.
		\end{itemize}
	\item Dopo la fase di fetch, viene la fase \textbf{execute}, dove si eseguono effettivamente operazioni sugli operandi;

	\item Nel caso di un errore in fase di fetch, o dell'incontro dell'istruzione HLT in fase execute, si dovrà andare in \textbf{stato di blocco}, cioè il processore dovrà smettere di rispondere agli ingressi e mantenere ferme le sue uscite.
\end{itemize}

\end{document}


\documentclass[a4paper,11pt]{article}
\usepackage[a4paper, margin=8em]{geometry}

% usa i pacchetti per la scrittura in italiano
\usepackage[french,italian]{babel}
\usepackage[T1]{fontenc}
\usepackage[utf8]{inputenc}
\frenchspacing 

% usa i pacchetti per la formattazione matematica
\usepackage{amsmath, amssymb, amsthm, amsfonts}

% usa altri pacchetti
\usepackage{gensymb}
\usepackage{hyperref}
\usepackage{standalone}

\usepackage{colortbl}

\usepackage{xstring}
\usepackage{karnaugh-map}

% imposta il titolo
\title{Appunti Reti Logiche}
\author{Luca Seggiani}
\date{2024}

% imposta lo stile
% usa helvetica
\usepackage[scaled]{helvet}
% usa palatino
\usepackage{palatino}
% usa un font monospazio guardabile
\usepackage{lmodern}

\renewcommand{\rmdefault}{ppl}
\renewcommand{\sfdefault}{phv}
\renewcommand{\ttdefault}{lmtt}

% circuiti
\usepackage{circuitikz}
\usetikzlibrary{babel}

% testo cerchiato
\newcommand*\circled[1]{\tikz[baseline=(char.base)]{
            \node[shape=circle,draw,inner sep=2pt] (char) {#1};}}

% disponi il titolo
\makeatletter
\renewcommand{\maketitle} {
	\begin{center} 
		\begin{minipage}[t]{.8\textwidth}
			\textsf{\huge\bfseries \@title} 
		\end{minipage}%
		\begin{minipage}[t]{.2\textwidth}
			\raggedleft \vspace{-1.65em}
			\textsf{\small \@author} \vfill
			\textsf{\small \@date}
		\end{minipage}
		\par
	\end{center}

	\thispagestyle{empty}
	\pagestyle{fancy}
}
\makeatother

% disponi teoremi
\usepackage{tcolorbox}
\newtcolorbox[auto counter, number within=section]{theorem}[2][]{%
	colback=blue!10, 
	colframe=blue!40!black, 
	sharp corners=northwest,
	fonttitle=\sffamily\bfseries, 
	title=Teorema~\thetcbcounter: #2, 
	#1
}

% disponi definizioni
\newtcolorbox[auto counter, number within=section]{definition}[2][]{%
	colback=red!10,
	colframe=red!40!black,
	sharp corners=northwest,
	fonttitle=\sffamily\bfseries,
	title=Definizione~\thetcbcounter: #2,
	#1
}

% disponi codice
\usepackage{listings}
\usepackage[table]{xcolor}

\definecolor{codegreen}{rgb}{0,0.6,0}
\definecolor{codegray}{rgb}{0.5,0.5,0.5}
\definecolor{codepurple}{rgb}{0.58,0,0.82}
\definecolor{backcolour}{rgb}{0.95,0.95,0.92}

\lstdefinestyle{codestyle}{
		backgroundcolor=\color{black!5}, 
		commentstyle=\color{codegreen},
		keywordstyle=\bfseries\color{magenta},
		numberstyle=\sffamily\tiny\color{black!60},
		stringstyle=\color{green!50!black},
		basicstyle=\ttfamily\footnotesize,
		breakatwhitespace=false,         
		breaklines=true,                 
		captionpos=b,                    
		keepspaces=true,                 
		numbers=left,                    
		numbersep=5pt,                  
		showspaces=false,                
		showstringspaces=false,
		showtabs=false,                  
		tabsize=2
}

\lstdefinestyle{shellstyle}{
		backgroundcolor=\color{black!5}, 
		basicstyle=\ttfamily\footnotesize\color{black}, 
		commentstyle=\color{black}, 
		keywordstyle=\color{black},
		numberstyle=\color{black!5},
		stringstyle=\color{black}, 
		showspaces=false,
		showstringspaces=false, 
		showtabs=false, 
		tabsize=2, 
		numbers=none, 
		breaklines=true
}


\lstdefinelanguage{assembler}{ 
  keywords={AAA, AAD, AAM, AAS, ADC, ADCB, ADCW, ADCL, ADD, ADDB, ADDW, ADDL, AND, ANDB, ANDW, ANDL,
        ARPL, BOUND, BSF, BSFL, BSFW, BSR, BSRL, BSRW, BSWAP, BT, BTC, BTCB, BTCW, BTCL, BTR, 
        BTRB, BTRW, BTRL, BTS, BTSB, BTSW, BTSL, CALL, CBW, CDQ, CLC, CLD, CLI, CLTS, CMC, CMP,
        CMPB, CMPW, CMPL, CMPS, CMPSB, CMPSD, CMPSW, CMPXCHG, CMPXCHGB, CMPXCHGW, CMPXCHGL,
        CMPXCHG8B, CPUID, CWDE, DAA, DAS, DEC, DECB, DECW, DECL, DIV, DIVB, DIVW, DIVL, ENTER,
        HLT, IDIV, IDIVB, IDIVW, IDIVL, IMUL, IMULB, IMULW, IMULL, IN, INB, INW, INL, INC, INCB,
        INCW, INCL, INS, INSB, INSD, INSW, INT, INT3, INTO, INVD, INVLPG, IRET, IRETD, JA, JAE,
        JB, JBE, JC, JCXZ, JE, JECXZ, JG, JGE, JL, JLE, JMP, JNA, JNAE, JNB, JNBE, JNC, JNE, JNG,
        JNGE, JNL, JNLE, JNO, JNP, JNS, JNZ, JO, JP, JPE, JPO, JS, JZ, LAHF, LAR, LCALL, LDS,
        LEA, LEAVE, LES, LFS, LGDT, LGS, LIDT, LMSW, LOCK, LODSB, LODSD, LODSW, LOOP, LOOPE,
        LOOPNE, LSL, LSS, LTR, MOV, MOVB, MOVW, MOVL, MOVSB, MOVSD, MOVSW, MOVSX, MOVSXB,
        MOVSXW, MOVSXL, MOVZX, MOVZXB, MOVZXW, MOVZXL, MUL, MULB, MULW, MULL, NEG, NEGB, NEGW,
        NEGL, NOP, NOT, NOTB, NOTW, NOTL, OR, ORB, ORW, ORL, OUT, OUTB, OUTW, OUTL, OUTSB, OUTSD,
        OUTSW, POP, POPL, POPW, POPB, POPA, POPAD, POPF, POPFD, PUSH, PUSHL, PUSHW, PUSHB, PUSHA, 
				PUSHAD, PUSHF, PUSHFD, RCL, RCLB, RCLW, MOVSL, MOVSB, MOVSW, STOSL, STOSB, STOSW, LODSB, LODSW,
				LODSL, INSB, INSW, INSL, OUTSB, OUTSL, OUTSW
        RCLL, RCR, RCRB, RCRW, RCRL, RDMSR, RDPMC, RDTSC, REP, REPE, REPNE, RET, ROL, ROLB, ROLW,
        ROLL, ROR, RORB, RORW, RORL, SAHF, SAL, SALB, SALW, SALL, SAR, SARB, SARW, SARL, SBB,
        SBBB, SBBW, SBBL, SCASB, SCASD, SCASW, SETA, SETAE, SETB, SETBE, SETC, SETE, SETG, SETGE,
        SETL, SETLE, SETNA, SETNAE, SETNB, SETNBE, SETNC, SETNE, SETNG, SETNGE, SETNL, SETNLE,
        SETNO, SETNP, SETNS, SETNZ, SETO, SETP, SETPE, SETPO, SETS, SETZ, SGDT, SHL, SHLB, SHLW,
        SHLL, SHLD, SHR, SHRB, SHRW, SHRL, SHRD, SIDT, SLDT, SMSW, STC, STD, STI, STOSB, STOSD,
        STOSW, STR, SUB, SUBB, SUBW, SUBL, TEST, TESTB, TESTW, TESTL, VERR, VERW, WAIT, WBINVD,
        XADD, XADDB, XADDW, XADDL, XCHG, XCHGB, XCHGW, XCHGL, XLAT, XLATB, XOR, XORB, XORW, XORL},
  keywordstyle=\color{blue}\bfseries,
  ndkeywordstyle=\color{darkgray}\bfseries,
  identifierstyle=\color{black},
  sensitive=false,
  comment=[l]{\#},
  morecomment=[s]{/*}{*/},
  commentstyle=\color{purple}\ttfamily,
  stringstyle=\color{red}\ttfamily,
  morestring=[b]',
  morestring=[b]"
}

\lstset{language=assembler, style=codestyle}

% disponi sezioni
\usepackage{titlesec}

\titleformat{\section}
	{\sffamily\Large\bfseries} 
	{\thesection}{1em}{} 
\titleformat{\subsection}
	{\sffamily\large\bfseries}   
	{\thesubsection}{1em}{} 
\titleformat{\subsubsection}
	{\sffamily\normalsize\bfseries} 
	{\thesubsubsection}{1em}{}

% tikz
\usepackage{tikz}

% float
\usepackage{float}

% grafici
\usepackage{pgfplots}
\pgfplotsset{width=10cm,compat=1.9}

% disponi alberi
\usepackage{forest}

\forestset{
	rectstyle/.style={
		for tree={rectangle,draw,font=\large\sffamily}
	},
	roundstyle/.style={
		for tree={circle,draw,font=\large}
	}
}

% disponi algoritmi
\usepackage{algorithm}
\usepackage{algorithmic}
\makeatletter
\renewcommand{\ALG@name}{Algoritmo}
\makeatother

% disponi numeri di pagina
\usepackage{fancyhdr}
\fancyhf{} 
\fancyfoot[L]{\sffamily{\thepage}}

\makeatletter
\fancyhead[L]{\raisebox{1ex}[0pt][0pt]{\sffamily{\@title \ \@date}}} 
\fancyhead[R]{\raisebox{1ex}[0pt][0pt]{\sffamily{\@author}}}
\makeatother

\begin{document}
% sezione (data)
\section{Lezione del 27-11-24}

% stili pagina
\thispagestyle{empty}
\pagestyle{fancy}

% testo
\subsection{Letture e scritture nello spazio di memoria}
Durante la fase di fetch, abbiamo eseguito solo \textbf{letture} in memoria.
Vediamo adesso che nella fase \textbf{execute} abbiamo bisogno di effettuare sia letture che scritture.
Vediamo come si effettuano a livello di $\mu$-istruzioni queste letture e scritture, in maniera compatibile con le temporizzazioni già definite sulle memorie.

\subsubsection{Lettura}
Innanzitutto ripassiamo le temporizzazioni nel caso della lettura:
\begin{itemize}
	\item A indirizzi stabili arriva il comando \lstinline|/mr|;
	\item \lstinline|/s| arriva con ritardo;
	\item A \lstinline|/mr| e \lstinline|/s| bassi si effettua la lettura, cioè le tri-state vanno in conduzione;
	\item I multiplexer alle uscite vanno a regime dopo gli indirizzi, da qui in puoi i dati sono buoni e si prelevano;
	\item Quando \lstinline|/mr| torna a 1 i dati tornano ad alta impedenza, da lì in poi \lstinline|/s| e gli indirizzi possono tornare instabili.
\end{itemize}

Definiamo allora un $\mu$-programma per la lettura in memoria:
\begin{lstlisting}[language=verilog, style=codestyle]	
mem_r0: begin A23_A0 <= address; DIR <= 0; MR <= 0; STAR <= mem_r1; end
mem_r1: begin STAR <= mem_r2; end // stato di wait, da qui in poi omesso 
mem_r2: begin cpu_register <= d7_d0; MR <= 1; ...; end 
\end{lstlisting}
Notiamo che allo stato R2 si possono cambiare tutte le linee (indirizzo, ecc...) tranne che la DIR, in quanto impostando solo a quel punto MR non si è sicuri che la RAM risponda in tempo.

\subsubsection{Scrittura}
Ricordiamo che la scrittura è distruttiva.
Ricordiamo quindi le temporizzazioni:
\begin{itemize}
	\item Si abbassa \lstinline|/s| e ci si assicura che gli indirizzi siano stabili;
	\item Si abbassa \lstinline|/mr|;
	\item I dati dovranno essere corretti fino al fronte di salita di \lstinline|/mr|.
\end{itemize}

Definiamo un'altro $\mu$-programma, stavolta per la scrittura in memoria:
\begin{lstlisting}[language=verilog, style=codestyle]	
mem_w0: begin A23_A0 <= address; D7_D0 <= new_byte; DIR <= 1; STAR <= mem_w1; end
mem_w1: begin MW <= 0; STAR <= mem_w2; end
mem_w2: begin MW <= 1; STAR <= mem_w3; end
mem_w3: begin DIR <= 0; ...; end
\end{lstlisting}
Notiamo di non poter abbassare DIR o gli indirizzi fino allo stato W3, in quanto non si può essere sicuri che a quel punto la RAM abbia finito di scrivere.

\subsection{Letture e scritture nello spazio di I/O}
Le letture e le scritture nello spazio di I/O sono diverse, in quanto qui \textbf{la lettura è distruttiva}.
Inoltre, dobbiamo ricordarci di operare sui registri IOR e IOW anzichè MR e MW.

\subsubsection{Lettura}
Scriviamo quindi un $\mu$-programma per la lettura nello spazio di I/O dove teniamo conto di dover abbassare IOR \textbf{dopo} che gli indirizzi sono stabili, in maniera simile alla lettura:
\begin{lstlisting}[language=verilog, style=codestyle]	
io_r0: begin A23_A0 <= {H'00, offset}; DIR <= 0; STAR <= io_r1; end
io_r1: begin IOR <= 0; STAR <= io_r2; end
io_r2: begin STAR <= io_r3; end
io_r3: begin cpu_register <= d7_d0; IOR <= 1; ...; end
\end{lstlisting}

\subsubsection{Scrittura}
Ridefiniamo quindi il $\mu$-programma di scrittura:
\begin{lstlisting}[language=verilog, style=codestyle]	
io_w0: begin A23_A0 <= {H'00, offset}; D7_D0 <= new_byte; DIR <= 1; STAR <= io_w1; end
io_w1: begin IOW <= 0; STAR <= io_w2; end
io_w2: begin IOW <= 1; STAR <= io_w3; end
io_w3: begin DIR <= 0; ...; end
\end{lstlisting}

Notiamo che, in questo caso, la scrittura si fa sul fronte di discesa anziché di salita, e quindi l'assegnamento di \lstinline|D7_D0| al nuovo byte va fatto esclusivamente in W0, e non in W1 com'era possibile nella scrittura nello spazio di memoria. 

\subsection{Accessi multipli in memoria}
Il processore potrebbe fare accessi non solo ad un byte, ma 2 byte (per operandi su 16 bit) o 3 byte (per indirizzi).
Occasionalmente dovremo leggere anche 4 byte, ma questo non è considerato nel corso.

Per fare letture su locazioni multiple, si usano $\mu$-sottoprogrammi di lettura/scrittura modulari.
Utilizzeremo:
\begin{itemize}
	\item Il registro MJR per contenere il $\mu$-indirizzo di ritorno;
	\item Il registro NUMLOC come contatore del numero di byte da leggere/scrivere;
	\item Il registro A23\_A0 per contenere l'indirizzo del primo byte da leggere/scrivere;
	\item I registri APP0, ..., APP3 per contenere i byte letti/da scrivere.
\end{itemize}

\subsubsection{Lettura}
Vediamo quindi il $\mu$-programma principale di lettura:
\begin{lstlisting}[language=verilog, style=codestyle]	
s_x: begin ... A23_A0 <= address; MJR <= s_x+1; STAR <= subprogram; end
s_x+1: begin ... /* qui si usa APP0 */ end
\end{lstlisting}

Definiamo 4 $\mu$-sottoprogrammi:
\begin{itemize}
	\item readB: legge 1 byte;
	\item readW: legge 2 byte;
	\item readM: legge 3 byte;
	\item readL: legge 4 byte.
\end{itemize}

I parametri di ingresso saranno l'indirizzo in memoria della prima locazione e la DIR impostata a 0, i parametri di uscita i registri APP da 0 a 3 (che conterranno i byte letti).

Vediamo quindi i $\mu$-sottoprogrammi di lettura:
\begin{lstlisting}[language=verilog, style=codestyle]	
readB: begin MR <= 0; NUMLOC <= 1; STAR <= read0; end;
readW: begin MR <= 0; NUMLOC <= 2; STAR <= read0; end;
readM: begin MR <= 0; NUMLOC <= 3; STAR <= read0; end;
readL: begin MR <= 0; NUMLOC <= 4; STAR <= read0; end;

read0: begin APP0 <= d7_d0; A23_A0 <= A23_A0 + 1; NUMLOC <= NUMLOC - 1; STAR <= ( NUMLOC == 1 ) ? read4 : read1; end
read1: begin APP1 <= d7_d0; A23_A0 <= A23_A0 + 1; NUMLOC <= NUMLOC - 1; STAR <= ( NUMLOC == 1 ) ? read4 : read2; end
read2: begin APP2 <= d7_d0; A23_A0 <= A23_A0 + 1; NUMLOC <= NUMLOC - 1; STAR <= ( NUMLOC == 1 ) ? read4 : read3; end
read3: begin APP3 <= d7_d0; A23_A0 <= A23_A0 + 1; STAR <= read4; end
read4: begin MR <= 1; STAR <= MJR; end
\end{lstlisting}

\subsubsection{Scrittura}
Vediamo il $\mu$-programma principale di scrittura:
\begin{lstlisting}[language=verilog, style=codestyle]	
s: begin ... APP1 <= datum(15:8); APP0 <= datum(7:0); A23_A= <= address; MJR = s_x+1; STAR <= subprogram; end
\end{lstlisting}

E definiamo i 4 $\mu$-sottoprogrammi:
\begin{itemize}
	\item writeB: scrive 1 byte;
	\item writeW: scrive 2 byte;
	\item writeM: scrive 3 byte;
	\item writeL: scrive 4 byte.
\end{itemize}

I parametri di ingresso saranno l'indirizzo in memoria della prima locazione, la DIR impostata a 0, e i registri APP da 0 a 3 (che conterranno i byte da scrivere).

Implementiamo i $\mu$-sottoprogrammi come:
\begin{lstlisting}[language=verilog, style=codestyle]	
writeB: begin D7_D0 <= APP0; DIR <= 1; NUMLOC <= 1; STAR <= write0; end 
writeW: begin D7_D0 <= APP0; DIR <= 2; NUMLOC <= 1; STAR <= write0; end
writeM: begin D7_D0 <= APP0; DIR <= 3; NUMLOC <= 1; STAR <= write0; end
writeL: begin D7_D0 <= APP0; DIR <= 4; NUMLOC <= 1; STAR <= write0; end

write0: begin MW <= 0; STAR <= write1; end;

write1: begin MW <= 1; STAR <= ( NUMLOC == 1 ) ? write11 : write2; end
write2: begin D7_D0 = APP1; A23_A0 <= A23_A0 + 1; NUMLOC <= NUMLOC + 1; STAR <= write3; end
write3: begin MW <= 0; STAR <= write4; end

write4: begin MW <= 1; STAR <= ( NUMLOC == 1 ) ? write11 : write5; end
write5: begin D7_D0 = APP2; A23_A0 <= A23_A0 + 1; NUMLOC <= NUMLOC + 1; STAR <= write6; end
write6: begin MW <= 0; STAR <= write7; end

write7: begin MW <= 1; STAR <= ( NUMLOC == 1 ) ? write11 : write7; end
write8: begin D7_D0 = APP3; A23_A0 <= A23_A0 + 1; NUMLOC <= NUMLOC + 1; STAR <= write8; end
write9: begin MW <= 0; STAR <= write9; end

write10: begin MW <= 1; STAR <= write11; end

write11: begin DIR <= 0; STAR <= MJR; end
\end{lstlisting}

\end{document}


\documentclass[a4paper,11pt]{article}
\usepackage[a4paper, margin=8em]{geometry}

% usa i pacchetti per la scrittura in italiano
\usepackage[french,italian]{babel}
\usepackage[T1]{fontenc}
\usepackage[utf8]{inputenc}
\frenchspacing 

% usa i pacchetti per la formattazione matematica
\usepackage{amsmath, amssymb, amsthm, amsfonts}

% usa altri pacchetti
\usepackage{gensymb}
\usepackage{hyperref}
\usepackage{standalone}

\usepackage{colortbl}

\usepackage{xstring}
\usepackage{karnaugh-map}

% imposta il titolo
\title{Appunti Reti Logiche}
\author{Luca Seggiani}
\date{2024}

% imposta lo stile
% usa helvetica
\usepackage[scaled]{helvet}
% usa palatino
\usepackage{palatino}
% usa un font monospazio guardabile
\usepackage{lmodern}

\renewcommand{\rmdefault}{ppl}
\renewcommand{\sfdefault}{phv}
\renewcommand{\ttdefault}{lmtt}

% circuiti
\usepackage{circuitikz}
\usetikzlibrary{babel}

% testo cerchiato
\newcommand*\circled[1]{\tikz[baseline=(char.base)]{
            \node[shape=circle,draw,inner sep=2pt] (char) {#1};}}

% disponi il titolo
\makeatletter
\renewcommand{\maketitle} {
	\begin{center} 
		\begin{minipage}[t]{.8\textwidth}
			\textsf{\huge\bfseries \@title} 
		\end{minipage}%
		\begin{minipage}[t]{.2\textwidth}
			\raggedleft \vspace{-1.65em}
			\textsf{\small \@author} \vfill
			\textsf{\small \@date}
		\end{minipage}
		\par
	\end{center}

	\thispagestyle{empty}
	\pagestyle{fancy}
}
\makeatother

% disponi teoremi
\usepackage{tcolorbox}
\newtcolorbox[auto counter, number within=section]{theorem}[2][]{%
	colback=blue!10, 
	colframe=blue!40!black, 
	sharp corners=northwest,
	fonttitle=\sffamily\bfseries, 
	title=Teorema~\thetcbcounter: #2, 
	#1
}

% disponi definizioni
\newtcolorbox[auto counter, number within=section]{definition}[2][]{%
	colback=red!10,
	colframe=red!40!black,
	sharp corners=northwest,
	fonttitle=\sffamily\bfseries,
	title=Definizione~\thetcbcounter: #2,
	#1
}

% disponi codice
\usepackage{listings}
\usepackage[table]{xcolor}

\definecolor{codegreen}{rgb}{0,0.6,0}
\definecolor{codegray}{rgb}{0.5,0.5,0.5}
\definecolor{codepurple}{rgb}{0.58,0,0.82}
\definecolor{backcolour}{rgb}{0.95,0.95,0.92}

\lstdefinestyle{codestyle}{
		backgroundcolor=\color{black!5}, 
		commentstyle=\color{codegreen},
		keywordstyle=\bfseries\color{magenta},
		numberstyle=\sffamily\tiny\color{black!60},
		stringstyle=\color{green!50!black},
		basicstyle=\ttfamily\footnotesize,
		breakatwhitespace=false,         
		breaklines=true,                 
		captionpos=b,                    
		keepspaces=true,                 
		numbers=left,                    
		numbersep=5pt,                  
		showspaces=false,                
		showstringspaces=false,
		showtabs=false,                  
		tabsize=2
}

\lstdefinestyle{shellstyle}{
		backgroundcolor=\color{black!5}, 
		basicstyle=\ttfamily\footnotesize\color{black}, 
		commentstyle=\color{black}, 
		keywordstyle=\color{black},
		numberstyle=\color{black!5},
		stringstyle=\color{black}, 
		showspaces=false,
		showstringspaces=false, 
		showtabs=false, 
		tabsize=2, 
		numbers=none, 
		breaklines=true
}


\lstdefinelanguage{assembler}{ 
  keywords={AAA, AAD, AAM, AAS, ADC, ADCB, ADCW, ADCL, ADD, ADDB, ADDW, ADDL, AND, ANDB, ANDW, ANDL,
        ARPL, BOUND, BSF, BSFL, BSFW, BSR, BSRL, BSRW, BSWAP, BT, BTC, BTCB, BTCW, BTCL, BTR, 
        BTRB, BTRW, BTRL, BTS, BTSB, BTSW, BTSL, CALL, CBW, CDQ, CLC, CLD, CLI, CLTS, CMC, CMP,
        CMPB, CMPW, CMPL, CMPS, CMPSB, CMPSD, CMPSW, CMPXCHG, CMPXCHGB, CMPXCHGW, CMPXCHGL,
        CMPXCHG8B, CPUID, CWDE, DAA, DAS, DEC, DECB, DECW, DECL, DIV, DIVB, DIVW, DIVL, ENTER,
        HLT, IDIV, IDIVB, IDIVW, IDIVL, IMUL, IMULB, IMULW, IMULL, IN, INB, INW, INL, INC, INCB,
        INCW, INCL, INS, INSB, INSD, INSW, INT, INT3, INTO, INVD, INVLPG, IRET, IRETD, JA, JAE,
        JB, JBE, JC, JCXZ, JE, JECXZ, JG, JGE, JL, JLE, JMP, JNA, JNAE, JNB, JNBE, JNC, JNE, JNG,
        JNGE, JNL, JNLE, JNO, JNP, JNS, JNZ, JO, JP, JPE, JPO, JS, JZ, LAHF, LAR, LCALL, LDS,
        LEA, LEAVE, LES, LFS, LGDT, LGS, LIDT, LMSW, LOCK, LODSB, LODSD, LODSW, LOOP, LOOPE,
        LOOPNE, LSL, LSS, LTR, MOV, MOVB, MOVW, MOVL, MOVSB, MOVSD, MOVSW, MOVSX, MOVSXB,
        MOVSXW, MOVSXL, MOVZX, MOVZXB, MOVZXW, MOVZXL, MUL, MULB, MULW, MULL, NEG, NEGB, NEGW,
        NEGL, NOP, NOT, NOTB, NOTW, NOTL, OR, ORB, ORW, ORL, OUT, OUTB, OUTW, OUTL, OUTSB, OUTSD,
        OUTSW, POP, POPL, POPW, POPB, POPA, POPAD, POPF, POPFD, PUSH, PUSHL, PUSHW, PUSHB, PUSHA, 
				PUSHAD, PUSHF, PUSHFD, RCL, RCLB, RCLW, MOVSL, MOVSB, MOVSW, STOSL, STOSB, STOSW, LODSB, LODSW,
				LODSL, INSB, INSW, INSL, OUTSB, OUTSL, OUTSW
        RCLL, RCR, RCRB, RCRW, RCRL, RDMSR, RDPMC, RDTSC, REP, REPE, REPNE, RET, ROL, ROLB, ROLW,
        ROLL, ROR, RORB, RORW, RORL, SAHF, SAL, SALB, SALW, SALL, SAR, SARB, SARW, SARL, SBB,
        SBBB, SBBW, SBBL, SCASB, SCASD, SCASW, SETA, SETAE, SETB, SETBE, SETC, SETE, SETG, SETGE,
        SETL, SETLE, SETNA, SETNAE, SETNB, SETNBE, SETNC, SETNE, SETNG, SETNGE, SETNL, SETNLE,
        SETNO, SETNP, SETNS, SETNZ, SETO, SETP, SETPE, SETPO, SETS, SETZ, SGDT, SHL, SHLB, SHLW,
        SHLL, SHLD, SHR, SHRB, SHRW, SHRL, SHRD, SIDT, SLDT, SMSW, STC, STD, STI, STOSB, STOSD,
        STOSW, STR, SUB, SUBB, SUBW, SUBL, TEST, TESTB, TESTW, TESTL, VERR, VERW, WAIT, WBINVD,
        XADD, XADDB, XADDW, XADDL, XCHG, XCHGB, XCHGW, XCHGL, XLAT, XLATB, XOR, XORB, XORW, XORL},
  keywordstyle=\color{blue}\bfseries,
  ndkeywordstyle=\color{darkgray}\bfseries,
  identifierstyle=\color{black},
  sensitive=false,
  comment=[l]{\#},
  morecomment=[s]{/*}{*/},
  commentstyle=\color{purple}\ttfamily,
  stringstyle=\color{red}\ttfamily,
  morestring=[b]',
  morestring=[b]"
}

\lstset{language=assembler, style=codestyle}

% disponi sezioni
\usepackage{titlesec}

\titleformat{\section}
	{\sffamily\Large\bfseries} 
	{\thesection}{1em}{} 
\titleformat{\subsection}
	{\sffamily\large\bfseries}   
	{\thesubsection}{1em}{} 
\titleformat{\subsubsection}
	{\sffamily\normalsize\bfseries} 
	{\thesubsubsection}{1em}{}

% tikz
\usepackage{tikz}

% float
\usepackage{float}

% grafici
\usepackage{pgfplots}
\pgfplotsset{width=10cm,compat=1.9}

% disponi alberi
\usepackage{forest}

\forestset{
	rectstyle/.style={
		for tree={rectangle,draw,font=\large\sffamily}
	},
	roundstyle/.style={
		for tree={circle,draw,font=\large}
	}
}

% disponi algoritmi
\usepackage{algorithm}
\usepackage{algorithmic}
\makeatletter
\renewcommand{\ALG@name}{Algoritmo}
\makeatother

% disponi numeri di pagina
\usepackage{fancyhdr}
\fancyhf{} 
\fancyfoot[L]{\sffamily{\thepage}}

\makeatletter
\fancyhead[L]{\raisebox{1ex}[0pt][0pt]{\sffamily{\@title \ \@date}}} 
\fancyhead[R]{\raisebox{1ex}[0pt][0pt]{\sffamily{\@author}}}
\makeatother

\begin{document}
% sezione (data)
\section{Lezione del 28-11-24}

% stili pagina
\thispagestyle{empty}
\pagestyle{fancy}

% testo
\subsection{Descrizione in Verilog del ciclo fetch/execute}
\subsubsection{Fase di fetch} 

# tutto verilog

Alla fine della fase di fetch saremo riusciti con successo a mettere:
\begin{itemize}
	\item Il codice operativo dell'istruzione in OPCODE;
	\item L'operando immediato o in memoria dell'istruzione in SOURCE;
	\item L'operando destinatario in DEST\_ADDR;
	\item IP sulla prossima istruzione da prelevare.
\end{itemize}

\subsubsection{Fase di esecuzion}
Nella fase di esecuzione, avremo quindi tutti gli operandi già inizializzati, e dovremo solo farli passare attraverso apposite reti combinatorie.

# altro verilog

\subsubsection{Formato F1}
Abbiamo visto come un'eccezione tra le istruzioni è rappresentata da quele in formato F1, in quanto il "fetch" effettivo dei loro operandi sorgenti e destinatari va fatto in fase di esecuzione.
Queste, ricordiamo, sono le istruzioni di I/O (operandi su indirizzi a 16 bit) e MOV sui registri da 24 bit.

# indovina? altro verilog

\subsection{Interfacce}
Veniamo adesso alla descrizione di interfacce che completano il calcolatore, cioè gli permettono di comunicare col mondo esterno.
Le interfacce possono essere di due tipi principali:
\begin{itemize}
	\item \textbf{Parallele}: un byte alla volta (quindi più bit \textit{in parallelo});
	\item \textbf{Seriali}: un bit alla volta.
\end{itemize}
Vedremo poi anche le interfacce per la conversione da \textbf{analogico a digitale} e viceversa, che trasformano da tensioni a gruppi di bit.

I collegamenti lato bus delle interfacce, come avevamo anticipato sono sempre uguali, mentre cambiano sul lato dispositivo.

\subsubsection{Visione funzionale di un interfaccia}
La visione funzionale di un interfaccia è quella dal punto di vista di chi deve interagirci, cioè come un insieme di registri su cui opererà il \textbf{procressore}:
\begin{itemize}
	\item \textbf{Receive Buffer Register (RBR)}: registro dove si vanno a \textit{leggere} informazioni \textbf{dall'interfaccia};
	\item \textbf{Transmit Buffer Register (TBR)}: registro dove si vanno a \textit{scrivere} informazioni \textbf{all'interfaccia}.
\end{itemize}

\subsubsection{Sincronizzazione processore-dispositivi}
Eseguendo un programma che contiene sequenze di istruzioni \lstinline|IN| o \lstinline|OUT|, il processore non può sapere se fra una \lstinline|IN| e l'altra (o fra una \lstinline|OUT| e l'altra) il dispositivo ha prodotto nuovi dati (o se ha processato quelli inviati).
Dovremo quindi implementare un doppio handshake, sia sul lato processore (\textit{handshake "software"}) che sul lato hardware (\textit{handshake "hardware"}).

\par\smallskip 

Dotiamo quindi le interfacce di registri di stato:
\begin{itemize}
	\item \textbf{Receive Status Register (RSR)}: contiene un bit di interesse, il flag \textbf{FI} di \textbf{ingresso pieno};
	\item \textbf{Transmit Status Register (TSR)}: contiene un altro bit di interesse, il flag \textbf{FO} di \textbf{uscita vuota}. 
\end{itemize}

I due flag FI e FO vengono controllati dall'interfaccia, e quindi impostati a 1 o a 0 quando questa rileva le condizioni opportune.

\subsubsection{Ingresso dati a controllo di programma}
Vediamo quindi un ciclo di ingresso dati.
Si parte con FI a 0.
Quando il dispositivo gestito dall'interfaccia scrive in RBR, l'interfaccia mette FI a 1. Questo segnala al processore che c'è un nuovo dato.
A questo punto, quando il processore accede in lettura al registro RBR, l'interfaccia riporta FI a 0.

Notiamo che su due letture consecutive il processore è in \textbf{attesa attiva} finché non FI non si alza nuovamente.
Esistono altri metodi di accesso in memoria che non richiedono l'attesa attiva da parte del processore, fra cui il meccanismo degli \textit{interrupt} e il \textit{DMA (Direct Memory Access))}.

\subsubsection{Uscita dati a controllo di programma}
Vediamo adesso un ciclo di uscita dati.
Il flag FO parte a 0.
L'interfaccia lo mette a 0 quando il processore scrive in TBR, per segnalare che il dispositivo non ha ancora elaborato.
Quando il dispositivo accede a TBR per la lettura, FO torna a 0.


\end{document}


\documentclass[a4paper,11pt]{article}
\usepackage[a4paper, margin=8em]{geometry}

% usa i pacchetti per la scrittura in italiano
\usepackage[french,italian]{babel}
\usepackage[T1]{fontenc}
\usepackage[utf8]{inputenc}
\frenchspacing 

% usa i pacchetti per la formattazione matematica
\usepackage{amsmath, amssymb, amsthm, amsfonts}

% usa altri pacchetti
\usepackage{gensymb}
\usepackage{hyperref}
\usepackage{standalone}

\usepackage{colortbl}

\usepackage{xstring}
\usepackage{karnaugh-map}

% imposta il titolo
\title{Appunti Reti Logiche}
\author{Luca Seggiani}
\date{2024}

% imposta lo stile
% usa helvetica
\usepackage[scaled]{helvet}
% usa palatino
\usepackage{palatino}
% usa un font monospazio guardabile
\usepackage{lmodern}

\renewcommand{\rmdefault}{ppl}
\renewcommand{\sfdefault}{phv}
\renewcommand{\ttdefault}{lmtt}

% circuiti
\usepackage{circuitikz}
\usetikzlibrary{babel}

% testo cerchiato
\newcommand*\circled[1]{\tikz[baseline=(char.base)]{
            \node[shape=circle,draw,inner sep=2pt] (char) {#1};}}

% disponi il titolo
\makeatletter
\renewcommand{\maketitle} {
	\begin{center} 
		\begin{minipage}[t]{.8\textwidth}
			\textsf{\huge\bfseries \@title} 
		\end{minipage}%
		\begin{minipage}[t]{.2\textwidth}
			\raggedleft \vspace{-1.65em}
			\textsf{\small \@author} \vfill
			\textsf{\small \@date}
		\end{minipage}
		\par
	\end{center}

	\thispagestyle{empty}
	\pagestyle{fancy}
}
\makeatother

% disponi teoremi
\usepackage{tcolorbox}
\newtcolorbox[auto counter, number within=section]{theorem}[2][]{%
	colback=blue!10, 
	colframe=blue!40!black, 
	sharp corners=northwest,
	fonttitle=\sffamily\bfseries, 
	title=Teorema~\thetcbcounter: #2, 
	#1
}

% disponi definizioni
\newtcolorbox[auto counter, number within=section]{definition}[2][]{%
	colback=red!10,
	colframe=red!40!black,
	sharp corners=northwest,
	fonttitle=\sffamily\bfseries,
	title=Definizione~\thetcbcounter: #2,
	#1
}

% disponi codice
\usepackage{listings}
\usepackage[table]{xcolor}

\definecolor{codegreen}{rgb}{0,0.6,0}
\definecolor{codegray}{rgb}{0.5,0.5,0.5}
\definecolor{codepurple}{rgb}{0.58,0,0.82}
\definecolor{backcolour}{rgb}{0.95,0.95,0.92}

\lstdefinestyle{codestyle}{
		backgroundcolor=\color{black!5}, 
		commentstyle=\color{codegreen},
		keywordstyle=\bfseries\color{magenta},
		numberstyle=\sffamily\tiny\color{black!60},
		stringstyle=\color{green!50!black},
		basicstyle=\ttfamily\footnotesize,
		breakatwhitespace=false,         
		breaklines=true,                 
		captionpos=b,                    
		keepspaces=true,                 
		numbers=left,                    
		numbersep=5pt,                  
		showspaces=false,                
		showstringspaces=false,
		showtabs=false,                  
		tabsize=2
}

\lstdefinestyle{shellstyle}{
		backgroundcolor=\color{black!5}, 
		basicstyle=\ttfamily\footnotesize\color{black}, 
		commentstyle=\color{black}, 
		keywordstyle=\color{black},
		numberstyle=\color{black!5},
		stringstyle=\color{black}, 
		showspaces=false,
		showstringspaces=false, 
		showtabs=false, 
		tabsize=2, 
		numbers=none, 
		breaklines=true
}


\lstdefinelanguage{assembler}{ 
  keywords={AAA, AAD, AAM, AAS, ADC, ADCB, ADCW, ADCL, ADD, ADDB, ADDW, ADDL, AND, ANDB, ANDW, ANDL,
        ARPL, BOUND, BSF, BSFL, BSFW, BSR, BSRL, BSRW, BSWAP, BT, BTC, BTCB, BTCW, BTCL, BTR, 
        BTRB, BTRW, BTRL, BTS, BTSB, BTSW, BTSL, CALL, CBW, CDQ, CLC, CLD, CLI, CLTS, CMC, CMP,
        CMPB, CMPW, CMPL, CMPS, CMPSB, CMPSD, CMPSW, CMPXCHG, CMPXCHGB, CMPXCHGW, CMPXCHGL,
        CMPXCHG8B, CPUID, CWDE, DAA, DAS, DEC, DECB, DECW, DECL, DIV, DIVB, DIVW, DIVL, ENTER,
        HLT, IDIV, IDIVB, IDIVW, IDIVL, IMUL, IMULB, IMULW, IMULL, IN, INB, INW, INL, INC, INCB,
        INCW, INCL, INS, INSB, INSD, INSW, INT, INT3, INTO, INVD, INVLPG, IRET, IRETD, JA, JAE,
        JB, JBE, JC, JCXZ, JE, JECXZ, JG, JGE, JL, JLE, JMP, JNA, JNAE, JNB, JNBE, JNC, JNE, JNG,
        JNGE, JNL, JNLE, JNO, JNP, JNS, JNZ, JO, JP, JPE, JPO, JS, JZ, LAHF, LAR, LCALL, LDS,
        LEA, LEAVE, LES, LFS, LGDT, LGS, LIDT, LMSW, LOCK, LODSB, LODSD, LODSW, LOOP, LOOPE,
        LOOPNE, LSL, LSS, LTR, MOV, MOVB, MOVW, MOVL, MOVSB, MOVSD, MOVSW, MOVSX, MOVSXB,
        MOVSXW, MOVSXL, MOVZX, MOVZXB, MOVZXW, MOVZXL, MUL, MULB, MULW, MULL, NEG, NEGB, NEGW,
        NEGL, NOP, NOT, NOTB, NOTW, NOTL, OR, ORB, ORW, ORL, OUT, OUTB, OUTW, OUTL, OUTSB, OUTSD,
        OUTSW, POP, POPL, POPW, POPB, POPA, POPAD, POPF, POPFD, PUSH, PUSHL, PUSHW, PUSHB, PUSHA, 
				PUSHAD, PUSHF, PUSHFD, RCL, RCLB, RCLW, MOVSL, MOVSB, MOVSW, STOSL, STOSB, STOSW, LODSB, LODSW,
				LODSL, INSB, INSW, INSL, OUTSB, OUTSL, OUTSW
        RCLL, RCR, RCRB, RCRW, RCRL, RDMSR, RDPMC, RDTSC, REP, REPE, REPNE, RET, ROL, ROLB, ROLW,
        ROLL, ROR, RORB, RORW, RORL, SAHF, SAL, SALB, SALW, SALL, SAR, SARB, SARW, SARL, SBB,
        SBBB, SBBW, SBBL, SCASB, SCASD, SCASW, SETA, SETAE, SETB, SETBE, SETC, SETE, SETG, SETGE,
        SETL, SETLE, SETNA, SETNAE, SETNB, SETNBE, SETNC, SETNE, SETNG, SETNGE, SETNL, SETNLE,
        SETNO, SETNP, SETNS, SETNZ, SETO, SETP, SETPE, SETPO, SETS, SETZ, SGDT, SHL, SHLB, SHLW,
        SHLL, SHLD, SHR, SHRB, SHRW, SHRL, SHRD, SIDT, SLDT, SMSW, STC, STD, STI, STOSB, STOSD,
        STOSW, STR, SUB, SUBB, SUBW, SUBL, TEST, TESTB, TESTW, TESTL, VERR, VERW, WAIT, WBINVD,
        XADD, XADDB, XADDW, XADDL, XCHG, XCHGB, XCHGW, XCHGL, XLAT, XLATB, XOR, XORB, XORW, XORL},
  keywordstyle=\color{blue}\bfseries,
  ndkeywordstyle=\color{darkgray}\bfseries,
  identifierstyle=\color{black},
  sensitive=false,
  comment=[l]{\#},
  morecomment=[s]{/*}{*/},
  commentstyle=\color{purple}\ttfamily,
  stringstyle=\color{red}\ttfamily,
  morestring=[b]',
  morestring=[b]"
}

\lstset{language=assembler, style=codestyle}

% disponi sezioni
\usepackage{titlesec}

\titleformat{\section}
	{\sffamily\Large\bfseries} 
	{\thesection}{1em}{} 
\titleformat{\subsection}
	{\sffamily\large\bfseries}   
	{\thesubsection}{1em}{} 
\titleformat{\subsubsection}
	{\sffamily\normalsize\bfseries} 
	{\thesubsubsection}{1em}{}

% tikz
\usepackage{tikz}

% float
\usepackage{float}

% grafici
\usepackage{pgfplots}
\pgfplotsset{width=10cm,compat=1.9}

% disponi alberi
\usepackage{forest}

\forestset{
	rectstyle/.style={
		for tree={rectangle,draw,font=\large\sffamily}
	},
	roundstyle/.style={
		for tree={circle,draw,font=\large}
	}
}

% disponi algoritmi
\usepackage{algorithm}
\usepackage{algorithmic}
\makeatletter
\renewcommand{\ALG@name}{Algoritmo}
\makeatother

% disponi numeri di pagina
\usepackage{fancyhdr}
\fancyhf{} 
\fancyfoot[L]{\sffamily{\thepage}}

\makeatletter
\fancyhead[L]{\raisebox{1ex}[0pt][0pt]{\sffamily{\@title \ \@date}}} 
\fancyhead[R]{\raisebox{1ex}[0pt][0pt]{\sffamily{\@author}}}
\makeatother

\begin{document}
% sezione (data)
\section{Lezione del 03-12-24}

% stili pagina
\thispagestyle{empty}
\pagestyle{fancy}

% testo
\subsection{Interfacce parallele}
\subsubsection{Interfacce di ingresso senza handshake}
Iniziamo a vedere le interfacce parallele di \textbf{ingresso} senza handshake.
Queste scambiano interi byte col processore, attraverso un solo registro.
Non c'è quindi nessuna linea di indirizzo, l'intero registro va direttamente al processore su una linea dati.
Abbiamo poi la linea di select \lstinline|/s| e la linea di I/O read \lstinline|/ior|.

Il registro in uscita è forchettato da una tri-state in modo da mantenere ad alta impedenza l'uscita del registro RBR quando il processore non sta leggendo dall'interfaccia. 
Inoltre, non vorremo nemmeno che il registro RBR si trovi a leggere dati quando il processore non sta leggendo.

Possiamo quindi ricavare, dal select e l'I/O read attraverso una porta NOR, un segnale di enable sia per la porta tri state che per il registro: quando entrambi sono bassi, si legge dal lato dispositivo dell interfaccia all'interno del registro, e si restituisce l'uscita del registro al processore.

Notiamo che l'aggiornamento dell'RBR avviene al \textit{fronte di salita} dell'enabler (quindi di \lstinline|/ior|), quindi una volta sola per ogni lettura.

\subsubsection{Interfacce di uscita senza handshake}
L'interfaccia parallela di \textbf{uscita} senza handshake è simile: si ha sempre un solo registro, TBR, che memorizza le linee dati in entrata qundo sono entrambi bassi il select e l'I/O write (\lstinline|/iow|), cosa ricavata attraverso un'altra porta NOR.
Notiamo che in questo caso TBR campiona sul \textit{fronte di discesa} dell'enabler (quindi di \lstinline|/iow|), anziché di salita.

Vediamo la descrizione Verilog:
\lstinputlisting[language=verilog, style=codestyle]{../verilog/12-03/parallel_interfaces/parallel_in.v}

\subsubsection{Interfacce di ingresso/uscita senza handshake}
Le intefacce di ingresso/uscita senza handshake si costruiscono unendo due interfacce, una di ingresso e una di uscita (e appunto dette \textbf{sottointerfacce} di ingresso e uscita), e selezionando l'interfaccia giusta attraverso un bit di indirizzo (\lstinline|a0|).
Il select per la selezione viene generato da una semplice rete combinatoria che prende in ingresso il \lstinline|/s| globale e il bit di indirizzo, con tabella di verità:
\begin{table}[h!]
	\center 
	\begin{tabular} { c  c | c  c }
		\lstinline|/s| & \lstinline|/a0| & \lstinline|/si| & \lstinline|/so| \\ 
		\hline 
		1 & - & 1 & 1 \\ 
		0 & 0 & 0 & 1 \\ 
		0 & 1 & 1 & 0 
	\end{tabular}
\end{table}
Dove \lstinline|/si| è il select della sottointerfaccia di ingresso, e \lstinline|/so| il select della sottointerfaccia di uscita.
Come vediamo dall'esempio, solitamente si mette a indirizzo \textbf{pari} la porta di \textit{ingresso}, e a indirizzo \textbf{dispari} la porta di \textit{uscita}.

Notiamo che un montaggio alternativo si ottiene ignorando il bit di indirizzo e accedendo direttamente alle due sottointerfacce con un unico select.
A questo punto sarà compito del processore discriminare fra \lstinline|/ior| e \lstinline|/iow| per selezionare la sottointerfaccia desiderata.

Vediamo la descrizione Verilog:
\lstinputlisting[language=verilog, style=codestyle]{../verilog/12-03/parallel_interfaces/parallel_in.v}

\subsubsection{Interfacce di ingresso/uscita}
Possiamo combinare le interfacce senza handshake viste finora in un unica interfaccia di ingresso/uscita.
In questo caso i registri RBR e TBR verranno unificati, a quanto visto dal processore, in un unico registro RTBR, mentre i registri di stato RSR e TSR verranno unificati in un unico RTSR.

In Verilog, mostrando un esempio con calcolo esplicito dei select (sulla base dell'LSB dell'offset) e uno a connessione diretta: 
\lstinputlisting[language=verilog, style=codestyle]{../verilog/12-03/parallel_interfaces/parallel_inout.v}

\subsubsection{Interfacce di ingresso con handshake}
Ricordiamo la visione funzionale delle interfacce di ingresso con handshake: dovremo avere i registri RBR e RSR, da cui ricaviamo il flag FI lato processore, mentre lato dispositivo dovremo avere sia le linee di ingresso dati che le linee di handshake, che assumiamo essere in forma \lstinline|/dav| e \lstinline|rfd|.

L'interfaccia si implementa quindi come una combinazione di una rete combinatoria, come avevamo visto per le interfacce senza handshake, per la generazione degli enable, e una rete sequenziale per la gestione degli handshake.

Il processore potrà accedere in lettura sia al RBR che al RSR: questo si discrimina attraeverso un bit di indirizzo (\lstinline|a0|).
Si ha quindi la rete combinatoria per la generazione degli enable, dove \lstinline|eb| è l'enable del buffer e \lstinline|es| è l'enable del registro di stato:
\begin{table}[h!]
	\center 
	\begin{tabular} { c  c  c | c  c }
		\lstinline|/s| & \lstinline|/ior| & \lstinline|/a0| & \lstinline|/es| & \lstinline|/eb| \\ 
		\hline 
		0 & 0 & 0 & 1 & 0 \\ 
		0 & 0 & 1 & 0 & 1 \\ 
		... & & & 0 & 0 
	\end{tabular}
\end{table}

Vediamo la descrizione Verilog:
\lstinputlisting[language=verilog, style=codestyle]{../verilog/12-03/parallel_interfaces/hs_parallel_in.v}

Notiamo che, nella sintesi, si ritarda il clock dell'interfaccia rispetto a quello del processore per evitare condizioni di \textit{corsa} all'interfaccia.

\subsubsection{Interfacce di uscita con handhsake}
L'interfaccia di uscita è realizzata in modo analogo: includiamo un registro TSR che contiene il flag FO lato processore, il registro TBR, e lato dispositivo le linee \lstinline|/dav| e \lstinline|rfd|, dove però è l'interfaccia, e non più il processore, a fare da produttore.

La struttura interna sarà del tutto simile a l'interfaccia di ingresso, con una parte combinatoria che si occupa degli enable e una parte sequenziale che si occupa degli handshake.
La parte combinatoria obbedirà alla tabella di verità:
\begin{table}[h!]
	\center 
	\begin{tabular} { c  c  c  c | c  c }
		\lstinline|/s| & \lstinline|/ior| & \lstinline|/iow| & \lstinline|/a0| & \lstinline|/es| & \lstinline|/eb| \\ 
		\hline 
		0 & 0 & 1 & 0 & 1 & 0 \\ 
		0 & 1 & 0 & 1 & 0 & 1 \\ 
		... & & & & 0 & 0 
	\end{tabular}
\end{table}

Notiamo che compare comunque la linea di uscita in quanto vogliamo leggere lo stato del TSR.

Vediamo la descrizione Verilog:
\lstinputlisting[language=verilog, style=codestyle]{../verilog/12-03/parallel_interfaces/hs_parallel_out.v}

\subsubsection{Interfacce di ingresso/uscita con handshake}
Possiamo combinare le interfacce con handhshake viste finora in un unica interfaccia di ingresso/uscita.
Combineremo il registro di stato in un unico registro RTSR dotato dei flag FO e FI, e useremo un bit di indirizzo (\lstinline|a0|) per distinguere fra le porte di ingresso e uscita. 

In Verilog: 
\lstinputlisting[language=verilog, style=codestyle]{../verilog/12-03/parallel_interfaces/hs_parallel_inout.v}

\subsection{Interfacce seriali}
Le interfacce seriali trasmettono informazioni un bit a volta.
Noi ne consideriamo una versione semplificata, l'interfaccia seriale start/stop.

Dal punto di vista fisico, un interfaccia seriale trasporta informazioni su due linee:
\begin{itemize}
	\item La linea di \textbf{massa}, che funge da riferimento;
	\item La linea \textbf{segnale}, che porta appunto il segnale riferito a massa.
\end{itemize}

Dal punto di vista logico, invece, a interessarci sarà solo la linea segnale.
Questa trasporta informazioni \textit{half duplex}, cioè da un trasmettitore a un ricevitore (la comunicazione nelle due direzioni richiede quindi due linee).

Ora, se il segnale è rappresentato da una sequenza di \textbf{marking} (1, linea in tensione) e \textbf{spacing} (0, linea a massa) trasmessi con un periodo $T$, il problema diventerà sincronizzare trasmettitore e ricevitore in modo che possano comunicare efficientemente. 

Quello che ci interesserà stabilire sarà quindi il \textbf{tempo di bit} $T$ e la \textbf{trama} del segnale.
Definiamo trama una sequenza di bit che rappresenta un frammento di comunicazione: iniziamo la trama con uno spacing (\textbf{bit di start}), e seguiamo poi con un numero che va da 5 a 8, stabilito in precedenza, di \textbf{bit utili} (solitamente LSB).
Infine, trasmettiamo un marking per segnalare la fine della trama (\textbf{bit di stop}).

I bit di start e di stop rappresentano un \textbf{overhead}: una trama di $n$ bit utili è lunga almeno $n+2$ per segnalare inizio e fine della trama.
Avremo quindi che la velocità netta della linea è $\frac{n}{n+2}$, che è comunque più efficiente di usare un clock secondario o effettuare un handshake per ogni bit. 

Converrebbe quindi usare $n$ arbitrariamente lunghi: purtroppo questo è fattibile fino ad un certo limite superiore, in quanto i clock di trasmettitore e ricevitore saranno necessariamente leggermente differenti in frequenza, ergo sul lungo termine si andrebbe ad accumulare un'errore troppo grande.

Infatti, l'impedenza dalla linea di trasmissione determinerà uno "smussamento" del segnale, motivo per cui preferiremo campionare ogni bit trasmesso nella posizione più centrale possibile rispetto alla sua forma d'onda.
Questo significherà che, posto $T$ il periodo del clock del trasmettitore, e $T + \Delta T$ il periodo del clock del ricevitore, vorremo:
$$
n \cdot \Delta T \leq \frac{T}{2} \implies \frac{\Delta T}{T} \leq \frac{1}{2n}
$$

Uno standard di \textit{interoperabilità} per le trasmissioni seriali è l EIA-RS232C, che fissa lo 0 logico a tensioni da 3V a 25V, e l'1 logico a tensione negativa da -25V a -3V.

Vediamo quindi un implementazione Verilog, usando un clock al ricevitore con periodo $T + \Delta T = \frac{T}{16}$ per trasmissioni di trame da 10 bit (8 bit utili).

Il trasmettitore sarà il seguente:
\lstinputlisting[language=verilog, style=codestyle]{../verilog/12-03/serial_interfaces/serial_transmitter.v}

Il ricevitore, invece, sarà il seguente:
\lstinputlisting[language=verilog, style=codestyle]{../verilog/12-03/serial_interfaces/serial_receiver.v}

\subsection{Vista funzionale delle interfacce seriali}
L'interfaccia seriale è effettivamente un'interfaccia di ingresso uscita con handshake, dove ingressi e uscite sono i bit \lstinline|txd| di trasmissione e \lstinline|rxd| di ricezione.
In particolare, questi sono gestiti rispettivamente da due sottointerfacce dette \textbf{ricevitore} e \textbf{trasmettitore}, sincronizzati da un \textbf{generatore di segnali di sincronizzazione} e interfacciati con una sottointerfaccia di gestione attraverso handshake \lstinline|dav| - \lstinline|/rfd|.
Nell'handshake lato sottointerfaccia di gestione, il \textbf{ricevitore} è \textit{produttore} e il \textbf{trasmettitore} è il \textit{ricevitore}, cioè il ricevitore \textit{restituisce} i dati ricevuti e il trasmettitore \textit{prende} i dati da trasmettere.
Lato processore si hanno comunque 8 bit di uscita, cioè un byte, e il bit di indirizzo che discrimina fra porte di ingresso / uscita (cioè fra ricevitore e trasmettitore).

L'intera interfaccia, ricavata a partire da un interfaccia di ingresso uscita parallela con handshake (che approvigiona un trasmettitore e un ricevitore seriali), sarà quindi descritta in Verilog come:
\lstinputlisting[language=verilog, style=codestyle]{../verilog/12-03/serial_interfaces/serial_interface.v}
sulla base delle definizioni di \lstinline|serial_receiver| e \lstinline|serial_transmitter| date alla scorsa lezione.

\end{document}


\documentclass[a4paper,11pt]{article}
\usepackage[a4paper, margin=8em]{geometry}

% usa i pacchetti per la scrittura in italiano
\usepackage[french,italian]{babel}
\usepackage[T1]{fontenc}
\usepackage[utf8]{inputenc}
\frenchspacing 

% usa i pacchetti per la formattazione matematica
\usepackage{amsmath, amssymb, amsthm, amsfonts}

% usa altri pacchetti
\usepackage{gensymb}
\usepackage{hyperref}
\usepackage{standalone}

\usepackage{colortbl}

\usepackage{xstring}
\usepackage{karnaugh-map}

% imposta il titolo
\title{Appunti Reti Logiche}
\author{Luca Seggiani}
\date{2024}

% imposta lo stile
% usa helvetica
\usepackage[scaled]{helvet}
% usa palatino
\usepackage{palatino}
% usa un font monospazio guardabile
\usepackage{lmodern}

\renewcommand{\rmdefault}{ppl}
\renewcommand{\sfdefault}{phv}
\renewcommand{\ttdefault}{lmtt}

% circuiti
\usepackage{circuitikz}
\usetikzlibrary{babel}

% testo cerchiato
\newcommand*\circled[1]{\tikz[baseline=(char.base)]{
            \node[shape=circle,draw,inner sep=2pt] (char) {#1};}}

% disponi il titolo
\makeatletter
\renewcommand{\maketitle} {
	\begin{center} 
		\begin{minipage}[t]{.8\textwidth}
			\textsf{\huge\bfseries \@title} 
		\end{minipage}%
		\begin{minipage}[t]{.2\textwidth}
			\raggedleft \vspace{-1.65em}
			\textsf{\small \@author} \vfill
			\textsf{\small \@date}
		\end{minipage}
		\par
	\end{center}

	\thispagestyle{empty}
	\pagestyle{fancy}
}
\makeatother

% disponi teoremi
\usepackage{tcolorbox}
\newtcolorbox[auto counter, number within=section]{theorem}[2][]{%
	colback=blue!10, 
	colframe=blue!40!black, 
	sharp corners=northwest,
	fonttitle=\sffamily\bfseries, 
	title=Teorema~\thetcbcounter: #2, 
	#1
}

% disponi definizioni
\newtcolorbox[auto counter, number within=section]{definition}[2][]{%
	colback=red!10,
	colframe=red!40!black,
	sharp corners=northwest,
	fonttitle=\sffamily\bfseries,
	title=Definizione~\thetcbcounter: #2,
	#1
}

% disponi codice
\usepackage{listings}
\usepackage[table]{xcolor}

\definecolor{codegreen}{rgb}{0,0.6,0}
\definecolor{codegray}{rgb}{0.5,0.5,0.5}
\definecolor{codepurple}{rgb}{0.58,0,0.82}
\definecolor{backcolour}{rgb}{0.95,0.95,0.92}

\lstdefinestyle{codestyle}{
		backgroundcolor=\color{black!5}, 
		commentstyle=\color{codegreen},
		keywordstyle=\bfseries\color{magenta},
		numberstyle=\sffamily\tiny\color{black!60},
		stringstyle=\color{green!50!black},
		basicstyle=\ttfamily\footnotesize,
		breakatwhitespace=false,         
		breaklines=true,                 
		captionpos=b,                    
		keepspaces=true,                 
		numbers=left,                    
		numbersep=5pt,                  
		showspaces=false,                
		showstringspaces=false,
		showtabs=false,                  
		tabsize=2
}

\lstdefinestyle{shellstyle}{
		backgroundcolor=\color{black!5}, 
		basicstyle=\ttfamily\footnotesize\color{black}, 
		commentstyle=\color{black}, 
		keywordstyle=\color{black},
		numberstyle=\color{black!5},
		stringstyle=\color{black}, 
		showspaces=false,
		showstringspaces=false, 
		showtabs=false, 
		tabsize=2, 
		numbers=none, 
		breaklines=true
}


\lstdefinelanguage{assembler}{ 
  keywords={AAA, AAD, AAM, AAS, ADC, ADCB, ADCW, ADCL, ADD, ADDB, ADDW, ADDL, AND, ANDB, ANDW, ANDL,
        ARPL, BOUND, BSF, BSFL, BSFW, BSR, BSRL, BSRW, BSWAP, BT, BTC, BTCB, BTCW, BTCL, BTR, 
        BTRB, BTRW, BTRL, BTS, BTSB, BTSW, BTSL, CALL, CBW, CDQ, CLC, CLD, CLI, CLTS, CMC, CMP,
        CMPB, CMPW, CMPL, CMPS, CMPSB, CMPSD, CMPSW, CMPXCHG, CMPXCHGB, CMPXCHGW, CMPXCHGL,
        CMPXCHG8B, CPUID, CWDE, DAA, DAS, DEC, DECB, DECW, DECL, DIV, DIVB, DIVW, DIVL, ENTER,
        HLT, IDIV, IDIVB, IDIVW, IDIVL, IMUL, IMULB, IMULW, IMULL, IN, INB, INW, INL, INC, INCB,
        INCW, INCL, INS, INSB, INSD, INSW, INT, INT3, INTO, INVD, INVLPG, IRET, IRETD, JA, JAE,
        JB, JBE, JC, JCXZ, JE, JECXZ, JG, JGE, JL, JLE, JMP, JNA, JNAE, JNB, JNBE, JNC, JNE, JNG,
        JNGE, JNL, JNLE, JNO, JNP, JNS, JNZ, JO, JP, JPE, JPO, JS, JZ, LAHF, LAR, LCALL, LDS,
        LEA, LEAVE, LES, LFS, LGDT, LGS, LIDT, LMSW, LOCK, LODSB, LODSD, LODSW, LOOP, LOOPE,
        LOOPNE, LSL, LSS, LTR, MOV, MOVB, MOVW, MOVL, MOVSB, MOVSD, MOVSW, MOVSX, MOVSXB,
        MOVSXW, MOVSXL, MOVZX, MOVZXB, MOVZXW, MOVZXL, MUL, MULB, MULW, MULL, NEG, NEGB, NEGW,
        NEGL, NOP, NOT, NOTB, NOTW, NOTL, OR, ORB, ORW, ORL, OUT, OUTB, OUTW, OUTL, OUTSB, OUTSD,
        OUTSW, POP, POPL, POPW, POPB, POPA, POPAD, POPF, POPFD, PUSH, PUSHL, PUSHW, PUSHB, PUSHA, 
				PUSHAD, PUSHF, PUSHFD, RCL, RCLB, RCLW, MOVSL, MOVSB, MOVSW, STOSL, STOSB, STOSW, LODSB, LODSW,
				LODSL, INSB, INSW, INSL, OUTSB, OUTSL, OUTSW
        RCLL, RCR, RCRB, RCRW, RCRL, RDMSR, RDPMC, RDTSC, REP, REPE, REPNE, RET, ROL, ROLB, ROLW,
        ROLL, ROR, RORB, RORW, RORL, SAHF, SAL, SALB, SALW, SALL, SAR, SARB, SARW, SARL, SBB,
        SBBB, SBBW, SBBL, SCASB, SCASD, SCASW, SETA, SETAE, SETB, SETBE, SETC, SETE, SETG, SETGE,
        SETL, SETLE, SETNA, SETNAE, SETNB, SETNBE, SETNC, SETNE, SETNG, SETNGE, SETNL, SETNLE,
        SETNO, SETNP, SETNS, SETNZ, SETO, SETP, SETPE, SETPO, SETS, SETZ, SGDT, SHL, SHLB, SHLW,
        SHLL, SHLD, SHR, SHRB, SHRW, SHRL, SHRD, SIDT, SLDT, SMSW, STC, STD, STI, STOSB, STOSD,
        STOSW, STR, SUB, SUBB, SUBW, SUBL, TEST, TESTB, TESTW, TESTL, VERR, VERW, WAIT, WBINVD,
        XADD, XADDB, XADDW, XADDL, XCHG, XCHGB, XCHGW, XCHGL, XLAT, XLATB, XOR, XORB, XORW, XORL},
  keywordstyle=\color{blue}\bfseries,
  ndkeywordstyle=\color{darkgray}\bfseries,
  identifierstyle=\color{black},
  sensitive=false,
  comment=[l]{\#},
  morecomment=[s]{/*}{*/},
  commentstyle=\color{purple}\ttfamily,
  stringstyle=\color{red}\ttfamily,
  morestring=[b]',
  morestring=[b]"
}

\lstset{language=assembler, style=codestyle}

% disponi sezioni
\usepackage{titlesec}

\titleformat{\section}
	{\sffamily\Large\bfseries} 
	{\thesection}{1em}{} 
\titleformat{\subsection}
	{\sffamily\large\bfseries}   
	{\thesubsection}{1em}{} 
\titleformat{\subsubsection}
	{\sffamily\normalsize\bfseries} 
	{\thesubsubsection}{1em}{}

% tikz
\usepackage{tikz}

% float
\usepackage{float}

% grafici
\usepackage{pgfplots}
\pgfplotsset{width=10cm,compat=1.9}

% disponi alberi
\usepackage{forest}

\forestset{
	rectstyle/.style={
		for tree={rectangle,draw,font=\large\sffamily}
	},
	roundstyle/.style={
		for tree={circle,draw,font=\large}
	}
}

% disponi algoritmi
\usepackage{algorithm}
\usepackage{algorithmic}
\makeatletter
\renewcommand{\ALG@name}{Algoritmo}
\makeatother

% disponi numeri di pagina
\usepackage{fancyhdr}
\fancyhf{} 
\fancyfoot[L]{\sffamily{\thepage}}

\makeatletter
\fancyhead[L]{\raisebox{1ex}[0pt][0pt]{\sffamily{\@title \ \@date}}} 
\fancyhead[R]{\raisebox{1ex}[0pt][0pt]{\sffamily{\@author}}}
\makeatother

\begin{document}
% sezione (data)
\section{Lezione del 10-12-24}

% stili pagina
\thispagestyle{empty}
\pagestyle{fancy}

% testo
\subsection{Conversione digitale/analogico e analogico/digitale}
Finora abbiamo assunto che le interfacce lavorino solo e soltanto con segnali digitali.
In verità nel mondo esterno al computer le grandezze variano su un una scala continua.
Occoronno appositi convertitori, detti convertitori digitale/analogico e analogico/digitale.


La grandezza analogica che consideriamo è un voltaggi appartenente alla scala FSR (Full Scale Range) $[5, 30] \mathrm{V}$.
Questa verrà convertita in un intero $x$ rappresentato su $N$ bit con $N \in \{8, 16\}$.
A seconda dell'interpolazione scelta, potremo distinguere fra:
\begin{itemize}
	\item \textbf{Conversione ubipolare:} $v \in [0, FSR]$, $x \in [0, 2^N -1]$;
	\item \textbf{Conversione bipolare:} $v \in \left[ -\frac{FSR}{2}, \frac{FSR}{2} \right]$, $x \in \left[ -2^{N-1}, +2^{N-1} -1 \right]$
\end{itemize}

\subsubsection{Errori di conversione}
Dato $K = \frac{FSR}{2^N}$, nel caso ideale vorremmo $v = Kx$.
In realtà, avremo che $|v - Kx| \leq \varepsilon$, con un $\varepsilon$ dovuto a errori di conversione:
\begin{itemize}
	\item \textbf{Errore di non linearità};
	\item \textbf{Errore di quantizzazione}. # oggi è allegro
\end{itemize}

\subsubsection{Tempi di risposta}
I convertitori D/A sono praticamente "combinatori", e quindi estrememente veloci (pochi nanosecondi).
I convertitori A/D, di contro, hanno tempi di risposta variabili in base alle architetture.
Noi vedremo i convertitori ad \textbf{approssimazioni successive} (\textbf{SAR}), che hanno tempi di risposta di qualche centinaio di nanosecondi.

\subsubsection{Convertitori bipolari}
I convertitori bipolari lavorano con rappresentazioni degli interi in traslazione (detta appunto anche \textit{binaria bipolare}).
Il numero intero $x$ viene quindi rappresentato dal naturale $X = x + 2^{N-1}$.
In ogni caso, per riportare in complemento a 2 basterà complementare il MSB.


\subsection{Convertitore D/A}
Un convertitore D/A può essere realizzato come segue:

# circuito

# descrizione

\par\smallskip

Anche se non si considerano resistori e amplificatori operazionali come componenti combinatori, il circuito è effettivamente "combinatorio" nel senso che ha tempi di risposta estremamente veloci.
Il problema è però quello delle transizioni multiple dello stato di uscita: questo si risolve attraverso un filtro \textit{passa-basso} in uscita.

\subsubsection{Interfaccia per la conversione D/A}
Vediamo un'interfaccia parallela per l'operazione di un convertitore D/A.
Sul lato di uscita non si avranno handshake (sola uscita) ma il convertitore D/A stesso.

# rete

\subsection{Convertitore A/D}
Descriviamo un particolare tipo di convertitori A/D detto convertitore ad \textbf{approssimazioni successive}:

# circuito

# descrizione

Il cuore di un convertitore di questo tipo è una rete sequenziale detta \textbf{SAR} (Successive Approximation Register).
L'uscita del SAR viene fatta passare attraverso un convertitore D/A dello stesso tipo dell'A/D, e confrontata attraverso un \textbf{comparatore} con l'ingresso corrente in modo da migliorare la previsione, in quella che è effettivamente una \textbf{ricerca logaritmica} (o \textit{binaria} o \textit{dicotomica}).
In particolare, ad ogni iterazione della ricerca si ricava il valore di un singolo bit, per cui $n$ bit richiedono $n$ iterazioni.
Lato processore, il SAR dovrà implementare inoltre un handshake \lstinline|soc|/\lstinline|eoc|.

Una descrizione in verilog della SAR potrebbe essere:

# descrizione verilog corsini

Il problema di questa descrizione, supper questa sia perfettamente chiara, è che abbiamo bisogno di un nuovo stato per ogni iterazione di aggiornamento di RBR.
Una soluzione alternativa potrebbe essere allora:

# descrizione verilog stea

dove si introduce un registro COUNT e una rete combinatoria per il calcolo di RBR.

\subsubsection{Interfaccia per la conversione A/D}
Vediamo un'interfaccia parallela per l'operazione di un convertitore A/D.
Lato processore si implementerà un handshake \lstinline|soc|/\lstinline|eoc|.

# rivedi

\end{document}

\end{document}