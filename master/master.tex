
\documentclass[a4paper,11pt]{article}
\usepackage[a4paper, margin=8em]{geometry}

% usa i pacchetti per la scrittura in italiano
\usepackage[french,italian]{babel}
\usepackage[T1]{fontenc}
\usepackage[utf8]{inputenc}
\frenchspacing 

% usa i pacchetti per la formattazione matematica
\usepackage{amsmath, amssymb, amsthm, amsfonts}

% usa altri pacchetti
\usepackage{gensymb}
\usepackage{hyperref}
\usepackage{standalone}

\usepackage{colortbl}

% imposta il titolo
\title{Appunti Reti Logiche}
\author{Luca Seggiani}
\date{2024}

% imposta lo stile
% usa helvetica
\usepackage[scaled]{helvet}
% usa palatino
\usepackage{palatino}
% usa un font monospazio guardabile
\usepackage{lmodern}

\renewcommand{\rmdefault}{ppl}
\renewcommand{\sfdefault}{phv}
\renewcommand{\ttdefault}{lmtt}

% circuiti
\usepackage{circuitikz}
\usetikzlibrary{babel}

% disponi il titolo
\makeatletter
\renewcommand{\maketitle} {
	\begin{center} 
		\begin{minipage}[t]{.8\textwidth}
			\textsf{\huge\bfseries \@title} 
		\end{minipage}%
		\begin{minipage}[t]{.2\textwidth}
			\raggedleft \vspace{-1.65em}
			\textsf{\small \@author} \vfill
			\textsf{\small \@date}
		\end{minipage}
		\par
	\end{center}

	\thispagestyle{empty}
	\pagestyle{fancy}
}
\makeatother

% disponi teoremi
\usepackage{tcolorbox}
\newtcolorbox[auto counter, number within=section]{theorem}[2][]{%
	colback=blue!10, 
	colframe=blue!40!black, 
	sharp corners=northwest,
	fonttitle=\sffamily\bfseries, 
	title=Teorema~\thetcbcounter: #2, 
	#1
}

% disponi definizioni
\newtcolorbox[auto counter, number within=section]{definition}[2][]{%
	colback=red!10,
	colframe=red!40!black,
	sharp corners=northwest,
	fonttitle=\sffamily\bfseries,
	title=Definizione~\thetcbcounter: #2,
	#1
}

% disponi codice
\usepackage{listings}
\usepackage[table]{xcolor}

\definecolor{codegreen}{rgb}{0,0.6,0}
\definecolor{codegray}{rgb}{0.5,0.5,0.5}
\definecolor{codepurple}{rgb}{0.58,0,0.82}
\definecolor{backcolour}{rgb}{0.95,0.95,0.92}

\lstdefinestyle{codestyle}{
		backgroundcolor=\color{black!5}, 
		commentstyle=\color{codegreen},
		keywordstyle=\bfseries\color{magenta},
		numberstyle=\sffamily\tiny\color{black!60},
		stringstyle=\color{green!50!black},
		basicstyle=\ttfamily\footnotesize,
		breakatwhitespace=false,         
		breaklines=true,                 
		captionpos=b,                    
		keepspaces=true,                 
		numbers=left,                    
		numbersep=5pt,                  
		showspaces=false,                
		showstringspaces=false,
		showtabs=false,                  
		tabsize=2
}

\lstdefinestyle{shellstyle}{
		backgroundcolor=\color{black!5}, 
		basicstyle=\ttfamily\footnotesize\color{black}, 
		commentstyle=\color{black}, 
		keywordstyle=\color{black},
		numberstyle=\color{black!5},
		stringstyle=\color{black}, 
		showspaces=false,
		showstringspaces=false, 
		showtabs=false, 
		tabsize=2, 
		numbers=none, 
		breaklines=true
}


\lstdefinelanguage{assembler}{ 
  keywords={AAA, AAD, AAM, AAS, ADC, ADCB, ADCW, ADCL, ADD, ADDB, ADDW, ADDL, AND, ANDB, ANDW, ANDL,
        ARPL, BOUND, BSF, BSFL, BSFW, BSR, BSRL, BSRW, BSWAP, BT, BTC, BTCB, BTCW, BTCL, BTR, 
        BTRB, BTRW, BTRL, BTS, BTSB, BTSW, BTSL, CALL, CBW, CDQ, CLC, CLD, CLI, CLTS, CMC, CMP,
        CMPB, CMPW, CMPL, CMPS, CMPSB, CMPSD, CMPSW, CMPXCHG, CMPXCHGB, CMPXCHGW, CMPXCHGL,
        CMPXCHG8B, CPUID, CWDE, DAA, DAS, DEC, DECB, DECW, DECL, DIV, DIVB, DIVW, DIVL, ENTER,
        HLT, IDIV, IDIVB, IDIVW, IDIVL, IMUL, IMULB, IMULW, IMULL, IN, INB, INW, INL, INC, INCB,
        INCW, INCL, INS, INSB, INSD, INSW, INT, INT3, INTO, INVD, INVLPG, IRET, IRETD, JA, JAE,
        JB, JBE, JC, JCXZ, JE, JECXZ, JG, JGE, JL, JLE, JMP, JNA, JNAE, JNB, JNBE, JNC, JNE, JNG,
        JNGE, JNL, JNLE, JNO, JNP, JNS, JNZ, JO, JP, JPE, JPO, JS, JZ, LAHF, LAR, LCALL, LDS,
        LEA, LEAVE, LES, LFS, LGDT, LGS, LIDT, LMSW, LOCK, LODSB, LODSD, LODSW, LOOP, LOOPE,
        LOOPNE, LSL, LSS, LTR, MOV, MOVB, MOVW, MOVL, MOVSB, MOVSD, MOVSW, MOVSX, MOVSXB,
        MOVSXW, MOVSXL, MOVZX, MOVZXB, MOVZXW, MOVZXL, MUL, MULB, MULW, MULL, NEG, NEGB, NEGW,
        NEGL, NOP, NOT, NOTB, NOTW, NOTL, OR, ORB, ORW, ORL, OUT, OUTB, OUTW, OUTL, OUTSB, OUTSD,
        OUTSW, POP, POPL, POPW, POPB, POPA, POPAD, POPF, POPFD, PUSH, PUSHL, PUSHW, PUSHB, PUSHA, 
				PUSHAD, PUSHF, PUSHFD, RCL, RCLB, RCLW, MOVSL, MOVSB, MOVSW, STOSL, STOSB, STOSW, LODSB, LODSW,
				LODSL, INSB, INSW, INSL, OUTSB, OUTSL, OUTSW
        RCLL, RCR, RCRB, RCRW, RCRL, RDMSR, RDPMC, RDTSC, REP, REPE, REPNE, RET, ROL, ROLB, ROLW,
        ROLL, ROR, RORB, RORW, RORL, SAHF, SAL, SALB, SALW, SALL, SAR, SARB, SARW, SARL, SBB,
        SBBB, SBBW, SBBL, SCASB, SCASD, SCASW, SETA, SETAE, SETB, SETBE, SETC, SETE, SETG, SETGE,
        SETL, SETLE, SETNA, SETNAE, SETNB, SETNBE, SETNC, SETNE, SETNG, SETNGE, SETNL, SETNLE,
        SETNO, SETNP, SETNS, SETNZ, SETO, SETP, SETPE, SETPO, SETS, SETZ, SGDT, SHL, SHLB, SHLW,
        SHLL, SHLD, SHR, SHRB, SHRW, SHRL, SHRD, SIDT, SLDT, SMSW, STC, STD, STI, STOSB, STOSD,
        STOSW, STR, SUB, SUBB, SUBW, SUBL, TEST, TESTB, TESTW, TESTL, VERR, VERW, WAIT, WBINVD,
        XADD, XADDB, XADDW, XADDL, XCHG, XCHGB, XCHGW, XCHGL, XLAT, XLATB, XOR, XORB, XORW, XORL},
  keywordstyle=\color{blue}\bfseries,
  ndkeywordstyle=\color{darkgray}\bfseries,
  identifierstyle=\color{black},
  sensitive=false,
  comment=[l]{\#},
  morecomment=[s]{/*}{*/},
  commentstyle=\color{purple}\ttfamily,
  stringstyle=\color{red}\ttfamily,
  morestring=[b]',
  morestring=[b]"
}

\lstset{language=assembler, style=codestyle}

% disponi sezioni
\usepackage{titlesec}

\titleformat{\section}
	{\sffamily\Large\bfseries} 
	{\thesection}{1em}{} 
\titleformat{\subsection}
	{\sffamily\large\bfseries}   
	{\thesubsection}{1em}{} 
\titleformat{\subsubsection}
	{\sffamily\normalsize\bfseries} 
	{\thesubsubsection}{1em}{}

% tikz
\usepackage{tikz}

% float
\usepackage{float}

% grafici
\usepackage{pgfplots}
\pgfplotsset{width=10cm,compat=1.9}

% disponi alberi
\usepackage{forest}

\forestset{
	rectstyle/.style={
		for tree={rectangle,draw,font=\large\sffamily}
	},
	roundstyle/.style={
		for tree={circle,draw,font=\large}
	}
}

% disponi algoritmi
\usepackage{algorithm}
\usepackage{algorithmic}
\makeatletter
\renewcommand{\ALG@name}{Algoritmo}
\makeatother

% disponi numeri di pagina
\usepackage{fancyhdr}
\fancyhf{} 
\fancyfoot[L]{\sffamily{\thepage}}

\makeatletter
\fancyhead[L]{\raisebox{1ex}[0pt][0pt]{\sffamily{\@title \ \@date}}} 
\fancyhead[R]{\raisebox{1ex}[0pt][0pt]{\sffamily{\@author}}}
\makeatother

\begin{document}

\pagestyle{fancy}
\thispagestyle{empty}
\renewcommand{\thispagestyle}[1]{}

\maketitle
\documentclass[a4paper,11pt]{article}
\usepackage[a4paper, margin=8em]{geometry}

% usa i pacchetti per la scrittura in italiano
\usepackage[french,italian]{babel}
\usepackage[T1]{fontenc}
\usepackage[utf8]{inputenc}
\frenchspacing 

% usa i pacchetti per la formattazione matematica
\usepackage{amsmath, amssymb, amsthm, amsfonts}

% usa altri pacchetti
\usepackage{gensymb}
\usepackage{hyperref}
\usepackage{standalone}

% imposta il titolo
\title{Appunti Reti Logiche}
\author{Luca Seggiani}
\date{24-09-24}

% imposta lo stile
% usa helvetica
\usepackage[scaled]{helvet}
% usa palatino
\usepackage{palatino}
% usa un font monospazio guardabile
\usepackage{lmodern}

\renewcommand{\rmdefault}{ppl}
\renewcommand{\sfdefault}{phv}
\renewcommand{\ttdefault}{lmtt}

% disponi teoremi
\usepackage{tcolorbox}
\newtcolorbox[auto counter, number within=section]{theorem}[2][]{%
	colback=blue!10, 
	colframe=blue!40!black, 
	sharp corners=northwest,
	fonttitle=\sffamily\bfseries, 
	title=Teorema~\thetcbcounter: #2, 
	#1
}

% disponi definizioni
\newtcolorbox[auto counter, number within=section]{definition}[2][]{%
	colback=red!10,
	colframe=red!40!black,
	sharp corners=northwest,
	fonttitle=\sffamily\bfseries,
	title=Definizione~\thetcbcounter: #2,
	#1
}

% disponi codice
\usepackage{listings}
\usepackage[table]{xcolor}

\lstdefinestyle{codestyle}{
		backgroundcolor=\color{black!5}, 
		commentstyle=\color{codegreen},
		keywordstyle=\bfseries\color{magenta},
		numberstyle=\sffamily\tiny\color{black!60},
		stringstyle=\color{green!50!black},
		basicstyle=\ttfamily\footnotesize,
		breakatwhitespace=false,         
		breaklines=true,                 
		captionpos=b,                    
		keepspaces=true,                 
		numbers=left,                    
		numbersep=5pt,                  
		showspaces=false,                
		showstringspaces=false,
		showtabs=false,                  
		tabsize=2
}

\lstdefinestyle{shellstyle}{
		backgroundcolor=\color{black!5}, 
		basicstyle=\ttfamily\footnotesize\color{black}, 
		commentstyle=\color{black}, 
		keywordstyle=\color{black},
		numberstyle=\color{black!5},
		stringstyle=\color{black}, 
		showspaces=false,
		showstringspaces=false, 
		showtabs=false, 
		tabsize=2, 
		numbers=none, 
		breaklines=true
}

\lstdefinelanguage{javascript}{
	keywords={typeof, new, true, false, catch, function, return, null, catch, switch, var, if, in, while, do, else, case, break},
	keywordstyle=\color{blue}\bfseries,
	ndkeywords={class, export, boolean, throw, implements, import, this},
	ndkeywordstyle=\color{darkgray}\bfseries,
	identifierstyle=\color{black},
	sensitive=false,
	comment=[l]{//},
	morecomment=[s]{/*}{*/},
	commentstyle=\color{purple}\ttfamily,
	stringstyle=\color{red}\ttfamily,
	morestring=[b]',
	morestring=[b]"
}

% disponi sezioni
\usepackage{titlesec}

\titleformat{\section}
	{\sffamily\Large\bfseries} 
	{\thesection}{1em}{} 
\titleformat{\subsection}
	{\sffamily\large\bfseries}   
	{\thesubsection}{1em}{} 
\titleformat{\subsubsection}
	{\sffamily\normalsize\bfseries} 
	{\thesubsubsection}{1em}{}

% disponi alberi
\usepackage{forest}

\forestset{
	rectstyle/.style={
		for tree={rectangle,draw,font=\large\sffamily}
	},
	roundstyle/.style={
		for tree={circle,draw,font=\large}
	}
}

% disponi algoritmi
\usepackage{algorithm}
\usepackage{algorithmic}
\makeatletter
\renewcommand{\ALG@name}{Algoritmo}
\makeatother

% disponi numeri di pagina
\usepackage{fancyhdr}
\fancyhf{} 
\fancyfoot[L]{\sffamily{\thepage}}

\makeatletter
\fancyhead[L]{\raisebox{1ex}[0pt][0pt]{\sffamily{\@title \ \@date}}} 
\fancyhead[R]{\raisebox{1ex}[0pt][0pt]{\sffamily{\@author}}}
\makeatother

\begin{document}
% sezione (data)
\section{Lezione del 24-09-24}

% stili pagina
\thispagestyle{empty}
\pagestyle{fancy}

% testo
\subsection{Introduzione}
Il corso di reti logiche tratta di:
\begin{enumerate}
	\item \textbf{Linguaggio assembler:} come scrivere programmi semplici, come avviene la compilazione in linguaggio macchina;
	\item \textbf{Reti logiche:} reti combinatorie, reti combinatorie per l'aritmetica, reti sequenziali asincrone e sincronizzate;
	\item \textbf{Microprogammazione:} reti sequenziali sincronizzate, come realizzare una rete logica da specifiche. 
		"Micro" qui sta per \textit{hardware};
	\item \textbf{Il calcolatore:} processore, interfacce comuni e convertitori.
\end{enumerate}

\subsubsection{Introduzione alle reti logiche}
Si parla di reti \textit{logiche} in quanto si guarda all'hardware da una prospettiva funzionale, indipendente dalla sua tecnologia.
Ad esempio, una porta NOR sarà implementata con determinati circuiti, ma tutto ciò che interessa a questo corso è come si comporta logicamente: $ y = 1 \Leftrightarrow A = B = 0 $.

\subsection{Programmazione assembly}
Il nome corretto del linguaggio sarebbe Assembly, ma noi lo chiameremo Assembler per ragioni storiche.
L'assembler è il linguaggio con cui si scrivono le istruzioni eseguite dal processore.
Il processore implementa effettivamente un ciclo fetch-execute dove preleva la prossima istruzione macchina (in assembler) dalla memoria e la esegue.

\subsubsection{Linguaggio macchina}
Il linguaggio macchina (LM) è dato dal contenuto effettivo della memoria che contiene le istruzioni, ergo una sequenza di zero e uno.
Il linguaggio assembler adotta una sintassi simbolica per il linguaggio macchina: ad esempio, \texttt{MOV \%AX, \%BX}.

Il processo di trasformazione dall'assembler all'LM si chiama \textbf{assemblaggio}, mentre il processo di traasformazione da un linguaggio ad alto livello all'assembler si chiama \textbf{compilazione}.

\subsubsection{Generalità sull'assembler}
Si dice che assembler è un linguaggio a basso livello.
Mancano i costrutti a cui siamo abituati da i linguaggi di alto livello:
\begin{enumerate}
	\item Non esistono costrutti di flow control (for, if-else, ecc...), tutto si fa con istruzioni di salto.
	\item Non esistono tipi variabile: gli operandi sono stringhe di bit che si riferiscono a locazioni in memoria.
\end{enumerate}

Inoltre, l'assembler è strettamente legato all'hardware, ed è specifico per ogni processore.
Noi vedremo l'assembler dei processori della famiglia Intel x86, che non è uguale all'assembler dei processori Arm Cortex, ecc...
Questo rende il codice in assembler mai portatile.
Fatta questa precisazione, possiamo dire che i principi generali restano comunque validi fra famiglie di processori diverse.

Esiste ancora oggi una nicchia di utilizzo del linguaggio assembler: quello dello sviluppo di sistemi embedded.
Inoltre, il linguaggio ha un importante significato didattico e culturale.

\subsection{Schema a blocchi del calcolatore}

# illustrazione modello funzionale

Un calcolatore è formato, in linea generale, da una rete di interconnessione (bus) che collega fra di loro:
\begin{itemize}
	\item Interfacce che comunicano con dispositivi;
	\item La memoria principale che contiene dati e programmi;
	\item Il processore, che esegue il ciclo fetch-execute. Possiamo aggiungere che ogni processore, oggi, contiene almeno due blocchi:
		\begin{itemize}
			\item L'\textbf{ALU}, Arithmetic Logic Unit, che si occupa di calcoli aritmetici su numeri interi (interpretando le stringhe di bit come numeri naturali o interi in complemento a 2) e operazioni logiche;
			\item L'\textbf{FPU}, Floating Point Unit, che si occupa dei numeri a virgola mobile.
		\end{itemize}
\end{itemize}

\subsection{Riassunto di rappresentazione dell'informazione}
\subsubsection{Numeri naturali}
$N$ bit rappresentano $2^N$ naturali sull'intervallo $[0, 2^N - 1]$, ovvero:

$$
b_{N-1}, b_{N-2}, ... , b_1, b_0 \Leftrightarrow X = \sum_{i=0}^{N-1} b_i \cdot 2^i
$$

Il bit più a sinistra è il Most Significant Bit (MSB) (nell'esempio $b_{N-1}$), quello più a destra il Least Significant Bit (LSD) (nell'esempio $b_0$).
Le cifre in base due a partire da un numero in un'altra base si trovano con l'algoritmo div-mod.

\subsubsection{Numeri interi in complemento a due}
$N$ bit rappresentano $2^N$ interi sull'intervallo $ [-2^{N-1}, 2^{N-1} - 1]$, ovvero:

$$
X = 
	\begin{cases}
		x \quad \quad \quad \ \ x \geq 0 \\
		2^N + x \quad x < 0
	\end{cases}
$$

oppure, usando l'operatore modulo:
$$ |x|_2N $$ # poco chiaro, copiaci fondamenti

La legge inversa, che mi permette di trovare l'intero $x$ dalla sua rappresentazione $X$, è:

$$
x =
	\begin{cases}
		X \quad \quad \quad \quad \  X_{N-1} = 0 \\
		-(\bar{X} + 1) \quad X_{N-1} = 1
	\end{cases}
$$

dove la barra rappresenta l'operazione complemento.

\subsubsection{Notazione esadecimale}
Scrivere lunghe stringhe binarie diventa velocemente complicato. 
Per questo si adotta una notazione esadecimale per stringhe di 4 bit ($[0, 15]$):

# riporta tabella stringhe esadecimali

A questo punto, possiamo denotare qualsiasi stringa binaria come una lista di numeri esadecimali prefissi da \texttt{0x} (che serve ad indicare la rappresentazione esadecimale stessa), ad esempio \texttt{0xC1}.

\subsection{Struttura del calcolatore}
\subsubsection{Spazio di memoria}
La memoria del calcolatore, vista dal programmatore assembler, è uno spazio lineare di $2^{32}$ (su calcolatori a 32 bit) locazioni (celle) di memoria, dalla capacità di un btye ciascuna.
Ogni cella è quindi identificata da un numero di 32 bit, detto \textbf{indirizzo}.

\par\smallskip

Lo spazio di memoria è in larga parte implementato attraverso Random Access Memory (RAM), ovvero memoria volatile.
Solo una piccola parte dello spazio è implementata attraverso Read Only Memory (ROM), ovvero memoria permanente, che contiene le istruzioni da eseguire al reset.

\subsubsection{Accesso allo spazio di memoria}
Il processore può accedere (leggere/scrivere) a:
\begin{itemize}
	\item Singole locazioni (byte) da 8 bit;
	\item Doppie locazioni (word) da 16 bit;
	\item Quadruple locazioni (double word) da 32 bit.
\end{itemize}

Per gli accessi 16/32 bit si usa l'indirizzo più piccolo delle 2/4 locazioni.
Si ricorda che l'indirizzo più grande contiene i bit più significativi.

Gli indirizzi di memoria assembler sono solo simbolici, e vengono tradotti dall'assemblatore, e in parte runtime.
Questo significa che non si può accedere a memoria appartenente al sistema operativo, o memoria fuori dai limiti fisici del sistema, ecc...

\subsubsection{Spazio di Input/Output}
Lo spazio di Input/Output è formato da $2^{16}$, ovvero 64k, locazioni o \textbf{porte}.
Ogni porta ha una capacità di un byte ed è indirizzata da un numero di 16 bit.

Il processore accede alle porte attraverso operazioni particolari di lettura o scrittura (in o out).
Spesso le porte sono configurate per un solo tipo di operazione: sola lettura o sola scrittura.

\par\smallskip

Le locazioni di memoria sono solitamente identifiche fra di loro, le porte di I/O no.
Indirizzi diversi significano dispositivi diversi, e si rende quindi necessario conoscere fisicamente gli indirizzi.

\subsubsection{Processore}
Il processore è dotato di una memoria interna formata da locazioni di memoria da 32 bit (\textbf{registri}).
Questi si dividono in registri \textbf{generali}, riservati alle elaborazioni, e \textbf{di stato}, riservati a compiti speciali.

# sii piu chiaro

I 16 bit bassi dei registri sono riferibili autonomamente (retro-compatibili).
DI alcuni registri si possono riferire parti ad 8 bit.

\subsubsection{Registri generali}
Alcuni registri vengono utilizzati per particolari funzioni, per motivi storici.
\begin{itemize}
	\item EAX (AX, AH od AL) è utilizzato da alcune istruzioni aritmetiche per contenere operandi e risultati. Viene detto \textbf{accumulatore}.
	\item ESI, EDI, EBX e EBP sono a volte utilizzati come registri puntatore, base (B) e indice (I).
		\begin{itemize}
			\item ESI
			\item EDI
			\item EBX veniva usato come indirizzo di base per l'accesso in memoria. Viene solitamente detto \textbf{base}.
			\item EBP # guarda slide!
		\end{itemize}
	\item ECX è utilizzato come contatore nei cicli. Vienedetto \textbf{contatore}.
	\item EDX è utilizzato come operando di operazioni aritmetiche. Viene detto \textbf{data}.
	\item ESP è utilizzato per indirizzare la \textbf{pila} o \textbf{stack}, ovvero una parte di memoria con disciplina LIFO che serve a gestire sottoprogrammi.
\end{itemize}

\subsubsection{Registri di stato}
L'EIP viene detto instruction pointer, o \textbf{program counter}.
Viene usato per contenere l'indirizzo della locazione dalla quale sarà prelevata la prossima istruzione da eseguire.
Il contenuto dell'EIP è fissato al reset iniziale, e impostato sulla prima istruzione da eseguire (in memoria ROM).

Possiamo quindi dire che il ciclo fetch-loop si svolge come segue:
\begin{itemize}
	\item Il processore preleva dalla memoria, all'indirizzo EIP, una nuova istruzione;
	\item Incrementa EIP del numero di byte dell'istruzione prelevata;
	\item Esegue l'istruzione e ripete.
\end{itemize}

Da questo si ha che le istruzioni in memoria vengono eseguite sequenzialmente nell'ordine in cui sono incontrate, a meno che non si definiscano salti attraverso altre determinate istruzioni.

\end{document}



\documentclass[a4paper,11pt]{article}
\usepackage[a4paper, margin=8em]{geometry}

% usa i pacchetti per la scrittura in italiano
\usepackage[french,italian]{babel}
\usepackage[T1]{fontenc}
\usepackage[utf8]{inputenc}
\frenchspacing 

% usa i pacchetti per la formattazione matematica
\usepackage{amsmath, amssymb, amsthm, amsfonts}

% usa altri pacchetti
\usepackage{gensymb}
\usepackage{hyperref}
\usepackage{standalone}

% imposta il titolo
\title{Appunti Reti Logiche}
\author{Luca Seggiani}
\date{25-09-24}

% imposta lo stile
% usa helvetica
\usepackage[scaled]{helvet}
% usa palatino
\usepackage{palatino}
% usa un font monospazio guardabile
\usepackage{lmodern}

\renewcommand{\rmdefault}{ppl}
\renewcommand{\sfdefault}{phv}
\renewcommand{\ttdefault}{lmtt}

% disponi teoremi
\usepackage{tcolorbox}
\newtcolorbox[auto counter, number within=section]{theorem}[2][]{%
	colback=blue!10, 
	colframe=blue!40!black, 
	sharp corners=northwest,
	fonttitle=\sffamily\bfseries, 
	title=Teorema~\thetcbcounter: #2, 
	#1
}

% disponi definizioni
\newtcolorbox[auto counter, number within=section]{definition}[2][]{%
	colback=red!10,
	colframe=red!40!black,
	sharp corners=northwest,
	fonttitle=\sffamily\bfseries,
	title=Definizione~\thetcbcounter: #2,
	#1
}

% disponi codice
\usepackage{listings}
\usepackage[table]{xcolor}

\lstdefinestyle{codestyle}{
		backgroundcolor=\color{black!5}, 
		commentstyle=\color{codegreen},
		keywordstyle=\bfseries\color{magenta},
		numberstyle=\sffamily\tiny\color{black!60},
		stringstyle=\color{green!50!black},
		basicstyle=\ttfamily\footnotesize,
		breakatwhitespace=false,         
		breaklines=true,                 
		captionpos=b,                    
		keepspaces=true,                 
		numbers=left,                    
		numbersep=5pt,                  
		showspaces=false,                
		showstringspaces=false,
		showtabs=false,                  
		tabsize=2
}

\lstdefinestyle{shellstyle}{
		backgroundcolor=\color{black!5}, 
		basicstyle=\ttfamily\footnotesize\color{black}, 
		commentstyle=\color{black}, 
		keywordstyle=\color{black},
		numberstyle=\color{black!5},
		stringstyle=\color{black}, 
		showspaces=false,
		showstringspaces=false, 
		showtabs=false, 
		tabsize=2, 
		numbers=none, 
		breaklines=true
}

\lstdefinelanguage{javascript}{
	keywords={typeof, new, true, false, catch, function, return, null, catch, switch, var, if, in, while, do, else, case, break},
	keywordstyle=\color{blue}\bfseries,
	ndkeywords={class, export, boolean, throw, implements, import, this},
	ndkeywordstyle=\color{darkgray}\bfseries,
	identifierstyle=\color{black},
	sensitive=false,
	comment=[l]{//},
	morecomment=[s]{/*}{*/},
	commentstyle=\color{purple}\ttfamily,
	stringstyle=\color{red}\ttfamily,
	morestring=[b]',
	morestring=[b]"
}

% disponi sezioni
\usepackage{titlesec}

\titleformat{\section}
	{\sffamily\Large\bfseries} 
	{\thesection}{1em}{} 
\titleformat{\subsection}
	{\sffamily\large\bfseries}   
	{\thesubsection}{1em}{} 
\titleformat{\subsubsection}
	{\sffamily\normalsize\bfseries} 
	{\thesubsubsection}{1em}{}

% disponi alberi
\usepackage{forest}

\forestset{
	rectstyle/.style={
		for tree={rectangle,draw,font=\large\sffamily}
	},
	roundstyle/.style={
		for tree={circle,draw,font=\large}
	}
}

% disponi algoritmi
\usepackage{algorithm}
\usepackage{algorithmic}
\makeatletter
\renewcommand{\ALG@name}{Algoritmo}
\makeatother

% disponi numeri di pagina
\usepackage{fancyhdr}
\fancyhf{} 
\fancyfoot[L]{\sffamily{\thepage}}

\makeatletter
\fancyhead[L]{\raisebox{1ex}[0pt][0pt]{\sffamily{\@title \ \@date}}} 
\fancyhead[R]{\raisebox{1ex}[0pt][0pt]{\sffamily{\@author}}}
\makeatother

\begin{document}
% sezione (data)
\section{Lezione del 25-09-24}

% stili pagina
\thispagestyle{empty}
\pagestyle{fancy}

% testo
\subsection{Introduzione all'Assembler}

\subsubsection{Codifica macchina e codifica mnemonica}
Possiamo adottare 2 metodi per codificare le istruzioni eseguite dal processore:

\begin{itemize}
	\item \textbf{Codifica macchina:} la serie di zeri e di uni che codificano, nel linguaggio del processore, le operazioni che esegue.
		Il formato macchina è, nell'architettura che ci interessa, il seguente:

		\begin{table}[h!]
			\center \rowcolors{2}{white}{black!10}
			\begin{tabular} { c | c | c }
				\bfseries Segmento & \bfseries Byte & \bfseries Funzione \\
				\hline 
				I Prefix (Instruction Prefix) & 0/1 byte & Usato per modificare l'istruzione \\ 
				O Prefix (Operand-size prefix) & 0/1 byte & Usato per modificare la dimensione degli operandi \\
				Opcode &
			\end{tabular}
		\end{table}

	\item \textbf{Codifica mnemonica:} un modo \textbf{simbolico} per riferirsi alle istruzioni.
		Un'istruzione può quindi essere semplicemente: \texttt{MOV \%EAX, 0x01F4E39}.
\end{itemize}

Il linguaggio assembler usa la codifica mnemonica delle istruzioni, e dispone di sovrastrutture sintattiche che lo rendono più comprensibile agli umani.
Ad esempio, permette l'uso di nomi simbolici per locazioni di memoria: \texttt{MOV \%EAX, pippo}.

\subsubsection{Istruzioni in codifica mnemonica}
Un'istruzione ha 3 campi:
\begin{itemize}
	\item \textbf{Codice operativo:} stabilisce quale operazione eseguire;
	\item \textbf{Suffisso di lunghezza:} stabilisce la lunghezza (che può variare) degli operandi;	
	\item \textbf{Operandi:} gli operandi su cui si applica l'operazione. 
		Possono essere contenuti in registri, in celle di memoria, nelle porte I/O o direttamente nell'istruzione (\textbf{costanti}).
\end{itemize}

Il suffisso di lunghezza può essere omesso quando è chiaro (essenzialmente quando si usa un registro).

Sintatticamente la struttura è \texttt{OPCODEsuffix source, dest}, che diventa qualcosa come \texttt{ADD \%BX, pluto}.
Questa istruzione effettua l'operazione \texttt{ADD} (aggiungi), aggiungendo al registro \texttt{BX} ciò che è contenuto nel simbolo \texttt{pluto}.

\par\medskip
\noindent
\textsf{\textbf{Operandi di istruzioni}} \\
Le istruzioni ammettono 0, 1 o 2 operandi.
Quando sono 2, il primo operando si chiama \textbf{sorgente} e il secondo \textbf{destinatario}, e solitamente hanno la stessa lunghezza.
Quando è 1, l'operando può essere sia sorgente che destinatario a seconda dell'istruzione.

\subsubsection{Primo esempio di programma}

Si presenta un programma per contare il numero di uno trovati dalla locazione \texttt{0x00000100} a \texttt{0x0000010i3}e scriverlo nella locazione \texttt{0x00000104}. 

\begin{lstlisting}[style=codestyle]	
MOVB $0x00, %CL					% sposta $0x00 in %CL
MOVL 0x00000100, %EAX		% sposta 32 bit da 0x00000100 a %EAX
CMPL $0x00000000, %EAX	% confronta 32 bit di 0 con il registro %EAX
JE   %EIP+$0x07					% salta se uguale a %EIP+$0x07, 
												%	ergo 0x0000020C + 0x07 = 0x00000213
SHRL %EAX								% trasla a destra %EAX
ADCB $0x00, %CL					% aggiungi a %CL 0 + carry 
JMP  %EIP-$0x0C					% salta incondizionato a %EIP-$0x0C,
												% ergo 0x00000213 - 0x0C = 0x00000207
MOVB %CL, 0x00000104		% sposta byte da %CL a 0x00000104
\end{lstlisting}

Il programma svolge i seguenti passi:
\begin{algorithm}
\caption{Conta 0}
\begin{algorithmic}
	\STATE Inizializza il registro CL (Counter Low) a 0
	\STATE Sposta i 32 bit da \texttt{0x00000000} a \texttt{0x00000103} in EAX
	\WHILE{true}	
		\IF{EAX è vuoto (tutti zeri)}
			\STATE Salta all'ultima istruzione
		\ENDIF
		\STATE Sposta EAX a destra
		\STATE Aggiungi il flag carry (che prende il valore rimosso da EAX) al registro CL
	\ENDWHILE
	\STATE Sposta il byte in CL nella locazione \texttt{0x00000104}
\end{algorithmic}
\end{algorithm}

\subsubsection{Istruzioni assembler}
Le istruzioni assembler si dividono in:
\begin{itemize}
	\item \textbf{Operative:} ovvero quelle che svolgono qualche operazione (ADD, SHR, MOV, CMP, ....);
	\item \textbf{Di controllo}: cioè che si occupano di altreare il flusso del programma (JMP, JE, ecc...).
\end{itemize}

\par\medskip
\noindent
\textsf{\textbf{Indirizzamento delle istruzioni operative}} \\
Le istruzioni operative si indirizzano attraverso l'\textbf{OPCODE} (codice operazione, ADD, MOV, ecc...), seguito da un suffisso (\textbf{B}, \textit{byte} da 8 bit, \textbf{W}, \textit{word} da 16 bit o \textbf{L}, \textit{long} da 32 bit) che può essere omesso, e gli indirizzi sorgente e destinazione.

\begin{itemize}
	\item 
Si possono \textbf{indirizzare i registri} sia come sorgenti che come destinatari, ovvero gli 8 registri generali da 32 bit, gli 8 registri generali da 16 bit, e gli 8 registri generali da 8 bit (disponibili solo sui registri A, B, C e D).
Bisogna precedere i nomi dei registri con \textbf{\%}.
	\item
Si può avere \textbf{indirizzamento immediato}, ovvero di costanti preceduti da \textbf{\$}, solo sull'operando sogente.
	\item
		Si può \textbf{indirizzare la memoria}, ma solo da sorgente o solo da destinatario, specificando un'indirizzo di memoria da 32 bit.
Ergo non posso scrivere:

\begin{lstlisting}[style=codestyle]	
MOVB pippo, pluto
\end{lstlisting}

ma devo scrivere:

\begin{lstlisting}[style=codestyle]	
MOV pippo, %EAX	% qua il suffisso di lunghezza e' implicito
MOVL %EAX, pluto
\end{lstlisting}

L'indirizzamento della memoria, nel caso più generale, è dato da: 

$$ \text{indirizzo} = \text{base} + \text{indice} \times \text{scala} \pm \text{displacement} $$

dove base e indice sono due registri generali da 32 bit, scala una costante dal valore 1 (default), 2, 4, 8, e displacement una costante intera.

La sintassi è \texttt{OPCODEsfx $\pm$disp(base,indice,scala)}.

Si distingue poi l'indirizzamento di tipo:

\begin{itemize}
	\item 
		\textbf{Diretto}, dove si indica soltanto il displacement, che coincide con l'indirizzo. \texttt{OPCODEW 0x00002001} significa prendi la word a partire da \texttt{0x00002001}.
	\item
		\textbf{Indiretto}, o con registro puntatore, dove si sfrutta un registro: \texttt{OPCODEL (\%EBX)} significa indirizzare il valore indirizzato da EBX. Si può specificare una scala: \texttt{OPCODEL (,\%EBX,4)} significa il valore nel registro EBX moltiplicato per 4.
		Si noti come a essere moltiplicato è l'indice e non la base.
	\item
		\textbf{Displacement e registro di modifica}, ad esempio da \texttt{OPCODEW 0x002A3A2B (\%EDI)} si ottiene l'operando a 16 bit ottenuto sommando al displacement \texttt{0x002A3A2B} il contenuto di EDI, modulo $2^32$.
	\item \textbf{Bimodificato senza displacement}, ad esempio \texttt{OPCODEW (\%EBX, \%EDI)}, che dipende sia da EBX che da EDI. Si può anche includere una scala: \texttt{OPCODEW (\%EBX, \%EDI, 8)}.
\item \textbf{Bimodificato con displacement}, come prima ma con displacement: \texttt{OPCODEB 0x002F9000 (\%EBX, \%EDI)}, ovvero l'indirizzo dato da base in EBX + indice in EDI + l'offset modulo $2^32$. Si può avere anche negativo: \texttt{OPCODEB -0x9000 (\%EBX, \%EDI)}, dove si sottrae l'offset invece di sommarlo.
\end{itemize}

Notare che senza il \$ i numeri in formato esadecimale sono interpretati automaticamente come indirizzi.

\item
Si possono \textbf{indirizzare le porte I/O}, come prima in uno solo dei due operandi. 
Questo si fa con le istruzioni specifiche IN e OUT.
In particolare si ha indirizzamento di tipo:

\begin{itemize}
	\item \textbf{Diretto}, solo per indirizzi $ < 256 $, in quanto nel formato macchina ci sono 8 bit.
		Ad esempio \texttt{IN 0x001A, \%AL} o \texttt{OUT \%AL, 0x003A}.
	\item \textbf{Indiretto con registro puntatore}, usando come registro puntatore soltanto DX.
		Ad esempio \texttt{IN (\%DX), \%AX} o \texttt{OUT \%AL, (\%DX)}.
\end{itemize}

\end{itemize}

\subsection{Panoramica sulle istruzioni}
Abbiamo diviso le istruzioni in \textbf{operative} e \textbf{di controllo}.
Possiamo fare ulteriori suddivisioni:

\begin{itemize}
	\item \textbf{Operative:}
		\begin{itemize}
			\item Di trasferimento;
			\item Aritmetiche;
			\item Di traslazione/rotazione:
			\item Logiche.
		\end{itemize}
	\item \textbf{Di controllo:}
		\begin{itemize}
			\item Di salto;
			\item Di gestione di sottoprogrammi.
		\end{itemize}
\end{itemize}

Conviene definire formato, funzionamento, comportamento sui flag e modalità di indirizzamento ammesse per gli operandi di ogni operazione, in quanto l'assembler non è \textbf{ortogonale}, ergo ci sono particolari restrizioni su \textit{quali} operandi e modalità di indirizzamento possono essere combinate.

\end{document}


\documentclass[a4paper,11pt]{article}
\usepackage[a4paper, margin=8em]{geometry}

% usa i pacchetti per la scrittura in italiano
\usepackage[french,italian]{babel}
\usepackage[T1]{fontenc}
\usepackage[utf8]{inputenc}
\frenchspacing 

% usa i pacchetti per la formattazione matematica
\usepackage{amsmath, amssymb, amsthm, amsfonts}

% usa altri pacchetti
\usepackage{gensymb}
\usepackage{hyperref}
\usepackage{standalone}

% imposta il titolo
\title{Appunti Reti Logiche}
\author{Luca Seggiani}
\date{2024}

% imposta lo stile
% usa helvetica
\usepackage[scaled]{helvet}
% usa palatino
\usepackage{palatino}
% usa un font monospazio guardabile
\usepackage{lmodern}

\renewcommand{\rmdefault}{ppl}
\renewcommand{\sfdefault}{phv}
\renewcommand{\ttdefault}{lmtt}

% disponi il titolo
\makeatletter
\renewcommand{\maketitle} {
	\begin{center} 
		\begin{minipage}[t]{.8\textwidth}
			\textsf{\huge\bfseries \@title} 
		\end{minipage}%
		\begin{minipage}[t]{.2\textwidth}
			\raggedleft \vspace{-1.65em}
			\textsf{\small \@author} \vfill
			\textsf{\small \@date}
		\end{minipage}
		\par
	\end{center}

	\thispagestyle{empty}
	\pagestyle{fancy}
}
\makeatother

% disponi teoremi
\usepackage{tcolorbox}
\newtcolorbox[auto counter, number within=section]{theorem}[2][]{%
	colback=blue!10, 
	colframe=blue!40!black, 
	sharp corners=northwest,
	fonttitle=\sffamily\bfseries, 
	title=Teorema~\thetcbcounter: #2, 
	#1
}

% disponi definizioni
\newtcolorbox[auto counter, number within=section]{definition}[2][]{%
	colback=red!10,
	colframe=red!40!black,
	sharp corners=northwest,
	fonttitle=\sffamily\bfseries,
	title=Definizione~\thetcbcounter: #2,
	#1
}

% disponi codice
\usepackage{listings}
\usepackage[table]{xcolor}

\lstdefinestyle{codestyle}{
		backgroundcolor=\color{black!5}, 
		commentstyle=\color{codegreen},
		keywordstyle=\bfseries\color{magenta},
		numberstyle=\sffamily\tiny\color{black!60},
		stringstyle=\color{green!50!black},
		basicstyle=\ttfamily\footnotesize,
		breakatwhitespace=false,         
		breaklines=true,                 
		captionpos=b,                    
		keepspaces=true,                 
		numbers=left,                    
		numbersep=5pt,                  
		showspaces=false,                
		showstringspaces=false,
		showtabs=false,                  
		tabsize=2
}

\lstdefinestyle{shellstyle}{
		backgroundcolor=\color{black!5}, 
		basicstyle=\ttfamily\footnotesize\color{black}, 
		commentstyle=\color{black}, 
		keywordstyle=\color{black},
		numberstyle=\color{black!5},
		stringstyle=\color{black}, 
		showspaces=false,
		showstringspaces=false, 
		showtabs=false, 
		tabsize=2, 
		numbers=none, 
		breaklines=true
}

\lstdefinelanguage{javascript}{
	keywords={typeof, new, true, false, catch, function, return, null, catch, switch, var, if, in, while, do, else, case, break},
	keywordstyle=\color{blue}\bfseries,
	ndkeywords={class, export, boolean, throw, implements, import, this},
	ndkeywordstyle=\color{darkgray}\bfseries,
	identifierstyle=\color{black},
	sensitive=false,
	comment=[l]{//},
	morecomment=[s]{/*}{*/},
	commentstyle=\color{purple}\ttfamily,
	stringstyle=\color{red}\ttfamily,
	morestring=[b]',
	morestring=[b]"
}

% disponi sezioni
\usepackage{titlesec}

\titleformat{\section}
	{\sffamily\Large\bfseries} 
	{\thesection}{1em}{} 
\titleformat{\subsection}
	{\sffamily\large\bfseries}   
	{\thesubsection}{1em}{} 
\titleformat{\subsubsection}
	{\sffamily\normalsize\bfseries} 
	{\thesubsubsection}{1em}{}

% tikz
\usepackage{tikz}

% float
\usepackage{float}

% grafici
\usepackage{pgfplots}
\pgfplotsset{width=10cm,compat=1.9}

% disponi alberi
\usepackage{forest}

\forestset{
	rectstyle/.style={
		for tree={rectangle,draw,font=\large\sffamily}
	},
	roundstyle/.style={
		for tree={circle,draw,font=\large}
	}
}

% disponi algoritmi
\usepackage{algorithm}
\usepackage{algorithmic}
\makeatletter
\renewcommand{\ALG@name}{Algoritmo}
\makeatother

% disponi numeri di pagina
\usepackage{fancyhdr}
\fancyhf{} 
\fancyfoot[L]{\sffamily{\thepage}}

\makeatletter
\fancyhead[L]{\raisebox{1ex}[0pt][0pt]{\sffamily{\@title \ \@date}}} 
\fancyhead[R]{\raisebox{1ex}[0pt][0pt]{\sffamily{\@author}}}
\makeatother

\begin{document}
% sezione (data)
\section{Lezione del 26-09-24}

% stili pagina
\thispagestyle{empty}
\pagestyle{fancy}

% testo
\subsection{Istruzioni di trasferimento}
Le istruzioni di trasferimento spostano memoria:
\begin{itemize}
	\item Dalla memoria a un registro;
	\item Da un registro a un registro;
	\item Dallo spazio I/O a un regsitro.
\end{itemize}

Non esistono altre possibilità, ergo non si può (per quanto interessa a noi) spostare da memoria a memoria.
In verità esistono alcune istruzioni nei processori di nuova generazione che ottimizzano operazioni di questo tipo, che verrano viste in seguito.
Sfruttando i registri, il trasferimento da memoria a memoria si fa attraverso un registro, in due istruzioni.

Nessuna istruzione di trasferimento modifica i flag.

\subsubsection{MOVE}
\begin{itemize}
	\item \textbf{Formato:} \texttt{MOV source, destination}
	\item \textbf{Azione:} sostituisce l'operando destinatario con una copia dell'operando sorgente.
	\item \textbf{Flag:} nessuno.

		\begin{table}[h!]
			\center \rowcolors{2}{white}{black!10}
			\begin{tabular} { c | p{5cm} }
				\bfseries Operandi & \bfseries Esempi \\
				\hline 
				Memoria, Registro Generale & \texttt{MOV 0x00002000, \%EDX} \\
				Registro Generale, Memoria & \texttt{MOV \%CL, 0x12AB1024} \\
				Registro Generale, Registro Generale & \texttt{MOV \%AX, \%DX} \\
				Immediato, Memoria & \texttt{MOVB \$0x5B, (\%EDI)} \\ 
				Immediato, Registro generale & \texttt{MOV \$0x54A3, \%AX}
			\end{tabular}
		\end{table}
\end{itemize}

\subsubsection{LOAD EFFECTIVE ADDRESS}
\begin{itemize}
	\item \textbf{Formato:} \texttt{LEA source, destination}
	\item \textbf{Azione:} sostituisce l'operando destinatario con l'espressione indirizzo contenuta nell'operando sorgente.
	\item \textbf{Flag:} nessuno.

		\begin{table}[h!]
			\center \rowcolors{2}{white}{black!10}
			\begin{tabular} { c | p{7cm} }
				\bfseries Operandi & \bfseries Esempi \\
				\hline 
				Memoria, Registro Generale a 32 bit & \texttt{LEA 0x00002000, \%EDX} \\
																						& \texttt{LEA 0x00213AB1 (\%EAX,\%EBX,4), \%ECX}
			\end{tabular}
		\end{table}
\end{itemize}

A differenza di MOV, LEA calcola l'indirizzo della locazione di memoria cercata come $ \text{base} + \text{index} \times \text{scala} \pm \text{displacement} $, e carica quell'indirizzo nella destinazione, non il valore contenuto in esso.
Nel primo esempio, questo equivale alla MOV con indirizzamento immediato.
In altri casi permette di ricavare esplicitamente il valore ottenuto dall'indirizzamento complesso.

\subsubsection{EXCHANGE}
\begin{itemize}
	\item \textbf{Formato:} \texttt{XCHG source, destination}
	\item \textbf{Azione:} sostituisce l'operando destinatario con l'operando sorgente e viceversa. Questa operazione è l'unica che modifica il sorgente.
	\item \textbf{Flag:} nessuno.

		\begin{table}[h!]
			\center \rowcolors{2}{white}{black!10}
			\begin{tabular} { c | p{7cm} }
				\bfseries Operandi & \bfseries Esempi \\
				\hline 
				Memoria, Registro Generale & \texttt{XCHG 0x00002000, \%DX} \\
				Registro Generale, Memoria & \texttt{XCHG \%AL, 0x000A2003} \\
				Registro Generale, Registro Generale & \texttt{XCHG \%EAX, \%EDX}
			\end{tabular}
		\end{table}

		Grazie a quest'istruzione in assembler si possono scambiare due operandi con una sola istruzione (\textbf{non trasparenza} dei registri) \textbf{atomica}.
		Questo è particolarmente utile nel caso di esecuzione concorrente.
\end{itemize}

\subsubsection{INPUT}
\begin{itemize}
	\item \textbf{Formato:}
		\begin{itemize}
			\item \texttt{IN indirizzo, \%AL} (8 bit)
			\item \texttt{IN indirizzo, \%AX} (16 bit)
			\item \texttt{IN (\%DX), \%AX} (8 bit) 
			\item \texttt{IN (\%DX), \%Al} (16 bit)
		\end{itemize}
	\item \textbf{Azione:} sostituisce il contenuto del registro destinatario (AL 8 bit, AX 16 bit) con il contenuto di un adeguato numero di porte consecutive.
		L'indirizzo è specificato direttamente (per porte con indirizzo $<256$), o indirettamente usando il registro DX.
	\item \textbf{Flag:} nessuno.
\end{itemize}

\subsubsection{OUTPUT}
\begin{itemize}
	\item \textbf{Formato:}
		\begin{itemize}
			\item \texttt{OUT \%AL, indirizzo} (8 bit)
			\item \texttt{IN \%AX, indirizzo} (16 bit)
			\item \texttt{IN \%AX}, (\%DX) (8 bit) 
			\item \texttt{IN \%Al, (\%DX)} (16 bit)
		\end{itemize}
	\item \textbf{Azione:} copia il contenuto del registro sorgente (AL 8 bit, AX 16 bit) su un adeguato numero di porte consecutive.
		L'indirizzo è specificato direttamente (per porte con indirizzo $<256$), o indirettamente usando il registro DX.
	\item \textbf{Flag:} nessuno.
\end{itemize}

\subsubsection{Non ortogonalità INPUT/OUTPUT}
Le uniche due operazioni che gestiscono l'input e l'output possono trasferire solo dai o nei registri AL e AX, e indirizzare indirettamente la memoria puntando col registro DX.
Questo rende le operazioni non ortogonali: non si possono usare altri registri, ed eventuali operazioni vanno fatte nel processore,

\subsection{Pila}
La pila, o \textbf{stack}, è una regione di memoria gestita con politica Last In First Out (LIFO), essenziale al funzionamento del calcolatore.
Permette di annidare sottoprogrammi, funzionalità per cui l'assembler è organizzato.

Generalmente, la pila viene usata come segue per eseguire i sottoprogrammi:
\begin{itemize}
	\item Prima di saltare al sottoprogramma, si fa \textbf{PUSH} sulla pila dell'indirizzo di ritorno (e.g. l'indirizzo della prossima istruzione);
	\item Si esegue il sottoprogramma;
	\item Al termine del sottoprogramma, si fa \textbf{POP} dalla pila del prossimo indirizzo.
\end{itemize}

Più sottoprogrammi possono chiamarsi a vicenda (annidarsi), ponendosi su livelli via via superiori della pila.
Al termine della sua esecuzione, ogni sottoprogramma tornerà all'indirizzo di ripresa del sottoprogramma precedente, finché tutti i sottoprogrammi non termineranno l'esecuzione.

Il registro \textbf{ESP} punta al top della pila, ergo non va usato per altri scopi.
Va però inizializzato prima che parta il programma.
Si deve inoltre notare che la pila in assembler si estende \textit{verso il basso}: aggiungere alla pila significa decrementare ESP, e rimuovere dalla pila significa incrementare ESP.
I frame successivi della pila si vanno a disporre via via sotto (o "a sinistra") del frame corrente.

Per lavorare sulla pila si usano le istruzioni:

\subsubsection{PUSH}
\begin{itemize}
	\item \textbf{Formato:} \texttt{PUSH source}
	\item \textbf{Azione:} decrementa ESP e copia il sorgente nell'indirizzo puntato da ESP.
		Il sorgente deve essere a 16 bit o a 32 bit.
		Nello specifico, compie le seguenti azioni:
		\begin{itemize}
			\item Decrementa l'indirizzo contenuto nel registro ESP di 2 o 4;
			\item Memorizza una copia dell'operando sorgente nella word o long il cui indirizzo è contenuto in ESP.
		\end{itemize}
	\item \textbf{Flag:} nessuno.
\end{itemize}

		\begin{table}[h!]
			\center \rowcolors{2}{white}{black!10}
			\begin{tabular} { c | p{5cm} }
				\bfseries Operandi & \bfseries Esempi \\
				\hline 
				Memoria & \texttt{PUSHW 0x3214200A} \\ 
				Immediato & \texttt{PUSHL \$0x4871A000} \\ 
				Registro Generale & \texttt{PUSH \%BX}
			\end{tabular}
		\end{table}

\subsubsection{POP}
\begin{itemize}
	\item \textbf{Formato:} \texttt{POP destination}
	\item \textbf{Azione:} copia una word o un long dall'indirzzo puntato dall'ESP nel destinatario e incrementa ESP.
		Nello specifico compie le seguenti azioni:
		\begin{itemize}
			\item Sostituisce all'operando destinatario una copia del contenuto nella word o long il cui indirizzo è contenuto in ESP;
			\item Incrementa di due o quattro l'indirizzo contenuto in ESP, rimuovendo la word o il long copiato.
		\end{itemize}
	\item \textbf{Flag:} nessuno.
\end{itemize}
	
		\begin{table}[h!]
			\center \rowcolors{2}{white}{black!10}
			\begin{tabular} { c | p{5cm} }
				\bfseries Operandi & \bfseries Esempi \\
				\hline 
				Memoria & \texttt{POPW 0x02AB2000} \\ 
				Registro Generale & \texttt{POP \%BX}
			\end{tabular}
		\end{table}

\par\medskip

\noindent
\textsf{\textbf{Dati temporanei nella pila}} \\
Solitamente la pila viene usata per memorizzare dati temporanei, visto che i registri sono pochi e spesso hanno scopi diversi in momenti diversi. Ad esempio:

\begin{lstlisting}[style=codestyle]	
# sto usando %EAX, mi serve un dato da una porta
PUSH %EAX
IN 0x001A, %AL
...
POP %EAX # ritorno da dove ero
\end{lstlisting}

\subsubsection{PUSHAD}
\begin{itemize}
	\item \textbf{Formato:} \texttt{PUSHAD}
	\item \textbf{Azione:}: salva nella pila corrente una copia degli 8 registri generali a 32 bit, nell'ordine: EAX, ECX, EDX, EBX, ESP, EBP, ESI, EDI.
	\item \textbf{Flag:} nessuno.
\end{itemize}

\subsubsection{POPAD}
\begin{itemize}
	\item \textbf{Formato:} \texttt{POPAD}
	\item \textbf{Azione:}: copia dalla pila corrente gli 8 registri generali a 32 bit, nell'ordine: EAX, ECX, EDX, EBX, ESP, EBP, ESI, EDI. 
	\item \textbf{Flag:} nessuno.
\end{itemize}

\subsection{Istruzioni aritmetiche}
Molte operazioni aritmetiche di base non distinguono numeri naturali e numeri interi, distinzione che viene fatta solo per moltiplicazioni e divisioni.

Le operazioni possono modificare i flag, e in questo caso i flag da controllare dipenderanno dal tipo di numeri su cui si è fatta l'operazione (informazione nota soltanto al programmatore).

Abbiamo quindi che un'operazione aritmetica si svolge seguendo i passi:
\begin{itemize}
	\item Si esegue l'operazione;
	\item Si controllano i flag interessati (OF, SF e ZF sugli interi, CF e ZF sui naturali) per verificarne l'esito.
\end{itemize}

Vediamo quindi le operazioni aritmetiche:

\subsubsection{ADD}
\begin{itemize}
	\item \textbf{Formato:} \texttt{ADD source, destination}
	\item \textbf{Azione:} modifica l'operando destinatario sommandovi l'operando sorgente.
		Il risultato è consistente sia che si interpretino i numeri come naturali, che come interi.
	\item \textbf{Flag:} attiva CF se, interpretando i numeri come naturali, si è verificato un riporto; attiva OF se, interpretando gli operandi come interi, si è verificato un traboccamento.
		Inoltre attiva opportunamente ZF e SF se il numero è rispettivamente zero o negativo (in complemento a 2).
\end{itemize}

		\begin{table}[h!]
			\center \rowcolors{2}{white}{black!10}
			\begin{tabular} { c | p{5cm} }
				\bfseries Operandi & \bfseries Esempi \\
				\hline
				Memoria, Registro Generale & \texttt{ADD 0x00002000, \%EDX} \\ 
				Registro Generale, Memoria & \texttt{ADD \%CL, 0x12AB1024} \\ 
				Registro Generale, Registro Generale & \texttt{ADD \%AX, \%DX} \\ 
				Immediato, Memoria & \texttt{ADDB \$0x5B, (\%EDI)} \\ 
				Immediato, Registro Generale & \texttt{ADD \$0x54A3, \%AX}
			\end{tabular}
		\end{table}

\par\medskip
\noindent
\textbf{\textsf{Funzionamento della ADD}} \\
Il passo elementare di una somma consiste nel sommare due addendi (propriamente due cifre degli addendi) e un riporto entrante per produrre:
	\begin{itemize}
		\item Una cifra;
		\item Un riporto uscente (cioè il riporto entrante per il prossimo passo).
	\end{itemize}
L'ultimo riporto, se non entra in memoria, attiva il carry flag (CF).

L'operazione di somma ha lo stesso effetto sia su naturali che su interi in complemento a 2: la differenza sta nel controllo dell'attivazione dei flag.
Il carry flag non ha infatti alcun significato nella somma fra interi: dobbiamo controllare l'OF.

In generale, si ha overflow (OF) quando il risultato esce dall'intervallo di rappresentabilità.
Si può capire se si è verificato un overflow controllando i segni degli operandi:
\begin{itemize}
	\item \textbf{Segni discordi:} non c'é overflow;
	\item \textbf{Segni concordi:} il risultato è concorde se è concorde con gli operandi.
\end{itemize}
La ADD imposta quindi OF secondo queste regole.
Il ZF viene poi impostato se il risultato è fatto da tutti zeri, e il SF viene impostato se il MSB è uno.

\subsubsection{INCREMENT}
\begin{itemize}
	\item \textbf{Formato:} \texttt{INC destination}
	\item \textbf{Azione:} equivale all'istruzione \texttt{ADD \$1, destination}. 
	\item \textbf{Flag:} modifica tutti i flag di ADD tranne CF (il riporto).
\end{itemize}

		\begin{table}[H]
			\center \rowcolors{2}{white}{black!10}
			\begin{tabular} { c | c }
				\bfseries Operandi & \bfseries Esempi \\
				\hline 
				Memoria & \texttt{INCB (\%ESI)} \\
				Registro Generale & \texttt{INC \%CX}
			\end{tabular}
		\end{table}

Quest'istruzione è più compatta di ADD, e storicamente era anche più veloce.
Questo deriva dal fatto che la circuiteria che implementava l'incremento era più efficiente di quella che implementa le somme.

\subsubsection{SUBTRACT}
\begin{itemize}
	\item \textbf{Formato:} \texttt{SUB source, destination}
	\item \textbf{Azione:} modifica l'operando destinatario sottraendovi l'operando sorgente. 
		Il risultato è consistente sia che si interpretino i numeri come naturali, che come interi.
	\item \textbf{Flag:} attiva CF se, interpretando i numeri come naturali, si è verificato un riporto; attiva OF se, interpretando gli operandi come interi, si è verificato un traboccamento.
\end{itemize}

		\begin{table}[h!]
			\center \rowcolors{2}{white}{black!10}
			\begin{tabular} { c | p{5cm} }
				\bfseries Operandi & \bfseries Esempi \\
				\hline
				Memoria, Registro Generale & \texttt{SUB 0x00002000, \%EDX} \\ 
				Registro Generale, Memoria & \texttt{SUB \%CL, 0x12AB1024} \\ 
				Registro Generale, Registro Generale & \texttt{SUB \%AX, \%DX} \\ 
				Immediato, Memoria & \texttt{SUBB \$0x5B, (\%EDI)} \\ 
				Immediato, Registro Generale & \texttt{SUB \$0x54A3, \%AX}
			\end{tabular}
		\end{table}

\par\medskip
\noindent
\textbf{\textsf{Funzionamento della SUBTRACT}} \\
Il passo elementare della sottrazione è effettivamente il contrario di quello della somma: si sottraggono il sottraendo e un prestito entrante al minuendo, producendo:
\begin{itemize}
	\item Una cifra;
	\item Un prestito uscente.
\end{itemize}

Il carry flag (CF) memorizza il prestito.
Se alla fine dell'operazione il CF è impostato, significa che il risultato è un numero intero.

Questo funziona anche sugli interi: in questo caso, come prima, non si controlla il CF, ma l'OF, che conterrà la seguente informazione:
\begin{itemize}
	\item La differenza di numeri concordi è sempre rappresentabile;
	\item La differenza di numeri discordi è rappresentabile solo se il risultato ha il segno del minuendo.
\end{itemize}

Il ZF e il SF vengono attivati secondo le regole già note.

\subsubsection{DECREMENT}
\begin{itemize}
	\item \textbf{Formato:} \texttt{DEC destination}
	\item \textbf{Azione:} equivale all'istruzione \texttt{SUB \$1, destination}. 
	\item \textbf{Flag:} modifica tutti i flag di SUBTRACT tranne CF (il prestito).
\end{itemize}

		\begin{table}[h!]
			\center \rowcolors{2}{white}{black!10}
			\begin{tabular} { c | c }
				\bfseries Operandi & \bfseries Esempi \\
				\hline 
				Memoria & \texttt{DECB (\%EDI)} \\
				Registro Generale & \texttt{DEC \%CX}
			\end{tabular}
		\end{table}

\subsubsection{ADD WITH CARRY}
\begin{itemize}
	\item \textbf{Formato:} \texttt{ADC source, destination}
	\item \textbf{Azione:} modifica l'operando destinatario sommandovi sia l'operando sorgente sia il contenuto del flag CF.
	\item \textbf{Flag:} modifica tutti i flag come ADD. 
\end{itemize}

\begin{table}[h!]
			\center \rowcolors{2}{white}{black!10}
			\begin{tabular} { c | p{5cm} }
				\bfseries Operandi & \bfseries Esempi \\
				\hline
				Memoria, Registro Generale & \texttt{ADC 0x00002000, \%EDX} \\ 
				Registro Generale, Memoria & \texttt{ADC \%CL, 0x12AB1024} \\ 
				Registro Generale, Registro Generale & \texttt{ADC \%AX, \%DX} \\ 
				Immediato, Memoria & \texttt{ADCB \$0x5B, (\%EDI)} \\ 
				Immediato, Registro Generale & \texttt{ADC \$0x54A3, \%AX}
			\end{tabular}
		\end{table}

Quest'istruzione è utile per effettuare somme di numeri più grandi di 32 bit.
In questo caso si:
\begin{itemize}
	\item Effettua la somma dei 32 bit meno significativi con ADD;
	\item Sommano i successivi 32 bit con ADC portandosi quindi dietro il carry.
\end{itemize}

\subsubsection{SUBTRACT WITH BORROW}
\begin{itemize}
	\item \textbf{Formato:} \texttt{SBB source, destination}
	\item \textbf{Azione:} modifica l'operando destinatario sottraendovi sia l'operando sorgente sia il contenuto del flag CF.
	\item \textbf{Flag:} modifica tutti i flag come SUBTRACT. 
\end{itemize}

		\begin{table}[H]
		\center \rowcolors{2}{white}{black!10}
			\begin{tabular} { c | p{5cm} }
				\bfseries Operandi & \bfseries Esempi \\
				\hline
				Memoria, Registro Generale & \texttt{SBB 0x00002000, \%EDX} \\ 
				Registro Generale, Memoria & \texttt{SBB \%CL, 0x12AB1024} \\ 
				Registro Generale, Registro Generale & \texttt{SBB \%AX, \%DX} \\ 
				Immediato, Memoria & \texttt{SBBB \$0x255B, (\%EDI)} \\ 
				Immediato, Registro Generale & \texttt{SBB \$0x54A3, \%AX}
			\end{tabular}
		\end{table}

Come ormai dovrebbe essere chiaro, è la duale dell'ADC, e si usa per effettuare sottrazioni di numeri più grandi di 32 bit.

\subsubsection{NEGATE}
\begin{itemize}
	\item \textbf{Formato:} \texttt{NEG destination}
	\item \textbf{Azione:} interpreta l'operando destinatario come un numero intero e lo sostituisce con il suo opposto in complemento a 2. 
	\item \textbf{Flag:} quando l'operazione non è possibile (l'intervallo di rappresentabilità degli interi in complemento a 2 non è simmetrico) imposta il flag OF.
		Imposta inoltre il flag CF quando l'operando è diverso da zero, e tutti gli altri flag in base a nullità e segno del risultato. 
\end{itemize}

		\begin{table}[H]
		\center \rowcolors{2}{white}{black!10}
			\begin{tabular} { c | p{5cm} }
				\bfseries Operandi & \bfseries Esempi \\
				\hline
				Memoria & \texttt{NEGB (\%EDI)} \\ 
				Registro Generale & \texttt{NEG \%CX}
			\end{tabular}
		\end{table}


\par\medskip
\noindent
\textbf{\textsf{Funzionamento della NEGATE}} \\
L'opposto di un numero $X$ in complemento a due è:
$$
-X = \bar{X} + 1
$$

Si ricordi che questo ha senso \textit{solamente} se il numero è rappresentato in complemento a due.

\subsubsection{COMPARE}
\begin{itemize}
	\item \textbf{Formato:} \texttt{CMP source, destination}
	\item \textbf{Azione:} verifica se l'operando destinatario è maggiore, uguale o minore dell'operando sorgente, sia interpretando gli operandi come naturali che come interi, e aggiorna i flag di conseguenza.
		Più propriamente, la compare si comporta come la SUB, ma senza sovrascrivere nessuno degli operandi.
	\item \textbf{Flag:} come la SUB. 
\end{itemize}


		\begin{table}[H]
		\center \rowcolors{2}{white}{black!10}
			\begin{tabular} { c | p{5cm} }
				\bfseries Operandi & \bfseries Esempi \\
				\hline
				Memoria, Registro Generale & \texttt{CMP 0x00002000, \%EDX} \\ 
				Registro Generale, Memoria & \texttt{CMP \%CL, 0x12AB1024} \\ 
				Registro Generale, Registro Generale & \texttt{CMP \%AX, \%DX} \\ 
				Immediato, Memoria & \texttt{CMPB \$0x255B, (\%EDI)} \\ 
				Immediato, Registro Generale & \texttt{CMP \$0x54A3, \%AX}
			\end{tabular}
		\end{table}

\subsubsection{Funzionamento della COMPARE}
Solitamente la CMP si usa nei salti condizionati come:
\begin{lstlisting}[style=codestyle]	
CMP %AX, %BX
JCOND # salto condizionato
\end{lstlisting}
\noindent
Ciò che fa la CMP è effettivamente creare un'oggetto temporaneo:
$$
\text{tmp} = \text{dest} - \text{source}
$$
che viene poi rimosso.

I flag restano però aggiornati, e questo valore può essere interpretato correttamente dalla JE per effettuare un salto condizionale.


\subsection{Moltiplicazioni}
Le moltiplicazioni, a differenza delle somme e delle differenze, sono diverse fra naturali ed interi.
Bisogna inoltre notare che le dimensioni il risultato della somma di un numero a $n$ cifre sta su $n$ o $n+1$ cifre, mentre il prodotto di due numeri a $n$ cifre sta su $2n$ cifre.
In altre parole, il numero di bit necessari a memorizzare il risultato non è più confrontabile con quello degli operatori.


\subsubsection{MULTIPLY}
\begin{itemize}
	\item \textbf{Formato:} \texttt{MUL source}
	\item \textbf{Azione:} considera l'operando sorgente come un moltiplicando, l'operando destinatario (implicito) come un moltiplicatore, e effettua la moltiplicazione assumendo i numeri naturali. Nello specifico:
	\begin{itemize}
		\item Sorgente a 8 bit, si ha $\text{AX} = \text{AL} \times \text{source}$;
		\item Sorgente a 16 bit, si ha $\text{DX}\_\text{AX} = \text{AX} \times \text{source}$;
		\item Sorgente a 32 bit, si ha $\text{EDX}\_\text{EAX} = \text{EAX} \times \text{source}$.
	\end{itemize}
	\item \textbf{Flag:} imposta CF e OF se il risultato non sta nel numero di bit di source. SF e ZF sono indefiniti.
\end{itemize}

		\begin{table}[H]
		\center \rowcolors{2}{white}{black!10}
			\begin{tabular} { c | p{5cm} }
				\bfseries Operandi & \bfseries Esempi \\
				\hline
				Memoria & \texttt{MULB (\%ESI)} \\ 
				Registro Generale & \texttt{MUL \%ECX}
			\end{tabular}
		\end{table}

\subsubsection{INTEGER MULTIPLY}
\begin{itemize}
	\item \textbf{Formato:} \texttt{MUL source}
	\item \textbf{Azione:} considera l'operando sorgente come un moltiplicando, l'operando destinatario (implicito) come un moltiplicatore, e effettua la moltiplicazione assumendo i numeri interi. Nello specifico:
	\begin{itemize}
		\item Sorgente a 8 bit, si ha $\text{AX} = \text{AL} \times \text{source}$;
		\item Sorgente a 16 bit, si ha $\text{DX}\_\text{AX} = \text{AX} \times \text{source}$;
		\item Sorgente a 32 bit, si ha $\text{EDX}\_\text{EAX} = \text{EAX} \times \text{source}$.
	\end{itemize}
	\item \textbf{Flag:} li imposta tutti, ma non è attendibile.
\end{itemize}

		\begin{table}[H]
		\center \rowcolors{2}{white}{black!10}
			\begin{tabular} { c | p{5cm} }
				\bfseries Operandi & \bfseries Esempi \\
				\hline
				Memoria & \texttt{IMULB (\%ESI)} \\ 
				Registro Generale & \texttt{IMUL \%ECX}
			\end{tabular}
		\end{table}

\par\medskip
\noindent
\textbf{\textsf{Funzionamento delle MULTIPLY e INTEGER MULTIPLY}} \\
Queste operazioni hanno sia un operando che il destinatario impliciti, in base al tipo dell'operando fornito.
Questo deriva dal fatto che il risultato di una moltiplicazione raramente sta nello stesso numero di bit dei fattori.
Di preciso, abbiamo visto i 3 tipi di moltiplicazione concessi:
\begin{itemize}
	\item Sorgente a 8 bit, si ha $\text{AX} = \text{AL} \times \text{source}$;
	\item Sorgente a 16 bit, si ha $\text{DX}\_\text{AX} = \text{AX} \times \text{source}$;
	\item Sorgente a 32 bit, si ha $\text{EDX}\_\text{EAX} = \text{EAX} \times \text{source}$.
\end{itemize}

La differenza fra le prime due operazioni e l'ultima, in particolare con sorgente a 16 bit, che usa una due registri da 16 bit separati, ha principalmente motivi storici (il registro EAX è stato introdotto dopo).

Si può rimettere il valore dai due registri a 16 bit in un registro a 32 bit attraverso la pila:
\begin{lstlisting}[style=codestyle]	
PUSH \%DX
PUSH \%AX
POP \%EAX
\end{lstlisting}

\end{document}


\documentclass[a4paper,11pt]{article}
\usepackage[a4paper, margin=8em]{geometry}

% usa i pacchetti per la scrittura in italiano
\usepackage[french,italian]{babel}
\usepackage[T1]{fontenc}
\usepackage[utf8]{inputenc}
\frenchspacing 

% usa i pacchetti per la formattazione matematica
\usepackage{amsmath, amssymb, amsthm, amsfonts}

% usa altri pacchetti
\usepackage{gensymb}
\usepackage{hyperref}
\usepackage{standalone}

% imposta il titolo
\title{Appunti Reti Logiche}
\author{Luca Seggiani}
\date{2024}

% imposta lo stile
% usa helvetica
\usepackage[scaled]{helvet}
% usa palatino
\usepackage{palatino}
% usa un font monospazio guardabile
\usepackage{lmodern}

\renewcommand{\rmdefault}{ppl}
\renewcommand{\sfdefault}{phv}
\renewcommand{\ttdefault}{lmtt}

% disponi il titolo
\makeatletter
\renewcommand{\maketitle} {
	\begin{center} 
		\begin{minipage}[t]{.8\textwidth}
			\textsf{\huge\bfseries \@title} 
		\end{minipage}%
		\begin{minipage}[t]{.2\textwidth}
			\raggedleft \vspace{-1.65em}
			\textsf{\small \@author} \vfill
			\textsf{\small \@date}
		\end{minipage}
		\par
	\end{center}

	\thispagestyle{empty}
	\pagestyle{fancy}
}
\makeatother

% disponi teoremi
\usepackage{tcolorbox}
\newtcolorbox[auto counter, number within=section]{theorem}[2][]{%
	colback=blue!10, 
	colframe=blue!40!black, 
	sharp corners=northwest,
	fonttitle=\sffamily\bfseries, 
	title=Teorema~\thetcbcounter: #2, 
	#1
}

% disponi definizioni
\newtcolorbox[auto counter, number within=section]{definition}[2][]{%
	colback=red!10,
	colframe=red!40!black,
	sharp corners=northwest,
	fonttitle=\sffamily\bfseries,
	title=Definizione~\thetcbcounter: #2,
	#1
}

% disponi codice
\usepackage{listings}
\usepackage[table]{xcolor}

\definecolor{codegreen}{rgb}{0,0.6,0}
\definecolor{codegray}{rgb}{0.5,0.5,0.5}
\definecolor{codepurple}{rgb}{0.58,0,0.82}
\definecolor{backcolour}{rgb}{0.95,0.95,0.92}

\lstdefinestyle{codestyle}{
		backgroundcolor=\color{black!5}, 
		commentstyle=\color{codegreen},
		keywordstyle=\bfseries\color{magenta},
		numberstyle=\sffamily\tiny\color{black!60},
		stringstyle=\color{green!50!black},
		basicstyle=\ttfamily\footnotesize,
		breakatwhitespace=false,         
		breaklines=true,                 
		captionpos=b,                    
		keepspaces=true,                 
		numbers=left,                    
		numbersep=5pt,                  
		showspaces=false,                
		showstringspaces=false,
		showtabs=false,                  
		tabsize=2
}

\lstdefinestyle{shellstyle}{
		backgroundcolor=\color{black!5}, 
		basicstyle=\ttfamily\footnotesize\color{black}, 
		commentstyle=\color{black}, 
		keywordstyle=\color{black},
		numberstyle=\color{black!5},
		stringstyle=\color{black}, 
		showspaces=false,
		showstringspaces=false, 
		showtabs=false, 
		tabsize=2, 
		numbers=none, 
		breaklines=true
}


\lstdefinelanguage{assembler}{
  keywords={AAA, AAD, AAM, AAS, ADC, ADCB, ADCW, ADCL, ADD, ADDB, ADDW, ADDL, AND, ANDB, ANDW, ANDL,
        ARPL, BOUND, BSF, BSFL, BSFW, BSR, BSRL, BSRW, BSWAP, BT, BTC, BTCB, BTCW, BTCL, BTR, 
        BTRB, BTRW, BTRL, BTS, BTSB, BTSW, BTSL, CALL, CBW, CDQ, CLC, CLD, CLI, CLTS, CMC, CMP,
        CMPB, CMPW, CMPL, CMPS, CMPSB, CMPSD, CMPSW, CMPXCHG, CMPXCHGB, CMPXCHGW, CMPXCHGL,
        CMPXCHG8B, CPUID, CWDE, DAA, DAS, DEC, DECB, DECW, DECL, DIV, DIVB, DIVW, DIVL, ENTER,
        HLT, IDIV, IDIVB, IDIVW, IDIVL, IMUL, IMULB, IMULW, IMULL, IN, INB, INW, INL, INC, INCB,
        INCW, INCL, INS, INSB, INSD, INSW, INT, INT3, INTO, INVD, INVLPG, IRET, IRETD, JA, JAE,
        JB, JBE, JC, JCXZ, JE, JECXZ, JG, JGE, JL, JLE, JMP, JNA, JNAE, JNB, JNBE, JNC, JNE, JNG,
        JNGE, JNL, JNLE, JNO, JNP, JNS, JNZ, JO, JP, JPE, JPO, JS, JZ, LAHF, LAR, LCALL, LDS,
        LEA, LEAVE, LES, LFS, LGDT, LGS, LIDT, LMSW, LOCK, LODSB, LODSD, LODSW, LOOP, LOOPE,
        LOOPNE, LSL, LSS, LTR, MOV, MOVB, MOVW, MOVL, MOVSB, MOVSD, MOVSW, MOVSX, MOVSXB,
        MOVSXW, MOVSXL, MOVZX, MOVZXB, MOVZXW, MOVZXL, MUL, MULB, MULW, MULL, NEG, NEGB, NEGW,
        NEGL, NOP, NOT, NOTB, NOTW, NOTL, OR, ORB, ORW, ORL, OUT, OUTB, OUTW, OUTL, OUTSB, OUTSD,
        OUTSW, POP, POPA, POPAD, POPF, POPFD, PUSH, PUSHA, PUSHAD, PUSHF, PUSHFD, RCL, RCLB, RCLW,
        RCLL, RCR, RCRB, RCRW, RCRL, RDMSR, RDPMC, RDTSC, REP, REPE, REPNE, RET, ROL, ROLB, ROLW,
        ROLL, ROR, RORB, RORW, RORL, SAHF, SAL, SALB, SALW, SALL, SAR, SARB, SARW, SARL, SBB,
        SBBB, SBBW, SBBL, SCASB, SCASD, SCASW, SETA, SETAE, SETB, SETBE, SETC, SETE, SETG, SETGE,
        SETL, SETLE, SETNA, SETNAE, SETNB, SETNBE, SETNC, SETNE, SETNG, SETNGE, SETNL, SETNLE,
        SETNO, SETNP, SETNS, SETNZ, SETO, SETP, SETPE, SETPO, SETS, SETZ, SGDT, SHL, SHLB, SHLW,
        SHLL, SHLD, SHR, SHRB, SHRW, SHRL, SHRD, SIDT, SLDT, SMSW, STC, STD, STI, STOSB, STOSD,
        STOSW, STR, SUB, SUBB, SUBW, SUBL, TEST, TESTB, TESTW, TESTL, VERR, VERW, WAIT, WBINVD,
        XADD, XADDB, XADDW, XADDL, XCHG, XCHGB, XCHGW, XCHGL, XLAT, XLATB, XOR, XORB, XORW, XORL},
  keywordstyle=\color{blue}\bfseries,
  ndkeywordstyle=\color{darkgray}\bfseries,
  identifierstyle=\color{black},
  sensitive=false,
  comment=[l]{\#},
  morecomment=[s]{/*}{*/},
  commentstyle=\color{purple}\ttfamily,
  stringstyle=\color{red}\ttfamily,
  morestring=[b]',
  morestring=[b]"
}

\lstset{language=assembler, style=codestyle}

% disponi sezioni
\usepackage{titlesec}

\titleformat{\section}
	{\sffamily\Large\bfseries} 
	{\thesection}{1em}{} 
\titleformat{\subsection}
	{\sffamily\large\bfseries}   
	{\thesubsection}{1em}{} 
\titleformat{\subsubsection}
	{\sffamily\normalsize\bfseries} 
	{\thesubsubsection}{1em}{}

% tikz
\usepackage{tikz}

% float
\usepackage{float}

% grafici
\usepackage{pgfplots}
\pgfplotsset{width=10cm,compat=1.9}

% disponi alberi
\usepackage{forest}

\forestset{
	rectstyle/.style={
		for tree={rectangle,draw,font=\large\sffamily}
	},
	roundstyle/.style={
		for tree={circle,draw,font=\large}
	}
}

% disponi algoritmi
\usepackage{algorithm}
\usepackage{algorithmic}
\makeatletter
\renewcommand{\ALG@name}{Algoritmo}
\makeatother

% disponi numeri di pagina
\usepackage{fancyhdr}
\fancyhf{} 
\fancyfoot[L]{\sffamily{\thepage}}

\makeatletter
\fancyhead[L]{\raisebox{1ex}[0pt][0pt]{\sffamily{\@title \ \@date}}} 
\fancyhead[R]{\raisebox{1ex}[0pt][0pt]{\sffamily{\@author}}}
\makeatother

\begin{document}
% sezione (data)
\section{Lezione del 27-09-24}

% stili pagina
\thispagestyle{empty}
\pagestyle{fancy}

% testo
\subsection{Divisioni}
La divisone è l'operazione più complessa fra le 4 operazioni aritmetiche fondamentali.
I risultati, di base, sono due: \textbf{quoziente} e \textbf{resto}.
Inoltre, l'operazione non è ben definita quando il divisore vale 0.

Facciamo innanzitutto delle considerazioni di dimensione dei risultati:
$$
X / Y \rightarrow (Q, R), \quad
0 \leq R \leq Y - 1, \quad
0 \leq Q \leq X
$$

In assembler, si assume il quoziente e il resto stiano sulla metà dei bit che rappresentano il dividendo.
Bisogna fare attenzione in quanto questo non è sempre il caso.

\subsubsection{DIVIDE}
\begin{itemize}
	\item \textbf{Formato:} \lstinline|DIV source|
	\item \textbf{Azione:} considera l'operando sorgente come un divisore, l'operando destinatario (implicito) come un dividendo, e effettua la divisione assumendo i numeri naturali. Nello specifico:
	\begin{itemize}
	\item Sorgente a 8 bit, si ha $\text{AL} = \text{AX} \div \text{source}$, e $ \text{AH} = \text{AX} \mod \text{source} $;
	\item Sorgente a 16 bit, si ha $\text{AX} = \text{DX\_AX} \div \text{source}$, e $ \text{DX} = \text{DX\_AX} \mod \text{source} $;
	\item Sorgente a 32 bit, si ha $\text{EAX} = \text{EDX\_EAX} \div \text{source}$, e $ \text{EDX} = \text{EDX\_EAX} \mod \text{source} $;
	\end{itemize}
		Nel caso il quoziente non sia esprimibile su un numero di bit pari a quello del divisore, allora si genera un'eccezione interna, che mette in esecuzione un sottoprogramma.
		Da lì in poi i risultati generati non sono più attendibili
	\item \textbf{Flag:} imposta tutti i bit, ma non è attendibile. 
\end{itemize}

		\begin{table}[H]
		\center \rowcolors{2}{white}{black!10}
			\begin{tabular} { c | p{10cm} }
				\bfseries Operandi & \bfseries Esempi \\
				\hline
				Memoria & \lstinline!DIVB (\%ESI)	\# AX destinazione implicita! \\ 
				Registro Generale & \lstinline|DIV \%ECX	\# EDX\_EAX destinazione implicita|
			\end{tabular}
		\end{table}

Attenzione: la destinazione implicita non è quella che va a contenere il risultato, ma quella che contiene il dividendo.
Negli esempi, le destinazioni quoziente resto sono rispettivamente AL e AH, EAX e EDX.

\subsubsection{INTEGER DIVIDE}
\begin{itemize}
	\item \textbf{Formato:} \lstinline|IMUL source|
	\item \textbf{Azione:} considera l'operando sorgente come un divisore, l'operando destinatario (implicito) come un dividendo, e effettua la divisione assumendo i numeri interi. Nello specifico:
	\begin{itemize}
	\item Sorgente a 8 bit, si ha $\text{AL} = \text{AX} / \text{source}$, e $ \text{AH} = \text{AX} \mod \text{source} $;
	\item Sorgente a 16 bit, si ha $\text{AX} = \text{DX\_AX} / \text{source}$, e $ \text{DX} = \text{DX\_AX} \mod \text{source} $;
	\item Sorgente a 32 bit, si ha $\text{EAX} = \text{EDX\_EAX} / \text{source}$, e $ \text{EDX} = \text{EDX\_EAX} \mod \text{source} $;
	\end{itemize}
	\item \textbf{Flag:} li imposta tutti, ma non è attendibile.
\end{itemize}

		\begin{table}[H]
		\center \rowcolors{2}{white}{black!10}
			\begin{tabular} { c | p{10cm} }
				\bfseries Operandi & \bfseries Esempi \\
				\hline
				Memoria & \lstinline|IDIVB (\%ESI)	\# AX destinazione implicita| \\ 
				Registro Generale & \lstinline|IDIV \%ECX	\# EDX\_EAX destinazione implicita|
			\end{tabular}
		\end{table}

Bisogna stare attenti ai segni della divisione intera.
Nella divisione intera il resto ha sempre il segno del dividendo, ed è minore in modulo del divisore.
Ciò significa che il quoziente si approssima sempre all'intero più vicino allo zero (\textit{per troncamento}).
Ad esempio, $-7 \ \mathrm{idiv} \ 3 = -2, -1$ e $7 \ \mathrm{idiv} \ -3 = -2, +1$.

\par\medskip
\noindent
\textbf{\textsf{Funzionamento delle DIVIDE e INTEGER DIVIDE}} \\
Esistono quindi, come per le moltiplicazioni, tre tipi di divisione, con operando e destinatario impliciti:
\begin{itemize}
	\item Sorgente a 8 bit, si ha $\text{AL} = \text{AX} / \text{source}$, e $ \text{AH} = \text{AX} \mod \text{source} $;
	\item Sorgente a 16 bit, si ha $\text{AX} = \text{DX\_AX} / \text{source}$, e $ \text{DX} = \text{DX\_AX} \mod \text{source} $;
	\item Sorgente a 32 bit, si ha $\text{EAX} = \text{EDX\_EAX} / \text{source}$, e $ \text{EDX} = \text{EDX\_EAX} \mod \text{source} $;
\end{itemize}

In tabella questo significa:

\begin{table}[h!]
	\center \rowcolors{2}{white}{black!10}
	\begin{tabular} { c | c | c | c | c }
		\bfseries Dim. sorgente (divisore) & \bfseries Dim. dividendo & \bfseries Dividendo & \bfseries Quoziente & \bfseries Resto \\ 
		\hline 
		8 bit & 16 bit & AX & AL & AH \\ 
		16 bit & 32 bit & DX\_AX & AX & DX \\ 
		32 bit & 64 bit & EDX\_EAX & EAX & EDX
	\end{tabular}
\end{table}

Se il quoziente non sta nel numero di bit previsto, viene sollevata un'eccezione, e il programma va in HALT.
Bisogna quindi decidere quali versioni usare tenendo conto delle dimensioni dei possibili quoziente.
Questo è importante in quanto non è cosi raro avere divisioni dove il quoziente non sta nella metà dei bit del dividendo, ad esempio:

\begin{lstlisting}[language=assembler,style=codestyle]	
MOV $3, %CL
MOV $15000, %AX
DIV %CL	# come metto 5000 su una locazione da 8 bit?
\end{lstlisting}

per risolvere il problema, dobbiamo costringere il processore ad usare un altro tipo di divisione, quindi:
\begin{lstlisting}[language=assembler,style=codestyle]	
MOV $3, %CX
MOV $15000, %AX
MOV $0, %DX	# devo ripulire DX, verra' usato il dividendo DX_AX 
DIV %CX	# il risultato va in AX, tutto bene 
\end{lstlisting}

\subsection{Note conclusive su moltiplicazioni e divisioni}
Dobbiamo quindi ricordarci, riguardo a moltiplicazioni e divisioni, di:
\begin{itemize}
	\item Scegliere con cura la versione che usiamo (sopratutto nel caso di divisioni dove il quoziente potrebbe non stare nella metà del numero di bit del dividendo);
	\item Azzerare di azzerare i registri DX o EDX prima della divisione, se è a più di 8 bit;
	\item Ricordare che il contenuto di DX o EDX viene modificato per operazioni su più di 8 bit.
\end{itemize}

\subsection{Estensione di campo}
Attraverso l'estensione di campo si rappresenta lo stesso numero su più cifre.
Questo è banale sui naturali (si aggiunge uno zero), ma più complicato per gli interi.
In questo caso si estende con il bit più significativo (quello di segno).

\subsubsection{CONVERT BYTE TO WORD}
\begin{itemize}
	\item \textbf{Formato:} \lstinline|CBX|
	\item \textbf{Azione:} interpreta il contenuto di AL come un numero intero a 8 bit, la rappresenta su 16 bit e quindi lo memorizza in AX.
	\item \textbf{Flag:} nessuno.
\end{itemize}

\subsubsection{CONVERT WORD TO DOUBLEWORD}
\begin{itemize}
	\item \textbf{Formato:} \lstinline|CWDE|
	\item \textbf{Azione:} interpreta il contenuto di AX come un numero intero a 16 bit, la rappresenta su 32 bit e quindi lo memorizza in EAX.
	\item \textbf{Flag:} nessuno.
\end{itemize}

Poniamo ad esempio di voler sommare due interi, uno in AX e l'altro in EBX:
\begin{lstlisting}[language=assembler,style=codestyle]	
MOV $-5, %AX
MOV $100000, $EBX
CWDE
ADD %EAX, %EBX
\end{lstlisting}

\subsection{Istruzioni di traslazione e rotazione}
Queste istruzioni variano l'ordine dei bit in un operando destinatario.
Hanno due formati: \lstinline|OCPODE source, destination| o \lstinline|OPCODE destination|.

Quando si specifica un sorgente, esso rappresenta il numero di iterazioni per cui si ripete l'operazione.
Il sorgente può essere ad indirizzamento immediato o essere il registro CL.
Inoltre, deve essere $\leq 31$ (sarebbe inutile fare $\geq32$ trasformazioni di 32 bit).
Quando è omesso, il sorgente vale di default 1.

\subsubsection{SHIFT LOGICAL LEFT}
\begin{itemize}
	\item \textbf{Formato:} \lstinline|SHL source, destination|
	\item \textbf{Azione:} interpreta l'operando sorgente come un naturale $n$, e per $n$ iterazioni:
		\begin{itemize}
			\item Sostituisce il bit in CF con il MSB;
			\item Sostituisce ogni bit (tranne il LSB) con il bit immediatamente a destra  (il meno significativo);
			\item Sostituisce il LSB con 0.
		\end{itemize}
	\item \textbf{Flag:} nessuno.
\end{itemize}

		\begin{table}[H]
		\center \rowcolors{2}{white}{black!10}
			\begin{tabular} { c | p{5cm} }
				\bfseries Operandi & \bfseries Esempi \\
				\hline
				Immediato, Registro Generale & \lstinline|SHL \$1, \%EAX| \\
				Immediato, Memoria & \lstinline|SHLB \$7, 0x00002000| \\
				Registro CL, Registro Generale & \lstinline|SHL \%CL, \%EAX| \\
				Registro CL, Memoria & \lstinline|SHLL \%CL, (\%EDI)| \\
				Memoria & \lstinline|SHLL (\%EDI)| \\ 
				Registro Generale & \lstinline|SHL \%AX|
			\end{tabular}
		\end{table}

La SHL è utile per effettuare moltiplicazioni per 2 (shift a sinistra in binario significa $\times 2$), tranne nei casi in cui il prodotto non sta sul numero di bit del destinatario.

Per questo si controlla il CF, facendo però attenzione che per $n$ iterazioni (date dal sorgente) vengono effettuati $n$ sovrascrizioni del CF.
Ergo, se la moltiplicazione fallisce, non sappiamo \textit{quando} fallisce.

\subsubsection{SHIFT ARITHMETIC LEFT}
\begin{itemize}
	\item \textbf{Formato:} \lstinline|SAL source, destination|
	\item \textbf{Azione:} è identica alla SHL. 
		Quindi equivale a moltiplicare per $2^\text{source}$.
	\item \textbf{Flag:} nessuno.
\end{itemize}

Esiste come duale della SAR, ma in questo caso non deve fare nulla di diverso dalla SHL.

\subsubsection{SHIFT LOGICAL RIGHT}
\begin{itemize}
	\item \textbf{Formato:} \lstinline|SHR source, destination|
	\item \textbf{Azione:} interpreta l'operando sorgente come un naturale $n$, e per $n$ iterazioni:
		\begin{itemize}
			\item Sostituisce il bit in CF con il LSB;
			\item Sostituisce ogni bit (tranne il MSB) con il bit immediatamente a sinistra (il più significativo);
			\item Sostituisce il MSB con 0.
		\end{itemize}
	\item \textbf{Flag:} nessuno.
\end{itemize}

		\begin{table}[H]
		\center \rowcolors{2}{white}{black!10}
			\begin{tabular} { c | p{5cm} }
				\bfseries Operandi & \bfseries Esempi \\
				\hline
				Immediato, Registro Generale & \lstinline|SHR \$1, \%EAX| \\
				Immediato, Memoria & \lstinline|SHRB \$7, 0x00002000| \\
				Registro CL, Registro Generale & \lstinline|SHR \%CL, \%EAX| \\
				Registro CL, Memoria & \lstinline|SHRL \%CL, (\%EDI)| \\
				Memoria & \lstinline|SHRL (\%EDI)| \\ 
				Registro Generale & \lstinline|SHR \%AX|
			\end{tabular}
		\end{table}

La SHR, come la SHL, è utile per effettuare divisioni per 2 (shift a destra in binario significa $\div 2$), concessa approssimazione del bit perso, tranne nei casi in cui il numero è un intero (lo 0 al MSB corrompe il segno). 
Per questo motivo si definisce la:

\subsubsection{SHIFT ARITHMETIC RIGHT}
\begin{itemize}
	\item \textbf{Formato:} \lstinline|SAR source, destination|
	\item \textbf{Azione:} è identica alla SHR, ma non sostituisce il MSB con 0, lasciandolo tale.
		Questo equivale a dividere per $2^\text{source}$.
	\item \textbf{Flag:} nessuno.
\end{itemize}

La SAR ci permette di dividere velocemente interi per 2, come avremmo fatto sui naturali con la SHR.

\subsubsection{Divisioni intere}
Le IDIV e SAR approssimano diversamente: la IDIV approssima per troncamento, mentre la SAR approssima sempre a sinistra.
Quindi, IDIV e SAR danno lo stesso quoziente solo quando il dividendo è positivo, o il resto nullo.

\subsection{Istruzioni di rotazione}
Le istruzioni di rotazione ruotano i bit, cioè effettuano uno shift con rientro dei bit in uscita dal lato opposto, con la possibilità di includere o meno CF nella rotazione.

\subsubsection{ROTATE LEFT}
\begin{itemize}
	\item \textbf{Formato:} \lstinline|ROL source, destination|
	\item \textbf{Azione:} interpreta l'operando sorgente come un naturale $n$, e per $n$ iterazioni ruota verso sinistra senza usare il carry.
	\item \textbf{Flag:} nessuno.
\end{itemize}

		\begin{table}[H]
		\center \rowcolors{2}{white}{black!10}
			\begin{tabular} { c | p{5cm} }
				\bfseries Operandi & \bfseries Esempi \\
				\hline
				Immediato, Registro Generale & \lstinline|ROL \$1, \%EAX| \\
				Immediato, Memoria & \lstinline|ROLB \$7, 0x00002000| \\
				Registro CL, Registro Generale & \lstinline|ROL \%CL, \%EAX| \\
				Registro CL, Memoria & \lstinline|ROLL \%CL, (\%EDI)| \\
				Memoria & \lstinline|ROLL (\%EDI)| \\ 
				Registro Generale & \lstinline|ROL \%AX|
			\end{tabular}
		\end{table}

\subsubsection{ROTATE RIGHT}
\begin{itemize}
	\item \textbf{Formato:} \lstinline|ROR source, destination|
	\item \textbf{Azione:} interpreta l'operando sorgente come un naturale $n$, e per $n$ iterazioni ruota verso destra senza usare il carry.
	\item \textbf{Flag:} nessuno.
\end{itemize}

		\begin{table}[H]
		\center \rowcolors{2}{white}{black!10}
			\begin{tabular} { c | p{5cm} }
				\bfseries Operandi & \bfseries Esempi \\
				\hline
				Immediato, Registro Generale & \lstinline|ROR \$1, \%EAX| \\
				Immediato, Memoria & \lstinline|RORB \$7, 0x00002000| \\
				Registro CL, Registro Generale & \lstinline|ROR \%CL, \%EAX| \\
				Registro CL, Memoria & \lstinline|RORL \%CL, (\%EDI)| \\
				Memoria & \lstinline|RORL (\%EDI)| \\ 
				Registro Generale & \lstinline|ROR \%AX|
			\end{tabular}
		\end{table}

\subsubsection{ROTATE CARRY LEFT}
\begin{itemize}
	\item \textbf{Formato:} \lstinline|RCL source, destination|
	\item \textbf{Azione:} interpreta l'operando sorgente come un naturale $n$, e per $n$ iterazioni ruota verso sinistra usando il carry.
	\item \textbf{Flag:} imposta il carry assumendolo a sinistra del MSB.
\end{itemize}

		\begin{table}[H]
		\center \rowcolors{2}{white}{black!10}
			\begin{tabular} { c | p{5cm} }
				\bfseries Operandi & \bfseries Esempi \\
				\hline
				Immediato, Registro Generale & \lstinline|RCL \$1, \%EAX| \\
				Immediato, Memoria & \lstinline|RCLB \$7, 0x00002000| \\
				Registro CL, Registro Generale & \lstinline|RCL \%CL, \%EAX| \\
				Registro CL, Memoria & \lstinline|RCLL \%CL, (\%EDI)| \\
				Memoria & \lstinline|RCLL (\%EDI)| \\ 
				Registro Generale & \lstinline|RCL \%AX|
			\end{tabular}
		\end{table}

\subsubsection{ROTATE CARRY RIGHT}
\begin{itemize}
	\item \textbf{Formato:} \lstinline|RCR source, destination|
	\item \textbf{Azione:} interpreta l'operando sorgente come un naturale $n$, e per $n$ iterazioni ruota verso destra usando il carry.
	\item \textbf{Flag:} imposta il carry assumendolo a destra del LSB.
\end{itemize}

		\begin{table}[H]
		\center \rowcolors{2}{white}{black!10}
			\begin{tabular} { c | p{5cm} }
				\bfseries Operandi & \bfseries Esempi \\
				\hline
				Immediato, Registro Generale & \lstinline|RCR \$1, \%EAX| \\
				Immediato, Memoria & \lstinline|RCRB \$7, 0x00002000| \\
				Registro CL, Registro Generale & \lstinline|RCR \%CL, \%EAX| \\
				Registro CL, Memoria & \lstinline|RCRL \%CL, (\%EDI)| \\
				Memoria & \lstinline|RCRL (\%EDI)| \\ 
				Registro Generale & \lstinline|RCR \%AX|
			\end{tabular}
		\end{table}

\subsection{Istruzioni logiche}
Queste istruzioni applicano gli operatori dell'algebra di Boole, e solitamente modificano flag.

\subsubsection{NOT}
\begin{itemize}
	\item \textbf{Formato:} \lstinline|NOT destination|
	\item \textbf{Azione:} modifica il destinatario applicandogli il NOT bit a bit. 
	\item \textbf{Flag:} nessuno. 
\end{itemize}

		\begin{table}[H]
		\center \rowcolors{2}{white}{black!10}
			\begin{tabular} { c | p{5cm} }
				\bfseries Operandi & \bfseries Esempi \\
				\hline
				Memoria & \lstinline|NOTL (\%ESI)| \\ 
				Registro Generale & \lstinline|NOT \%CX| 
			\end{tabular}
		\end{table}

\subsubsection{AND}
\begin{itemize}
	\item \textbf{Formato:} \lstinline|AND source, destination|
	\item \textbf{Azione:} modifica il destinatario applicando l'AND bit a bit degli operandi. 
	\item \textbf{Flag:} modifica tutti i flag (annulla CF e OF).
\end{itemize}

		\begin{table}[H]
		\center \rowcolors{2}{white}{black!10}
			\begin{tabular} { c | p{5cm} }
				\bfseries Operandi & \bfseries Esempi \\
				\hline
				Memoria, Registro Generale & \lstinline|AND 0x00002000, \%EDX| \\ 
				Registro Generale, Memoria & \lstinline|AND \%CL, 0x12AB1024| \\ 
				Registro Generale, Registro Generale & \lstinline|AND \%AX, \%DX| \\ 
				Immediato, Memoria & \lstinline|AND 5x5B, (\%EDI)| \\ 
				Immediato, Registro Generale & \lstinline|AND \$0x45AB54A3, \%EAX|
			\end{tabular}
		\end{table}

\subsubsection{OR}
\begin{itemize}
	\item \textbf{Formato:} \lstinline|OR source, destination|
	\item \textbf{Azione:} modifica il destinatario applicando l'OR bit a bit degli operandi. 
	\item \textbf{Flag:} modifica tutti i flag (annulla CF e OF).
\end{itemize}

		\begin{table}[H]
		\center \rowcolors{2}{white}{black!10}
			\begin{tabular} { c | p{5cm} }
				\bfseries Operandi & \bfseries Esempi \\
				\hline
				Memoria, Registro Generale & \lstinline|OR 0x00002000, \%EDX| \\ 
				Registro Generale, Memoria & \lstinline|OR \%CL, 0x12AB1024| \\ 
				Registro Generale, Registro Generale & \lstinline|OR \%AX, \%DX| \\ 
				Immediato, Memoria & \lstinline|OR 5x5B, (\%EDI)| \\ 
				Immediato, Registro Generale & \lstinline|OR \$0x45AB54A3, \%EAX|
			\end{tabular}
		\end{table}

\subsubsection{XOR}
\begin{itemize}
	\item \textbf{Formato:} \lstinline|XOR source, destination|
	\item \textbf{Azione:} modifica il destinatario applicando l'OR bit a bit degli operandi. 
	\item \textbf{Flag:} modifica tutti i flag (annulla CF e OF).
\end{itemize}

		\begin{table}[H]
		\center \rowcolors{2}{white}{black!10}
			\begin{tabular} { c | p{5cm} }
				\bfseries Operandi & \bfseries Esempi \\
				\hline
				Memoria, Registro Generale & \lstinline|XOR 0x00002000, \%EDX| \\ 
				Registro Generale, Memoria & \lstinline|XOR \%CL, 0x12AB1024| \\ 
				Registro Generale, Registro Generale & \lstinline|XOR \%AX, \%DX| \\ 
				Immediato, Memoria & \lstinline|XOR 5x5B, (\%EDI)| \\ 
				Immediato, Registro Generale & \lstinline|XOR \$0x45AB54A3, \%EAX|
			\end{tabular}
		\end{table}

\subsubsection{Uso delle istruzioni logiche}
Le istruzioni logiche vengono usate per operare su singoli bit degli operandi, usando uno specifico operatore sorgente immediato detto maschera (\textbf{bitmask}).
Nello specifico:
\begin{itemize}
	\item \textbf{AND:} 
		\begin{itemize}
			\item si usa per testare singoli bit di un operando.
			Ad esempio, si può implementare un salto condizionale se il quinto bit di AL vale zero:
			\begin{lstlisting}[language=assembler,style=codestyle]	
AND $0x20, %AL	# 0x20 = 00100000
JZ # vale zero
\end{lstlisting} 
			\item si usa per resettare singoli bit di un operando.
			Ad esempio, si può resettare il sesto bit di BH:
			\begin{lstlisting}[language=assembler,style=codestyle]	
AND $0xBF, $BH	# 0xBF = 10111111
\end{lstlisting}
			\item si usa per l'estensione di operandi \textit{naturali}.
				Ad esempio, si possono sommare due numeri naturali, di cui uno in AL e l'altro in EBX:
				\begin{lstlisting}[language=assembler,style=codestyle]	
MOV $5, $AL
MOV $100000, %EBX
AND $0x000000FF, $EAX
ADD %EAX, %EBX	
\end{lstlisting}
		\end{itemize} 
	\item \textbf{OR:} si usa per settare singoli bit di un operando.
		Ad esempio, si può settare il quarto bit di CL:
		\begin{lstlisting}[language=assembler,style=codestyle]	
OR $0x10, %CL	# =x10 = 00010000
\end{lstlisting}
	\item \textbf{XOR:}
		\begin{itemize}
			\item si usa per invertire singoli bit.
		Ad esempio, si può invertire il quinto bit del registro AH:
		\begin{lstlisting}[language=assembler,style=codestyle]	
XOR $0x20, %AH	# 0x20 = 00100000
\end{lstlisting}
	\item si usa per resettare registri.
		Ad esempio, si può resettare EAX come:
		\begin{lstlisting}[language=assembler,style=codestyle]	
XOR %EAX, %EAX	# equivale a dire MOV $0, %EAX, ma occupa 
								# 1 byte invece di 5
\end{lstlisting}
		\end{itemize}
\end{itemize}

\subsection{Istruzioni di controllo}
Le istruzioni di controllo permettono di alterare il flusso del programma, che altrimenti scorrerebbe normalmente in sequenza (le istruzioni vengono eseguite come vengono lette in memoria).

Conosciamo il ciclo fetch-execute: il processore carica un'istruzione, incrementa EIP, e la esegue.
Alcune istruzioni alterano il valore di EIP, implementando quindi alterazioni del flusso di esecuzione:
\begin{itemize}
	\item \textbf{Istruzioni di salto:} JMP, Jcon;
	\item \textbf{Istruzioni di gestione sottoprgrammi}: CALL, RET.
\end{itemize}

\subsubsection{JUMP}
\begin{itemize}
	\item \textbf{Formato:} \lstinline|JMP  \%EIP +/- displacement|, \lstinline|JMP *extended\_register|, \lstinline|JMP *memory|
	\item \textbf{Azione:} calcola un'indrizzo di salto e lo immette nel registro EIP. 
	\item \textbf{Flag:} nessuno. 
\end{itemize}

Solitamente le istruzioni di salto si riferiscono ad un nome simbolico, ed è quindi compito dell'assemblatore ricondurre la sintassi ad una delle forme sopra riportate.

\subsubsection{JUMP if CONDITION MET}
\begin{itemize}
	\item \textbf{Formato:} \lstinline|Jcon \%EIP +/- displacement|
	\item \textbf{Azione:} esamina il contenuto dei flag.
		Se da questo esame risulta che la condizione \textit{con} è soddisfatta, si comporta come \lstinline|JMP \%EIP +/- displacement|, altrimenti non fa nulla.
	\item \textbf{Flag:} nessuno. 
\end{itemize}

I prossimi paragrafi riguardano tutti i di condizione supportati.

\subsubsection{Condizioni sui flag}
Esistono le seguenti condizioni sui singoli flag:

\begin{table}[h!]
	\center \rowcolors{1}{white}{black!5}
	\begin{tabular} { c  p{10cm} }
		\bfseries Condizione & \bfseries Funzionamento \\
		\hline 
		JZ & Jump If Zero, la condizione è soddisfatta se ZF è impostato, ergo se il risultato dell'istruzione precedente è stato 0. \\ 
		JNZ & Jump If Not Zero, la condizione è soddisfatta se ZF non è impostato, ergo se il risultato dell'istruzione precedente non è stato 0. \\ 
		JC & Jump if Carry, la condizione è soddisfatta se CF è impostato. \\
		JNC & Jump if No Carry, la condizione è soddisfatta se CF non è impostato. \\ 
		JO & Jump if Overflow, la condizione è soddisfatta se OF è impostato. \\
		JNO & Jump if No Overflow, la condizione è soddisfatta se OF non è impostato. \\ 
		JS & Jump if Sign, la condizione è soddisfatta se SF è impostato. \\
		JNS & Jump if No Sign, la condizione è soddisfatta se SF non è impostato. \\ 
	\end{tabular}
\end{table}

\par\medskip
\noindent
\textbf{\textsf{Esempi}} \\
\begin{itemize}
	\item 
\begin{lstlisting}[language=assembler,style=codestyle]	
ADD %AX, %BX
JC ...
# continua
\end{lstlisting}
Se la somma dei contenuti di AX e BX presi come naturali non è rappresentabile su 16 bit, salta.

	\item 
\begin{lstlisting}[language=assembler,style=codestyle]	
ADD %AX, %BX
JO ...
# continua
\end{lstlisting}
Se la somma dei contenuti di AX e BX presi come interi non è rappresentabile su 16 bit, salta.

	\item 
\begin{lstlisting}[language=assembler,style=codestyle]	
SUB %AL, %BL
JS ...
# continua
\end{lstlisting}
Se la somma differenza dei contenuti di BL ed AL (in quest'ordine) presi come interi è negativa, salta.
\end{itemize}

\subsubsection{Condizioni sui naturali}
Esistono le seguenti condizioni sui confronti fra naturali:

\begin{table}[h!]
	\center \rowcolors{2}{white}{black!5}
	\begin{tabular} { c  p{10cm} }
		\bfseries Condizione & \bfseries Funzionamento \\
		\hline 
		JE & Jump if Equal, la condizione è soddisfatta se ZF contiene 1, cioè dopo CMP su due numeri uguali. \\
		JNE & Jump if Not Equal, la condizione è soddisfatta se ZF contiene 0, cioè dopo CMP su due numeri non uguali. \\ 
		JA & Jump if Above, la condizione è soddisfatta se CF contiene 0 e ZF contiene 0, cioè dopo CMP su un destinatario maggiore del sorgente. \\
		JAE & Jump if Above or Equal, la condizione è soddisfatta se CF contiene 0, cioè dopo CMP su un destinatario maggiore o uguale del sorgente. \\ 
		JB & Jump if Below, la condizione è soddisfatta se CF contiene 1, cioè dopo CMP su un destinatario minore del sorgente. \\
		JBE & Jump if Below or Equal, la condizione è soddisfatta se CF contiene 1 o ZF contiene 1, cioè dopo CMP su un destinatario minore o uguale del sorgente. \\ 
	\end{tabular}
\end{table}

Tutte queste condizioni seguono sempre una CMP, che aggiorna i flag in modo da permettere il confronto.
I risultati dei confronti possono sempre evincersi dai flag.

\par\medskip
\noindent
\textbf{\textsf{Esempi}} \\
\begin{itemize}
	\item 
\begin{lstlisting}[language=assembler,style=codestyle]	
CMP %AX, %BX
JAE ...
# continua
\end{lstlisting}
Se BX è maggiore o uguale di AX, presi come naturali, salta.

	\item 
\begin{lstlisting}[language=assembler,style=codestyle]	
CMP %EDX, %ECX
JB ...
# continua
\end{lstlisting}
Se ECX è minore stretto di EDX, presi come naturali, salta.
\end{itemize}

\subsubsection{Condizioni sugli interi}
Esistono le seguenti condizioni sui confronti fra interi:

\begin{table}[h!]
	\center \rowcolors{2}{white}{black!5}
	\begin{tabular} { c  p{10cm} }
		\bfseries Condizione & \bfseries Funzionamento \\
		\hline 
		JE & Jump if Equal, la condizione è soddisfatta se ZF contiene 1, cioè dopo CMP su due numeri uguali. \\
		JNE & Jump if Not Equal, la condizione è soddisfatta se ZF contiene 0, cioè dopo CMP su due numeri non uguali. \\ 
		JG & Jump if Greater, la condizione è soddisfatta se ZF contiene 0 e se SF è uguale a OF, cioè dopo CMP su un destinatario maggiore del sorgente. \\
		JGE & Jump if Greater or Equal, la condizione è soddisfatta se SF è uguale a OF, cioè dopo CMP su un destinatario maggiore o uguale del sorgente. \\ 
		JL & Jump if Less, la condizione è soddisfatta se SF è diverso da OF, cioè dopo CMP su un destinatario minore del sorgente. \\
		JLE & Jump if Less or Equal, la condizione è soddisfatta se ZF contiene 1 o se Sf è diverso da OF, cioè dopo CMP su un destinatario minore o uguale del sorgente. \\ 
	\end{tabular}
\end{table}

Come prima, queste operazioni seguono sempre una CMP ed evincono il risultato del confronto dai flag.

\par\medskip
\noindent
\textbf{\textsf{Esempi}} \\
\begin{itemize}
	\item 
\begin{lstlisting}[language=assembler,style=codestyle]	
CMP %AX, %BX
JGE ...
# continua
\end{lstlisting}
Se BX è maggiore o uguale di AX, presi come interi, salta.

	\item 
\begin{lstlisting}[language=assembler,style=codestyle]	
CMP %EDX, %ECX
JL ...
# continua
\end{lstlisting}
Se ECX è minore stretto di EDX, presi come interi, salta.
\end{itemize}
\end{document}


\documentclass[a4paper,11pt]{article}
\usepackage[a4paper, margin=8em]{geometry}

% usa i pacchetti per la scrittura in italiano
\usepackage[french,italian]{babel}
\usepackage[T1]{fontenc}
\usepackage[utf8]{inputenc}
\frenchspacing 

% usa i pacchetti per la formattazione matematica
\usepackage{amsmath, amssymb, amsthm, amsfonts}

% usa altri pacchetti
\usepackage{gensymb}
\usepackage{hyperref}
\usepackage{standalone}

% imposta il titolo
\title{Appunti Reti Logiche}
\author{Luca Seggiani}
\date{2024}

% imposta lo stile
% usa helvetica
\usepackage[scaled]{helvet}
% usa palatino
\usepackage{palatino}
% usa un font monospazio guardabile
\usepackage{lmodern}

\renewcommand{\rmdefault}{ppl}
\renewcommand{\sfdefault}{phv}
\renewcommand{\ttdefault}{lmtt}

% disponi il titolo
\makeatletter
\renewcommand{\maketitle} {
	\begin{center} 
		\begin{minipage}[t]{.8\textwidth}
			\textsf{\huge\bfseries \@title} 
		\end{minipage}%
		\begin{minipage}[t]{.2\textwidth}
			\raggedleft \vspace{-1.65em}
			\textsf{\small \@author} \vfill
			\textsf{\small \@date}
		\end{minipage}
		\par
	\end{center}

	\thispagestyle{empty}
	\pagestyle{fancy}
}
\makeatother

% disponi teoremi
\usepackage{tcolorbox}
\newtcolorbox[auto counter, number within=section]{theorem}[2][]{%
	colback=blue!10, 
	colframe=blue!40!black, 
	sharp corners=northwest,
	fonttitle=\sffamily\bfseries, 
	title=Teorema~\thetcbcounter: #2, 
	#1
}

% disponi definizioni
\newtcolorbox[auto counter, number within=section]{definition}[2][]{%
	colback=red!10,
	colframe=red!40!black,
	sharp corners=northwest,
	fonttitle=\sffamily\bfseries,
	title=Definizione~\thetcbcounter: #2,
	#1
}

% disponi codice
\usepackage{listings}
\usepackage[table]{xcolor}

\definecolor{codegreen}{rgb}{0,0.6,0}
\definecolor{codegray}{rgb}{0.5,0.5,0.5}
\definecolor{codepurple}{rgb}{0.58,0,0.82}
\definecolor{backcolour}{rgb}{0.95,0.95,0.92}

\lstdefinestyle{codestyle}{
		backgroundcolor=\color{black!5}, 
		commentstyle=\color{codegreen},
		keywordstyle=\bfseries\color{magenta},
		numberstyle=\sffamily\tiny\color{black!60},
		stringstyle=\color{green!50!black},
		basicstyle=\ttfamily\footnotesize,
		breakatwhitespace=false,         
		breaklines=true,                 
		captionpos=b,                    
		keepspaces=true,                 
		numbers=left,                    
		numbersep=5pt,                  
		showspaces=false,                
		showstringspaces=false,
		showtabs=false,                  
		tabsize=2
}

\lstdefinestyle{shellstyle}{
		backgroundcolor=\color{black!5}, 
		basicstyle=\ttfamily\footnotesize\color{black}, 
		commentstyle=\color{black}, 
		keywordstyle=\color{black},
		numberstyle=\color{black!5},
		stringstyle=\color{black}, 
		showspaces=false,
		showstringspaces=false, 
		showtabs=false, 
		tabsize=2, 
		numbers=none, 
		breaklines=true
}


\lstdefinelanguage{assembler}{
  keywords={AAA, AAD, AAM, AAS, ADC, ADCB, ADCW, ADCL, ADD, ADDB, ADDW, ADDL, AND, ANDB, ANDW, ANDL,
        ARPL, BOUND, BSF, BSFL, BSFW, BSR, BSRL, BSRW, BSWAP, BT, BTC, BTCB, BTCW, BTCL, BTR, 
        BTRB, BTRW, BTRL, BTS, BTSB, BTSW, BTSL, CALL, CBW, CDQ, CLC, CLD, CLI, CLTS, CMC, CMP,
        CMPB, CMPW, CMPL, CMPS, CMPSB, CMPSD, CMPSW, CMPXCHG, CMPXCHGB, CMPXCHGW, CMPXCHGL,
        CMPXCHG8B, CPUID, CWDE, DAA, DAS, DEC, DECB, DECW, DECL, DIV, DIVB, DIVW, DIVL, ENTER,
        HLT, IDIV, IDIVB, IDIVW, IDIVL, IMUL, IMULB, IMULW, IMULL, IN, INB, INW, INL, INC, INCB,
        INCW, INCL, INS, INSB, INSD, INSW, INT, INT3, INTO, INVD, INVLPG, IRET, IRETD, JA, JAE,
        JB, JBE, JC, JCXZ, JE, JECXZ, JG, JGE, JL, JLE, JMP, JNA, JNAE, JNB, JNBE, JNC, JNE, JNG,
        JNGE, JNL, JNLE, JNO, JNP, JNS, JNZ, JO, JP, JPE, JPO, JS, JZ, LAHF, LAR, LCALL, LDS,
        LEA, LEAVE, LES, LFS, LGDT, LGS, LIDT, LMSW, LOCK, LODSB, LODSD, LODSW, LOOP, LOOPE,
        LOOPNE, LSL, LSS, LTR, MOV, MOVB, MOVW, MOVL, MOVSB, MOVSD, MOVSW, MOVSX, MOVSXB,
        MOVSXW, MOVSXL, MOVZX, MOVZXB, MOVZXW, MOVZXL, MUL, MULB, MULW, MULL, NEG, NEGB, NEGW,
        NEGL, NOP, NOT, NOTB, NOTW, NOTL, OR, ORB, ORW, ORL, OUT, OUTB, OUTW, OUTL, OUTSB, OUTSD,
        OUTSW, POP, POPL, POPW, POPB, POPA, POPAD, POPF, POPFD, PUSH, PUSHL, PUSHW, PUSHB, PUSHA, 
				PUSHAD, PUSHF, PUSHFD, RCL, RCLB, RCLW,
        RCLL, RCR, RCRB, RCRW, RCRL, RDMSR, RDPMC, RDTSC, REP, REPE, REPNE, RET, ROL, ROLB, ROLW,
        ROLL, ROR, RORB, RORW, RORL, SAHF, SAL, SALB, SALW, SALL, SAR, SARB, SARW, SARL, SBB,
        SBBB, SBBW, SBBL, SCASB, SCASD, SCASW, SETA, SETAE, SETB, SETBE, SETC, SETE, SETG, SETGE,
        SETL, SETLE, SETNA, SETNAE, SETNB, SETNBE, SETNC, SETNE, SETNG, SETNGE, SETNL, SETNLE,
        SETNO, SETNP, SETNS, SETNZ, SETO, SETP, SETPE, SETPO, SETS, SETZ, SGDT, SHL, SHLB, SHLW,
        SHLL, SHLD, SHR, SHRB, SHRW, SHRL, SHRD, SIDT, SLDT, SMSW, STC, STD, STI, STOSB, STOSD,
        STOSW, STR, SUB, SUBB, SUBW, SUBL, TEST, TESTB, TESTW, TESTL, VERR, VERW, WAIT, WBINVD,
        XADD, XADDB, XADDW, XADDL, XCHG, XCHGB, XCHGW, XCHGL, XLAT, XLATB, XOR, XORB, XORW, XORL},
  keywordstyle=\color{blue}\bfseries,
  ndkeywordstyle=\color{darkgray}\bfseries,
  identifierstyle=\color{black},
  sensitive=false,
  comment=[l]{\#},
  morecomment=[s]{/*}{*/},
  commentstyle=\color{purple}\ttfamily,
  stringstyle=\color{red}\ttfamily,
  morestring=[b]',
  morestring=[b]"
}

\lstset{language=assembler, style=codestyle}

% disponi sezioni
\usepackage{titlesec}

\titleformat{\section}
	{\sffamily\Large\bfseries} 
	{\thesection}{1em}{} 
\titleformat{\subsection}
	{\sffamily\large\bfseries}   
	{\thesubsection}{1em}{} 
\titleformat{\subsubsection}
	{\sffamily\normalsize\bfseries} 
	{\thesubsubsection}{1em}{}

% tikz
\usepackage{tikz}

% float
\usepackage{float}

% grafici
\usepackage{pgfplots}
\pgfplotsset{width=10cm,compat=1.9}

% disponi alberi
\usepackage{forest}

\forestset{
	rectstyle/.style={
		for tree={rectangle,draw,font=\large\sffamily}
	},
	roundstyle/.style={
		for tree={circle,draw,font=\large}
	}
}

% disponi algoritmi
\usepackage{algorithm}
\usepackage{algorithmic}
\makeatletter
\renewcommand{\ALG@name}{Algoritmo}
\makeatother

% disponi numeri di pagina
\usepackage{fancyhdr}
\fancyhf{} 
\fancyfoot[L]{\sffamily{\thepage}}

\makeatletter
\fancyhead[L]{\raisebox{1ex}[0pt][0pt]{\sffamily{\@title \ \@date}}} 
\fancyhead[R]{\raisebox{1ex}[0pt][0pt]{\sffamily{\@author}}}
\makeatother

\begin{document}
% sezione (data)
\section{Lezione del 01-10-24}

% stili pagina
\thispagestyle{empty}
\pagestyle{fancy}

% testo
\subsection{Istruzioni per sottoprogrammi}
Nei sottoprogrammi vengono coninvolte due istruzioni \lstinline|CALL|, e \lstinline|RET|.
Entrambe si riferiscono alla pila.


\subsubsection{CALL}
\begin{itemize}
	\item \textbf{Formato:} \lstinline|CALL %EIP +/- $displacement|, \lstinline|CALL *extended_register|, \lstinline|CALL *memory| 
	\item \textbf{Azione:} effettua la chiamata di un sottoprogramma, ovvero:
		\begin{itemize}
			\item Salva il valore corrente di EIP nella pila;
			\item Modifica EIP come farebbe JMP.
		\end{itemize}
	\item \textbf{Flag:} nessuno.
\end{itemize}

		\begin{table}[H]
		\center \rowcolors{2}{white}{black!10}
			\begin{tabular} { c | p{5cm} }
				\bfseries Operandi & \bfseries Esempi \\
				\hline
				Displacement & \lstinline|CALL 0x00400010| \\ 
				Registro & \lstinline|CALL *%EAX| \\ 
				Memoria & \lstinline|CALL *0x00400010|
			\end{tabular}
		\end{table}

\subsubsection{RET}
\begin{itemize}
	\item \textbf{Formato:} \lstinline|RET|
	\item \textbf{Azione:} ritorna da un sottoprogramma, ovvero:
		\begin{itemize}
			\item Rimuove un long dalla pila;
			\item Lo inserisce in EIP.
		\end{itemize}
	\item \textbf{Flag:} nessuno.
\end{itemize}

\par\medskip 
Esistono poi altre istruzioni di controllo, ovvero:

\subsubsection{NOP}
\begin{itemize}
	\item \textbf{Formato:} \lstinline|NOP|
	\item \textbf{Azione:} è l'istruzione nulla. 
	\item \textbf{Flag:} nessuno.
\end{itemize}

\subsubsection{HLT}
\begin{itemize}
	\item \textbf{Formato:} \lstinline|HLT|
	\item \textbf{Azione:} arresta l'esecuzione fino al prossimo interrupt. 
	\item \textbf{Flag:} nessuno.
\end{itemize}

\subsubsection{HCF}
\begin{itemize}
	\item \textbf{Formato:} \lstinline|HCF|
	\item \textbf{Azione:} arresta l'esecuzione e causa l'autocombustione spontanea del processore. 
	\item \textbf{Flag:} nessuno.
\end{itemize}

\subsection{Istruzioni privilegiate}
Il codice in assembler può girare secondo due modalità sul sistema:
\begin{itemize}
	\item \textbf{Sistema:} con accesso totale a tutte le istruzioni;
	\item \textbf{Utente:} senza l'accesso ad alcune istruzioni dette privilegiate.
\end{itemize}

Tra le istruzioni privilegiate ci sono \lstinline|HLT|, \lstinline|IN| e \lstinline|OUT|.
La \lstinline|HLT| non è un grande problema, ma lo sono \lstinline|IN| e \lstinline|OUT|.
Per ottenere input e output dal sistema, adoperiamo quindi determinati sottoprogrammi di servizio atti a fornire esattamente queste informazioni.

L'uso di sottoprogrammi di servizio per l'input/output è dovuto al fatto che le interfacce sono sistemi complessi, facili da portare in stato inconsistente, mentre i sottoprogrammi si assicurano di farne un corretto uso.

\subsection{Struttura di un programma assembler}
Vediamo adesso come strutturare un programma assembler scritto nell'ambiente GAS (Gnu Assembler).
Un programma assembler è diviso in due sezioni
\begin{itemize}
	\item \textbf{Sezione dati:} qui si dichiarano le variabili, ergo nomi simbolici per indirizzi di memoria che contengono i dati del programma;
	\item \textbf{Sezione codice:} istruzioni.


In un programma abbiamo bisogno di:
\end{itemize}


\begin{itemize}
	\item \textbf{Istruzioni}, viste finora;
	\item \textbf{Direttive}, necessarie all'assemblaggio e alla dichiarazione di variabili.
\end{itemize}

Ad esempio, potremo avere:
\begin{lstlisting}	
.GLOBAL _main

.DATA
...

.TEXT
_main:	NOP
...
				RET
\end{lstlisting}

Le linee che iniziano col punto sono direttive, le altre istruzioni.
Una riga qualsiasi del codice è fatta come:
\begin{lstlisting}	
nome:	OPCODE operandi # commento [\CR]
\end{lstlisting}
dove abbiamo una label, l'istruzione e un commento.

Tutto qui può mancare, tranne il ritorno carrello.
Tutte le righe, inclusa l'ultima, vanno terminate.
Inoltre, l'ultima riga dovrebbe essere una RET, che restituisce l'esecuzione al chiamante (qui l'ambiente).

Conviene iniziare il programma con una NOP, per assicurarsi che in fase di inizializzazione esso non faccia effettivamente nulla.

Vediamo ad esempio il programma visto prima per il conteggio degli uni, reso in questa struttura:
\begin{lstlisting} 
.GLOBAL _main
.DATA
dato:				.LONG 0x0F0F0101
conteggio:	.BYTE 0x00

.TEXT
_main:			NOP
						MOVB $0x00, %CL
						MOVL dato, %EAX
comp:				CMPL $0x00, %EAX
						JE fine
						SHRL %EAX
						ADCB $0x00, %CL
						JMP comp
fine:				MOVB %CL, conteggio
						RET
\end{lstlisting}

\subsubsection{Direttive}
Tutte le direttive iniziano con il carattere punto.
Esse sono:
\begin{itemize}
	\item \textbf{Dichiarazione di variabili:}
		Variabili dichiarate di seguito sono sempre consecutive in memoria. Si ha, di base:
		\begin{itemize}
			\item \lstinline|.BYTE|: riserva 1 byte;
			\item \lstinline|.WORD|: riserva 2 byte;
			\item \lstinline|.LONG|: riserva 4 byte.
		\end{itemize}
	\textsf{\textbf{Esempi}}
\begin{lstlisting}	
var0:	.WORD									# scalare, 2 byte, valore 0x0000 
														#	(considerato brutto, non inizializzare
														#	 si fa con .FILL)
var1: .BYTE 0x30						# scalare, 1 byte, valore 0x30
var2: .BYTE 0x30,0x31				# vettore, 2 componenti da 1 byte, 
														#	valore 0x30 e 0x31
var3:	.WORD 0x1020, 0x32AB	# vettore, 2 componenti da 2 byte, 
														#	valore 0x1020e 0x32AB
var4: .LONG var3+2					# scalare, 4 byte, valore 0xAB
\end{lstlisting}
\par\smallskip 
Esistono altri modi di inizializzare variabili particolari:
\begin{itemize}
	\item \lstinline|.FILL numero, dim, espressione|: dichiara \lstinline|numero| variabili di lunghezza \lstinline|dim| e le inizializza ad espression (0 di default).
		Dim può essere 1, 2 o 4.
	\item \textbf{ASCII}: si può usare la codifica ASCII fra single tick ', coi caratteri speciali dopo sequenze di escape, per indicare singoli byte. Ad esempio:
\begin{lstlisting}	
var5: .BYTE 'S', 'o', 'n', 'n', 'o'				# vettore, 4 componenti 
																					# da 1 byte
var6: .BYTE 0x53, 0x6F, 0x6E, 0x6E, 0x6F	# vettore, 4 componenti 
																					# da 1 byte
var7: .ASCII "Stea"												# vettore, 4 componenti
																					# da 1 byte
var8: .ASCIZ "Stea"												# vettore, 5 componenti 
																					# da 1 byte (include il 
																					# terminatore)
	\end{lstlisting} 
\end{itemize}
\item \textbf{Altre direttive:}
	\begin{itemize}
		\item \lstinline|.INCLUDE "path"|: include un sorgente nel presente file, prima dell'assemblamento;
		\item \lstinline|.SET nome, espressione|: serve a creare \textbf{costanti simboliche}. 
			Tali costanti hanno nome \lstinline|nome| e valore \lstinline|espressione|. Ad esempio:
\begin{lstlisting}	
.SET dimensione, 4
.SET n_iter, (100 * dimensione)
...
MOV $n_iter, %CX	# e' accesso immediato
\end{lstlisting}
	\end{itemize}
\end{itemize}

\subsection{Costanti numeriche}
Possiamo indicare costanti numeriche attraverso le seguenti convenzioni:
\begin{itemize}
	\item \textbf{Naturali:} non hanno segno, e vengono convertite nella loro rappresentazione in base 2;
	\item \textbf{Intere:} hanno un segno + o - davanti, e vengono convertite nella loro rappresentazione in complemento a 2.
\end{itemize}

Inoltre possiamo scrivere costanti in base 2, 8, 10 e 16 attraverso i prefissi \lstinline|0b|, \lstinline|0|, nessun prefisso e \lstinline|0x|.

Le variabili, quando non sono della dimensione giusta, vengono solitamente troncate (con avviso dall'assemblatore) o estese (senza avvisi dall'assemblatore).

\subsection{Controllo di flusso}
I costrutti di flusso a cui siamo abituati vengono implementati attraverso istruzioni di salto.
Conviene comunque ragionare in costrutti ad alto livello, e limitarsi a tradurli in assembler.
Da qui in puoi useremo una sintassi pseudo-C per indicare questi costrutti ad alto livello.

\subsubsection{If-then-else}
Prendiamo la sintassi:
\begin{lstlisting}[language=C++, style=codestyle]	
if(%AX < variabile) {
	//ramo if
	...
} else {
	//ramo else
	...
}
//prosegui
...
\end{lstlisting}
potremo tradurla in due modi:
\begin{itemize}
	\item Invertendo i rami then e else:
\begin{lstlisting}	
					CMP variabile, %AX
					JB ramothen
ramoelse:	... # ramo else
					JMP segue
ramothen: ... # ramo then
segue:		# prosegui
					...
\end{lstlisting}
	\item Invertendo la condizione:
\begin{lstlisting}[language=C++, style=codestyle]	
					CMP variabile, %AX
					JAE ramoelse
ramothen:	... # ramo then
					JMP segue
ramoelse: ... # ramo else
segue:		... # prosegui
\end{lstlisting}
\end{itemize}

\subsubsection{Ciclo for}
Prendiamo:
\begin{lstlisting}[language=C++, style=codestyle]	
for(int i = 0; i < variabile; i++) {
	//iter
	...
}
//prosegui
...
\end{lstlisting}
si rende attraverso il registro CX, come:
\begin{lstlisting}	
				MOV $0, %CX
ciclo:	CMP var, %CX
				JE segue
				...	# iter
				INC %CX
				JMP ciclo
segue:	... # prosegui
\end{lstlisting}

\subsubsection{Ciclo do-while}
Prendiamo infine:
\begin{lstlisting}[language=C++, style=codestyle]	
do {
	//iter
	...
} while(AX < var)
//prosegui
...
\end{lstlisting}
si rende come:
\begin{lstlisting}	
ciclo:	... # iter
				CMP var, %AX
				JB ciclo
				... # prosegui
\end{lstlisting}

\subsubsection{Un piatto di spaghetti}
In assembler ci è concesso fare ciò che non è permesso da linguaggi strutturati come il C o il Pascal.
In questi linguaggi, un costrutto ha un solo punto di ingresso e un solo punto di uscita.

In assembler, invece, possiamo saltare fuori e dentro cicli e costrutti quando e dove vogliamo, ed è il programmatore che deve pensare a cosa il programma sta effettivamente facendo. Ad esempio, nessuno ci vieta di dire:
\begin{lstlisting}	
ciclo: 	... # inizio ciclo
			 	...
label1:	... # meta' ciclo
				CMP var, %AX
				JB ciclo
				...
				JMP label1 # salto dentro un ciclo a meta' esecuzione?
\end{lstlisting}
\par\medskip
In assembler abbiamo a disposizione un'istruzione dedicata per i loop, che è:

\subsubsection{LOOP}
\begin{itemize}
	\item \textbf{Formato:} \lstinline|LOOP destination|
	\item \textbf{Azione:} decrementa ECX e salta alla destinazione se ECX $\neq0$. ECX va inizializzato al numero di iterazioni desiderate, e non va toccato durante il ciclo. 
	\item \textbf{Flag:} nessuno.
\end{itemize}

Si nota che la LOOP decrementa sempre ECX, quindi si applica difficilmente a cicli FOR dove vogliamo che la variabile di controllo incrementi, e ci serve che il suo valore nel corpo del ciclo. Si noti la differenza nei due esempi:

\begin{minipage}{0.45\textwidth}
\begin{lstlisting}[language=C++, style=codestyle]	
for(int i = var; i > 0; i--) {
	//iter (usa i)
}
\end{lstlisting}
diventa:
\begin{lstlisting}
				MOV var, %ECX
ciclo:	... # iter
				LOOP ciclo
\end{lstlisting}
\end{minipage}%
\hfill % This adds horizontal space between the two minipages
\begin{minipage}{0.45\textwidth}
\begin{lstlisting}[language=C++, style=codestyle]	
for(int i = 0; i < var; i++) {
	//iter (usa i)
}
\end{lstlisting}
diventa:
\begin{lstlisting}
				MOV $0, %EBX # usa EBX
ciclo:	... # iter
				INC EBX
				CMP var, %EBX
				JE ciclo
\end{lstlisting}
\end{minipage}

\subsubsection{LOOP condizionali}
Esistono versioni condizionali della LOOP, che sono \lstinline|LOOPE| e \lstinline|LOOPNE|, simili alle Jump condizionali. In questo caso, oltre al registro ECX, si verifica la condizione e nel caso si salta. Ad esempio:

\begin{lstlisting}	
				MOV $10, %ECX
ciclo: 	CMP src, dest
				LOOPcond ciclo
\end{lstlisting}

\par\smallskip
Queste istruzioni non sono indispensabili, in quanto possono essere rimpiazzate facilmente dalla \lstinline|CMP| unita ad un Jump condizionale.

\subsection{Passaggio di argomenti a sottoprogrammi}
Le \lstinline|CALL| e \lstinline|RET| prima definite non fornisicono modi per passare parametri ai sottoprogrammi, o restituire valori ai chiamanti.

Dobbiamo quindi stabilire delle convenzioni, scegliendo se:
\begin{itemize}
	\item Usare locazioni di memoria condivise;
	\item Usare registri;
	\item Usare la pila (che non verrà visto nel corso).
\end{itemize}

In assembler non esiste il concetto di visibilità o variabili locali, tutta la memoria è indirizzabile a qualsiasi livello.
Comunque, quando si scrive un sottoprogramma, bisogna specificare i parametri di ingresso e di uscita con un'opportuno commento, come:
\begin{lstlisting}	
# sottoprogramma "sottoprog", [descrizione]
# ingresso: %AX, [descrizione]
#					  %EBX, [descrizione]
# uscita:	  CAX, [descrizione]

sottoprog: 	...
						MOV ..., %CX # preparo il ritorno
						RET
\end{lstlisting} 

adesso potremo usare il sottoprogramma come:
\begin{lstlisting}	
MOV ..., %AX # preparo i parametri
MOV ..., %EBX
CALL sottoprog # chiamo
MOV %CX, var # var contiene il ritorno
\end{lstlisting}

\end{document}


\documentclass[a4paper,11pt]{article}
\usepackage[a4paper, margin=8em]{geometry}

% usa i pacchetti per la scrittura in italiano
\usepackage[french,italian]{babel}
\usepackage[T1]{fontenc}
\usepackage[utf8]{inputenc}
\frenchspacing 

% usa i pacchetti per la formattazione matematica
\usepackage{amsmath, amssymb, amsthm, amsfonts}

% usa altri pacchetti
\usepackage{gensymb}
\usepackage{hyperref}
\usepackage{standalone}

% imposta il titolo
\title{Appunti Reti Logiche}
\author{Luca Seggiani}
\date{2024}

% imposta lo stile
% usa helvetica
\usepackage[scaled]{helvet}
% usa palatino
\usepackage{palatino}
% usa un font monospazio guardabile
\usepackage{lmodern}

\renewcommand{\rmdefault}{ppl}
\renewcommand{\sfdefault}{phv}
\renewcommand{\ttdefault}{lmtt}

% disponi il titolo
\makeatletter
\renewcommand{\maketitle} {
	\begin{center} 
		\begin{minipage}[t]{.8\textwidth}
			\textsf{\huge\bfseries \@title} 
		\end{minipage}%
		\begin{minipage}[t]{.2\textwidth}
			\raggedleft \vspace{-1.65em}
			\textsf{\small \@author} \vfill
			\textsf{\small \@date}
		\end{minipage}
		\par
	\end{center}

	\thispagestyle{empty}
	\pagestyle{fancy}
}
\makeatother

% disponi teoremi
\usepackage{tcolorbox}
\newtcolorbox[auto counter, number within=section]{theorem}[2][]{%
	colback=blue!10, 
	colframe=blue!40!black, 
	sharp corners=northwest,
	fonttitle=\sffamily\bfseries, 
	title=Teorema~\thetcbcounter: #2, 
	#1
}

% disponi definizioni
\newtcolorbox[auto counter, number within=section]{definition}[2][]{%
	colback=red!10,
	colframe=red!40!black,
	sharp corners=northwest,
	fonttitle=\sffamily\bfseries,
	title=Definizione~\thetcbcounter: #2,
	#1
}

% disponi codice
\usepackage{listings}
\usepackage[table]{xcolor}

\definecolor{codegreen}{rgb}{0,0.6,0}
\definecolor{codegray}{rgb}{0.5,0.5,0.5}
\definecolor{codepurple}{rgb}{0.58,0,0.82}
\definecolor{backcolour}{rgb}{0.95,0.95,0.92}

\lstdefinestyle{codestyle}{
		backgroundcolor=\color{black!5}, 
		commentstyle=\color{codegreen},
		keywordstyle=\bfseries\color{magenta},
		numberstyle=\sffamily\tiny\color{black!60},
		stringstyle=\color{green!50!black},
		basicstyle=\ttfamily\footnotesize,
		breakatwhitespace=false,         
		breaklines=true,                 
		captionpos=b,                    
		keepspaces=true,                 
		numbers=left,                    
		numbersep=5pt,                  
		showspaces=false,                
		showstringspaces=false,
		showtabs=false,                  
		tabsize=2
}

\lstdefinestyle{shellstyle}{
		backgroundcolor=\color{black!5}, 
		basicstyle=\ttfamily\footnotesize\color{black}, 
		commentstyle=\color{black}, 
		keywordstyle=\color{black},
		numberstyle=\color{black!5},
		stringstyle=\color{black}, 
		showspaces=false,
		showstringspaces=false, 
		showtabs=false, 
		tabsize=2, 
		numbers=none, 
		breaklines=true
}


\lstdefinelanguage{assembler}{ 
  keywords={AAA, AAD, AAM, AAS, ADC, ADCB, ADCW, ADCL, ADD, ADDB, ADDW, ADDL, AND, ANDB, ANDW, ANDL,
        ARPL, BOUND, BSF, BSFL, BSFW, BSR, BSRL, BSRW, BSWAP, BT, BTC, BTCB, BTCW, BTCL, BTR, 
        BTRB, BTRW, BTRL, BTS, BTSB, BTSW, BTSL, CALL, CBW, CDQ, CLC, CLD, CLI, CLTS, CMC, CMP,
        CMPB, CMPW, CMPL, CMPS, CMPSB, CMPSD, CMPSW, CMPXCHG, CMPXCHGB, CMPXCHGW, CMPXCHGL,
        CMPXCHG8B, CPUID, CWDE, DAA, DAS, DEC, DECB, DECW, DECL, DIV, DIVB, DIVW, DIVL, ENTER,
        HLT, IDIV, IDIVB, IDIVW, IDIVL, IMUL, IMULB, IMULW, IMULL, IN, INB, INW, INL, INC, INCB,
        INCW, INCL, INS, INSB, INSD, INSW, INT, INT3, INTO, INVD, INVLPG, IRET, IRETD, JA, JAE,
        JB, JBE, JC, JCXZ, JE, JECXZ, JG, JGE, JL, JLE, JMP, JNA, JNAE, JNB, JNBE, JNC, JNE, JNG,
        JNGE, JNL, JNLE, JNO, JNP, JNS, JNZ, JO, JP, JPE, JPO, JS, JZ, LAHF, LAR, LCALL, LDS,
        LEA, LEAVE, LES, LFS, LGDT, LGS, LIDT, LMSW, LOCK, LODSB, LODSD, LODSW, LOOP, LOOPE,
        LOOPNE, LSL, LSS, LTR, MOV, MOVB, MOVW, MOVL, MOVSB, MOVSD, MOVSW, MOVSX, MOVSXB,
        MOVSXW, MOVSXL, MOVZX, MOVZXB, MOVZXW, MOVZXL, MUL, MULB, MULW, MULL, NEG, NEGB, NEGW,
        NEGL, NOP, NOT, NOTB, NOTW, NOTL, OR, ORB, ORW, ORL, OUT, OUTB, OUTW, OUTL, OUTSB, OUTSD,
        OUTSW, POP, POPL, POPW, POPB, POPA, POPAD, POPF, POPFD, PUSH, PUSHL, PUSHW, PUSHB, PUSHA, 
				PUSHAD, PUSHF, PUSHFD, RCL, RCLB, RCLW, MOVSL, MOVSB, MOVSW, STOSL, STOSB, STOSW, LODSB, LODSW,
				LODSL, INSB, INSW, INSL, OUTSB, OUTSL, OUTSW
        RCLL, RCR, RCRB, RCRW, RCRL, RDMSR, RDPMC, RDTSC, REP, REPE, REPNE, RET, ROL, ROLB, ROLW,
        ROLL, ROR, RORB, RORW, RORL, SAHF, SAL, SALB, SALW, SALL, SAR, SARB, SARW, SARL, SBB,
        SBBB, SBBW, SBBL, SCASB, SCASD, SCASW, SETA, SETAE, SETB, SETBE, SETC, SETE, SETG, SETGE,
        SETL, SETLE, SETNA, SETNAE, SETNB, SETNBE, SETNC, SETNE, SETNG, SETNGE, SETNL, SETNLE,
        SETNO, SETNP, SETNS, SETNZ, SETO, SETP, SETPE, SETPO, SETS, SETZ, SGDT, SHL, SHLB, SHLW,
        SHLL, SHLD, SHR, SHRB, SHRW, SHRL, SHRD, SIDT, SLDT, SMSW, STC, STD, STI, STOSB, STOSD,
        STOSW, STR, SUB, SUBB, SUBW, SUBL, TEST, TESTB, TESTW, TESTL, VERR, VERW, WAIT, WBINVD,
        XADD, XADDB, XADDW, XADDL, XCHG, XCHGB, XCHGW, XCHGL, XLAT, XLATB, XOR, XORB, XORW, XORL},
  keywordstyle=\color{blue}\bfseries,
  ndkeywordstyle=\color{darkgray}\bfseries,
  identifierstyle=\color{black},
  sensitive=false,
  comment=[l]{\#},
  morecomment=[s]{/*}{*/},
  commentstyle=\color{purple}\ttfamily,
  stringstyle=\color{red}\ttfamily,
  morestring=[b]',
  morestring=[b]"
}

\lstset{language=assembler, style=codestyle}

% disponi sezioni
\usepackage{titlesec}

\titleformat{\section}
	{\sffamily\Large\bfseries} 
	{\thesection}{1em}{} 
\titleformat{\subsection}
	{\sffamily\large\bfseries}   
	{\thesubsection}{1em}{} 
\titleformat{\subsubsection}
	{\sffamily\normalsize\bfseries} 
	{\thesubsubsection}{1em}{}

% tikz
\usepackage{tikz}

% float
\usepackage{float}

% grafici
\usepackage{pgfplots}
\pgfplotsset{width=10cm,compat=1.9}

% disponi alberi
\usepackage{forest}

\forestset{
	rectstyle/.style={
		for tree={rectangle,draw,font=\large\sffamily}
	},
	roundstyle/.style={
		for tree={circle,draw,font=\large}
	}
}

% disponi algoritmi
\usepackage{algorithm}
\usepackage{algorithmic}
\makeatletter
\renewcommand{\ALG@name}{Algoritmo}
\makeatother

% disponi numeri di pagina
\usepackage{fancyhdr}
\fancyhf{} 
\fancyfoot[L]{\sffamily{\thepage}}

\makeatletter
\fancyhead[L]{\raisebox{1ex}[0pt][0pt]{\sffamily{\@title \ \@date}}} 
\fancyhead[R]{\raisebox{1ex}[0pt][0pt]{\sffamily{\@author}}}
\makeatother

\begin{document}
% sezione (data)
\section{Lezione del 02-10-24}

% stili pagina
\thispagestyle{empty}
\pagestyle{fancy}

% testo
\subsection{Effetti collaterali}
I sottoprogrammi non dovrebbero avere effetti collaterali, ergo dovrebbero lasciare i registri come li trovano.
Per fare ciò, si sfrutta la pila per immagazzinare i loro valori precedenti:
\begin{lstlisting}	
sottoprog:	PUSH ... # fai push dei registri
						PUSH ...
						...	# esegui il sottoprogramma
						MOV ..., %CX

						POP ... # riprendi i resisti
						POP ...
						RET
\end{lstlisting}

Sono fondamentali due linee guida:
\begin{itemize}
	\item Bisogna stare attenti ad operazioni come \lstinline|IDIV| e \lstinline|IMUL|, che sporcano registri come EDX implictamente;
	\item Bisogna far corrispondere una \lstinline|POP| ad ogni \lstinline|PUSH|, altrimenti si lascia la pila in uno stato inconsistente per il prossimo \lstinline|RET|.
\end{itemize}

\subsection{Sottoprogramma principale}
Il \lstinline|_main| va in esecuzione come un sottoprogramma, ergo deve terminare con una \lstinline|RET| e lasciare in EAX un valore di ritorno (0 significa tutto ok, $\neq0$ significa codice di errore).
Per quanto ci riguarda, basterà scrivere \lstinline|XOR %EAX, %EAX|.

\subsection{Dichiarazione dello stack}
Lo stack esiste se viene:
\begin{enumerate}
	\item Dichiarato con una direttiva;
	\item Inizializzato con il registro ESP.
\end{enumerate}

Dichiarare significa allocare abbastanza memoria, e inizializzare significa impostare ESP alla cella successiva al fondo dello stack (si ricorda che lo stack si evolve verso sinistra). Ad esempio, potremo avere:

\begin{lstlisting}	
.DATA
...
mystack:	.FILL 1024, 4 #dichiarazione stack
.SET			initial_esp, (mystack + 1024*4)

.TEXT
_main:		NOP
					MOV $initial_esp, %ESP	# inizializzazione stack
\end{lstlisting}

Lo stack può essere grande a piacere del programmatore.
Nel nostro ambiente (ma non in generale) possiamo omettere la dichiarazione.

La pila può essere anche usata per il passaggio dei documanti (è il metodo che usano i compilatori). 
Questo risulta difficile da fare a mano, e quindi è sconsigliato per programmi più semplici.

\subsection{Sottoprogrammi di Input/Output}
In assembler non esistono istruzioni di ingresso e uscita (tranne le \lstinline|IN| e \lstinline|OUT|, che però sappiamo essere privilegiate).
Si usano quindi i servizi del sistema (DOS), ovvero sottoprogrammi scritti da altri che girano in modalità sistema.
Questi servizi sono molto primitivi: permettono l'uscita di singoli caratteri.
Esistono quindi sottoprogrammi (leggermente) più sofisticati per l'output di numeri, ecc...

\subsubsection{I/O tastiera e video}
Le informazioni che entrano ed escono da interfacce sono solo codifice ASCII di singoli caratteri.
Infatti in assembler non esiste il concetto di I/O tipato di variabili.

Ricevere il numero 32 significa ottenere i caratteri '3' e '2', mentre stamparlo significa inviare i caratteri '3' e '2'.
Questo chiaramente sui decimale si traduce in moltiplicazioni per 10 (in entrata) e divisioni per 10 con resto (in uscita) atte ad ottenere queste cifre.

\subsubsection{I/O di caratteri e stringhe}
Nel corso si userà il file di utilità \lstinline|.INCLUDE "./files/utility.s"|.
Questo file mette a disposizione alcuni sottoprogrammi fra cui:
\begin{itemize}
	\item \textbf{inchar:} mette in AL la codifica ASCII del tasto premuto;
	\item \textbf{outchar:} mette sul video la codifica ascii contenuta in AL;
	\item \textbf{newline:} stampa \lstinline|0x0D| (Carriage Return) e \lstinline|0x0A| (Line Feed), ergo va a capo;
	\item \textbf{pauseN:} mette in pausa il programma e stampa a video:
\begin{lstlisting}	
Checkpoint number N. Press any key to continue
\end{lstlisting}
dove N deve essere una cifra decimale.
\end{itemize}

Sopra questi sottoprogrammi sono state scritte routine più complesse:
\begin{itemize}
	
	\item \textbf{inline:} 
	\begin{itemize}
		\item \textbf{Descrizione:} porta una stringa di massimo 80 caratteri in un buffer di memoria, digitando con eco su video.
		\item \textbf{Parametri di ingresso:} 
			\begin{itemize}
				\item EBX: indirizzo di memoria del buffer;
				\item CX: numero di caratteri da leggere (massimo 80, una linea).
			\end{itemize}
	\end{itemize}
		Questo programma legge effettivamente 78 caratteri utili, in quanto gli ultimi 2 sono obbligatoriamente il nuova linea.
		Il programma inoltre gestisce la pressione dei tasti invio (finisci di ottenere caratteri) e backspace (cancella caratteri).
	
	\item \textbf{outline, outmess:}
		\begin{itemize}
		\item \textbf{Descrizione:} stampa a video massimo 80 caratteri da un buffer di memoria. Si ferma prima se trova un carattere di ritorno carrello, andando anche a capo. 
		\item \textbf{Parametri di ingresso:} 
			\begin{itemize}
				\item EBX: indirizzo di memoria del buffer;
			\end{itemize}
		\end{itemize}
		
	\item \textbf{inbyte, inword, inlong:}
		\begin{itemize}
		\item \textbf{Descrizione:} prelevano da tastiera (con eco sul video) 2, 4 o 8 caratteri.
			Interpretano tale sequenza di caratteri come un numero esadecimale a 2, 4 o 8 cifre.
			Ignorano tutti gli altri caratteri.
		\item \textbf{Parametri di ingresso:} 
			\begin{itemize}
				\item AL, AX, o EAX: il numero esadecimale digitato.
			\end{itemize}
		\end{itemize}

	\item \textbf{outbyte, outword, outlong:}
		\begin{itemize}
		\item \textbf{Descrizione:} stampano a video 2, 4 o 8 caratteri, corrispondenti a cifre esadecimali.
		\item \textbf{Parametri di ingresso:} 
			\begin{itemize}
				\item AL, AX, o EAX: il numero esadecimale da stampare.
			\end{itemize}
		\end{itemize}


	\item \textbf{indecimal\_byte, indecimal\_word, indecimal\_long:}
		\begin{itemize}
		\item \textbf{Descrizione:} prelevano da tastiera (con eco sul video) fino a 3, 5 o 10 cifre decimali.
			Interpretano tale sequenza di caratteri come un numero decimale.
		\item \textbf{Parametri di ingresso:} 
			\begin{itemize}
				\item AL, AX, o EAX: il numero decimale digitato.
			\end{itemize}
		\end{itemize}
	Se il numero decimale è troppo grande viene troncato.
	Inoltre si può usare invio per dare ingresso a meno cifre.

	
	\item \textbf{outdecimal\_byte, outdecimal\_word, outdecimal\_long:}
		\begin{itemize}
		\item \textbf{Descrizione:} stampano a video caratteri corrispondenti a cifre decimali.
		\item \textbf{Parametri di ingresso:} 
			\begin{itemize}
				\item AL, AX, o EAX: il numero decimale da stampare.
			\end{itemize}
		\end{itemize}

\end{itemize}

\subsection{Manipolazione di stringhe e vettori}
In assembler non esistono tipi di dati né strutture dati.
Si supporta però il concetto di vettore: si dichiarano vettori di variabili di una certa dimensione, e si indirizzano i loro elementi attraverso l'indirizzamento complesso ($\text{displacement} + \text{base} + \text{indice} \times \text{scala}$).

In verità esistono istruzioni stringa, che servono a copiare interi buffer di memoria, che sfruttano i registri ESI e EDI.
Ad esempio, copiare un vettore a mano significherebbe:
\begin{lstlisting}	
vett_sorg:	.FILL 1000,4
vett_dest:	.FILL 1000,4

						MOV $1000, %ECX
						LEA vett_sorg, %ESI
						LEA vett_dest, %EDI
ciclo:			MOV (%ESI), %EAX
						MOV %EAX , (%EDI)
						ADD $4, %ESI
						ADD $4, %EDI
						LOOP ciclo
\end{lstlisting}
ma abbiamo la possibilità di scrivere la stessa cosa come:
\begin{lstlisting}	
vett_sorg:	.FILL 1000,4
vett_dest:	.FILL 1000,4

						MOV $1000, %ECX
						LEA vett_sorg, %ESI
						LEA vett_dest, %EDI
						REP MOVSL
\end{lstlisting}
dove l'istruzione \lstinline|REP MOVSL| indica ripetizione (prefisso \lstinline|REP|), di movimento da stringa a stringa su long (\lstinline|MOVSL|) finché ECX $\neq 0$.

\subsubsection{Direction Flag}
Esiste un'altro bit utile nel registro dei flag: il Direction Flag, o DF.
Si imposta con le istruzioni:
\begin{itemize}
	\item \textbf{\textsf{STD}}: \textsf{SET DIRECTION FLAG}, la imposta ad 1;
	\item \textbf{\textsf{CLD}}: \textsf{CLEAR DIRECTION FLAG}, la imposta a  0;
\end{itemize}

Si usa questo flag per dare indicazioni alla prossima istruzione:

\subsubsection{MOVE DATA FROM STRING TO STRING (with REPEAT)}
\begin{itemize}
	\item \textbf{Formato:} \lstinline|MOVSsuf|, \lstinline|REP MOVSsuf| 
	\item \textbf{Azione:} copia il numero di byte indicato dal suffisso \textit{suf} dall'indirizzo di memoria puntato da ESI all'indirizzo di memoria puntato da EDI.
		Successivamente, SE DF è 1, sottrae da ESI e EDI il numero di byte indicati da \textit{suf}, altrimenti li somma.

		Se si include il prefisso, le operazioni vengono ripetute decrementando ECX (come per \lstinline|LOOP|).
	\item \textbf{Flag:} nessuno.
\end{itemize}
\par\medskip
Esistono poi altre istruzioni di stringa, fra cui:

\subsubsection{LOAD DATA FROM STRING}
\begin{itemize}
	\item \textbf{Formato:} \lstinline|LODSsuf| 
	\item \textbf{Azione:} copia in AL, AX, oppure EAX, il contenuto della memoria all'indirizzo puntato da ESI. Successivamente incrementa o decrementa ESI di 1, 2 o 4 a seconda di DF.
	\item \textbf{Flag:} nessuno.
\end{itemize}

\subsubsection{STORE DATA TO STRING}
\begin{itemize}
	\item \textbf{Formato:} \lstinline|LODSsuf| 
	\item \textbf{Azione:} copia il registro AL, AX, oppure EAX, in memoria all'indirizzo puntato da EDI. Successivamente incrementa o decrementa EDI di 1, 2 o 4 a seconda di DF.
	\item \textbf{Flag:} nessuno.
\end{itemize}

Si dovrebbe essere notato che ESI sta per sorgente, ed EDI per destinatario.
Vediamo quindi degli esempi:
\par\smallskip
\begin{minipage}[t]{0.45\textwidth}
Copia un vettore da una parte all'altra, eseguendo un'operazione su tutti i suoi elementi:
\begin{lstlisting}	
				MOV $1000, %CX
				LEA buffer_src, %ESI
				LEA buffer_dst, %EDI
				CLD
ciclo:	LODSL
				...	#modifica %EAX
				STOSL
				LOOP ciclo
\end{lstlisting}

\end{minipage}%
\hfill % This adds horizontal space between the two minipages
\begin{minipage}[t]{0.45\textwidth}
Riempi un buffer in memoria di zeri:
\begin{lstlisting}	
MOV $1000, %ECX
LEA buffer, %EDI
XOR %EAX, %EAX
CLD
REP STOSL
\end{lstlisting}
\end{minipage}

\subsubsection{Istruzioni stringa per l'I/O}
Esistono delle istruzioni stringa di ingresso e uscita: 


\subsubsection{INSERT STRING}
\begin{itemize}
	\item \textbf{Formato:} \lstinline|INSsuf| 
	\item \textbf{Azione:} fa ingresso di 1, 2 o 4 byte dalla porta di I/O il cui offset è contenuto in DX. L'operando viene inserito in memoria a partire dall'indirizzo contenuto in EDI.
		Successivamente incrementa o decrementa EDI di 1, 2, o 4 a seconda di DF.
	\item \textbf{Flag:} nessuno.
\end{itemize}

\subsubsection{OUTPUT STRING}
\begin{itemize}
	\item \textbf{Formato:} \lstinline|INSsuf| 
	\item \textbf{Azione:} fa uscita di 1, 2 o 4 byte dall'indirizzo di memoria contenuto in EDI. L'operando viene inserito nella porta di I/O il cui offset è contenuto in DX.
		Successivamente incrementa o decrementa ESI di 1, 2, o 4 a seconda di DF. 

	\item \textbf{Flag:} nessuno.
\end{itemize}

\subsubsection{Istruzioni di confronto su stringhe}
Vediamo infine alcune istruzioni per effettuare confronti su e fra stringhe:

\subsubsection{COMPARE STRINGS}
\begin{itemize}
	\item \textbf{Formato:} \lstinline|CMPSsuf| 
	\item \textbf{Azione:} confronta il valore delle locazioni (singole, doppie o quadruple) indicate da ESI (sorgente) ed EDI (destinatario). 
		Successivamente incrementa o decrementa ESI di 1, 2, o 4 a seconda di DF. 

	\item \textbf{Flag:} nessuno.
\end{itemize}

\subsubsection{SCAN STRING}
\begin{itemize}
	\item \textbf{Formato:} \lstinline|SCASsuf| 
	\item \textbf{Azione:} confronta il contenuto del registro AL, AX o EAX con la locazione (singola, doppia o quadrupla) di memoria indirizzata da EDI. L'algoritmo di confronto è lo stesso di CMP. 
		Successivamente incrementa o decrementa ESI di 1, 2, o 4 a seconda di DF. 

	\item \textbf{Flag:} nessuno.
\end{itemize}

Quest'espressione si usa per trovare valori noti dentro un vettore con, DF $=0$ che cerca la prima occorrenza, e DF $=1$ che cerca l'ultima. Ad esempio, poniamo di voler trovare il primo elemento differente fra due vettori:
\begin{lstlisting}	
arrayl: .WORD 1, 2, 3, 4, 5, 6, 7, 8, 9, 10
array2: .WORD 1, 2, 3, 4, 7, 6, 7, 8, 9, 10

CLD
LEA array 1, %ESI
LEA array2, %EDI
MOV $10, %ECX
REPE CMPSW
\end{lstlisting}

dove si noti che alla fine del ciclo EDI e ESI puntano all'elemento successivo. 

\subsubsection{Prefissi di ripetizione}
Vediamo nel dettaglio il prefisso \lstinline|REP|, e le sue varianti \lstinline|REPE| e \lstinline|REPNE|.
Bisogna ricordare che questi prefissi si applicano ad istruzioni, non a blocchi di codice.
\begin{itemize}
	\item \lstinline|REP|: si può usare con \lstinline|MOVS|, \lstinline|LODS|, \lstinline|STOS|, \lstinline|INS| e \lstinline|OUTS|, anche se l'utilizzo con \lstinline|LODS| è privo di senso (almeno che non si voglia ottenere l'ultimo elemento...).
	\item \lstinline|REPE| e \lstinline|REPNE|: si può usare con \lstinline|CMPS| e \lstinline|SCAS|, ed effettua al massimo ECX ripetizioni, finché la condizione specificata è vera.
\end{itemize}

\subsubsection{Perchè due direzioni?}
L'uso di due direzioni di scorrimento di stringhe attraverso il flag DF è utile, sopratutto nel caso si debbano fare traslazioni del vettore (copia di buffer \textbf{parzialmente sovrapposti}).
Infatti, cercando si spostare il vettore a destra spostandoci verso destra, finiremo per copiare sempre gli stessi dati.

\subsection{Note sull'efficienza}
Un compilatore ottimizza il codice in alto livello per il sistema su cui quel codice dovrà girare.
Un assemblatore, invece, traduce le istruzioni una per una.

\subsubsection{Tempo di esecuzione di un processo}
Un processo è un programma in esecuzione con dei dati.
In questo, dipende dai dati, dallo stato del sistema, e da cosa sta facendo il processore (chi lo sta usando?).
Questo rende il calcolatore una macchina poco prevedibile, e il tempo di esecuzione del processo difficile da calcolare a priori. Di base, infatti:
\begin{itemize}
	\item Il clock non va a velocità costante;
	\item Il vostro processo non necessariamente gira su un solo core;
	\item Altri meccanismi introducono variabilità considerevoli:
		\begin{itemize}
			\item Memorie cache;
			\item Code di prefetch;
			\item Esecuzione in pipeline: eseguire un'istruzione significa fare fetch dell'istruzione, recuperare l'OPCODE, il sorgente, scrivere sul destinatario, ecc... conviene eseguire queste operazioni in pipeline, cioè eseguendo in parallelo più istruzioni possibili contemporaneamente;
			\item Esecuzione non sequenziale: il processore non esegue necessariamente il codice nell'ordine in cui è scritto: se possibile, modifica l'ordine in modo dal caricare in modo più efficiente possibile la pipeline;
			\item Branch prediction: quando si esegue in pipeline, le istruzioni condizionali creano forti bottleneck di prestazioni. Per ovviare a questo problema, il processore cerca di predire il tipo della prossima istruzione, pagando un prezzo nel caso si sbagli, ma ottenendo un significativo incremento di velocità nel caso abbia successo.
		\end{itemize}
\end{itemize}

\subsubsection{Lunghezza delle istruzioni e tempo di fetch}
Il numero di byte occupati da un'istruzione dipende dall'OPCODE e dal tipo di indirizzamento.
Se gli operandi sono \textbf{registri}, le istruzioni stanno normalmente su 1 byte; gli operandi \textbf{immediati} devono essere codificati (in 1, 2 o 4 byte); i \textbf{displacement} occupano 4 byte.

La lunghezza delle istruzioni, oltre alle dimensioni dei file binari, influenza anche il tempo di fetch delle stesse, e va quindi tenuto in considerazione.

\subsubsection{Tempo di esecuzione delle istruzioni}
Il tempo di esecuzione delle istruzioni dipende molto dall'architettura specifica del processore (anche in processori della stessa famiglia).

Abbiamo che le istruzioni ALU (escluse MUL e DIV) costano poco, su O(1) cicli di clock. Le MUL e DIV costano sui O(10) cicli di clock, e per questo vengono tradotte in procedure alternative (attraverso LEA o le istruzioni di shift) dai compilatori attraverso apposite tablle di corrispondenza.

Le operazioni più costose sono quelle dellA FPU (Floating Point Unit), che richiedono sulle O(100) istruzioni.

Anche le istruzioni condizionali sono molto costose, ma per i motivi visti prima che rallentano le pipeline.


\end{document}



\documentclass[a4paper,11pt]{article}
\usepackage[a4paper, margin=8em]{geometry}

% usa i pacchetti per la scrittura in italiano
\usepackage[french,italian]{babel}
\usepackage[T1]{fontenc}
\usepackage[utf8]{inputenc}
\frenchspacing 

% usa i pacchetti per la formattazione matematica
\usepackage{amsmath, amssymb, amsthm, amsfonts}

% usa altri pacchetti
\usepackage{gensymb}
\usepackage{hyperref}
\usepackage{standalone}

% imposta il titolo
\title{Appunti Reti Logiche}
\author{Luca Seggiani}
\date{2024}

% imposta lo stile
% usa helvetica
\usepackage[scaled]{helvet}
% usa palatino
\usepackage{palatino}
% usa un font monospazio guardabile
\usepackage{lmodern}

\renewcommand{\rmdefault}{ppl}
\renewcommand{\sfdefault}{phv}
\renewcommand{\ttdefault}{lmtt}

% disponi il titolo
\makeatletter
\renewcommand{\maketitle} {
	\begin{center} 
		\begin{minipage}[t]{.8\textwidth}
			\textsf{\huge\bfseries \@title} 
		\end{minipage}%
		\begin{minipage}[t]{.2\textwidth}
			\raggedleft \vspace{-1.65em}
			\textsf{\small \@author} \vfill
			\textsf{\small \@date}
		\end{minipage}
		\par
	\end{center}

	\thispagestyle{empty}
	\pagestyle{fancy}
}
\makeatother

% disponi teoremi
\usepackage{tcolorbox}
\newtcolorbox[auto counter, number within=section]{theorem}[2][]{%
	colback=blue!10, 
	colframe=blue!40!black, 
	sharp corners=northwest,
	fonttitle=\sffamily\bfseries, 
	title=Teorema~\thetcbcounter: #2, 
	#1
}

% disponi definizioni
\newtcolorbox[auto counter, number within=section]{definition}[2][]{%
	colback=red!10,
	colframe=red!40!black,
	sharp corners=northwest,
	fonttitle=\sffamily\bfseries,
	title=Definizione~\thetcbcounter: #2,
	#1
}

% disponi codice
\usepackage{listings}
\usepackage[table]{xcolor}

\definecolor{codegreen}{rgb}{0,0.6,0}
\definecolor{codegray}{rgb}{0.5,0.5,0.5}
\definecolor{codepurple}{rgb}{0.58,0,0.82}
\definecolor{backcolour}{rgb}{0.95,0.95,0.92}

\lstdefinestyle{codestyle}{
		backgroundcolor=\color{black!5}, 
		commentstyle=\color{codegreen},
		keywordstyle=\bfseries\color{magenta},
		numberstyle=\sffamily\tiny\color{black!60},
		stringstyle=\color{green!50!black},
		basicstyle=\ttfamily\footnotesize,
		breakatwhitespace=false,         
		breaklines=true,                 
		captionpos=b,                    
		keepspaces=true,                 
		numbers=left,                    
		numbersep=5pt,                  
		showspaces=false,                
		showstringspaces=false,
		showtabs=false,                  
		tabsize=2
}

\lstdefinestyle{shellstyle}{
		backgroundcolor=\color{black!5}, 
		basicstyle=\ttfamily\footnotesize\color{black}, 
		commentstyle=\color{black}, 
		keywordstyle=\color{black},
		numberstyle=\color{black!5},
		stringstyle=\color{black}, 
		showspaces=false,
		showstringspaces=false, 
		showtabs=false, 
		tabsize=2, 
		numbers=none, 
		breaklines=true
}


\lstdefinelanguage{assembler}{ 
  keywords={AAA, AAD, AAM, AAS, ADC, ADCB, ADCW, ADCL, ADD, ADDB, ADDW, ADDL, AND, ANDB, ANDW, ANDL,
        ARPL, BOUND, BSF, BSFL, BSFW, BSR, BSRL, BSRW, BSWAP, BT, BTC, BTCB, BTCW, BTCL, BTR, 
        BTRB, BTRW, BTRL, BTS, BTSB, BTSW, BTSL, CALL, CBW, CDQ, CLC, CLD, CLI, CLTS, CMC, CMP,
        CMPB, CMPW, CMPL, CMPS, CMPSB, CMPSD, CMPSW, CMPXCHG, CMPXCHGB, CMPXCHGW, CMPXCHGL,
        CMPXCHG8B, CPUID, CWDE, DAA, DAS, DEC, DECB, DECW, DECL, DIV, DIVB, DIVW, DIVL, ENTER,
        HLT, IDIV, IDIVB, IDIVW, IDIVL, IMUL, IMULB, IMULW, IMULL, IN, INB, INW, INL, INC, INCB,
        INCW, INCL, INS, INSB, INSD, INSW, INT, INT3, INTO, INVD, INVLPG, IRET, IRETD, JA, JAE,
        JB, JBE, JC, JCXZ, JE, JECXZ, JG, JGE, JL, JLE, JMP, JNA, JNAE, JNB, JNBE, JNC, JNE, JNG,
        JNGE, JNL, JNLE, JNO, JNP, JNS, JNZ, JO, JP, JPE, JPO, JS, JZ, LAHF, LAR, LCALL, LDS,
        LEA, LEAVE, LES, LFS, LGDT, LGS, LIDT, LMSW, LOCK, LODSB, LODSD, LODSW, LOOP, LOOPE,
        LOOPNE, LSL, LSS, LTR, MOV, MOVB, MOVW, MOVL, MOVSB, MOVSD, MOVSW, MOVSX, MOVSXB,
        MOVSXW, MOVSXL, MOVZX, MOVZXB, MOVZXW, MOVZXL, MUL, MULB, MULW, MULL, NEG, NEGB, NEGW,
        NEGL, NOP, NOT, NOTB, NOTW, NOTL, OR, ORB, ORW, ORL, OUT, OUTB, OUTW, OUTL, OUTSB, OUTSD,
        OUTSW, POP, POPL, POPW, POPB, POPA, POPAD, POPF, POPFD, PUSH, PUSHL, PUSHW, PUSHB, PUSHA, 
				PUSHAD, PUSHF, PUSHFD, RCL, RCLB, RCLW, MOVSL, MOVSB, MOVSW, STOSL, STOSB, STOSW, LODSB, LODSW,
				LODSL, INSB, INSW, INSL, OUTSB, OUTSL, OUTSW
        RCLL, RCR, RCRB, RCRW, RCRL, RDMSR, RDPMC, RDTSC, REP, REPE, REPNE, RET, ROL, ROLB, ROLW,
        ROLL, ROR, RORB, RORW, RORL, SAHF, SAL, SALB, SALW, SALL, SAR, SARB, SARW, SARL, SBB,
        SBBB, SBBW, SBBL, SCASB, SCASD, SCASW, SETA, SETAE, SETB, SETBE, SETC, SETE, SETG, SETGE,
        SETL, SETLE, SETNA, SETNAE, SETNB, SETNBE, SETNC, SETNE, SETNG, SETNGE, SETNL, SETNLE,
        SETNO, SETNP, SETNS, SETNZ, SETO, SETP, SETPE, SETPO, SETS, SETZ, SGDT, SHL, SHLB, SHLW,
        SHLL, SHLD, SHR, SHRB, SHRW, SHRL, SHRD, SIDT, SLDT, SMSW, STC, STD, STI, STOSB, STOSD,
        STOSW, STR, SUB, SUBB, SUBW, SUBL, TEST, TESTB, TESTW, TESTL, VERR, VERW, WAIT, WBINVD,
        XADD, XADDB, XADDW, XADDL, XCHG, XCHGB, XCHGW, XCHGL, XLAT, XLATB, XOR, XORB, XORW, XORL},
  keywordstyle=\color{blue}\bfseries,
  ndkeywordstyle=\color{darkgray}\bfseries,
  identifierstyle=\color{black},
  sensitive=false,
  comment=[l]{\#},
  morecomment=[s]{/*}{*/},
  commentstyle=\color{purple}\ttfamily,
  stringstyle=\color{red}\ttfamily,
  morestring=[b]',
  morestring=[b]"
}

\lstset{language=assembler, style=codestyle}

% disponi sezioni
\usepackage{titlesec}

\titleformat{\section}
	{\sffamily\Large\bfseries} 
	{\thesection}{1em}{} 
\titleformat{\subsection}
	{\sffamily\large\bfseries}   
	{\thesubsection}{1em}{} 
\titleformat{\subsubsection}
	{\sffamily\normalsize\bfseries} 
	{\thesubsubsection}{1em}{}

% tikz
\usepackage{tikz}

% float
\usepackage{float}

% grafici
\usepackage{pgfplots}
\pgfplotsset{width=10cm,compat=1.9}

% disponi alberi
\usepackage{forest}

\forestset{
	rectstyle/.style={
		for tree={rectangle,draw,font=\large\sffamily}
	},
	roundstyle/.style={
		for tree={circle,draw,font=\large}
	}
}

% disponi algoritmi
\usepackage{algorithm}
\usepackage{algorithmic}
\makeatletter
\renewcommand{\ALG@name}{Algoritmo}
\makeatother

% disponi numeri di pagina
\usepackage{fancyhdr}
\fancyhf{} 
\fancyfoot[L]{\sffamily{\thepage}}

\makeatletter
\fancyhead[L]{\raisebox{1ex}[0pt][0pt]{\sffamily{\@title \ \@date}}} 
\fancyhead[R]{\raisebox{1ex}[0pt][0pt]{\sffamily{\@author}}}
\makeatother

\begin{document}
% sezione (data)
\section{Lezione del 03-10-24}

% stili pagina
\thispagestyle{empty}
\pagestyle{fancy}

% testo
\subsection{Assembler a 64 bit}
Finora abbiamo studiato il linguaggio assembler a 32 bit (registri estesi EAX, EBX, ecc...).
Vediamo adesso alcune caratteristiche dell'assembler a 64 bit.

Nei processori a 64 bit Intel-AMD x86 abbiamo 16 registri generali a 64 bit, con prefisso R, e che quindi si indicano come RAX, RBX, ecc...
Di questi si può indirizzare la parte estesa dei 32 bit meno significativi (EAX), i 16 bit meno significativi (AX), e gli 8 bit meno significativi (AL).
Per RAX, RBX, RCX e RDX si possono inoltre indirizzare gli 8 bit precedenti ad AL, BL, CL e DL usando AH, BH, CH e DH, ma questo è sconsigliato in quanto ci sono diverse limitazioni (non sono compatibili col prefisso REX).

Una lista completa dei registri genrali è la seguente, inclusi i nomi dei sottoregistri di dimensione minore:

\begin{table}[h!]
	\center \rowcolors{2}{white}{black!10}
	\begin{tabular} { c || c | c | c | c }
		\bfseries 64 bit & \bfseries 32 bit & \bfseries 16 bit & \bfseries 8 bit & \bfseries 8 bit (legacy) \\
		\hline 
		RAX & EAX & AX & AL & AH \\
		RBX & EBX & BX & BL & BH \\
		RCX & ECX & CX & CL & CH \\
		RDX & EDX & DX & DL & DH \\
		RSP & ESP & SP & SPL & \\
		RBP & EBP & BP & BPL & \\
		RSI & ESI & SI & SIL & \\
		RDI & EDI & DI & DIL & \\
		R8 &R8D &R8W &R8B & \\
		R9 &R9D &R9W &R9B & \\
		R10&R10D&R10W&R10B & \\
		R11&R11D&R11W&R11B & \\
		R12&R12D&R12W&R12B & \\
		R13&R13D&R13W&R13B & \\
		R14&R14D&R14W&R14B & \\
		R15&R15D&R15W&R15B & \\
	\end{tabular}
\end{table}

Ricordiamo poi i registri RIP, l'instruction pointer, e RFLAGS che è il registro dei flag.

\subsubsection{Spazio indirizzabile}
Tecnicamente con architettura a 64 bit si potrebbero indirizzare $2^{64}$ byte distinti, ma i processori moderni permettono di indirizzarne solo $2^{48} = 256 \ \mathrm{TiB}$, con alcuni modelli più recenti che arrivano a $2^{57}= 128 \ \mathrm{PiB}$.
I 48 (o 57) bit occupati sono i meno significativi, e i restanti 16 (o 7) devono avere il valore del bit più significativo utilizzato.
Questo significa che sono indirizzabili effettivamente due porzioni contigue ma separate fra di loro di memoria:
\begin{table}[h!]
	\center \rowcolors{2}{white}{black!10}
	\begin{tabular} { c | p{4cm} | p{4cm} }
		& \bfseries 48 bit & \bfseries 57 bit \\
		\hline
		\bfseries Regione alta & \texttt{0000 0000 0000 0000} \texttt{0000 7fff ffff ffff} & \texttt{0000 0000 0000 0000} \texttt{01ff ffff ffff ffff} \\
		\bfseries Regione bassa & \texttt{ffff 8000 0000 0000} \texttt{ffff ffff ffff ffff} & \texttt{fe00 0000 0000 0000} \texttt{ffff ffff ffff ffff} \\
	\end{tabular}
\end{table}

Lo spazio I/O, infine, è di $2^{16} = 64 \ \mathrm{KiB}$ locazioni.

\subsubsection{Istruzioni}
Le operazioni possono possono usare 1, 2, 4 o 8 byte per un operando (rispettivamente Byte, Word, Long e Quad).

Notiamo che non possiamo usare displacement o operandi immediati a 64 bit: siamo limitati a 32 bit.
Per ovviare a questo problema esiste una versione alternativa della \lstinline|MOV|:

\subsubsection{MOVABS}
\begin{itemize}
	\item \textbf{Formato:} \lstinline|MOVABS $const, destination|
	\item \textbf{Azione:} porta una costante a 64 bit (che ci permette di scrivere) in un indirizzo generale.
	\item \textbf{Flag:} nessuno.
\end{itemize}

		\begin{table}[H]
		\center \rowcolors{2}{white}{black!10}
			\begin{tabular} { c | p{5cm} }
				\bfseries Operandi & \bfseries Esempi \\
				\hline
				Immediato & \lstinline|MOVABS $0xffff8105402300ef, %RBX| \\ 
				Memoria & \lstinline|CALL 0x00ef0b2a, %RAX| \\ 
				Registro & \lstinline|CALL %RAX, 0x00ef0b2a|
			\end{tabular}
		\end{table}

\par\smallskip 
In generale, in assember a 64 bit si usano registri con valori base di 64 bit, e poi si indirizza con displacement a 32 bit, che in complemento a 2 concedono $\pm 2^{32}$, ergo $\pm 2 \mathrm{GB}$ di memoria indirizzabile rispetto alla base.

\subsection{Reti logiche}
Una rete logica è un modello astratto di un sistema fisico, costituito da dispositivi tra loro interconnessi.
Le informazioni vengono codificate da questi dispositivi attraverso fenomeni fisici che si presentano in due aspetti distinti (corrente forte / corrente debole, tensione forte / tensione debole, magnetizzazione / non magnetizzazione, ecc...).

\subsubsection{Caratterizzazione di rete logica}
Una rete logica è caratterizzata da:
\begin{itemize}
	\item Un'insieme di $N$ variabili di ingresso. Il loro valore all'istante temporale $t$ si chiama stato di ingresso. L'insieme di tutti i $2^N$ stati di ingresso si indicherà come $X.X = \{ x_{N-1} x_{N-2} ... x_1 x_0 \}$. 
	\item Un'insieme di $M$ variabili di uscita. Il loro valore all'istante temporale $t$ si chiama stato di uscita. L'insieme di tutti i $2^M$ stati di uscita si indicherà come $Z.Z = \{ x_{M-1} x_{M-2} ... x_1 x_0 \}$. 
	\item Una legge di evoluzione che determina come le uscite si evolvono in funzione degli ingressi.
\end{itemize}

Possiamo classificare le reti logiche in base a 2 criteri riguardanti l'evoluzione nel tempo:
\begin{itemize}
	\item \textbf{Presenza/assenza di memoria}:
		\begin{itemize}
			\item \textbf{Reti combinatorie:} analoghe a funzioni matematiche, le loro uscite dipendono solo dai loro ingressi in un qualsiasi istanti $t$;
			\item \textbf{Reti sequenziali:} lo stato di uscita dipende dalla storia degli ingressi precedenti, ergo sono reti con memoria.
		\end{itemize}
	\item \textbf{Temporizzazione della legge di evoluzione:}
		\begin{itemize}
			\item \textbf{Reti asincrone:} l'aggiornamento delle uscite avviene costantemente nel tempo;
			\item \textbf{Reti sincronizzate:} l'aggiornamento delle uscite avviene ad istanti di sincronizzazione discreti nel tempo.
		\end{itemize}
\end{itemize}

I modelli sono ortogonali, ergo possiamo avere qualsiasi delle 4 combinazioni di queste caratteristiche:
\begin{itemize}
	\item Reti combinatorie (si considerano le sincronizzate come caso particolare);
	\item Reti sequenziali asincrone;
	\item Reti sequenziali sincronizzate.
\end{itemize}

Quindi in sostanza una rete logica comunica con l'esterno attraverso variabili logiche (0 e 1).
L'interpretazione di questi messaggi è una convenzione del progettista, programatore, ecc...

Usiamo le reti logiche per modellizzare circuiti elettronici all'interno del calcolatore, che codificano le informazioni in tensione.
Notiamo quindi che una rete logica fisica ha, oltre agli ingressi e alle uscite, i collegamenti ai terminali positivi e negativi di un generatore di tensione, che noi ignoreremo. 

\subsection{Transizione dei segnali}
Una variabile logica (per noi il voltaggio su un circuito) può settarsi (andare a 1), restare settato per tempi paragonabili a $\Delta T$, e resettarsi (andare a 0) in un qualsiasi momento temporale $t$:

\begin{center}
\begin{tikzpicture}
    \begin{axis}[
        xlabel={$t$},
        ylabel={$V$},
        xmin=0, xmax=14,
        ymin=-1, ymax=6,
        grid=major,
        domain=0:14,
				xtick={5,8},
				ytick={0,5},
				xticklabels={$t_1$,$t_2$},
				yticklabels={$0$, $V_{max}$},
        samples=100,
        legend pos=south west,
    		width=14cm,
				height=7cm
			]
    \addplot[blue, thick] {5 * (x >= 5) * (x <=  8)};
    \end{axis}
\end{tikzpicture}
\end{center}

In un sistema fisico reale, durante la transizione c'è un periodo di indecisione in cui il voltaggio sale o scende fisicamente fino al valore necessario, sotto l'atto di una qualche potenza. Vediamo il grafico a $\Delta t << \Delta T$:

\begin{center}
\begin{tikzpicture}
    \begin{axis}[
        xlabel={$t$},
        ylabel={$V$},
        xmin=4.8, xmax=5.2,
        ymin=-1, ymax=6,
        grid=major,
        domain=0:14,
				xtick={5,8},
				ytick={0,5},
				xticklabels={$t_1$,$t_2$},
				yticklabels={$0$, $V_{max}$},
        samples=100,
        legend pos=south west,
    		width=14cm,
				height=7cm
			]
    \addplot[blue, thick] {5 * (x >= 5) * (x <=  8)};
    \end{axis}
\end{tikzpicture}
\end{center}

Decidiamo di ignorare questo problema, in quanto abbiamo visto che il $\Delta t$ di transizione è molto più piccolo del $\Delta t$ di stasi delle variabili.

Il problema si presenta nel caso si parli di \textbf{contemporaneità}. 
Supponiamo di avere una rete logica con due ingressi $x_0$ e $x_1$ e un'uscita $z_0$.
Abbiamo che prima dell'istante $t_1$ lo stato di ingresso è $(1,0)$, e che subito dopo lo stesso stato è $(0, 1)$.
Nell'istante di transizione non abbiamo la sicurezza che le singole transizioni delle due variabili della rete avvengano contemporaneamente:

\begin{tikzpicture}
    \begin{axis}[
        xlabel={$t$},
        ylabel={$V$},
        xmin=0, xmax=8,
        ymin=-1, ymax=6,
        grid=major,
        domain=0:8,
        samples=100,
				xtick={3},
				ytick={0,5},
				xticklabels={$t_1$},
				yticklabels={$0$, $V_{max}$},
        legend pos=south west,
    		width=14cm,
				height=7cm
			]
    \addplot[blue, thick] {5 * (x >= 3)};
    \addplot[red, thick] {5 * (x >= 3.5) + 0.25};
    \end{axis}
\end{tikzpicture}

Questa considerazione è importante nel caso delle reti logiche asincrone, dove considerare le transizioni come contemporanee potrebbe portare alla comparsa di stati di uscita spuri, e nelle reti sequenziali, dove potrebbe portare ad evoluzioni imprevedibili del sistema.

\subsection{Il linguaggio Verilog}
Per descrivere le reti logiche fa comodo adottare una \textbf{notazione testuale}.
Per reti semplici useremo disegni o espressioni algebriche: per reti complesse introduciamo un \textbf{linguaggio di descrizione hardware}, il \textbf{Verilog}.
Questo linguaggio è più \textbf{compatto}, e può essere \textbf{interpretato} automaticamente da una macchina, permettendoci di effettuare prove (e realizzare \textbf{diagrammi di temporizzazione}).

% cosa voglio sapere del verilog? 
% le basi (cos'è un modulo, cos'è un testbench, ecc...)
% BENE gli assegnamenti continui, procedurali, bloccanti, non bloccanti, ecc...

Non si riporteranno appunti riguardanti operatori e sintassi particolarmente specifiche del Verilog, in quanto esistono testi sicuramente più utili e approfonditi: procederemo principalmente per esempi, esplicitando quando si rende necessario particolarità del linguaggio.

\subsubsection{Struttura di una sintesi Verilog}
Il linguaggio Verilog descrive \textbf{moduli}.
Un modulo è formato da un insieme di \textbf{input} e \textbf{output}, e da una \textbf{struttura interna} che descrive la legge di evoluzione degli output in funzione degli input.
Ad esempio, una rete basilare potrebbe essere:
\begin{lstlisting}[language=verilog, style=codestyle]	
module rete(x, z);
	input x;
	output z;
	wire y;
	assign y = x;
	assign z = y;
endmodule
\end{lstlisting}

Nell'esempio, si definisce un modulo \lstinline|rete|, formato da un input $x$ e un output $z$. 
La realizzazione interna della rete è formata da un filo $y$ a cui sono connessi sia l'input che l'output.
Il funzionamento della rete è quindi semplicemente quello di replicare il suo ingresso.

In particolare, diciamo che la parola chiave \lstinline|assign| rapprenta un \textbf{assegnamento continuo}.
Più avanti vedremo i diversi tipi di assegnamento e le differenze fra di loro.

\subsection{Reti combinatorie}
Il primo tipo di reti logiche che andiamo a studiare sono le \textbf{reti combinatorie}.
Una rete combinatoria è caratterizzata da:
\begin{itemize}
	\item Un'insieme di $N$ variabili logiche di ingresso;
	\item Un'insieme di $M$ variabili logiche di uscita;
	\item Una descrizione funzionale $F: X \rightarrow Z$ che mappa stati di ingresso a stati di uscita;
	\item Una legge di evoluzione nel tempo che adegua $F(X)$ allo stato di ingresso $X$ continuamente.
\end{itemize}

\subsubsection{Tempo di attraversamento}
Il tempo di attraversamento (o di accesso) è una caratteristica di tutte le reti logiche asincrone: è il tempo necessario perché la rete si "accorga" della variazione degli ingressi e aggiorni di conseguenza le sue uscite.

Questo tempo è solitamente non nullo, ed è quindi necessario attendere che la rete arrivi a \textbf{regime} prima di valutare le uscite.
Questo vincolo prende il nome di \textbf{pilotaggio in modo fondamentale}: si dice che è una rete è pilotata in modo fondamentale quando chi la pilota aspetta sempre che essa arrivi a regime prima di valutare le sue uscite.
\end{document}


\documentclass[a4paper,11pt]{article}
\usepackage[a4paper, margin=8em]{geometry}

% usa i pacchetti per la scrittura in italiano
\usepackage[french,italian]{babel}
\usepackage[T1]{fontenc}
\usepackage[utf8]{inputenc}
\frenchspacing 

% usa i pacchetti per la formattazione matematica
\usepackage{amsmath, amssymb, amsthm, amsfonts}

% usa altri pacchetti
\usepackage{gensymb}
\usepackage{hyperref}
\usepackage{standalone}

\usepackage{colortbl}

% circuiti
\usepackage{circuitikz}
\usetikzlibrary{babel}

% imposta il titolo
\title{Appunti Reti Logiche}
\author{Luca Seggiani}
\date{2024}

% imposta lo stile
% usa helvetica
\usepackage[scaled]{helvet}
% usa palatino
\usepackage{palatino}
% usa un font monospazio guardabile
\usepackage{lmodern}

\renewcommand{\rmdefault}{ppl}
\renewcommand{\sfdefault}{phv}
\renewcommand{\ttdefault}{lmtt}

% disponi il titolo
\makeatletter
\renewcommand{\maketitle} {
	\begin{center} 
		\begin{minipage}[t]{.8\textwidth}
			\textsf{\huge\bfseries \@title} 
		\end{minipage}%
		\begin{minipage}[t]{.2\textwidth}
			\raggedleft \vspace{-1.65em}
			\textsf{\small \@author} \vfill
			\textsf{\small \@date}
		\end{minipage}
		\par
	\end{center}

	\thispagestyle{empty}
	\pagestyle{fancy}
}
\makeatother

% disponi teoremi
\usepackage{tcolorbox}
\newtcolorbox[auto counter, number within=section]{theorem}[2][]{%
	colback=blue!10, 
	colframe=blue!40!black, 
	sharp corners=northwest,
	fonttitle=\sffamily\bfseries, 
	title=Teorema~\thetcbcounter: #2, 
	#1
}

% disponi definizioni
\newtcolorbox[auto counter, number within=section]{definition}[2][]{%
	colback=red!10,
	colframe=red!40!black,
	sharp corners=northwest,
	fonttitle=\sffamily\bfseries,
	title=Definizione~\thetcbcounter: #2,
	#1
}

% disponi codice
\usepackage{listings}
\usepackage[table]{xcolor}

\definecolor{codegreen}{rgb}{0,0.6,0}
\definecolor{codegray}{rgb}{0.5,0.5,0.5}
\definecolor{codepurple}{rgb}{0.58,0,0.82}
\definecolor{backcolour}{rgb}{0.95,0.95,0.92}

\lstdefinestyle{codestyle}{
		backgroundcolor=\color{black!5}, 
		commentstyle=\color{codegreen},
		keywordstyle=\bfseries\color{magenta},
		numberstyle=\sffamily\tiny\color{black!60},
		stringstyle=\color{green!50!black},
		basicstyle=\ttfamily\footnotesize,
		breakatwhitespace=false,         
		breaklines=true,                 
		captionpos=b,                    
		keepspaces=true,                 
		numbers=left,                    
		numbersep=5pt,                  
		showspaces=false,                
		showstringspaces=false,
		showtabs=false,                  
		tabsize=2
}

\lstdefinestyle{shellstyle}{
		backgroundcolor=\color{black!5}, 
		basicstyle=\ttfamily\footnotesize\color{black}, 
		commentstyle=\color{black}, 
		keywordstyle=\color{black},
		numberstyle=\color{black!5},
		stringstyle=\color{black}, 
		showspaces=false,
		showstringspaces=false, 
		showtabs=false, 
		tabsize=2, 
		numbers=none, 
		breaklines=true
}


\lstdefinelanguage{assembler}{ 
  keywords={AAA, AAD, AAM, AAS, ADC, ADCB, ADCW, ADCL, ADD, ADDB, ADDW, ADDL, AND, ANDB, ANDW, ANDL,
        ARPL, BOUND, BSF, BSFL, BSFW, BSR, BSRL, BSRW, BSWAP, BT, BTC, BTCB, BTCW, BTCL, BTR, 
        BTRB, BTRW, BTRL, BTS, BTSB, BTSW, BTSL, CALL, CBW, CDQ, CLC, CLD, CLI, CLTS, CMC, CMP,
        CMPB, CMPW, CMPL, CMPS, CMPSB, CMPSD, CMPSW, CMPXCHG, CMPXCHGB, CMPXCHGW, CMPXCHGL,
        CMPXCHG8B, CPUID, CWDE, DAA, DAS, DEC, DECB, DECW, DECL, DIV, DIVB, DIVW, DIVL, ENTER,
        HLT, IDIV, IDIVB, IDIVW, IDIVL, IMUL, IMULB, IMULW, IMULL, IN, INB, INW, INL, INC, INCB,
        INCW, INCL, INS, INSB, INSD, INSW, INT, INT3, INTO, INVD, INVLPG, IRET, IRETD, JA, JAE,
        JB, JBE, JC, JCXZ, JE, JECXZ, JG, JGE, JL, JLE, JMP, JNA, JNAE, JNB, JNBE, JNC, JNE, JNG,
        JNGE, JNL, JNLE, JNO, JNP, JNS, JNZ, JO, JP, JPE, JPO, JS, JZ, LAHF, LAR, LCALL, LDS,
        LEA, LEAVE, LES, LFS, LGDT, LGS, LIDT, LMSW, LOCK, LODSB, LODSD, LODSW, LOOP, LOOPE,
        LOOPNE, LSL, LSS, LTR, MOV, MOVB, MOVW, MOVL, MOVSB, MOVSD, MOVSW, MOVSX, MOVSXB,
        MOVSXW, MOVSXL, MOVZX, MOVZXB, MOVZXW, MOVZXL, MUL, MULB, MULW, MULL, NEG, NEGB, NEGW,
        NEGL, NOP, NOT, NOTB, NOTW, NOTL, OR, ORB, ORW, ORL, OUT, OUTB, OUTW, OUTL, OUTSB, OUTSD,
        OUTSW, POP, POPL, POPW, POPB, POPA, POPAD, POPF, POPFD, PUSH, PUSHL, PUSHW, PUSHB, PUSHA, 
				PUSHAD, PUSHF, PUSHFD, RCL, RCLB, RCLW, MOVSL, MOVSB, MOVSW, STOSL, STOSB, STOSW, LODSB, LODSW,
				LODSL, INSB, INSW, INSL, OUTSB, OUTSL, OUTSW
        RCLL, RCR, RCRB, RCRW, RCRL, RDMSR, RDPMC, RDTSC, REP, REPE, REPNE, RET, ROL, ROLB, ROLW,
        ROLL, ROR, RORB, RORW, RORL, SAHF, SAL, SALB, SALW, SALL, SAR, SARB, SARW, SARL, SBB,
        SBBB, SBBW, SBBL, SCASB, SCASD, SCASW, SETA, SETAE, SETB, SETBE, SETC, SETE, SETG, SETGE,
        SETL, SETLE, SETNA, SETNAE, SETNB, SETNBE, SETNC, SETNE, SETNG, SETNGE, SETNL, SETNLE,
        SETNO, SETNP, SETNS, SETNZ, SETO, SETP, SETPE, SETPO, SETS, SETZ, SGDT, SHL, SHLB, SHLW,
        SHLL, SHLD, SHR, SHRB, SHRW, SHRL, SHRD, SIDT, SLDT, SMSW, STC, STD, STI, STOSB, STOSD,
        STOSW, STR, SUB, SUBB, SUBW, SUBL, TEST, TESTB, TESTW, TESTL, VERR, VERW, WAIT, WBINVD,
        XADD, XADDB, XADDW, XADDL, XCHG, XCHGB, XCHGW, XCHGL, XLAT, XLATB, XOR, XORB, XORW, XORL},
  keywordstyle=\color{blue}\bfseries,
  ndkeywordstyle=\color{darkgray}\bfseries,
  identifierstyle=\color{black},
  sensitive=false,
  comment=[l]{\#},
  morecomment=[s]{/*}{*/},
  commentstyle=\color{purple}\ttfamily,
  stringstyle=\color{red}\ttfamily,
  morestring=[b]',
  morestring=[b]"
}

\lstset{language=assembler, style=codestyle}

% disponi sezioni
\usepackage{titlesec}

\titleformat{\section}
	{\sffamily\Large\bfseries} 
	{\thesection}{1em}{} 
\titleformat{\subsection}
	{\sffamily\large\bfseries}   
	{\thesubsection}{1em}{} 
\titleformat{\subsubsection}
	{\sffamily\normalsize\bfseries} 
	{\thesubsubsection}{1em}{}

% tikz
\usepackage{tikz}

% float
\usepackage{float}

% grafici
\usepackage{pgfplots}
\pgfplotsset{width=10cm,compat=1.9}

% disponi alberi
\usepackage{forest}

\forestset{
	rectstyle/.style={
		for tree={rectangle,draw,font=\large\sffamily}
	},
	roundstyle/.style={
		for tree={circle,draw,font=\large}
	}
}

% disponi algoritmi
\usepackage{algorithm}
\usepackage{algorithmic}
\makeatletter
\renewcommand{\ALG@name}{Algoritmo}
\makeatother

% disponi numeri di pagina
\usepackage{fancyhdr}
\fancyhf{} 
\fancyfoot[L]{\sffamily{\thepage}}

\makeatletter
\fancyhead[L]{\raisebox{1ex}[0pt][0pt]{\sffamily{\@title \ \@date}}} 
\fancyhead[R]{\raisebox{1ex}[0pt][0pt]{\sffamily{\@author}}}
\makeatother

\begin{document}
% sezione (data)
\section{Lezione del 08-10-24}

% stili pagina
\thispagestyle{empty}
\pagestyle{fancy}

% testo
\subsection{Descrizione funzionale}
La caratteristica più importante di una rete combinatoria è la funzione $F$, cioé la descrizione funzionale.
Esistono più modi per esprimere questa funzione:
\begin{itemize}
	\item A parole;
	\item Usando notazioni testuali (e..g. il Verilog);
	\item Attraverso \textbf{tabelle di verità}.
		In una tabella di verità contiene due insiemi di colonne: gli ingressi e le uscite.
		Ogni riga mostra una configurazione di stati di ingresso e il corrispondente stato d'uscita. Ad esempio:
	\begin{table}[H]
		\center 
		\begin{tabular} { c  c  c | c c }
			$x_2$ & $x_1$ & $x_0$ & $z_1$ & $z_0$ \\ 
			\hline 
			$0$ & $0$ & $0$ & $0$ & $0$ \\
			$0$ & $0$ & $1$ & $-$ & $1$ \\
			$0$ & $1$ & $0$ & $1$ & $0$ \\
			...
		\end{tabular}
	\end{table}
	Si dice che la variabile di uscita \textbf{riconosce} particolari stati quando si attiva in presenza di essi.
	Inoltre, i trattini indicano stati \textbf{non specificati}, in inglese DC, \textit{don't care}.
	Questi non equivalgono alla fascia di indeterminazione, ma a uno dei due stati accettati, anche se non è importante quale.
	I \textit{don't care} vanno conservati, e non fissati a variabili come $0$ o $1$, in quanto è importante mantenere il funzionamento interno delle reti il più semplice possibile. 
\end{itemize}

\subsubsection{Descrizione e sintesi}
Una \textbf{descrizione} di una rete deve essere formale, in modo che si possa capire esattamente cosa fa quella rete.
La \textbf{sintesi} di una rete è il progetto stesso di realizzazione della rete, cioè quali componenti combinare in quale modo, ecc...
Prima si fa la descrizione, e poi la sintesi.

Notiamo una proprietà fondamentale: ogni rete combinatoria di $N$ ingressi e $M$ uscite può essere realizzata interconnettendo $M$ reti combinatorie ad $N$ ingressi ed una uscita.
Questo ci permette di trattare tutte le reti con reti con una sola uscita.

\subsection{Reti a 0 ingressi}
Le reti a 0 ingresso di uscita si chiamano \textbf{generatori di costante}, e rappresentano un caso degenere.
Si indicano come: 

\begin{center}
	\begin{circuitikz}
		\draw[->] (0.5,0) to (1,0);
    \draw (0,0) node[draw, rectangle, minimum width = 1cm, minimum height = 1cm] {1};

		\draw[->] (2.5,0) to (3,0);
    \draw (2,0) node[draw, rectangle, minimum width = 1cm, minimum height = 1cm] {0};
	\end{circuitikz}
\end{center}

La loro uscita chiaramente vale $1$ o $0$ costante.
Fisicamente, i generatori di costante si realizzano collegando resistori in serie al VCC (genera $1$) o a massa (genera $0$), ergo:

\begin{center}
	\begin{circuitikz}
		\draw[->] (1,0) to (1.5,0);
    \draw (0,0) node[draw, rectangle, minimum width = 2cm, minimum height = 2cm] {};

		\draw (0,-1) node[below] {1};
		
		\draw (0, 1) to[ american voltage source, l=VCC, transform shape, scale=0.5] (0,0);
		\draw (0,0) to [ R, transform shape, scale=0.5] (2,0);

	\end{circuitikz}
	\hspace{1cm}
	\begin{circuitikz}
		\draw[->] (1,0) to (1.5,0);
    \draw (0,0) node[draw, rectangle, minimum width = 2cm, minimum height = 2cm] {};

		\draw (0,-1) node[below] {0};

		\draw (0,0) to [ R, transform shape, scale=0.5] (2,0);
		\draw (0, 0) node[ground] {};

	\end{circuitikz}
\end{center}

\subsection{Reti a 1 ingresso}
\subsubsection{Invertitore}
L'invertitore è una rete descritta dalla tabella di verità:

\begin{table}[H]
	\center 
	\begin{tabular} { c | c }
		$x$ & $z$ \\ 
		\hline 
		$0$ & $1$ \\
		$1$ & $0$ \\
	\end{tabular}
\end{table}

e indicata come:

\begin{center}
	\begin{circuitikz}
			\draw
			(0,0) node[not port] (mynot) {};
	\end{circuitikz}
\end{center}

Essenzialmente nega il suo ingresso.

\subsubsection{Elemento neutro}
L'elemento neutro, detto anche \textit{buffer}, è una rete descritta dalla tabella di verità:

\begin{table}[H]
	\center 
	\begin{tabular} { c | c }
		$x$ & $z$ \\ 
		\hline 
		$0$ & $0$ \\
		$1$ & $1$ \\
	\end{tabular}
\end{table}

e indicata come:

\begin{center}
	\begin{circuitikz}
			\draw
			(0,0) node[buffer port] (mynot) {};
	\end{circuitikz}
\end{center}

Lascia il suo ingresso invariato.
Può avere un utilità come rete di rallentamento, in quanto, inevitabilmente, si perde tempo per attraversarla (pensa alla NOP).
Questo è utile per le temporizzazioni delle reti.

Inoltre, dal punto di vista elettrico, l'elemento neutro ha anche un utilità per la \textbf{rigenerazione} dei segnali.
Infatti, essendo collegato a massa e al VCC, può prendere segnali scadenti (vicini alla fascia di indeterminazione) e trasformarli in segnali di buona qualità (vicini al fondoscala).
Questa proprietà, veramente, è comune a tutte le reti logiche, ma l'elemento neutro è l'unico che non ha altri effetti collaterali.

\subsubsection{Reti costanti}
Si possono interpretare i generatori di costante come reti ad un ingresso degeneri.
Effettivamente, restano tali a se stesse, in quanto gli ingressi sono ignorati.
Le loro tabelle di verità sono:

\begin{center}
\begin{minipage}[t]{0.2\textwidth} % Left half of the page
	Generatore di 1:
\begin{table}[H]
	\center 
	\begin{tabular} { c | c }
		$x$ & $z$ \\ 
		\hline 
		$0$ & $1$ \\
		$1$ & $1$ \\
	\end{tabular}
\end{table}
\end{minipage}%
\hspace{2cm}
\begin{minipage}[t]{0.2\textwidth} % Right half of the page
	Generatore di 0:
\begin{table}[H]
	\center 
	\begin{tabular} { c | c }
		$x$ & $z$ \\ 
		\hline 
		$0$ & $0$ \\
		$1$ & $0$ \\
	\end{tabular}
\end{table}
\end{minipage}
\end{center}

\subsection{Reti a 2 ingressi}
La prima domanda da porsi quando si parla di reti a 2 (come $N$) ingressi, è quante reti possiamo creare in tutto.
Su $N$ ingressi, la tabella di verità avrà $2^N$ righe.
Le configurazioni possibili di $0$ e $1$ su $2^N$ righe sono $2^{2^N}$.
Ergo, nel caso $N=2$, abbiamo $2^{2^2} = 16$ possibili combinazioni, che sono:

\begin{table}[H]
	\center 
	\begin{tabular} { c  c | c >{\columncolor{green!40!white}}c c c 
										c c >{\columncolor{red!40!white}}c >{\columncolor{blue!40!white}}c
										>{\columncolor{purple!40!white}}c >{\columncolor{orange!40!white}}c c c 
										c c >{\columncolor{cyan!40!white}}c c }
		$x_1$ & $x_0$ & $z^0$ & $z^1$ & $z^2$ & $z^3$ & $z^4$ & $z^5$ & $z^6$ & $z^7$ & $z^8$ & $z^9$ & $z^{10}$ & $z^{11}$ & $z^{12}$ & $z^{13}$ & $z^{14}$ & $z^{15}$ \\ 
		\hline 
		0 & 0 & 0 & 0 & 0 & 0 & 0 & 0 & 0 & 0 & 1 & 1 & 1 & 1 & 1 & 1 & 1 & 1 \\  
		0 & 1 & 0 & 0 & 0 & 0 & 1 & 1 & 1 & 1 & 0 & 0 & 0 & 0 & 1 & 1 & 1 & 1 \\ 
		1 & 0 & 0 & 0 & 1 & 1 & 0 & 0 & 1 & 1 & 0 & 0 & 1 & 1 & 0 & 0 & 1 & 1 \\ 
		1 & 1 & 0 & 1 & 0 & 1 & 0 & 1 & 0 & 1 & 0 & 1 & 0 & 1 & 0 & 1 & 0 & 1 \\ 
	\end{tabular}
\end{table}

Ad alcune di queste corrispondono nomi speciali.
Vediamole nel dettaglio:

\subsubsection{Porta AND}
La porta AND, indicata in \color{green!50!black} $z^1$ \color{black}, corrisponde al $\wedge$ logico, ergo $z = 1 \Leftrightarrow x_0 = x_1 = 1$.
Si indica come:

\begin{center}
	\begin{circuitikz}
			\draw
			(0,0) node[and port] (mynot) {};
	\end{circuitikz}
\end{center}

e ha tabella di verità:
\begin{table}[H]
	\center
	\begin{tabular} { c  c | c }
		$x_1$ & $x_0$ & $z$ \\ 
		\hline 
		0 & 0 & 0 \\ 
		0 & 1 & 0 \\ 
		1 & 0 & 0 \\ 
		1 & 1 & 1 \\
	\end{tabular}
\end{table}

\subsubsection{Porta XOR}
La porta XOR, indicata in \color{red!50!black} $z^6$ \color{black}, corrisponde all'\textit{aut} logico, cioè esclusivo, ergo $z = 1 \Leftrightarrow x_0 \neq x_1$.
Si indica come:

\begin{center}
	\begin{circuitikz}
			\draw
			(0,0) node[xor port] (mynot) {};
	\end{circuitikz}
\end{center}

e ha tabella di verità:
\begin{table}[H]
	\center
	\begin{tabular} { c  c | c }
		$x_1$ & $x_0$ & $z$ \\ 
		\hline 
		0 & 0 & 0 \\ 
		0 & 1 & 1 \\ 
		1 & 0 & 1 \\ 
		1 & 1 & 0 \\
	\end{tabular}
\end{table}

\subsubsection{Porta OR}
La porta OR, indicata in \color{blue!50!black} $z^7$ \color{black}, corrisponde al $\lor$ logico, ergo $z = 0 \Leftrightarrow x_0 = x_1 = 0$.
Si indica come:

\begin{center}
	\begin{circuitikz}
			\draw
			(0,0) node[or port] (mynot) {};
	\end{circuitikz}
\end{center}

e ha tabella di verità:
\begin{table}[H]
	\center
	\begin{tabular} { c  c | c }
		$x_1$ & $x_0$ & $z$ \\ 
		\hline 
		0 & 0 & 0 \\ 
		0 & 1 & 1 \\ 
		1 & 0 & 1 \\ 
		1 & 1 & 1 \\
	\end{tabular}
\end{table}

\subsubsection{Porta NOR}
La porta NOR, indicata in \color{purple!50!black} $z^8$ \color{black}, corrisponde alla negazione dell'$\lor$ logico, ergo $z = 1 \Leftrightarrow x_0 = x_1 = 0$.
Si indica come:

\begin{center}
	\begin{circuitikz}
			\draw
			(0,0) node[nor port] (mynot) {};
	\end{circuitikz}
\end{center}

e ha tabella di verità:
\begin{table}[H]
	\center
	\begin{tabular} { c  c | c }
		$x_1$ & $x_0$ & $z$ \\ 
		\hline 
		0 & 0 & 1 \\ 
		0 & 1 & 0 \\ 
		1 & 0 & 0 \\ 
		1 & 1 & 0 \\
	\end{tabular}
\end{table}

\subsubsection{Porta XNOR}
La porta XNOR, indicata in \color{orange!50!black} $z^9$ \color{black}, corrisponde alla negazione dell'\textit{aut} logico, ergo $z = 1 \Leftrightarrow x_0 = x_1$.
Si indica come:

\begin{center}
	\begin{circuitikz}
			\draw
			(0,0) node[xnor port] (mynot) {};
	\end{circuitikz}
\end{center}

e ha tabella di verità:
\begin{table}[H]
	\center
	\begin{tabular} { c  c | c }
		$x_1$ & $x_0$ & $z$ \\ 
		\hline 
		0 & 0 & 1 \\ 
		0 & 1 & 0 \\ 
		1 & 0 & 0 \\ 
		1 & 1 & 1 \\
	\end{tabular}
\end{table}

\subsubsection{Porta NAND}
La porta NAND, indicata in \color{cyan!50!black} $z^14$ \color{black}, corrisponde alla negazione dell'$\wedge$ logico, ergo $z = 0 \Leftrightarrow x_0 = x_1 = 1$.
Si indica come:

\begin{center}
	\begin{circuitikz}
			\draw
			(0,0) node[nand port] (mynot) {};
	\end{circuitikz} 
\end{center}

e ha tabella di verità:
\begin{table}[H]
	\center
	\begin{tabular} { c  c | c }
		$x_1$ & $x_0$ & $z$ \\ 
		\hline 
		0 & 0 & 1 \\ 
		0 & 1 & 1 \\ 
		1 & 0 & 1 \\ 
		1 & 1 & 0 \\
	\end{tabular}
\end{table}

\par\smallskip
Si dovrebbe essere notato che un pallino finale indica negazione.
A volte si usa solo questa notazione, invece di tutta la porta NOT.

\subsubsection{Casi degeneri}
Alcuni casi speciali della tabella delle possibili reti a due porte sono degeneri: abbiamo due generatori di costante ($z^0$ e $z^{15}$), due elementi neutri, rispettivamente su $x_1$ e $x_0$ ($z^3$ e $z_5$), e due inversori sugli stessi ingressi ($z_{10}$ e $z_{12}$).

\subsection{AND e OR a più ingressi}
Posso pensare di estendere AND e OR ad $N$ ingressi:
\begin{itemize}
	\item \textbf{AND a $N$ ingressi:} l'uscita vale $1$ se tutti gli $N$ ingressi valgono $1$;
	\item \textbf{OR a $N$ ingressi:} l'uscita vale $1$ se almeno un'ingresso vale $1$;
\end{itemize}

Questo può essere realizzato concatenando più porte logiche dello stesso tipo, come segue:

\begin{center}
	\begin{circuitikz} 
		\node (short) at (-1.5, 1.28) {}; 
		\draw (0,0) node[and port] (myand2) {}
		(2,1) node[and port] (myand3) {}
		(short) -- (myand3.in 1)
		(myand2.out) -- (myand3.in 2);
	\end{circuitikz} 
\end{center}

La dimostrazione è semplice dalla tabella di verità, o dalle proprietà degli operatori logici.

Una nota va fatta sulle combinazioni di più di 3 ingressi, infatti una rete del genere è sconveniente:

\begin{center}
	\begin{circuitikz} 
		\node (short1) at (-3.5, 1.28) {}; 
		\node (short2) at (-3.5, 0.28) {}; 
		\draw (0,0) node[and port] (myand1) {}
		(-2,-1) node[and port] (myand2) {}
		(2,1) node[and port] (myand3) {}
		(short1) -- (myand3.in 1)
		(short2) -- (myand1.in 1)
		(myand2.out) -- (myand1.in 2)
		(myand1.out) -- (myand3.in 2);
	\end{circuitikz} 
\end{center}


in quanto il segnale deve attraversare al massimo 3 livelli di logica, mentre disponendo le porte come:

\begin{center}
	\begin{circuitikz} \draw
			(0,2) node[and port] (myand1) {}
			(0,0) node[and port] (myand2) {}
			(2,1) node[and port] (myand3) {}
			(myand1.out) -- (myand3.in 1)
			(myand2.out) -- (myand3.in 2);
	\end{circuitikz} 
\end{center}

il segnale dovrà attraversare al massimo 2 livelli di logica.

Conviene quindi disporre gli $N$ ingressi e le relative porte come un'albero binario bilanciato, in modo da minimizzare gli attraversamenti di livelli di logica.
Si noti che questo discorso vale per AND e OR: non per NAND, NOR, XOR o XNOR. 

Possiamo osservare velocemente cosa accade se si collegano queste porte fra di loro:
\begin{itemize}
	\item \textbf{NAND:} un singolo NAND può formare un NOT quando i suoi ingressi sono uniti insieme.
		Se si mettono 2 NAND in serie (a \textit{cascata}) in questo modo, si ottiene di nuovo un AND;
	\item \textbf{NOR:} un singolo NOR può formare un NAND nello stesso modo del NAND.
		Se si mettono 2 NOR a cascata, si ottiene di nuovo un NOR;
	\item \textbf{XOR:} con $\geq 2$ XOR, si crea effettivamente un controllore di parità, ergo una rete che si attiva quando un numero dispari dei suoi ingressi sono accesi;
	\item \textbf{XNOR:} con $\geq 2$ XNOR, si ha l'opposto che con gli XOR: si crea una rete che si attiva quando un numero pari dei suoi ingressi sono accesi.
\end{itemize}

Queste porte si indicano solitamente come con gli input su unica orizzontale, che risulta più compatto.

\subsection{Algebra di Boole}
L'algebra di Boole adopera gli operatori logici conosciuti, applicati ad elementi del campo binario $GF(2) = \{0 , 1\}$

Vediamo questi operatori:
\begin{itemize}
	\item \textbf{Complemento logico:} si indica come $\overline{x}$, oppure $!x$ o $/x$. 
		Si definisce come: $$ \overline{0} = 1, \quad \bar{1} = 0 $$
	\item\ \textbf{Somma logica:} si indica con $x + y$, e ha tabella di verità:
	\begin{table}[H]
		\center
		\begin{tabular} { c  c | c }
			$x$ & $y$ & $ x + y $ \\ 
			\hline 
			0 & 0 & 0 \\ 
			0 & 1 & 1 \\ 
			1 & 0 & 1 \\ 
			1 & 1 & 1 \\
		\end{tabular}
	\end{table}
		cioè equivale all'OR.
	\item \textbf{Prodotto logico:} si indica con $x \cdot y$, e ha tabella di verità:
	\begin{table}[H]
		\center
		\begin{tabular} { c  c | c }
			$x$ & $y$ & $ x \cdot y $ \\ 
			\hline 
			0 & 0 & 0 \\ 
			0 & 1 & 0 \\ 
			1 & 0 & 0 \\ 
			1 & 1 & 1 \\
		\end{tabular}
	\end{table}
		cioè equivale all'AND.

\end{itemize}

\par\smallskip

Su questi operatori valgono le proprietà:
\begin{enumerate}
	\item \textbf{Involutiva del complemento:} $\overline{\bar{x}} = x$;
	\item \textbf{Commutativa della somma e del prodotto:} $ x + y = y + x, \quad x \cdot y = y \cdot x$;
	\item \textbf{Associativa della somma:} $ x + y + z = (x + y) + z = x + (y + z)$;
	\item \textbf{Associativa del prodotto:} $ x \cdot y \cdot z = (x \cdot y) \cdot z = x \cdot (y \cdot z)$;
	\item \textbf{Distributiva della somma rispetto al prodotto:} $ x \cdot (y + z) = (x \cdot y) + (x \cdot z) $;
	\item \textbf{Distributiva del prodotto rispetto alla somma:} $ x + (y \cdot z) = (x + y) \cdot (x + z) $. Bisogna fare attenzione in quanto questa non vale in $\mathbb{R}$;
	\item \textbf{Complementazione:} $ x \cdot \overline{x} = 0, \quad x + \bar{x} = 1 $;
	\item \textbf{Unione e intersezione:} $ x + 0 = x, \quad x + 1 = 1 $, cioè $0$ è l'elemento neutro e $1$ l'elemento assorbente della somma (non lo è in $\mathbb{R}$); \\
																				$ x \cdot 0 = 0, \quad x \cdot 1 = x $, cioè $1$ è l'elemento neutro e $0$ l'elemento assorbente del prodotto;
	\item \textbf{Idempotenza:} $x + x = x$, \quad $x \cdot x = x$, altra che non vale in $\mathbb{R}$;
	\item \textbf{Leggi di De Morgan:} $\overline{x \cdot x} = \overline{x} + \bar{x}$ e $\overline{x + x} = \bar{x} \cdot \bar{x}$.
\end{enumerate}

\subsubsection{Teoremi di De Morgan}
Le leggi di De Morgan comuni della logica si estendono ad $N$ variabili come:
\begin{enumerate}
	\item $\overline{x_0 \cdot x_1 \cdot ... \cdot x_n} = \overline{x}_0 + \bar{x}_1 + ... + \bar{x}_n$
	\item $\overline{x_0 + x_1 + ... + x_{n}} = \overline{x}_0 \cdot \bar{x}_1 \cdot ... \cdot \bar{x}_n$
\end{enumerate}

\noindent
\textbf{\textsf{Dimostrazione per induzione}} \\
Richiamiamo le basi dell'induzione:
\begin{itemize}
	\item Si dimostra che una proprietà vale per un certo numero $n_0$ (passo base);
	\item Si dimostra che se vale per un certo $n \geq n_0$, allora vale anche per $n + 1$.
\end{itemize}

Partiamo con le dimostrazioni classiche ottenute con le tabelle di verità:

\begin{table}[H]
	\center
	\begin{tabular} { c  c | c | c | c | c | c }
		$x$ & $y$ & $ x \cdot y $ & $\overline{x \cdot y}$ & $\overline{x}$ & $\bar{y}$ & $\bar{x} + \bar{y}$ \\ 
		\hline 
		0 & 0 & 0 & 1 & 1 & 1 & 1 \\  
		0 & 1 & 0 & 1 & 1 & 0 & 1 \\ 
		1 & 0 & 0 & 1 & 0 & 1 & 1 \\ 
		1 & 1 & 1 & 0 & 0 & 1 & 0
	\end{tabular}
\end{table}

che ci portano a $n_0 = 2$.
Posso quindi porre l'ipotesi:

$$
\overline{x_0 \cdot ... \cdot x_{n-1}} = \overline{x}_0 + ... + \bar{x}_{n-1}
$$

e la tesi:

$$
\overline{x_0 \cdot ... \cdot x_{n-1} \cdot x_n} = \overline{x}_0 + ... + \bar{x}_{n-1} + \cdot x_n
$$

A questo punto faccio il passo induttivo, sfruttando l'associatività del prodotto (o della somma), e quindi riscrivendo la tesi come:

$$
\overline{\alpha \cdot x_n}, \quad \alpha = x_0 + ... x_{n-1} 
$$
dove notiamo la variabile introdotta $\alpha$, se complementata, rispetta:
$$
\overline{\alpha} = \overline{x_0 \cdot ... \cdot x_{n-1}} = \bar{x}_0 + ... + \bar{x}_{n-1}
$$
dall'ipotesi.

Possiamo quindi svolgere il passaggio:
$$
\overline{\alpha \cdot x_n} = \overline{\alpha} + \bar{x}_n =  \bar{x}_0 + ... + \bar{x}_{n-1} + \bar{x}_n
$$
che conferma la tesi.

\subsubsection{Algebra di Boole e reti combinatorie}
Esiste una corrispondenza fra l'algebra di Boole e le reti combinatorie.
In particolare, si ha che:
\begin{itemize}
	\item \textbf{Data una rete combinatoria}, (comunque complessa), è sempre possibile trovare un'espressione booleana che mette in relazione ogni sua uscita con gli ingressi (in verità un'espressione per ogni uscita);
	\item \textbf{Data un'espressione booleana}, p sempre possibile sintetizzare una rete combinatoria (ad un'uscita) in cui la relazione tra ingresso ed uscita data è dall'espressione.
\end{itemize}

Si noti che, effettivamente, espressioni logiche equivalenti $\Leftrightarrow$ reti logiche che svolgono lo stesso compito, ma non per questo l'equivalenza è totale: ci conviene creare reti che usano meno componenti possibili, in quanto queste le rende più affidabili, più economiche e meno dispendiose di energia.
Le proprietà dell'algebra di Boole possono quindi essere usate per ridurre il numero di porte logiche, attraverso un processo che chiameremo \textbf{minimizzazione}.

\end{document}


\documentclass[a4paper,11pt]{article}
\usepackage[a4paper, margin=8em]{geometry}

% usa i pacchetti per la scrittura in italiano
\usepackage[french,italian]{babel}
\usepackage[T1]{fontenc}
\usepackage[utf8]{inputenc}
\frenchspacing 

% usa i pacchetti per la formattazione matematica
\usepackage{amsmath, amssymb, amsthm, amsfonts}

% usa altri pacchetti
\usepackage{gensymb}
\usepackage{hyperref}
\usepackage{standalone}

\usepackage{colortbl}

% imposta il titolo
\title{Appunti Reti Logiche}
\author{Luca Seggiani}
\date{2024}

% imposta lo stile
% usa helvetica
\usepackage[scaled]{helvet}
% usa palatino
\usepackage{palatino}
% usa un font monospazio guardabile
\usepackage{lmodern}

\renewcommand{\rmdefault}{ppl}
\renewcommand{\sfdefault}{phv}
\renewcommand{\ttdefault}{lmtt}

% circuiti
\usepackage{circuitikz}
\usetikzlibrary{babel}

% disponi il titolo
\makeatletter
\renewcommand{\maketitle} {
	\begin{center} 
		\begin{minipage}[t]{.8\textwidth}
			\textsf{\huge\bfseries \@title} 
		\end{minipage}%
		\begin{minipage}[t]{.2\textwidth}
			\raggedleft \vspace{-1.65em}
			\textsf{\small \@author} \vfill
			\textsf{\small \@date}
		\end{minipage}
		\par
	\end{center}

	\thispagestyle{empty}
	\pagestyle{fancy}
}
\makeatother

% disponi teoremi
\usepackage{tcolorbox}
\newtcolorbox[auto counter, number within=section]{theorem}[2][]{%
	colback=blue!10, 
	colframe=blue!40!black, 
	sharp corners=northwest,
	fonttitle=\sffamily\bfseries, 
	title=Teorema~\thetcbcounter: #2, 
	#1
}

% disponi definizioni
\newtcolorbox[auto counter, number within=section]{definition}[2][]{%
	colback=red!10,
	colframe=red!40!black,
	sharp corners=northwest,
	fonttitle=\sffamily\bfseries,
	title=Definizione~\thetcbcounter: #2,
	#1
}

% disponi codice
\usepackage{listings}
\usepackage[table]{xcolor}

\definecolor{codegreen}{rgb}{0,0.6,0}
\definecolor{codegray}{rgb}{0.5,0.5,0.5}
\definecolor{codepurple}{rgb}{0.58,0,0.82}
\definecolor{backcolour}{rgb}{0.95,0.95,0.92}

\lstdefinestyle{codestyle}{
		backgroundcolor=\color{black!5}, 
		commentstyle=\color{codegreen},
		keywordstyle=\bfseries\color{magenta},
		numberstyle=\sffamily\tiny\color{black!60},
		stringstyle=\color{green!50!black},
		basicstyle=\ttfamily\footnotesize,
		breakatwhitespace=false,         
		breaklines=true,                 
		captionpos=b,                    
		keepspaces=true,                 
		numbers=left,                    
		numbersep=5pt,                  
		showspaces=false,                
		showstringspaces=false,
		showtabs=false,                  
		tabsize=2
}

\lstdefinestyle{shellstyle}{
		backgroundcolor=\color{black!5}, 
		basicstyle=\ttfamily\footnotesize\color{black}, 
		commentstyle=\color{black}, 
		keywordstyle=\color{black},
		numberstyle=\color{black!5},
		stringstyle=\color{black}, 
		showspaces=false,
		showstringspaces=false, 
		showtabs=false, 
		tabsize=2, 
		numbers=none, 
		breaklines=true
}


\lstdefinelanguage{assembler}{ 
  keywords={AAA, AAD, AAM, AAS, ADC, ADCB, ADCW, ADCL, ADD, ADDB, ADDW, ADDL, AND, ANDB, ANDW, ANDL,
        ARPL, BOUND, BSF, BSFL, BSFW, BSR, BSRL, BSRW, BSWAP, BT, BTC, BTCB, BTCW, BTCL, BTR, 
        BTRB, BTRW, BTRL, BTS, BTSB, BTSW, BTSL, CALL, CBW, CDQ, CLC, CLD, CLI, CLTS, CMC, CMP,
        CMPB, CMPW, CMPL, CMPS, CMPSB, CMPSD, CMPSW, CMPXCHG, CMPXCHGB, CMPXCHGW, CMPXCHGL,
        CMPXCHG8B, CPUID, CWDE, DAA, DAS, DEC, DECB, DECW, DECL, DIV, DIVB, DIVW, DIVL, ENTER,
        HLT, IDIV, IDIVB, IDIVW, IDIVL, IMUL, IMULB, IMULW, IMULL, IN, INB, INW, INL, INC, INCB,
        INCW, INCL, INS, INSB, INSD, INSW, INT, INT3, INTO, INVD, INVLPG, IRET, IRETD, JA, JAE,
        JB, JBE, JC, JCXZ, JE, JECXZ, JG, JGE, JL, JLE, JMP, JNA, JNAE, JNB, JNBE, JNC, JNE, JNG,
        JNGE, JNL, JNLE, JNO, JNP, JNS, JNZ, JO, JP, JPE, JPO, JS, JZ, LAHF, LAR, LCALL, LDS,
        LEA, LEAVE, LES, LFS, LGDT, LGS, LIDT, LMSW, LOCK, LODSB, LODSD, LODSW, LOOP, LOOPE,
        LOOPNE, LSL, LSS, LTR, MOV, MOVB, MOVW, MOVL, MOVSB, MOVSD, MOVSW, MOVSX, MOVSXB,
        MOVSXW, MOVSXL, MOVZX, MOVZXB, MOVZXW, MOVZXL, MUL, MULB, MULW, MULL, NEG, NEGB, NEGW,
        NEGL, NOP, NOT, NOTB, NOTW, NOTL, OR, ORB, ORW, ORL, OUT, OUTB, OUTW, OUTL, OUTSB, OUTSD,
        OUTSW, POP, POPL, POPW, POPB, POPA, POPAD, POPF, POPFD, PUSH, PUSHL, PUSHW, PUSHB, PUSHA, 
				PUSHAD, PUSHF, PUSHFD, RCL, RCLB, RCLW, MOVSL, MOVSB, MOVSW, STOSL, STOSB, STOSW, LODSB, LODSW,
				LODSL, INSB, INSW, INSL, OUTSB, OUTSL, OUTSW
        RCLL, RCR, RCRB, RCRW, RCRL, RDMSR, RDPMC, RDTSC, REP, REPE, REPNE, RET, ROL, ROLB, ROLW,
        ROLL, ROR, RORB, RORW, RORL, SAHF, SAL, SALB, SALW, SALL, SAR, SARB, SARW, SARL, SBB,
        SBBB, SBBW, SBBL, SCASB, SCASD, SCASW, SETA, SETAE, SETB, SETBE, SETC, SETE, SETG, SETGE,
        SETL, SETLE, SETNA, SETNAE, SETNB, SETNBE, SETNC, SETNE, SETNG, SETNGE, SETNL, SETNLE,
        SETNO, SETNP, SETNS, SETNZ, SETO, SETP, SETPE, SETPO, SETS, SETZ, SGDT, SHL, SHLB, SHLW,
        SHLL, SHLD, SHR, SHRB, SHRW, SHRL, SHRD, SIDT, SLDT, SMSW, STC, STD, STI, STOSB, STOSD,
        STOSW, STR, SUB, SUBB, SUBW, SUBL, TEST, TESTB, TESTW, TESTL, VERR, VERW, WAIT, WBINVD,
        XADD, XADDB, XADDW, XADDL, XCHG, XCHGB, XCHGW, XCHGL, XLAT, XLATB, XOR, XORB, XORW, XORL},
  keywordstyle=\color{blue}\bfseries,
  ndkeywordstyle=\color{darkgray}\bfseries,
  identifierstyle=\color{black},
  sensitive=false,
  comment=[l]{\#},
  morecomment=[s]{/*}{*/},
  commentstyle=\color{purple}\ttfamily,
  stringstyle=\color{red}\ttfamily,
  morestring=[b]',
  morestring=[b]"
}

\lstset{language=assembler, style=codestyle}

% disponi sezioni
\usepackage{titlesec}

\titleformat{\section}
	{\sffamily\Large\bfseries} 
	{\thesection}{1em}{} 
\titleformat{\subsection}
	{\sffamily\large\bfseries}   
	{\thesubsection}{1em}{} 
\titleformat{\subsubsection}
	{\sffamily\normalsize\bfseries} 
	{\thesubsubsection}{1em}{}

% tikz
\usepackage{tikz}

% float
\usepackage{float}

% grafici
\usepackage{pgfplots}
\pgfplotsset{width=10cm,compat=1.9}

% disponi alberi
\usepackage{forest}

\forestset{
	rectstyle/.style={
		for tree={rectangle,draw,font=\large\sffamily}
	},
	roundstyle/.style={
		for tree={circle,draw,font=\large}
	}
}

% disponi algoritmi
\usepackage{algorithm}
\usepackage{algorithmic}
\makeatletter
\renewcommand{\ALG@name}{Algoritmo}
\makeatother

% disponi numeri di pagina
\usepackage{fancyhdr}
\fancyhf{} 
\fancyfoot[L]{\sffamily{\thepage}}

\makeatletter
\fancyhead[L]{\raisebox{1ex}[0pt][0pt]{\sffamily{\@title \ \@date}}} 
\fancyhead[R]{\raisebox{1ex}[0pt][0pt]{\sffamily{\@author}}}
\makeatother

\begin{document}
% sezione (data)
\section{Lezione del 09-10-24}

% stili pagina
\thispagestyle{empty}
\pagestyle{fancy}

% testo
\subsection{Decoder}
Un decoder è una rete con $N$ ingressi e $p$ uscite con $p = 2^N$.
Si indica come:

\begin{center}
	\begin{circuitikz}
		\node[trapezium, trapezium angle=60, minimum height=1cm, minimum width=2cm, draw] (decoder) at (0,0) {};
		\node (xn) at (-0.5,1.5) {$x_{N-1}$};
		\node (x) at (0.5,1.5) {$x_0$};
		
		\draw (xn) -- (-0.5, 0.5);
		\draw (x) -- (0.5, 0.5);

		\node (zn) at (-1,-1.5) {$z_{p-1}$};
		\node (z) at (1,-1.5) {$z_0$};
		\node (j) at(0, -1.5) {$z_j$};

		\draw (zn) -- (-1, -0.5);
		\draw (z) -- (1, -0.5);
		\draw (j) -- (0, -0.5);

		\node at (0.11, 1.45) {$...$};
		\node at (-0.4, -1.55) {$...$};
		\node at (0.5, -1.55) {$...$};

	\end{circuitikz}
\end{center}

La sua legge di corrispondenza stabilisce che ogni uscita riconosce uno ed un solo stato di ingresso, in particolare l'uscita $j$-esima ($z_j$) riconosce lo stato di ingresso i cui bit sono la codifica di $j$ in base 2, cioè:
$$
(x_{n-1}, ..., x_0)_{2} = j
$$

Ad esempio, un decoder da 2 a 4 ha tabella di verità:

\begin{table}[h!]
	\center 
	\begin{tabular} { c c | c c c c }
		$x_1$ &$x_0$ &$z_0$ &$z_1$ &$z_2$ &$z_3$ \\
	\hline 
	0 & 0 & 1 & 0 & 0 & 0 \\
	0 & 1 & 0 & 1 & 0 & 0 \\
	1 & 0 & 0 & 0 & 1 & 0 \\
	1 & 1 & 0 & 0 & 0 & 1 \\
	\end{tabular}
\end{table}
che equivale alla codifica \textit{one-hot} del binario in ingresso (cioè ogni numero codificato da $n$ bit viene mandato al $j$-esimo di $p$ output che corrispondono uno ad uno ai numeri rappresentabili).

Vediamo di passare da questa descrizione ad una sintesi della rete. Abbiamo che:
\[
	\begin{cases}
			
z_3 = x_1 \cdot x_0 \\ 
z_2 = x_1 \cdot \overline{x}_0 \\ 
z_1 = \overline{x_1} \cdot x_0 \\ 
z_0 = \overline{x_1} \cdot \overline{x_0} \\ 
	\end{cases}
\]
cioè ogni "indice" del decoder corrisponde al prodotto dei due ingressi opportunamente negati: l'ultima uscità avra tutti i bit attivi (sarebbe $2^N -1$ considerando numeri naturali), ergo prende il prodotto di tutti gli ingressi.
Di contro, la prima uscita ($0$) avrà tutti i bit disattivi, quindi prenderà il prodotto di tutti gli ingressi negati.
Gli altri numeri vengono indirizzati prendendo il prodotto e complementando i bit che quel particolare numero si aspetterebbe come $0$.
Notiamo che, sebbene si abbiano 4 negazioni, nella rete fisica conviene negare gli input in entrata risparmiando 2 invertitori.

Per le figure, rimandiamo a \url{https://github.com/Guray00/IngegneriaInformatica/blob/master/SECONDO%20ANNO/I%20SEMESTRE/Reti%20Logiche/Diapositive%20OCR/Reti%20combinatorie%20ocr.pdf}.

Generalizziamo quindi questa struttura a decoder da $N$ a $2^N$, applicando quanto detto prima. Si avrà:
\[
	\begin{cases}
		z_0 = \overline{x_{N-1}} \cdot \overline{x_{N-2}} \cdot ... \cdot \overline{x_1} \cdot \overline{x_0}	\\
		z_1 = \overline{x_{N-1}} \cdot \overline{x_{N-2}} \cdot ... \cdot \overline{x_1} \cdot x_0	\\
		... \\ 
		z_{p-2} = x_{N-1} \cdot x_{N-2} \cdot ... \cdot x_1 \cdot \overline{x_0} \\
		z_{p-1} = x_{N-1} \cdot x_{N-2} \cdot ... \cdot x_1 \cdot x_0	\\
	\end{cases}
\]

Vediamo quindi le codifiche in Verilog di decoder a diversi valori di $N$.
Si definisce innanzitutto il caso banale di $N = 1$, che finora non è stato trattato.
Questo servirà a definire, in maniera gerarchica (ma come vedremo imperfetta), decoder più complessi:

\lstinputlisting[language=verilog, style=codestyle]{../verilog/10-09/decoders/b1to2_decoder.v}

Possiamo quindi definire il decoder da 2 a 4 visto prima:

\lstinputlisting[language=verilog, style=codestyle]{../verilog/10-09/decoders/b2to4_decoder.v}

un decoder da 3 a 8:

\lstinputlisting[language=verilog, style=codestyle]{../verilog/10-09/decoders/b3to8_decoder.v}

e infine, ad evidenziare quanto velocemente esplode il numero di termini (cioè esponenzialmente), un decoder da 4 a 16:

\lstinputlisting[language=verilog, style=codestyle]{../verilog/10-09/decoders/b4to16_decoder.v}

\subsubsection{Decoder con enabler}
Il problema dei decoder come appena descritti è che sono poco agili nell'espansione: non si possono costruire, come avevamo visto per i gli AND o gli OR, reti di più decoder combinati, a meno di non ridursi a ritrovare quelli che sono effettivamente i mintermini della tabella di verità (come si nota dagli esempi).
Introduciamo per questo motivo il decoder con \textbf{enabler}:

\begin{center}
	\begin{circuitikz}
		\node[trapezium, trapezium angle=60, minimum height=1cm, minimum width=2cm, draw] (decoder) at (0,0) {};
		\node (xn) at (-0.5,1.5) {$x_{N-1}$};
		\node (x) at (0.5,1.5) {$x_0$};
		
		\draw (xn) -- (-0.5, 0.5);
		\draw (x) -- (0.5, 0.5);

		\node (zn) at (-1,-1.5) {$z_{p-1}$};
		\node (z) at (1,-1.5) {$z_0$};
		\node (j) at(0, -1.5) {$z_j$};

		\draw (zn) -- (-1, -0.5);
		\draw (z) -- (1, -0.5);
		\draw (j) -- (0, -0.5);

		\node (e) at(-2, 0) {$e$};
		\draw (e) -- (-0.79,0);

		\node at (0.11, 1.45) {$...$};
		\node at (-0.4, -1.55) {$...$};
		\node at (0.5, -1.55) {$...$};

	\end{circuitikz}
\end{center}

Questi decoder hanno $N + 1$ ingressi, cioè quelli normali più l'enabler, che ha il compito di "accendere" il decoder stesso.
Fisicamente, potremmo semplicemente inserire il decoder $e$ come ingresso aggiuntivo agli AND già predisposti, per avere che:
\[
	z_i =
	\begin{cases}
			y_i \quad e = 1 \\
			0 \quad \ e = 0
	\end{cases}
\]
e quindi:
\[
	\begin{cases}
		z_0 = e \cdot \overline{x_{N-1}} \cdot \overline{x_{N-2}} \cdot ... \cdot \overline{x_1} \cdot \overline{x_0}	\\
		z_1 = e \cdot \overline{x_{N-1}} \cdot \overline{x_{N-2}} \cdot ... \cdot \overline{x_1} \cdot x_0	\\
		... \\ 
		z_{p-2} = e \cdot x_{N-1} \cdot x_{N-2} \cdot ... \cdot x_1 \cdot \overline{x_0}	\\
		z_{p-1} = e \cdot x_{N-1} \cdot x_{N-2} \cdot ... \cdot x_1 \cdot x_0	\\
	\end{cases}
\]

Adesso basta accorgersi che reti di decoder con $N > 2$ possono crearsi concatenando decoder a decoder, cioè usando un decoder con i bit più significativi in entrata per generare l'enabler di $N$ nuovi decoder, i quali ricevono i bit meno significativi in entrata. 

Ad esempio, se vogliamo creare un decoder \texttt{4to16} a partire da decoder \texttt{2to4}, useremo 4 decoder, con gli stessi input ($x_0$ e $x_1$), abilitati da un quinto decoder con input $x_2$ e $x_3$.

Vediamo un'esempio pratico, dato dalle implementazioni in Verilog degli stessi decoder visti prima, ma stavolta dotati di enabler (e posti in cascata, nelle sintesi gerarchiche, attraverso tali enabler).
Si inizia col decoder 1 a 2:

\lstinputlisting[language=verilog, style=codestyle]{../verilog/10-09/enb_decoders/b1to2_enb_decoder.v}

Possiamo quindi definire il decoder da 2 a 4:

\lstinputlisting[language=verilog, style=codestyle]{../verilog/10-09/enb_decoders/b2to4_enb_decoder.v}

un decoder da 3 a 8:

\lstinputlisting[language=verilog, style=codestyle]{../verilog/10-09/enb_decoders/b3to8_enb_decoder.v}

e infine, di cui notiamo la sintesi gerarchica molto più immediata rispetto al caso senza enabler, un decoder da 4 a 16:

\lstinputlisting[language=verilog, style=codestyle]{../verilog/10-09/enb_decoders/b4to16_enb_decoder.v}

\subsection{Demultiplexer}
Il demultiplexer è una rete con $N+1$ ingressi e $p = 2^N$ uscite:

\begin{center}
	\begin{circuitikz}
		\node[rectangle, minimum height=2cm, minimum width=2cm, draw] (multiplex) at (0,0) {};
		\node (x) at (0,1.5) {$x$};	
		\draw (x) -- (0, 1);

		\node (zn) at (-1,-1.5) {$z_{p-1}$};
		\node (z) at (1,-1.5) {$z_0$};
		\node (j) at(0, -1.5) {$z_j$};

		\draw (zn) -- (-1, -1);
		\draw (z) -- (1, -1);
		\draw (j) -- (0, -1);

		\node (bn) at(-2, 0.5) {$b_{N-1}$};
		\draw (bn) -- (-1,0.5);

		\node (b) at(-2, -0.5) {$b_0$};
		\draw (b) -- (-1,-0.5);
		
		\node at (-0.4, -1.55) {$...$};
		\node at (0.5, -1.55) {$...$};
		\node at (-2, 0) {$...$};

		\draw[dashed] (0, 1) -- (-1, -1);
		\draw[dashed] (0, 1) -- (1, -1);
	\end{circuitikz}
\end{center}

Chiamiamo $x$ la \textbf{variabile da commutare}, e le altre \textbf{variabili di comando} ($b$).
La $j$-esima uscita insegue la variabile da commutare se e solo se:
$$
(b_{n-1}, ..., b_0)_{2} = j
$$
altrimenti vale 0.
Questo significa che il demultiplexer invia il suo input, $x$, all'output $z_j$ tale che i controlli $b_{N-1} ... b_0$ sono la codifica binaria di $j$.

Il multiplexer, fisicamente, è identico ad un decoder con enabler: si fa la parte di decoding con il:
\[
	\begin{cases}
		z_0 = \overline{x_{N-1}} \cdot \overline{x_{N-2}} \cdot ... \cdot \overline{x_1} \cdot \overline{x_0}	\\
		z_1 = \overline{x_{N-1}} \cdot \overline{x_{N-2}} \cdot ... \cdot \overline{x_1} \cdot x_0	\\
		... \\ 
		z_{p-2} = x_{N-1} \cdot x_{N-2} \cdot ... \cdot x_1 \cdot \overline{x_0}	\\
		z_{p-1} = x_{N-1} \cdot x_{N-2} \cdot ... \cdot x_1 \cdot x_0	\\
	\end{cases}
\]
di prima, e si moltiplica per $x$ per ottenere il comportamento desiderato:
\[
	\begin{cases}
		z_0 = x \cdot \overline{x_{N-1}} \cdot \overline{x_{N-2}} \cdot ... \cdot \overline{x}_1 \cdot \overline{x_0}	\\
		z_1 = x \cdot \overline{x_{N-1}} \cdot \overline{x_{N-2}} \cdot ... \cdot \overline{x_1} \cdot x_0	\\
		... \\ 
		z_{p-2} = x \cdot x_{N-1} \cdot x_{N-2} \cdot ... \cdot x_1 \cdot \overline{x_0}	\\
		z_{p-1} = x \cdot x_{N-1} \cdot x_{N-2} \cdot ... \cdot x_1 \cdot x_0	\\
	\end{cases}
\]

Con $x = e$ questo è un decoder con enabler $x$.

Vediamo infatti l'implementazione in Verilog di un demultiplexer da 1 a 2:

\lstinputlisting[language=verilog, style=codestyle]{../verilog/10-09/demuxers/b1to2_demuxer.v}

e come se ne può ricavare uno da 1 a 4:

\lstinputlisting[language=verilog, style=codestyle]{../verilog/10-09/demuxers/b1to4_demuxer.v}

\subsection{Multiplexer}
Il multiplexer è il duale del demultiplexer: una rete con $N + 2^N$ ingressi e $1$ uscita:

\begin{center}
	\begin{circuitikz}
		\node[rectangle, minimum height=2cm, minimum width=2cm, draw] (multiplex) at (0,0) {};
		\node (x) at (0,-1.5) {$x$};	
		\draw (x) -- (0, -1);

		\node (zn) at (-1,1.5) {$z_{p-1}$};
		\node (z) at (1,1.5) {$z_0$};
		\node (j) at(0, 1.5) {$z_j$};

		\draw (zn) -- (-1, 1);
		\draw (z) -- (1, 1);
		\draw (j) -- (0, 1);

		\node (bn) at(-2, 0.5) {$b_{N-1}$};
		\draw (bn) -- (-1,0.5);

		\node (b) at(-2, -0.5) {$b_0$};
		\draw (b) -- (-1,-0.5);
		
		\node at (-0.4, 1.45) {$...$};
		\node at (0.5, 1.45) {$...$};
		\node at (-2, 0) {$...$};

		\draw[dashed] (-1, 1) -- (0, -1);
		\draw[dashed] (1, 1) -- (0, -1);
	\end{circuitikz}
\end{center}

Gli ingressi $b_i$ si chiamano variabili di comando, e selezionano l'ingresso connesso all'uscita come:
$$
z = x_i \Leftrightarrow (b_{N-1}, ..., b_1, b_0) = i
$$

Abbiamo detto che il multiplexer è il duale del demultiplexer: se quest'ultimo prendeva un segnale $x$ e lo inviava al $j$-esimo output sulla base della codifica di $j$ ottenuta alle variabili di controllo, il multiplexer prende il $j$-esimo ingresso, secondo gli stessi canoni, e lo invia alla linea $x$ di uscita.

Alla base della sintesi di un multiplexer sta un decoder: infatti, abbiamo che quest'ultimo seleziona uno solo (\textit{one-hot}) degli output, che possiamo moltiplicare (mettiamo una AND) per l'ingresso corrispondente.
Visto che solo uno degli output in uscita dagli AND è attivo in un dato momento, possiamo ricombinare il segnale finlle con un unico grande OR.

Come prima, possiamo eliminare gli AND in cascata dal decoder connettendoli agli AND già contenuti in esso.

Otteniamo quindi la descrizione algebrica (si noti che adesso abbiamo fatto sintesi $\rightarrow$ descrizione, mentre fino a questo punto avevamo fatto l'operazione inversa, descrizione $\rightarrow$ sintesi):

\[
	\begin{aligned}
		z = x_0 \cdot \overline{b_{N-1}} \cdot \overline{b_{N-2}} \cdot ... \cdot \overline{b_1} \cdot \overline{b_0}	+ \\
		x_1 \cdot \overline{b_{N-1}} \cdot \overline{b_{N-2}} \cdot ... \cdot \overline{b_1} \cdot b_0	+\\
		... + \\
		x_{p-2} \cdot b_{N-1} \cdot b_{N-2} \cdot ... \cdot b_1 \cdot \overline{b_0}	+\\
		x_{p-1} \cdot b_{N-1} \cdot b_{N-2} \cdot ... \cdot b_1 \cdot b_0	\\
	\end{aligned}
\]

Notiamo che il multiplexer è una rete a 2 livelli di logica: il segnale passerà al massimo da una AND e una OR.
Le NOT sugli ingressi non si contano, in quanto in una rete fisica le variabili di comando proverranno da registri, che forniscono già una versione negata del loro output senza bisogno di ulteriori inversori.

Vediamo quindi un'implementazione in Verilog di un multiplexer da 2 a 1:

\lstinputlisting[language=verilog, style=codestyle]{../verilog/10-09/muxers/b2to1_muxer.v}

e come se ne può ricavare uno da 4 a 1:

\lstinputlisting[language=verilog, style=codestyle]{../verilog/10-09/muxers/b4to1_muxer.v}

\subsubsection{Multiplexer come rete combinatoria universale}
Dimostriamo il seguente teorema:
\begin{theorem}{Multiplexer come rete combinatoria universale}	
Un multiplexer con $N$ variabili di comando è in grado di realizzare qualunque legge combinatoria ad $N$ ingressi ed un uscita, connettendo i $2^N$ ingressi a generatori di costante.
\end{theorem}

Abbiamo che:
\begin{itemize}
	\item Un multiplexer si ricava con porte AND, OR e NOT a due livelli di logica; 
	\item Un multiplexer realizza qualsiasi rete combinatoria ad un'uscita;
	\item una rete a più uscite può essere scomposta in più reti con le uscite messe "in parellelo".
\end{itemize}
Allora qualsiasi rete combinatoria può essere creata combinando AND, OR e NOT su due livelli di logica.

Inoltre, si può dimostrare che per qualsiasi tabella di verità ad $N$ ingressi, si può trovare una rete che la implementa tramite un multiplexer a $N-1$ variabili di comando, e al più porte NOT.

\subsection{Modello strutturale universale per reti combinatorie}
Vediamo adesso un modo per sintetizzare una rete logica ad $N$ ingressi ed $M$ uscite a partire da una tabella di verità.
Si prende prima di tutto un decoder con $N$ ingressi, e si creano $M$ linee parallele alle $2^N$ (che è anche il numero delle righe della tabella di verità) linee di uscita del decoder.
Si combinano quindi queste linee di uscita attraverso OR su ogni intersezione che corrisponde ad una certa cella della tabella di verità.

\subsubsection{Riduzione dei costi}
Definiamo informalmente il costo come ridotto quando si usano meno porte logiche.
Troviamo quindi un modo per ridurre il costo della rete creata.
Avremo che, inizialmente, tutte le uscite si presentano in una forma canonica \textbf{SP}, che sta per Somma di Prodotti, del tipo:
$$ 
z_j = x_{n-1} \cdot ... \cdot x_0 + ... + x_{n-1} \cdot ... \cdot x_0
$$
con la possibilità di complementare qualsiasi $x$. 
Questa forma equivale effettivamente a una forma normale disgiuntiva.

Possiamo quindi usare le proprietà dell'algebra di Boole per raggruppare e semplificare i termini.
Vogliamo un algoritmo che ci permetta di eseguire questi passaggi in modo ordinato, e ci porti sempre alla soluzione ottimale.

\end{document}

\end{document}