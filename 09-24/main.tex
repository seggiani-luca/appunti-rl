\documentclass[a4paper,11pt]{article}
\usepackage[a4paper, margin=8em]{geometry}

% usa i pacchetti per la scrittura in italiano
\usepackage[french,italian]{babel}
\usepackage[T1]{fontenc}
\usepackage[utf8]{inputenc}
\frenchspacing 

% usa i pacchetti per la formattazione matematica
\usepackage{amsmath, amssymb, amsthm, amsfonts}

% usa altri pacchetti
\usepackage{gensymb}
\usepackage{hyperref}
\usepackage{standalone}

% imposta il titolo
\title{Appunti Reti Logiche}
\author{Luca Seggiani}
\date{24-09-24}

% imposta lo stile
% usa helvetica
\usepackage[scaled]{helvet}
% usa palatino
\usepackage{palatino}
% usa un font monospazio guardabile
\usepackage{lmodern}

\renewcommand{\rmdefault}{ppl}
\renewcommand{\sfdefault}{phv}
\renewcommand{\ttdefault}{lmtt}

% disponi teoremi
\usepackage{tcolorbox}
\newtcolorbox[auto counter, number within=section]{theorem}[2][]{%
	colback=blue!10, 
	colframe=blue!40!black, 
	sharp corners=northwest,
	fonttitle=\sffamily\bfseries, 
	title=Teorema~\thetcbcounter: #2, 
	#1
}

% disponi definizioni
\newtcolorbox[auto counter, number within=section]{definition}[2][]{%
	colback=red!10,
	colframe=red!40!black,
	sharp corners=northwest,
	fonttitle=\sffamily\bfseries,
	title=Definizione~\thetcbcounter: #2,
	#1
}

% disponi codice
\usepackage{listings}
\usepackage[table]{xcolor}

\lstdefinestyle{codestyle}{
		backgroundcolor=\color{black!5}, 
		commentstyle=\color{codegreen},
		keywordstyle=\bfseries\color{magenta},
		numberstyle=\sffamily\tiny\color{black!60},
		stringstyle=\color{green!50!black},
		basicstyle=\ttfamily\footnotesize,
		breakatwhitespace=false,         
		breaklines=true,                 
		captionpos=b,                    
		keepspaces=true,                 
		numbers=left,                    
		numbersep=5pt,                  
		showspaces=false,                
		showstringspaces=false,
		showtabs=false,                  
		tabsize=2
}

\lstdefinestyle{shellstyle}{
		backgroundcolor=\color{black!5}, 
		basicstyle=\ttfamily\footnotesize\color{black}, 
		commentstyle=\color{black}, 
		keywordstyle=\color{black},
		numberstyle=\color{black!5},
		stringstyle=\color{black}, 
		showspaces=false,
		showstringspaces=false, 
		showtabs=false, 
		tabsize=2, 
		numbers=none, 
		breaklines=true
}

\lstdefinelanguage{javascript}{
	keywords={typeof, new, true, false, catch, function, return, null, catch, switch, var, if, in, while, do, else, case, break},
	keywordstyle=\color{blue}\bfseries,
	ndkeywords={class, export, boolean, throw, implements, import, this},
	ndkeywordstyle=\color{darkgray}\bfseries,
	identifierstyle=\color{black},
	sensitive=false,
	comment=[l]{//},
	morecomment=[s]{/*}{*/},
	commentstyle=\color{purple}\ttfamily,
	stringstyle=\color{red}\ttfamily,
	morestring=[b]',
	morestring=[b]"
}

% disponi sezioni
\usepackage{titlesec}

\titleformat{\section}
	{\sffamily\Large\bfseries} 
	{\thesection}{1em}{} 
\titleformat{\subsection}
	{\sffamily\large\bfseries}   
	{\thesubsection}{1em}{} 
\titleformat{\subsubsection}
	{\sffamily\normalsize\bfseries} 
	{\thesubsubsection}{1em}{}

% disponi alberi
\usepackage{forest}

\forestset{
	rectstyle/.style={
		for tree={rectangle,draw,font=\large\sffamily}
	},
	roundstyle/.style={
		for tree={circle,draw,font=\large}
	}
}

% tikz
\usepackage{tikz}

% float
\usepackage{float}

% grafici
\usepackage{pgfplots}
\pgfplotsset{width=10cm,compat=1.9}

% disponi algoritmi
\usepackage{algorithm}
\usepackage{algorithmic}
\makeatletter
\renewcommand{\ALG@name}{Algoritmo}
\makeatother

% disponi numeri di pagina
\usepackage{fancyhdr}
\fancyhf{} 
\fancyfoot[L]{\sffamily{\thepage}}

\makeatletter
\fancyhead[L]{\raisebox{1ex}[0pt][0pt]{\sffamily{\@title \ \@date}}} 
\fancyhead[R]{\raisebox{1ex}[0pt][0pt]{\sffamily{\@author}}}
\makeatother

\begin{document}
% sezione (data)
\section{Lezione del 24-09-24}

% stili pagina
\thispagestyle{empty}
\pagestyle{fancy}

% testo
\subsection{Introduzione}
Il corso di reti logiche tratta di:
\begin{enumerate}
	\item \textbf{Linguaggio assembler:} come scrivere programmi semplici, come avviene la compilazione in linguaggio macchina;
	\item \textbf{Reti logiche:} reti combinatorie, reti combinatorie per l'aritmetica, reti sequenziali asincrone e sincronizzate;
	\item \textbf{Microprogammazione:} reti sequenziali sincronizzate, come realizzare una rete logica da specifiche. 
		"Micro" qui sta per \textit{hardware};
	\item \textbf{Il calcolatore:} processore, interfacce comuni e convertitori.
\end{enumerate}

\subsubsection{Introduzione alle reti logiche}
Si parla di reti \textit{logiche} in quanto si guarda all'hardware da una prospettiva funzionale, indipendente dalla sua tecnologia.
Ad esempio, una porta NOR sarà implementata con determinati circuiti, ma tutto ciò che interessa a questo corso è come si comporta logicamente: $ y = 1 \Leftrightarrow A = B = 0 $.

\subsection{Programmazione assembly}
Il nome corretto del linguaggio sarebbe Assembly, ma noi lo chiameremo Assembler per ragioni storiche.
L'assembler è il linguaggio con cui si scrivono le istruzioni eseguite dal processore.
Il processore implementa effettivamente un ciclo fetch-execute dove preleva la prossima istruzione macchina (in assembler) dalla memoria e la esegue.

\subsubsection{Linguaggio macchina}
Il linguaggio macchina (LM) è dato dal contenuto effettivo della memoria che contiene le istruzioni, ergo una sequenza di zero e uno.
Il linguaggio assembler adotta una sintassi simbolica per il linguaggio macchina: ad esempio, \texttt{MOV \%AX, \%BX}.

Il processo di trasformazione dall'assembler all'LM si chiama \textbf{assemblaggio}, mentre il processo di traasformazione da un linguaggio ad alto livello all'assembler si chiama \textbf{compilazione}.

\subsubsection{Generalità sull'assembler}
Si dice che assembler è un linguaggio a basso livello.
Mancano i costrutti a cui siamo abituati da i linguaggi di alto livello:
\begin{enumerate}
	\item Non esistono costrutti di flow control (for, if-else, ecc...), tutto si fa con istruzioni di salto.
	\item Non esistono tipi variabile: gli operandi sono stringhe di bit che si riferiscono a locazioni in memoria.
\end{enumerate}

Inoltre, l'assembler è strettamente legato all'hardware, ed è specifico per ogni processore.
Noi vedremo l'assembler dei processori della famiglia Intel x86, che non è uguale all'assembler dei processori Arm Cortex, ecc...
Questo rende il codice in assembler mai portatile.
Fatta questa precisazione, possiamo dire che i principi generali restano comunque validi fra famiglie di processori diverse.

Esiste ancora oggi una nicchia di utilizzo del linguaggio assembler: quello dello sviluppo di sistemi embedded.
Inoltre, il linguaggio ha un importante significato didattico e culturale.

\subsection{Schema a blocchi del calcolatore}

\begin{center}

\begin{tikzpicture}[scale=2]

	% bus
	\draw[thick] (0,0) rectangle (5,1) node[midway] {\textsf{Rete di interconnessione (Bus)}};

	 % componenti
	\draw[thick] (0, -2) rectangle (1.5, -0.3) node[midway] {};
	
	\draw[thick] (0.1, -0.8) rectangle (1.4, -0.4) node[midway] {\textsf{Interfacce}};
	\draw[thick, rounded corners=10pt] (0.1, -1.4) rectangle (1.4, -1) node[midway] {\textsf{Dispositivi}};
	
	\node at (0.8, -1.7) {\parbox{3cm}{\textsf{Sottosistema \\ di ingresso/uscita}}};

	\draw[thick] (1.8, -2) rectangle (3.3, -0.3) node[midway] {\parbox{1.5cm}{\textsf{Memoria \\ Principale}}};
	\draw[thick] (3.6, -2) rectangle (5, -0.3) node[midway] {\parbox{1.75cm}{\textsf{Processore \\ ALU \\ FPU}}};

	% frecce
	\draw[->, thick] (0.75, 0) -- (0.75, -0.4) node[midway, above] {};
	
	\draw[->, thick] (0.75, -0.8) -- (0.75, -1) node[midway, above] {};
	
	\draw[->, thick] (2.55, 0) -- (2.55, -0.3) node[midway, above] {};
	\draw[->, thick] (4.3, 0) -- (4.3, -0.3) node[midway, above] {};

\end{tikzpicture}

\end{center}
\noindent
Un calcolatore è formato, in linea generale, da una rete di interconnessione (bus) che collega fra di loro:
\begin{itemize}
	\item Interfacce che comunicano con dispositivi;
	\item La memoria principale che contiene dati e programmi;
	\item Il processore, che esegue il ciclo fetch-execute. Possiamo aggiungere che ogni processore, oggi, contiene almeno due blocchi:
		\begin{itemize}
			\item L'\textbf{ALU}, Arithmetic Logic Unit, che si occupa di calcoli aritmetici su numeri interi (interpretando le stringhe di bit come numeri naturali o interi in complemento a 2) e operazioni logiche;
			\item L'\textbf{FPU}, Floating Point Unit, che si occupa dei numeri a virgola mobile.
		\end{itemize}
\end{itemize}

\subsection{Riassunto di rappresentazione dell'informazione}

Da qui in poi $x$ è il numero rappresentato e $X$ la sequenza di bit rappresentante.

\subsubsection{Numeri naturali}

\textbf{\textsf{Intervallo di rappresentabilità}} \\
$n$ bit rappresentano $2^n$ naturali sull'intervallo $[0, 2^n- 1]$.

\par\medskip
\noindent
\textbf{\textsf{Trasformazione diretta}} \\
Per portare un'intero in rappresentazione binaria nel suo corrispondente in base 10, si sa che presi $n$ bit $b_{n-1}, b_{n-2}, ... , b_1, b_0$ della rappresentazione $X$, essi rappresentano il naturale $x$:
$$
x =  b_{n-1} \cdot 2^{n-1} + b_{n-2} \cdot 2^{n-2} + ... + b_1 \cdot 2 + b_0 = \sum_{i=0}^{n-1} b_n \cdot 2^i
$$

Il bit più a sinistra è il Most Significant Bit (MSB), cioé $b_{n-1}$, quello più a destra il Least Significant Bit (LSD) cioè $b_0$.

\par\medskip
\noindent
\textbf{\textsf{Trasformazione inversa}} \\
Per portare un'intero in base 10 nella sua rappresentazione binaria, si usa l'algoritmo DIV-MOD:

\begin{algorithm}
\caption{DIV-MOD}
\begin{algorithmic} % The number indicates where to start line numbering
    \STATE \textbf{Input:} $x$ in base 10
    \STATE \textbf{Output:} $X$ rappresentazione in base 2
    
    % Body
    \STATE Inizializza $q \gets x$, $r \gets 0$, and $i \gets 0$ % Initialize variables
    \STATE Crea un'array vuota $R$ per i resti 

    \WHILE{$q \neq 0$}
        \STATE $r \gets q \mod 2$ % Calculate remainder
        \STATE Metti $r$ in $R[i]$ % Store the remainder
        \STATE $q \gets q / 2$ % Update the quotient
        \STATE $i \gets i + 1$ % Increment index
    \ENDWHILE
    
    \STATE Gli $R[n-1], R[n-2], ..., R[0]$ rimasti (letti al contrario) sono le cifre di $X$.
\end{algorithmic}
\end{algorithm}

\subsubsection{Numeri interi in complemento a due}

\textbf{\textsf{Intervallo di rappresentabilità}} \\
$n$ bit rappresentano $2^n$ interi sull'intervallo $ [-2^{n-1}, 2^{n-1} - 1]$.

\par\medskip
\noindent
\textbf{\textsf{Trasformazione diretta}} \\
Per portare un intero $x$ in base 10 nella sua rappresentazione in complemento a due $X$ su $n$ bit, si decide alternativamente rispetto al segno di $x$ di rappresentare il naturale $N$ in $X$:

\[
	N =
	\begin{cases}
		x \hspace{1.6cm} x \geq 0 \\ 
		2^{n} + x \hspace {0.75cm} x < 0 
	\end{cases}, \quad 
	X = N_2
\]

dove si nota che nella seconda espressione $2^n + x$ equivale a $2^n - |x|$, dalla negatività di $x$.

\par\smallskip

Alternativamente, sui soli numeri negativi:
\begin{itemize}
	\item Si converte $x$ in rappresentazione binaria.
	\item Si trova il complemento, ovvero la rappresentazione che inverte tutti i bit (che equivale alla rappresentazione in complemento a 2 dell'opposto $- 1$). 
	\item A questo punto si aggiunge 1, ignorando qualsiasi overflow.
\end{itemize}
La rappresentazione $X$ trovata è il complemento a 2 di $x$.
Simbolicamente:
\[
	X =
	\begin{cases}
		x_2 \hspace{1.6cm} x \geq 0 \\ 
		(\bar{x} + 1)_2 \hspace {0.95cm} x < 0 
	\end{cases}
\]

\par\medskip
\noindent
\textbf{\textsf{Trasformazione inversa}}

Per portare la rappresentazione in complemento a due $X$ su $n$ bit di un intero $x$ all'intero stesso, ci si comporta come per le rappresentazioni di naturali, ma prendendo il bit più significativo dagli $n$ bit $b_{n-1}, b_{n-2}, ... , b_1, b_0$ della rappresentazione $X$ con valenza negativa:
$$
x = -b_{n-1} \cdot 2^{n-1} + b_{n-2} \cdot 2^{n-2} + ... + b_1 \cdot 2 + b_0 = -b_{n-1} \cdot 2^{n-1} + \sum_{i=0}^{n-2} b_n \cdot 2^i
$$


\par\smallskip

Alternativamente, si nota che il bit più significativo della rappresentazione sarà impostato a 0 per numeri positivi e 1 per numeri negativi.
Ciò significa che avremo:
$$
x =
	\begin{cases}
		X_{10} \hspace{1.67cm} X_{n-1} = 0 \\
		-(\bar{X} + 1)_{10} \quad X_{n-1} = 1
	\end{cases}
$$

dove la barra rappresenta l'operazione complemento.

\subsubsection{Rappresentazioni di interi e naturali, diagramma a farfalla}
La rappresentazione in complemento 2 su $n$ bit è effettivamente una funzione dal dominio $[-2^{n_1}, 2^{n-1} - 1]$ degli interi al codominio $[0, 2^{n} -1]$ dei naturali.
Tale funzione prende il nome di \textit{diagramma a farfalla}:

\begin{center}
	\begin{tikzpicture} [scale=0.9]
    \begin{axis}[
        axis lines=middle,
        xlabel={Intero},
        ylabel={Naturale},
				xtick={-0.5,0.5},
				ytick={0,0.5,1},
				xticklabels={$-2^{n-1}$, $2^{n-1} - 1$},
				yticklabels={0, $2^{n-1}$, $2^{n} - 1$},
				axis line style = {-}, % Use this line to remove arrows
				] 


		\addplot[domain=-0.5:0, black, thick] {x+1};
		\addplot[domain=0:0.5, black, thick] {x};

    \end{axis}
\end{tikzpicture}
\end{center}

da cui notiamo la relazione fra un'intero e il naturale che lo rappresenta in complemento a 2.

\subsubsection{Valori notevoli del complemento a 2}

Vale la pena notare alcuni valori notevoli del complemento a 2 su $n$ bit.
\begin{itemize}
	\item Innanzitutto, 0 rimane 0, ergo una fila di di $n$ zeri.
	\item Uno zero seguito da $n-1$ uni è il numero più positivo positivo, ergo $2^{n-1} -1$.
	\item Aggiungendo uno, si arriva ad un uno seguito da $n - 1$ zeri, che è il numero negativo possibile, ergo $-2^{n-1}$. 
		Notare che questo combacia col prendere il numero più positivo $2^{n-1} -1$, e ricavare uno meno del suo opposto $-2^{n-1}$, che abbiamo appurato essere ciò che accade quando si complementa (e infatti i due numeri sono l'uno il complemento dell'altro).
	\item Infine, una sequenza di $n$ uno rappresenta il più piccolo numero negativo, ergo $-1$.
\end{itemize}

Si nota che, al pari dei naturali, la rappresentazione dei numeri interi in complemento a 2 è effettivamente ciclica.

\subsubsection{Notazione esadecimale}
Scrivere lunghe stringhe binarie diventa velocemente complicato. 
Per questo si adotta una notazione esadecimale per stringhe di 4 bit ($[0, 15]$):

\begin{table}[h!]
    \centering
    \rowcolors{2}{white}{black!10}
    \begin{tabular}{ c | c | c }
        \bfseries Decimale & \bfseries Binario & \bfseries Esadecimale \\
        \hline
        0  & 0000 & \texttt{0x0} \\
        1  & 0001 & \texttt{0x1} \\
        2  & 0010 & \texttt{0x2} \\
        3  & 0011 & \texttt{0x3} \\
        4  & 0100 & \texttt{0x4} \\
        5  & 0101 & \texttt{0x5} \\
        6  & 0110 & \texttt{0x6} \\
        7  & 0111 & \texttt{0x7} \\
        8  & 1000 & \texttt{0x8} \\
        9  & 1001 & \texttt{0x9} \\
        10 & 1010 & \texttt{0xA} \\
        11 & 1011 & \texttt{0xB} \\
        12 & 1100 & \texttt{0xC} \\
        13 & 1101 & \texttt{0xD} \\
        14 & 1110 & \texttt{0xE} \\
        15 & 1111 & \texttt{0xF} \\
    \end{tabular}
\end{table}

A questo punto, possiamo denotare qualsiasi stringa binaria come una lista di numeri esadecimali prefissi da \texttt{0x} (che serve ad indicare la rappresentazione esadecimale stessa), ad esempio \texttt{0xC1} (11000001).

\subsubsection{Nota sulle potenze di 2}
Conviene ricordare le prime potenze di 2:

\begin{table}[H]
    \centering
    \rowcolors{2}{white}{black!10}
    \begin{tabular}{ c | c  }
        \bfseries Esponente & \bfseries Valore \\
        \hline
				0 & 1 \\
				1 & 2 \\
				2 & 4 \\
				3 & 8 \\
				4 & 16 \\
				5 & 32 \\
				6 & 64 \\
				7 & 128 \\
				8 & 256 \\
				9 & 512 \\
				10 & 1024 $\approx$ 1000 \\
				11 & 2048 \\
				12 & 4096 \\
				13 & 8192 \\
    \end{tabular}
\end{table}

e inoltre ricordare che, visto $2^10 = 1024 \approx 1000$, le unità di misura usuali diventano:

\begin{table}[H]
    \centering
    \rowcolors{2}{white}{black!10}
    \begin{tabular}{ c | c }
        \bfseries Unità & \bfseries Potenza \\
        \hline
				$2^{10}$ & 1 KB \\
				$2^{20}$ & 1 MB \\
				$2^{30}$ & 1 GB \\
    \end{tabular}
\end{table}

e cosi via.

\subsection{Struttura del calcolatore}
\subsubsection{Spazio di memoria}
La memoria del calcolatore, vista dal programmatore assembler, è uno spazio lineare di $2^{32}$ (su calcolatori a 32 bit) locazioni (celle) di memoria, dalla capacità di un byte ciascuna.
Ogni cella è quindi identificata da un numero di 32 bit, detto \textbf{indirizzo}.

\par\smallskip

Lo spazio di memoria è in larga parte implementato attraverso Random Access Memory (RAM), ovvero memoria volatile.
Solo una piccola parte dello spazio è implementata attraverso Read Only Memory (ROM), ovvero memoria permanente, che contiene le istruzioni da eseguire al reset.

\par\medskip
\noindent
\textbf{\textsf{Accesso allo spazio di memoria}} \\
Il processore può accedere (leggere/scrivere) a:
\begin{itemize}
	\item Singole locazioni (byte) da 8 bit;
	\item Doppie locazioni (word) da 16 bit;
	\item Quadruple locazioni (double word) da 32 bit.
\end{itemize}

Per gli accessi 16/32 bit si usa l'indirizzo più piccolo delle 2/4 locazioni.
Si ricorda che l'indirizzo più grande contiene i bit più significativi.

Gli indirizzi di memoria assembler sono solo simbolici, e vengono tradotti dall'assemblatore, e in parte runtime.
Questo significa che non si può accedere a memoria appartenente al sistema operativo, o memoria fuori dai limiti fisici del sistema, ecc...

\subsubsection{Spazio di Input/Output}
Lo spazio di Input/Output è formato da $2^{16}$, ovvero 64k, locazioni o \textbf{porte}.
Ogni porta ha una capacità di un byte ed è indirizzata da un numero di 16 bit.

Il processore accede alle porte attraverso operazioni particolari di lettura o scrittura (in o out).
Spesso le porte sono configurate per un solo tipo di operazione: sola lettura o sola scrittura.

\par\smallskip

Le locazioni di memoria sono solitamente identiche fra di loro, le porte di I/O no.
Indirizzi diversi significano dispositivi diversi, e si rende quindi necessario conoscere fisicamente gli indirizzi.

\subsubsection{Processore}
Il processore è dotato di una memoria interna formata da locazioni di memoria da 32 bit (\textbf{registri}).
Questi si dividono in registri \textbf{generali}, riservati alle elaborazioni, e \textbf{di stato}, riservati a compiti speciali.

\par\medskip
\noindent
\textbf{\textsf{Registri generali}} \\
I registri iniziano generalmente con la lettera \textbf{E}, che sta per \textit{Extended}.
Questo perché storicamente i registri erano da 16 bit, e successivamente sono stati estesi a 32 bit.
Possiamo quindi riferirci a più sezioni dello stesso registro:
\begin{itemize}
	\item \textbf{EAX:} tutti i 32 bit del registro esteso;
	\item \textbf{AL:} la parte \textit{bassa} del registro \textbf{AX}, ergo quella meno significativa, da 8 bit;
	\item \textbf{AH:} la parte \textit{alta} del registro \textbf{AX}, ergo quella più significativa, da 8 bit;
	\item \textbf{AX:} il registro AX legacy, che combina \textbf{AL} e \textbf{AH}, da 16 bit.
 \end{itemize}

Alcuni registri vengono storicamente utilizzati per particolari funzioni:
\begin{itemize}
	\item \textbf{EAX} è utilizzato da alcune istruzioni aritmetiche per contenere operandi e risultati. Viene detto \textbf{accumulatore}.
	\item \textbf{ESI}, \textbf{EDI}, \textbf{EBX}, \textbf{EBP} vengono detti registri puntatore, dove \textbf{B} sta per base e \textbf{I} per indice. In particolare:
		\begin{itemize}
			\item \textbf{ESI}, \textbf{EDI} vengono utilizzati come registri indice per accessi in memoria.
			\item \textbf{EBX} è utilizzato come indirizzo di base per l'accesso in memoria. Viene solitamente detto \textbf{base}.
			\item \textbf{EBP} è utlizzato sempre come indirizzo di base per l'accesso in memoria.
		\end{itemize}
	\item \textbf{ECX} è utilizzato come contatore nei cicli. Viene detto \textbf{contatore}.
	\item \textbf{EDX} è utilizzato come operando di operazioni aritmetiche. Viene detto \textbf{data}.
	\item \textbf{ESP} è utilizzato per indirizzare la \textbf{pila} o \textbf{stack}, ovvero una parte di memoria con disciplina LIFO che serve a gestire sottoprogrammi.
\end{itemize}

\par\medskip
\noindent
\textbf{\textsf{Registri di stato}} \\
Ricordiamo due registri di stato:
\begin{itemize}
	\item  L'\textbf{EIP} viene detto \textbf{instruction pointer}, o \textbf{program counter}.
Viene usato per contenere l'indirizzo della locazione dalla quale sarà prelevata la prossima istruzione da eseguire.
Il contenuto dell'EIP è fissato al reset iniziale, e impostato sulla prima istruzione da eseguire (in memoria ROM) all'indirizzo \texttt{0xFFFF0000}.
Un po' di celle in memoria centrale da questo indirizzo in poi sono implementate in ROM.

Possiamo quindi dire che il ciclo fetch-loop si svolge come segue:
\begin{itemize}
	\item Il processore preleva dalla memoria, all'indirizzo EIP, una nuova istruzione;
	\item Incrementa EIP del numero di byte dell'istruzione prelevata;
	\item Esegue l'istruzione e ripete.
\end{itemize}

Da questo si ha che le istruzioni in memoria vengono eseguite sequenzialmente nell'ordine in cui sono incontrate, a meno che non si definiscano salti attraverso altre determinate istruzioni.
	
\item L'\textbf{EF} viene detto \textbf{extended flag}.
Consiste di 32 elementi detti \textbf{flag}, fra cui ricordiamo:
\begin{itemize}
	\item \textbf{OF:} flag di overflow (traboccamento) delle operazioni aritmetiche;
	\item \textbf{SF:} flag di segno, impostato quando l'ultima operazione restituisce un complemento a 2 con $MSB = 1$ (ergo negativo);
	\item \textbf{ZF:} flag zero, che viene impostato quando l'ultima operazione restituisce qualcosa di nullo;
	\item \textbf{CF:} flag di carry (riporto), che viene impostato quando l'ultima operazione richiede un riporto o un prestito.
\end{itemize}

\end{itemize}

I flag \textbf{OF} e \textbf{SF} sono significativi per operazioni su interi.
Il flag \textbf{CF} è significativo per operazioni su naturali.
Il flag \textbf{ZF} è significativo per entrambi i tipi di operazione.

Al reset i flag visti finora sono impostati a 0.

\end{document}


