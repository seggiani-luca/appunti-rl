
\documentclass[a4paper,11pt]{article}
\usepackage[a4paper, margin=8em]{geometry}

% usa i pacchetti per la scrittura in italiano
\usepackage[french,italian]{babel}
\usepackage[T1]{fontenc}
\usepackage[utf8]{inputenc}
\frenchspacing 

% usa i pacchetti per la formattazione matematica
\usepackage{amsmath, amssymb, amsthm, amsfonts}

% usa altri pacchetti
\usepackage{gensymb}
\usepackage{hyperref}
\usepackage{standalone}

\usepackage{colortbl}

\usepackage{xstring}
\usepackage{karnaugh-map}

% imposta il titolo
\title{Appunti Reti Logiche}
\author{Luca Seggiani}
\date{2024}

% imposta lo stile
% usa helvetica
\usepackage[scaled]{helvet}
% usa palatino
\usepackage{palatino}
% usa un font monospazio guardabile
\usepackage{lmodern}

\renewcommand{\rmdefault}{ppl}
\renewcommand{\sfdefault}{phv}
\renewcommand{\ttdefault}{lmtt}

% circuiti
\usepackage{circuitikz}
\usetikzlibrary{babel}

% disponi il titolo
\makeatletter
\renewcommand{\maketitle} {
	\begin{center} 
		\begin{minipage}[t]{.8\textwidth}
			\textsf{\huge\bfseries \@title} 
		\end{minipage}%
		\begin{minipage}[t]{.2\textwidth}
			\raggedleft \vspace{-1.65em}
			\textsf{\small \@author} \vfill
			\textsf{\small \@date}
		\end{minipage}
		\par
	\end{center}

	\thispagestyle{empty}
	\pagestyle{fancy}
}
\makeatother

% disponi teoremi
\usepackage{tcolorbox}
\newtcolorbox[auto counter, number within=section]{theorem}[2][]{%
	colback=blue!10, 
	colframe=blue!40!black, 
	sharp corners=northwest,
	fonttitle=\sffamily\bfseries, 
	title=Teorema~\thetcbcounter: #2, 
	#1
}

% disponi definizioni
\newtcolorbox[auto counter, number within=section]{definition}[2][]{%
	colback=red!10,
	colframe=red!40!black,
	sharp corners=northwest,
	fonttitle=\sffamily\bfseries,
	title=Definizione~\thetcbcounter: #2,
	#1
}

% disponi codice
\usepackage{listings}
\usepackage[table]{xcolor}

\definecolor{codegreen}{rgb}{0,0.6,0}
\definecolor{codegray}{rgb}{0.5,0.5,0.5}
\definecolor{codepurple}{rgb}{0.58,0,0.82}
\definecolor{backcolour}{rgb}{0.95,0.95,0.92}

\lstdefinestyle{codestyle}{
		backgroundcolor=\color{black!5}, 
		commentstyle=\color{codegreen},
		keywordstyle=\bfseries\color{magenta},
		numberstyle=\sffamily\tiny\color{black!60},
		stringstyle=\color{green!50!black},
		basicstyle=\ttfamily\footnotesize,
		breakatwhitespace=false,         
		breaklines=true,                 
		captionpos=b,                    
		keepspaces=true,                 
		numbers=left,                    
		numbersep=5pt,                  
		showspaces=false,                
		showstringspaces=false,
		showtabs=false,                  
		tabsize=2
}

\lstdefinestyle{shellstyle}{
		backgroundcolor=\color{black!5}, 
		basicstyle=\ttfamily\footnotesize\color{black}, 
		commentstyle=\color{black}, 
		keywordstyle=\color{black},
		numberstyle=\color{black!5},
		stringstyle=\color{black}, 
		showspaces=false,
		showstringspaces=false, 
		showtabs=false, 
		tabsize=2, 
		numbers=none, 
		breaklines=true
}


\lstdefinelanguage{assembler}{ 
  keywords={AAA, AAD, AAM, AAS, ADC, ADCB, ADCW, ADCL, ADD, ADDB, ADDW, ADDL, AND, ANDB, ANDW, ANDL,
        ARPL, BOUND, BSF, BSFL, BSFW, BSR, BSRL, BSRW, BSWAP, BT, BTC, BTCB, BTCW, BTCL, BTR, 
        BTRB, BTRW, BTRL, BTS, BTSB, BTSW, BTSL, CALL, CBW, CDQ, CLC, CLD, CLI, CLTS, CMC, CMP,
        CMPB, CMPW, CMPL, CMPS, CMPSB, CMPSD, CMPSW, CMPXCHG, CMPXCHGB, CMPXCHGW, CMPXCHGL,
        CMPXCHG8B, CPUID, CWDE, DAA, DAS, DEC, DECB, DECW, DECL, DIV, DIVB, DIVW, DIVL, ENTER,
        HLT, IDIV, IDIVB, IDIVW, IDIVL, IMUL, IMULB, IMULW, IMULL, IN, INB, INW, INL, INC, INCB,
        INCW, INCL, INS, INSB, INSD, INSW, INT, INT3, INTO, INVD, INVLPG, IRET, IRETD, JA, JAE,
        JB, JBE, JC, JCXZ, JE, JECXZ, JG, JGE, JL, JLE, JMP, JNA, JNAE, JNB, JNBE, JNC, JNE, JNG,
        JNGE, JNL, JNLE, JNO, JNP, JNS, JNZ, JO, JP, JPE, JPO, JS, JZ, LAHF, LAR, LCALL, LDS,
        LEA, LEAVE, LES, LFS, LGDT, LGS, LIDT, LMSW, LOCK, LODSB, LODSD, LODSW, LOOP, LOOPE,
        LOOPNE, LSL, LSS, LTR, MOV, MOVB, MOVW, MOVL, MOVSB, MOVSD, MOVSW, MOVSX, MOVSXB,
        MOVSXW, MOVSXL, MOVZX, MOVZXB, MOVZXW, MOVZXL, MUL, MULB, MULW, MULL, NEG, NEGB, NEGW,
        NEGL, NOP, NOT, NOTB, NOTW, NOTL, OR, ORB, ORW, ORL, OUT, OUTB, OUTW, OUTL, OUTSB, OUTSD,
        OUTSW, POP, POPL, POPW, POPB, POPA, POPAD, POPF, POPFD, PUSH, PUSHL, PUSHW, PUSHB, PUSHA, 
				PUSHAD, PUSHF, PUSHFD, RCL, RCLB, RCLW, MOVSL, MOVSB, MOVSW, STOSL, STOSB, STOSW, LODSB, LODSW,
				LODSL, INSB, INSW, INSL, OUTSB, OUTSL, OUTSW
        RCLL, RCR, RCRB, RCRW, RCRL, RDMSR, RDPMC, RDTSC, REP, REPE, REPNE, RET, ROL, ROLB, ROLW,
        ROLL, ROR, RORB, RORW, RORL, SAHF, SAL, SALB, SALW, SALL, SAR, SARB, SARW, SARL, SBB,
        SBBB, SBBW, SBBL, SCASB, SCASD, SCASW, SETA, SETAE, SETB, SETBE, SETC, SETE, SETG, SETGE,
        SETL, SETLE, SETNA, SETNAE, SETNB, SETNBE, SETNC, SETNE, SETNG, SETNGE, SETNL, SETNLE,
        SETNO, SETNP, SETNS, SETNZ, SETO, SETP, SETPE, SETPO, SETS, SETZ, SGDT, SHL, SHLB, SHLW,
        SHLL, SHLD, SHR, SHRB, SHRW, SHRL, SHRD, SIDT, SLDT, SMSW, STC, STD, STI, STOSB, STOSD,
        STOSW, STR, SUB, SUBB, SUBW, SUBL, TEST, TESTB, TESTW, TESTL, VERR, VERW, WAIT, WBINVD,
        XADD, XADDB, XADDW, XADDL, XCHG, XCHGB, XCHGW, XCHGL, XLAT, XLATB, XOR, XORB, XORW, XORL},
  keywordstyle=\color{blue}\bfseries,
  ndkeywordstyle=\color{darkgray}\bfseries,
  identifierstyle=\color{black},
  sensitive=false,
  comment=[l]{\#},
  morecomment=[s]{/*}{*/},
  commentstyle=\color{purple}\ttfamily,
  stringstyle=\color{red}\ttfamily,
  morestring=[b]',
  morestring=[b]"
}

\lstset{language=assembler, style=codestyle}

% disponi sezioni
\usepackage{titlesec}

\titleformat{\section}
	{\sffamily\Large\bfseries} 
	{\thesection}{1em}{} 
\titleformat{\subsection}
	{\sffamily\large\bfseries}   
	{\thesubsection}{1em}{} 
\titleformat{\subsubsection}
	{\sffamily\normalsize\bfseries} 
	{\thesubsubsection}{1em}{}

% tikz
\usepackage{tikz}

% float
\usepackage{float}

% grafici
\usepackage{pgfplots}
\pgfplotsset{width=10cm,compat=1.9}

% disponi alberi
\usepackage{forest}

\forestset{
	rectstyle/.style={
		for tree={rectangle,draw,font=\large\sffamily}
	},
	roundstyle/.style={
		for tree={circle,draw,font=\large}
	}
}

% disponi algoritmi
\usepackage{algorithm}
\usepackage{algorithmic}
\makeatletter
\renewcommand{\ALG@name}{Algoritmo}
\makeatother

% disponi numeri di pagina
\usepackage{fancyhdr}
\fancyhf{} 
\fancyfoot[L]{\sffamily{\thepage}}

\makeatletter
\fancyhead[L]{\raisebox{1ex}[0pt][0pt]{\sffamily{\@title \ \@date}}} 
\fancyhead[R]{\raisebox{1ex}[0pt][0pt]{\sffamily{\@author}}}
\makeatother

\begin{document}
% sezione (data)
\section{Lezione del 08-11-24}

% stili pagina
\thispagestyle{empty}
\pagestyle{fancy}

% testo
\subsection{Contatori e divisione in frequenza}
Si ha che i contatori \textit{contano} i cicli di clock a cui sono sottoposti (cioè incrementano per ogni ciclo).
Possiamo usare un contatore per \textbf{dividere} la frequenza del clock per un dato valore.
Ad esempio, il MSB di un contatore in base 3 va 3 volte più lento del clock che lo pilota.
In generale, si ha che per un contatore a $N$ cifre in base 2 che riceve clock a periodo $T$, l'MSB è a periodo $2^N \cdot T$.

Si potrebbe pensare di usare l'uscita di riporto del contatore in MSB come uscita divisa del clock: questo non è raccomandabile in quanto l'uscita di riporto è un \textbf{uscita combinatoria}, che non è né \textbf{stabile} né a \textbf{temporizzazione certa} (a differenza dell'uscita di un registro).

\subsection{Registro multifunzionale}
Un registro multifunzionale è una rete che, all'arrivo del clock, memorizza nello stesso registro una tra $K$ funzioni combinatorie possibili, scelte impostando un certo numero di variabili di comando $W = \log_2 K$.

L'implementazione effettiva del registro è data da un multiplexer da $0$ a $K-1$ reti combinatorie, dove $W$ è la variabile di comando, la cui uscita viene inviata a un certo registro (che spedisce poi la sua uscita in retroazione alle reti combinatorie funzionali, e cosi via).

\subsection{Modello di reti sequenziali sincronizzate}
Definiamo tre modelli per le RSS (che abbiamo già introdotto parlando delle regole di pilotaggio):

\subsubsection{Modello di Moore}
Una RSS di Moore è definita a partire da:
\begin{itemize}
	\item Un insieme di $N$ variabili logiche di ingressi:
	\item Un insieme di $M$ variabili logiche di uscita;
	\item Un \textbf{meccanismo di marcatura}, che a ogni istante marca uno \textbf{stato interno presente}, scelto fra $K$ finito stati interni $S \equiv \{ S_0, ..., S_{K-1} \}$;
	\item Una legge di evoluzione nel tempo $A: X \times S \rightarrow S$, che mappa una coppia, data da un $X$ stato di ingresso e un elemento $s \in S$ stato interno, ad un nuovo stato interno (diciamo $s' \in S$);
	\item Una legge di evoluzione nel tempo $B:S \rightarrow Z$, che mappa uno stato interno $s \in S$ a uno stato di uscita $Z$.
\end{itemize}

La rete riceve \textbf{segnali di sincronizzazione}, come ad esempio le transizioni da 0 a 1 del segnale di clock (avevamo detto il \textit{leading edge}).
La legge di temporizzazione di una RSS di Moore è quindi la seguente: dato un elemento $s \in S$, stato interno marcato ad un certo istante, e dato $X$ ingresso ad un certo istante immediatamente precedente l'arrivo di un segnale di sincronizzazione:

\begin{enumerate}
	\item Si individua il nuovo stato interno da marcare $s' = A(X, s)$;
	\item Si aspetta $T_{prop}$ dopo l'arrivo del segnale di sincronizzazione;
	\item Si promuove $s'$ al rango di \textbf{stato interno marcato}.
\end{enumerate}

Lo stato interno marcato viene memorizzato in un apposito registro, detto \textbf{STAR} (da \textit{STAtus Register}).
Questo viene implementato con una batteria di D flip-flop non trasparenti.

In una RSS di Moore si hanno quindi i seguenti vincoli di pilotaggio:
\[
	\begin{cases}
		T \geq T_{hold} + T_{a\_monte} + T_A + T_{setup} \\ 
		T \geq T_{prop} + T_A + T_{setup} \\ 
		T \geq T_{prop} + T_Z + T_{a\_valle}
	\end{cases}
\]

che riguardano rispettivamente i tempi ingresso-STAR, STAR-STAR e STAR-uscita.

\subsubsection{Flip-flop JK}
Un'esempio di RSS di Moore è il flip-flop JK, che valuta due ingressi $j$ e $k$ e si comporta come segue:
\begin{table}[h!]
	\center \rowcolors{2}{white}{black!10}
	\begin{tabular} { c | c | c }
		$j$ & $k$ & Azione \\ 
		\hline 
		0 & 0 & Conserva \\ 
		1 & 0 & Setta \\ 
		0 & 1 & Resetta \\ 
		1 & 1 & Commuta
	\end{tabular}
\end{table}

Un modo di vedere questa rete è come un registro multifunzionale ad un bit, con tabella di applicazione:
\begin{table}[h!]
	\center 
	\begin{tabular} { c  c | c  c }
		$q$ & $q'$ & $j$ & $k$ \\ 
		\hline 
		0 & 0 & 0 & - \\ 
		0 & 1 & 1 & - \\ 
		1 & 0 & - & 1 \\ 
		1 & 1 & - & 0
	\end{tabular}
\end{table}

# quando puoi fai un ripassone

Vediamone la sintesi: visto che conosciamo soltanto la sintesi di reti combinatorie attraverso le mappe di Karnaugh, basterà sintetizzare le due reti combinatorie, dalla definizione di rete di Moore. \textbf{RCA} e \textbf{RCB} che implementano le funzioni $A$ e $B$.
Il registro STAR conterrà a questo punto lo stato interno, che nel caso del flip-flop JK ridurrà RCB a un cortocircuito per ogni uscita del registro.

Per la sintesi di RCA, invece, dovremo consultare la tabella di flusso:
# riportala e applica Karnaugh

\end{document}
